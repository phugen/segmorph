\chapter{Introduction}
Medical and biological image analysis usually is a time-consuming task that can only be carried out by domain experts. However, with the rise of sophisticated, computer-assisted image analysis and machine learning algorithms, good results have been achieved even on complex or noisy data. \textbf {TODO cite here}\\

\noindent This thesis deals with the special case of identifying different parts of \textit{Drosophila melanogaster} cells in images that were created using fluorescence microscopy. To obtain the images, the microscopy samples are stained using a fluorescent that is then exposed to illumination of a certain wavelength. The samples respond by emitting light of a different wavelength which is then filtered and used to create an image.

The microscopy images in question possess four regions of interest: The background, containing no cell material, the cell proper, the \textit{lamellipodium} and the \textit{filopodia}. The latter two terms describe part of the cytoplasm of a cell, which can be exuded from the cell proper in the form of a broad, translucent area, the \textit{lamellipodium}, and long, thin spikes traversing the \textit{lamellipodium} and the space beyond it, the \textit{filopodia}. These cell parts play roles in cell movement during wound healing and cell infection. Even for human experts, correctly identifying these areas is not always possible. Unfortunately, the stained samples are sensitive to light: if the samples are illuminated for too long, they deteriorate and become unusuable. Also, samples are often stacked while performing microscopy to create 3D models. When viewed as individual 2D images, both aforementioned circumstances lead to images that are often non-uniformly lit, sometimes noisy and which often contain ``ghost'' cells that are actually parts of cells in the sample stack layer below the layer the actual image was taken from. In addition, since the cells are alive and moving about, movement blur also occurs occasionally. All of this combined makes automated image segmentation and cell analysis a difficult task.\\
\textbf{TODO: add labelled cell image here, showing ghost cells etc}\\

\noindent In chapter \ref{chapter_concepts}, different approaches to the image segmentation problem are explained. The focus is on the function of Multi-Layer Perceptrons and the way they can be trained efficiently, as this approach was seen as the most promising one. The following chapter \ref{chapter_architecture} describes the architecture of the network that was believed to suit the task at hand, and also focuses on the individual layer types and loss functions the network works with. Then, in chapter \ref{chapter_training}, information about the data sets used for training and validating the network as well as information about data pre-processing, training parameters and optimization is given. In chapter \ref{chapter_results}, the hardware details of the training- and testing environment are listed and the results of all proposed approaches are compared to each other. In chapter \ref{chapter_conclusion}, a conclusion is derived from the achieved results and finally, in chapter \ref{chapter_futurework}, some ideas that are worth considering but went untested in the course of this thesis are explained. 