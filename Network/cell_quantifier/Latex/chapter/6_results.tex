\chapter {Results}
\label{chap:results}

	\section{Hardware}
All variants of the U-Net were trained on a NVIDIA TITAN X GPU (12 GB VRAM) using Caffe's CUDA/cuDNN support. \textbf{TODO: Processor, RAM? What else?}

	\section {Segmentation quality evaluation}

\noindent As there are many possible variations of the U-Net architecture, these variations were tested iteratively, choosing the best network of a number of networks and modifying it further. This was done because an exhaustive search for the best combination of weight initialization, activation functions, hyperparameters like the learning rate and techniques such as Dropout and Batch Normalization would have exceeded the time limit of this thesis. Also, at the time of writing, it was not yet known how these different approaches interact precisely. For example, Batch Normalization and specialized weight initialization schemes have the same goal but achieve it in different ways, making it unclear whether one or the other performs better in practice.\\

\noindent To compare the performance of all methods on the validation set with each other, the \textit{Micro} and \textit{Macro} variants of the F-Measure \cite{micromacro} are a suitable way to quantify how well the segmentation works. The Micro F-Measure is defined by the Precision and Recall quantities (see Section \textbf{\ref{subsec:fmeasure}}) of the validation set:

\[ F_{1\mu} = 2 \left ( \frac{PR_\mu \cdot RC_\mu}{PR_\mu + RC_\mu} \right ) \]

\noindent Here, $PR_\mu$ and $RC_\mu$ denote the micro-average Precision and Recall over the entire validation set. $PR_\mu$ and $RC_\mu$ are calculated by taking the sum of all $TP$, $FP$ and $FN$ values for all images and deriving the Precision and Recall over all $n$ validation images from these sums, i.e.

\[ PR_\mu = \frac{\sum_{i=1}^{n} TP_i}{\sum_{i=1}^{n} (TP_i + FP_i)} \text{ and }  RC_\mu = \frac{\sum_{i=1}^{n} TP_i}{\sum_{i=1}^{n} (TP_i + FN_i)} \]

\noindent The Macro F-Measure likewise is defined as

\[ F_{1M} = 2 \left ( \frac{PR_M \cdot RC_M}{PR_M + RC_M} \right ) \]

\noindent where $PR_M$ and $RC_M$ are the macro-average Precision and Recall. These are calculated for each sample independently, summed, and averaged over all $n$ samples:

\[ PR_M = \frac{1}{n} \sum_{i=1}^{n} \frac{TP_i}{TP_i + FP_i} \text { and } RC_M = \frac{1}{n} \sum_{i=1}^{n} \frac{TP_i}{TP_i + FN_i} \] 

\noindent \cite[pp. 317-318]{information_retrieval} highlights that the Micro F-Measure is dominated by ``large'' classes, meaning classes that occur often in the ground truth data. This shifts the focus of the segmentation effectiveness evaluation towards whether the large classes are segmented correctly. As most pixels in the validation images are background pixels and the correct segmentation of the non-background class pixels is of more interest, the Macro F-Measure is therefore chosen for assessing which method performs best because it is biased towards smaller classes rather than large ones, but for completeness, both quantities are listed.\\

\noindent The first test run pitted two nearly identical U-Net networks against each other, using ReLU activations and Dropout with $p = 0.5$. The only difference was the choice of the loss function. The \textbf{Unet\_Weighted} networks used the weighted Cross-Entropy Loss, while the \textbf{Unet\_F1} networks employed the multi-class F-Measure. Both networks were trained for 80,000 iterations (or $\approx$ 30 epochs) on both the 3-class and the 4-class training set, while testing the network on the respective validation set every 1,000 iterations.

\textbf{Unet\_Weighted} used an initial learning rate of 0.001, a step learning rate decay of a factor $\zeta = 0.1$ every 20,000 iterations and a momentum modifier $\gamma = 0.99$, while \textbf{Unet\_F1} used an initial learning rate of 0.0001, $\zeta = 0.3$ every 20,000 iterations and $\gamma = 0.99$. Both networks used L2 gradient regularization and a mini-batch size of 5.\\

\noindent The progress of the training is shown in Figure \textbf{\ref{fig:weighted_f1_training}}. The F-Measure scores provide the insight that the proper cells (green) can be discerned from the background much more easily than the Lamellopodia (red) and Filopodia (blue), which mirrors the problems when manually labelling the images. Reducing the dataset to three classes by merging the red and blue classes improves the results for both networks.\\

\begin {figure}[!ht]
	\begin {subfigure}[b]{0.4\linewidth}
		\scalebox{0.685}{%% Creator: Matplotlib, PGF backend
%%
%% To include the figure in your LaTeX document, write
%%   \input{<filename>.pgf}
%%
%% Make sure the required packages are loaded in your preamble
%%   \usepackage{pgf}
%%
%% Figures using additional raster images can only be included by \input if
%% they are in the same directory as the main LaTeX file. For loading figures
%% from other directories you can use the `import` package
%%   \usepackage{import}
%% and then include the figures with
%%   \import{<path to file>}{<filename>.pgf}
%%
%% Matplotlib used the following preamble
%%   \usepackage{fontspec}
%%   \setmainfont{DejaVu Serif}
%%   \setsansfont{DejaVu Sans}
%%   \setmonofont{DejaVu Sans Mono}
%%
\begingroup%
\makeatletter%
\begin{pgfpicture}%
\pgfpathrectangle{\pgfpointorigin}{\pgfqpoint{5.000000in}{4.000000in}}%
\pgfusepath{use as bounding box, clip}%
\begin{pgfscope}%
\pgfsetbuttcap%
\pgfsetmiterjoin%
\definecolor{currentfill}{rgb}{1.000000,1.000000,1.000000}%
\pgfsetfillcolor{currentfill}%
\pgfsetlinewidth{0.000000pt}%
\definecolor{currentstroke}{rgb}{1.000000,1.000000,1.000000}%
\pgfsetstrokecolor{currentstroke}%
\pgfsetdash{}{0pt}%
\pgfpathmoveto{\pgfqpoint{0.000000in}{0.000000in}}%
\pgfpathlineto{\pgfqpoint{5.000000in}{0.000000in}}%
\pgfpathlineto{\pgfqpoint{5.000000in}{4.000000in}}%
\pgfpathlineto{\pgfqpoint{0.000000in}{4.000000in}}%
\pgfpathclose%
\pgfusepath{fill}%
\end{pgfscope}%
\begin{pgfscope}%
\pgfsetbuttcap%
\pgfsetmiterjoin%
\definecolor{currentfill}{rgb}{1.000000,1.000000,1.000000}%
\pgfsetfillcolor{currentfill}%
\pgfsetlinewidth{0.000000pt}%
\definecolor{currentstroke}{rgb}{0.000000,0.000000,0.000000}%
\pgfsetstrokecolor{currentstroke}%
\pgfsetstrokeopacity{0.000000}%
\pgfsetdash{}{0pt}%
\pgfpathmoveto{\pgfqpoint{0.625000in}{0.440000in}}%
\pgfpathlineto{\pgfqpoint{4.500000in}{0.440000in}}%
\pgfpathlineto{\pgfqpoint{4.500000in}{3.520000in}}%
\pgfpathlineto{\pgfqpoint{0.625000in}{3.520000in}}%
\pgfpathclose%
\pgfusepath{fill}%
\end{pgfscope}%
\begin{pgfscope}%
\pgfsetbuttcap%
\pgfsetroundjoin%
\definecolor{currentfill}{rgb}{0.000000,0.000000,0.000000}%
\pgfsetfillcolor{currentfill}%
\pgfsetlinewidth{0.803000pt}%
\definecolor{currentstroke}{rgb}{0.000000,0.000000,0.000000}%
\pgfsetstrokecolor{currentstroke}%
\pgfsetdash{}{0pt}%
\pgfsys@defobject{currentmarker}{\pgfqpoint{0.000000in}{-0.048611in}}{\pgfqpoint{0.000000in}{0.000000in}}{%
\pgfpathmoveto{\pgfqpoint{0.000000in}{0.000000in}}%
\pgfpathlineto{\pgfqpoint{0.000000in}{-0.048611in}}%
\pgfusepath{stroke,fill}%
}%
\begin{pgfscope}%
\pgfsys@transformshift{0.801092in}{0.440000in}%
\pgfsys@useobject{currentmarker}{}%
\end{pgfscope}%
\end{pgfscope}%
\begin{pgfscope}%
\pgftext[x=0.801092in,y=0.342778in,,top]{\sffamily\fontsize{10.000000}{12.000000}\selectfont 0}%
\end{pgfscope}%
\begin{pgfscope}%
\pgfsetbuttcap%
\pgfsetroundjoin%
\definecolor{currentfill}{rgb}{0.000000,0.000000,0.000000}%
\pgfsetfillcolor{currentfill}%
\pgfsetlinewidth{0.803000pt}%
\definecolor{currentstroke}{rgb}{0.000000,0.000000,0.000000}%
\pgfsetstrokecolor{currentstroke}%
\pgfsetdash{}{0pt}%
\pgfsys@defobject{currentmarker}{\pgfqpoint{0.000000in}{-0.048611in}}{\pgfqpoint{0.000000in}{0.000000in}}{%
\pgfpathmoveto{\pgfqpoint{0.000000in}{0.000000in}}%
\pgfpathlineto{\pgfqpoint{0.000000in}{-0.048611in}}%
\pgfusepath{stroke,fill}%
}%
\begin{pgfscope}%
\pgfsys@transformshift{1.388206in}{0.440000in}%
\pgfsys@useobject{currentmarker}{}%
\end{pgfscope}%
\end{pgfscope}%
\begin{pgfscope}%
\pgftext[x=1.388206in,y=0.342778in,,top]{\sffamily\fontsize{10.000000}{12.000000}\selectfont 5}%
\end{pgfscope}%
\begin{pgfscope}%
\pgfsetbuttcap%
\pgfsetroundjoin%
\definecolor{currentfill}{rgb}{0.000000,0.000000,0.000000}%
\pgfsetfillcolor{currentfill}%
\pgfsetlinewidth{0.803000pt}%
\definecolor{currentstroke}{rgb}{0.000000,0.000000,0.000000}%
\pgfsetstrokecolor{currentstroke}%
\pgfsetdash{}{0pt}%
\pgfsys@defobject{currentmarker}{\pgfqpoint{0.000000in}{-0.048611in}}{\pgfqpoint{0.000000in}{0.000000in}}{%
\pgfpathmoveto{\pgfqpoint{0.000000in}{0.000000in}}%
\pgfpathlineto{\pgfqpoint{0.000000in}{-0.048611in}}%
\pgfusepath{stroke,fill}%
}%
\begin{pgfscope}%
\pgfsys@transformshift{1.975320in}{0.440000in}%
\pgfsys@useobject{currentmarker}{}%
\end{pgfscope}%
\end{pgfscope}%
\begin{pgfscope}%
\pgftext[x=1.975320in,y=0.342778in,,top]{\sffamily\fontsize{10.000000}{12.000000}\selectfont 10}%
\end{pgfscope}%
\begin{pgfscope}%
\pgfsetbuttcap%
\pgfsetroundjoin%
\definecolor{currentfill}{rgb}{0.000000,0.000000,0.000000}%
\pgfsetfillcolor{currentfill}%
\pgfsetlinewidth{0.803000pt}%
\definecolor{currentstroke}{rgb}{0.000000,0.000000,0.000000}%
\pgfsetstrokecolor{currentstroke}%
\pgfsetdash{}{0pt}%
\pgfsys@defobject{currentmarker}{\pgfqpoint{0.000000in}{-0.048611in}}{\pgfqpoint{0.000000in}{0.000000in}}{%
\pgfpathmoveto{\pgfqpoint{0.000000in}{0.000000in}}%
\pgfpathlineto{\pgfqpoint{0.000000in}{-0.048611in}}%
\pgfusepath{stroke,fill}%
}%
\begin{pgfscope}%
\pgfsys@transformshift{2.562434in}{0.440000in}%
\pgfsys@useobject{currentmarker}{}%
\end{pgfscope}%
\end{pgfscope}%
\begin{pgfscope}%
\pgftext[x=2.562434in,y=0.342778in,,top]{\sffamily\fontsize{10.000000}{12.000000}\selectfont 15}%
\end{pgfscope}%
\begin{pgfscope}%
\pgfsetbuttcap%
\pgfsetroundjoin%
\definecolor{currentfill}{rgb}{0.000000,0.000000,0.000000}%
\pgfsetfillcolor{currentfill}%
\pgfsetlinewidth{0.803000pt}%
\definecolor{currentstroke}{rgb}{0.000000,0.000000,0.000000}%
\pgfsetstrokecolor{currentstroke}%
\pgfsetdash{}{0pt}%
\pgfsys@defobject{currentmarker}{\pgfqpoint{0.000000in}{-0.048611in}}{\pgfqpoint{0.000000in}{0.000000in}}{%
\pgfpathmoveto{\pgfqpoint{0.000000in}{0.000000in}}%
\pgfpathlineto{\pgfqpoint{0.000000in}{-0.048611in}}%
\pgfusepath{stroke,fill}%
}%
\begin{pgfscope}%
\pgfsys@transformshift{3.149548in}{0.440000in}%
\pgfsys@useobject{currentmarker}{}%
\end{pgfscope}%
\end{pgfscope}%
\begin{pgfscope}%
\pgftext[x=3.149548in,y=0.342778in,,top]{\sffamily\fontsize{10.000000}{12.000000}\selectfont 20}%
\end{pgfscope}%
\begin{pgfscope}%
\pgfsetbuttcap%
\pgfsetroundjoin%
\definecolor{currentfill}{rgb}{0.000000,0.000000,0.000000}%
\pgfsetfillcolor{currentfill}%
\pgfsetlinewidth{0.803000pt}%
\definecolor{currentstroke}{rgb}{0.000000,0.000000,0.000000}%
\pgfsetstrokecolor{currentstroke}%
\pgfsetdash{}{0pt}%
\pgfsys@defobject{currentmarker}{\pgfqpoint{0.000000in}{-0.048611in}}{\pgfqpoint{0.000000in}{0.000000in}}{%
\pgfpathmoveto{\pgfqpoint{0.000000in}{0.000000in}}%
\pgfpathlineto{\pgfqpoint{0.000000in}{-0.048611in}}%
\pgfusepath{stroke,fill}%
}%
\begin{pgfscope}%
\pgfsys@transformshift{3.736662in}{0.440000in}%
\pgfsys@useobject{currentmarker}{}%
\end{pgfscope}%
\end{pgfscope}%
\begin{pgfscope}%
\pgftext[x=3.736662in,y=0.342778in,,top]{\sffamily\fontsize{10.000000}{12.000000}\selectfont 25}%
\end{pgfscope}%
\begin{pgfscope}%
\pgfsetbuttcap%
\pgfsetroundjoin%
\definecolor{currentfill}{rgb}{0.000000,0.000000,0.000000}%
\pgfsetfillcolor{currentfill}%
\pgfsetlinewidth{0.803000pt}%
\definecolor{currentstroke}{rgb}{0.000000,0.000000,0.000000}%
\pgfsetstrokecolor{currentstroke}%
\pgfsetdash{}{0pt}%
\pgfsys@defobject{currentmarker}{\pgfqpoint{0.000000in}{-0.048611in}}{\pgfqpoint{0.000000in}{0.000000in}}{%
\pgfpathmoveto{\pgfqpoint{0.000000in}{0.000000in}}%
\pgfpathlineto{\pgfqpoint{0.000000in}{-0.048611in}}%
\pgfusepath{stroke,fill}%
}%
\begin{pgfscope}%
\pgfsys@transformshift{4.323776in}{0.440000in}%
\pgfsys@useobject{currentmarker}{}%
\end{pgfscope}%
\end{pgfscope}%
\begin{pgfscope}%
\pgftext[x=4.323776in,y=0.342778in,,top]{\sffamily\fontsize{10.000000}{12.000000}\selectfont 30}%
\end{pgfscope}%
\begin{pgfscope}%
\pgftext[x=2.562500in,y=0.152809in,,top]{\sffamily\fontsize{10.000000}{12.000000}\selectfont Epochs}%
\end{pgfscope}%
\begin{pgfscope}%
\pgfsetbuttcap%
\pgfsetroundjoin%
\definecolor{currentfill}{rgb}{0.000000,0.000000,0.000000}%
\pgfsetfillcolor{currentfill}%
\pgfsetlinewidth{0.803000pt}%
\definecolor{currentstroke}{rgb}{0.000000,0.000000,0.000000}%
\pgfsetstrokecolor{currentstroke}%
\pgfsetdash{}{0pt}%
\pgfsys@defobject{currentmarker}{\pgfqpoint{-0.048611in}{0.000000in}}{\pgfqpoint{0.000000in}{0.000000in}}{%
\pgfpathmoveto{\pgfqpoint{0.000000in}{0.000000in}}%
\pgfpathlineto{\pgfqpoint{-0.048611in}{0.000000in}}%
\pgfusepath{stroke,fill}%
}%
\begin{pgfscope}%
\pgfsys@transformshift{0.625000in}{0.459181in}%
\pgfsys@useobject{currentmarker}{}%
\end{pgfscope}%
\end{pgfscope}%
\begin{pgfscope}%
\pgftext[x=0.306898in,y=0.406419in,left,base]{\sffamily\fontsize{10.000000}{12.000000}\selectfont 0.0}%
\end{pgfscope}%
\begin{pgfscope}%
\pgfsetbuttcap%
\pgfsetroundjoin%
\definecolor{currentfill}{rgb}{0.000000,0.000000,0.000000}%
\pgfsetfillcolor{currentfill}%
\pgfsetlinewidth{0.803000pt}%
\definecolor{currentstroke}{rgb}{0.000000,0.000000,0.000000}%
\pgfsetstrokecolor{currentstroke}%
\pgfsetdash{}{0pt}%
\pgfsys@defobject{currentmarker}{\pgfqpoint{-0.048611in}{0.000000in}}{\pgfqpoint{0.000000in}{0.000000in}}{%
\pgfpathmoveto{\pgfqpoint{0.000000in}{0.000000in}}%
\pgfpathlineto{\pgfqpoint{-0.048611in}{0.000000in}}%
\pgfusepath{stroke,fill}%
}%
\begin{pgfscope}%
\pgfsys@transformshift{0.625000in}{0.841783in}%
\pgfsys@useobject{currentmarker}{}%
\end{pgfscope}%
\end{pgfscope}%
\begin{pgfscope}%
\pgftext[x=0.306898in,y=0.789021in,left,base]{\sffamily\fontsize{10.000000}{12.000000}\selectfont 0.2}%
\end{pgfscope}%
\begin{pgfscope}%
\pgfsetbuttcap%
\pgfsetroundjoin%
\definecolor{currentfill}{rgb}{0.000000,0.000000,0.000000}%
\pgfsetfillcolor{currentfill}%
\pgfsetlinewidth{0.803000pt}%
\definecolor{currentstroke}{rgb}{0.000000,0.000000,0.000000}%
\pgfsetstrokecolor{currentstroke}%
\pgfsetdash{}{0pt}%
\pgfsys@defobject{currentmarker}{\pgfqpoint{-0.048611in}{0.000000in}}{\pgfqpoint{0.000000in}{0.000000in}}{%
\pgfpathmoveto{\pgfqpoint{0.000000in}{0.000000in}}%
\pgfpathlineto{\pgfqpoint{-0.048611in}{0.000000in}}%
\pgfusepath{stroke,fill}%
}%
\begin{pgfscope}%
\pgfsys@transformshift{0.625000in}{1.224385in}%
\pgfsys@useobject{currentmarker}{}%
\end{pgfscope}%
\end{pgfscope}%
\begin{pgfscope}%
\pgftext[x=0.306898in,y=1.171624in,left,base]{\sffamily\fontsize{10.000000}{12.000000}\selectfont 0.4}%
\end{pgfscope}%
\begin{pgfscope}%
\pgfsetbuttcap%
\pgfsetroundjoin%
\definecolor{currentfill}{rgb}{0.000000,0.000000,0.000000}%
\pgfsetfillcolor{currentfill}%
\pgfsetlinewidth{0.803000pt}%
\definecolor{currentstroke}{rgb}{0.000000,0.000000,0.000000}%
\pgfsetstrokecolor{currentstroke}%
\pgfsetdash{}{0pt}%
\pgfsys@defobject{currentmarker}{\pgfqpoint{-0.048611in}{0.000000in}}{\pgfqpoint{0.000000in}{0.000000in}}{%
\pgfpathmoveto{\pgfqpoint{0.000000in}{0.000000in}}%
\pgfpathlineto{\pgfqpoint{-0.048611in}{0.000000in}}%
\pgfusepath{stroke,fill}%
}%
\begin{pgfscope}%
\pgfsys@transformshift{0.625000in}{1.606988in}%
\pgfsys@useobject{currentmarker}{}%
\end{pgfscope}%
\end{pgfscope}%
\begin{pgfscope}%
\pgftext[x=0.306898in,y=1.554226in,left,base]{\sffamily\fontsize{10.000000}{12.000000}\selectfont 0.6}%
\end{pgfscope}%
\begin{pgfscope}%
\pgfsetbuttcap%
\pgfsetroundjoin%
\definecolor{currentfill}{rgb}{0.000000,0.000000,0.000000}%
\pgfsetfillcolor{currentfill}%
\pgfsetlinewidth{0.803000pt}%
\definecolor{currentstroke}{rgb}{0.000000,0.000000,0.000000}%
\pgfsetstrokecolor{currentstroke}%
\pgfsetdash{}{0pt}%
\pgfsys@defobject{currentmarker}{\pgfqpoint{-0.048611in}{0.000000in}}{\pgfqpoint{0.000000in}{0.000000in}}{%
\pgfpathmoveto{\pgfqpoint{0.000000in}{0.000000in}}%
\pgfpathlineto{\pgfqpoint{-0.048611in}{0.000000in}}%
\pgfusepath{stroke,fill}%
}%
\begin{pgfscope}%
\pgfsys@transformshift{0.625000in}{1.989590in}%
\pgfsys@useobject{currentmarker}{}%
\end{pgfscope}%
\end{pgfscope}%
\begin{pgfscope}%
\pgftext[x=0.306898in,y=1.936829in,left,base]{\sffamily\fontsize{10.000000}{12.000000}\selectfont 0.8}%
\end{pgfscope}%
\begin{pgfscope}%
\pgfsetbuttcap%
\pgfsetroundjoin%
\definecolor{currentfill}{rgb}{0.000000,0.000000,0.000000}%
\pgfsetfillcolor{currentfill}%
\pgfsetlinewidth{0.803000pt}%
\definecolor{currentstroke}{rgb}{0.000000,0.000000,0.000000}%
\pgfsetstrokecolor{currentstroke}%
\pgfsetdash{}{0pt}%
\pgfsys@defobject{currentmarker}{\pgfqpoint{-0.048611in}{0.000000in}}{\pgfqpoint{0.000000in}{0.000000in}}{%
\pgfpathmoveto{\pgfqpoint{0.000000in}{0.000000in}}%
\pgfpathlineto{\pgfqpoint{-0.048611in}{0.000000in}}%
\pgfusepath{stroke,fill}%
}%
\begin{pgfscope}%
\pgfsys@transformshift{0.625000in}{2.372193in}%
\pgfsys@useobject{currentmarker}{}%
\end{pgfscope}%
\end{pgfscope}%
\begin{pgfscope}%
\pgftext[x=0.306898in,y=2.319431in,left,base]{\sffamily\fontsize{10.000000}{12.000000}\selectfont 1.0}%
\end{pgfscope}%
\begin{pgfscope}%
\pgfsetbuttcap%
\pgfsetroundjoin%
\definecolor{currentfill}{rgb}{0.000000,0.000000,0.000000}%
\pgfsetfillcolor{currentfill}%
\pgfsetlinewidth{0.803000pt}%
\definecolor{currentstroke}{rgb}{0.000000,0.000000,0.000000}%
\pgfsetstrokecolor{currentstroke}%
\pgfsetdash{}{0pt}%
\pgfsys@defobject{currentmarker}{\pgfqpoint{-0.048611in}{0.000000in}}{\pgfqpoint{0.000000in}{0.000000in}}{%
\pgfpathmoveto{\pgfqpoint{0.000000in}{0.000000in}}%
\pgfpathlineto{\pgfqpoint{-0.048611in}{0.000000in}}%
\pgfusepath{stroke,fill}%
}%
\begin{pgfscope}%
\pgfsys@transformshift{0.625000in}{2.754795in}%
\pgfsys@useobject{currentmarker}{}%
\end{pgfscope}%
\end{pgfscope}%
\begin{pgfscope}%
\pgftext[x=0.306898in,y=2.702034in,left,base]{\sffamily\fontsize{10.000000}{12.000000}\selectfont 1.2}%
\end{pgfscope}%
\begin{pgfscope}%
\pgfsetbuttcap%
\pgfsetroundjoin%
\definecolor{currentfill}{rgb}{0.000000,0.000000,0.000000}%
\pgfsetfillcolor{currentfill}%
\pgfsetlinewidth{0.803000pt}%
\definecolor{currentstroke}{rgb}{0.000000,0.000000,0.000000}%
\pgfsetstrokecolor{currentstroke}%
\pgfsetdash{}{0pt}%
\pgfsys@defobject{currentmarker}{\pgfqpoint{-0.048611in}{0.000000in}}{\pgfqpoint{0.000000in}{0.000000in}}{%
\pgfpathmoveto{\pgfqpoint{0.000000in}{0.000000in}}%
\pgfpathlineto{\pgfqpoint{-0.048611in}{0.000000in}}%
\pgfusepath{stroke,fill}%
}%
\begin{pgfscope}%
\pgfsys@transformshift{0.625000in}{3.137398in}%
\pgfsys@useobject{currentmarker}{}%
\end{pgfscope}%
\end{pgfscope}%
\begin{pgfscope}%
\pgftext[x=0.306898in,y=3.084636in,left,base]{\sffamily\fontsize{10.000000}{12.000000}\selectfont 1.4}%
\end{pgfscope}%
\begin{pgfscope}%
\pgfsetbuttcap%
\pgfsetroundjoin%
\definecolor{currentfill}{rgb}{0.000000,0.000000,0.000000}%
\pgfsetfillcolor{currentfill}%
\pgfsetlinewidth{0.803000pt}%
\definecolor{currentstroke}{rgb}{0.000000,0.000000,0.000000}%
\pgfsetstrokecolor{currentstroke}%
\pgfsetdash{}{0pt}%
\pgfsys@defobject{currentmarker}{\pgfqpoint{-0.048611in}{0.000000in}}{\pgfqpoint{0.000000in}{0.000000in}}{%
\pgfpathmoveto{\pgfqpoint{0.000000in}{0.000000in}}%
\pgfpathlineto{\pgfqpoint{-0.048611in}{0.000000in}}%
\pgfusepath{stroke,fill}%
}%
\begin{pgfscope}%
\pgfsys@transformshift{0.625000in}{3.520000in}%
\pgfsys@useobject{currentmarker}{}%
\end{pgfscope}%
\end{pgfscope}%
\begin{pgfscope}%
\pgftext[x=0.306898in,y=3.467238in,left,base]{\sffamily\fontsize{10.000000}{12.000000}\selectfont 1.6}%
\end{pgfscope}%
\begin{pgfscope}%
\pgftext[x=0.251343in,y=1.980000in,,bottom,rotate=90.000000]{\sffamily\fontsize{10.000000}{12.000000}\selectfont Cross-Entropy loss}%
\end{pgfscope}%
\begin{pgfscope}%
\pgfpathrectangle{\pgfqpoint{0.625000in}{0.440000in}}{\pgfqpoint{3.875000in}{3.080000in}} %
\pgfusepath{clip}%
\pgfsetrectcap%
\pgfsetroundjoin%
\pgfsetlinewidth{1.505625pt}%
\definecolor{currentstroke}{rgb}{0.901961,0.901961,0.980392}%
\pgfsetstrokecolor{currentstroke}%
\pgfsetdash{}{0pt}%
\pgfpathmoveto{\pgfqpoint{0.805496in}{1.245008in}}%
\pgfpathlineto{\pgfqpoint{0.809899in}{1.176530in}}%
\pgfpathlineto{\pgfqpoint{0.814303in}{1.052075in}}%
\pgfpathlineto{\pgfqpoint{0.818706in}{2.242249in}}%
\pgfpathlineto{\pgfqpoint{0.823110in}{1.355767in}}%
\pgfpathlineto{\pgfqpoint{0.827513in}{1.046334in}}%
\pgfpathlineto{\pgfqpoint{0.831917in}{1.182255in}}%
\pgfpathlineto{\pgfqpoint{0.836320in}{1.361239in}}%
\pgfpathlineto{\pgfqpoint{0.840724in}{1.111893in}}%
\pgfpathlineto{\pgfqpoint{0.845127in}{1.090859in}}%
\pgfpathlineto{\pgfqpoint{0.849530in}{0.960003in}}%
\pgfpathlineto{\pgfqpoint{0.858337in}{1.121879in}}%
\pgfpathlineto{\pgfqpoint{0.862741in}{0.969706in}}%
\pgfpathlineto{\pgfqpoint{0.867144in}{0.911145in}}%
\pgfpathlineto{\pgfqpoint{0.871548in}{1.664537in}}%
\pgfpathlineto{\pgfqpoint{0.875951in}{1.339698in}}%
\pgfpathlineto{\pgfqpoint{0.880355in}{1.175426in}}%
\pgfpathlineto{\pgfqpoint{0.884758in}{1.314915in}}%
\pgfpathlineto{\pgfqpoint{0.889162in}{1.244788in}}%
\pgfpathlineto{\pgfqpoint{0.893565in}{1.282281in}}%
\pgfpathlineto{\pgfqpoint{0.897969in}{1.114010in}}%
\pgfpathlineto{\pgfqpoint{0.902372in}{1.329978in}}%
\pgfpathlineto{\pgfqpoint{0.906775in}{0.879871in}}%
\pgfpathlineto{\pgfqpoint{0.911179in}{1.082369in}}%
\pgfpathlineto{\pgfqpoint{0.915582in}{0.806015in}}%
\pgfpathlineto{\pgfqpoint{0.919986in}{0.783499in}}%
\pgfpathlineto{\pgfqpoint{0.924389in}{0.854024in}}%
\pgfpathlineto{\pgfqpoint{0.928793in}{1.417458in}}%
\pgfpathlineto{\pgfqpoint{0.933196in}{1.136498in}}%
\pgfpathlineto{\pgfqpoint{0.937600in}{1.018287in}}%
\pgfpathlineto{\pgfqpoint{0.942003in}{1.077621in}}%
\pgfpathlineto{\pgfqpoint{0.946407in}{1.076791in}}%
\pgfpathlineto{\pgfqpoint{0.950810in}{0.884011in}}%
\pgfpathlineto{\pgfqpoint{0.955214in}{0.907627in}}%
\pgfpathlineto{\pgfqpoint{0.959617in}{0.839535in}}%
\pgfpathlineto{\pgfqpoint{0.964021in}{0.826488in}}%
\pgfpathlineto{\pgfqpoint{0.968424in}{0.849320in}}%
\pgfpathlineto{\pgfqpoint{0.972827in}{0.780261in}}%
\pgfpathlineto{\pgfqpoint{0.977231in}{0.760997in}}%
\pgfpathlineto{\pgfqpoint{0.981634in}{1.214170in}}%
\pgfpathlineto{\pgfqpoint{0.990441in}{0.961536in}}%
\pgfpathlineto{\pgfqpoint{0.994845in}{1.023896in}}%
\pgfpathlineto{\pgfqpoint{0.999248in}{0.853519in}}%
\pgfpathlineto{\pgfqpoint{1.003652in}{0.844521in}}%
\pgfpathlineto{\pgfqpoint{1.008055in}{0.909486in}}%
\pgfpathlineto{\pgfqpoint{1.012459in}{0.747091in}}%
\pgfpathlineto{\pgfqpoint{1.016862in}{0.690093in}}%
\pgfpathlineto{\pgfqpoint{1.021266in}{0.809036in}}%
\pgfpathlineto{\pgfqpoint{1.025669in}{0.797149in}}%
\pgfpathlineto{\pgfqpoint{1.030072in}{0.757180in}}%
\pgfpathlineto{\pgfqpoint{1.034476in}{0.793152in}}%
\pgfpathlineto{\pgfqpoint{1.038879in}{1.150392in}}%
\pgfpathlineto{\pgfqpoint{1.043283in}{0.931752in}}%
\pgfpathlineto{\pgfqpoint{1.047686in}{0.877663in}}%
\pgfpathlineto{\pgfqpoint{1.052090in}{0.962441in}}%
\pgfpathlineto{\pgfqpoint{1.056493in}{0.753088in}}%
\pgfpathlineto{\pgfqpoint{1.060897in}{0.903753in}}%
\pgfpathlineto{\pgfqpoint{1.065300in}{0.813897in}}%
\pgfpathlineto{\pgfqpoint{1.069704in}{0.859806in}}%
\pgfpathlineto{\pgfqpoint{1.074107in}{0.706634in}}%
\pgfpathlineto{\pgfqpoint{1.078511in}{0.819973in}}%
\pgfpathlineto{\pgfqpoint{1.082914in}{0.883001in}}%
\pgfpathlineto{\pgfqpoint{1.087317in}{0.712247in}}%
\pgfpathlineto{\pgfqpoint{1.091721in}{1.105497in}}%
\pgfpathlineto{\pgfqpoint{1.096124in}{1.028625in}}%
\pgfpathlineto{\pgfqpoint{1.100528in}{0.786836in}}%
\pgfpathlineto{\pgfqpoint{1.104931in}{0.806608in}}%
\pgfpathlineto{\pgfqpoint{1.109335in}{0.811305in}}%
\pgfpathlineto{\pgfqpoint{1.113738in}{0.751716in}}%
\pgfpathlineto{\pgfqpoint{1.118142in}{0.727308in}}%
\pgfpathlineto{\pgfqpoint{1.122545in}{0.653357in}}%
\pgfpathlineto{\pgfqpoint{1.126949in}{0.743804in}}%
\pgfpathlineto{\pgfqpoint{1.131352in}{0.797474in}}%
\pgfpathlineto{\pgfqpoint{1.135756in}{0.767001in}}%
\pgfpathlineto{\pgfqpoint{1.140159in}{0.776283in}}%
\pgfpathlineto{\pgfqpoint{1.144563in}{0.753863in}}%
\pgfpathlineto{\pgfqpoint{1.148966in}{0.895051in}}%
\pgfpathlineto{\pgfqpoint{1.153369in}{0.850885in}}%
\pgfpathlineto{\pgfqpoint{1.157773in}{0.770634in}}%
\pgfpathlineto{\pgfqpoint{1.162176in}{0.801721in}}%
\pgfpathlineto{\pgfqpoint{1.166580in}{0.741028in}}%
\pgfpathlineto{\pgfqpoint{1.170983in}{0.804743in}}%
\pgfpathlineto{\pgfqpoint{1.175387in}{0.682182in}}%
\pgfpathlineto{\pgfqpoint{1.188597in}{0.873575in}}%
\pgfpathlineto{\pgfqpoint{1.193001in}{0.810469in}}%
\pgfpathlineto{\pgfqpoint{1.197404in}{0.693955in}}%
\pgfpathlineto{\pgfqpoint{1.201808in}{0.975933in}}%
\pgfpathlineto{\pgfqpoint{1.206211in}{0.890257in}}%
\pgfpathlineto{\pgfqpoint{1.210614in}{0.769555in}}%
\pgfpathlineto{\pgfqpoint{1.215018in}{0.840230in}}%
\pgfpathlineto{\pgfqpoint{1.219421in}{0.750012in}}%
\pgfpathlineto{\pgfqpoint{1.223825in}{0.725252in}}%
\pgfpathlineto{\pgfqpoint{1.228228in}{0.790093in}}%
\pgfpathlineto{\pgfqpoint{1.232632in}{0.663754in}}%
\pgfpathlineto{\pgfqpoint{1.237035in}{0.687441in}}%
\pgfpathlineto{\pgfqpoint{1.241439in}{0.947785in}}%
\pgfpathlineto{\pgfqpoint{1.245842in}{0.901490in}}%
\pgfpathlineto{\pgfqpoint{1.250246in}{0.633918in}}%
\pgfpathlineto{\pgfqpoint{1.254649in}{0.706336in}}%
\pgfpathlineto{\pgfqpoint{1.259053in}{0.880036in}}%
\pgfpathlineto{\pgfqpoint{1.263456in}{0.714738in}}%
\pgfpathlineto{\pgfqpoint{1.267860in}{0.701640in}}%
\pgfpathlineto{\pgfqpoint{1.272263in}{0.748906in}}%
\pgfpathlineto{\pgfqpoint{1.276666in}{0.776825in}}%
\pgfpathlineto{\pgfqpoint{1.281070in}{0.748895in}}%
\pgfpathlineto{\pgfqpoint{1.285473in}{0.657899in}}%
\pgfpathlineto{\pgfqpoint{1.289877in}{0.658566in}}%
\pgfpathlineto{\pgfqpoint{1.294280in}{1.036101in}}%
\pgfpathlineto{\pgfqpoint{1.298684in}{0.708461in}}%
\pgfpathlineto{\pgfqpoint{1.303087in}{0.728835in}}%
\pgfpathlineto{\pgfqpoint{1.307491in}{0.693852in}}%
\pgfpathlineto{\pgfqpoint{1.311894in}{0.990612in}}%
\pgfpathlineto{\pgfqpoint{1.316298in}{0.877178in}}%
\pgfpathlineto{\pgfqpoint{1.320701in}{0.831671in}}%
\pgfpathlineto{\pgfqpoint{1.325105in}{0.851417in}}%
\pgfpathlineto{\pgfqpoint{1.329508in}{0.859008in}}%
\pgfpathlineto{\pgfqpoint{1.333911in}{0.664278in}}%
\pgfpathlineto{\pgfqpoint{1.338315in}{0.791176in}}%
\pgfpathlineto{\pgfqpoint{1.342718in}{0.779838in}}%
\pgfpathlineto{\pgfqpoint{1.347122in}{0.705565in}}%
\pgfpathlineto{\pgfqpoint{1.351525in}{0.736525in}}%
\pgfpathlineto{\pgfqpoint{1.355929in}{0.940554in}}%
\pgfpathlineto{\pgfqpoint{1.360332in}{0.768362in}}%
\pgfpathlineto{\pgfqpoint{1.364736in}{0.706615in}}%
\pgfpathlineto{\pgfqpoint{1.369139in}{1.287532in}}%
\pgfpathlineto{\pgfqpoint{1.373543in}{0.807521in}}%
\pgfpathlineto{\pgfqpoint{1.377946in}{0.728043in}}%
\pgfpathlineto{\pgfqpoint{1.382350in}{0.808120in}}%
\pgfpathlineto{\pgfqpoint{1.386753in}{0.669526in}}%
\pgfpathlineto{\pgfqpoint{1.391157in}{0.749662in}}%
\pgfpathlineto{\pgfqpoint{1.395560in}{0.746901in}}%
\pgfpathlineto{\pgfqpoint{1.399963in}{0.756187in}}%
\pgfpathlineto{\pgfqpoint{1.404367in}{1.127296in}}%
\pgfpathlineto{\pgfqpoint{1.413174in}{0.884372in}}%
\pgfpathlineto{\pgfqpoint{1.417577in}{0.689492in}}%
\pgfpathlineto{\pgfqpoint{1.421981in}{1.078566in}}%
\pgfpathlineto{\pgfqpoint{1.426384in}{0.830219in}}%
\pgfpathlineto{\pgfqpoint{1.430788in}{0.836153in}}%
\pgfpathlineto{\pgfqpoint{1.439595in}{0.659106in}}%
\pgfpathlineto{\pgfqpoint{1.443998in}{0.697167in}}%
\pgfpathlineto{\pgfqpoint{1.448402in}{0.668085in}}%
\pgfpathlineto{\pgfqpoint{1.452805in}{0.697450in}}%
\pgfpathlineto{\pgfqpoint{1.457208in}{0.662742in}}%
\pgfpathlineto{\pgfqpoint{1.461612in}{0.734689in}}%
\pgfpathlineto{\pgfqpoint{1.466015in}{0.770529in}}%
\pgfpathlineto{\pgfqpoint{1.470419in}{0.720332in}}%
\pgfpathlineto{\pgfqpoint{1.474822in}{0.720907in}}%
\pgfpathlineto{\pgfqpoint{1.479226in}{0.824271in}}%
\pgfpathlineto{\pgfqpoint{1.483629in}{0.870478in}}%
\pgfpathlineto{\pgfqpoint{1.488033in}{0.812032in}}%
\pgfpathlineto{\pgfqpoint{1.492436in}{0.718593in}}%
\pgfpathlineto{\pgfqpoint{1.496840in}{0.715717in}}%
\pgfpathlineto{\pgfqpoint{1.501243in}{0.665679in}}%
\pgfpathlineto{\pgfqpoint{1.505647in}{0.802951in}}%
\pgfpathlineto{\pgfqpoint{1.510050in}{0.722071in}}%
\pgfpathlineto{\pgfqpoint{1.514454in}{1.185190in}}%
\pgfpathlineto{\pgfqpoint{1.518857in}{1.046284in}}%
\pgfpathlineto{\pgfqpoint{1.523260in}{0.817472in}}%
\pgfpathlineto{\pgfqpoint{1.527664in}{0.760886in}}%
\pgfpathlineto{\pgfqpoint{1.532067in}{0.882232in}}%
\pgfpathlineto{\pgfqpoint{1.536471in}{0.757205in}}%
\pgfpathlineto{\pgfqpoint{1.540874in}{0.670400in}}%
\pgfpathlineto{\pgfqpoint{1.545278in}{0.690324in}}%
\pgfpathlineto{\pgfqpoint{1.549681in}{0.752246in}}%
\pgfpathlineto{\pgfqpoint{1.554085in}{0.728757in}}%
\pgfpathlineto{\pgfqpoint{1.558488in}{0.636082in}}%
\pgfpathlineto{\pgfqpoint{1.562892in}{0.655595in}}%
\pgfpathlineto{\pgfqpoint{1.567295in}{0.645980in}}%
\pgfpathlineto{\pgfqpoint{1.571699in}{0.691453in}}%
\pgfpathlineto{\pgfqpoint{1.576102in}{0.674607in}}%
\pgfpathlineto{\pgfqpoint{1.580505in}{0.683873in}}%
\pgfpathlineto{\pgfqpoint{1.584909in}{0.819231in}}%
\pgfpathlineto{\pgfqpoint{1.589312in}{0.715958in}}%
\pgfpathlineto{\pgfqpoint{1.593716in}{0.700846in}}%
\pgfpathlineto{\pgfqpoint{1.598119in}{0.901976in}}%
\pgfpathlineto{\pgfqpoint{1.602523in}{0.646949in}}%
\pgfpathlineto{\pgfqpoint{1.606926in}{0.708297in}}%
\pgfpathlineto{\pgfqpoint{1.611330in}{0.662304in}}%
\pgfpathlineto{\pgfqpoint{1.615733in}{0.715932in}}%
\pgfpathlineto{\pgfqpoint{1.620137in}{0.657661in}}%
\pgfpathlineto{\pgfqpoint{1.624540in}{0.796963in}}%
\pgfpathlineto{\pgfqpoint{1.628944in}{0.788115in}}%
\pgfpathlineto{\pgfqpoint{1.633347in}{0.796716in}}%
\pgfpathlineto{\pgfqpoint{1.637751in}{0.767554in}}%
\pgfpathlineto{\pgfqpoint{1.642154in}{0.825105in}}%
\pgfpathlineto{\pgfqpoint{1.646557in}{0.802823in}}%
\pgfpathlineto{\pgfqpoint{1.650961in}{0.722011in}}%
\pgfpathlineto{\pgfqpoint{1.655364in}{0.757697in}}%
\pgfpathlineto{\pgfqpoint{1.659768in}{0.738628in}}%
\pgfpathlineto{\pgfqpoint{1.664171in}{0.768641in}}%
\pgfpathlineto{\pgfqpoint{1.668575in}{0.589800in}}%
\pgfpathlineto{\pgfqpoint{1.672978in}{0.688897in}}%
\pgfpathlineto{\pgfqpoint{1.677382in}{0.680311in}}%
\pgfpathlineto{\pgfqpoint{1.681785in}{0.704161in}}%
\pgfpathlineto{\pgfqpoint{1.690592in}{0.625308in}}%
\pgfpathlineto{\pgfqpoint{1.694996in}{0.861456in}}%
\pgfpathlineto{\pgfqpoint{1.699399in}{0.843801in}}%
\pgfpathlineto{\pgfqpoint{1.703802in}{0.865969in}}%
\pgfpathlineto{\pgfqpoint{1.717013in}{0.655157in}}%
\pgfpathlineto{\pgfqpoint{1.721416in}{0.638520in}}%
\pgfpathlineto{\pgfqpoint{1.725820in}{0.606408in}}%
\pgfpathlineto{\pgfqpoint{1.730223in}{0.664544in}}%
\pgfpathlineto{\pgfqpoint{1.734627in}{0.638567in}}%
\pgfpathlineto{\pgfqpoint{1.739030in}{0.841691in}}%
\pgfpathlineto{\pgfqpoint{1.743434in}{0.811460in}}%
\pgfpathlineto{\pgfqpoint{1.747837in}{0.684994in}}%
\pgfpathlineto{\pgfqpoint{1.752241in}{0.991080in}}%
\pgfpathlineto{\pgfqpoint{1.756644in}{0.818905in}}%
\pgfpathlineto{\pgfqpoint{1.761048in}{0.722061in}}%
\pgfpathlineto{\pgfqpoint{1.765451in}{0.776109in}}%
\pgfpathlineto{\pgfqpoint{1.774258in}{0.610284in}}%
\pgfpathlineto{\pgfqpoint{1.783065in}{0.710307in}}%
\pgfpathlineto{\pgfqpoint{1.787468in}{0.703449in}}%
\pgfpathlineto{\pgfqpoint{1.791872in}{0.734781in}}%
\pgfpathlineto{\pgfqpoint{1.796275in}{0.819902in}}%
\pgfpathlineto{\pgfqpoint{1.800679in}{0.809803in}}%
\pgfpathlineto{\pgfqpoint{1.805082in}{0.745291in}}%
\pgfpathlineto{\pgfqpoint{1.809486in}{0.798055in}}%
\pgfpathlineto{\pgfqpoint{1.813889in}{0.739990in}}%
\pgfpathlineto{\pgfqpoint{1.818293in}{0.651945in}}%
\pgfpathlineto{\pgfqpoint{1.822696in}{0.779796in}}%
\pgfpathlineto{\pgfqpoint{1.831503in}{0.629995in}}%
\pgfpathlineto{\pgfqpoint{1.835906in}{0.635786in}}%
\pgfpathlineto{\pgfqpoint{1.840310in}{0.663462in}}%
\pgfpathlineto{\pgfqpoint{1.844713in}{0.663092in}}%
\pgfpathlineto{\pgfqpoint{1.849117in}{0.780584in}}%
\pgfpathlineto{\pgfqpoint{1.853520in}{0.750521in}}%
\pgfpathlineto{\pgfqpoint{1.857924in}{0.700584in}}%
\pgfpathlineto{\pgfqpoint{1.862327in}{0.843866in}}%
\pgfpathlineto{\pgfqpoint{1.866731in}{0.673989in}}%
\pgfpathlineto{\pgfqpoint{1.871134in}{0.670743in}}%
\pgfpathlineto{\pgfqpoint{1.875538in}{0.644275in}}%
\pgfpathlineto{\pgfqpoint{1.884345in}{0.683694in}}%
\pgfpathlineto{\pgfqpoint{1.888748in}{0.658673in}}%
\pgfpathlineto{\pgfqpoint{1.893151in}{0.694449in}}%
\pgfpathlineto{\pgfqpoint{1.897555in}{0.795498in}}%
\pgfpathlineto{\pgfqpoint{1.901958in}{0.737979in}}%
\pgfpathlineto{\pgfqpoint{1.906362in}{0.836884in}}%
\pgfpathlineto{\pgfqpoint{1.910765in}{0.779972in}}%
\pgfpathlineto{\pgfqpoint{1.915169in}{0.691051in}}%
\pgfpathlineto{\pgfqpoint{1.919572in}{0.700903in}}%
\pgfpathlineto{\pgfqpoint{1.923976in}{0.738792in}}%
\pgfpathlineto{\pgfqpoint{1.928379in}{0.612745in}}%
\pgfpathlineto{\pgfqpoint{1.932783in}{0.724068in}}%
\pgfpathlineto{\pgfqpoint{1.937186in}{0.719746in}}%
\pgfpathlineto{\pgfqpoint{1.941590in}{0.658581in}}%
\pgfpathlineto{\pgfqpoint{1.945993in}{0.710151in}}%
\pgfpathlineto{\pgfqpoint{1.950396in}{0.638428in}}%
\pgfpathlineto{\pgfqpoint{1.954800in}{0.646951in}}%
\pgfpathlineto{\pgfqpoint{1.959203in}{0.780423in}}%
\pgfpathlineto{\pgfqpoint{1.963607in}{0.780957in}}%
\pgfpathlineto{\pgfqpoint{1.968010in}{0.787434in}}%
\pgfpathlineto{\pgfqpoint{1.972414in}{0.705663in}}%
\pgfpathlineto{\pgfqpoint{1.976817in}{0.750817in}}%
\pgfpathlineto{\pgfqpoint{1.981221in}{0.769157in}}%
\pgfpathlineto{\pgfqpoint{1.985624in}{0.711407in}}%
\pgfpathlineto{\pgfqpoint{1.990028in}{0.700408in}}%
\pgfpathlineto{\pgfqpoint{1.994431in}{0.647634in}}%
\pgfpathlineto{\pgfqpoint{1.998835in}{0.655886in}}%
\pgfpathlineto{\pgfqpoint{2.003238in}{0.713328in}}%
\pgfpathlineto{\pgfqpoint{2.007642in}{0.684241in}}%
\pgfpathlineto{\pgfqpoint{2.012045in}{0.694512in}}%
\pgfpathlineto{\pgfqpoint{2.016448in}{0.692295in}}%
\pgfpathlineto{\pgfqpoint{2.020852in}{0.597919in}}%
\pgfpathlineto{\pgfqpoint{2.025255in}{0.714958in}}%
\pgfpathlineto{\pgfqpoint{2.029659in}{0.724031in}}%
\pgfpathlineto{\pgfqpoint{2.034062in}{0.687877in}}%
\pgfpathlineto{\pgfqpoint{2.038466in}{0.719555in}}%
\pgfpathlineto{\pgfqpoint{2.042869in}{0.700576in}}%
\pgfpathlineto{\pgfqpoint{2.047273in}{0.689391in}}%
\pgfpathlineto{\pgfqpoint{2.051676in}{0.636057in}}%
\pgfpathlineto{\pgfqpoint{2.056080in}{0.721202in}}%
\pgfpathlineto{\pgfqpoint{2.060483in}{0.648117in}}%
\pgfpathlineto{\pgfqpoint{2.064887in}{0.729281in}}%
\pgfpathlineto{\pgfqpoint{2.069290in}{0.721877in}}%
\pgfpathlineto{\pgfqpoint{2.073693in}{0.819586in}}%
\pgfpathlineto{\pgfqpoint{2.078097in}{0.731079in}}%
\pgfpathlineto{\pgfqpoint{2.082500in}{0.754630in}}%
\pgfpathlineto{\pgfqpoint{2.086904in}{0.903944in}}%
\pgfpathlineto{\pgfqpoint{2.091307in}{0.780270in}}%
\pgfpathlineto{\pgfqpoint{2.095711in}{0.720020in}}%
\pgfpathlineto{\pgfqpoint{2.100114in}{0.712131in}}%
\pgfpathlineto{\pgfqpoint{2.104518in}{0.661235in}}%
\pgfpathlineto{\pgfqpoint{2.108921in}{0.657260in}}%
\pgfpathlineto{\pgfqpoint{2.113325in}{0.688279in}}%
\pgfpathlineto{\pgfqpoint{2.117728in}{0.676384in}}%
\pgfpathlineto{\pgfqpoint{2.122132in}{0.633141in}}%
\pgfpathlineto{\pgfqpoint{2.126535in}{0.726815in}}%
\pgfpathlineto{\pgfqpoint{2.130938in}{0.664866in}}%
\pgfpathlineto{\pgfqpoint{2.135342in}{0.670974in}}%
\pgfpathlineto{\pgfqpoint{2.139745in}{0.749450in}}%
\pgfpathlineto{\pgfqpoint{2.144149in}{0.773714in}}%
\pgfpathlineto{\pgfqpoint{2.148552in}{0.824813in}}%
\pgfpathlineto{\pgfqpoint{2.152956in}{0.745122in}}%
\pgfpathlineto{\pgfqpoint{2.157359in}{0.785288in}}%
\pgfpathlineto{\pgfqpoint{2.161763in}{0.620853in}}%
\pgfpathlineto{\pgfqpoint{2.166166in}{0.622749in}}%
\pgfpathlineto{\pgfqpoint{2.170570in}{0.599836in}}%
\pgfpathlineto{\pgfqpoint{2.174973in}{0.696346in}}%
\pgfpathlineto{\pgfqpoint{2.179377in}{0.707438in}}%
\pgfpathlineto{\pgfqpoint{2.183780in}{0.795471in}}%
\pgfpathlineto{\pgfqpoint{2.188184in}{0.560799in}}%
\pgfpathlineto{\pgfqpoint{2.192587in}{0.766397in}}%
\pgfpathlineto{\pgfqpoint{2.196990in}{0.752870in}}%
\pgfpathlineto{\pgfqpoint{2.201394in}{0.700012in}}%
\pgfpathlineto{\pgfqpoint{2.205797in}{0.785487in}}%
\pgfpathlineto{\pgfqpoint{2.210201in}{0.820853in}}%
\pgfpathlineto{\pgfqpoint{2.214604in}{0.741101in}}%
\pgfpathlineto{\pgfqpoint{2.219008in}{0.591460in}}%
\pgfpathlineto{\pgfqpoint{2.223411in}{0.767640in}}%
\pgfpathlineto{\pgfqpoint{2.227815in}{0.743705in}}%
\pgfpathlineto{\pgfqpoint{2.232218in}{0.705710in}}%
\pgfpathlineto{\pgfqpoint{2.236622in}{0.704058in}}%
\pgfpathlineto{\pgfqpoint{2.241025in}{0.648193in}}%
\pgfpathlineto{\pgfqpoint{2.245429in}{0.710910in}}%
\pgfpathlineto{\pgfqpoint{2.249832in}{0.614779in}}%
\pgfpathlineto{\pgfqpoint{2.254235in}{0.730568in}}%
\pgfpathlineto{\pgfqpoint{2.258639in}{0.719790in}}%
\pgfpathlineto{\pgfqpoint{2.263042in}{0.678592in}}%
\pgfpathlineto{\pgfqpoint{2.267446in}{0.834380in}}%
\pgfpathlineto{\pgfqpoint{2.271849in}{0.618892in}}%
\pgfpathlineto{\pgfqpoint{2.276253in}{0.595143in}}%
\pgfpathlineto{\pgfqpoint{2.280656in}{0.681054in}}%
\pgfpathlineto{\pgfqpoint{2.285060in}{0.579766in}}%
\pgfpathlineto{\pgfqpoint{2.289463in}{0.672413in}}%
\pgfpathlineto{\pgfqpoint{2.293867in}{0.722631in}}%
\pgfpathlineto{\pgfqpoint{2.298270in}{0.716615in}}%
\pgfpathlineto{\pgfqpoint{2.302674in}{0.726893in}}%
\pgfpathlineto{\pgfqpoint{2.307077in}{0.706952in}}%
\pgfpathlineto{\pgfqpoint{2.311481in}{0.724712in}}%
\pgfpathlineto{\pgfqpoint{2.315884in}{0.785317in}}%
\pgfpathlineto{\pgfqpoint{2.320287in}{0.936127in}}%
\pgfpathlineto{\pgfqpoint{2.324691in}{0.636405in}}%
\pgfpathlineto{\pgfqpoint{2.329094in}{0.625359in}}%
\pgfpathlineto{\pgfqpoint{2.333498in}{0.789330in}}%
\pgfpathlineto{\pgfqpoint{2.337901in}{0.699447in}}%
\pgfpathlineto{\pgfqpoint{2.342305in}{0.745264in}}%
\pgfpathlineto{\pgfqpoint{2.346708in}{0.702240in}}%
\pgfpathlineto{\pgfqpoint{2.351112in}{0.646082in}}%
\pgfpathlineto{\pgfqpoint{2.355515in}{0.665836in}}%
\pgfpathlineto{\pgfqpoint{2.359919in}{0.636711in}}%
\pgfpathlineto{\pgfqpoint{2.364322in}{0.641947in}}%
\pgfpathlineto{\pgfqpoint{2.368726in}{0.629786in}}%
\pgfpathlineto{\pgfqpoint{2.373129in}{0.683585in}}%
\pgfpathlineto{\pgfqpoint{2.377532in}{0.797826in}}%
\pgfpathlineto{\pgfqpoint{2.381936in}{0.642034in}}%
\pgfpathlineto{\pgfqpoint{2.386339in}{0.664319in}}%
\pgfpathlineto{\pgfqpoint{2.390743in}{0.677901in}}%
\pgfpathlineto{\pgfqpoint{2.395146in}{0.593600in}}%
\pgfpathlineto{\pgfqpoint{2.399550in}{0.697048in}}%
\pgfpathlineto{\pgfqpoint{2.403953in}{0.636085in}}%
\pgfpathlineto{\pgfqpoint{2.412760in}{0.765825in}}%
\pgfpathlineto{\pgfqpoint{2.417164in}{0.766936in}}%
\pgfpathlineto{\pgfqpoint{2.421567in}{0.774759in}}%
\pgfpathlineto{\pgfqpoint{2.425971in}{0.679833in}}%
\pgfpathlineto{\pgfqpoint{2.430374in}{0.889316in}}%
\pgfpathlineto{\pgfqpoint{2.434778in}{0.689197in}}%
\pgfpathlineto{\pgfqpoint{2.439181in}{0.662853in}}%
\pgfpathlineto{\pgfqpoint{2.443584in}{0.680250in}}%
\pgfpathlineto{\pgfqpoint{2.447988in}{1.041760in}}%
\pgfpathlineto{\pgfqpoint{2.452391in}{0.698309in}}%
\pgfpathlineto{\pgfqpoint{2.456795in}{0.693048in}}%
\pgfpathlineto{\pgfqpoint{2.461198in}{0.599978in}}%
\pgfpathlineto{\pgfqpoint{2.465602in}{0.727152in}}%
\pgfpathlineto{\pgfqpoint{2.470005in}{0.730371in}}%
\pgfpathlineto{\pgfqpoint{2.478812in}{0.723293in}}%
\pgfpathlineto{\pgfqpoint{2.492023in}{0.673641in}}%
\pgfpathlineto{\pgfqpoint{2.496426in}{0.674479in}}%
\pgfpathlineto{\pgfqpoint{2.505233in}{0.599379in}}%
\pgfpathlineto{\pgfqpoint{2.509636in}{0.653219in}}%
\pgfpathlineto{\pgfqpoint{2.514040in}{0.667810in}}%
\pgfpathlineto{\pgfqpoint{2.518443in}{0.659984in}}%
\pgfpathlineto{\pgfqpoint{2.522847in}{0.718528in}}%
\pgfpathlineto{\pgfqpoint{2.527250in}{0.704123in}}%
\pgfpathlineto{\pgfqpoint{2.531654in}{0.676348in}}%
\pgfpathlineto{\pgfqpoint{2.536057in}{0.634188in}}%
\pgfpathlineto{\pgfqpoint{2.540461in}{0.756472in}}%
\pgfpathlineto{\pgfqpoint{2.544864in}{0.840145in}}%
\pgfpathlineto{\pgfqpoint{2.549268in}{0.614145in}}%
\pgfpathlineto{\pgfqpoint{2.553671in}{0.650532in}}%
\pgfpathlineto{\pgfqpoint{2.558075in}{1.210267in}}%
\pgfpathlineto{\pgfqpoint{2.562478in}{0.691732in}}%
\pgfpathlineto{\pgfqpoint{2.566881in}{0.678856in}}%
\pgfpathlineto{\pgfqpoint{2.571285in}{0.656843in}}%
\pgfpathlineto{\pgfqpoint{2.575688in}{0.700674in}}%
\pgfpathlineto{\pgfqpoint{2.580092in}{0.703868in}}%
\pgfpathlineto{\pgfqpoint{2.584495in}{0.714962in}}%
\pgfpathlineto{\pgfqpoint{2.588899in}{0.719415in}}%
\pgfpathlineto{\pgfqpoint{2.593302in}{0.630294in}}%
\pgfpathlineto{\pgfqpoint{2.597706in}{0.691876in}}%
\pgfpathlineto{\pgfqpoint{2.602109in}{0.711614in}}%
\pgfpathlineto{\pgfqpoint{2.606513in}{0.685869in}}%
\pgfpathlineto{\pgfqpoint{2.610916in}{0.614707in}}%
\pgfpathlineto{\pgfqpoint{2.615320in}{0.611528in}}%
\pgfpathlineto{\pgfqpoint{2.619723in}{0.699464in}}%
\pgfpathlineto{\pgfqpoint{2.624126in}{0.810817in}}%
\pgfpathlineto{\pgfqpoint{2.628530in}{0.616528in}}%
\pgfpathlineto{\pgfqpoint{2.632933in}{0.725918in}}%
\pgfpathlineto{\pgfqpoint{2.637337in}{0.714218in}}%
\pgfpathlineto{\pgfqpoint{2.641740in}{0.680256in}}%
\pgfpathlineto{\pgfqpoint{2.646144in}{0.711241in}}%
\pgfpathlineto{\pgfqpoint{2.650547in}{0.718935in}}%
\pgfpathlineto{\pgfqpoint{2.654951in}{0.825260in}}%
\pgfpathlineto{\pgfqpoint{2.659354in}{0.650100in}}%
\pgfpathlineto{\pgfqpoint{2.663758in}{0.648205in}}%
\pgfpathlineto{\pgfqpoint{2.668161in}{1.105398in}}%
\pgfpathlineto{\pgfqpoint{2.672565in}{0.652221in}}%
\pgfpathlineto{\pgfqpoint{2.676968in}{0.786244in}}%
\pgfpathlineto{\pgfqpoint{2.681372in}{0.611674in}}%
\pgfpathlineto{\pgfqpoint{2.685775in}{0.644430in}}%
\pgfpathlineto{\pgfqpoint{2.690178in}{0.738853in}}%
\pgfpathlineto{\pgfqpoint{2.694582in}{0.654088in}}%
\pgfpathlineto{\pgfqpoint{2.698985in}{0.622542in}}%
\pgfpathlineto{\pgfqpoint{2.703389in}{0.661811in}}%
\pgfpathlineto{\pgfqpoint{2.707792in}{0.772513in}}%
\pgfpathlineto{\pgfqpoint{2.712196in}{0.613455in}}%
\pgfpathlineto{\pgfqpoint{2.716599in}{0.623163in}}%
\pgfpathlineto{\pgfqpoint{2.721003in}{0.680109in}}%
\pgfpathlineto{\pgfqpoint{2.725406in}{0.610694in}}%
\pgfpathlineto{\pgfqpoint{2.729810in}{0.683994in}}%
\pgfpathlineto{\pgfqpoint{2.734213in}{0.655936in}}%
\pgfpathlineto{\pgfqpoint{2.738617in}{0.666953in}}%
\pgfpathlineto{\pgfqpoint{2.743020in}{0.780031in}}%
\pgfpathlineto{\pgfqpoint{2.747423in}{0.815272in}}%
\pgfpathlineto{\pgfqpoint{2.751827in}{0.752761in}}%
\pgfpathlineto{\pgfqpoint{2.756230in}{0.718193in}}%
\pgfpathlineto{\pgfqpoint{2.760634in}{0.697597in}}%
\pgfpathlineto{\pgfqpoint{2.765037in}{0.788060in}}%
\pgfpathlineto{\pgfqpoint{2.769441in}{0.693350in}}%
\pgfpathlineto{\pgfqpoint{2.773844in}{0.651318in}}%
\pgfpathlineto{\pgfqpoint{2.778248in}{0.639649in}}%
\pgfpathlineto{\pgfqpoint{2.782651in}{0.657946in}}%
\pgfpathlineto{\pgfqpoint{2.787055in}{0.778717in}}%
\pgfpathlineto{\pgfqpoint{2.791458in}{0.656649in}}%
\pgfpathlineto{\pgfqpoint{2.795862in}{0.689255in}}%
\pgfpathlineto{\pgfqpoint{2.800265in}{0.774759in}}%
\pgfpathlineto{\pgfqpoint{2.804669in}{0.683692in}}%
\pgfpathlineto{\pgfqpoint{2.809072in}{0.735951in}}%
\pgfpathlineto{\pgfqpoint{2.813475in}{0.690519in}}%
\pgfpathlineto{\pgfqpoint{2.817879in}{0.774963in}}%
\pgfpathlineto{\pgfqpoint{2.822282in}{0.632321in}}%
\pgfpathlineto{\pgfqpoint{2.826686in}{0.654490in}}%
\pgfpathlineto{\pgfqpoint{2.831089in}{0.665158in}}%
\pgfpathlineto{\pgfqpoint{2.835493in}{0.602658in}}%
\pgfpathlineto{\pgfqpoint{2.839896in}{0.687772in}}%
\pgfpathlineto{\pgfqpoint{2.844300in}{0.734149in}}%
\pgfpathlineto{\pgfqpoint{2.848703in}{0.697297in}}%
\pgfpathlineto{\pgfqpoint{2.853107in}{0.696059in}}%
\pgfpathlineto{\pgfqpoint{2.857510in}{0.769314in}}%
\pgfpathlineto{\pgfqpoint{2.861914in}{0.664816in}}%
\pgfpathlineto{\pgfqpoint{2.866317in}{0.623348in}}%
\pgfpathlineto{\pgfqpoint{2.870720in}{0.817748in}}%
\pgfpathlineto{\pgfqpoint{2.875124in}{0.691696in}}%
\pgfpathlineto{\pgfqpoint{2.879527in}{0.705039in}}%
\pgfpathlineto{\pgfqpoint{2.883931in}{0.650945in}}%
\pgfpathlineto{\pgfqpoint{2.888334in}{0.686731in}}%
\pgfpathlineto{\pgfqpoint{2.892738in}{0.681708in}}%
\pgfpathlineto{\pgfqpoint{2.897141in}{0.662597in}}%
\pgfpathlineto{\pgfqpoint{2.901545in}{0.671710in}}%
\pgfpathlineto{\pgfqpoint{2.905948in}{0.665623in}}%
\pgfpathlineto{\pgfqpoint{2.910352in}{0.662882in}}%
\pgfpathlineto{\pgfqpoint{2.919159in}{0.767876in}}%
\pgfpathlineto{\pgfqpoint{2.927966in}{0.649457in}}%
\pgfpathlineto{\pgfqpoint{2.932369in}{0.692421in}}%
\pgfpathlineto{\pgfqpoint{2.936772in}{0.646849in}}%
\pgfpathlineto{\pgfqpoint{2.945579in}{0.627975in}}%
\pgfpathlineto{\pgfqpoint{2.949983in}{0.667249in}}%
\pgfpathlineto{\pgfqpoint{2.954386in}{0.687948in}}%
\pgfpathlineto{\pgfqpoint{2.958790in}{0.647691in}}%
\pgfpathlineto{\pgfqpoint{2.963193in}{0.863291in}}%
\pgfpathlineto{\pgfqpoint{2.967597in}{0.627467in}}%
\pgfpathlineto{\pgfqpoint{2.972000in}{0.638818in}}%
\pgfpathlineto{\pgfqpoint{2.976404in}{0.705347in}}%
\pgfpathlineto{\pgfqpoint{2.980807in}{0.715914in}}%
\pgfpathlineto{\pgfqpoint{2.989614in}{0.612952in}}%
\pgfpathlineto{\pgfqpoint{2.994017in}{0.638952in}}%
\pgfpathlineto{\pgfqpoint{2.998421in}{0.652657in}}%
\pgfpathlineto{\pgfqpoint{3.002824in}{0.660223in}}%
\pgfpathlineto{\pgfqpoint{3.007228in}{0.795488in}}%
\pgfpathlineto{\pgfqpoint{3.011631in}{0.539219in}}%
\pgfpathlineto{\pgfqpoint{3.016035in}{0.741767in}}%
\pgfpathlineto{\pgfqpoint{3.020438in}{0.702590in}}%
\pgfpathlineto{\pgfqpoint{3.024842in}{0.626276in}}%
\pgfpathlineto{\pgfqpoint{3.029245in}{0.633022in}}%
\pgfpathlineto{\pgfqpoint{3.033649in}{0.648803in}}%
\pgfpathlineto{\pgfqpoint{3.038052in}{0.607545in}}%
\pgfpathlineto{\pgfqpoint{3.042456in}{0.612196in}}%
\pgfpathlineto{\pgfqpoint{3.046859in}{0.654256in}}%
\pgfpathlineto{\pgfqpoint{3.051263in}{0.676080in}}%
\pgfpathlineto{\pgfqpoint{3.055666in}{0.605307in}}%
\pgfpathlineto{\pgfqpoint{3.060069in}{0.618048in}}%
\pgfpathlineto{\pgfqpoint{3.064473in}{0.694577in}}%
\pgfpathlineto{\pgfqpoint{3.068876in}{0.622210in}}%
\pgfpathlineto{\pgfqpoint{3.073280in}{0.792184in}}%
\pgfpathlineto{\pgfqpoint{3.077683in}{0.716680in}}%
\pgfpathlineto{\pgfqpoint{3.082087in}{0.712192in}}%
\pgfpathlineto{\pgfqpoint{3.086490in}{0.824040in}}%
\pgfpathlineto{\pgfqpoint{3.090894in}{0.647843in}}%
\pgfpathlineto{\pgfqpoint{3.095297in}{0.691744in}}%
\pgfpathlineto{\pgfqpoint{3.099701in}{0.642048in}}%
\pgfpathlineto{\pgfqpoint{3.104104in}{0.623270in}}%
\pgfpathlineto{\pgfqpoint{3.108508in}{0.806434in}}%
\pgfpathlineto{\pgfqpoint{3.112911in}{0.616447in}}%
\pgfpathlineto{\pgfqpoint{3.117314in}{0.763533in}}%
\pgfpathlineto{\pgfqpoint{3.126121in}{0.758207in}}%
\pgfpathlineto{\pgfqpoint{3.130525in}{0.795084in}}%
\pgfpathlineto{\pgfqpoint{3.134928in}{0.652710in}}%
\pgfpathlineto{\pgfqpoint{3.143735in}{0.752857in}}%
\pgfpathlineto{\pgfqpoint{3.148139in}{0.625374in}}%
\pgfpathlineto{\pgfqpoint{3.152542in}{0.605789in}}%
\pgfpathlineto{\pgfqpoint{3.156946in}{0.641716in}}%
\pgfpathlineto{\pgfqpoint{3.161349in}{0.709244in}}%
\pgfpathlineto{\pgfqpoint{3.165753in}{0.599887in}}%
\pgfpathlineto{\pgfqpoint{3.170156in}{0.709799in}}%
\pgfpathlineto{\pgfqpoint{3.174559in}{0.678555in}}%
\pgfpathlineto{\pgfqpoint{3.178963in}{0.738662in}}%
\pgfpathlineto{\pgfqpoint{3.183366in}{0.731452in}}%
\pgfpathlineto{\pgfqpoint{3.187770in}{0.746505in}}%
\pgfpathlineto{\pgfqpoint{3.192173in}{0.681394in}}%
\pgfpathlineto{\pgfqpoint{3.196577in}{0.728035in}}%
\pgfpathlineto{\pgfqpoint{3.200980in}{0.668504in}}%
\pgfpathlineto{\pgfqpoint{3.205384in}{0.749411in}}%
\pgfpathlineto{\pgfqpoint{3.209787in}{0.662694in}}%
\pgfpathlineto{\pgfqpoint{3.214191in}{0.672116in}}%
\pgfpathlineto{\pgfqpoint{3.218594in}{0.749272in}}%
\pgfpathlineto{\pgfqpoint{3.222998in}{0.641397in}}%
\pgfpathlineto{\pgfqpoint{3.227401in}{0.758140in}}%
\pgfpathlineto{\pgfqpoint{3.231805in}{0.799471in}}%
\pgfpathlineto{\pgfqpoint{3.236208in}{0.633990in}}%
\pgfpathlineto{\pgfqpoint{3.240611in}{0.733415in}}%
\pgfpathlineto{\pgfqpoint{3.245015in}{0.743427in}}%
\pgfpathlineto{\pgfqpoint{3.249418in}{0.803775in}}%
\pgfpathlineto{\pgfqpoint{3.258225in}{0.651886in}}%
\pgfpathlineto{\pgfqpoint{3.262629in}{0.636492in}}%
\pgfpathlineto{\pgfqpoint{3.267032in}{0.657673in}}%
\pgfpathlineto{\pgfqpoint{3.271436in}{0.667027in}}%
\pgfpathlineto{\pgfqpoint{3.275839in}{0.629025in}}%
\pgfpathlineto{\pgfqpoint{3.280243in}{0.709789in}}%
\pgfpathlineto{\pgfqpoint{3.284646in}{0.641601in}}%
\pgfpathlineto{\pgfqpoint{3.289050in}{0.726497in}}%
\pgfpathlineto{\pgfqpoint{3.293453in}{0.705399in}}%
\pgfpathlineto{\pgfqpoint{3.297856in}{0.706319in}}%
\pgfpathlineto{\pgfqpoint{3.302260in}{0.642163in}}%
\pgfpathlineto{\pgfqpoint{3.306663in}{0.680679in}}%
\pgfpathlineto{\pgfqpoint{3.311067in}{0.676218in}}%
\pgfpathlineto{\pgfqpoint{3.315470in}{0.663917in}}%
\pgfpathlineto{\pgfqpoint{3.319874in}{0.590335in}}%
\pgfpathlineto{\pgfqpoint{3.324277in}{0.730849in}}%
\pgfpathlineto{\pgfqpoint{3.328681in}{0.580370in}}%
\pgfpathlineto{\pgfqpoint{3.337488in}{0.778830in}}%
\pgfpathlineto{\pgfqpoint{3.341891in}{0.558922in}}%
\pgfpathlineto{\pgfqpoint{3.346295in}{0.651465in}}%
\pgfpathlineto{\pgfqpoint{3.350698in}{0.622970in}}%
\pgfpathlineto{\pgfqpoint{3.355102in}{0.670974in}}%
\pgfpathlineto{\pgfqpoint{3.359505in}{0.692193in}}%
\pgfpathlineto{\pgfqpoint{3.363908in}{0.634448in}}%
\pgfpathlineto{\pgfqpoint{3.372715in}{0.639859in}}%
\pgfpathlineto{\pgfqpoint{3.377119in}{0.633285in}}%
\pgfpathlineto{\pgfqpoint{3.381522in}{0.593817in}}%
\pgfpathlineto{\pgfqpoint{3.385926in}{0.608505in}}%
\pgfpathlineto{\pgfqpoint{3.390329in}{0.665541in}}%
\pgfpathlineto{\pgfqpoint{3.394733in}{0.757733in}}%
\pgfpathlineto{\pgfqpoint{3.399136in}{0.651894in}}%
\pgfpathlineto{\pgfqpoint{3.403540in}{0.756796in}}%
\pgfpathlineto{\pgfqpoint{3.407943in}{0.692822in}}%
\pgfpathlineto{\pgfqpoint{3.412347in}{0.712678in}}%
\pgfpathlineto{\pgfqpoint{3.416750in}{0.754488in}}%
\pgfpathlineto{\pgfqpoint{3.421153in}{0.636585in}}%
\pgfpathlineto{\pgfqpoint{3.425557in}{0.646037in}}%
\pgfpathlineto{\pgfqpoint{3.429960in}{0.604129in}}%
\pgfpathlineto{\pgfqpoint{3.434364in}{0.772478in}}%
\pgfpathlineto{\pgfqpoint{3.438767in}{0.586222in}}%
\pgfpathlineto{\pgfqpoint{3.443171in}{0.622131in}}%
\pgfpathlineto{\pgfqpoint{3.447574in}{0.727905in}}%
\pgfpathlineto{\pgfqpoint{3.451978in}{0.646557in}}%
\pgfpathlineto{\pgfqpoint{3.456381in}{0.651142in}}%
\pgfpathlineto{\pgfqpoint{3.460785in}{0.734543in}}%
\pgfpathlineto{\pgfqpoint{3.465188in}{0.638389in}}%
\pgfpathlineto{\pgfqpoint{3.469592in}{0.702759in}}%
\pgfpathlineto{\pgfqpoint{3.473995in}{0.739423in}}%
\pgfpathlineto{\pgfqpoint{3.478399in}{0.647974in}}%
\pgfpathlineto{\pgfqpoint{3.487205in}{0.580305in}}%
\pgfpathlineto{\pgfqpoint{3.491609in}{0.610053in}}%
\pgfpathlineto{\pgfqpoint{3.496012in}{0.663035in}}%
\pgfpathlineto{\pgfqpoint{3.500416in}{0.688252in}}%
\pgfpathlineto{\pgfqpoint{3.504819in}{0.825295in}}%
\pgfpathlineto{\pgfqpoint{3.509223in}{0.623623in}}%
\pgfpathlineto{\pgfqpoint{3.513626in}{0.700943in}}%
\pgfpathlineto{\pgfqpoint{3.518030in}{0.697083in}}%
\pgfpathlineto{\pgfqpoint{3.522433in}{0.701576in}}%
\pgfpathlineto{\pgfqpoint{3.526837in}{0.724586in}}%
\pgfpathlineto{\pgfqpoint{3.531240in}{0.655777in}}%
\pgfpathlineto{\pgfqpoint{3.535644in}{0.675573in}}%
\pgfpathlineto{\pgfqpoint{3.540047in}{0.658532in}}%
\pgfpathlineto{\pgfqpoint{3.544450in}{0.656563in}}%
\pgfpathlineto{\pgfqpoint{3.548854in}{0.602755in}}%
\pgfpathlineto{\pgfqpoint{3.553257in}{0.650832in}}%
\pgfpathlineto{\pgfqpoint{3.557661in}{0.638587in}}%
\pgfpathlineto{\pgfqpoint{3.562064in}{0.607446in}}%
\pgfpathlineto{\pgfqpoint{3.570871in}{0.797826in}}%
\pgfpathlineto{\pgfqpoint{3.575275in}{0.654526in}}%
\pgfpathlineto{\pgfqpoint{3.579678in}{0.754282in}}%
\pgfpathlineto{\pgfqpoint{3.584082in}{0.812856in}}%
\pgfpathlineto{\pgfqpoint{3.588485in}{0.662838in}}%
\pgfpathlineto{\pgfqpoint{3.592889in}{0.635135in}}%
\pgfpathlineto{\pgfqpoint{3.597292in}{0.627948in}}%
\pgfpathlineto{\pgfqpoint{3.601696in}{0.640878in}}%
\pgfpathlineto{\pgfqpoint{3.606099in}{0.699032in}}%
\pgfpathlineto{\pgfqpoint{3.610502in}{0.667066in}}%
\pgfpathlineto{\pgfqpoint{3.614906in}{0.745275in}}%
\pgfpathlineto{\pgfqpoint{3.619309in}{0.694081in}}%
\pgfpathlineto{\pgfqpoint{3.623713in}{0.628151in}}%
\pgfpathlineto{\pgfqpoint{3.628116in}{0.628510in}}%
\pgfpathlineto{\pgfqpoint{3.632520in}{0.649306in}}%
\pgfpathlineto{\pgfqpoint{3.636923in}{0.649053in}}%
\pgfpathlineto{\pgfqpoint{3.641327in}{0.680088in}}%
\pgfpathlineto{\pgfqpoint{3.645730in}{0.685015in}}%
\pgfpathlineto{\pgfqpoint{3.650134in}{0.650148in}}%
\pgfpathlineto{\pgfqpoint{3.654537in}{0.636088in}}%
\pgfpathlineto{\pgfqpoint{3.658941in}{0.637814in}}%
\pgfpathlineto{\pgfqpoint{3.663344in}{0.674833in}}%
\pgfpathlineto{\pgfqpoint{3.667747in}{0.694043in}}%
\pgfpathlineto{\pgfqpoint{3.672151in}{0.643251in}}%
\pgfpathlineto{\pgfqpoint{3.676554in}{0.662851in}}%
\pgfpathlineto{\pgfqpoint{3.680958in}{0.728691in}}%
\pgfpathlineto{\pgfqpoint{3.685361in}{0.664898in}}%
\pgfpathlineto{\pgfqpoint{3.689765in}{0.694841in}}%
\pgfpathlineto{\pgfqpoint{3.694168in}{0.690965in}}%
\pgfpathlineto{\pgfqpoint{3.698572in}{0.658182in}}%
\pgfpathlineto{\pgfqpoint{3.702975in}{0.682276in}}%
\pgfpathlineto{\pgfqpoint{3.707379in}{0.613211in}}%
\pgfpathlineto{\pgfqpoint{3.711782in}{0.644418in}}%
\pgfpathlineto{\pgfqpoint{3.716186in}{0.647455in}}%
\pgfpathlineto{\pgfqpoint{3.720589in}{0.624899in}}%
\pgfpathlineto{\pgfqpoint{3.724993in}{0.783362in}}%
\pgfpathlineto{\pgfqpoint{3.729396in}{0.701238in}}%
\pgfpathlineto{\pgfqpoint{3.733799in}{0.804479in}}%
\pgfpathlineto{\pgfqpoint{3.738203in}{0.717458in}}%
\pgfpathlineto{\pgfqpoint{3.747010in}{0.756042in}}%
\pgfpathlineto{\pgfqpoint{3.751413in}{0.696530in}}%
\pgfpathlineto{\pgfqpoint{3.760220in}{0.662293in}}%
\pgfpathlineto{\pgfqpoint{3.764624in}{0.690049in}}%
\pgfpathlineto{\pgfqpoint{3.769027in}{0.649926in}}%
\pgfpathlineto{\pgfqpoint{3.773431in}{0.725975in}}%
\pgfpathlineto{\pgfqpoint{3.777834in}{0.893272in}}%
\pgfpathlineto{\pgfqpoint{3.782238in}{0.699717in}}%
\pgfpathlineto{\pgfqpoint{3.786641in}{0.690755in}}%
\pgfpathlineto{\pgfqpoint{3.791044in}{0.687325in}}%
\pgfpathlineto{\pgfqpoint{3.795448in}{0.673011in}}%
\pgfpathlineto{\pgfqpoint{3.799851in}{0.776044in}}%
\pgfpathlineto{\pgfqpoint{3.804255in}{0.702053in}}%
\pgfpathlineto{\pgfqpoint{3.808658in}{0.668321in}}%
\pgfpathlineto{\pgfqpoint{3.813062in}{0.654484in}}%
\pgfpathlineto{\pgfqpoint{3.817465in}{0.618979in}}%
\pgfpathlineto{\pgfqpoint{3.821869in}{0.626659in}}%
\pgfpathlineto{\pgfqpoint{3.826272in}{0.694910in}}%
\pgfpathlineto{\pgfqpoint{3.830676in}{0.662857in}}%
\pgfpathlineto{\pgfqpoint{3.835079in}{0.891181in}}%
\pgfpathlineto{\pgfqpoint{3.839483in}{0.640810in}}%
\pgfpathlineto{\pgfqpoint{3.843886in}{0.861852in}}%
\pgfpathlineto{\pgfqpoint{3.848290in}{0.841645in}}%
\pgfpathlineto{\pgfqpoint{3.852693in}{0.744254in}}%
\pgfpathlineto{\pgfqpoint{3.857096in}{0.724100in}}%
\pgfpathlineto{\pgfqpoint{3.861500in}{0.677576in}}%
\pgfpathlineto{\pgfqpoint{3.865903in}{0.722474in}}%
\pgfpathlineto{\pgfqpoint{3.870307in}{0.752596in}}%
\pgfpathlineto{\pgfqpoint{3.874710in}{0.670220in}}%
\pgfpathlineto{\pgfqpoint{3.879114in}{0.631469in}}%
\pgfpathlineto{\pgfqpoint{3.883517in}{0.682961in}}%
\pgfpathlineto{\pgfqpoint{3.887921in}{0.640047in}}%
\pgfpathlineto{\pgfqpoint{3.892324in}{0.624046in}}%
\pgfpathlineto{\pgfqpoint{3.901131in}{0.769033in}}%
\pgfpathlineto{\pgfqpoint{3.909938in}{0.793017in}}%
\pgfpathlineto{\pgfqpoint{3.914341in}{0.766929in}}%
\pgfpathlineto{\pgfqpoint{3.918745in}{0.657983in}}%
\pgfpathlineto{\pgfqpoint{3.923148in}{0.646445in}}%
\pgfpathlineto{\pgfqpoint{3.927552in}{0.627359in}}%
\pgfpathlineto{\pgfqpoint{3.931955in}{0.575226in}}%
\pgfpathlineto{\pgfqpoint{3.936359in}{0.775015in}}%
\pgfpathlineto{\pgfqpoint{3.940762in}{0.705860in}}%
\pgfpathlineto{\pgfqpoint{3.945166in}{0.878498in}}%
\pgfpathlineto{\pgfqpoint{3.949569in}{0.615327in}}%
\pgfpathlineto{\pgfqpoint{3.953973in}{0.723947in}}%
\pgfpathlineto{\pgfqpoint{3.958376in}{0.720552in}}%
\pgfpathlineto{\pgfqpoint{3.962780in}{0.660263in}}%
\pgfpathlineto{\pgfqpoint{3.967183in}{0.651618in}}%
\pgfpathlineto{\pgfqpoint{3.971587in}{0.635052in}}%
\pgfpathlineto{\pgfqpoint{3.975990in}{0.724177in}}%
\pgfpathlineto{\pgfqpoint{3.984797in}{0.644810in}}%
\pgfpathlineto{\pgfqpoint{3.989200in}{0.663081in}}%
\pgfpathlineto{\pgfqpoint{3.993604in}{0.669470in}}%
\pgfpathlineto{\pgfqpoint{3.998007in}{0.730457in}}%
\pgfpathlineto{\pgfqpoint{4.002411in}{0.648956in}}%
\pgfpathlineto{\pgfqpoint{4.006814in}{0.666763in}}%
\pgfpathlineto{\pgfqpoint{4.011218in}{0.658568in}}%
\pgfpathlineto{\pgfqpoint{4.015621in}{0.745966in}}%
\pgfpathlineto{\pgfqpoint{4.020025in}{0.681256in}}%
\pgfpathlineto{\pgfqpoint{4.024428in}{0.645097in}}%
\pgfpathlineto{\pgfqpoint{4.028832in}{0.652563in}}%
\pgfpathlineto{\pgfqpoint{4.033235in}{0.635781in}}%
\pgfpathlineto{\pgfqpoint{4.037638in}{0.610104in}}%
\pgfpathlineto{\pgfqpoint{4.042042in}{0.618337in}}%
\pgfpathlineto{\pgfqpoint{4.046445in}{1.054552in}}%
\pgfpathlineto{\pgfqpoint{4.050849in}{0.682238in}}%
\pgfpathlineto{\pgfqpoint{4.055252in}{0.778305in}}%
\pgfpathlineto{\pgfqpoint{4.059656in}{0.618256in}}%
\pgfpathlineto{\pgfqpoint{4.064059in}{0.727628in}}%
\pgfpathlineto{\pgfqpoint{4.068463in}{0.645226in}}%
\pgfpathlineto{\pgfqpoint{4.072866in}{0.643903in}}%
\pgfpathlineto{\pgfqpoint{4.077270in}{0.738201in}}%
\pgfpathlineto{\pgfqpoint{4.081673in}{0.634216in}}%
\pgfpathlineto{\pgfqpoint{4.086077in}{0.693754in}}%
\pgfpathlineto{\pgfqpoint{4.090480in}{0.636387in}}%
\pgfpathlineto{\pgfqpoint{4.094884in}{0.623917in}}%
\pgfpathlineto{\pgfqpoint{4.099287in}{0.644212in}}%
\pgfpathlineto{\pgfqpoint{4.103690in}{0.638543in}}%
\pgfpathlineto{\pgfqpoint{4.108094in}{0.688773in}}%
\pgfpathlineto{\pgfqpoint{4.112497in}{0.688853in}}%
\pgfpathlineto{\pgfqpoint{4.116901in}{0.640572in}}%
\pgfpathlineto{\pgfqpoint{4.121304in}{0.614779in}}%
\pgfpathlineto{\pgfqpoint{4.125708in}{0.601270in}}%
\pgfpathlineto{\pgfqpoint{4.130111in}{0.620830in}}%
\pgfpathlineto{\pgfqpoint{4.134515in}{0.649506in}}%
\pgfpathlineto{\pgfqpoint{4.138918in}{0.625964in}}%
\pgfpathlineto{\pgfqpoint{4.143322in}{0.622145in}}%
\pgfpathlineto{\pgfqpoint{4.147725in}{0.608408in}}%
\pgfpathlineto{\pgfqpoint{4.152129in}{0.681180in}}%
\pgfpathlineto{\pgfqpoint{4.156532in}{1.165111in}}%
\pgfpathlineto{\pgfqpoint{4.160935in}{0.659161in}}%
\pgfpathlineto{\pgfqpoint{4.165339in}{0.622922in}}%
\pgfpathlineto{\pgfqpoint{4.169742in}{0.642027in}}%
\pgfpathlineto{\pgfqpoint{4.174146in}{0.764759in}}%
\pgfpathlineto{\pgfqpoint{4.178549in}{0.680187in}}%
\pgfpathlineto{\pgfqpoint{4.182953in}{0.715048in}}%
\pgfpathlineto{\pgfqpoint{4.187356in}{0.803041in}}%
\pgfpathlineto{\pgfqpoint{4.191760in}{0.641308in}}%
\pgfpathlineto{\pgfqpoint{4.200567in}{0.695774in}}%
\pgfpathlineto{\pgfqpoint{4.204970in}{0.651054in}}%
\pgfpathlineto{\pgfqpoint{4.209374in}{0.633602in}}%
\pgfpathlineto{\pgfqpoint{4.213777in}{0.676946in}}%
\pgfpathlineto{\pgfqpoint{4.218180in}{0.659419in}}%
\pgfpathlineto{\pgfqpoint{4.222584in}{0.609777in}}%
\pgfpathlineto{\pgfqpoint{4.226987in}{0.634151in}}%
\pgfpathlineto{\pgfqpoint{4.231391in}{0.740979in}}%
\pgfpathlineto{\pgfqpoint{4.235794in}{0.689054in}}%
\pgfpathlineto{\pgfqpoint{4.240198in}{0.701689in}}%
\pgfpathlineto{\pgfqpoint{4.244601in}{0.804158in}}%
\pgfpathlineto{\pgfqpoint{4.249005in}{0.611167in}}%
\pgfpathlineto{\pgfqpoint{4.253408in}{0.656426in}}%
\pgfpathlineto{\pgfqpoint{4.257812in}{0.639172in}}%
\pgfpathlineto{\pgfqpoint{4.262215in}{0.660401in}}%
\pgfpathlineto{\pgfqpoint{4.266619in}{1.304529in}}%
\pgfpathlineto{\pgfqpoint{4.271022in}{0.641482in}}%
\pgfpathlineto{\pgfqpoint{4.279829in}{0.700008in}}%
\pgfpathlineto{\pgfqpoint{4.284232in}{0.695388in}}%
\pgfpathlineto{\pgfqpoint{4.288636in}{0.665348in}}%
\pgfpathlineto{\pgfqpoint{4.293039in}{0.659697in}}%
\pgfpathlineto{\pgfqpoint{4.297443in}{0.745017in}}%
\pgfpathlineto{\pgfqpoint{4.301846in}{0.659098in}}%
\pgfpathlineto{\pgfqpoint{4.306250in}{0.675297in}}%
\pgfpathlineto{\pgfqpoint{4.310653in}{0.700123in}}%
\pgfpathlineto{\pgfqpoint{4.315057in}{0.656550in}}%
\pgfpathlineto{\pgfqpoint{4.319460in}{0.661608in}}%
\pgfpathlineto{\pgfqpoint{4.319460in}{0.661608in}}%
\pgfusepath{stroke}%
\end{pgfscope}%
\begin{pgfscope}%
\pgfpathrectangle{\pgfqpoint{0.625000in}{0.440000in}}{\pgfqpoint{3.875000in}{3.080000in}} %
\pgfusepath{clip}%
\pgfsetrectcap%
\pgfsetroundjoin%
\pgfsetlinewidth{1.505625pt}%
\definecolor{currentstroke}{rgb}{0.000000,0.000000,1.000000}%
\pgfsetstrokecolor{currentstroke}%
\pgfsetdash{}{0pt}%
\pgfpathmoveto{\pgfqpoint{0.801136in}{2.523609in}}%
\pgfpathlineto{\pgfqpoint{1.021266in}{0.992699in}}%
\pgfpathlineto{\pgfqpoint{1.241439in}{0.855940in}}%
\pgfpathlineto{\pgfqpoint{1.461612in}{0.804168in}}%
\pgfpathlineto{\pgfqpoint{1.681785in}{0.759697in}}%
\pgfpathlineto{\pgfqpoint{1.901958in}{0.718490in}}%
\pgfpathlineto{\pgfqpoint{2.122132in}{0.702749in}}%
\pgfpathlineto{\pgfqpoint{2.342305in}{0.695022in}}%
\pgfpathlineto{\pgfqpoint{2.562478in}{0.690125in}}%
\pgfpathlineto{\pgfqpoint{2.782651in}{0.689824in}}%
\pgfpathlineto{\pgfqpoint{3.002824in}{0.689396in}}%
\pgfpathlineto{\pgfqpoint{3.222998in}{0.689025in}}%
\pgfpathlineto{\pgfqpoint{3.443171in}{0.687625in}}%
\pgfpathlineto{\pgfqpoint{3.663344in}{0.680728in}}%
\pgfpathlineto{\pgfqpoint{3.883517in}{0.679174in}}%
\pgfpathlineto{\pgfqpoint{4.103690in}{0.678188in}}%
\pgfpathlineto{\pgfqpoint{4.323864in}{0.677521in}}%
\pgfusepath{stroke}%
\end{pgfscope}%
\begin{pgfscope}%
\pgfsetrectcap%
\pgfsetmiterjoin%
\pgfsetlinewidth{0.803000pt}%
\definecolor{currentstroke}{rgb}{0.000000,0.000000,0.000000}%
\pgfsetstrokecolor{currentstroke}%
\pgfsetdash{}{0pt}%
\pgfpathmoveto{\pgfqpoint{0.625000in}{0.440000in}}%
\pgfpathlineto{\pgfqpoint{0.625000in}{3.520000in}}%
\pgfusepath{stroke}%
\end{pgfscope}%
\begin{pgfscope}%
\pgfsetrectcap%
\pgfsetmiterjoin%
\pgfsetlinewidth{0.803000pt}%
\definecolor{currentstroke}{rgb}{0.000000,0.000000,0.000000}%
\pgfsetstrokecolor{currentstroke}%
\pgfsetdash{}{0pt}%
\pgfpathmoveto{\pgfqpoint{4.500000in}{0.440000in}}%
\pgfpathlineto{\pgfqpoint{4.500000in}{3.520000in}}%
\pgfusepath{stroke}%
\end{pgfscope}%
\begin{pgfscope}%
\pgfsetrectcap%
\pgfsetmiterjoin%
\pgfsetlinewidth{0.803000pt}%
\definecolor{currentstroke}{rgb}{0.000000,0.000000,0.000000}%
\pgfsetstrokecolor{currentstroke}%
\pgfsetdash{}{0pt}%
\pgfpathmoveto{\pgfqpoint{0.625000in}{0.440000in}}%
\pgfpathlineto{\pgfqpoint{4.500000in}{0.440000in}}%
\pgfusepath{stroke}%
\end{pgfscope}%
\begin{pgfscope}%
\pgfsetrectcap%
\pgfsetmiterjoin%
\pgfsetlinewidth{0.803000pt}%
\definecolor{currentstroke}{rgb}{0.000000,0.000000,0.000000}%
\pgfsetstrokecolor{currentstroke}%
\pgfsetdash{}{0pt}%
\pgfpathmoveto{\pgfqpoint{0.625000in}{3.520000in}}%
\pgfpathlineto{\pgfqpoint{4.500000in}{3.520000in}}%
\pgfusepath{stroke}%
\end{pgfscope}%
\end{pgfpicture}%
\makeatother%
\endgroup%
}
		\caption{\textbf{Unet\_Weighted\_3}}
	\end {subfigure}\hspace{1.3cm}
	\begin {subfigure}[b]{0.4\linewidth}
		\scalebox{0.70}{%% Creator: Matplotlib, PGF backend
%%
%% To include the figure in your LaTeX document, write
%%   \input{<filename>.pgf}
%%
%% Make sure the required packages are loaded in your preamble
%%   \usepackage{pgf}
%%
%% Figures using additional raster images can only be included by \input if
%% they are in the same directory as the main LaTeX file. For loading figures
%% from other directories you can use the `import` package
%%   \usepackage{import}
%% and then include the figures with
%%   \import{<path to file>}{<filename>.pgf}
%%
%% Matplotlib used the following preamble
%%   \usepackage{fontspec}
%%   \setmainfont{DejaVu Serif}
%%   \setsansfont{DejaVu Sans}
%%   \setmonofont{DejaVu Sans Mono}
%%
\begingroup%
\makeatletter%
\begin{pgfpicture}%
\pgfpathrectangle{\pgfpointorigin}{\pgfqpoint{5.000000in}{4.000000in}}%
\pgfusepath{use as bounding box, clip}%
\begin{pgfscope}%
\pgfsetbuttcap%
\pgfsetmiterjoin%
\definecolor{currentfill}{rgb}{1.000000,1.000000,1.000000}%
\pgfsetfillcolor{currentfill}%
\pgfsetlinewidth{0.000000pt}%
\definecolor{currentstroke}{rgb}{1.000000,1.000000,1.000000}%
\pgfsetstrokecolor{currentstroke}%
\pgfsetdash{}{0pt}%
\pgfpathmoveto{\pgfqpoint{0.000000in}{0.000000in}}%
\pgfpathlineto{\pgfqpoint{5.000000in}{0.000000in}}%
\pgfpathlineto{\pgfqpoint{5.000000in}{4.000000in}}%
\pgfpathlineto{\pgfqpoint{0.000000in}{4.000000in}}%
\pgfpathclose%
\pgfusepath{fill}%
\end{pgfscope}%
\begin{pgfscope}%
\pgfsetbuttcap%
\pgfsetmiterjoin%
\definecolor{currentfill}{rgb}{1.000000,1.000000,1.000000}%
\pgfsetfillcolor{currentfill}%
\pgfsetlinewidth{0.000000pt}%
\definecolor{currentstroke}{rgb}{0.000000,0.000000,0.000000}%
\pgfsetstrokecolor{currentstroke}%
\pgfsetstrokeopacity{0.000000}%
\pgfsetdash{}{0pt}%
\pgfpathmoveto{\pgfqpoint{0.625000in}{0.440000in}}%
\pgfpathlineto{\pgfqpoint{4.500000in}{0.440000in}}%
\pgfpathlineto{\pgfqpoint{4.500000in}{3.520000in}}%
\pgfpathlineto{\pgfqpoint{0.625000in}{3.520000in}}%
\pgfpathclose%
\pgfusepath{fill}%
\end{pgfscope}%
\begin{pgfscope}%
\pgfsetbuttcap%
\pgfsetroundjoin%
\definecolor{currentfill}{rgb}{0.000000,0.000000,0.000000}%
\pgfsetfillcolor{currentfill}%
\pgfsetlinewidth{0.803000pt}%
\definecolor{currentstroke}{rgb}{0.000000,0.000000,0.000000}%
\pgfsetstrokecolor{currentstroke}%
\pgfsetdash{}{0pt}%
\pgfsys@defobject{currentmarker}{\pgfqpoint{0.000000in}{-0.048611in}}{\pgfqpoint{0.000000in}{0.000000in}}{%
\pgfpathmoveto{\pgfqpoint{0.000000in}{0.000000in}}%
\pgfpathlineto{\pgfqpoint{0.000000in}{-0.048611in}}%
\pgfusepath{stroke,fill}%
}%
\begin{pgfscope}%
\pgfsys@transformshift{0.796722in}{0.440000in}%
\pgfsys@useobject{currentmarker}{}%
\end{pgfscope}%
\end{pgfscope}%
\begin{pgfscope}%
\pgftext[x=0.796722in,y=0.342778in,,top]{\sffamily\fontsize{10.000000}{12.000000}\selectfont 0}%
\end{pgfscope}%
\begin{pgfscope}%
\pgfsetbuttcap%
\pgfsetroundjoin%
\definecolor{currentfill}{rgb}{0.000000,0.000000,0.000000}%
\pgfsetfillcolor{currentfill}%
\pgfsetlinewidth{0.803000pt}%
\definecolor{currentstroke}{rgb}{0.000000,0.000000,0.000000}%
\pgfsetstrokecolor{currentstroke}%
\pgfsetdash{}{0pt}%
\pgfsys@defobject{currentmarker}{\pgfqpoint{0.000000in}{-0.048611in}}{\pgfqpoint{0.000000in}{0.000000in}}{%
\pgfpathmoveto{\pgfqpoint{0.000000in}{0.000000in}}%
\pgfpathlineto{\pgfqpoint{0.000000in}{-0.048611in}}%
\pgfusepath{stroke,fill}%
}%
\begin{pgfscope}%
\pgfsys@transformshift{1.385300in}{0.440000in}%
\pgfsys@useobject{currentmarker}{}%
\end{pgfscope}%
\end{pgfscope}%
\begin{pgfscope}%
\pgftext[x=1.385300in,y=0.342778in,,top]{\sffamily\fontsize{10.000000}{12.000000}\selectfont 5}%
\end{pgfscope}%
\begin{pgfscope}%
\pgfsetbuttcap%
\pgfsetroundjoin%
\definecolor{currentfill}{rgb}{0.000000,0.000000,0.000000}%
\pgfsetfillcolor{currentfill}%
\pgfsetlinewidth{0.803000pt}%
\definecolor{currentstroke}{rgb}{0.000000,0.000000,0.000000}%
\pgfsetstrokecolor{currentstroke}%
\pgfsetdash{}{0pt}%
\pgfsys@defobject{currentmarker}{\pgfqpoint{0.000000in}{-0.048611in}}{\pgfqpoint{0.000000in}{0.000000in}}{%
\pgfpathmoveto{\pgfqpoint{0.000000in}{0.000000in}}%
\pgfpathlineto{\pgfqpoint{0.000000in}{-0.048611in}}%
\pgfusepath{stroke,fill}%
}%
\begin{pgfscope}%
\pgfsys@transformshift{1.973878in}{0.440000in}%
\pgfsys@useobject{currentmarker}{}%
\end{pgfscope}%
\end{pgfscope}%
\begin{pgfscope}%
\pgftext[x=1.973878in,y=0.342778in,,top]{\sffamily\fontsize{10.000000}{12.000000}\selectfont 10}%
\end{pgfscope}%
\begin{pgfscope}%
\pgfsetbuttcap%
\pgfsetroundjoin%
\definecolor{currentfill}{rgb}{0.000000,0.000000,0.000000}%
\pgfsetfillcolor{currentfill}%
\pgfsetlinewidth{0.803000pt}%
\definecolor{currentstroke}{rgb}{0.000000,0.000000,0.000000}%
\pgfsetstrokecolor{currentstroke}%
\pgfsetdash{}{0pt}%
\pgfsys@defobject{currentmarker}{\pgfqpoint{0.000000in}{-0.048611in}}{\pgfqpoint{0.000000in}{0.000000in}}{%
\pgfpathmoveto{\pgfqpoint{0.000000in}{0.000000in}}%
\pgfpathlineto{\pgfqpoint{0.000000in}{-0.048611in}}%
\pgfusepath{stroke,fill}%
}%
\begin{pgfscope}%
\pgfsys@transformshift{2.562456in}{0.440000in}%
\pgfsys@useobject{currentmarker}{}%
\end{pgfscope}%
\end{pgfscope}%
\begin{pgfscope}%
\pgftext[x=2.562456in,y=0.342778in,,top]{\sffamily\fontsize{10.000000}{12.000000}\selectfont 15}%
\end{pgfscope}%
\begin{pgfscope}%
\pgfsetbuttcap%
\pgfsetroundjoin%
\definecolor{currentfill}{rgb}{0.000000,0.000000,0.000000}%
\pgfsetfillcolor{currentfill}%
\pgfsetlinewidth{0.803000pt}%
\definecolor{currentstroke}{rgb}{0.000000,0.000000,0.000000}%
\pgfsetstrokecolor{currentstroke}%
\pgfsetdash{}{0pt}%
\pgfsys@defobject{currentmarker}{\pgfqpoint{0.000000in}{-0.048611in}}{\pgfqpoint{0.000000in}{0.000000in}}{%
\pgfpathmoveto{\pgfqpoint{0.000000in}{0.000000in}}%
\pgfpathlineto{\pgfqpoint{0.000000in}{-0.048611in}}%
\pgfusepath{stroke,fill}%
}%
\begin{pgfscope}%
\pgfsys@transformshift{3.151034in}{0.440000in}%
\pgfsys@useobject{currentmarker}{}%
\end{pgfscope}%
\end{pgfscope}%
\begin{pgfscope}%
\pgftext[x=3.151034in,y=0.342778in,,top]{\sffamily\fontsize{10.000000}{12.000000}\selectfont 20}%
\end{pgfscope}%
\begin{pgfscope}%
\pgfsetbuttcap%
\pgfsetroundjoin%
\definecolor{currentfill}{rgb}{0.000000,0.000000,0.000000}%
\pgfsetfillcolor{currentfill}%
\pgfsetlinewidth{0.803000pt}%
\definecolor{currentstroke}{rgb}{0.000000,0.000000,0.000000}%
\pgfsetstrokecolor{currentstroke}%
\pgfsetdash{}{0pt}%
\pgfsys@defobject{currentmarker}{\pgfqpoint{0.000000in}{-0.048611in}}{\pgfqpoint{0.000000in}{0.000000in}}{%
\pgfpathmoveto{\pgfqpoint{0.000000in}{0.000000in}}%
\pgfpathlineto{\pgfqpoint{0.000000in}{-0.048611in}}%
\pgfusepath{stroke,fill}%
}%
\begin{pgfscope}%
\pgfsys@transformshift{3.739612in}{0.440000in}%
\pgfsys@useobject{currentmarker}{}%
\end{pgfscope}%
\end{pgfscope}%
\begin{pgfscope}%
\pgftext[x=3.739612in,y=0.342778in,,top]{\sffamily\fontsize{10.000000}{12.000000}\selectfont 25}%
\end{pgfscope}%
\begin{pgfscope}%
\pgfsetbuttcap%
\pgfsetroundjoin%
\definecolor{currentfill}{rgb}{0.000000,0.000000,0.000000}%
\pgfsetfillcolor{currentfill}%
\pgfsetlinewidth{0.803000pt}%
\definecolor{currentstroke}{rgb}{0.000000,0.000000,0.000000}%
\pgfsetstrokecolor{currentstroke}%
\pgfsetdash{}{0pt}%
\pgfsys@defobject{currentmarker}{\pgfqpoint{0.000000in}{-0.048611in}}{\pgfqpoint{0.000000in}{0.000000in}}{%
\pgfpathmoveto{\pgfqpoint{0.000000in}{0.000000in}}%
\pgfpathlineto{\pgfqpoint{0.000000in}{-0.048611in}}%
\pgfusepath{stroke,fill}%
}%
\begin{pgfscope}%
\pgfsys@transformshift{4.328190in}{0.440000in}%
\pgfsys@useobject{currentmarker}{}%
\end{pgfscope}%
\end{pgfscope}%
\begin{pgfscope}%
\pgftext[x=4.328190in,y=0.342778in,,top]{\sffamily\fontsize{10.000000}{12.000000}\selectfont 30}%
\end{pgfscope}%
\begin{pgfscope}%
\pgftext[x=2.562500in,y=0.152809in,,top]{\sffamily\fontsize{10.000000}{12.000000}\selectfont Epochs}%
\end{pgfscope}%
\begin{pgfscope}%
\pgfsetbuttcap%
\pgfsetroundjoin%
\definecolor{currentfill}{rgb}{0.000000,0.000000,0.000000}%
\pgfsetfillcolor{currentfill}%
\pgfsetlinewidth{0.803000pt}%
\definecolor{currentstroke}{rgb}{0.000000,0.000000,0.000000}%
\pgfsetstrokecolor{currentstroke}%
\pgfsetdash{}{0pt}%
\pgfsys@defobject{currentmarker}{\pgfqpoint{-0.048611in}{0.000000in}}{\pgfqpoint{0.000000in}{0.000000in}}{%
\pgfpathmoveto{\pgfqpoint{0.000000in}{0.000000in}}%
\pgfpathlineto{\pgfqpoint{-0.048611in}{0.000000in}}%
\pgfusepath{stroke,fill}%
}%
\begin{pgfscope}%
\pgfsys@transformshift{0.625000in}{0.805664in}%
\pgfsys@useobject{currentmarker}{}%
\end{pgfscope}%
\end{pgfscope}%
\begin{pgfscope}%
\pgftext[x=0.306898in,y=0.752902in,left,base]{\sffamily\fontsize{10.000000}{12.000000}\selectfont 0.2}%
\end{pgfscope}%
\begin{pgfscope}%
\pgfsetbuttcap%
\pgfsetroundjoin%
\definecolor{currentfill}{rgb}{0.000000,0.000000,0.000000}%
\pgfsetfillcolor{currentfill}%
\pgfsetlinewidth{0.803000pt}%
\definecolor{currentstroke}{rgb}{0.000000,0.000000,0.000000}%
\pgfsetstrokecolor{currentstroke}%
\pgfsetdash{}{0pt}%
\pgfsys@defobject{currentmarker}{\pgfqpoint{-0.048611in}{0.000000in}}{\pgfqpoint{0.000000in}{0.000000in}}{%
\pgfpathmoveto{\pgfqpoint{0.000000in}{0.000000in}}%
\pgfpathlineto{\pgfqpoint{-0.048611in}{0.000000in}}%
\pgfusepath{stroke,fill}%
}%
\begin{pgfscope}%
\pgfsys@transformshift{0.625000in}{1.286296in}%
\pgfsys@useobject{currentmarker}{}%
\end{pgfscope}%
\end{pgfscope}%
\begin{pgfscope}%
\pgftext[x=0.306898in,y=1.233534in,left,base]{\sffamily\fontsize{10.000000}{12.000000}\selectfont 0.4}%
\end{pgfscope}%
\begin{pgfscope}%
\pgfsetbuttcap%
\pgfsetroundjoin%
\definecolor{currentfill}{rgb}{0.000000,0.000000,0.000000}%
\pgfsetfillcolor{currentfill}%
\pgfsetlinewidth{0.803000pt}%
\definecolor{currentstroke}{rgb}{0.000000,0.000000,0.000000}%
\pgfsetstrokecolor{currentstroke}%
\pgfsetdash{}{0pt}%
\pgfsys@defobject{currentmarker}{\pgfqpoint{-0.048611in}{0.000000in}}{\pgfqpoint{0.000000in}{0.000000in}}{%
\pgfpathmoveto{\pgfqpoint{0.000000in}{0.000000in}}%
\pgfpathlineto{\pgfqpoint{-0.048611in}{0.000000in}}%
\pgfusepath{stroke,fill}%
}%
\begin{pgfscope}%
\pgfsys@transformshift{0.625000in}{1.766928in}%
\pgfsys@useobject{currentmarker}{}%
\end{pgfscope}%
\end{pgfscope}%
\begin{pgfscope}%
\pgftext[x=0.306898in,y=1.714166in,left,base]{\sffamily\fontsize{10.000000}{12.000000}\selectfont 0.6}%
\end{pgfscope}%
\begin{pgfscope}%
\pgfsetbuttcap%
\pgfsetroundjoin%
\definecolor{currentfill}{rgb}{0.000000,0.000000,0.000000}%
\pgfsetfillcolor{currentfill}%
\pgfsetlinewidth{0.803000pt}%
\definecolor{currentstroke}{rgb}{0.000000,0.000000,0.000000}%
\pgfsetstrokecolor{currentstroke}%
\pgfsetdash{}{0pt}%
\pgfsys@defobject{currentmarker}{\pgfqpoint{-0.048611in}{0.000000in}}{\pgfqpoint{0.000000in}{0.000000in}}{%
\pgfpathmoveto{\pgfqpoint{0.000000in}{0.000000in}}%
\pgfpathlineto{\pgfqpoint{-0.048611in}{0.000000in}}%
\pgfusepath{stroke,fill}%
}%
\begin{pgfscope}%
\pgfsys@transformshift{0.625000in}{2.247559in}%
\pgfsys@useobject{currentmarker}{}%
\end{pgfscope}%
\end{pgfscope}%
\begin{pgfscope}%
\pgftext[x=0.306898in,y=2.194798in,left,base]{\sffamily\fontsize{10.000000}{12.000000}\selectfont 0.8}%
\end{pgfscope}%
\begin{pgfscope}%
\pgfsetbuttcap%
\pgfsetroundjoin%
\definecolor{currentfill}{rgb}{0.000000,0.000000,0.000000}%
\pgfsetfillcolor{currentfill}%
\pgfsetlinewidth{0.803000pt}%
\definecolor{currentstroke}{rgb}{0.000000,0.000000,0.000000}%
\pgfsetstrokecolor{currentstroke}%
\pgfsetdash{}{0pt}%
\pgfsys@defobject{currentmarker}{\pgfqpoint{-0.048611in}{0.000000in}}{\pgfqpoint{0.000000in}{0.000000in}}{%
\pgfpathmoveto{\pgfqpoint{0.000000in}{0.000000in}}%
\pgfpathlineto{\pgfqpoint{-0.048611in}{0.000000in}}%
\pgfusepath{stroke,fill}%
}%
\begin{pgfscope}%
\pgfsys@transformshift{0.625000in}{2.728191in}%
\pgfsys@useobject{currentmarker}{}%
\end{pgfscope}%
\end{pgfscope}%
\begin{pgfscope}%
\pgftext[x=0.306898in,y=2.675430in,left,base]{\sffamily\fontsize{10.000000}{12.000000}\selectfont 1.0}%
\end{pgfscope}%
\begin{pgfscope}%
\pgfsetbuttcap%
\pgfsetroundjoin%
\definecolor{currentfill}{rgb}{0.000000,0.000000,0.000000}%
\pgfsetfillcolor{currentfill}%
\pgfsetlinewidth{0.803000pt}%
\definecolor{currentstroke}{rgb}{0.000000,0.000000,0.000000}%
\pgfsetstrokecolor{currentstroke}%
\pgfsetdash{}{0pt}%
\pgfsys@defobject{currentmarker}{\pgfqpoint{-0.048611in}{0.000000in}}{\pgfqpoint{0.000000in}{0.000000in}}{%
\pgfpathmoveto{\pgfqpoint{0.000000in}{0.000000in}}%
\pgfpathlineto{\pgfqpoint{-0.048611in}{0.000000in}}%
\pgfusepath{stroke,fill}%
}%
\begin{pgfscope}%
\pgfsys@transformshift{0.625000in}{3.208823in}%
\pgfsys@useobject{currentmarker}{}%
\end{pgfscope}%
\end{pgfscope}%
\begin{pgfscope}%
\pgftext[x=0.306898in,y=3.156061in,left,base]{\sffamily\fontsize{10.000000}{12.000000}\selectfont 1.2}%
\end{pgfscope}%
\begin{pgfscope}%
\pgftext[x=0.251343in,y=1.980000in,,bottom,rotate=90.000000]{\sffamily\fontsize{10.000000}{12.000000}\selectfont Cross-Entropy loss}%
\end{pgfscope}%
\begin{pgfscope}%
\pgfpathrectangle{\pgfqpoint{0.625000in}{0.440000in}}{\pgfqpoint{3.875000in}{3.080000in}} %
\pgfusepath{clip}%
\pgfsetrectcap%
\pgfsetroundjoin%
\pgfsetlinewidth{1.505625pt}%
\definecolor{currentstroke}{rgb}{0.901961,0.901961,0.980392}%
\pgfsetstrokecolor{currentstroke}%
\pgfsetdash{}{0pt}%
\pgfpathmoveto{\pgfqpoint{0.801136in}{1.935074in}}%
\pgfpathlineto{\pgfqpoint{0.805551in}{1.634167in}}%
\pgfpathlineto{\pgfqpoint{0.809965in}{1.493289in}}%
\pgfpathlineto{\pgfqpoint{0.814380in}{2.374093in}}%
\pgfpathlineto{\pgfqpoint{0.823209in}{1.632598in}}%
\pgfpathlineto{\pgfqpoint{0.827623in}{1.798637in}}%
\pgfpathlineto{\pgfqpoint{0.832037in}{2.125943in}}%
\pgfpathlineto{\pgfqpoint{0.836452in}{1.441174in}}%
\pgfpathlineto{\pgfqpoint{0.840866in}{1.512190in}}%
\pgfpathlineto{\pgfqpoint{0.845281in}{1.327176in}}%
\pgfpathlineto{\pgfqpoint{0.849695in}{1.474348in}}%
\pgfpathlineto{\pgfqpoint{0.854110in}{1.342628in}}%
\pgfpathlineto{\pgfqpoint{0.858524in}{1.324842in}}%
\pgfpathlineto{\pgfqpoint{0.862939in}{1.063583in}}%
\pgfpathlineto{\pgfqpoint{0.867353in}{2.640572in}}%
\pgfpathlineto{\pgfqpoint{0.871767in}{1.979672in}}%
\pgfpathlineto{\pgfqpoint{0.876182in}{1.758233in}}%
\pgfpathlineto{\pgfqpoint{0.880596in}{1.732029in}}%
\pgfpathlineto{\pgfqpoint{0.885011in}{1.602467in}}%
\pgfpathlineto{\pgfqpoint{0.889425in}{3.380000in}}%
\pgfpathlineto{\pgfqpoint{0.893840in}{1.848736in}}%
\pgfpathlineto{\pgfqpoint{0.898254in}{1.728547in}}%
\pgfpathlineto{\pgfqpoint{0.902669in}{1.177685in}}%
\pgfpathlineto{\pgfqpoint{0.907083in}{1.545895in}}%
\pgfpathlineto{\pgfqpoint{0.911497in}{1.231037in}}%
\pgfpathlineto{\pgfqpoint{0.915912in}{1.235947in}}%
\pgfpathlineto{\pgfqpoint{0.920326in}{1.243474in}}%
\pgfpathlineto{\pgfqpoint{0.924741in}{2.086185in}}%
\pgfpathlineto{\pgfqpoint{0.929155in}{1.707646in}}%
\pgfpathlineto{\pgfqpoint{0.933570in}{1.651050in}}%
\pgfpathlineto{\pgfqpoint{0.937984in}{1.618780in}}%
\pgfpathlineto{\pgfqpoint{0.942399in}{1.616009in}}%
\pgfpathlineto{\pgfqpoint{0.946813in}{1.473922in}}%
\pgfpathlineto{\pgfqpoint{0.951228in}{1.620727in}}%
\pgfpathlineto{\pgfqpoint{0.955642in}{1.411303in}}%
\pgfpathlineto{\pgfqpoint{0.960056in}{1.332126in}}%
\pgfpathlineto{\pgfqpoint{0.964471in}{1.377856in}}%
\pgfpathlineto{\pgfqpoint{0.968885in}{1.289014in}}%
\pgfpathlineto{\pgfqpoint{0.973300in}{1.495640in}}%
\pgfpathlineto{\pgfqpoint{0.977714in}{2.448341in}}%
\pgfpathlineto{\pgfqpoint{0.982129in}{2.031128in}}%
\pgfpathlineto{\pgfqpoint{0.986543in}{1.836424in}}%
\pgfpathlineto{\pgfqpoint{0.990958in}{1.743761in}}%
\pgfpathlineto{\pgfqpoint{0.995372in}{1.394181in}}%
\pgfpathlineto{\pgfqpoint{0.999786in}{1.714034in}}%
\pgfpathlineto{\pgfqpoint{1.004201in}{1.649855in}}%
\pgfpathlineto{\pgfqpoint{1.008615in}{1.149042in}}%
\pgfpathlineto{\pgfqpoint{1.013030in}{1.239775in}}%
\pgfpathlineto{\pgfqpoint{1.017444in}{1.548862in}}%
\pgfpathlineto{\pgfqpoint{1.021859in}{1.380586in}}%
\pgfpathlineto{\pgfqpoint{1.026273in}{1.362976in}}%
\pgfpathlineto{\pgfqpoint{1.030688in}{1.266597in}}%
\pgfpathlineto{\pgfqpoint{1.035102in}{2.085637in}}%
\pgfpathlineto{\pgfqpoint{1.039516in}{1.904441in}}%
\pgfpathlineto{\pgfqpoint{1.043931in}{1.917082in}}%
\pgfpathlineto{\pgfqpoint{1.048345in}{1.691202in}}%
\pgfpathlineto{\pgfqpoint{1.052760in}{1.347288in}}%
\pgfpathlineto{\pgfqpoint{1.057174in}{1.436851in}}%
\pgfpathlineto{\pgfqpoint{1.061589in}{1.452770in}}%
\pgfpathlineto{\pgfqpoint{1.066003in}{1.161536in}}%
\pgfpathlineto{\pgfqpoint{1.070418in}{1.250474in}}%
\pgfpathlineto{\pgfqpoint{1.074832in}{1.408650in}}%
\pgfpathlineto{\pgfqpoint{1.079246in}{1.261632in}}%
\pgfpathlineto{\pgfqpoint{1.083661in}{1.351203in}}%
\pgfpathlineto{\pgfqpoint{1.088075in}{2.154386in}}%
\pgfpathlineto{\pgfqpoint{1.092490in}{1.938261in}}%
\pgfpathlineto{\pgfqpoint{1.096904in}{1.538899in}}%
\pgfpathlineto{\pgfqpoint{1.101319in}{1.507535in}}%
\pgfpathlineto{\pgfqpoint{1.105733in}{1.596604in}}%
\pgfpathlineto{\pgfqpoint{1.110148in}{1.601424in}}%
\pgfpathlineto{\pgfqpoint{1.114562in}{1.221372in}}%
\pgfpathlineto{\pgfqpoint{1.118976in}{0.963083in}}%
\pgfpathlineto{\pgfqpoint{1.123391in}{1.150051in}}%
\pgfpathlineto{\pgfqpoint{1.127805in}{1.420971in}}%
\pgfpathlineto{\pgfqpoint{1.132220in}{1.268596in}}%
\pgfpathlineto{\pgfqpoint{1.136634in}{1.192599in}}%
\pgfpathlineto{\pgfqpoint{1.141049in}{1.052593in}}%
\pgfpathlineto{\pgfqpoint{1.145463in}{1.681599in}}%
\pgfpathlineto{\pgfqpoint{1.149878in}{1.743511in}}%
\pgfpathlineto{\pgfqpoint{1.154292in}{1.408653in}}%
\pgfpathlineto{\pgfqpoint{1.158706in}{1.597875in}}%
\pgfpathlineto{\pgfqpoint{1.163121in}{1.212629in}}%
\pgfpathlineto{\pgfqpoint{1.171950in}{1.016493in}}%
\pgfpathlineto{\pgfqpoint{1.176364in}{1.027141in}}%
\pgfpathlineto{\pgfqpoint{1.180779in}{1.164852in}}%
\pgfpathlineto{\pgfqpoint{1.185193in}{1.199775in}}%
\pgfpathlineto{\pgfqpoint{1.189608in}{1.140071in}}%
\pgfpathlineto{\pgfqpoint{1.194022in}{1.038316in}}%
\pgfpathlineto{\pgfqpoint{1.198436in}{2.040390in}}%
\pgfpathlineto{\pgfqpoint{1.202851in}{1.649394in}}%
\pgfpathlineto{\pgfqpoint{1.207265in}{1.365509in}}%
\pgfpathlineto{\pgfqpoint{1.220509in}{1.036845in}}%
\pgfpathlineto{\pgfqpoint{1.224923in}{1.329267in}}%
\pgfpathlineto{\pgfqpoint{1.229338in}{1.048320in}}%
\pgfpathlineto{\pgfqpoint{1.233752in}{0.875079in}}%
\pgfpathlineto{\pgfqpoint{1.238166in}{1.118397in}}%
\pgfpathlineto{\pgfqpoint{1.242581in}{1.272139in}}%
\pgfpathlineto{\pgfqpoint{1.246995in}{0.820554in}}%
\pgfpathlineto{\pgfqpoint{1.251410in}{1.082277in}}%
\pgfpathlineto{\pgfqpoint{1.255824in}{1.538257in}}%
\pgfpathlineto{\pgfqpoint{1.260239in}{1.249551in}}%
\pgfpathlineto{\pgfqpoint{1.264653in}{1.146586in}}%
\pgfpathlineto{\pgfqpoint{1.269068in}{1.123083in}}%
\pgfpathlineto{\pgfqpoint{1.273482in}{0.958130in}}%
\pgfpathlineto{\pgfqpoint{1.277896in}{1.038220in}}%
\pgfpathlineto{\pgfqpoint{1.282311in}{0.883550in}}%
\pgfpathlineto{\pgfqpoint{1.286725in}{0.956988in}}%
\pgfpathlineto{\pgfqpoint{1.291140in}{1.226142in}}%
\pgfpathlineto{\pgfqpoint{1.295554in}{1.102252in}}%
\pgfpathlineto{\pgfqpoint{1.299969in}{1.067207in}}%
\pgfpathlineto{\pgfqpoint{1.304383in}{0.996249in}}%
\pgfpathlineto{\pgfqpoint{1.308798in}{1.796342in}}%
\pgfpathlineto{\pgfqpoint{1.313212in}{1.406151in}}%
\pgfpathlineto{\pgfqpoint{1.317626in}{1.121225in}}%
\pgfpathlineto{\pgfqpoint{1.322041in}{1.332953in}}%
\pgfpathlineto{\pgfqpoint{1.326455in}{1.066500in}}%
\pgfpathlineto{\pgfqpoint{1.330870in}{0.908947in}}%
\pgfpathlineto{\pgfqpoint{1.335284in}{1.069925in}}%
\pgfpathlineto{\pgfqpoint{1.339699in}{0.918639in}}%
\pgfpathlineto{\pgfqpoint{1.344113in}{0.968427in}}%
\pgfpathlineto{\pgfqpoint{1.348528in}{0.943946in}}%
\pgfpathlineto{\pgfqpoint{1.352942in}{1.173934in}}%
\pgfpathlineto{\pgfqpoint{1.357356in}{1.082349in}}%
\pgfpathlineto{\pgfqpoint{1.361771in}{0.948899in}}%
\pgfpathlineto{\pgfqpoint{1.366185in}{1.598125in}}%
\pgfpathlineto{\pgfqpoint{1.370600in}{1.000666in}}%
\pgfpathlineto{\pgfqpoint{1.375014in}{0.988857in}}%
\pgfpathlineto{\pgfqpoint{1.379429in}{1.197179in}}%
\pgfpathlineto{\pgfqpoint{1.383843in}{0.907154in}}%
\pgfpathlineto{\pgfqpoint{1.388258in}{1.001865in}}%
\pgfpathlineto{\pgfqpoint{1.392672in}{0.980662in}}%
\pgfpathlineto{\pgfqpoint{1.397086in}{0.979335in}}%
\pgfpathlineto{\pgfqpoint{1.401501in}{1.531339in}}%
\pgfpathlineto{\pgfqpoint{1.405915in}{1.082967in}}%
\pgfpathlineto{\pgfqpoint{1.410330in}{1.123559in}}%
\pgfpathlineto{\pgfqpoint{1.414744in}{0.923832in}}%
\pgfpathlineto{\pgfqpoint{1.419159in}{2.541098in}}%
\pgfpathlineto{\pgfqpoint{1.423573in}{1.384140in}}%
\pgfpathlineto{\pgfqpoint{1.427988in}{1.108657in}}%
\pgfpathlineto{\pgfqpoint{1.432402in}{1.231434in}}%
\pgfpathlineto{\pgfqpoint{1.436816in}{0.829018in}}%
\pgfpathlineto{\pgfqpoint{1.441231in}{0.813501in}}%
\pgfpathlineto{\pgfqpoint{1.445645in}{0.806010in}}%
\pgfpathlineto{\pgfqpoint{1.450060in}{0.972224in}}%
\pgfpathlineto{\pgfqpoint{1.454474in}{0.885545in}}%
\pgfpathlineto{\pgfqpoint{1.458889in}{1.121475in}}%
\pgfpathlineto{\pgfqpoint{1.463303in}{1.036973in}}%
\pgfpathlineto{\pgfqpoint{1.467718in}{1.015070in}}%
\pgfpathlineto{\pgfqpoint{1.472132in}{0.898089in}}%
\pgfpathlineto{\pgfqpoint{1.476546in}{1.588065in}}%
\pgfpathlineto{\pgfqpoint{1.485375in}{0.854390in}}%
\pgfpathlineto{\pgfqpoint{1.489790in}{1.120067in}}%
\pgfpathlineto{\pgfqpoint{1.494204in}{0.877280in}}%
\pgfpathlineto{\pgfqpoint{1.498619in}{0.711729in}}%
\pgfpathlineto{\pgfqpoint{1.503033in}{1.120418in}}%
\pgfpathlineto{\pgfqpoint{1.507448in}{0.918216in}}%
\pgfpathlineto{\pgfqpoint{1.511862in}{1.744619in}}%
\pgfpathlineto{\pgfqpoint{1.516276in}{0.990856in}}%
\pgfpathlineto{\pgfqpoint{1.520691in}{1.134519in}}%
\pgfpathlineto{\pgfqpoint{1.525105in}{0.980136in}}%
\pgfpathlineto{\pgfqpoint{1.529520in}{1.156684in}}%
\pgfpathlineto{\pgfqpoint{1.533934in}{1.137422in}}%
\pgfpathlineto{\pgfqpoint{1.538349in}{0.953838in}}%
\pgfpathlineto{\pgfqpoint{1.542763in}{1.066575in}}%
\pgfpathlineto{\pgfqpoint{1.547178in}{0.909209in}}%
\pgfpathlineto{\pgfqpoint{1.551592in}{0.836424in}}%
\pgfpathlineto{\pgfqpoint{1.556006in}{0.862076in}}%
\pgfpathlineto{\pgfqpoint{1.560421in}{0.911934in}}%
\pgfpathlineto{\pgfqpoint{1.564835in}{0.842764in}}%
\pgfpathlineto{\pgfqpoint{1.569250in}{0.961365in}}%
\pgfpathlineto{\pgfqpoint{1.573664in}{0.979963in}}%
\pgfpathlineto{\pgfqpoint{1.578079in}{0.920395in}}%
\pgfpathlineto{\pgfqpoint{1.582493in}{1.110942in}}%
\pgfpathlineto{\pgfqpoint{1.586908in}{1.169000in}}%
\pgfpathlineto{\pgfqpoint{1.591322in}{1.273984in}}%
\pgfpathlineto{\pgfqpoint{1.595737in}{1.342321in}}%
\pgfpathlineto{\pgfqpoint{1.600151in}{0.999046in}}%
\pgfpathlineto{\pgfqpoint{1.608980in}{0.860494in}}%
\pgfpathlineto{\pgfqpoint{1.613394in}{0.898618in}}%
\pgfpathlineto{\pgfqpoint{1.617809in}{0.818612in}}%
\pgfpathlineto{\pgfqpoint{1.622223in}{1.095269in}}%
\pgfpathlineto{\pgfqpoint{1.626638in}{1.144310in}}%
\pgfpathlineto{\pgfqpoint{1.631052in}{1.146071in}}%
\pgfpathlineto{\pgfqpoint{1.635467in}{1.053454in}}%
\pgfpathlineto{\pgfqpoint{1.639881in}{1.519652in}}%
\pgfpathlineto{\pgfqpoint{1.644295in}{1.303723in}}%
\pgfpathlineto{\pgfqpoint{1.648710in}{0.994353in}}%
\pgfpathlineto{\pgfqpoint{1.653124in}{0.953002in}}%
\pgfpathlineto{\pgfqpoint{1.657539in}{0.923493in}}%
\pgfpathlineto{\pgfqpoint{1.661953in}{0.999022in}}%
\pgfpathlineto{\pgfqpoint{1.666368in}{0.729476in}}%
\pgfpathlineto{\pgfqpoint{1.670782in}{0.906111in}}%
\pgfpathlineto{\pgfqpoint{1.675197in}{0.937484in}}%
\pgfpathlineto{\pgfqpoint{1.679611in}{0.949522in}}%
\pgfpathlineto{\pgfqpoint{1.684025in}{0.945458in}}%
\pgfpathlineto{\pgfqpoint{1.688440in}{0.863015in}}%
\pgfpathlineto{\pgfqpoint{1.697269in}{1.582704in}}%
\pgfpathlineto{\pgfqpoint{1.701683in}{1.095422in}}%
\pgfpathlineto{\pgfqpoint{1.706098in}{1.030439in}}%
\pgfpathlineto{\pgfqpoint{1.710512in}{1.012400in}}%
\pgfpathlineto{\pgfqpoint{1.714927in}{0.828345in}}%
\pgfpathlineto{\pgfqpoint{1.719341in}{0.838640in}}%
\pgfpathlineto{\pgfqpoint{1.723755in}{0.689365in}}%
\pgfpathlineto{\pgfqpoint{1.728170in}{0.844807in}}%
\pgfpathlineto{\pgfqpoint{1.732584in}{0.805157in}}%
\pgfpathlineto{\pgfqpoint{1.736999in}{1.025389in}}%
\pgfpathlineto{\pgfqpoint{1.741413in}{1.066974in}}%
\pgfpathlineto{\pgfqpoint{1.745828in}{0.898133in}}%
\pgfpathlineto{\pgfqpoint{1.750242in}{1.702886in}}%
\pgfpathlineto{\pgfqpoint{1.754657in}{1.189489in}}%
\pgfpathlineto{\pgfqpoint{1.759071in}{1.010649in}}%
\pgfpathlineto{\pgfqpoint{1.763485in}{1.002435in}}%
\pgfpathlineto{\pgfqpoint{1.767900in}{1.015109in}}%
\pgfpathlineto{\pgfqpoint{1.772314in}{0.832262in}}%
\pgfpathlineto{\pgfqpoint{1.776729in}{0.848212in}}%
\pgfpathlineto{\pgfqpoint{1.781143in}{0.939361in}}%
\pgfpathlineto{\pgfqpoint{1.785558in}{0.861299in}}%
\pgfpathlineto{\pgfqpoint{1.794387in}{1.155213in}}%
\pgfpathlineto{\pgfqpoint{1.798801in}{1.007416in}}%
\pgfpathlineto{\pgfqpoint{1.803215in}{0.905020in}}%
\pgfpathlineto{\pgfqpoint{1.807630in}{1.273939in}}%
\pgfpathlineto{\pgfqpoint{1.812044in}{0.992512in}}%
\pgfpathlineto{\pgfqpoint{1.816459in}{0.809682in}}%
\pgfpathlineto{\pgfqpoint{1.820873in}{1.056455in}}%
\pgfpathlineto{\pgfqpoint{1.825288in}{0.852264in}}%
\pgfpathlineto{\pgfqpoint{1.834117in}{0.707973in}}%
\pgfpathlineto{\pgfqpoint{1.838531in}{0.890452in}}%
\pgfpathlineto{\pgfqpoint{1.842945in}{0.860783in}}%
\pgfpathlineto{\pgfqpoint{1.847360in}{1.153387in}}%
\pgfpathlineto{\pgfqpoint{1.851774in}{1.062696in}}%
\pgfpathlineto{\pgfqpoint{1.856189in}{0.936499in}}%
\pgfpathlineto{\pgfqpoint{1.860603in}{1.278348in}}%
\pgfpathlineto{\pgfqpoint{1.865018in}{0.955818in}}%
\pgfpathlineto{\pgfqpoint{1.869432in}{0.981174in}}%
\pgfpathlineto{\pgfqpoint{1.873847in}{0.891394in}}%
\pgfpathlineto{\pgfqpoint{1.878261in}{0.898967in}}%
\pgfpathlineto{\pgfqpoint{1.882675in}{0.846018in}}%
\pgfpathlineto{\pgfqpoint{1.887090in}{0.886701in}}%
\pgfpathlineto{\pgfqpoint{1.891504in}{0.993358in}}%
\pgfpathlineto{\pgfqpoint{1.895919in}{1.033952in}}%
\pgfpathlineto{\pgfqpoint{1.900333in}{1.031402in}}%
\pgfpathlineto{\pgfqpoint{1.904748in}{1.213550in}}%
\pgfpathlineto{\pgfqpoint{1.909162in}{1.134137in}}%
\pgfpathlineto{\pgfqpoint{1.913577in}{0.857469in}}%
\pgfpathlineto{\pgfqpoint{1.917991in}{0.891865in}}%
\pgfpathlineto{\pgfqpoint{1.922405in}{0.988290in}}%
\pgfpathlineto{\pgfqpoint{1.926820in}{0.767019in}}%
\pgfpathlineto{\pgfqpoint{1.931234in}{0.962854in}}%
\pgfpathlineto{\pgfqpoint{1.935649in}{0.899305in}}%
\pgfpathlineto{\pgfqpoint{1.940063in}{0.851841in}}%
\pgfpathlineto{\pgfqpoint{1.944478in}{0.950908in}}%
\pgfpathlineto{\pgfqpoint{1.948892in}{0.871092in}}%
\pgfpathlineto{\pgfqpoint{1.953307in}{0.834156in}}%
\pgfpathlineto{\pgfqpoint{1.957721in}{1.282689in}}%
\pgfpathlineto{\pgfqpoint{1.962135in}{0.961583in}}%
\pgfpathlineto{\pgfqpoint{1.966550in}{1.119615in}}%
\pgfpathlineto{\pgfqpoint{1.970964in}{1.233337in}}%
\pgfpathlineto{\pgfqpoint{1.975379in}{1.041219in}}%
\pgfpathlineto{\pgfqpoint{1.979793in}{1.135606in}}%
\pgfpathlineto{\pgfqpoint{1.984208in}{0.912472in}}%
\pgfpathlineto{\pgfqpoint{1.988622in}{0.883541in}}%
\pgfpathlineto{\pgfqpoint{1.997451in}{0.753989in}}%
\pgfpathlineto{\pgfqpoint{2.001865in}{0.976324in}}%
\pgfpathlineto{\pgfqpoint{2.006280in}{0.889416in}}%
\pgfpathlineto{\pgfqpoint{2.010694in}{0.943567in}}%
\pgfpathlineto{\pgfqpoint{2.015109in}{0.952338in}}%
\pgfpathlineto{\pgfqpoint{2.019523in}{0.811078in}}%
\pgfpathlineto{\pgfqpoint{2.023938in}{0.927059in}}%
\pgfpathlineto{\pgfqpoint{2.028352in}{0.993127in}}%
\pgfpathlineto{\pgfqpoint{2.032767in}{0.906498in}}%
\pgfpathlineto{\pgfqpoint{2.037181in}{0.961059in}}%
\pgfpathlineto{\pgfqpoint{2.041595in}{0.849050in}}%
\pgfpathlineto{\pgfqpoint{2.046010in}{0.955539in}}%
\pgfpathlineto{\pgfqpoint{2.050424in}{0.801341in}}%
\pgfpathlineto{\pgfqpoint{2.054839in}{0.980431in}}%
\pgfpathlineto{\pgfqpoint{2.059253in}{0.797551in}}%
\pgfpathlineto{\pgfqpoint{2.063668in}{1.003324in}}%
\pgfpathlineto{\pgfqpoint{2.068082in}{1.114165in}}%
\pgfpathlineto{\pgfqpoint{2.072497in}{1.103788in}}%
\pgfpathlineto{\pgfqpoint{2.076911in}{0.988040in}}%
\pgfpathlineto{\pgfqpoint{2.081325in}{1.073592in}}%
\pgfpathlineto{\pgfqpoint{2.085740in}{1.284320in}}%
\pgfpathlineto{\pgfqpoint{2.090154in}{1.166611in}}%
\pgfpathlineto{\pgfqpoint{2.094569in}{0.946383in}}%
\pgfpathlineto{\pgfqpoint{2.098983in}{0.987533in}}%
\pgfpathlineto{\pgfqpoint{2.103398in}{0.894167in}}%
\pgfpathlineto{\pgfqpoint{2.107812in}{0.758096in}}%
\pgfpathlineto{\pgfqpoint{2.112227in}{0.886316in}}%
\pgfpathlineto{\pgfqpoint{2.116641in}{0.828962in}}%
\pgfpathlineto{\pgfqpoint{2.121055in}{0.859696in}}%
\pgfpathlineto{\pgfqpoint{2.125470in}{1.023537in}}%
\pgfpathlineto{\pgfqpoint{2.129884in}{0.947167in}}%
\pgfpathlineto{\pgfqpoint{2.134299in}{0.758557in}}%
\pgfpathlineto{\pgfqpoint{2.138713in}{1.008195in}}%
\pgfpathlineto{\pgfqpoint{2.143128in}{1.031664in}}%
\pgfpathlineto{\pgfqpoint{2.147542in}{1.154413in}}%
\pgfpathlineto{\pgfqpoint{2.151957in}{0.866752in}}%
\pgfpathlineto{\pgfqpoint{2.156371in}{1.063167in}}%
\pgfpathlineto{\pgfqpoint{2.160785in}{0.723995in}}%
\pgfpathlineto{\pgfqpoint{2.165200in}{0.799024in}}%
\pgfpathlineto{\pgfqpoint{2.169614in}{0.670361in}}%
\pgfpathlineto{\pgfqpoint{2.178443in}{1.169084in}}%
\pgfpathlineto{\pgfqpoint{2.187272in}{0.713765in}}%
\pgfpathlineto{\pgfqpoint{2.191687in}{1.103156in}}%
\pgfpathlineto{\pgfqpoint{2.196101in}{1.097131in}}%
\pgfpathlineto{\pgfqpoint{2.200515in}{1.018024in}}%
\pgfpathlineto{\pgfqpoint{2.204930in}{1.152327in}}%
\pgfpathlineto{\pgfqpoint{2.209344in}{1.199780in}}%
\pgfpathlineto{\pgfqpoint{2.218173in}{0.712984in}}%
\pgfpathlineto{\pgfqpoint{2.222588in}{1.035019in}}%
\pgfpathlineto{\pgfqpoint{2.227002in}{0.902941in}}%
\pgfpathlineto{\pgfqpoint{2.231417in}{0.973710in}}%
\pgfpathlineto{\pgfqpoint{2.235831in}{1.007450in}}%
\pgfpathlineto{\pgfqpoint{2.240246in}{0.898983in}}%
\pgfpathlineto{\pgfqpoint{2.244660in}{0.995324in}}%
\pgfpathlineto{\pgfqpoint{2.249074in}{0.780022in}}%
\pgfpathlineto{\pgfqpoint{2.253489in}{1.041954in}}%
\pgfpathlineto{\pgfqpoint{2.257903in}{1.020725in}}%
\pgfpathlineto{\pgfqpoint{2.262318in}{0.855914in}}%
\pgfpathlineto{\pgfqpoint{2.266732in}{1.015301in}}%
\pgfpathlineto{\pgfqpoint{2.271147in}{0.790168in}}%
\pgfpathlineto{\pgfqpoint{2.275561in}{0.697990in}}%
\pgfpathlineto{\pgfqpoint{2.279976in}{0.858211in}}%
\pgfpathlineto{\pgfqpoint{2.284390in}{0.673327in}}%
\pgfpathlineto{\pgfqpoint{2.288804in}{0.935975in}}%
\pgfpathlineto{\pgfqpoint{2.293219in}{0.952917in}}%
\pgfpathlineto{\pgfqpoint{2.297633in}{1.019439in}}%
\pgfpathlineto{\pgfqpoint{2.302048in}{1.109471in}}%
\pgfpathlineto{\pgfqpoint{2.306462in}{0.988883in}}%
\pgfpathlineto{\pgfqpoint{2.310877in}{0.967175in}}%
\pgfpathlineto{\pgfqpoint{2.319706in}{1.276082in}}%
\pgfpathlineto{\pgfqpoint{2.324120in}{0.816163in}}%
\pgfpathlineto{\pgfqpoint{2.328534in}{0.716355in}}%
\pgfpathlineto{\pgfqpoint{2.332949in}{1.061507in}}%
\pgfpathlineto{\pgfqpoint{2.337363in}{0.997830in}}%
\pgfpathlineto{\pgfqpoint{2.341778in}{1.082094in}}%
\pgfpathlineto{\pgfqpoint{2.350607in}{0.902547in}}%
\pgfpathlineto{\pgfqpoint{2.355021in}{0.880147in}}%
\pgfpathlineto{\pgfqpoint{2.359436in}{0.806625in}}%
\pgfpathlineto{\pgfqpoint{2.363850in}{0.832329in}}%
\pgfpathlineto{\pgfqpoint{2.368264in}{0.817225in}}%
\pgfpathlineto{\pgfqpoint{2.372679in}{0.756104in}}%
\pgfpathlineto{\pgfqpoint{2.377093in}{1.038160in}}%
\pgfpathlineto{\pgfqpoint{2.381508in}{0.886408in}}%
\pgfpathlineto{\pgfqpoint{2.385922in}{0.801516in}}%
\pgfpathlineto{\pgfqpoint{2.390337in}{0.863371in}}%
\pgfpathlineto{\pgfqpoint{2.394751in}{0.631692in}}%
\pgfpathlineto{\pgfqpoint{2.399166in}{0.952956in}}%
\pgfpathlineto{\pgfqpoint{2.403580in}{0.889911in}}%
\pgfpathlineto{\pgfqpoint{2.407994in}{1.115342in}}%
\pgfpathlineto{\pgfqpoint{2.412409in}{1.215470in}}%
\pgfpathlineto{\pgfqpoint{2.416823in}{0.996172in}}%
\pgfpathlineto{\pgfqpoint{2.421238in}{1.081289in}}%
\pgfpathlineto{\pgfqpoint{2.425652in}{0.816572in}}%
\pgfpathlineto{\pgfqpoint{2.430067in}{1.193243in}}%
\pgfpathlineto{\pgfqpoint{2.434481in}{0.904097in}}%
\pgfpathlineto{\pgfqpoint{2.438896in}{0.788830in}}%
\pgfpathlineto{\pgfqpoint{2.443310in}{0.896330in}}%
\pgfpathlineto{\pgfqpoint{2.447724in}{1.375549in}}%
\pgfpathlineto{\pgfqpoint{2.452139in}{0.973342in}}%
\pgfpathlineto{\pgfqpoint{2.456553in}{0.991289in}}%
\pgfpathlineto{\pgfqpoint{2.460968in}{0.829414in}}%
\pgfpathlineto{\pgfqpoint{2.465382in}{1.027338in}}%
\pgfpathlineto{\pgfqpoint{2.474211in}{0.913902in}}%
\pgfpathlineto{\pgfqpoint{2.478626in}{0.951247in}}%
\pgfpathlineto{\pgfqpoint{2.483040in}{0.739104in}}%
\pgfpathlineto{\pgfqpoint{2.487454in}{1.027706in}}%
\pgfpathlineto{\pgfqpoint{2.491869in}{0.875932in}}%
\pgfpathlineto{\pgfqpoint{2.496283in}{0.840459in}}%
\pgfpathlineto{\pgfqpoint{2.500698in}{0.736869in}}%
\pgfpathlineto{\pgfqpoint{2.505112in}{0.695453in}}%
\pgfpathlineto{\pgfqpoint{2.509527in}{0.860761in}}%
\pgfpathlineto{\pgfqpoint{2.513941in}{0.864099in}}%
\pgfpathlineto{\pgfqpoint{2.522770in}{1.129184in}}%
\pgfpathlineto{\pgfqpoint{2.527184in}{0.902980in}}%
\pgfpathlineto{\pgfqpoint{2.531599in}{0.981984in}}%
\pgfpathlineto{\pgfqpoint{2.536013in}{0.815921in}}%
\pgfpathlineto{\pgfqpoint{2.540428in}{1.172809in}}%
\pgfpathlineto{\pgfqpoint{2.544842in}{1.148352in}}%
\pgfpathlineto{\pgfqpoint{2.549257in}{0.740192in}}%
\pgfpathlineto{\pgfqpoint{2.553671in}{0.827052in}}%
\pgfpathlineto{\pgfqpoint{2.558086in}{1.642790in}}%
\pgfpathlineto{\pgfqpoint{2.562500in}{1.033527in}}%
\pgfpathlineto{\pgfqpoint{2.566914in}{1.008519in}}%
\pgfpathlineto{\pgfqpoint{2.571329in}{0.938424in}}%
\pgfpathlineto{\pgfqpoint{2.580158in}{0.990520in}}%
\pgfpathlineto{\pgfqpoint{2.588987in}{0.986305in}}%
\pgfpathlineto{\pgfqpoint{2.593401in}{0.820001in}}%
\pgfpathlineto{\pgfqpoint{2.597816in}{0.983423in}}%
\pgfpathlineto{\pgfqpoint{2.602230in}{0.937544in}}%
\pgfpathlineto{\pgfqpoint{2.606644in}{0.871193in}}%
\pgfpathlineto{\pgfqpoint{2.611059in}{0.700658in}}%
\pgfpathlineto{\pgfqpoint{2.615473in}{0.789265in}}%
\pgfpathlineto{\pgfqpoint{2.624302in}{1.059409in}}%
\pgfpathlineto{\pgfqpoint{2.628717in}{0.765399in}}%
\pgfpathlineto{\pgfqpoint{2.633131in}{1.054146in}}%
\pgfpathlineto{\pgfqpoint{2.641960in}{0.902617in}}%
\pgfpathlineto{\pgfqpoint{2.646374in}{1.030172in}}%
\pgfpathlineto{\pgfqpoint{2.650789in}{1.078168in}}%
\pgfpathlineto{\pgfqpoint{2.655203in}{1.102375in}}%
\pgfpathlineto{\pgfqpoint{2.659618in}{0.861347in}}%
\pgfpathlineto{\pgfqpoint{2.664032in}{0.810792in}}%
\pgfpathlineto{\pgfqpoint{2.668447in}{1.537584in}}%
\pgfpathlineto{\pgfqpoint{2.672861in}{0.833841in}}%
\pgfpathlineto{\pgfqpoint{2.677276in}{1.066820in}}%
\pgfpathlineto{\pgfqpoint{2.681690in}{0.813328in}}%
\pgfpathlineto{\pgfqpoint{2.686104in}{0.784434in}}%
\pgfpathlineto{\pgfqpoint{2.690519in}{1.023878in}}%
\pgfpathlineto{\pgfqpoint{2.694933in}{0.871897in}}%
\pgfpathlineto{\pgfqpoint{2.699348in}{0.797517in}}%
\pgfpathlineto{\pgfqpoint{2.703762in}{0.914671in}}%
\pgfpathlineto{\pgfqpoint{2.708177in}{1.078288in}}%
\pgfpathlineto{\pgfqpoint{2.712591in}{0.848260in}}%
\pgfpathlineto{\pgfqpoint{2.717006in}{0.718162in}}%
\pgfpathlineto{\pgfqpoint{2.721420in}{0.908558in}}%
\pgfpathlineto{\pgfqpoint{2.725834in}{0.746746in}}%
\pgfpathlineto{\pgfqpoint{2.730249in}{0.926834in}}%
\pgfpathlineto{\pgfqpoint{2.734663in}{0.863849in}}%
\pgfpathlineto{\pgfqpoint{2.739078in}{0.904919in}}%
\pgfpathlineto{\pgfqpoint{2.743492in}{1.105672in}}%
\pgfpathlineto{\pgfqpoint{2.747907in}{1.098782in}}%
\pgfpathlineto{\pgfqpoint{2.752321in}{0.993257in}}%
\pgfpathlineto{\pgfqpoint{2.756736in}{1.037893in}}%
\pgfpathlineto{\pgfqpoint{2.761150in}{0.941562in}}%
\pgfpathlineto{\pgfqpoint{2.765564in}{1.079413in}}%
\pgfpathlineto{\pgfqpoint{2.769979in}{0.925469in}}%
\pgfpathlineto{\pgfqpoint{2.774393in}{0.817918in}}%
\pgfpathlineto{\pgfqpoint{2.778808in}{0.846467in}}%
\pgfpathlineto{\pgfqpoint{2.783222in}{0.819448in}}%
\pgfpathlineto{\pgfqpoint{2.787637in}{1.020650in}}%
\pgfpathlineto{\pgfqpoint{2.792051in}{0.833627in}}%
\pgfpathlineto{\pgfqpoint{2.796466in}{0.848959in}}%
\pgfpathlineto{\pgfqpoint{2.800880in}{1.026639in}}%
\pgfpathlineto{\pgfqpoint{2.805294in}{0.877064in}}%
\pgfpathlineto{\pgfqpoint{2.809709in}{0.976610in}}%
\pgfpathlineto{\pgfqpoint{2.814123in}{0.935586in}}%
\pgfpathlineto{\pgfqpoint{2.818538in}{1.105948in}}%
\pgfpathlineto{\pgfqpoint{2.822952in}{0.810117in}}%
\pgfpathlineto{\pgfqpoint{2.827367in}{0.774319in}}%
\pgfpathlineto{\pgfqpoint{2.831781in}{0.855320in}}%
\pgfpathlineto{\pgfqpoint{2.836196in}{0.689464in}}%
\pgfpathlineto{\pgfqpoint{2.840610in}{0.940452in}}%
\pgfpathlineto{\pgfqpoint{2.845024in}{0.964609in}}%
\pgfpathlineto{\pgfqpoint{2.849439in}{0.868341in}}%
\pgfpathlineto{\pgfqpoint{2.853853in}{0.906750in}}%
\pgfpathlineto{\pgfqpoint{2.858268in}{1.028228in}}%
\pgfpathlineto{\pgfqpoint{2.862682in}{0.892053in}}%
\pgfpathlineto{\pgfqpoint{2.867097in}{0.822633in}}%
\pgfpathlineto{\pgfqpoint{2.871511in}{1.169918in}}%
\pgfpathlineto{\pgfqpoint{2.875926in}{1.020948in}}%
\pgfpathlineto{\pgfqpoint{2.880340in}{1.038905in}}%
\pgfpathlineto{\pgfqpoint{2.884754in}{0.809079in}}%
\pgfpathlineto{\pgfqpoint{2.889169in}{0.926043in}}%
\pgfpathlineto{\pgfqpoint{2.893583in}{0.926189in}}%
\pgfpathlineto{\pgfqpoint{2.897998in}{0.815103in}}%
\pgfpathlineto{\pgfqpoint{2.906827in}{0.871049in}}%
\pgfpathlineto{\pgfqpoint{2.911241in}{0.883348in}}%
\pgfpathlineto{\pgfqpoint{2.915656in}{0.946277in}}%
\pgfpathlineto{\pgfqpoint{2.920070in}{0.987237in}}%
\pgfpathlineto{\pgfqpoint{2.924485in}{0.971191in}}%
\pgfpathlineto{\pgfqpoint{2.928899in}{0.888224in}}%
\pgfpathlineto{\pgfqpoint{2.933313in}{0.963686in}}%
\pgfpathlineto{\pgfqpoint{2.937728in}{0.791473in}}%
\pgfpathlineto{\pgfqpoint{2.942142in}{0.751852in}}%
\pgfpathlineto{\pgfqpoint{2.946557in}{0.798738in}}%
\pgfpathlineto{\pgfqpoint{2.950971in}{0.895410in}}%
\pgfpathlineto{\pgfqpoint{2.955386in}{0.885925in}}%
\pgfpathlineto{\pgfqpoint{2.959800in}{0.830390in}}%
\pgfpathlineto{\pgfqpoint{2.964215in}{1.135877in}}%
\pgfpathlineto{\pgfqpoint{2.968629in}{0.814099in}}%
\pgfpathlineto{\pgfqpoint{2.973043in}{0.830613in}}%
\pgfpathlineto{\pgfqpoint{2.981872in}{1.033161in}}%
\pgfpathlineto{\pgfqpoint{2.986287in}{0.945554in}}%
\pgfpathlineto{\pgfqpoint{2.990701in}{0.798209in}}%
\pgfpathlineto{\pgfqpoint{2.995116in}{0.803633in}}%
\pgfpathlineto{\pgfqpoint{2.999530in}{0.846376in}}%
\pgfpathlineto{\pgfqpoint{3.003945in}{0.849671in}}%
\pgfpathlineto{\pgfqpoint{3.008359in}{1.073400in}}%
\pgfpathlineto{\pgfqpoint{3.012773in}{0.580000in}}%
\pgfpathlineto{\pgfqpoint{3.017188in}{1.014409in}}%
\pgfpathlineto{\pgfqpoint{3.021602in}{0.943824in}}%
\pgfpathlineto{\pgfqpoint{3.026017in}{0.806483in}}%
\pgfpathlineto{\pgfqpoint{3.030431in}{0.788782in}}%
\pgfpathlineto{\pgfqpoint{3.034846in}{0.837155in}}%
\pgfpathlineto{\pgfqpoint{3.039260in}{0.790863in}}%
\pgfpathlineto{\pgfqpoint{3.043675in}{0.871890in}}%
\pgfpathlineto{\pgfqpoint{3.048089in}{0.755104in}}%
\pgfpathlineto{\pgfqpoint{3.052503in}{0.751804in}}%
\pgfpathlineto{\pgfqpoint{3.056918in}{0.754474in}}%
\pgfpathlineto{\pgfqpoint{3.061332in}{0.776807in}}%
\pgfpathlineto{\pgfqpoint{3.065747in}{0.893956in}}%
\pgfpathlineto{\pgfqpoint{3.070161in}{0.858831in}}%
\pgfpathlineto{\pgfqpoint{3.074576in}{1.102617in}}%
\pgfpathlineto{\pgfqpoint{3.078990in}{0.944131in}}%
\pgfpathlineto{\pgfqpoint{3.083405in}{0.950454in}}%
\pgfpathlineto{\pgfqpoint{3.087819in}{1.146602in}}%
\pgfpathlineto{\pgfqpoint{3.092233in}{0.880830in}}%
\pgfpathlineto{\pgfqpoint{3.096648in}{0.938559in}}%
\pgfpathlineto{\pgfqpoint{3.101062in}{0.881945in}}%
\pgfpathlineto{\pgfqpoint{3.105477in}{0.798524in}}%
\pgfpathlineto{\pgfqpoint{3.109891in}{1.124681in}}%
\pgfpathlineto{\pgfqpoint{3.114306in}{0.758985in}}%
\pgfpathlineto{\pgfqpoint{3.118720in}{1.012158in}}%
\pgfpathlineto{\pgfqpoint{3.123135in}{1.025233in}}%
\pgfpathlineto{\pgfqpoint{3.127549in}{1.028401in}}%
\pgfpathlineto{\pgfqpoint{3.131963in}{1.119680in}}%
\pgfpathlineto{\pgfqpoint{3.136378in}{0.785035in}}%
\pgfpathlineto{\pgfqpoint{3.140792in}{0.941498in}}%
\pgfpathlineto{\pgfqpoint{3.145207in}{0.994288in}}%
\pgfpathlineto{\pgfqpoint{3.149621in}{0.764983in}}%
\pgfpathlineto{\pgfqpoint{3.154036in}{0.744482in}}%
\pgfpathlineto{\pgfqpoint{3.158450in}{0.779864in}}%
\pgfpathlineto{\pgfqpoint{3.162865in}{0.901197in}}%
\pgfpathlineto{\pgfqpoint{3.167279in}{0.779654in}}%
\pgfpathlineto{\pgfqpoint{3.171693in}{0.873702in}}%
\pgfpathlineto{\pgfqpoint{3.180522in}{0.972794in}}%
\pgfpathlineto{\pgfqpoint{3.184937in}{0.996054in}}%
\pgfpathlineto{\pgfqpoint{3.189351in}{0.982154in}}%
\pgfpathlineto{\pgfqpoint{3.193766in}{0.930933in}}%
\pgfpathlineto{\pgfqpoint{3.198180in}{1.014047in}}%
\pgfpathlineto{\pgfqpoint{3.202595in}{0.847426in}}%
\pgfpathlineto{\pgfqpoint{3.207009in}{1.068336in}}%
\pgfpathlineto{\pgfqpoint{3.211423in}{0.953593in}}%
\pgfpathlineto{\pgfqpoint{3.215838in}{0.875194in}}%
\pgfpathlineto{\pgfqpoint{3.220252in}{1.075082in}}%
\pgfpathlineto{\pgfqpoint{3.224667in}{0.800908in}}%
\pgfpathlineto{\pgfqpoint{3.229081in}{1.016286in}}%
\pgfpathlineto{\pgfqpoint{3.233496in}{1.086870in}}%
\pgfpathlineto{\pgfqpoint{3.237910in}{0.784884in}}%
\pgfpathlineto{\pgfqpoint{3.242325in}{1.027514in}}%
\pgfpathlineto{\pgfqpoint{3.246739in}{0.930506in}}%
\pgfpathlineto{\pgfqpoint{3.251153in}{1.130134in}}%
\pgfpathlineto{\pgfqpoint{3.255568in}{0.970071in}}%
\pgfpathlineto{\pgfqpoint{3.259982in}{0.744342in}}%
\pgfpathlineto{\pgfqpoint{3.264397in}{0.816771in}}%
\pgfpathlineto{\pgfqpoint{3.268811in}{0.869814in}}%
\pgfpathlineto{\pgfqpoint{3.273226in}{0.893321in}}%
\pgfpathlineto{\pgfqpoint{3.277640in}{0.844629in}}%
\pgfpathlineto{\pgfqpoint{3.282055in}{0.913965in}}%
\pgfpathlineto{\pgfqpoint{3.286469in}{0.844650in}}%
\pgfpathlineto{\pgfqpoint{3.290883in}{1.075649in}}%
\pgfpathlineto{\pgfqpoint{3.295298in}{0.975608in}}%
\pgfpathlineto{\pgfqpoint{3.299712in}{0.937381in}}%
\pgfpathlineto{\pgfqpoint{3.304127in}{0.840224in}}%
\pgfpathlineto{\pgfqpoint{3.308541in}{0.923214in}}%
\pgfpathlineto{\pgfqpoint{3.312956in}{0.795176in}}%
\pgfpathlineto{\pgfqpoint{3.317370in}{0.968701in}}%
\pgfpathlineto{\pgfqpoint{3.321785in}{0.717119in}}%
\pgfpathlineto{\pgfqpoint{3.326199in}{0.903177in}}%
\pgfpathlineto{\pgfqpoint{3.330613in}{0.706385in}}%
\pgfpathlineto{\pgfqpoint{3.339442in}{1.021477in}}%
\pgfpathlineto{\pgfqpoint{3.343857in}{0.746472in}}%
\pgfpathlineto{\pgfqpoint{3.348271in}{0.878244in}}%
\pgfpathlineto{\pgfqpoint{3.352686in}{0.856303in}}%
\pgfpathlineto{\pgfqpoint{3.357100in}{0.874589in}}%
\pgfpathlineto{\pgfqpoint{3.361515in}{0.983519in}}%
\pgfpathlineto{\pgfqpoint{3.365929in}{0.830111in}}%
\pgfpathlineto{\pgfqpoint{3.370343in}{0.841134in}}%
\pgfpathlineto{\pgfqpoint{3.374758in}{0.841723in}}%
\pgfpathlineto{\pgfqpoint{3.379172in}{0.760047in}}%
\pgfpathlineto{\pgfqpoint{3.383587in}{0.711491in}}%
\pgfpathlineto{\pgfqpoint{3.388001in}{0.773079in}}%
\pgfpathlineto{\pgfqpoint{3.392416in}{0.853655in}}%
\pgfpathlineto{\pgfqpoint{3.396830in}{1.026452in}}%
\pgfpathlineto{\pgfqpoint{3.401245in}{0.952684in}}%
\pgfpathlineto{\pgfqpoint{3.405659in}{1.034046in}}%
\pgfpathlineto{\pgfqpoint{3.414488in}{0.925533in}}%
\pgfpathlineto{\pgfqpoint{3.418902in}{1.002127in}}%
\pgfpathlineto{\pgfqpoint{3.423317in}{0.824952in}}%
\pgfpathlineto{\pgfqpoint{3.427731in}{0.875567in}}%
\pgfpathlineto{\pgfqpoint{3.432146in}{0.662690in}}%
\pgfpathlineto{\pgfqpoint{3.436560in}{1.033327in}}%
\pgfpathlineto{\pgfqpoint{3.440975in}{0.681589in}}%
\pgfpathlineto{\pgfqpoint{3.445389in}{0.843307in}}%
\pgfpathlineto{\pgfqpoint{3.449803in}{0.945559in}}%
\pgfpathlineto{\pgfqpoint{3.454218in}{0.907406in}}%
\pgfpathlineto{\pgfqpoint{3.458632in}{0.792235in}}%
\pgfpathlineto{\pgfqpoint{3.463047in}{1.011586in}}%
\pgfpathlineto{\pgfqpoint{3.467461in}{0.845376in}}%
\pgfpathlineto{\pgfqpoint{3.471876in}{0.855912in}}%
\pgfpathlineto{\pgfqpoint{3.476290in}{0.955489in}}%
\pgfpathlineto{\pgfqpoint{3.480705in}{0.856082in}}%
\pgfpathlineto{\pgfqpoint{3.489533in}{0.581185in}}%
\pgfpathlineto{\pgfqpoint{3.493948in}{0.747289in}}%
\pgfpathlineto{\pgfqpoint{3.498362in}{0.858605in}}%
\pgfpathlineto{\pgfqpoint{3.502777in}{0.884413in}}%
\pgfpathlineto{\pgfqpoint{3.507191in}{1.105376in}}%
\pgfpathlineto{\pgfqpoint{3.511606in}{0.800978in}}%
\pgfpathlineto{\pgfqpoint{3.516020in}{0.957149in}}%
\pgfpathlineto{\pgfqpoint{3.520435in}{0.960742in}}%
\pgfpathlineto{\pgfqpoint{3.524849in}{0.942348in}}%
\pgfpathlineto{\pgfqpoint{3.529263in}{0.960776in}}%
\pgfpathlineto{\pgfqpoint{3.533678in}{0.755846in}}%
\pgfpathlineto{\pgfqpoint{3.538092in}{0.909190in}}%
\pgfpathlineto{\pgfqpoint{3.542507in}{0.853059in}}%
\pgfpathlineto{\pgfqpoint{3.546921in}{0.913397in}}%
\pgfpathlineto{\pgfqpoint{3.551336in}{0.748231in}}%
\pgfpathlineto{\pgfqpoint{3.555750in}{0.874757in}}%
\pgfpathlineto{\pgfqpoint{3.560165in}{0.788820in}}%
\pgfpathlineto{\pgfqpoint{3.564579in}{0.744753in}}%
\pgfpathlineto{\pgfqpoint{3.568994in}{1.053583in}}%
\pgfpathlineto{\pgfqpoint{3.573408in}{1.093228in}}%
\pgfpathlineto{\pgfqpoint{3.577822in}{0.868836in}}%
\pgfpathlineto{\pgfqpoint{3.582237in}{0.979797in}}%
\pgfpathlineto{\pgfqpoint{3.586651in}{1.061958in}}%
\pgfpathlineto{\pgfqpoint{3.591066in}{0.831935in}}%
\pgfpathlineto{\pgfqpoint{3.595480in}{0.864654in}}%
\pgfpathlineto{\pgfqpoint{3.599895in}{0.773313in}}%
\pgfpathlineto{\pgfqpoint{3.604309in}{0.892492in}}%
\pgfpathlineto{\pgfqpoint{3.608724in}{0.918793in}}%
\pgfpathlineto{\pgfqpoint{3.613138in}{0.881133in}}%
\pgfpathlineto{\pgfqpoint{3.621967in}{0.998342in}}%
\pgfpathlineto{\pgfqpoint{3.626381in}{0.882906in}}%
\pgfpathlineto{\pgfqpoint{3.630796in}{0.830859in}}%
\pgfpathlineto{\pgfqpoint{3.635210in}{0.876019in}}%
\pgfpathlineto{\pgfqpoint{3.639625in}{0.838183in}}%
\pgfpathlineto{\pgfqpoint{3.644039in}{0.772584in}}%
\pgfpathlineto{\pgfqpoint{3.648454in}{0.918819in}}%
\pgfpathlineto{\pgfqpoint{3.652868in}{0.796205in}}%
\pgfpathlineto{\pgfqpoint{3.657282in}{0.801182in}}%
\pgfpathlineto{\pgfqpoint{3.666111in}{0.875053in}}%
\pgfpathlineto{\pgfqpoint{3.670526in}{0.882591in}}%
\pgfpathlineto{\pgfqpoint{3.674940in}{0.935480in}}%
\pgfpathlineto{\pgfqpoint{3.679355in}{0.950562in}}%
\pgfpathlineto{\pgfqpoint{3.683769in}{1.000781in}}%
\pgfpathlineto{\pgfqpoint{3.688184in}{0.858807in}}%
\pgfpathlineto{\pgfqpoint{3.692598in}{0.955371in}}%
\pgfpathlineto{\pgfqpoint{3.697012in}{0.882639in}}%
\pgfpathlineto{\pgfqpoint{3.701427in}{0.851060in}}%
\pgfpathlineto{\pgfqpoint{3.705841in}{1.001368in}}%
\pgfpathlineto{\pgfqpoint{3.710256in}{0.781363in}}%
\pgfpathlineto{\pgfqpoint{3.714670in}{0.830123in}}%
\pgfpathlineto{\pgfqpoint{3.723499in}{0.735903in}}%
\pgfpathlineto{\pgfqpoint{3.727914in}{1.032123in}}%
\pgfpathlineto{\pgfqpoint{3.732328in}{1.021218in}}%
\pgfpathlineto{\pgfqpoint{3.736742in}{1.116611in}}%
\pgfpathlineto{\pgfqpoint{3.741157in}{0.816663in}}%
\pgfpathlineto{\pgfqpoint{3.745571in}{0.966149in}}%
\pgfpathlineto{\pgfqpoint{3.749986in}{1.031537in}}%
\pgfpathlineto{\pgfqpoint{3.758815in}{0.830919in}}%
\pgfpathlineto{\pgfqpoint{3.763229in}{0.817913in}}%
\pgfpathlineto{\pgfqpoint{3.767644in}{0.949635in}}%
\pgfpathlineto{\pgfqpoint{3.772058in}{0.810999in}}%
\pgfpathlineto{\pgfqpoint{3.776472in}{0.939352in}}%
\pgfpathlineto{\pgfqpoint{3.780887in}{1.129490in}}%
\pgfpathlineto{\pgfqpoint{3.785301in}{1.074875in}}%
\pgfpathlineto{\pgfqpoint{3.789716in}{0.839640in}}%
\pgfpathlineto{\pgfqpoint{3.794130in}{0.911968in}}%
\pgfpathlineto{\pgfqpoint{3.798545in}{0.813359in}}%
\pgfpathlineto{\pgfqpoint{3.802959in}{0.988448in}}%
\pgfpathlineto{\pgfqpoint{3.807374in}{0.943579in}}%
\pgfpathlineto{\pgfqpoint{3.811788in}{0.809694in}}%
\pgfpathlineto{\pgfqpoint{3.816202in}{0.864195in}}%
\pgfpathlineto{\pgfqpoint{3.820617in}{0.739743in}}%
\pgfpathlineto{\pgfqpoint{3.825031in}{0.694958in}}%
\pgfpathlineto{\pgfqpoint{3.829446in}{0.888828in}}%
\pgfpathlineto{\pgfqpoint{3.833860in}{0.821717in}}%
\pgfpathlineto{\pgfqpoint{3.838275in}{1.192758in}}%
\pgfpathlineto{\pgfqpoint{3.842689in}{0.850663in}}%
\pgfpathlineto{\pgfqpoint{3.847104in}{1.151339in}}%
\pgfpathlineto{\pgfqpoint{3.851518in}{1.010151in}}%
\pgfpathlineto{\pgfqpoint{3.855932in}{1.007676in}}%
\pgfpathlineto{\pgfqpoint{3.860347in}{1.091825in}}%
\pgfpathlineto{\pgfqpoint{3.864761in}{0.985435in}}%
\pgfpathlineto{\pgfqpoint{3.869176in}{0.960504in}}%
\pgfpathlineto{\pgfqpoint{3.873590in}{1.033440in}}%
\pgfpathlineto{\pgfqpoint{3.878005in}{0.952413in}}%
\pgfpathlineto{\pgfqpoint{3.882419in}{0.726271in}}%
\pgfpathlineto{\pgfqpoint{3.886834in}{0.860607in}}%
\pgfpathlineto{\pgfqpoint{3.891248in}{0.774096in}}%
\pgfpathlineto{\pgfqpoint{3.895662in}{0.774894in}}%
\pgfpathlineto{\pgfqpoint{3.900077in}{0.952764in}}%
\pgfpathlineto{\pgfqpoint{3.904491in}{0.981875in}}%
\pgfpathlineto{\pgfqpoint{3.908906in}{0.931082in}}%
\pgfpathlineto{\pgfqpoint{3.913320in}{1.061951in}}%
\pgfpathlineto{\pgfqpoint{3.917735in}{1.044425in}}%
\pgfpathlineto{\pgfqpoint{3.922149in}{0.767896in}}%
\pgfpathlineto{\pgfqpoint{3.926564in}{0.765322in}}%
\pgfpathlineto{\pgfqpoint{3.930978in}{0.794023in}}%
\pgfpathlineto{\pgfqpoint{3.935392in}{0.686772in}}%
\pgfpathlineto{\pgfqpoint{3.939807in}{0.986672in}}%
\pgfpathlineto{\pgfqpoint{3.944221in}{0.895057in}}%
\pgfpathlineto{\pgfqpoint{3.948636in}{1.154266in}}%
\pgfpathlineto{\pgfqpoint{3.953050in}{0.846900in}}%
\pgfpathlineto{\pgfqpoint{3.957465in}{0.963753in}}%
\pgfpathlineto{\pgfqpoint{3.961879in}{0.944809in}}%
\pgfpathlineto{\pgfqpoint{3.966294in}{0.920953in}}%
\pgfpathlineto{\pgfqpoint{3.970708in}{0.878619in}}%
\pgfpathlineto{\pgfqpoint{3.975122in}{0.822827in}}%
\pgfpathlineto{\pgfqpoint{3.979537in}{1.042884in}}%
\pgfpathlineto{\pgfqpoint{3.983951in}{0.880022in}}%
\pgfpathlineto{\pgfqpoint{3.988366in}{0.802888in}}%
\pgfpathlineto{\pgfqpoint{3.992780in}{0.844989in}}%
\pgfpathlineto{\pgfqpoint{3.997195in}{0.848726in}}%
\pgfpathlineto{\pgfqpoint{4.001609in}{0.980888in}}%
\pgfpathlineto{\pgfqpoint{4.006024in}{0.883084in}}%
\pgfpathlineto{\pgfqpoint{4.010438in}{0.904749in}}%
\pgfpathlineto{\pgfqpoint{4.014852in}{0.821046in}}%
\pgfpathlineto{\pgfqpoint{4.019267in}{0.957676in}}%
\pgfpathlineto{\pgfqpoint{4.023681in}{0.940390in}}%
\pgfpathlineto{\pgfqpoint{4.028096in}{0.874163in}}%
\pgfpathlineto{\pgfqpoint{4.036925in}{0.781308in}}%
\pgfpathlineto{\pgfqpoint{4.041339in}{0.799709in}}%
\pgfpathlineto{\pgfqpoint{4.045754in}{0.746710in}}%
\pgfpathlineto{\pgfqpoint{4.050168in}{1.363944in}}%
\pgfpathlineto{\pgfqpoint{4.054582in}{0.874058in}}%
\pgfpathlineto{\pgfqpoint{4.058997in}{0.988419in}}%
\pgfpathlineto{\pgfqpoint{4.063411in}{0.823656in}}%
\pgfpathlineto{\pgfqpoint{4.067826in}{0.993769in}}%
\pgfpathlineto{\pgfqpoint{4.072240in}{0.900509in}}%
\pgfpathlineto{\pgfqpoint{4.076655in}{0.851853in}}%
\pgfpathlineto{\pgfqpoint{4.081069in}{0.946402in}}%
\pgfpathlineto{\pgfqpoint{4.085484in}{0.781853in}}%
\pgfpathlineto{\pgfqpoint{4.089898in}{0.966255in}}%
\pgfpathlineto{\pgfqpoint{4.094312in}{0.740163in}}%
\pgfpathlineto{\pgfqpoint{4.098727in}{0.733540in}}%
\pgfpathlineto{\pgfqpoint{4.103141in}{0.822560in}}%
\pgfpathlineto{\pgfqpoint{4.107556in}{0.772157in}}%
\pgfpathlineto{\pgfqpoint{4.111970in}{0.955710in}}%
\pgfpathlineto{\pgfqpoint{4.116385in}{0.862229in}}%
\pgfpathlineto{\pgfqpoint{4.120799in}{0.813436in}}%
\pgfpathlineto{\pgfqpoint{4.125214in}{0.788640in}}%
\pgfpathlineto{\pgfqpoint{4.129628in}{0.774401in}}%
\pgfpathlineto{\pgfqpoint{4.134042in}{0.866706in}}%
\pgfpathlineto{\pgfqpoint{4.138457in}{0.807538in}}%
\pgfpathlineto{\pgfqpoint{4.142871in}{0.813623in}}%
\pgfpathlineto{\pgfqpoint{4.147286in}{0.711924in}}%
\pgfpathlineto{\pgfqpoint{4.151700in}{0.723320in}}%
\pgfpathlineto{\pgfqpoint{4.156115in}{0.813255in}}%
\pgfpathlineto{\pgfqpoint{4.160529in}{1.594609in}}%
\pgfpathlineto{\pgfqpoint{4.164944in}{0.855453in}}%
\pgfpathlineto{\pgfqpoint{4.169358in}{0.746943in}}%
\pgfpathlineto{\pgfqpoint{4.173772in}{0.826569in}}%
\pgfpathlineto{\pgfqpoint{4.178187in}{1.053490in}}%
\pgfpathlineto{\pgfqpoint{4.182601in}{0.886215in}}%
\pgfpathlineto{\pgfqpoint{4.187016in}{0.971451in}}%
\pgfpathlineto{\pgfqpoint{4.191430in}{1.102663in}}%
\pgfpathlineto{\pgfqpoint{4.195845in}{0.811734in}}%
\pgfpathlineto{\pgfqpoint{4.200259in}{0.861009in}}%
\pgfpathlineto{\pgfqpoint{4.204674in}{0.877033in}}%
\pgfpathlineto{\pgfqpoint{4.209088in}{0.883461in}}%
\pgfpathlineto{\pgfqpoint{4.213503in}{0.866767in}}%
\pgfpathlineto{\pgfqpoint{4.217917in}{0.866598in}}%
\pgfpathlineto{\pgfqpoint{4.222331in}{0.891238in}}%
\pgfpathlineto{\pgfqpoint{4.226746in}{0.835566in}}%
\pgfpathlineto{\pgfqpoint{4.231160in}{0.835891in}}%
\pgfpathlineto{\pgfqpoint{4.235575in}{1.034339in}}%
\pgfpathlineto{\pgfqpoint{4.239989in}{0.904571in}}%
\pgfpathlineto{\pgfqpoint{4.244404in}{0.918177in}}%
\pgfpathlineto{\pgfqpoint{4.248818in}{1.044564in}}%
\pgfpathlineto{\pgfqpoint{4.253233in}{0.748334in}}%
\pgfpathlineto{\pgfqpoint{4.257647in}{0.832923in}}%
\pgfpathlineto{\pgfqpoint{4.262061in}{0.822250in}}%
\pgfpathlineto{\pgfqpoint{4.266476in}{0.901937in}}%
\pgfpathlineto{\pgfqpoint{4.270890in}{1.776422in}}%
\pgfpathlineto{\pgfqpoint{4.275305in}{0.841694in}}%
\pgfpathlineto{\pgfqpoint{4.279719in}{0.861328in}}%
\pgfpathlineto{\pgfqpoint{4.284134in}{0.973767in}}%
\pgfpathlineto{\pgfqpoint{4.288548in}{0.987357in}}%
\pgfpathlineto{\pgfqpoint{4.292963in}{0.872404in}}%
\pgfpathlineto{\pgfqpoint{4.297377in}{0.915577in}}%
\pgfpathlineto{\pgfqpoint{4.301791in}{1.039119in}}%
\pgfpathlineto{\pgfqpoint{4.306206in}{0.827064in}}%
\pgfpathlineto{\pgfqpoint{4.310620in}{0.869396in}}%
\pgfpathlineto{\pgfqpoint{4.315035in}{0.756401in}}%
\pgfpathlineto{\pgfqpoint{4.319449in}{0.908687in}}%
\pgfpathlineto{\pgfqpoint{4.323864in}{0.933740in}}%
\pgfpathlineto{\pgfqpoint{4.323864in}{0.933740in}}%
\pgfusepath{stroke}%
\end{pgfscope}%
\begin{pgfscope}%
\pgfpathrectangle{\pgfqpoint{0.625000in}{0.440000in}}{\pgfqpoint{3.875000in}{3.080000in}} %
\pgfusepath{clip}%
\pgfsetrectcap%
\pgfsetroundjoin%
\pgfsetlinewidth{1.505625pt}%
\definecolor{currentstroke}{rgb}{0.000000,0.000000,1.000000}%
\pgfsetstrokecolor{currentstroke}%
\pgfsetdash{}{0pt}%
\pgfpathmoveto{\pgfqpoint{0.840866in}{1.799029in}}%
\pgfpathlineto{\pgfqpoint{0.885011in}{1.774449in}}%
\pgfpathlineto{\pgfqpoint{0.929155in}{1.743795in}}%
\pgfpathlineto{\pgfqpoint{0.973300in}{2.054134in}}%
\pgfpathlineto{\pgfqpoint{1.017444in}{1.807875in}}%
\pgfpathlineto{\pgfqpoint{1.061589in}{1.824358in}}%
\pgfpathlineto{\pgfqpoint{1.105733in}{1.613123in}}%
\pgfpathlineto{\pgfqpoint{1.149878in}{1.626415in}}%
\pgfpathlineto{\pgfqpoint{1.194022in}{1.527184in}}%
\pgfpathlineto{\pgfqpoint{1.238166in}{1.282386in}}%
\pgfpathlineto{\pgfqpoint{1.282311in}{1.524396in}}%
\pgfpathlineto{\pgfqpoint{1.326455in}{1.291748in}}%
\pgfpathlineto{\pgfqpoint{1.370600in}{1.438920in}}%
\pgfpathlineto{\pgfqpoint{1.414744in}{1.157695in}}%
\pgfpathlineto{\pgfqpoint{1.458889in}{1.136108in}}%
\pgfpathlineto{\pgfqpoint{1.503033in}{1.375549in}}%
\pgfpathlineto{\pgfqpoint{1.547178in}{1.113826in}}%
\pgfpathlineto{\pgfqpoint{1.591322in}{1.383088in}}%
\pgfpathlineto{\pgfqpoint{1.635467in}{1.073703in}}%
\pgfpathlineto{\pgfqpoint{1.679611in}{1.108983in}}%
\pgfpathlineto{\pgfqpoint{1.723755in}{1.062691in}}%
\pgfpathlineto{\pgfqpoint{1.767900in}{1.045009in}}%
\pgfpathlineto{\pgfqpoint{1.812044in}{1.096206in}}%
\pgfpathlineto{\pgfqpoint{1.856189in}{1.013828in}}%
\pgfpathlineto{\pgfqpoint{1.900333in}{1.102603in}}%
\pgfpathlineto{\pgfqpoint{1.944478in}{0.995014in}}%
\pgfpathlineto{\pgfqpoint{1.988622in}{1.074683in}}%
\pgfpathlineto{\pgfqpoint{2.032767in}{1.032698in}}%
\pgfpathlineto{\pgfqpoint{2.076911in}{0.965524in}}%
\pgfpathlineto{\pgfqpoint{2.121055in}{1.058209in}}%
\pgfpathlineto{\pgfqpoint{2.165200in}{0.986730in}}%
\pgfpathlineto{\pgfqpoint{2.209344in}{1.096739in}}%
\pgfpathlineto{\pgfqpoint{2.253489in}{1.037014in}}%
\pgfpathlineto{\pgfqpoint{2.297633in}{0.964207in}}%
\pgfpathlineto{\pgfqpoint{2.341778in}{1.038321in}}%
\pgfpathlineto{\pgfqpoint{2.385922in}{0.988763in}}%
\pgfpathlineto{\pgfqpoint{2.430067in}{1.118070in}}%
\pgfpathlineto{\pgfqpoint{2.474211in}{1.046345in}}%
\pgfpathlineto{\pgfqpoint{2.518356in}{0.957582in}}%
\pgfpathlineto{\pgfqpoint{2.562500in}{1.029799in}}%
\pgfpathlineto{\pgfqpoint{2.606644in}{0.938659in}}%
\pgfpathlineto{\pgfqpoint{2.650789in}{1.035430in}}%
\pgfpathlineto{\pgfqpoint{2.694933in}{0.916882in}}%
\pgfpathlineto{\pgfqpoint{2.739078in}{0.998734in}}%
\pgfpathlineto{\pgfqpoint{2.783222in}{0.972015in}}%
\pgfpathlineto{\pgfqpoint{2.827367in}{0.954592in}}%
\pgfpathlineto{\pgfqpoint{2.871511in}{1.043478in}}%
\pgfpathlineto{\pgfqpoint{2.915656in}{0.922097in}}%
\pgfpathlineto{\pgfqpoint{2.959800in}{1.000632in}}%
\pgfpathlineto{\pgfqpoint{3.003945in}{0.979194in}}%
\pgfpathlineto{\pgfqpoint{3.048089in}{0.957003in}}%
\pgfpathlineto{\pgfqpoint{3.092233in}{1.045845in}}%
\pgfpathlineto{\pgfqpoint{3.136378in}{0.924695in}}%
\pgfpathlineto{\pgfqpoint{3.180522in}{1.004311in}}%
\pgfpathlineto{\pgfqpoint{3.224667in}{0.982683in}}%
\pgfpathlineto{\pgfqpoint{3.268811in}{0.958752in}}%
\pgfpathlineto{\pgfqpoint{3.312956in}{1.046302in}}%
\pgfpathlineto{\pgfqpoint{3.357100in}{0.926648in}}%
\pgfpathlineto{\pgfqpoint{3.401245in}{1.007265in}}%
\pgfpathlineto{\pgfqpoint{3.445389in}{0.987518in}}%
\pgfpathlineto{\pgfqpoint{3.489533in}{0.996681in}}%
\pgfpathlineto{\pgfqpoint{3.533678in}{0.974935in}}%
\pgfpathlineto{\pgfqpoint{3.577822in}{0.957844in}}%
\pgfpathlineto{\pgfqpoint{3.621967in}{0.972128in}}%
\pgfpathlineto{\pgfqpoint{3.666111in}{0.975096in}}%
\pgfpathlineto{\pgfqpoint{3.710256in}{0.989256in}}%
\pgfpathlineto{\pgfqpoint{3.754400in}{0.975339in}}%
\pgfpathlineto{\pgfqpoint{3.798545in}{0.960696in}}%
\pgfpathlineto{\pgfqpoint{3.842689in}{0.975661in}}%
\pgfpathlineto{\pgfqpoint{3.886834in}{0.977312in}}%
\pgfpathlineto{\pgfqpoint{3.930978in}{0.990121in}}%
\pgfpathlineto{\pgfqpoint{3.975122in}{0.976894in}}%
\pgfpathlineto{\pgfqpoint{4.019267in}{0.962597in}}%
\pgfpathlineto{\pgfqpoint{4.063411in}{0.979605in}}%
\pgfpathlineto{\pgfqpoint{4.107556in}{0.977800in}}%
\pgfpathlineto{\pgfqpoint{4.151700in}{0.990174in}}%
\pgfpathlineto{\pgfqpoint{4.195845in}{0.978513in}}%
\pgfpathlineto{\pgfqpoint{4.239989in}{0.964373in}}%
\pgfpathlineto{\pgfqpoint{4.284134in}{0.982320in}}%
\pgfusepath{stroke}%
\end{pgfscope}%
\begin{pgfscope}%
\pgfsetrectcap%
\pgfsetmiterjoin%
\pgfsetlinewidth{0.803000pt}%
\definecolor{currentstroke}{rgb}{0.000000,0.000000,0.000000}%
\pgfsetstrokecolor{currentstroke}%
\pgfsetdash{}{0pt}%
\pgfpathmoveto{\pgfqpoint{0.625000in}{0.440000in}}%
\pgfpathlineto{\pgfqpoint{0.625000in}{3.520000in}}%
\pgfusepath{stroke}%
\end{pgfscope}%
\begin{pgfscope}%
\pgfsetrectcap%
\pgfsetmiterjoin%
\pgfsetlinewidth{0.803000pt}%
\definecolor{currentstroke}{rgb}{0.000000,0.000000,0.000000}%
\pgfsetstrokecolor{currentstroke}%
\pgfsetdash{}{0pt}%
\pgfpathmoveto{\pgfqpoint{4.500000in}{0.440000in}}%
\pgfpathlineto{\pgfqpoint{4.500000in}{3.520000in}}%
\pgfusepath{stroke}%
\end{pgfscope}%
\begin{pgfscope}%
\pgfsetrectcap%
\pgfsetmiterjoin%
\pgfsetlinewidth{0.803000pt}%
\definecolor{currentstroke}{rgb}{0.000000,0.000000,0.000000}%
\pgfsetstrokecolor{currentstroke}%
\pgfsetdash{}{0pt}%
\pgfpathmoveto{\pgfqpoint{0.625000in}{0.440000in}}%
\pgfpathlineto{\pgfqpoint{4.500000in}{0.440000in}}%
\pgfusepath{stroke}%
\end{pgfscope}%
\begin{pgfscope}%
\pgfsetrectcap%
\pgfsetmiterjoin%
\pgfsetlinewidth{0.803000pt}%
\definecolor{currentstroke}{rgb}{0.000000,0.000000,0.000000}%
\pgfsetstrokecolor{currentstroke}%
\pgfsetdash{}{0pt}%
\pgfpathmoveto{\pgfqpoint{0.625000in}{3.520000in}}%
\pgfpathlineto{\pgfqpoint{4.500000in}{3.520000in}}%
\pgfusepath{stroke}%
\end{pgfscope}%
\end{pgfpicture}%
\makeatother%
\endgroup%
}
		\caption{\textbf{Unet\_Weighted\_4}}
	\end {subfigure}

	\begin {subfigure}[b]{0.4\linewidth}
		\scalebox{0.70}{%% Creator: Matplotlib, PGF backend
%%
%% To include the figure in your LaTeX document, write
%%   \input{<filename>.pgf}
%%
%% Make sure the required packages are loaded in your preamble
%%   \usepackage{pgf}
%%
%% Figures using additional raster images can only be included by \input if
%% they are in the same directory as the main LaTeX file. For loading figures
%% from other directories you can use the `import` package
%%   \usepackage{import}
%% and then include the figures with
%%   \import{<path to file>}{<filename>.pgf}
%%
%% Matplotlib used the following preamble
%%   \usepackage{fontspec}
%%   \setmainfont{DejaVu Serif}
%%   \setsansfont{DejaVu Sans}
%%   \setmonofont{DejaVu Sans Mono}
%%
\begingroup%
\makeatletter%
\begin{pgfpicture}%
\pgfpathrectangle{\pgfpointorigin}{\pgfqpoint{5.000000in}{4.000000in}}%
\pgfusepath{use as bounding box, clip}%
\begin{pgfscope}%
\pgfsetbuttcap%
\pgfsetmiterjoin%
\definecolor{currentfill}{rgb}{1.000000,1.000000,1.000000}%
\pgfsetfillcolor{currentfill}%
\pgfsetlinewidth{0.000000pt}%
\definecolor{currentstroke}{rgb}{1.000000,1.000000,1.000000}%
\pgfsetstrokecolor{currentstroke}%
\pgfsetdash{}{0pt}%
\pgfpathmoveto{\pgfqpoint{0.000000in}{0.000000in}}%
\pgfpathlineto{\pgfqpoint{5.000000in}{0.000000in}}%
\pgfpathlineto{\pgfqpoint{5.000000in}{4.000000in}}%
\pgfpathlineto{\pgfqpoint{0.000000in}{4.000000in}}%
\pgfpathclose%
\pgfusepath{fill}%
\end{pgfscope}%
\begin{pgfscope}%
\pgfsetbuttcap%
\pgfsetmiterjoin%
\definecolor{currentfill}{rgb}{1.000000,1.000000,1.000000}%
\pgfsetfillcolor{currentfill}%
\pgfsetlinewidth{0.000000pt}%
\definecolor{currentstroke}{rgb}{0.000000,0.000000,0.000000}%
\pgfsetstrokecolor{currentstroke}%
\pgfsetstrokeopacity{0.000000}%
\pgfsetdash{}{0pt}%
\pgfpathmoveto{\pgfqpoint{0.625000in}{0.440000in}}%
\pgfpathlineto{\pgfqpoint{4.500000in}{0.440000in}}%
\pgfpathlineto{\pgfqpoint{4.500000in}{3.520000in}}%
\pgfpathlineto{\pgfqpoint{0.625000in}{3.520000in}}%
\pgfpathclose%
\pgfusepath{fill}%
\end{pgfscope}%
\begin{pgfscope}%
\pgfsetbuttcap%
\pgfsetroundjoin%
\definecolor{currentfill}{rgb}{0.000000,0.000000,0.000000}%
\pgfsetfillcolor{currentfill}%
\pgfsetlinewidth{0.803000pt}%
\definecolor{currentstroke}{rgb}{0.000000,0.000000,0.000000}%
\pgfsetstrokecolor{currentstroke}%
\pgfsetdash{}{0pt}%
\pgfsys@defobject{currentmarker}{\pgfqpoint{0.000000in}{-0.048611in}}{\pgfqpoint{0.000000in}{0.000000in}}{%
\pgfpathmoveto{\pgfqpoint{0.000000in}{0.000000in}}%
\pgfpathlineto{\pgfqpoint{0.000000in}{-0.048611in}}%
\pgfusepath{stroke,fill}%
}%
\begin{pgfscope}%
\pgfsys@transformshift{0.796722in}{0.440000in}%
\pgfsys@useobject{currentmarker}{}%
\end{pgfscope}%
\end{pgfscope}%
\begin{pgfscope}%
\pgftext[x=0.796722in,y=0.342778in,,top]{\sffamily\fontsize{10.000000}{12.000000}\selectfont 0}%
\end{pgfscope}%
\begin{pgfscope}%
\pgfsetbuttcap%
\pgfsetroundjoin%
\definecolor{currentfill}{rgb}{0.000000,0.000000,0.000000}%
\pgfsetfillcolor{currentfill}%
\pgfsetlinewidth{0.803000pt}%
\definecolor{currentstroke}{rgb}{0.000000,0.000000,0.000000}%
\pgfsetstrokecolor{currentstroke}%
\pgfsetdash{}{0pt}%
\pgfsys@defobject{currentmarker}{\pgfqpoint{0.000000in}{-0.048611in}}{\pgfqpoint{0.000000in}{0.000000in}}{%
\pgfpathmoveto{\pgfqpoint{0.000000in}{0.000000in}}%
\pgfpathlineto{\pgfqpoint{0.000000in}{-0.048611in}}%
\pgfusepath{stroke,fill}%
}%
\begin{pgfscope}%
\pgfsys@transformshift{1.385300in}{0.440000in}%
\pgfsys@useobject{currentmarker}{}%
\end{pgfscope}%
\end{pgfscope}%
\begin{pgfscope}%
\pgftext[x=1.385300in,y=0.342778in,,top]{\sffamily\fontsize{10.000000}{12.000000}\selectfont 5}%
\end{pgfscope}%
\begin{pgfscope}%
\pgfsetbuttcap%
\pgfsetroundjoin%
\definecolor{currentfill}{rgb}{0.000000,0.000000,0.000000}%
\pgfsetfillcolor{currentfill}%
\pgfsetlinewidth{0.803000pt}%
\definecolor{currentstroke}{rgb}{0.000000,0.000000,0.000000}%
\pgfsetstrokecolor{currentstroke}%
\pgfsetdash{}{0pt}%
\pgfsys@defobject{currentmarker}{\pgfqpoint{0.000000in}{-0.048611in}}{\pgfqpoint{0.000000in}{0.000000in}}{%
\pgfpathmoveto{\pgfqpoint{0.000000in}{0.000000in}}%
\pgfpathlineto{\pgfqpoint{0.000000in}{-0.048611in}}%
\pgfusepath{stroke,fill}%
}%
\begin{pgfscope}%
\pgfsys@transformshift{1.973878in}{0.440000in}%
\pgfsys@useobject{currentmarker}{}%
\end{pgfscope}%
\end{pgfscope}%
\begin{pgfscope}%
\pgftext[x=1.973878in,y=0.342778in,,top]{\sffamily\fontsize{10.000000}{12.000000}\selectfont 10}%
\end{pgfscope}%
\begin{pgfscope}%
\pgfsetbuttcap%
\pgfsetroundjoin%
\definecolor{currentfill}{rgb}{0.000000,0.000000,0.000000}%
\pgfsetfillcolor{currentfill}%
\pgfsetlinewidth{0.803000pt}%
\definecolor{currentstroke}{rgb}{0.000000,0.000000,0.000000}%
\pgfsetstrokecolor{currentstroke}%
\pgfsetdash{}{0pt}%
\pgfsys@defobject{currentmarker}{\pgfqpoint{0.000000in}{-0.048611in}}{\pgfqpoint{0.000000in}{0.000000in}}{%
\pgfpathmoveto{\pgfqpoint{0.000000in}{0.000000in}}%
\pgfpathlineto{\pgfqpoint{0.000000in}{-0.048611in}}%
\pgfusepath{stroke,fill}%
}%
\begin{pgfscope}%
\pgfsys@transformshift{2.562456in}{0.440000in}%
\pgfsys@useobject{currentmarker}{}%
\end{pgfscope}%
\end{pgfscope}%
\begin{pgfscope}%
\pgftext[x=2.562456in,y=0.342778in,,top]{\sffamily\fontsize{10.000000}{12.000000}\selectfont 15}%
\end{pgfscope}%
\begin{pgfscope}%
\pgfsetbuttcap%
\pgfsetroundjoin%
\definecolor{currentfill}{rgb}{0.000000,0.000000,0.000000}%
\pgfsetfillcolor{currentfill}%
\pgfsetlinewidth{0.803000pt}%
\definecolor{currentstroke}{rgb}{0.000000,0.000000,0.000000}%
\pgfsetstrokecolor{currentstroke}%
\pgfsetdash{}{0pt}%
\pgfsys@defobject{currentmarker}{\pgfqpoint{0.000000in}{-0.048611in}}{\pgfqpoint{0.000000in}{0.000000in}}{%
\pgfpathmoveto{\pgfqpoint{0.000000in}{0.000000in}}%
\pgfpathlineto{\pgfqpoint{0.000000in}{-0.048611in}}%
\pgfusepath{stroke,fill}%
}%
\begin{pgfscope}%
\pgfsys@transformshift{3.151034in}{0.440000in}%
\pgfsys@useobject{currentmarker}{}%
\end{pgfscope}%
\end{pgfscope}%
\begin{pgfscope}%
\pgftext[x=3.151034in,y=0.342778in,,top]{\sffamily\fontsize{10.000000}{12.000000}\selectfont 20}%
\end{pgfscope}%
\begin{pgfscope}%
\pgfsetbuttcap%
\pgfsetroundjoin%
\definecolor{currentfill}{rgb}{0.000000,0.000000,0.000000}%
\pgfsetfillcolor{currentfill}%
\pgfsetlinewidth{0.803000pt}%
\definecolor{currentstroke}{rgb}{0.000000,0.000000,0.000000}%
\pgfsetstrokecolor{currentstroke}%
\pgfsetdash{}{0pt}%
\pgfsys@defobject{currentmarker}{\pgfqpoint{0.000000in}{-0.048611in}}{\pgfqpoint{0.000000in}{0.000000in}}{%
\pgfpathmoveto{\pgfqpoint{0.000000in}{0.000000in}}%
\pgfpathlineto{\pgfqpoint{0.000000in}{-0.048611in}}%
\pgfusepath{stroke,fill}%
}%
\begin{pgfscope}%
\pgfsys@transformshift{3.739612in}{0.440000in}%
\pgfsys@useobject{currentmarker}{}%
\end{pgfscope}%
\end{pgfscope}%
\begin{pgfscope}%
\pgftext[x=3.739612in,y=0.342778in,,top]{\sffamily\fontsize{10.000000}{12.000000}\selectfont 25}%
\end{pgfscope}%
\begin{pgfscope}%
\pgfsetbuttcap%
\pgfsetroundjoin%
\definecolor{currentfill}{rgb}{0.000000,0.000000,0.000000}%
\pgfsetfillcolor{currentfill}%
\pgfsetlinewidth{0.803000pt}%
\definecolor{currentstroke}{rgb}{0.000000,0.000000,0.000000}%
\pgfsetstrokecolor{currentstroke}%
\pgfsetdash{}{0pt}%
\pgfsys@defobject{currentmarker}{\pgfqpoint{0.000000in}{-0.048611in}}{\pgfqpoint{0.000000in}{0.000000in}}{%
\pgfpathmoveto{\pgfqpoint{0.000000in}{0.000000in}}%
\pgfpathlineto{\pgfqpoint{0.000000in}{-0.048611in}}%
\pgfusepath{stroke,fill}%
}%
\begin{pgfscope}%
\pgfsys@transformshift{4.328190in}{0.440000in}%
\pgfsys@useobject{currentmarker}{}%
\end{pgfscope}%
\end{pgfscope}%
\begin{pgfscope}%
\pgftext[x=4.328190in,y=0.342778in,,top]{\sffamily\fontsize{10.000000}{12.000000}\selectfont 30}%
\end{pgfscope}%
\begin{pgfscope}%
\pgftext[x=2.562500in,y=0.152809in,,top]{\sffamily\fontsize{10.000000}{12.000000}\selectfont Epochs}%
\end{pgfscope}%
\begin{pgfscope}%
\pgfsetbuttcap%
\pgfsetroundjoin%
\definecolor{currentfill}{rgb}{0.000000,0.000000,0.000000}%
\pgfsetfillcolor{currentfill}%
\pgfsetlinewidth{0.803000pt}%
\definecolor{currentstroke}{rgb}{0.000000,0.000000,0.000000}%
\pgfsetstrokecolor{currentstroke}%
\pgfsetdash{}{0pt}%
\pgfsys@defobject{currentmarker}{\pgfqpoint{-0.048611in}{0.000000in}}{\pgfqpoint{0.000000in}{0.000000in}}{%
\pgfpathmoveto{\pgfqpoint{0.000000in}{0.000000in}}%
\pgfpathlineto{\pgfqpoint{-0.048611in}{0.000000in}}%
\pgfusepath{stroke,fill}%
}%
\begin{pgfscope}%
\pgfsys@transformshift{0.625000in}{0.580000in}%
\pgfsys@useobject{currentmarker}{}%
\end{pgfscope}%
\end{pgfscope}%
\begin{pgfscope}%
\pgftext[x=0.306898in,y=0.527238in,left,base]{\sffamily\fontsize{10.000000}{12.000000}\selectfont 0.0}%
\end{pgfscope}%
\begin{pgfscope}%
\pgfsetbuttcap%
\pgfsetroundjoin%
\definecolor{currentfill}{rgb}{0.000000,0.000000,0.000000}%
\pgfsetfillcolor{currentfill}%
\pgfsetlinewidth{0.803000pt}%
\definecolor{currentstroke}{rgb}{0.000000,0.000000,0.000000}%
\pgfsetstrokecolor{currentstroke}%
\pgfsetdash{}{0pt}%
\pgfsys@defobject{currentmarker}{\pgfqpoint{-0.048611in}{0.000000in}}{\pgfqpoint{0.000000in}{0.000000in}}{%
\pgfpathmoveto{\pgfqpoint{0.000000in}{0.000000in}}%
\pgfpathlineto{\pgfqpoint{-0.048611in}{0.000000in}}%
\pgfusepath{stroke,fill}%
}%
\begin{pgfscope}%
\pgfsys@transformshift{0.625000in}{1.142709in}%
\pgfsys@useobject{currentmarker}{}%
\end{pgfscope}%
\end{pgfscope}%
\begin{pgfscope}%
\pgftext[x=0.306898in,y=1.089948in,left,base]{\sffamily\fontsize{10.000000}{12.000000}\selectfont 0.2}%
\end{pgfscope}%
\begin{pgfscope}%
\pgfsetbuttcap%
\pgfsetroundjoin%
\definecolor{currentfill}{rgb}{0.000000,0.000000,0.000000}%
\pgfsetfillcolor{currentfill}%
\pgfsetlinewidth{0.803000pt}%
\definecolor{currentstroke}{rgb}{0.000000,0.000000,0.000000}%
\pgfsetstrokecolor{currentstroke}%
\pgfsetdash{}{0pt}%
\pgfsys@defobject{currentmarker}{\pgfqpoint{-0.048611in}{0.000000in}}{\pgfqpoint{0.000000in}{0.000000in}}{%
\pgfpathmoveto{\pgfqpoint{0.000000in}{0.000000in}}%
\pgfpathlineto{\pgfqpoint{-0.048611in}{0.000000in}}%
\pgfusepath{stroke,fill}%
}%
\begin{pgfscope}%
\pgfsys@transformshift{0.625000in}{1.705419in}%
\pgfsys@useobject{currentmarker}{}%
\end{pgfscope}%
\end{pgfscope}%
\begin{pgfscope}%
\pgftext[x=0.306898in,y=1.652657in,left,base]{\sffamily\fontsize{10.000000}{12.000000}\selectfont 0.4}%
\end{pgfscope}%
\begin{pgfscope}%
\pgfsetbuttcap%
\pgfsetroundjoin%
\definecolor{currentfill}{rgb}{0.000000,0.000000,0.000000}%
\pgfsetfillcolor{currentfill}%
\pgfsetlinewidth{0.803000pt}%
\definecolor{currentstroke}{rgb}{0.000000,0.000000,0.000000}%
\pgfsetstrokecolor{currentstroke}%
\pgfsetdash{}{0pt}%
\pgfsys@defobject{currentmarker}{\pgfqpoint{-0.048611in}{0.000000in}}{\pgfqpoint{0.000000in}{0.000000in}}{%
\pgfpathmoveto{\pgfqpoint{0.000000in}{0.000000in}}%
\pgfpathlineto{\pgfqpoint{-0.048611in}{0.000000in}}%
\pgfusepath{stroke,fill}%
}%
\begin{pgfscope}%
\pgfsys@transformshift{0.625000in}{2.268128in}%
\pgfsys@useobject{currentmarker}{}%
\end{pgfscope}%
\end{pgfscope}%
\begin{pgfscope}%
\pgftext[x=0.306898in,y=2.215367in,left,base]{\sffamily\fontsize{10.000000}{12.000000}\selectfont 0.6}%
\end{pgfscope}%
\begin{pgfscope}%
\pgfsetbuttcap%
\pgfsetroundjoin%
\definecolor{currentfill}{rgb}{0.000000,0.000000,0.000000}%
\pgfsetfillcolor{currentfill}%
\pgfsetlinewidth{0.803000pt}%
\definecolor{currentstroke}{rgb}{0.000000,0.000000,0.000000}%
\pgfsetstrokecolor{currentstroke}%
\pgfsetdash{}{0pt}%
\pgfsys@defobject{currentmarker}{\pgfqpoint{-0.048611in}{0.000000in}}{\pgfqpoint{0.000000in}{0.000000in}}{%
\pgfpathmoveto{\pgfqpoint{0.000000in}{0.000000in}}%
\pgfpathlineto{\pgfqpoint{-0.048611in}{0.000000in}}%
\pgfusepath{stroke,fill}%
}%
\begin{pgfscope}%
\pgfsys@transformshift{0.625000in}{2.830838in}%
\pgfsys@useobject{currentmarker}{}%
\end{pgfscope}%
\end{pgfscope}%
\begin{pgfscope}%
\pgftext[x=0.306898in,y=2.778076in,left,base]{\sffamily\fontsize{10.000000}{12.000000}\selectfont 0.8}%
\end{pgfscope}%
\begin{pgfscope}%
\pgfsetbuttcap%
\pgfsetroundjoin%
\definecolor{currentfill}{rgb}{0.000000,0.000000,0.000000}%
\pgfsetfillcolor{currentfill}%
\pgfsetlinewidth{0.803000pt}%
\definecolor{currentstroke}{rgb}{0.000000,0.000000,0.000000}%
\pgfsetstrokecolor{currentstroke}%
\pgfsetdash{}{0pt}%
\pgfsys@defobject{currentmarker}{\pgfqpoint{-0.048611in}{0.000000in}}{\pgfqpoint{0.000000in}{0.000000in}}{%
\pgfpathmoveto{\pgfqpoint{0.000000in}{0.000000in}}%
\pgfpathlineto{\pgfqpoint{-0.048611in}{0.000000in}}%
\pgfusepath{stroke,fill}%
}%
\begin{pgfscope}%
\pgfsys@transformshift{0.625000in}{3.393547in}%
\pgfsys@useobject{currentmarker}{}%
\end{pgfscope}%
\end{pgfscope}%
\begin{pgfscope}%
\pgftext[x=0.306898in,y=3.340786in,left,base]{\sffamily\fontsize{10.000000}{12.000000}\selectfont 1.0}%
\end{pgfscope}%
\begin{pgfscope}%
\pgftext[x=0.251343in,y=1.980000in,,bottom,rotate=90.000000]{\sffamily\fontsize{10.000000}{12.000000}\selectfont F-Measure score}%
\end{pgfscope}%
\begin{pgfscope}%
\pgfpathrectangle{\pgfqpoint{0.625000in}{0.440000in}}{\pgfqpoint{3.875000in}{3.080000in}} %
\pgfusepath{clip}%
\pgfsetrectcap%
\pgfsetroundjoin%
\pgfsetlinewidth{1.505625pt}%
\definecolor{currentstroke}{rgb}{0.752941,0.752941,0.752941}%
\pgfsetstrokecolor{currentstroke}%
\pgfsetdash{}{0pt}%
\pgfpathmoveto{\pgfqpoint{0.801136in}{3.203821in}}%
\pgfpathlineto{\pgfqpoint{0.805551in}{3.219310in}}%
\pgfpathlineto{\pgfqpoint{0.809965in}{3.286807in}}%
\pgfpathlineto{\pgfqpoint{0.814380in}{2.830686in}}%
\pgfpathlineto{\pgfqpoint{0.818794in}{3.243920in}}%
\pgfpathlineto{\pgfqpoint{0.823209in}{3.268842in}}%
\pgfpathlineto{\pgfqpoint{0.827623in}{3.252589in}}%
\pgfpathlineto{\pgfqpoint{0.836452in}{3.323726in}}%
\pgfpathlineto{\pgfqpoint{0.840866in}{3.275308in}}%
\pgfpathlineto{\pgfqpoint{0.845281in}{3.304203in}}%
\pgfpathlineto{\pgfqpoint{0.849695in}{3.285338in}}%
\pgfpathlineto{\pgfqpoint{0.854110in}{3.275519in}}%
\pgfpathlineto{\pgfqpoint{0.858524in}{3.297501in}}%
\pgfpathlineto{\pgfqpoint{0.862939in}{3.261336in}}%
\pgfpathlineto{\pgfqpoint{0.867353in}{2.721011in}}%
\pgfpathlineto{\pgfqpoint{0.871767in}{2.921077in}}%
\pgfpathlineto{\pgfqpoint{0.876182in}{3.279160in}}%
\pgfpathlineto{\pgfqpoint{0.880596in}{3.157572in}}%
\pgfpathlineto{\pgfqpoint{0.885011in}{3.300771in}}%
\pgfpathlineto{\pgfqpoint{0.889425in}{3.317159in}}%
\pgfpathlineto{\pgfqpoint{0.893840in}{3.292034in}}%
\pgfpathlineto{\pgfqpoint{0.898254in}{3.285479in}}%
\pgfpathlineto{\pgfqpoint{0.902669in}{3.329981in}}%
\pgfpathlineto{\pgfqpoint{0.907083in}{3.179554in}}%
\pgfpathlineto{\pgfqpoint{0.911497in}{3.278932in}}%
\pgfpathlineto{\pgfqpoint{0.915912in}{3.301797in}}%
\pgfpathlineto{\pgfqpoint{0.920326in}{3.248168in}}%
\pgfpathlineto{\pgfqpoint{0.924741in}{2.567614in}}%
\pgfpathlineto{\pgfqpoint{0.929155in}{3.315651in}}%
\pgfpathlineto{\pgfqpoint{0.933570in}{3.303052in}}%
\pgfpathlineto{\pgfqpoint{0.937984in}{3.218612in}}%
\pgfpathlineto{\pgfqpoint{0.942399in}{3.296030in}}%
\pgfpathlineto{\pgfqpoint{0.946813in}{3.334609in}}%
\pgfpathlineto{\pgfqpoint{0.951228in}{3.293182in}}%
\pgfpathlineto{\pgfqpoint{0.955642in}{3.336007in}}%
\pgfpathlineto{\pgfqpoint{0.960056in}{3.314489in}}%
\pgfpathlineto{\pgfqpoint{0.964471in}{3.259952in}}%
\pgfpathlineto{\pgfqpoint{0.968885in}{3.313561in}}%
\pgfpathlineto{\pgfqpoint{0.973300in}{3.244286in}}%
\pgfpathlineto{\pgfqpoint{0.977714in}{2.353472in}}%
\pgfpathlineto{\pgfqpoint{0.982129in}{2.691570in}}%
\pgfpathlineto{\pgfqpoint{0.986543in}{3.230370in}}%
\pgfpathlineto{\pgfqpoint{0.990958in}{3.175351in}}%
\pgfpathlineto{\pgfqpoint{0.995372in}{3.338919in}}%
\pgfpathlineto{\pgfqpoint{0.999786in}{3.314844in}}%
\pgfpathlineto{\pgfqpoint{1.004201in}{3.310745in}}%
\pgfpathlineto{\pgfqpoint{1.008615in}{3.351563in}}%
\pgfpathlineto{\pgfqpoint{1.013030in}{3.323431in}}%
\pgfpathlineto{\pgfqpoint{1.017444in}{3.317165in}}%
\pgfpathlineto{\pgfqpoint{1.021859in}{3.181282in}}%
\pgfpathlineto{\pgfqpoint{1.026273in}{3.282958in}}%
\pgfpathlineto{\pgfqpoint{1.030688in}{3.224909in}}%
\pgfpathlineto{\pgfqpoint{1.035102in}{3.122105in}}%
\pgfpathlineto{\pgfqpoint{1.039516in}{3.248506in}}%
\pgfpathlineto{\pgfqpoint{1.043931in}{3.239193in}}%
\pgfpathlineto{\pgfqpoint{1.048345in}{3.145114in}}%
\pgfpathlineto{\pgfqpoint{1.052760in}{3.318409in}}%
\pgfpathlineto{\pgfqpoint{1.057174in}{3.346949in}}%
\pgfpathlineto{\pgfqpoint{1.061589in}{3.326078in}}%
\pgfpathlineto{\pgfqpoint{1.066003in}{3.334530in}}%
\pgfpathlineto{\pgfqpoint{1.074832in}{3.230539in}}%
\pgfpathlineto{\pgfqpoint{1.079246in}{3.214366in}}%
\pgfpathlineto{\pgfqpoint{1.083661in}{3.176915in}}%
\pgfpathlineto{\pgfqpoint{1.088075in}{2.235058in}}%
\pgfpathlineto{\pgfqpoint{1.092490in}{2.674627in}}%
\pgfpathlineto{\pgfqpoint{1.096904in}{3.249342in}}%
\pgfpathlineto{\pgfqpoint{1.105733in}{3.314765in}}%
\pgfpathlineto{\pgfqpoint{1.110148in}{3.262720in}}%
\pgfpathlineto{\pgfqpoint{1.114562in}{3.311665in}}%
\pgfpathlineto{\pgfqpoint{1.118976in}{3.347731in}}%
\pgfpathlineto{\pgfqpoint{1.123391in}{3.338255in}}%
\pgfpathlineto{\pgfqpoint{1.127805in}{3.291711in}}%
\pgfpathlineto{\pgfqpoint{1.132220in}{3.204685in}}%
\pgfpathlineto{\pgfqpoint{1.136634in}{3.221884in}}%
\pgfpathlineto{\pgfqpoint{1.141049in}{3.231431in}}%
\pgfpathlineto{\pgfqpoint{1.145463in}{3.305438in}}%
\pgfpathlineto{\pgfqpoint{1.149878in}{3.228344in}}%
\pgfpathlineto{\pgfqpoint{1.154292in}{3.268181in}}%
\pgfpathlineto{\pgfqpoint{1.158706in}{3.218938in}}%
\pgfpathlineto{\pgfqpoint{1.163121in}{3.329626in}}%
\pgfpathlineto{\pgfqpoint{1.167535in}{3.347771in}}%
\pgfpathlineto{\pgfqpoint{1.171950in}{3.350863in}}%
\pgfpathlineto{\pgfqpoint{1.176364in}{3.320372in}}%
\pgfpathlineto{\pgfqpoint{1.185193in}{3.224352in}}%
\pgfpathlineto{\pgfqpoint{1.189608in}{3.226127in}}%
\pgfpathlineto{\pgfqpoint{1.194022in}{3.236231in}}%
\pgfpathlineto{\pgfqpoint{1.198436in}{3.076719in}}%
\pgfpathlineto{\pgfqpoint{1.202851in}{3.016884in}}%
\pgfpathlineto{\pgfqpoint{1.207265in}{3.240924in}}%
\pgfpathlineto{\pgfqpoint{1.211680in}{3.273941in}}%
\pgfpathlineto{\pgfqpoint{1.220509in}{3.303083in}}%
\pgfpathlineto{\pgfqpoint{1.224923in}{3.265123in}}%
\pgfpathlineto{\pgfqpoint{1.229338in}{3.328653in}}%
\pgfpathlineto{\pgfqpoint{1.233752in}{3.342479in}}%
\pgfpathlineto{\pgfqpoint{1.238166in}{3.213038in}}%
\pgfpathlineto{\pgfqpoint{1.242581in}{3.200724in}}%
\pgfpathlineto{\pgfqpoint{1.246995in}{3.332564in}}%
\pgfpathlineto{\pgfqpoint{1.251410in}{3.286649in}}%
\pgfpathlineto{\pgfqpoint{1.255824in}{3.288993in}}%
\pgfpathlineto{\pgfqpoint{1.260239in}{3.285051in}}%
\pgfpathlineto{\pgfqpoint{1.264653in}{3.273389in}}%
\pgfpathlineto{\pgfqpoint{1.269068in}{3.276422in}}%
\pgfpathlineto{\pgfqpoint{1.273482in}{3.339493in}}%
\pgfpathlineto{\pgfqpoint{1.277896in}{3.319576in}}%
\pgfpathlineto{\pgfqpoint{1.282311in}{3.322317in}}%
\pgfpathlineto{\pgfqpoint{1.286725in}{3.307267in}}%
\pgfpathlineto{\pgfqpoint{1.291140in}{3.318448in}}%
\pgfpathlineto{\pgfqpoint{1.295554in}{3.253365in}}%
\pgfpathlineto{\pgfqpoint{1.299969in}{3.275412in}}%
\pgfpathlineto{\pgfqpoint{1.304383in}{3.268553in}}%
\pgfpathlineto{\pgfqpoint{1.308798in}{3.246134in}}%
\pgfpathlineto{\pgfqpoint{1.313212in}{3.196855in}}%
\pgfpathlineto{\pgfqpoint{1.317626in}{3.272635in}}%
\pgfpathlineto{\pgfqpoint{1.322041in}{3.269245in}}%
\pgfpathlineto{\pgfqpoint{1.326455in}{3.314990in}}%
\pgfpathlineto{\pgfqpoint{1.330870in}{3.347864in}}%
\pgfpathlineto{\pgfqpoint{1.335284in}{3.296632in}}%
\pgfpathlineto{\pgfqpoint{1.339699in}{3.325828in}}%
\pgfpathlineto{\pgfqpoint{1.344113in}{3.303832in}}%
\pgfpathlineto{\pgfqpoint{1.348528in}{3.213683in}}%
\pgfpathlineto{\pgfqpoint{1.352942in}{3.188220in}}%
\pgfpathlineto{\pgfqpoint{1.357356in}{3.233316in}}%
\pgfpathlineto{\pgfqpoint{1.361771in}{3.327589in}}%
\pgfpathlineto{\pgfqpoint{1.366185in}{3.282308in}}%
\pgfpathlineto{\pgfqpoint{1.370600in}{3.276270in}}%
\pgfpathlineto{\pgfqpoint{1.375014in}{3.258066in}}%
\pgfpathlineto{\pgfqpoint{1.379429in}{3.210062in}}%
\pgfpathlineto{\pgfqpoint{1.383843in}{3.315466in}}%
\pgfpathlineto{\pgfqpoint{1.388258in}{3.338748in}}%
\pgfpathlineto{\pgfqpoint{1.392672in}{3.311735in}}%
\pgfpathlineto{\pgfqpoint{1.397086in}{3.311074in}}%
\pgfpathlineto{\pgfqpoint{1.401501in}{3.256494in}}%
\pgfpathlineto{\pgfqpoint{1.405915in}{3.276644in}}%
\pgfpathlineto{\pgfqpoint{1.410330in}{3.286621in}}%
\pgfpathlineto{\pgfqpoint{1.419159in}{3.261794in}}%
\pgfpathlineto{\pgfqpoint{1.423573in}{3.228516in}}%
\pgfpathlineto{\pgfqpoint{1.427988in}{3.229396in}}%
\pgfpathlineto{\pgfqpoint{1.432402in}{3.286784in}}%
\pgfpathlineto{\pgfqpoint{1.441231in}{3.350331in}}%
\pgfpathlineto{\pgfqpoint{1.445645in}{3.348618in}}%
\pgfpathlineto{\pgfqpoint{1.450060in}{3.341044in}}%
\pgfpathlineto{\pgfqpoint{1.454474in}{3.339381in}}%
\pgfpathlineto{\pgfqpoint{1.458889in}{3.178542in}}%
\pgfpathlineto{\pgfqpoint{1.463303in}{3.303595in}}%
\pgfpathlineto{\pgfqpoint{1.467718in}{3.297996in}}%
\pgfpathlineto{\pgfqpoint{1.472132in}{3.324840in}}%
\pgfpathlineto{\pgfqpoint{1.476546in}{3.325637in}}%
\pgfpathlineto{\pgfqpoint{1.480961in}{3.223899in}}%
\pgfpathlineto{\pgfqpoint{1.485375in}{3.287426in}}%
\pgfpathlineto{\pgfqpoint{1.489790in}{3.195229in}}%
\pgfpathlineto{\pgfqpoint{1.494204in}{3.303345in}}%
\pgfpathlineto{\pgfqpoint{1.498619in}{3.372029in}}%
\pgfpathlineto{\pgfqpoint{1.503033in}{3.317536in}}%
\pgfpathlineto{\pgfqpoint{1.507448in}{3.325051in}}%
\pgfpathlineto{\pgfqpoint{1.511862in}{3.153087in}}%
\pgfpathlineto{\pgfqpoint{1.516276in}{3.269706in}}%
\pgfpathlineto{\pgfqpoint{1.520691in}{3.223055in}}%
\pgfpathlineto{\pgfqpoint{1.525105in}{3.257501in}}%
\pgfpathlineto{\pgfqpoint{1.529520in}{3.311383in}}%
\pgfpathlineto{\pgfqpoint{1.533934in}{3.279157in}}%
\pgfpathlineto{\pgfqpoint{1.538349in}{3.228319in}}%
\pgfpathlineto{\pgfqpoint{1.542763in}{3.251500in}}%
\pgfpathlineto{\pgfqpoint{1.547178in}{3.317061in}}%
\pgfpathlineto{\pgfqpoint{1.551592in}{3.330451in}}%
\pgfpathlineto{\pgfqpoint{1.556006in}{3.348491in}}%
\pgfpathlineto{\pgfqpoint{1.560421in}{3.340174in}}%
\pgfpathlineto{\pgfqpoint{1.564835in}{3.343722in}}%
\pgfpathlineto{\pgfqpoint{1.569250in}{3.244446in}}%
\pgfpathlineto{\pgfqpoint{1.578079in}{3.297166in}}%
\pgfpathlineto{\pgfqpoint{1.582493in}{3.264121in}}%
\pgfpathlineto{\pgfqpoint{1.586908in}{3.300312in}}%
\pgfpathlineto{\pgfqpoint{1.591322in}{3.255498in}}%
\pgfpathlineto{\pgfqpoint{1.595737in}{3.299569in}}%
\pgfpathlineto{\pgfqpoint{1.600151in}{3.261654in}}%
\pgfpathlineto{\pgfqpoint{1.604565in}{3.332761in}}%
\pgfpathlineto{\pgfqpoint{1.608980in}{3.354028in}}%
\pgfpathlineto{\pgfqpoint{1.613394in}{3.341525in}}%
\pgfpathlineto{\pgfqpoint{1.617809in}{3.348395in}}%
\pgfpathlineto{\pgfqpoint{1.622223in}{3.319545in}}%
\pgfpathlineto{\pgfqpoint{1.626638in}{3.235018in}}%
\pgfpathlineto{\pgfqpoint{1.631052in}{3.222368in}}%
\pgfpathlineto{\pgfqpoint{1.635467in}{3.300832in}}%
\pgfpathlineto{\pgfqpoint{1.639881in}{3.308758in}}%
\pgfpathlineto{\pgfqpoint{1.644295in}{3.221223in}}%
\pgfpathlineto{\pgfqpoint{1.648710in}{3.205917in}}%
\pgfpathlineto{\pgfqpoint{1.653124in}{3.293424in}}%
\pgfpathlineto{\pgfqpoint{1.657539in}{3.325344in}}%
\pgfpathlineto{\pgfqpoint{1.661953in}{3.332015in}}%
\pgfpathlineto{\pgfqpoint{1.666368in}{3.365775in}}%
\pgfpathlineto{\pgfqpoint{1.670782in}{3.343477in}}%
\pgfpathlineto{\pgfqpoint{1.675197in}{3.334339in}}%
\pgfpathlineto{\pgfqpoint{1.679611in}{3.257653in}}%
\pgfpathlineto{\pgfqpoint{1.684025in}{3.272745in}}%
\pgfpathlineto{\pgfqpoint{1.688440in}{3.322390in}}%
\pgfpathlineto{\pgfqpoint{1.692854in}{3.218443in}}%
\pgfpathlineto{\pgfqpoint{1.697269in}{3.286869in}}%
\pgfpathlineto{\pgfqpoint{1.701683in}{3.240029in}}%
\pgfpathlineto{\pgfqpoint{1.706098in}{3.300211in}}%
\pgfpathlineto{\pgfqpoint{1.710512in}{3.294544in}}%
\pgfpathlineto{\pgfqpoint{1.714927in}{3.357030in}}%
\pgfpathlineto{\pgfqpoint{1.719341in}{3.351938in}}%
\pgfpathlineto{\pgfqpoint{1.723755in}{3.367066in}}%
\pgfpathlineto{\pgfqpoint{1.728170in}{3.339105in}}%
\pgfpathlineto{\pgfqpoint{1.732584in}{3.366692in}}%
\pgfpathlineto{\pgfqpoint{1.736999in}{3.295363in}}%
\pgfpathlineto{\pgfqpoint{1.741413in}{3.286556in}}%
\pgfpathlineto{\pgfqpoint{1.745828in}{3.309774in}}%
\pgfpathlineto{\pgfqpoint{1.750242in}{3.276633in}}%
\pgfpathlineto{\pgfqpoint{1.754657in}{3.266369in}}%
\pgfpathlineto{\pgfqpoint{1.759071in}{3.233198in}}%
\pgfpathlineto{\pgfqpoint{1.763485in}{3.297259in}}%
\pgfpathlineto{\pgfqpoint{1.767900in}{3.344116in}}%
\pgfpathlineto{\pgfqpoint{1.772314in}{3.354746in}}%
\pgfpathlineto{\pgfqpoint{1.776729in}{3.356819in}}%
\pgfpathlineto{\pgfqpoint{1.781143in}{3.328121in}}%
\pgfpathlineto{\pgfqpoint{1.785558in}{3.320806in}}%
\pgfpathlineto{\pgfqpoint{1.789972in}{3.276554in}}%
\pgfpathlineto{\pgfqpoint{1.794387in}{3.255574in}}%
\pgfpathlineto{\pgfqpoint{1.798801in}{3.296854in}}%
\pgfpathlineto{\pgfqpoint{1.803215in}{3.317781in}}%
\pgfpathlineto{\pgfqpoint{1.807630in}{3.326261in}}%
\pgfpathlineto{\pgfqpoint{1.812044in}{3.238200in}}%
\pgfpathlineto{\pgfqpoint{1.816459in}{3.349293in}}%
\pgfpathlineto{\pgfqpoint{1.820873in}{3.290743in}}%
\pgfpathlineto{\pgfqpoint{1.825288in}{3.342650in}}%
\pgfpathlineto{\pgfqpoint{1.829702in}{3.362936in}}%
\pgfpathlineto{\pgfqpoint{1.834117in}{3.356327in}}%
\pgfpathlineto{\pgfqpoint{1.838531in}{3.339913in}}%
\pgfpathlineto{\pgfqpoint{1.842945in}{3.337850in}}%
\pgfpathlineto{\pgfqpoint{1.847360in}{3.320623in}}%
\pgfpathlineto{\pgfqpoint{1.851774in}{3.273960in}}%
\pgfpathlineto{\pgfqpoint{1.856189in}{3.309017in}}%
\pgfpathlineto{\pgfqpoint{1.860603in}{3.285394in}}%
\pgfpathlineto{\pgfqpoint{1.865018in}{3.328028in}}%
\pgfpathlineto{\pgfqpoint{1.869432in}{3.318797in}}%
\pgfpathlineto{\pgfqpoint{1.873847in}{3.350781in}}%
\pgfpathlineto{\pgfqpoint{1.878261in}{3.357545in}}%
\pgfpathlineto{\pgfqpoint{1.882675in}{3.346266in}}%
\pgfpathlineto{\pgfqpoint{1.887090in}{3.353308in}}%
\pgfpathlineto{\pgfqpoint{1.891504in}{3.334291in}}%
\pgfpathlineto{\pgfqpoint{1.895919in}{3.289449in}}%
\pgfpathlineto{\pgfqpoint{1.900333in}{3.278091in}}%
\pgfpathlineto{\pgfqpoint{1.904748in}{3.229126in}}%
\pgfpathlineto{\pgfqpoint{1.909162in}{3.278363in}}%
\pgfpathlineto{\pgfqpoint{1.913577in}{3.295771in}}%
\pgfpathlineto{\pgfqpoint{1.917991in}{3.323619in}}%
\pgfpathlineto{\pgfqpoint{1.922405in}{3.266682in}}%
\pgfpathlineto{\pgfqpoint{1.926820in}{3.362877in}}%
\pgfpathlineto{\pgfqpoint{1.931234in}{3.335225in}}%
\pgfpathlineto{\pgfqpoint{1.935649in}{3.327730in}}%
\pgfpathlineto{\pgfqpoint{1.940063in}{3.352953in}}%
\pgfpathlineto{\pgfqpoint{1.948892in}{3.351091in}}%
\pgfpathlineto{\pgfqpoint{1.953307in}{3.343441in}}%
\pgfpathlineto{\pgfqpoint{1.957721in}{3.311124in}}%
\pgfpathlineto{\pgfqpoint{1.962135in}{3.299825in}}%
\pgfpathlineto{\pgfqpoint{1.966550in}{3.299780in}}%
\pgfpathlineto{\pgfqpoint{1.975379in}{3.323785in}}%
\pgfpathlineto{\pgfqpoint{1.979793in}{3.320266in}}%
\pgfpathlineto{\pgfqpoint{1.984208in}{3.330482in}}%
\pgfpathlineto{\pgfqpoint{1.988622in}{3.336733in}}%
\pgfpathlineto{\pgfqpoint{1.993037in}{3.359906in}}%
\pgfpathlineto{\pgfqpoint{1.997451in}{3.363873in}}%
\pgfpathlineto{\pgfqpoint{2.001865in}{3.320595in}}%
\pgfpathlineto{\pgfqpoint{2.006280in}{3.344136in}}%
\pgfpathlineto{\pgfqpoint{2.010694in}{3.340574in}}%
\pgfpathlineto{\pgfqpoint{2.015109in}{3.285952in}}%
\pgfpathlineto{\pgfqpoint{2.019523in}{3.359571in}}%
\pgfpathlineto{\pgfqpoint{2.023938in}{3.267196in}}%
\pgfpathlineto{\pgfqpoint{2.028352in}{3.230049in}}%
\pgfpathlineto{\pgfqpoint{2.032767in}{3.320055in}}%
\pgfpathlineto{\pgfqpoint{2.037181in}{3.330867in}}%
\pgfpathlineto{\pgfqpoint{2.041595in}{3.326191in}}%
\pgfpathlineto{\pgfqpoint{2.046010in}{3.346291in}}%
\pgfpathlineto{\pgfqpoint{2.050424in}{3.355969in}}%
\pgfpathlineto{\pgfqpoint{2.054839in}{3.347416in}}%
\pgfpathlineto{\pgfqpoint{2.059253in}{3.349622in}}%
\pgfpathlineto{\pgfqpoint{2.063668in}{3.321768in}}%
\pgfpathlineto{\pgfqpoint{2.068082in}{3.331247in}}%
\pgfpathlineto{\pgfqpoint{2.072497in}{3.293092in}}%
\pgfpathlineto{\pgfqpoint{2.076911in}{3.319464in}}%
\pgfpathlineto{\pgfqpoint{2.081325in}{3.316037in}}%
\pgfpathlineto{\pgfqpoint{2.085740in}{3.310598in}}%
\pgfpathlineto{\pgfqpoint{2.090154in}{3.269625in}}%
\pgfpathlineto{\pgfqpoint{2.094569in}{3.339733in}}%
\pgfpathlineto{\pgfqpoint{2.098983in}{3.331410in}}%
\pgfpathlineto{\pgfqpoint{2.103398in}{3.366112in}}%
\pgfpathlineto{\pgfqpoint{2.107812in}{3.362553in}}%
\pgfpathlineto{\pgfqpoint{2.112227in}{3.341046in}}%
\pgfpathlineto{\pgfqpoint{2.116641in}{3.345210in}}%
\pgfpathlineto{\pgfqpoint{2.121055in}{3.339606in}}%
\pgfpathlineto{\pgfqpoint{2.125470in}{3.318684in}}%
\pgfpathlineto{\pgfqpoint{2.129884in}{3.314869in}}%
\pgfpathlineto{\pgfqpoint{2.134299in}{3.338869in}}%
\pgfpathlineto{\pgfqpoint{2.138713in}{3.221201in}}%
\pgfpathlineto{\pgfqpoint{2.143128in}{3.265868in}}%
\pgfpathlineto{\pgfqpoint{2.147542in}{3.283720in}}%
\pgfpathlineto{\pgfqpoint{2.151957in}{3.333216in}}%
\pgfpathlineto{\pgfqpoint{2.156371in}{3.329618in}}%
\pgfpathlineto{\pgfqpoint{2.160785in}{3.362981in}}%
\pgfpathlineto{\pgfqpoint{2.165200in}{3.354363in}}%
\pgfpathlineto{\pgfqpoint{2.169614in}{3.361802in}}%
\pgfpathlineto{\pgfqpoint{2.174029in}{3.349760in}}%
\pgfpathlineto{\pgfqpoint{2.178443in}{3.341356in}}%
\pgfpathlineto{\pgfqpoint{2.182858in}{3.321625in}}%
\pgfpathlineto{\pgfqpoint{2.187272in}{3.367694in}}%
\pgfpathlineto{\pgfqpoint{2.191687in}{3.280040in}}%
\pgfpathlineto{\pgfqpoint{2.196101in}{3.319976in}}%
\pgfpathlineto{\pgfqpoint{2.200515in}{3.311175in}}%
\pgfpathlineto{\pgfqpoint{2.204930in}{3.336103in}}%
\pgfpathlineto{\pgfqpoint{2.209344in}{3.324691in}}%
\pgfpathlineto{\pgfqpoint{2.213759in}{3.327617in}}%
\pgfpathlineto{\pgfqpoint{2.218173in}{3.365212in}}%
\pgfpathlineto{\pgfqpoint{2.222588in}{3.346007in}}%
\pgfpathlineto{\pgfqpoint{2.227002in}{3.316296in}}%
\pgfpathlineto{\pgfqpoint{2.231417in}{3.345272in}}%
\pgfpathlineto{\pgfqpoint{2.235831in}{3.305157in}}%
\pgfpathlineto{\pgfqpoint{2.240246in}{3.337943in}}%
\pgfpathlineto{\pgfqpoint{2.244660in}{3.249131in}}%
\pgfpathlineto{\pgfqpoint{2.249074in}{3.299285in}}%
\pgfpathlineto{\pgfqpoint{2.253489in}{3.262216in}}%
\pgfpathlineto{\pgfqpoint{2.257903in}{3.297864in}}%
\pgfpathlineto{\pgfqpoint{2.262318in}{3.362328in}}%
\pgfpathlineto{\pgfqpoint{2.266732in}{3.318825in}}%
\pgfpathlineto{\pgfqpoint{2.271147in}{3.361085in}}%
\pgfpathlineto{\pgfqpoint{2.275561in}{3.372769in}}%
\pgfpathlineto{\pgfqpoint{2.279976in}{3.361622in}}%
\pgfpathlineto{\pgfqpoint{2.284390in}{3.376210in}}%
\pgfpathlineto{\pgfqpoint{2.288804in}{3.344611in}}%
\pgfpathlineto{\pgfqpoint{2.293219in}{3.294021in}}%
\pgfpathlineto{\pgfqpoint{2.297633in}{3.314042in}}%
\pgfpathlineto{\pgfqpoint{2.302048in}{3.247105in}}%
\pgfpathlineto{\pgfqpoint{2.306462in}{3.302073in}}%
\pgfpathlineto{\pgfqpoint{2.310877in}{3.309639in}}%
\pgfpathlineto{\pgfqpoint{2.315291in}{3.327820in}}%
\pgfpathlineto{\pgfqpoint{2.319706in}{3.310624in}}%
\pgfpathlineto{\pgfqpoint{2.324120in}{3.358167in}}%
\pgfpathlineto{\pgfqpoint{2.328534in}{3.358569in}}%
\pgfpathlineto{\pgfqpoint{2.332949in}{3.315072in}}%
\pgfpathlineto{\pgfqpoint{2.337363in}{3.344960in}}%
\pgfpathlineto{\pgfqpoint{2.341778in}{3.304110in}}%
\pgfpathlineto{\pgfqpoint{2.346192in}{3.294733in}}%
\pgfpathlineto{\pgfqpoint{2.350607in}{3.321957in}}%
\pgfpathlineto{\pgfqpoint{2.355021in}{3.314973in}}%
\pgfpathlineto{\pgfqpoint{2.359436in}{3.320147in}}%
\pgfpathlineto{\pgfqpoint{2.363850in}{3.303480in}}%
\pgfpathlineto{\pgfqpoint{2.368264in}{3.335408in}}%
\pgfpathlineto{\pgfqpoint{2.372679in}{3.352889in}}%
\pgfpathlineto{\pgfqpoint{2.377093in}{3.344251in}}%
\pgfpathlineto{\pgfqpoint{2.385922in}{3.357337in}}%
\pgfpathlineto{\pgfqpoint{2.390337in}{3.350171in}}%
\pgfpathlineto{\pgfqpoint{2.394751in}{3.368585in}}%
\pgfpathlineto{\pgfqpoint{2.399166in}{3.335009in}}%
\pgfpathlineto{\pgfqpoint{2.403580in}{3.280420in}}%
\pgfpathlineto{\pgfqpoint{2.407994in}{3.332572in}}%
\pgfpathlineto{\pgfqpoint{2.412409in}{3.263691in}}%
\pgfpathlineto{\pgfqpoint{2.421238in}{3.296848in}}%
\pgfpathlineto{\pgfqpoint{2.425652in}{3.353795in}}%
\pgfpathlineto{\pgfqpoint{2.430067in}{3.343354in}}%
\pgfpathlineto{\pgfqpoint{2.434481in}{3.363819in}}%
\pgfpathlineto{\pgfqpoint{2.438896in}{3.352861in}}%
\pgfpathlineto{\pgfqpoint{2.443310in}{3.333593in}}%
\pgfpathlineto{\pgfqpoint{2.447724in}{3.343511in}}%
\pgfpathlineto{\pgfqpoint{2.452139in}{3.325209in}}%
\pgfpathlineto{\pgfqpoint{2.456553in}{3.288444in}}%
\pgfpathlineto{\pgfqpoint{2.460968in}{3.350058in}}%
\pgfpathlineto{\pgfqpoint{2.465382in}{3.332673in}}%
\pgfpathlineto{\pgfqpoint{2.469797in}{3.295157in}}%
\pgfpathlineto{\pgfqpoint{2.474211in}{3.287735in}}%
\pgfpathlineto{\pgfqpoint{2.478626in}{3.290982in}}%
\pgfpathlineto{\pgfqpoint{2.487454in}{3.355612in}}%
\pgfpathlineto{\pgfqpoint{2.491869in}{3.351355in}}%
\pgfpathlineto{\pgfqpoint{2.496283in}{3.344237in}}%
\pgfpathlineto{\pgfqpoint{2.505112in}{3.367823in}}%
\pgfpathlineto{\pgfqpoint{2.509527in}{3.339899in}}%
\pgfpathlineto{\pgfqpoint{2.513941in}{3.342000in}}%
\pgfpathlineto{\pgfqpoint{2.518356in}{3.357081in}}%
\pgfpathlineto{\pgfqpoint{2.522770in}{3.306158in}}%
\pgfpathlineto{\pgfqpoint{2.527184in}{3.301319in}}%
\pgfpathlineto{\pgfqpoint{2.531599in}{3.340346in}}%
\pgfpathlineto{\pgfqpoint{2.536013in}{3.359425in}}%
\pgfpathlineto{\pgfqpoint{2.544842in}{3.328279in}}%
\pgfpathlineto{\pgfqpoint{2.549257in}{3.362998in}}%
\pgfpathlineto{\pgfqpoint{2.553671in}{3.354442in}}%
\pgfpathlineto{\pgfqpoint{2.558086in}{3.229185in}}%
\pgfpathlineto{\pgfqpoint{2.562500in}{3.300641in}}%
\pgfpathlineto{\pgfqpoint{2.566914in}{3.274748in}}%
\pgfpathlineto{\pgfqpoint{2.571329in}{3.346071in}}%
\pgfpathlineto{\pgfqpoint{2.575743in}{3.328340in}}%
\pgfpathlineto{\pgfqpoint{2.580158in}{3.278347in}}%
\pgfpathlineto{\pgfqpoint{2.584572in}{3.314225in}}%
\pgfpathlineto{\pgfqpoint{2.588987in}{3.291641in}}%
\pgfpathlineto{\pgfqpoint{2.593401in}{3.364590in}}%
\pgfpathlineto{\pgfqpoint{2.597816in}{3.352951in}}%
\pgfpathlineto{\pgfqpoint{2.602230in}{3.345188in}}%
\pgfpathlineto{\pgfqpoint{2.611059in}{3.368279in}}%
\pgfpathlineto{\pgfqpoint{2.615473in}{3.370347in}}%
\pgfpathlineto{\pgfqpoint{2.624302in}{3.294342in}}%
\pgfpathlineto{\pgfqpoint{2.628717in}{3.367626in}}%
\pgfpathlineto{\pgfqpoint{2.633131in}{3.282091in}}%
\pgfpathlineto{\pgfqpoint{2.637546in}{3.338993in}}%
\pgfpathlineto{\pgfqpoint{2.641960in}{3.340005in}}%
\pgfpathlineto{\pgfqpoint{2.646374in}{3.333140in}}%
\pgfpathlineto{\pgfqpoint{2.650789in}{3.344108in}}%
\pgfpathlineto{\pgfqpoint{2.655203in}{3.309678in}}%
\pgfpathlineto{\pgfqpoint{2.659618in}{3.351935in}}%
\pgfpathlineto{\pgfqpoint{2.664032in}{3.350013in}}%
\pgfpathlineto{\pgfqpoint{2.668447in}{3.278386in}}%
\pgfpathlineto{\pgfqpoint{2.672861in}{3.320223in}}%
\pgfpathlineto{\pgfqpoint{2.677276in}{3.301958in}}%
\pgfpathlineto{\pgfqpoint{2.681690in}{3.343463in}}%
\pgfpathlineto{\pgfqpoint{2.686104in}{3.340239in}}%
\pgfpathlineto{\pgfqpoint{2.690519in}{3.303421in}}%
\pgfpathlineto{\pgfqpoint{2.694933in}{3.355508in}}%
\pgfpathlineto{\pgfqpoint{2.699348in}{3.358873in}}%
\pgfpathlineto{\pgfqpoint{2.703762in}{3.350840in}}%
\pgfpathlineto{\pgfqpoint{2.708177in}{3.328762in}}%
\pgfpathlineto{\pgfqpoint{2.712591in}{3.362936in}}%
\pgfpathlineto{\pgfqpoint{2.717006in}{3.363276in}}%
\pgfpathlineto{\pgfqpoint{2.721420in}{3.330090in}}%
\pgfpathlineto{\pgfqpoint{2.725834in}{3.362519in}}%
\pgfpathlineto{\pgfqpoint{2.730249in}{3.311960in}}%
\pgfpathlineto{\pgfqpoint{2.734663in}{3.346831in}}%
\pgfpathlineto{\pgfqpoint{2.739078in}{3.346603in}}%
\pgfpathlineto{\pgfqpoint{2.743492in}{3.290709in}}%
\pgfpathlineto{\pgfqpoint{2.747907in}{3.316996in}}%
\pgfpathlineto{\pgfqpoint{2.752321in}{3.304642in}}%
\pgfpathlineto{\pgfqpoint{2.761150in}{3.355908in}}%
\pgfpathlineto{\pgfqpoint{2.765564in}{3.342740in}}%
\pgfpathlineto{\pgfqpoint{2.769979in}{3.346879in}}%
\pgfpathlineto{\pgfqpoint{2.774393in}{3.348221in}}%
\pgfpathlineto{\pgfqpoint{2.778808in}{3.354144in}}%
\pgfpathlineto{\pgfqpoint{2.783222in}{3.332645in}}%
\pgfpathlineto{\pgfqpoint{2.787637in}{3.283943in}}%
\pgfpathlineto{\pgfqpoint{2.792051in}{3.330656in}}%
\pgfpathlineto{\pgfqpoint{2.796466in}{3.342073in}}%
\pgfpathlineto{\pgfqpoint{2.800880in}{3.301699in}}%
\pgfpathlineto{\pgfqpoint{2.805294in}{3.323949in}}%
\pgfpathlineto{\pgfqpoint{2.809709in}{3.273096in}}%
\pgfpathlineto{\pgfqpoint{2.814123in}{3.321802in}}%
\pgfpathlineto{\pgfqpoint{2.818538in}{3.321262in}}%
\pgfpathlineto{\pgfqpoint{2.822952in}{3.362117in}}%
\pgfpathlineto{\pgfqpoint{2.827367in}{3.341162in}}%
\pgfpathlineto{\pgfqpoint{2.831781in}{3.332187in}}%
\pgfpathlineto{\pgfqpoint{2.836196in}{3.364616in}}%
\pgfpathlineto{\pgfqpoint{2.840610in}{3.314768in}}%
\pgfpathlineto{\pgfqpoint{2.845024in}{3.313662in}}%
\pgfpathlineto{\pgfqpoint{2.849439in}{3.347011in}}%
\pgfpathlineto{\pgfqpoint{2.853853in}{3.331193in}}%
\pgfpathlineto{\pgfqpoint{2.858268in}{3.293393in}}%
\pgfpathlineto{\pgfqpoint{2.867097in}{3.354298in}}%
\pgfpathlineto{\pgfqpoint{2.871511in}{3.322522in}}%
\pgfpathlineto{\pgfqpoint{2.875926in}{3.360049in}}%
\pgfpathlineto{\pgfqpoint{2.880340in}{3.336708in}}%
\pgfpathlineto{\pgfqpoint{2.884754in}{3.359759in}}%
\pgfpathlineto{\pgfqpoint{2.889169in}{3.341739in}}%
\pgfpathlineto{\pgfqpoint{2.893583in}{3.282502in}}%
\pgfpathlineto{\pgfqpoint{2.897998in}{3.341843in}}%
\pgfpathlineto{\pgfqpoint{2.902412in}{3.345759in}}%
\pgfpathlineto{\pgfqpoint{2.911241in}{3.296404in}}%
\pgfpathlineto{\pgfqpoint{2.915656in}{3.324539in}}%
\pgfpathlineto{\pgfqpoint{2.920070in}{3.270784in}}%
\pgfpathlineto{\pgfqpoint{2.928899in}{3.355660in}}%
\pgfpathlineto{\pgfqpoint{2.933313in}{3.346885in}}%
\pgfpathlineto{\pgfqpoint{2.937728in}{3.359107in}}%
\pgfpathlineto{\pgfqpoint{2.942142in}{3.364967in}}%
\pgfpathlineto{\pgfqpoint{2.946557in}{3.358656in}}%
\pgfpathlineto{\pgfqpoint{2.950971in}{3.318783in}}%
\pgfpathlineto{\pgfqpoint{2.955386in}{3.340954in}}%
\pgfpathlineto{\pgfqpoint{2.959800in}{3.328295in}}%
\pgfpathlineto{\pgfqpoint{2.968629in}{3.334300in}}%
\pgfpathlineto{\pgfqpoint{2.973043in}{3.333304in}}%
\pgfpathlineto{\pgfqpoint{2.977458in}{3.351026in}}%
\pgfpathlineto{\pgfqpoint{2.981872in}{3.351817in}}%
\pgfpathlineto{\pgfqpoint{2.986287in}{3.359498in}}%
\pgfpathlineto{\pgfqpoint{2.990701in}{3.357402in}}%
\pgfpathlineto{\pgfqpoint{2.995116in}{3.359903in}}%
\pgfpathlineto{\pgfqpoint{3.003945in}{3.326126in}}%
\pgfpathlineto{\pgfqpoint{3.008359in}{3.279162in}}%
\pgfpathlineto{\pgfqpoint{3.012773in}{3.380000in}}%
\pgfpathlineto{\pgfqpoint{3.017188in}{3.297611in}}%
\pgfpathlineto{\pgfqpoint{3.021602in}{3.316439in}}%
\pgfpathlineto{\pgfqpoint{3.026017in}{3.349239in}}%
\pgfpathlineto{\pgfqpoint{3.030431in}{3.329899in}}%
\pgfpathlineto{\pgfqpoint{3.039260in}{3.361644in}}%
\pgfpathlineto{\pgfqpoint{3.043675in}{3.359689in}}%
\pgfpathlineto{\pgfqpoint{3.048089in}{3.360356in}}%
\pgfpathlineto{\pgfqpoint{3.052503in}{3.359399in}}%
\pgfpathlineto{\pgfqpoint{3.056918in}{3.351361in}}%
\pgfpathlineto{\pgfqpoint{3.061332in}{3.349763in}}%
\pgfpathlineto{\pgfqpoint{3.065747in}{3.330155in}}%
\pgfpathlineto{\pgfqpoint{3.070161in}{3.345503in}}%
\pgfpathlineto{\pgfqpoint{3.074576in}{3.323473in}}%
\pgfpathlineto{\pgfqpoint{3.078990in}{3.327384in}}%
\pgfpathlineto{\pgfqpoint{3.083405in}{3.299969in}}%
\pgfpathlineto{\pgfqpoint{3.092233in}{3.359017in}}%
\pgfpathlineto{\pgfqpoint{3.096648in}{3.357615in}}%
\pgfpathlineto{\pgfqpoint{3.101062in}{3.351701in}}%
\pgfpathlineto{\pgfqpoint{3.105477in}{3.362083in}}%
\pgfpathlineto{\pgfqpoint{3.109891in}{3.327254in}}%
\pgfpathlineto{\pgfqpoint{3.114306in}{3.337431in}}%
\pgfpathlineto{\pgfqpoint{3.118720in}{3.288411in}}%
\pgfpathlineto{\pgfqpoint{3.123135in}{3.327212in}}%
\pgfpathlineto{\pgfqpoint{3.127549in}{3.332600in}}%
\pgfpathlineto{\pgfqpoint{3.131963in}{3.256229in}}%
\pgfpathlineto{\pgfqpoint{3.136378in}{3.329705in}}%
\pgfpathlineto{\pgfqpoint{3.140792in}{3.325606in}}%
\pgfpathlineto{\pgfqpoint{3.145207in}{3.277972in}}%
\pgfpathlineto{\pgfqpoint{3.149621in}{3.352239in}}%
\pgfpathlineto{\pgfqpoint{3.154036in}{3.368633in}}%
\pgfpathlineto{\pgfqpoint{3.158450in}{3.362643in}}%
\pgfpathlineto{\pgfqpoint{3.162865in}{3.334809in}}%
\pgfpathlineto{\pgfqpoint{3.167279in}{3.359757in}}%
\pgfpathlineto{\pgfqpoint{3.171693in}{3.318189in}}%
\pgfpathlineto{\pgfqpoint{3.176108in}{3.311662in}}%
\pgfpathlineto{\pgfqpoint{3.180522in}{3.339339in}}%
\pgfpathlineto{\pgfqpoint{3.189351in}{3.314149in}}%
\pgfpathlineto{\pgfqpoint{3.193766in}{3.302290in}}%
\pgfpathlineto{\pgfqpoint{3.202595in}{3.337265in}}%
\pgfpathlineto{\pgfqpoint{3.207009in}{3.329004in}}%
\pgfpathlineto{\pgfqpoint{3.215838in}{3.358240in}}%
\pgfpathlineto{\pgfqpoint{3.220252in}{3.355185in}}%
\pgfpathlineto{\pgfqpoint{3.224667in}{3.342898in}}%
\pgfpathlineto{\pgfqpoint{3.229081in}{3.279787in}}%
\pgfpathlineto{\pgfqpoint{3.237910in}{3.349321in}}%
\pgfpathlineto{\pgfqpoint{3.242325in}{3.273012in}}%
\pgfpathlineto{\pgfqpoint{3.246739in}{3.327733in}}%
\pgfpathlineto{\pgfqpoint{3.251153in}{3.316481in}}%
\pgfpathlineto{\pgfqpoint{3.255568in}{3.254924in}}%
\pgfpathlineto{\pgfqpoint{3.259982in}{3.360195in}}%
\pgfpathlineto{\pgfqpoint{3.264397in}{3.365074in}}%
\pgfpathlineto{\pgfqpoint{3.268811in}{3.352531in}}%
\pgfpathlineto{\pgfqpoint{3.273226in}{3.351957in}}%
\pgfpathlineto{\pgfqpoint{3.277640in}{3.353350in}}%
\pgfpathlineto{\pgfqpoint{3.282055in}{3.306409in}}%
\pgfpathlineto{\pgfqpoint{3.286469in}{3.351285in}}%
\pgfpathlineto{\pgfqpoint{3.290883in}{3.327691in}}%
\pgfpathlineto{\pgfqpoint{3.295298in}{3.325184in}}%
\pgfpathlineto{\pgfqpoint{3.299712in}{3.335855in}}%
\pgfpathlineto{\pgfqpoint{3.304127in}{3.313747in}}%
\pgfpathlineto{\pgfqpoint{3.308541in}{3.309166in}}%
\pgfpathlineto{\pgfqpoint{3.312956in}{3.358921in}}%
\pgfpathlineto{\pgfqpoint{3.317370in}{3.351583in}}%
\pgfpathlineto{\pgfqpoint{3.321785in}{3.368403in}}%
\pgfpathlineto{\pgfqpoint{3.326199in}{3.356093in}}%
\pgfpathlineto{\pgfqpoint{3.330613in}{3.378000in}}%
\pgfpathlineto{\pgfqpoint{3.335028in}{3.299876in}}%
\pgfpathlineto{\pgfqpoint{3.339442in}{3.277311in}}%
\pgfpathlineto{\pgfqpoint{3.343857in}{3.372713in}}%
\pgfpathlineto{\pgfqpoint{3.348271in}{3.340714in}}%
\pgfpathlineto{\pgfqpoint{3.352686in}{3.335352in}}%
\pgfpathlineto{\pgfqpoint{3.357100in}{3.342104in}}%
\pgfpathlineto{\pgfqpoint{3.361515in}{3.330527in}}%
\pgfpathlineto{\pgfqpoint{3.365929in}{3.332310in}}%
\pgfpathlineto{\pgfqpoint{3.370343in}{3.362092in}}%
\pgfpathlineto{\pgfqpoint{3.379172in}{3.365499in}}%
\pgfpathlineto{\pgfqpoint{3.383587in}{3.365097in}}%
\pgfpathlineto{\pgfqpoint{3.388001in}{3.359380in}}%
\pgfpathlineto{\pgfqpoint{3.392416in}{3.324939in}}%
\pgfpathlineto{\pgfqpoint{3.396830in}{3.305390in}}%
\pgfpathlineto{\pgfqpoint{3.401245in}{3.316059in}}%
\pgfpathlineto{\pgfqpoint{3.405659in}{3.309259in}}%
\pgfpathlineto{\pgfqpoint{3.410073in}{3.331489in}}%
\pgfpathlineto{\pgfqpoint{3.414488in}{3.297656in}}%
\pgfpathlineto{\pgfqpoint{3.418902in}{3.302506in}}%
\pgfpathlineto{\pgfqpoint{3.423317in}{3.368569in}}%
\pgfpathlineto{\pgfqpoint{3.427731in}{3.367525in}}%
\pgfpathlineto{\pgfqpoint{3.432146in}{3.370969in}}%
\pgfpathlineto{\pgfqpoint{3.436560in}{3.354397in}}%
\pgfpathlineto{\pgfqpoint{3.440975in}{3.369928in}}%
\pgfpathlineto{\pgfqpoint{3.445389in}{3.325234in}}%
\pgfpathlineto{\pgfqpoint{3.449803in}{3.311966in}}%
\pgfpathlineto{\pgfqpoint{3.454218in}{3.344009in}}%
\pgfpathlineto{\pgfqpoint{3.458632in}{3.356760in}}%
\pgfpathlineto{\pgfqpoint{3.463047in}{3.290484in}}%
\pgfpathlineto{\pgfqpoint{3.467461in}{3.346862in}}%
\pgfpathlineto{\pgfqpoint{3.471876in}{3.338911in}}%
\pgfpathlineto{\pgfqpoint{3.476290in}{3.289165in}}%
\pgfpathlineto{\pgfqpoint{3.480705in}{3.352250in}}%
\pgfpathlineto{\pgfqpoint{3.485119in}{3.365099in}}%
\pgfpathlineto{\pgfqpoint{3.489533in}{3.371399in}}%
\pgfpathlineto{\pgfqpoint{3.493948in}{3.368870in}}%
\pgfpathlineto{\pgfqpoint{3.498362in}{3.350503in}}%
\pgfpathlineto{\pgfqpoint{3.507191in}{3.242223in}}%
\pgfpathlineto{\pgfqpoint{3.511606in}{3.358637in}}%
\pgfpathlineto{\pgfqpoint{3.516020in}{3.311676in}}%
\pgfpathlineto{\pgfqpoint{3.520435in}{3.335366in}}%
\pgfpathlineto{\pgfqpoint{3.524849in}{3.302090in}}%
\pgfpathlineto{\pgfqpoint{3.529263in}{3.287685in}}%
\pgfpathlineto{\pgfqpoint{3.533678in}{3.362733in}}%
\pgfpathlineto{\pgfqpoint{3.538092in}{3.359219in}}%
\pgfpathlineto{\pgfqpoint{3.542507in}{3.365491in}}%
\pgfpathlineto{\pgfqpoint{3.546921in}{3.357599in}}%
\pgfpathlineto{\pgfqpoint{3.551336in}{3.366782in}}%
\pgfpathlineto{\pgfqpoint{3.555750in}{3.287921in}}%
\pgfpathlineto{\pgfqpoint{3.560165in}{3.342358in}}%
\pgfpathlineto{\pgfqpoint{3.564579in}{3.361411in}}%
\pgfpathlineto{\pgfqpoint{3.568994in}{3.335411in}}%
\pgfpathlineto{\pgfqpoint{3.573408in}{3.279742in}}%
\pgfpathlineto{\pgfqpoint{3.577822in}{3.332763in}}%
\pgfpathlineto{\pgfqpoint{3.582237in}{3.327024in}}%
\pgfpathlineto{\pgfqpoint{3.586651in}{3.290152in}}%
\pgfpathlineto{\pgfqpoint{3.591066in}{3.362252in}}%
\pgfpathlineto{\pgfqpoint{3.599895in}{3.366807in}}%
\pgfpathlineto{\pgfqpoint{3.608724in}{3.333484in}}%
\pgfpathlineto{\pgfqpoint{3.613138in}{3.335636in}}%
\pgfpathlineto{\pgfqpoint{3.617552in}{3.286073in}}%
\pgfpathlineto{\pgfqpoint{3.621967in}{3.283639in}}%
\pgfpathlineto{\pgfqpoint{3.626381in}{3.353539in}}%
\pgfpathlineto{\pgfqpoint{3.630796in}{3.367905in}}%
\pgfpathlineto{\pgfqpoint{3.635210in}{3.344715in}}%
\pgfpathlineto{\pgfqpoint{3.639625in}{3.333152in}}%
\pgfpathlineto{\pgfqpoint{3.644039in}{3.365657in}}%
\pgfpathlineto{\pgfqpoint{3.648454in}{3.359287in}}%
\pgfpathlineto{\pgfqpoint{3.652868in}{3.360406in}}%
\pgfpathlineto{\pgfqpoint{3.657282in}{3.357331in}}%
\pgfpathlineto{\pgfqpoint{3.661697in}{3.367311in}}%
\pgfpathlineto{\pgfqpoint{3.666111in}{3.318032in}}%
\pgfpathlineto{\pgfqpoint{3.670526in}{3.330645in}}%
\pgfpathlineto{\pgfqpoint{3.674940in}{3.356138in}}%
\pgfpathlineto{\pgfqpoint{3.679355in}{3.354962in}}%
\pgfpathlineto{\pgfqpoint{3.683769in}{3.318946in}}%
\pgfpathlineto{\pgfqpoint{3.688184in}{3.337476in}}%
\pgfpathlineto{\pgfqpoint{3.692598in}{3.346766in}}%
\pgfpathlineto{\pgfqpoint{3.697012in}{3.342217in}}%
\pgfpathlineto{\pgfqpoint{3.701427in}{3.362925in}}%
\pgfpathlineto{\pgfqpoint{3.705841in}{3.345436in}}%
\pgfpathlineto{\pgfqpoint{3.710256in}{3.360713in}}%
\pgfpathlineto{\pgfqpoint{3.714670in}{3.348736in}}%
\pgfpathlineto{\pgfqpoint{3.719085in}{3.347622in}}%
\pgfpathlineto{\pgfqpoint{3.723499in}{3.352126in}}%
\pgfpathlineto{\pgfqpoint{3.727914in}{3.314855in}}%
\pgfpathlineto{\pgfqpoint{3.732328in}{3.267678in}}%
\pgfpathlineto{\pgfqpoint{3.736742in}{3.286337in}}%
\pgfpathlineto{\pgfqpoint{3.741157in}{3.344535in}}%
\pgfpathlineto{\pgfqpoint{3.745571in}{3.328343in}}%
\pgfpathlineto{\pgfqpoint{3.749986in}{3.299935in}}%
\pgfpathlineto{\pgfqpoint{3.754400in}{3.358338in}}%
\pgfpathlineto{\pgfqpoint{3.758815in}{3.361518in}}%
\pgfpathlineto{\pgfqpoint{3.767644in}{3.350652in}}%
\pgfpathlineto{\pgfqpoint{3.772058in}{3.358395in}}%
\pgfpathlineto{\pgfqpoint{3.776472in}{3.308786in}}%
\pgfpathlineto{\pgfqpoint{3.780887in}{3.287808in}}%
\pgfpathlineto{\pgfqpoint{3.785301in}{3.337285in}}%
\pgfpathlineto{\pgfqpoint{3.789716in}{3.346471in}}%
\pgfpathlineto{\pgfqpoint{3.794130in}{3.334516in}}%
\pgfpathlineto{\pgfqpoint{3.798545in}{3.354613in}}%
\pgfpathlineto{\pgfqpoint{3.802959in}{3.345928in}}%
\pgfpathlineto{\pgfqpoint{3.807374in}{3.319602in}}%
\pgfpathlineto{\pgfqpoint{3.811788in}{3.353713in}}%
\pgfpathlineto{\pgfqpoint{3.816202in}{3.347425in}}%
\pgfpathlineto{\pgfqpoint{3.820617in}{3.362176in}}%
\pgfpathlineto{\pgfqpoint{3.825031in}{3.360007in}}%
\pgfpathlineto{\pgfqpoint{3.829446in}{3.348356in}}%
\pgfpathlineto{\pgfqpoint{3.833860in}{3.332187in}}%
\pgfpathlineto{\pgfqpoint{3.838275in}{3.272556in}}%
\pgfpathlineto{\pgfqpoint{3.842689in}{3.324112in}}%
\pgfpathlineto{\pgfqpoint{3.847104in}{3.267987in}}%
\pgfpathlineto{\pgfqpoint{3.851518in}{3.316647in}}%
\pgfpathlineto{\pgfqpoint{3.855932in}{3.306305in}}%
\pgfpathlineto{\pgfqpoint{3.860347in}{3.301713in}}%
\pgfpathlineto{\pgfqpoint{3.864761in}{3.339080in}}%
\pgfpathlineto{\pgfqpoint{3.878005in}{3.350438in}}%
\pgfpathlineto{\pgfqpoint{3.882419in}{3.362795in}}%
\pgfpathlineto{\pgfqpoint{3.886834in}{3.303258in}}%
\pgfpathlineto{\pgfqpoint{3.891248in}{3.353114in}}%
\pgfpathlineto{\pgfqpoint{3.895662in}{3.366720in}}%
\pgfpathlineto{\pgfqpoint{3.900077in}{3.327139in}}%
\pgfpathlineto{\pgfqpoint{3.904491in}{3.298281in}}%
\pgfpathlineto{\pgfqpoint{3.913320in}{3.318963in}}%
\pgfpathlineto{\pgfqpoint{3.917735in}{3.325763in}}%
\pgfpathlineto{\pgfqpoint{3.922149in}{3.359537in}}%
\pgfpathlineto{\pgfqpoint{3.926564in}{3.357942in}}%
\pgfpathlineto{\pgfqpoint{3.930978in}{3.355100in}}%
\pgfpathlineto{\pgfqpoint{3.935392in}{3.372994in}}%
\pgfpathlineto{\pgfqpoint{3.939807in}{3.322528in}}%
\pgfpathlineto{\pgfqpoint{3.944221in}{3.316602in}}%
\pgfpathlineto{\pgfqpoint{3.948636in}{3.287159in}}%
\pgfpathlineto{\pgfqpoint{3.953050in}{3.353440in}}%
\pgfpathlineto{\pgfqpoint{3.957465in}{3.325049in}}%
\pgfpathlineto{\pgfqpoint{3.966294in}{3.343356in}}%
\pgfpathlineto{\pgfqpoint{3.970708in}{3.338109in}}%
\pgfpathlineto{\pgfqpoint{3.975122in}{3.355758in}}%
\pgfpathlineto{\pgfqpoint{3.979537in}{3.334497in}}%
\pgfpathlineto{\pgfqpoint{3.983951in}{3.361979in}}%
\pgfpathlineto{\pgfqpoint{3.988366in}{3.356667in}}%
\pgfpathlineto{\pgfqpoint{3.992780in}{3.360401in}}%
\pgfpathlineto{\pgfqpoint{4.001609in}{3.310694in}}%
\pgfpathlineto{\pgfqpoint{4.006024in}{3.356608in}}%
\pgfpathlineto{\pgfqpoint{4.010438in}{3.350888in}}%
\pgfpathlineto{\pgfqpoint{4.014852in}{3.340470in}}%
\pgfpathlineto{\pgfqpoint{4.019267in}{3.296696in}}%
\pgfpathlineto{\pgfqpoint{4.028096in}{3.362756in}}%
\pgfpathlineto{\pgfqpoint{4.032510in}{3.365153in}}%
\pgfpathlineto{\pgfqpoint{4.036925in}{3.352427in}}%
\pgfpathlineto{\pgfqpoint{4.041339in}{3.354805in}}%
\pgfpathlineto{\pgfqpoint{4.045754in}{3.362297in}}%
\pgfpathlineto{\pgfqpoint{4.050168in}{3.335436in}}%
\pgfpathlineto{\pgfqpoint{4.054582in}{3.318659in}}%
\pgfpathlineto{\pgfqpoint{4.058997in}{3.308339in}}%
\pgfpathlineto{\pgfqpoint{4.063411in}{3.353148in}}%
\pgfpathlineto{\pgfqpoint{4.067826in}{3.284069in}}%
\pgfpathlineto{\pgfqpoint{4.072240in}{3.335042in}}%
\pgfpathlineto{\pgfqpoint{4.076655in}{3.349788in}}%
\pgfpathlineto{\pgfqpoint{4.081069in}{3.324500in}}%
\pgfpathlineto{\pgfqpoint{4.085484in}{3.373709in}}%
\pgfpathlineto{\pgfqpoint{4.089898in}{3.339462in}}%
\pgfpathlineto{\pgfqpoint{4.094312in}{3.365873in}}%
\pgfpathlineto{\pgfqpoint{4.098727in}{3.364376in}}%
\pgfpathlineto{\pgfqpoint{4.107556in}{3.343143in}}%
\pgfpathlineto{\pgfqpoint{4.111970in}{3.294122in}}%
\pgfpathlineto{\pgfqpoint{4.116385in}{3.350075in}}%
\pgfpathlineto{\pgfqpoint{4.120799in}{3.338087in}}%
\pgfpathlineto{\pgfqpoint{4.125214in}{3.340914in}}%
\pgfpathlineto{\pgfqpoint{4.129628in}{3.349104in}}%
\pgfpathlineto{\pgfqpoint{4.134042in}{3.354943in}}%
\pgfpathlineto{\pgfqpoint{4.138457in}{3.353961in}}%
\pgfpathlineto{\pgfqpoint{4.142871in}{3.364495in}}%
\pgfpathlineto{\pgfqpoint{4.147286in}{3.370018in}}%
\pgfpathlineto{\pgfqpoint{4.151700in}{3.350078in}}%
\pgfpathlineto{\pgfqpoint{4.156115in}{3.358735in}}%
\pgfpathlineto{\pgfqpoint{4.160529in}{3.288630in}}%
\pgfpathlineto{\pgfqpoint{4.164944in}{3.347470in}}%
\pgfpathlineto{\pgfqpoint{4.169358in}{3.359008in}}%
\pgfpathlineto{\pgfqpoint{4.173772in}{3.346339in}}%
\pgfpathlineto{\pgfqpoint{4.178187in}{3.293028in}}%
\pgfpathlineto{\pgfqpoint{4.182601in}{3.324421in}}%
\pgfpathlineto{\pgfqpoint{4.187016in}{3.327209in}}%
\pgfpathlineto{\pgfqpoint{4.191430in}{3.312298in}}%
\pgfpathlineto{\pgfqpoint{4.195845in}{3.369936in}}%
\pgfpathlineto{\pgfqpoint{4.200259in}{3.347000in}}%
\pgfpathlineto{\pgfqpoint{4.204674in}{3.346513in}}%
\pgfpathlineto{\pgfqpoint{4.209088in}{3.344603in}}%
\pgfpathlineto{\pgfqpoint{4.213503in}{3.359022in}}%
\pgfpathlineto{\pgfqpoint{4.222331in}{3.311839in}}%
\pgfpathlineto{\pgfqpoint{4.226746in}{3.346944in}}%
\pgfpathlineto{\pgfqpoint{4.231160in}{3.334392in}}%
\pgfpathlineto{\pgfqpoint{4.235575in}{3.303379in}}%
\pgfpathlineto{\pgfqpoint{4.239989in}{3.333757in}}%
\pgfpathlineto{\pgfqpoint{4.244404in}{3.347222in}}%
\pgfpathlineto{\pgfqpoint{4.248818in}{3.317930in}}%
\pgfpathlineto{\pgfqpoint{4.253233in}{3.368532in}}%
\pgfpathlineto{\pgfqpoint{4.257647in}{3.363237in}}%
\pgfpathlineto{\pgfqpoint{4.262061in}{3.352745in}}%
\pgfpathlineto{\pgfqpoint{4.266476in}{3.361591in}}%
\pgfpathlineto{\pgfqpoint{4.270890in}{3.181586in}}%
\pgfpathlineto{\pgfqpoint{4.275305in}{3.337096in}}%
\pgfpathlineto{\pgfqpoint{4.279719in}{3.293514in}}%
\pgfpathlineto{\pgfqpoint{4.284134in}{3.318684in}}%
\pgfpathlineto{\pgfqpoint{4.288548in}{3.304335in}}%
\pgfpathlineto{\pgfqpoint{4.292963in}{3.347892in}}%
\pgfpathlineto{\pgfqpoint{4.297377in}{3.357832in}}%
\pgfpathlineto{\pgfqpoint{4.301791in}{3.328442in}}%
\pgfpathlineto{\pgfqpoint{4.306206in}{3.359121in}}%
\pgfpathlineto{\pgfqpoint{4.310620in}{3.348165in}}%
\pgfpathlineto{\pgfqpoint{4.315035in}{3.366430in}}%
\pgfpathlineto{\pgfqpoint{4.319449in}{3.335321in}}%
\pgfpathlineto{\pgfqpoint{4.323864in}{3.352700in}}%
\pgfpathlineto{\pgfqpoint{4.323864in}{3.352700in}}%
\pgfusepath{stroke}%
\end{pgfscope}%
\begin{pgfscope}%
\pgfpathrectangle{\pgfqpoint{0.625000in}{0.440000in}}{\pgfqpoint{3.875000in}{3.080000in}} %
\pgfusepath{clip}%
\pgfsetrectcap%
\pgfsetroundjoin%
\pgfsetlinewidth{1.505625pt}%
\definecolor{currentstroke}{rgb}{1.000000,0.894118,0.882353}%
\pgfsetstrokecolor{currentstroke}%
\pgfsetdash{}{0pt}%
\pgfpathmoveto{\pgfqpoint{0.801136in}{0.580000in}}%
\pgfpathlineto{\pgfqpoint{0.805551in}{0.580000in}}%
\pgfpathlineto{\pgfqpoint{0.809965in}{1.714301in}}%
\pgfpathlineto{\pgfqpoint{0.814380in}{1.954066in}}%
\pgfpathlineto{\pgfqpoint{0.818794in}{1.987322in}}%
\pgfpathlineto{\pgfqpoint{0.823209in}{2.622272in}}%
\pgfpathlineto{\pgfqpoint{0.827623in}{2.563120in}}%
\pgfpathlineto{\pgfqpoint{0.832037in}{2.058575in}}%
\pgfpathlineto{\pgfqpoint{0.836452in}{2.028130in}}%
\pgfpathlineto{\pgfqpoint{0.840866in}{2.248073in}}%
\pgfpathlineto{\pgfqpoint{0.849695in}{2.018657in}}%
\pgfpathlineto{\pgfqpoint{0.854110in}{1.040353in}}%
\pgfpathlineto{\pgfqpoint{0.858524in}{2.359580in}}%
\pgfpathlineto{\pgfqpoint{0.862939in}{1.940398in}}%
\pgfpathlineto{\pgfqpoint{0.871767in}{2.286788in}}%
\pgfpathlineto{\pgfqpoint{0.876182in}{2.423836in}}%
\pgfpathlineto{\pgfqpoint{0.880596in}{2.231397in}}%
\pgfpathlineto{\pgfqpoint{0.885011in}{2.242829in}}%
\pgfpathlineto{\pgfqpoint{0.889425in}{2.151937in}}%
\pgfpathlineto{\pgfqpoint{0.893840in}{2.428323in}}%
\pgfpathlineto{\pgfqpoint{0.898254in}{1.634897in}}%
\pgfpathlineto{\pgfqpoint{0.902669in}{1.687699in}}%
\pgfpathlineto{\pgfqpoint{0.907083in}{1.489175in}}%
\pgfpathlineto{\pgfqpoint{0.911497in}{1.470322in}}%
\pgfpathlineto{\pgfqpoint{0.915912in}{2.130332in}}%
\pgfpathlineto{\pgfqpoint{0.920326in}{2.522746in}}%
\pgfpathlineto{\pgfqpoint{0.924741in}{2.335946in}}%
\pgfpathlineto{\pgfqpoint{0.933570in}{2.724618in}}%
\pgfpathlineto{\pgfqpoint{0.937984in}{2.440939in}}%
\pgfpathlineto{\pgfqpoint{0.942399in}{2.574220in}}%
\pgfpathlineto{\pgfqpoint{0.946813in}{2.323783in}}%
\pgfpathlineto{\pgfqpoint{0.951228in}{2.613992in}}%
\pgfpathlineto{\pgfqpoint{0.955642in}{2.486184in}}%
\pgfpathlineto{\pgfqpoint{0.960056in}{1.792014in}}%
\pgfpathlineto{\pgfqpoint{0.964471in}{1.414518in}}%
\pgfpathlineto{\pgfqpoint{0.968885in}{2.479744in}}%
\pgfpathlineto{\pgfqpoint{0.973300in}{2.173894in}}%
\pgfpathlineto{\pgfqpoint{0.977714in}{2.066546in}}%
\pgfpathlineto{\pgfqpoint{0.982129in}{2.328380in}}%
\pgfpathlineto{\pgfqpoint{0.986543in}{2.396744in}}%
\pgfpathlineto{\pgfqpoint{0.990958in}{2.389952in}}%
\pgfpathlineto{\pgfqpoint{0.995372in}{2.642243in}}%
\pgfpathlineto{\pgfqpoint{1.004201in}{2.489540in}}%
\pgfpathlineto{\pgfqpoint{1.008615in}{1.366173in}}%
\pgfpathlineto{\pgfqpoint{1.013030in}{2.115761in}}%
\pgfpathlineto{\pgfqpoint{1.017444in}{1.639376in}}%
\pgfpathlineto{\pgfqpoint{1.021859in}{2.107275in}}%
\pgfpathlineto{\pgfqpoint{1.026273in}{2.376593in}}%
\pgfpathlineto{\pgfqpoint{1.030688in}{2.317253in}}%
\pgfpathlineto{\pgfqpoint{1.035102in}{1.547331in}}%
\pgfpathlineto{\pgfqpoint{1.039516in}{2.333217in}}%
\pgfpathlineto{\pgfqpoint{1.043931in}{2.348044in}}%
\pgfpathlineto{\pgfqpoint{1.048345in}{2.357911in}}%
\pgfpathlineto{\pgfqpoint{1.052760in}{2.330823in}}%
\pgfpathlineto{\pgfqpoint{1.057174in}{2.784046in}}%
\pgfpathlineto{\pgfqpoint{1.061589in}{2.184361in}}%
\pgfpathlineto{\pgfqpoint{1.066003in}{2.025437in}}%
\pgfpathlineto{\pgfqpoint{1.070418in}{1.943746in}}%
\pgfpathlineto{\pgfqpoint{1.074832in}{1.983375in}}%
\pgfpathlineto{\pgfqpoint{1.079246in}{2.603844in}}%
\pgfpathlineto{\pgfqpoint{1.083661in}{2.370271in}}%
\pgfpathlineto{\pgfqpoint{1.088075in}{2.405300in}}%
\pgfpathlineto{\pgfqpoint{1.092490in}{2.365269in}}%
\pgfpathlineto{\pgfqpoint{1.096904in}{2.822687in}}%
\pgfpathlineto{\pgfqpoint{1.101319in}{2.813504in}}%
\pgfpathlineto{\pgfqpoint{1.105733in}{2.232247in}}%
\pgfpathlineto{\pgfqpoint{1.110148in}{2.371295in}}%
\pgfpathlineto{\pgfqpoint{1.114562in}{2.327581in}}%
\pgfpathlineto{\pgfqpoint{1.118976in}{2.107660in}}%
\pgfpathlineto{\pgfqpoint{1.127805in}{1.875112in}}%
\pgfpathlineto{\pgfqpoint{1.132220in}{2.135290in}}%
\pgfpathlineto{\pgfqpoint{1.136634in}{2.311522in}}%
\pgfpathlineto{\pgfqpoint{1.141049in}{2.050059in}}%
\pgfpathlineto{\pgfqpoint{1.145463in}{1.021434in}}%
\pgfpathlineto{\pgfqpoint{1.149878in}{2.464995in}}%
\pgfpathlineto{\pgfqpoint{1.154292in}{2.608447in}}%
\pgfpathlineto{\pgfqpoint{1.158706in}{2.700520in}}%
\pgfpathlineto{\pgfqpoint{1.163121in}{2.338667in}}%
\pgfpathlineto{\pgfqpoint{1.167535in}{2.379399in}}%
\pgfpathlineto{\pgfqpoint{1.171950in}{2.005647in}}%
\pgfpathlineto{\pgfqpoint{1.176364in}{2.153907in}}%
\pgfpathlineto{\pgfqpoint{1.180779in}{2.154610in}}%
\pgfpathlineto{\pgfqpoint{1.185193in}{1.874007in}}%
\pgfpathlineto{\pgfqpoint{1.189608in}{2.521165in}}%
\pgfpathlineto{\pgfqpoint{1.194022in}{2.576516in}}%
\pgfpathlineto{\pgfqpoint{1.198436in}{2.437780in}}%
\pgfpathlineto{\pgfqpoint{1.202851in}{2.439833in}}%
\pgfpathlineto{\pgfqpoint{1.207265in}{2.459444in}}%
\pgfpathlineto{\pgfqpoint{1.211680in}{2.344122in}}%
\pgfpathlineto{\pgfqpoint{1.216094in}{2.111760in}}%
\pgfpathlineto{\pgfqpoint{1.220509in}{2.560954in}}%
\pgfpathlineto{\pgfqpoint{1.224923in}{2.327103in}}%
\pgfpathlineto{\pgfqpoint{1.229338in}{2.452039in}}%
\pgfpathlineto{\pgfqpoint{1.238166in}{1.486595in}}%
\pgfpathlineto{\pgfqpoint{1.242581in}{2.167645in}}%
\pgfpathlineto{\pgfqpoint{1.246995in}{1.935812in}}%
\pgfpathlineto{\pgfqpoint{1.251410in}{2.727612in}}%
\pgfpathlineto{\pgfqpoint{1.255824in}{1.015757in}}%
\pgfpathlineto{\pgfqpoint{1.260239in}{2.799318in}}%
\pgfpathlineto{\pgfqpoint{1.264653in}{2.636062in}}%
\pgfpathlineto{\pgfqpoint{1.269068in}{2.778694in}}%
\pgfpathlineto{\pgfqpoint{1.273482in}{2.561826in}}%
\pgfpathlineto{\pgfqpoint{1.277896in}{2.401161in}}%
\pgfpathlineto{\pgfqpoint{1.282311in}{2.503839in}}%
\pgfpathlineto{\pgfqpoint{1.286725in}{2.533758in}}%
\pgfpathlineto{\pgfqpoint{1.291140in}{2.254111in}}%
\pgfpathlineto{\pgfqpoint{1.295554in}{1.665196in}}%
\pgfpathlineto{\pgfqpoint{1.299969in}{1.826778in}}%
\pgfpathlineto{\pgfqpoint{1.304383in}{2.627359in}}%
\pgfpathlineto{\pgfqpoint{1.308798in}{1.630469in}}%
\pgfpathlineto{\pgfqpoint{1.313212in}{2.344474in}}%
\pgfpathlineto{\pgfqpoint{1.317626in}{2.357827in}}%
\pgfpathlineto{\pgfqpoint{1.322041in}{2.462345in}}%
\pgfpathlineto{\pgfqpoint{1.326455in}{2.727797in}}%
\pgfpathlineto{\pgfqpoint{1.330870in}{2.682575in}}%
\pgfpathlineto{\pgfqpoint{1.335284in}{2.367835in}}%
\pgfpathlineto{\pgfqpoint{1.339699in}{2.141997in}}%
\pgfpathlineto{\pgfqpoint{1.344113in}{2.092262in}}%
\pgfpathlineto{\pgfqpoint{1.348528in}{1.312538in}}%
\pgfpathlineto{\pgfqpoint{1.352942in}{1.808986in}}%
\pgfpathlineto{\pgfqpoint{1.361771in}{2.524057in}}%
\pgfpathlineto{\pgfqpoint{1.366185in}{0.868408in}}%
\pgfpathlineto{\pgfqpoint{1.370600in}{2.349775in}}%
\pgfpathlineto{\pgfqpoint{1.375014in}{2.130647in}}%
\pgfpathlineto{\pgfqpoint{1.379429in}{2.399642in}}%
\pgfpathlineto{\pgfqpoint{1.383843in}{2.269901in}}%
\pgfpathlineto{\pgfqpoint{1.388258in}{2.713355in}}%
\pgfpathlineto{\pgfqpoint{1.392672in}{2.314923in}}%
\pgfpathlineto{\pgfqpoint{1.397086in}{2.368144in}}%
\pgfpathlineto{\pgfqpoint{1.401501in}{2.303149in}}%
\pgfpathlineto{\pgfqpoint{1.405915in}{1.617006in}}%
\pgfpathlineto{\pgfqpoint{1.410330in}{2.321468in}}%
\pgfpathlineto{\pgfqpoint{1.414744in}{2.478787in}}%
\pgfpathlineto{\pgfqpoint{1.419159in}{1.179131in}}%
\pgfpathlineto{\pgfqpoint{1.423573in}{2.245755in}}%
\pgfpathlineto{\pgfqpoint{1.432402in}{2.794622in}}%
\pgfpathlineto{\pgfqpoint{1.436816in}{2.419061in}}%
\pgfpathlineto{\pgfqpoint{1.441231in}{2.480748in}}%
\pgfpathlineto{\pgfqpoint{1.445645in}{2.467052in}}%
\pgfpathlineto{\pgfqpoint{1.450060in}{2.542708in}}%
\pgfpathlineto{\pgfqpoint{1.458889in}{1.695884in}}%
\pgfpathlineto{\pgfqpoint{1.463303in}{2.277579in}}%
\pgfpathlineto{\pgfqpoint{1.467718in}{2.421796in}}%
\pgfpathlineto{\pgfqpoint{1.472132in}{2.150930in}}%
\pgfpathlineto{\pgfqpoint{1.476546in}{0.669899in}}%
\pgfpathlineto{\pgfqpoint{1.480961in}{2.198231in}}%
\pgfpathlineto{\pgfqpoint{1.485375in}{2.042029in}}%
\pgfpathlineto{\pgfqpoint{1.489790in}{2.518360in}}%
\pgfpathlineto{\pgfqpoint{1.494204in}{2.315146in}}%
\pgfpathlineto{\pgfqpoint{1.498619in}{2.829560in}}%
\pgfpathlineto{\pgfqpoint{1.503033in}{2.110527in}}%
\pgfpathlineto{\pgfqpoint{1.507448in}{2.112624in}}%
\pgfpathlineto{\pgfqpoint{1.511862in}{2.262555in}}%
\pgfpathlineto{\pgfqpoint{1.516276in}{1.646275in}}%
\pgfpathlineto{\pgfqpoint{1.520691in}{2.719236in}}%
\pgfpathlineto{\pgfqpoint{1.525105in}{2.357262in}}%
\pgfpathlineto{\pgfqpoint{1.529520in}{0.964120in}}%
\pgfpathlineto{\pgfqpoint{1.533934in}{2.440337in}}%
\pgfpathlineto{\pgfqpoint{1.542763in}{2.747911in}}%
\pgfpathlineto{\pgfqpoint{1.547178in}{2.239396in}}%
\pgfpathlineto{\pgfqpoint{1.551592in}{2.317115in}}%
\pgfpathlineto{\pgfqpoint{1.556006in}{2.737940in}}%
\pgfpathlineto{\pgfqpoint{1.560421in}{2.345602in}}%
\pgfpathlineto{\pgfqpoint{1.564835in}{2.344153in}}%
\pgfpathlineto{\pgfqpoint{1.569250in}{1.812868in}}%
\pgfpathlineto{\pgfqpoint{1.573664in}{2.751490in}}%
\pgfpathlineto{\pgfqpoint{1.578079in}{2.704439in}}%
\pgfpathlineto{\pgfqpoint{1.582493in}{2.491350in}}%
\pgfpathlineto{\pgfqpoint{1.586908in}{0.826314in}}%
\pgfpathlineto{\pgfqpoint{1.591322in}{2.546377in}}%
\pgfpathlineto{\pgfqpoint{1.595737in}{2.626780in}}%
\pgfpathlineto{\pgfqpoint{1.600151in}{2.782985in}}%
\pgfpathlineto{\pgfqpoint{1.604565in}{2.836904in}}%
\pgfpathlineto{\pgfqpoint{1.608980in}{2.858993in}}%
\pgfpathlineto{\pgfqpoint{1.613394in}{2.608337in}}%
\pgfpathlineto{\pgfqpoint{1.617809in}{2.563478in}}%
\pgfpathlineto{\pgfqpoint{1.622223in}{2.577092in}}%
\pgfpathlineto{\pgfqpoint{1.626638in}{2.322388in}}%
\pgfpathlineto{\pgfqpoint{1.631052in}{2.897871in}}%
\pgfpathlineto{\pgfqpoint{1.635467in}{2.733137in}}%
\pgfpathlineto{\pgfqpoint{1.639881in}{1.192037in}}%
\pgfpathlineto{\pgfqpoint{1.644295in}{2.047482in}}%
\pgfpathlineto{\pgfqpoint{1.648710in}{2.577641in}}%
\pgfpathlineto{\pgfqpoint{1.653124in}{2.746232in}}%
\pgfpathlineto{\pgfqpoint{1.657539in}{2.540052in}}%
\pgfpathlineto{\pgfqpoint{1.666368in}{2.956935in}}%
\pgfpathlineto{\pgfqpoint{1.670782in}{1.948464in}}%
\pgfpathlineto{\pgfqpoint{1.675197in}{2.403221in}}%
\pgfpathlineto{\pgfqpoint{1.679611in}{2.050987in}}%
\pgfpathlineto{\pgfqpoint{1.684025in}{2.456490in}}%
\pgfpathlineto{\pgfqpoint{1.688440in}{2.674604in}}%
\pgfpathlineto{\pgfqpoint{1.692854in}{2.742405in}}%
\pgfpathlineto{\pgfqpoint{1.697269in}{1.342570in}}%
\pgfpathlineto{\pgfqpoint{1.701683in}{2.613730in}}%
\pgfpathlineto{\pgfqpoint{1.706098in}{2.819280in}}%
\pgfpathlineto{\pgfqpoint{1.710512in}{2.719672in}}%
\pgfpathlineto{\pgfqpoint{1.714927in}{2.944632in}}%
\pgfpathlineto{\pgfqpoint{1.719341in}{2.979821in}}%
\pgfpathlineto{\pgfqpoint{1.723755in}{2.139974in}}%
\pgfpathlineto{\pgfqpoint{1.728170in}{2.122871in}}%
\pgfpathlineto{\pgfqpoint{1.732584in}{2.805893in}}%
\pgfpathlineto{\pgfqpoint{1.736999in}{2.002659in}}%
\pgfpathlineto{\pgfqpoint{1.741413in}{2.752486in}}%
\pgfpathlineto{\pgfqpoint{1.745828in}{2.235576in}}%
\pgfpathlineto{\pgfqpoint{1.750242in}{1.454214in}}%
\pgfpathlineto{\pgfqpoint{1.754657in}{2.530765in}}%
\pgfpathlineto{\pgfqpoint{1.759071in}{2.693255in}}%
\pgfpathlineto{\pgfqpoint{1.763485in}{2.734685in}}%
\pgfpathlineto{\pgfqpoint{1.772314in}{2.992825in}}%
\pgfpathlineto{\pgfqpoint{1.776729in}{2.955897in}}%
\pgfpathlineto{\pgfqpoint{1.781143in}{2.702633in}}%
\pgfpathlineto{\pgfqpoint{1.785558in}{1.921384in}}%
\pgfpathlineto{\pgfqpoint{1.789972in}{2.159801in}}%
\pgfpathlineto{\pgfqpoint{1.794387in}{2.778036in}}%
\pgfpathlineto{\pgfqpoint{1.798801in}{2.377502in}}%
\pgfpathlineto{\pgfqpoint{1.803215in}{2.437006in}}%
\pgfpathlineto{\pgfqpoint{1.807630in}{1.200137in}}%
\pgfpathlineto{\pgfqpoint{1.812044in}{2.682263in}}%
\pgfpathlineto{\pgfqpoint{1.816459in}{2.836082in}}%
\pgfpathlineto{\pgfqpoint{1.820873in}{2.525678in}}%
\pgfpathlineto{\pgfqpoint{1.825288in}{2.628420in}}%
\pgfpathlineto{\pgfqpoint{1.829702in}{2.894143in}}%
\pgfpathlineto{\pgfqpoint{1.834117in}{1.826421in}}%
\pgfpathlineto{\pgfqpoint{1.838531in}{2.526510in}}%
\pgfpathlineto{\pgfqpoint{1.842945in}{2.765198in}}%
\pgfpathlineto{\pgfqpoint{1.847360in}{2.555822in}}%
\pgfpathlineto{\pgfqpoint{1.851774in}{2.701778in}}%
\pgfpathlineto{\pgfqpoint{1.856189in}{2.089651in}}%
\pgfpathlineto{\pgfqpoint{1.860603in}{1.213017in}}%
\pgfpathlineto{\pgfqpoint{1.865018in}{3.046887in}}%
\pgfpathlineto{\pgfqpoint{1.869432in}{3.009985in}}%
\pgfpathlineto{\pgfqpoint{1.873847in}{3.076835in}}%
\pgfpathlineto{\pgfqpoint{1.878261in}{2.911660in}}%
\pgfpathlineto{\pgfqpoint{1.882675in}{2.674199in}}%
\pgfpathlineto{\pgfqpoint{1.887090in}{2.947502in}}%
\pgfpathlineto{\pgfqpoint{1.891504in}{2.819617in}}%
\pgfpathlineto{\pgfqpoint{1.895919in}{2.203152in}}%
\pgfpathlineto{\pgfqpoint{1.904748in}{2.827248in}}%
\pgfpathlineto{\pgfqpoint{1.909162in}{2.527661in}}%
\pgfpathlineto{\pgfqpoint{1.913577in}{2.768236in}}%
\pgfpathlineto{\pgfqpoint{1.917991in}{1.706806in}}%
\pgfpathlineto{\pgfqpoint{1.922405in}{2.952709in}}%
\pgfpathlineto{\pgfqpoint{1.926820in}{2.938025in}}%
\pgfpathlineto{\pgfqpoint{1.931234in}{2.821888in}}%
\pgfpathlineto{\pgfqpoint{1.935649in}{2.586287in}}%
\pgfpathlineto{\pgfqpoint{1.940063in}{2.851517in}}%
\pgfpathlineto{\pgfqpoint{1.944478in}{2.561998in}}%
\pgfpathlineto{\pgfqpoint{1.948892in}{2.874622in}}%
\pgfpathlineto{\pgfqpoint{1.953307in}{2.568894in}}%
\pgfpathlineto{\pgfqpoint{1.957721in}{2.900344in}}%
\pgfpathlineto{\pgfqpoint{1.966550in}{2.415378in}}%
\pgfpathlineto{\pgfqpoint{1.970964in}{1.211965in}}%
\pgfpathlineto{\pgfqpoint{1.975379in}{2.820717in}}%
\pgfpathlineto{\pgfqpoint{1.979793in}{2.807564in}}%
\pgfpathlineto{\pgfqpoint{1.984208in}{2.551922in}}%
\pgfpathlineto{\pgfqpoint{1.993037in}{2.877990in}}%
\pgfpathlineto{\pgfqpoint{1.997451in}{2.941247in}}%
\pgfpathlineto{\pgfqpoint{2.001865in}{2.490460in}}%
\pgfpathlineto{\pgfqpoint{2.006280in}{2.409965in}}%
\pgfpathlineto{\pgfqpoint{2.010694in}{2.212243in}}%
\pgfpathlineto{\pgfqpoint{2.015109in}{2.533792in}}%
\pgfpathlineto{\pgfqpoint{2.019523in}{3.070369in}}%
\pgfpathlineto{\pgfqpoint{2.023938in}{2.518782in}}%
\pgfpathlineto{\pgfqpoint{2.028352in}{2.678507in}}%
\pgfpathlineto{\pgfqpoint{2.032767in}{3.025881in}}%
\pgfpathlineto{\pgfqpoint{2.037181in}{2.883488in}}%
\pgfpathlineto{\pgfqpoint{2.041595in}{2.538935in}}%
\pgfpathlineto{\pgfqpoint{2.046010in}{2.836651in}}%
\pgfpathlineto{\pgfqpoint{2.050424in}{2.993506in}}%
\pgfpathlineto{\pgfqpoint{2.054839in}{2.517490in}}%
\pgfpathlineto{\pgfqpoint{2.059253in}{2.205735in}}%
\pgfpathlineto{\pgfqpoint{2.063668in}{2.463377in}}%
\pgfpathlineto{\pgfqpoint{2.068082in}{2.899345in}}%
\pgfpathlineto{\pgfqpoint{2.072497in}{2.695337in}}%
\pgfpathlineto{\pgfqpoint{2.076911in}{2.622219in}}%
\pgfpathlineto{\pgfqpoint{2.081325in}{1.340083in}}%
\pgfpathlineto{\pgfqpoint{2.085740in}{2.807719in}}%
\pgfpathlineto{\pgfqpoint{2.090154in}{2.739853in}}%
\pgfpathlineto{\pgfqpoint{2.094569in}{2.771970in}}%
\pgfpathlineto{\pgfqpoint{2.098983in}{2.899365in}}%
\pgfpathlineto{\pgfqpoint{2.103398in}{2.962399in}}%
\pgfpathlineto{\pgfqpoint{2.112227in}{2.561874in}}%
\pgfpathlineto{\pgfqpoint{2.116641in}{1.930497in}}%
\pgfpathlineto{\pgfqpoint{2.121055in}{2.402667in}}%
\pgfpathlineto{\pgfqpoint{2.125470in}{2.538153in}}%
\pgfpathlineto{\pgfqpoint{2.129884in}{2.999572in}}%
\pgfpathlineto{\pgfqpoint{2.134299in}{2.518472in}}%
\pgfpathlineto{\pgfqpoint{2.138713in}{2.778714in}}%
\pgfpathlineto{\pgfqpoint{2.143128in}{2.802339in}}%
\pgfpathlineto{\pgfqpoint{2.147542in}{2.770549in}}%
\pgfpathlineto{\pgfqpoint{2.151957in}{2.418839in}}%
\pgfpathlineto{\pgfqpoint{2.156371in}{2.775107in}}%
\pgfpathlineto{\pgfqpoint{2.160785in}{2.911660in}}%
\pgfpathlineto{\pgfqpoint{2.165200in}{2.633248in}}%
\pgfpathlineto{\pgfqpoint{2.169614in}{2.188361in}}%
\pgfpathlineto{\pgfqpoint{2.174029in}{2.780678in}}%
\pgfpathlineto{\pgfqpoint{2.178443in}{2.963435in}}%
\pgfpathlineto{\pgfqpoint{2.191687in}{1.869800in}}%
\pgfpathlineto{\pgfqpoint{2.196101in}{3.022370in}}%
\pgfpathlineto{\pgfqpoint{2.200515in}{3.008910in}}%
\pgfpathlineto{\pgfqpoint{2.204930in}{2.966130in}}%
\pgfpathlineto{\pgfqpoint{2.209344in}{2.951111in}}%
\pgfpathlineto{\pgfqpoint{2.213759in}{2.629723in}}%
\pgfpathlineto{\pgfqpoint{2.218173in}{2.872357in}}%
\pgfpathlineto{\pgfqpoint{2.222588in}{2.691714in}}%
\pgfpathlineto{\pgfqpoint{2.227002in}{2.056052in}}%
\pgfpathlineto{\pgfqpoint{2.231417in}{2.828533in}}%
\pgfpathlineto{\pgfqpoint{2.235831in}{3.001747in}}%
\pgfpathlineto{\pgfqpoint{2.240246in}{2.883476in}}%
\pgfpathlineto{\pgfqpoint{2.244660in}{2.706808in}}%
\pgfpathlineto{\pgfqpoint{2.249074in}{2.966681in}}%
\pgfpathlineto{\pgfqpoint{2.253489in}{2.913432in}}%
\pgfpathlineto{\pgfqpoint{2.257903in}{3.014157in}}%
\pgfpathlineto{\pgfqpoint{2.262318in}{3.021909in}}%
\pgfpathlineto{\pgfqpoint{2.266732in}{2.593411in}}%
\pgfpathlineto{\pgfqpoint{2.271147in}{2.519592in}}%
\pgfpathlineto{\pgfqpoint{2.275561in}{2.794546in}}%
\pgfpathlineto{\pgfqpoint{2.279976in}{2.596357in}}%
\pgfpathlineto{\pgfqpoint{2.288804in}{2.925289in}}%
\pgfpathlineto{\pgfqpoint{2.293219in}{2.582174in}}%
\pgfpathlineto{\pgfqpoint{2.297633in}{2.431773in}}%
\pgfpathlineto{\pgfqpoint{2.302048in}{2.824631in}}%
\pgfpathlineto{\pgfqpoint{2.306462in}{2.966546in}}%
\pgfpathlineto{\pgfqpoint{2.310877in}{2.858523in}}%
\pgfpathlineto{\pgfqpoint{2.315291in}{2.640932in}}%
\pgfpathlineto{\pgfqpoint{2.319706in}{2.627674in}}%
\pgfpathlineto{\pgfqpoint{2.324120in}{2.878159in}}%
\pgfpathlineto{\pgfqpoint{2.328534in}{2.967315in}}%
\pgfpathlineto{\pgfqpoint{2.332949in}{2.096434in}}%
\pgfpathlineto{\pgfqpoint{2.341778in}{3.008814in}}%
\pgfpathlineto{\pgfqpoint{2.346192in}{2.650923in}}%
\pgfpathlineto{\pgfqpoint{2.350607in}{2.909192in}}%
\pgfpathlineto{\pgfqpoint{2.355021in}{2.647470in}}%
\pgfpathlineto{\pgfqpoint{2.359436in}{2.944376in}}%
\pgfpathlineto{\pgfqpoint{2.363850in}{3.012905in}}%
\pgfpathlineto{\pgfqpoint{2.368264in}{3.130818in}}%
\pgfpathlineto{\pgfqpoint{2.372679in}{2.702413in}}%
\pgfpathlineto{\pgfqpoint{2.377093in}{2.674450in}}%
\pgfpathlineto{\pgfqpoint{2.381508in}{2.889511in}}%
\pgfpathlineto{\pgfqpoint{2.385922in}{2.600358in}}%
\pgfpathlineto{\pgfqpoint{2.390337in}{2.641148in}}%
\pgfpathlineto{\pgfqpoint{2.394751in}{2.395498in}}%
\pgfpathlineto{\pgfqpoint{2.399166in}{2.905906in}}%
\pgfpathlineto{\pgfqpoint{2.407994in}{2.775197in}}%
\pgfpathlineto{\pgfqpoint{2.412409in}{2.854370in}}%
\pgfpathlineto{\pgfqpoint{2.416823in}{2.878291in}}%
\pgfpathlineto{\pgfqpoint{2.421238in}{2.794020in}}%
\pgfpathlineto{\pgfqpoint{2.425652in}{2.483021in}}%
\pgfpathlineto{\pgfqpoint{2.430067in}{2.611674in}}%
\pgfpathlineto{\pgfqpoint{2.434481in}{2.883116in}}%
\pgfpathlineto{\pgfqpoint{2.438896in}{2.454053in}}%
\pgfpathlineto{\pgfqpoint{2.447724in}{2.289798in}}%
\pgfpathlineto{\pgfqpoint{2.452139in}{2.981455in}}%
\pgfpathlineto{\pgfqpoint{2.456553in}{2.303748in}}%
\pgfpathlineto{\pgfqpoint{2.460968in}{2.796693in}}%
\pgfpathlineto{\pgfqpoint{2.465382in}{2.754385in}}%
\pgfpathlineto{\pgfqpoint{2.469797in}{2.917917in}}%
\pgfpathlineto{\pgfqpoint{2.474211in}{2.880829in}}%
\pgfpathlineto{\pgfqpoint{2.478626in}{2.885215in}}%
\pgfpathlineto{\pgfqpoint{2.483040in}{2.392315in}}%
\pgfpathlineto{\pgfqpoint{2.487454in}{2.866449in}}%
\pgfpathlineto{\pgfqpoint{2.491869in}{2.885837in}}%
\pgfpathlineto{\pgfqpoint{2.496283in}{2.462151in}}%
\pgfpathlineto{\pgfqpoint{2.500698in}{2.246399in}}%
\pgfpathlineto{\pgfqpoint{2.505112in}{2.316701in}}%
\pgfpathlineto{\pgfqpoint{2.509527in}{2.873519in}}%
\pgfpathlineto{\pgfqpoint{2.513941in}{2.612656in}}%
\pgfpathlineto{\pgfqpoint{2.518356in}{3.042428in}}%
\pgfpathlineto{\pgfqpoint{2.522770in}{3.010258in}}%
\pgfpathlineto{\pgfqpoint{2.527184in}{3.090570in}}%
\pgfpathlineto{\pgfqpoint{2.531599in}{3.035695in}}%
\pgfpathlineto{\pgfqpoint{2.536013in}{2.907932in}}%
\pgfpathlineto{\pgfqpoint{2.540428in}{2.899784in}}%
\pgfpathlineto{\pgfqpoint{2.544842in}{2.787807in}}%
\pgfpathlineto{\pgfqpoint{2.549257in}{2.433033in}}%
\pgfpathlineto{\pgfqpoint{2.553671in}{2.798288in}}%
\pgfpathlineto{\pgfqpoint{2.558086in}{2.044198in}}%
\pgfpathlineto{\pgfqpoint{2.562500in}{2.997757in}}%
\pgfpathlineto{\pgfqpoint{2.566914in}{2.764345in}}%
\pgfpathlineto{\pgfqpoint{2.571329in}{2.954178in}}%
\pgfpathlineto{\pgfqpoint{2.575743in}{2.999527in}}%
\pgfpathlineto{\pgfqpoint{2.580158in}{2.875655in}}%
\pgfpathlineto{\pgfqpoint{2.584572in}{2.970125in}}%
\pgfpathlineto{\pgfqpoint{2.588987in}{2.896610in}}%
\pgfpathlineto{\pgfqpoint{2.593401in}{2.897654in}}%
\pgfpathlineto{\pgfqpoint{2.597816in}{2.790694in}}%
\pgfpathlineto{\pgfqpoint{2.606644in}{2.459239in}}%
\pgfpathlineto{\pgfqpoint{2.611059in}{2.146912in}}%
\pgfpathlineto{\pgfqpoint{2.615473in}{2.967658in}}%
\pgfpathlineto{\pgfqpoint{2.619888in}{2.884588in}}%
\pgfpathlineto{\pgfqpoint{2.624302in}{2.484659in}}%
\pgfpathlineto{\pgfqpoint{2.628717in}{2.911007in}}%
\pgfpathlineto{\pgfqpoint{2.633131in}{2.879906in}}%
\pgfpathlineto{\pgfqpoint{2.637546in}{3.112857in}}%
\pgfpathlineto{\pgfqpoint{2.641960in}{3.026779in}}%
\pgfpathlineto{\pgfqpoint{2.646374in}{2.975924in}}%
\pgfpathlineto{\pgfqpoint{2.655203in}{2.652960in}}%
\pgfpathlineto{\pgfqpoint{2.659618in}{2.981605in}}%
\pgfpathlineto{\pgfqpoint{2.664032in}{2.860242in}}%
\pgfpathlineto{\pgfqpoint{2.668447in}{2.363097in}}%
\pgfpathlineto{\pgfqpoint{2.672861in}{2.903982in}}%
\pgfpathlineto{\pgfqpoint{2.677276in}{2.702999in}}%
\pgfpathlineto{\pgfqpoint{2.681690in}{2.798690in}}%
\pgfpathlineto{\pgfqpoint{2.686104in}{2.684961in}}%
\pgfpathlineto{\pgfqpoint{2.690519in}{2.978529in}}%
\pgfpathlineto{\pgfqpoint{2.694933in}{3.109832in}}%
\pgfpathlineto{\pgfqpoint{2.699348in}{3.127242in}}%
\pgfpathlineto{\pgfqpoint{2.703762in}{3.118332in}}%
\pgfpathlineto{\pgfqpoint{2.708177in}{2.732679in}}%
\pgfpathlineto{\pgfqpoint{2.712591in}{2.844258in}}%
\pgfpathlineto{\pgfqpoint{2.717006in}{2.588597in}}%
\pgfpathlineto{\pgfqpoint{2.721420in}{2.728484in}}%
\pgfpathlineto{\pgfqpoint{2.725834in}{2.497458in}}%
\pgfpathlineto{\pgfqpoint{2.730249in}{2.931509in}}%
\pgfpathlineto{\pgfqpoint{2.734663in}{2.781556in}}%
\pgfpathlineto{\pgfqpoint{2.739078in}{2.764714in}}%
\pgfpathlineto{\pgfqpoint{2.743492in}{2.942434in}}%
\pgfpathlineto{\pgfqpoint{2.747907in}{2.896967in}}%
\pgfpathlineto{\pgfqpoint{2.752321in}{2.937595in}}%
\pgfpathlineto{\pgfqpoint{2.756736in}{2.937361in}}%
\pgfpathlineto{\pgfqpoint{2.761150in}{2.835632in}}%
\pgfpathlineto{\pgfqpoint{2.765564in}{2.637786in}}%
\pgfpathlineto{\pgfqpoint{2.769979in}{2.919414in}}%
\pgfpathlineto{\pgfqpoint{2.774393in}{2.245918in}}%
\pgfpathlineto{\pgfqpoint{2.778808in}{2.685175in}}%
\pgfpathlineto{\pgfqpoint{2.783222in}{2.888808in}}%
\pgfpathlineto{\pgfqpoint{2.787637in}{2.888057in}}%
\pgfpathlineto{\pgfqpoint{2.792051in}{2.254128in}}%
\pgfpathlineto{\pgfqpoint{2.796466in}{2.220371in}}%
\pgfpathlineto{\pgfqpoint{2.800880in}{2.986309in}}%
\pgfpathlineto{\pgfqpoint{2.805294in}{2.961490in}}%
\pgfpathlineto{\pgfqpoint{2.809709in}{2.846898in}}%
\pgfpathlineto{\pgfqpoint{2.814123in}{3.014788in}}%
\pgfpathlineto{\pgfqpoint{2.818538in}{2.813774in}}%
\pgfpathlineto{\pgfqpoint{2.822952in}{3.037178in}}%
\pgfpathlineto{\pgfqpoint{2.827367in}{2.514491in}}%
\pgfpathlineto{\pgfqpoint{2.831781in}{2.551357in}}%
\pgfpathlineto{\pgfqpoint{2.836196in}{2.264164in}}%
\pgfpathlineto{\pgfqpoint{2.840610in}{2.484836in}}%
\pgfpathlineto{\pgfqpoint{2.845024in}{2.883482in}}%
\pgfpathlineto{\pgfqpoint{2.849439in}{2.781435in}}%
\pgfpathlineto{\pgfqpoint{2.853853in}{3.044515in}}%
\pgfpathlineto{\pgfqpoint{2.858268in}{2.916724in}}%
\pgfpathlineto{\pgfqpoint{2.862682in}{3.116635in}}%
\pgfpathlineto{\pgfqpoint{2.867097in}{3.113926in}}%
\pgfpathlineto{\pgfqpoint{2.871511in}{2.794754in}}%
\pgfpathlineto{\pgfqpoint{2.875926in}{2.948860in}}%
\pgfpathlineto{\pgfqpoint{2.880340in}{2.946705in}}%
\pgfpathlineto{\pgfqpoint{2.884754in}{2.571313in}}%
\pgfpathlineto{\pgfqpoint{2.893583in}{2.946860in}}%
\pgfpathlineto{\pgfqpoint{2.897998in}{2.947198in}}%
\pgfpathlineto{\pgfqpoint{2.902412in}{1.944300in}}%
\pgfpathlineto{\pgfqpoint{2.911241in}{3.013544in}}%
\pgfpathlineto{\pgfqpoint{2.915656in}{2.966904in}}%
\pgfpathlineto{\pgfqpoint{2.920070in}{2.949589in}}%
\pgfpathlineto{\pgfqpoint{2.924485in}{3.083511in}}%
\pgfpathlineto{\pgfqpoint{2.928899in}{2.881189in}}%
\pgfpathlineto{\pgfqpoint{2.933313in}{2.974880in}}%
\pgfpathlineto{\pgfqpoint{2.937728in}{2.681174in}}%
\pgfpathlineto{\pgfqpoint{2.942142in}{2.555555in}}%
\pgfpathlineto{\pgfqpoint{2.950971in}{2.514587in}}%
\pgfpathlineto{\pgfqpoint{2.955386in}{2.872245in}}%
\pgfpathlineto{\pgfqpoint{2.959800in}{2.801602in}}%
\pgfpathlineto{\pgfqpoint{2.964215in}{2.914569in}}%
\pgfpathlineto{\pgfqpoint{2.968629in}{3.133018in}}%
\pgfpathlineto{\pgfqpoint{2.977458in}{2.985411in}}%
\pgfpathlineto{\pgfqpoint{2.981872in}{2.803898in}}%
\pgfpathlineto{\pgfqpoint{2.986287in}{2.831707in}}%
\pgfpathlineto{\pgfqpoint{2.990701in}{3.045033in}}%
\pgfpathlineto{\pgfqpoint{2.995116in}{2.727845in}}%
\pgfpathlineto{\pgfqpoint{2.999530in}{2.656854in}}%
\pgfpathlineto{\pgfqpoint{3.003945in}{2.395647in}}%
\pgfpathlineto{\pgfqpoint{3.008359in}{2.858109in}}%
\pgfpathlineto{\pgfqpoint{3.012773in}{2.185252in}}%
\pgfpathlineto{\pgfqpoint{3.021602in}{2.950675in}}%
\pgfpathlineto{\pgfqpoint{3.026017in}{3.118931in}}%
\pgfpathlineto{\pgfqpoint{3.030431in}{3.100122in}}%
\pgfpathlineto{\pgfqpoint{3.034846in}{3.114747in}}%
\pgfpathlineto{\pgfqpoint{3.039260in}{2.956449in}}%
\pgfpathlineto{\pgfqpoint{3.043675in}{3.110850in}}%
\pgfpathlineto{\pgfqpoint{3.048089in}{2.754183in}}%
\pgfpathlineto{\pgfqpoint{3.052503in}{2.640287in}}%
\pgfpathlineto{\pgfqpoint{3.056918in}{2.436530in}}%
\pgfpathlineto{\pgfqpoint{3.061332in}{2.088239in}}%
\pgfpathlineto{\pgfqpoint{3.065747in}{2.823956in}}%
\pgfpathlineto{\pgfqpoint{3.070161in}{2.954158in}}%
\pgfpathlineto{\pgfqpoint{3.074576in}{2.899035in}}%
\pgfpathlineto{\pgfqpoint{3.078990in}{2.925635in}}%
\pgfpathlineto{\pgfqpoint{3.083405in}{2.831066in}}%
\pgfpathlineto{\pgfqpoint{3.087819in}{2.762820in}}%
\pgfpathlineto{\pgfqpoint{3.092233in}{2.849703in}}%
\pgfpathlineto{\pgfqpoint{3.096648in}{2.858475in}}%
\pgfpathlineto{\pgfqpoint{3.101062in}{3.022592in}}%
\pgfpathlineto{\pgfqpoint{3.105477in}{2.291259in}}%
\pgfpathlineto{\pgfqpoint{3.109891in}{2.505066in}}%
\pgfpathlineto{\pgfqpoint{3.114306in}{2.429291in}}%
\pgfpathlineto{\pgfqpoint{3.118720in}{2.423985in}}%
\pgfpathlineto{\pgfqpoint{3.123135in}{2.524476in}}%
\pgfpathlineto{\pgfqpoint{3.127549in}{2.856042in}}%
\pgfpathlineto{\pgfqpoint{3.136378in}{2.844244in}}%
\pgfpathlineto{\pgfqpoint{3.140792in}{2.841698in}}%
\pgfpathlineto{\pgfqpoint{3.145207in}{2.769595in}}%
\pgfpathlineto{\pgfqpoint{3.149621in}{2.849551in}}%
\pgfpathlineto{\pgfqpoint{3.154036in}{2.952130in}}%
\pgfpathlineto{\pgfqpoint{3.158450in}{2.835131in}}%
\pgfpathlineto{\pgfqpoint{3.162865in}{2.768456in}}%
\pgfpathlineto{\pgfqpoint{3.167279in}{2.069706in}}%
\pgfpathlineto{\pgfqpoint{3.171693in}{2.278243in}}%
\pgfpathlineto{\pgfqpoint{3.176108in}{2.616786in}}%
\pgfpathlineto{\pgfqpoint{3.180522in}{2.727547in}}%
\pgfpathlineto{\pgfqpoint{3.184937in}{2.949420in}}%
\pgfpathlineto{\pgfqpoint{3.189351in}{2.874248in}}%
\pgfpathlineto{\pgfqpoint{3.193766in}{2.982184in}}%
\pgfpathlineto{\pgfqpoint{3.198180in}{2.970269in}}%
\pgfpathlineto{\pgfqpoint{3.202595in}{2.647105in}}%
\pgfpathlineto{\pgfqpoint{3.211423in}{3.007033in}}%
\pgfpathlineto{\pgfqpoint{3.215838in}{2.508073in}}%
\pgfpathlineto{\pgfqpoint{3.220252in}{2.740486in}}%
\pgfpathlineto{\pgfqpoint{3.224667in}{2.027235in}}%
\pgfpathlineto{\pgfqpoint{3.229081in}{2.728889in}}%
\pgfpathlineto{\pgfqpoint{3.233496in}{2.830067in}}%
\pgfpathlineto{\pgfqpoint{3.237910in}{2.749290in}}%
\pgfpathlineto{\pgfqpoint{3.242325in}{2.953525in}}%
\pgfpathlineto{\pgfqpoint{3.246739in}{2.778905in}}%
\pgfpathlineto{\pgfqpoint{3.251153in}{2.765229in}}%
\pgfpathlineto{\pgfqpoint{3.255568in}{2.829392in}}%
\pgfpathlineto{\pgfqpoint{3.259982in}{2.613246in}}%
\pgfpathlineto{\pgfqpoint{3.264397in}{3.073852in}}%
\pgfpathlineto{\pgfqpoint{3.268811in}{3.032125in}}%
\pgfpathlineto{\pgfqpoint{3.273226in}{3.036868in}}%
\pgfpathlineto{\pgfqpoint{3.277640in}{2.657807in}}%
\pgfpathlineto{\pgfqpoint{3.282055in}{2.769514in}}%
\pgfpathlineto{\pgfqpoint{3.286469in}{2.422888in}}%
\pgfpathlineto{\pgfqpoint{3.290883in}{2.828111in}}%
\pgfpathlineto{\pgfqpoint{3.295298in}{3.011628in}}%
\pgfpathlineto{\pgfqpoint{3.299712in}{2.971912in}}%
\pgfpathlineto{\pgfqpoint{3.304127in}{3.043919in}}%
\pgfpathlineto{\pgfqpoint{3.308541in}{3.033959in}}%
\pgfpathlineto{\pgfqpoint{3.312956in}{2.660480in}}%
\pgfpathlineto{\pgfqpoint{3.317370in}{2.913913in}}%
\pgfpathlineto{\pgfqpoint{3.321785in}{2.763383in}}%
\pgfpathlineto{\pgfqpoint{3.326199in}{2.763355in}}%
\pgfpathlineto{\pgfqpoint{3.330613in}{2.729038in}}%
\pgfpathlineto{\pgfqpoint{3.335028in}{2.414374in}}%
\pgfpathlineto{\pgfqpoint{3.339442in}{2.748395in}}%
\pgfpathlineto{\pgfqpoint{3.343857in}{2.781663in}}%
\pgfpathlineto{\pgfqpoint{3.348271in}{2.684494in}}%
\pgfpathlineto{\pgfqpoint{3.352686in}{3.154950in}}%
\pgfpathlineto{\pgfqpoint{3.357100in}{2.995377in}}%
\pgfpathlineto{\pgfqpoint{3.361515in}{2.987642in}}%
\pgfpathlineto{\pgfqpoint{3.365929in}{3.148650in}}%
\pgfpathlineto{\pgfqpoint{3.370343in}{2.892902in}}%
\pgfpathlineto{\pgfqpoint{3.374758in}{2.987761in}}%
\pgfpathlineto{\pgfqpoint{3.379172in}{2.750728in}}%
\pgfpathlineto{\pgfqpoint{3.383587in}{2.750716in}}%
\pgfpathlineto{\pgfqpoint{3.388001in}{2.588549in}}%
\pgfpathlineto{\pgfqpoint{3.392416in}{2.495266in}}%
\pgfpathlineto{\pgfqpoint{3.396830in}{2.754087in}}%
\pgfpathlineto{\pgfqpoint{3.401245in}{2.940029in}}%
\pgfpathlineto{\pgfqpoint{3.405659in}{3.036559in}}%
\pgfpathlineto{\pgfqpoint{3.410073in}{2.941222in}}%
\pgfpathlineto{\pgfqpoint{3.414488in}{2.926892in}}%
\pgfpathlineto{\pgfqpoint{3.418902in}{2.952895in}}%
\pgfpathlineto{\pgfqpoint{3.423317in}{3.082135in}}%
\pgfpathlineto{\pgfqpoint{3.427731in}{2.955143in}}%
\pgfpathlineto{\pgfqpoint{3.432146in}{2.683965in}}%
\pgfpathlineto{\pgfqpoint{3.436560in}{2.911038in}}%
\pgfpathlineto{\pgfqpoint{3.440975in}{2.380788in}}%
\pgfpathlineto{\pgfqpoint{3.445389in}{3.104936in}}%
\pgfpathlineto{\pgfqpoint{3.449803in}{2.821801in}}%
\pgfpathlineto{\pgfqpoint{3.454218in}{2.965584in}}%
\pgfpathlineto{\pgfqpoint{3.458632in}{2.833229in}}%
\pgfpathlineto{\pgfqpoint{3.467461in}{3.117544in}}%
\pgfpathlineto{\pgfqpoint{3.476290in}{2.869822in}}%
\pgfpathlineto{\pgfqpoint{3.480705in}{2.646280in}}%
\pgfpathlineto{\pgfqpoint{3.485119in}{2.901745in}}%
\pgfpathlineto{\pgfqpoint{3.489533in}{2.325035in}}%
\pgfpathlineto{\pgfqpoint{3.493948in}{2.404048in}}%
\pgfpathlineto{\pgfqpoint{3.498362in}{2.274805in}}%
\pgfpathlineto{\pgfqpoint{3.502777in}{2.894728in}}%
\pgfpathlineto{\pgfqpoint{3.507191in}{2.833294in}}%
\pgfpathlineto{\pgfqpoint{3.511606in}{2.651677in}}%
\pgfpathlineto{\pgfqpoint{3.516020in}{3.055198in}}%
\pgfpathlineto{\pgfqpoint{3.520435in}{2.989052in}}%
\pgfpathlineto{\pgfqpoint{3.524849in}{3.003016in}}%
\pgfpathlineto{\pgfqpoint{3.529263in}{2.995104in}}%
\pgfpathlineto{\pgfqpoint{3.533678in}{2.705852in}}%
\pgfpathlineto{\pgfqpoint{3.538092in}{2.973091in}}%
\pgfpathlineto{\pgfqpoint{3.542507in}{3.008924in}}%
\pgfpathlineto{\pgfqpoint{3.546921in}{3.055688in}}%
\pgfpathlineto{\pgfqpoint{3.551336in}{2.153614in}}%
\pgfpathlineto{\pgfqpoint{3.555750in}{2.934793in}}%
\pgfpathlineto{\pgfqpoint{3.560165in}{2.885603in}}%
\pgfpathlineto{\pgfqpoint{3.564579in}{2.858458in}}%
\pgfpathlineto{\pgfqpoint{3.568994in}{2.859601in}}%
\pgfpathlineto{\pgfqpoint{3.573408in}{2.877320in}}%
\pgfpathlineto{\pgfqpoint{3.577822in}{3.071877in}}%
\pgfpathlineto{\pgfqpoint{3.582237in}{2.962087in}}%
\pgfpathlineto{\pgfqpoint{3.586651in}{2.788094in}}%
\pgfpathlineto{\pgfqpoint{3.591066in}{2.664197in}}%
\pgfpathlineto{\pgfqpoint{3.595480in}{3.063957in}}%
\pgfpathlineto{\pgfqpoint{3.599895in}{2.531521in}}%
\pgfpathlineto{\pgfqpoint{3.604309in}{2.898127in}}%
\pgfpathlineto{\pgfqpoint{3.613138in}{2.504359in}}%
\pgfpathlineto{\pgfqpoint{3.617552in}{2.840094in}}%
\pgfpathlineto{\pgfqpoint{3.621967in}{2.440469in}}%
\pgfpathlineto{\pgfqpoint{3.626381in}{3.142874in}}%
\pgfpathlineto{\pgfqpoint{3.630796in}{3.144275in}}%
\pgfpathlineto{\pgfqpoint{3.635210in}{3.117797in}}%
\pgfpathlineto{\pgfqpoint{3.639625in}{3.063316in}}%
\pgfpathlineto{\pgfqpoint{3.644039in}{2.549297in}}%
\pgfpathlineto{\pgfqpoint{3.648454in}{2.921563in}}%
\pgfpathlineto{\pgfqpoint{3.652868in}{2.949840in}}%
\pgfpathlineto{\pgfqpoint{3.657282in}{2.809576in}}%
\pgfpathlineto{\pgfqpoint{3.661697in}{2.964366in}}%
\pgfpathlineto{\pgfqpoint{3.666111in}{2.744197in}}%
\pgfpathlineto{\pgfqpoint{3.670526in}{2.787194in}}%
\pgfpathlineto{\pgfqpoint{3.674940in}{2.911437in}}%
\pgfpathlineto{\pgfqpoint{3.679355in}{2.465651in}}%
\pgfpathlineto{\pgfqpoint{3.683769in}{3.004439in}}%
\pgfpathlineto{\pgfqpoint{3.688184in}{3.129105in}}%
\pgfpathlineto{\pgfqpoint{3.692598in}{3.091338in}}%
\pgfpathlineto{\pgfqpoint{3.697012in}{3.000241in}}%
\pgfpathlineto{\pgfqpoint{3.701427in}{3.058369in}}%
\pgfpathlineto{\pgfqpoint{3.705841in}{2.994060in}}%
\pgfpathlineto{\pgfqpoint{3.710256in}{2.682288in}}%
\pgfpathlineto{\pgfqpoint{3.714670in}{2.608449in}}%
\pgfpathlineto{\pgfqpoint{3.719085in}{2.620331in}}%
\pgfpathlineto{\pgfqpoint{3.723499in}{2.537635in}}%
\pgfpathlineto{\pgfqpoint{3.727914in}{2.405103in}}%
\pgfpathlineto{\pgfqpoint{3.732328in}{2.631641in}}%
\pgfpathlineto{\pgfqpoint{3.736742in}{2.917340in}}%
\pgfpathlineto{\pgfqpoint{3.745571in}{2.803850in}}%
\pgfpathlineto{\pgfqpoint{3.749986in}{2.797289in}}%
\pgfpathlineto{\pgfqpoint{3.754400in}{2.889219in}}%
\pgfpathlineto{\pgfqpoint{3.758815in}{2.870554in}}%
\pgfpathlineto{\pgfqpoint{3.763229in}{2.948219in}}%
\pgfpathlineto{\pgfqpoint{3.767644in}{2.704718in}}%
\pgfpathlineto{\pgfqpoint{3.772058in}{2.601024in}}%
\pgfpathlineto{\pgfqpoint{3.776472in}{2.423023in}}%
\pgfpathlineto{\pgfqpoint{3.780887in}{2.749220in}}%
\pgfpathlineto{\pgfqpoint{3.785301in}{2.941523in}}%
\pgfpathlineto{\pgfqpoint{3.789716in}{2.587671in}}%
\pgfpathlineto{\pgfqpoint{3.794130in}{3.011282in}}%
\pgfpathlineto{\pgfqpoint{3.798545in}{3.094774in}}%
\pgfpathlineto{\pgfqpoint{3.802959in}{2.898231in}}%
\pgfpathlineto{\pgfqpoint{3.807374in}{2.863154in}}%
\pgfpathlineto{\pgfqpoint{3.811788in}{2.642198in}}%
\pgfpathlineto{\pgfqpoint{3.816202in}{2.857668in}}%
\pgfpathlineto{\pgfqpoint{3.820617in}{2.501934in}}%
\pgfpathlineto{\pgfqpoint{3.825031in}{2.305419in}}%
\pgfpathlineto{\pgfqpoint{3.829446in}{2.547924in}}%
\pgfpathlineto{\pgfqpoint{3.833860in}{2.412365in}}%
\pgfpathlineto{\pgfqpoint{3.838275in}{2.747498in}}%
\pgfpathlineto{\pgfqpoint{3.847104in}{2.920148in}}%
\pgfpathlineto{\pgfqpoint{3.851518in}{2.684455in}}%
\pgfpathlineto{\pgfqpoint{3.855932in}{2.806306in}}%
\pgfpathlineto{\pgfqpoint{3.860347in}{2.976208in}}%
\pgfpathlineto{\pgfqpoint{3.864761in}{2.932902in}}%
\pgfpathlineto{\pgfqpoint{3.869176in}{2.945248in}}%
\pgfpathlineto{\pgfqpoint{3.873590in}{3.008049in}}%
\pgfpathlineto{\pgfqpoint{3.878005in}{2.915582in}}%
\pgfpathlineto{\pgfqpoint{3.882419in}{2.446465in}}%
\pgfpathlineto{\pgfqpoint{3.886834in}{2.919363in}}%
\pgfpathlineto{\pgfqpoint{3.891248in}{2.411785in}}%
\pgfpathlineto{\pgfqpoint{3.895662in}{2.791960in}}%
\pgfpathlineto{\pgfqpoint{3.900077in}{2.838533in}}%
\pgfpathlineto{\pgfqpoint{3.904491in}{2.932027in}}%
\pgfpathlineto{\pgfqpoint{3.908906in}{2.770141in}}%
\pgfpathlineto{\pgfqpoint{3.913320in}{2.791150in}}%
\pgfpathlineto{\pgfqpoint{3.917735in}{2.867636in}}%
\pgfpathlineto{\pgfqpoint{3.922149in}{2.528351in}}%
\pgfpathlineto{\pgfqpoint{3.926564in}{2.818576in}}%
\pgfpathlineto{\pgfqpoint{3.930978in}{2.734564in}}%
\pgfpathlineto{\pgfqpoint{3.935392in}{2.754914in}}%
\pgfpathlineto{\pgfqpoint{3.939807in}{2.674666in}}%
\pgfpathlineto{\pgfqpoint{3.944221in}{2.458791in}}%
\pgfpathlineto{\pgfqpoint{3.957465in}{3.115262in}}%
\pgfpathlineto{\pgfqpoint{3.961879in}{2.977435in}}%
\pgfpathlineto{\pgfqpoint{3.966294in}{3.088927in}}%
\pgfpathlineto{\pgfqpoint{3.970708in}{3.103102in}}%
\pgfpathlineto{\pgfqpoint{3.975122in}{3.036708in}}%
\pgfpathlineto{\pgfqpoint{3.979537in}{2.944103in}}%
\pgfpathlineto{\pgfqpoint{3.983951in}{3.038990in}}%
\pgfpathlineto{\pgfqpoint{3.988366in}{2.328665in}}%
\pgfpathlineto{\pgfqpoint{3.992780in}{2.555695in}}%
\pgfpathlineto{\pgfqpoint{3.997195in}{2.523261in}}%
\pgfpathlineto{\pgfqpoint{4.001609in}{2.772085in}}%
\pgfpathlineto{\pgfqpoint{4.006024in}{2.725085in}}%
\pgfpathlineto{\pgfqpoint{4.010438in}{2.799078in}}%
\pgfpathlineto{\pgfqpoint{4.014852in}{3.007976in}}%
\pgfpathlineto{\pgfqpoint{4.019267in}{2.867282in}}%
\pgfpathlineto{\pgfqpoint{4.023681in}{3.062044in}}%
\pgfpathlineto{\pgfqpoint{4.028096in}{3.137115in}}%
\pgfpathlineto{\pgfqpoint{4.032510in}{2.908044in}}%
\pgfpathlineto{\pgfqpoint{4.036925in}{2.996553in}}%
\pgfpathlineto{\pgfqpoint{4.041339in}{2.897733in}}%
\pgfpathlineto{\pgfqpoint{4.045754in}{2.686753in}}%
\pgfpathlineto{\pgfqpoint{4.050168in}{2.100050in}}%
\pgfpathlineto{\pgfqpoint{4.054582in}{2.840981in}}%
\pgfpathlineto{\pgfqpoint{4.058997in}{2.937485in}}%
\pgfpathlineto{\pgfqpoint{4.063411in}{2.848180in}}%
\pgfpathlineto{\pgfqpoint{4.072240in}{3.072097in}}%
\pgfpathlineto{\pgfqpoint{4.076655in}{3.020460in}}%
\pgfpathlineto{\pgfqpoint{4.081069in}{2.881521in}}%
\pgfpathlineto{\pgfqpoint{4.085484in}{3.008423in}}%
\pgfpathlineto{\pgfqpoint{4.089898in}{2.860265in}}%
\pgfpathlineto{\pgfqpoint{4.094312in}{2.854463in}}%
\pgfpathlineto{\pgfqpoint{4.098727in}{2.250259in}}%
\pgfpathlineto{\pgfqpoint{4.103141in}{2.528089in}}%
\pgfpathlineto{\pgfqpoint{4.107556in}{2.468875in}}%
\pgfpathlineto{\pgfqpoint{4.111970in}{3.073228in}}%
\pgfpathlineto{\pgfqpoint{4.120799in}{2.336309in}}%
\pgfpathlineto{\pgfqpoint{4.125214in}{2.602583in}}%
\pgfpathlineto{\pgfqpoint{4.129628in}{3.173722in}}%
\pgfpathlineto{\pgfqpoint{4.134042in}{3.206013in}}%
\pgfpathlineto{\pgfqpoint{4.138457in}{2.993092in}}%
\pgfpathlineto{\pgfqpoint{4.142871in}{3.075594in}}%
\pgfpathlineto{\pgfqpoint{4.147286in}{3.038503in}}%
\pgfpathlineto{\pgfqpoint{4.151700in}{2.443747in}}%
\pgfpathlineto{\pgfqpoint{4.156115in}{2.622281in}}%
\pgfpathlineto{\pgfqpoint{4.160529in}{2.269206in}}%
\pgfpathlineto{\pgfqpoint{4.164944in}{2.406091in}}%
\pgfpathlineto{\pgfqpoint{4.169358in}{2.831989in}}%
\pgfpathlineto{\pgfqpoint{4.173772in}{2.701212in}}%
\pgfpathlineto{\pgfqpoint{4.178187in}{2.931093in}}%
\pgfpathlineto{\pgfqpoint{4.182601in}{2.996674in}}%
\pgfpathlineto{\pgfqpoint{4.187016in}{2.902102in}}%
\pgfpathlineto{\pgfqpoint{4.191430in}{2.769927in}}%
\pgfpathlineto{\pgfqpoint{4.195845in}{3.087158in}}%
\pgfpathlineto{\pgfqpoint{4.200259in}{2.794546in}}%
\pgfpathlineto{\pgfqpoint{4.204674in}{2.854075in}}%
\pgfpathlineto{\pgfqpoint{4.209088in}{2.984545in}}%
\pgfpathlineto{\pgfqpoint{4.213503in}{2.101552in}}%
\pgfpathlineto{\pgfqpoint{4.217917in}{2.392828in}}%
\pgfpathlineto{\pgfqpoint{4.222331in}{2.827639in}}%
\pgfpathlineto{\pgfqpoint{4.226746in}{2.877726in}}%
\pgfpathlineto{\pgfqpoint{4.231160in}{2.278493in}}%
\pgfpathlineto{\pgfqpoint{4.235575in}{2.326259in}}%
\pgfpathlineto{\pgfqpoint{4.239989in}{3.025397in}}%
\pgfpathlineto{\pgfqpoint{4.244404in}{2.961738in}}%
\pgfpathlineto{\pgfqpoint{4.248818in}{2.751676in}}%
\pgfpathlineto{\pgfqpoint{4.253233in}{3.068599in}}%
\pgfpathlineto{\pgfqpoint{4.257647in}{2.969864in}}%
\pgfpathlineto{\pgfqpoint{4.262061in}{2.790708in}}%
\pgfpathlineto{\pgfqpoint{4.266476in}{2.902189in}}%
\pgfpathlineto{\pgfqpoint{4.270890in}{2.215051in}}%
\pgfpathlineto{\pgfqpoint{4.275305in}{2.543074in}}%
\pgfpathlineto{\pgfqpoint{4.279719in}{2.650239in}}%
\pgfpathlineto{\pgfqpoint{4.284134in}{2.897263in}}%
\pgfpathlineto{\pgfqpoint{4.288548in}{2.749220in}}%
\pgfpathlineto{\pgfqpoint{4.292963in}{3.171876in}}%
\pgfpathlineto{\pgfqpoint{4.297377in}{3.062455in}}%
\pgfpathlineto{\pgfqpoint{4.301791in}{2.904888in}}%
\pgfpathlineto{\pgfqpoint{4.306206in}{2.650585in}}%
\pgfpathlineto{\pgfqpoint{4.310620in}{2.964017in}}%
\pgfpathlineto{\pgfqpoint{4.315035in}{3.002414in}}%
\pgfpathlineto{\pgfqpoint{4.319449in}{2.848060in}}%
\pgfpathlineto{\pgfqpoint{4.323864in}{2.641441in}}%
\pgfpathlineto{\pgfqpoint{4.323864in}{2.641441in}}%
\pgfusepath{stroke}%
\end{pgfscope}%
\begin{pgfscope}%
\pgfpathrectangle{\pgfqpoint{0.625000in}{0.440000in}}{\pgfqpoint{3.875000in}{3.080000in}} %
\pgfusepath{clip}%
\pgfsetrectcap%
\pgfsetroundjoin%
\pgfsetlinewidth{1.505625pt}%
\definecolor{currentstroke}{rgb}{0.941176,1.000000,0.941176}%
\pgfsetstrokecolor{currentstroke}%
\pgfsetdash{}{0pt}%
\pgfpathmoveto{\pgfqpoint{0.801136in}{2.409357in}}%
\pgfpathlineto{\pgfqpoint{0.805551in}{2.592485in}}%
\pgfpathlineto{\pgfqpoint{0.809965in}{2.592919in}}%
\pgfpathlineto{\pgfqpoint{0.814380in}{2.607332in}}%
\pgfpathlineto{\pgfqpoint{0.823209in}{3.019295in}}%
\pgfpathlineto{\pgfqpoint{0.827623in}{2.923564in}}%
\pgfpathlineto{\pgfqpoint{0.832037in}{2.882402in}}%
\pgfpathlineto{\pgfqpoint{0.836452in}{2.390495in}}%
\pgfpathlineto{\pgfqpoint{0.840866in}{2.558309in}}%
\pgfpathlineto{\pgfqpoint{0.845281in}{2.533347in}}%
\pgfpathlineto{\pgfqpoint{0.849695in}{3.085922in}}%
\pgfpathlineto{\pgfqpoint{0.854110in}{2.630263in}}%
\pgfpathlineto{\pgfqpoint{0.862939in}{2.361743in}}%
\pgfpathlineto{\pgfqpoint{0.867353in}{2.829268in}}%
\pgfpathlineto{\pgfqpoint{0.871767in}{2.958913in}}%
\pgfpathlineto{\pgfqpoint{0.876182in}{2.771143in}}%
\pgfpathlineto{\pgfqpoint{0.880596in}{3.065389in}}%
\pgfpathlineto{\pgfqpoint{0.885011in}{2.803127in}}%
\pgfpathlineto{\pgfqpoint{0.889425in}{2.782102in}}%
\pgfpathlineto{\pgfqpoint{0.893840in}{2.891211in}}%
\pgfpathlineto{\pgfqpoint{0.898254in}{2.443159in}}%
\pgfpathlineto{\pgfqpoint{0.902669in}{2.310453in}}%
\pgfpathlineto{\pgfqpoint{0.907083in}{2.433244in}}%
\pgfpathlineto{\pgfqpoint{0.911497in}{2.466573in}}%
\pgfpathlineto{\pgfqpoint{0.915912in}{2.307366in}}%
\pgfpathlineto{\pgfqpoint{0.920326in}{2.661502in}}%
\pgfpathlineto{\pgfqpoint{0.924741in}{2.869071in}}%
\pgfpathlineto{\pgfqpoint{0.929155in}{2.929017in}}%
\pgfpathlineto{\pgfqpoint{0.933570in}{2.897198in}}%
\pgfpathlineto{\pgfqpoint{0.937984in}{2.774983in}}%
\pgfpathlineto{\pgfqpoint{0.942399in}{3.088418in}}%
\pgfpathlineto{\pgfqpoint{0.946813in}{2.562245in}}%
\pgfpathlineto{\pgfqpoint{0.951228in}{3.150370in}}%
\pgfpathlineto{\pgfqpoint{0.955642in}{2.795879in}}%
\pgfpathlineto{\pgfqpoint{0.960056in}{2.596714in}}%
\pgfpathlineto{\pgfqpoint{0.964471in}{2.868024in}}%
\pgfpathlineto{\pgfqpoint{0.968885in}{2.994595in}}%
\pgfpathlineto{\pgfqpoint{0.973300in}{2.538657in}}%
\pgfpathlineto{\pgfqpoint{0.977714in}{2.839909in}}%
\pgfpathlineto{\pgfqpoint{0.982129in}{3.011513in}}%
\pgfpathlineto{\pgfqpoint{0.986543in}{2.940693in}}%
\pgfpathlineto{\pgfqpoint{0.995372in}{3.047928in}}%
\pgfpathlineto{\pgfqpoint{0.999786in}{3.138758in}}%
\pgfpathlineto{\pgfqpoint{1.004201in}{3.009501in}}%
\pgfpathlineto{\pgfqpoint{1.008615in}{2.193305in}}%
\pgfpathlineto{\pgfqpoint{1.013030in}{2.643588in}}%
\pgfpathlineto{\pgfqpoint{1.017444in}{2.845178in}}%
\pgfpathlineto{\pgfqpoint{1.021859in}{3.195296in}}%
\pgfpathlineto{\pgfqpoint{1.026273in}{3.228392in}}%
\pgfpathlineto{\pgfqpoint{1.030688in}{2.170535in}}%
\pgfpathlineto{\pgfqpoint{1.035102in}{2.070105in}}%
\pgfpathlineto{\pgfqpoint{1.039516in}{3.050261in}}%
\pgfpathlineto{\pgfqpoint{1.043931in}{3.175441in}}%
\pgfpathlineto{\pgfqpoint{1.048345in}{2.901607in}}%
\pgfpathlineto{\pgfqpoint{1.052760in}{2.476449in}}%
\pgfpathlineto{\pgfqpoint{1.057174in}{3.116314in}}%
\pgfpathlineto{\pgfqpoint{1.061589in}{2.887235in}}%
\pgfpathlineto{\pgfqpoint{1.066003in}{2.874611in}}%
\pgfpathlineto{\pgfqpoint{1.070418in}{3.047813in}}%
\pgfpathlineto{\pgfqpoint{1.079246in}{2.688264in}}%
\pgfpathlineto{\pgfqpoint{1.083661in}{2.730287in}}%
\pgfpathlineto{\pgfqpoint{1.088075in}{2.714239in}}%
\pgfpathlineto{\pgfqpoint{1.092490in}{3.106022in}}%
\pgfpathlineto{\pgfqpoint{1.096904in}{2.896163in}}%
\pgfpathlineto{\pgfqpoint{1.101319in}{2.793122in}}%
\pgfpathlineto{\pgfqpoint{1.105733in}{2.333495in}}%
\pgfpathlineto{\pgfqpoint{1.110148in}{3.139017in}}%
\pgfpathlineto{\pgfqpoint{1.114562in}{2.678608in}}%
\pgfpathlineto{\pgfqpoint{1.123391in}{2.377457in}}%
\pgfpathlineto{\pgfqpoint{1.127805in}{3.051924in}}%
\pgfpathlineto{\pgfqpoint{1.132220in}{2.618212in}}%
\pgfpathlineto{\pgfqpoint{1.136634in}{2.701139in}}%
\pgfpathlineto{\pgfqpoint{1.141049in}{2.720136in}}%
\pgfpathlineto{\pgfqpoint{1.145463in}{1.456791in}}%
\pgfpathlineto{\pgfqpoint{1.149878in}{3.017925in}}%
\pgfpathlineto{\pgfqpoint{1.154292in}{3.233192in}}%
\pgfpathlineto{\pgfqpoint{1.158706in}{3.009397in}}%
\pgfpathlineto{\pgfqpoint{1.163121in}{3.023141in}}%
\pgfpathlineto{\pgfqpoint{1.167535in}{2.575793in}}%
\pgfpathlineto{\pgfqpoint{1.171950in}{2.427834in}}%
\pgfpathlineto{\pgfqpoint{1.176364in}{2.622179in}}%
\pgfpathlineto{\pgfqpoint{1.180779in}{2.688948in}}%
\pgfpathlineto{\pgfqpoint{1.185193in}{3.019891in}}%
\pgfpathlineto{\pgfqpoint{1.189608in}{2.732108in}}%
\pgfpathlineto{\pgfqpoint{1.194022in}{2.215583in}}%
\pgfpathlineto{\pgfqpoint{1.198436in}{3.006617in}}%
\pgfpathlineto{\pgfqpoint{1.202851in}{3.154086in}}%
\pgfpathlineto{\pgfqpoint{1.207265in}{2.536634in}}%
\pgfpathlineto{\pgfqpoint{1.211680in}{2.924636in}}%
\pgfpathlineto{\pgfqpoint{1.216094in}{2.508698in}}%
\pgfpathlineto{\pgfqpoint{1.220509in}{3.187303in}}%
\pgfpathlineto{\pgfqpoint{1.224923in}{3.022764in}}%
\pgfpathlineto{\pgfqpoint{1.229338in}{3.166044in}}%
\pgfpathlineto{\pgfqpoint{1.233752in}{2.992892in}}%
\pgfpathlineto{\pgfqpoint{1.238166in}{2.951044in}}%
\pgfpathlineto{\pgfqpoint{1.242581in}{3.085143in}}%
\pgfpathlineto{\pgfqpoint{1.246995in}{2.631391in}}%
\pgfpathlineto{\pgfqpoint{1.251410in}{2.719359in}}%
\pgfpathlineto{\pgfqpoint{1.255824in}{2.273952in}}%
\pgfpathlineto{\pgfqpoint{1.260239in}{2.938816in}}%
\pgfpathlineto{\pgfqpoint{1.264653in}{2.635440in}}%
\pgfpathlineto{\pgfqpoint{1.269068in}{3.021661in}}%
\pgfpathlineto{\pgfqpoint{1.273482in}{2.695017in}}%
\pgfpathlineto{\pgfqpoint{1.277896in}{2.664582in}}%
\pgfpathlineto{\pgfqpoint{1.282311in}{2.736235in}}%
\pgfpathlineto{\pgfqpoint{1.286725in}{3.245780in}}%
\pgfpathlineto{\pgfqpoint{1.291140in}{2.930640in}}%
\pgfpathlineto{\pgfqpoint{1.295554in}{3.079254in}}%
\pgfpathlineto{\pgfqpoint{1.299969in}{2.207283in}}%
\pgfpathlineto{\pgfqpoint{1.304383in}{2.130363in}}%
\pgfpathlineto{\pgfqpoint{1.313212in}{2.965559in}}%
\pgfpathlineto{\pgfqpoint{1.317626in}{3.000177in}}%
\pgfpathlineto{\pgfqpoint{1.322041in}{3.014627in}}%
\pgfpathlineto{\pgfqpoint{1.326455in}{2.493600in}}%
\pgfpathlineto{\pgfqpoint{1.330870in}{3.126466in}}%
\pgfpathlineto{\pgfqpoint{1.335284in}{3.031852in}}%
\pgfpathlineto{\pgfqpoint{1.339699in}{2.773838in}}%
\pgfpathlineto{\pgfqpoint{1.344113in}{3.155718in}}%
\pgfpathlineto{\pgfqpoint{1.348528in}{3.006789in}}%
\pgfpathlineto{\pgfqpoint{1.352942in}{2.195044in}}%
\pgfpathlineto{\pgfqpoint{1.357356in}{3.269594in}}%
\pgfpathlineto{\pgfqpoint{1.361771in}{2.191144in}}%
\pgfpathlineto{\pgfqpoint{1.366185in}{2.151712in}}%
\pgfpathlineto{\pgfqpoint{1.370600in}{2.729950in}}%
\pgfpathlineto{\pgfqpoint{1.375014in}{3.049996in}}%
\pgfpathlineto{\pgfqpoint{1.379429in}{3.072339in}}%
\pgfpathlineto{\pgfqpoint{1.383843in}{2.480990in}}%
\pgfpathlineto{\pgfqpoint{1.388258in}{2.578389in}}%
\pgfpathlineto{\pgfqpoint{1.392672in}{3.143127in}}%
\pgfpathlineto{\pgfqpoint{1.397086in}{3.099475in}}%
\pgfpathlineto{\pgfqpoint{1.401501in}{2.971811in}}%
\pgfpathlineto{\pgfqpoint{1.405915in}{3.113653in}}%
\pgfpathlineto{\pgfqpoint{1.410330in}{2.435284in}}%
\pgfpathlineto{\pgfqpoint{1.414744in}{2.696449in}}%
\pgfpathlineto{\pgfqpoint{1.419159in}{2.050236in}}%
\pgfpathlineto{\pgfqpoint{1.423573in}{3.042481in}}%
\pgfpathlineto{\pgfqpoint{1.427988in}{3.074207in}}%
\pgfpathlineto{\pgfqpoint{1.432402in}{3.073374in}}%
\pgfpathlineto{\pgfqpoint{1.436816in}{1.932953in}}%
\pgfpathlineto{\pgfqpoint{1.441231in}{2.875354in}}%
\pgfpathlineto{\pgfqpoint{1.445645in}{2.711797in}}%
\pgfpathlineto{\pgfqpoint{1.450060in}{2.871488in}}%
\pgfpathlineto{\pgfqpoint{1.454474in}{2.848119in}}%
\pgfpathlineto{\pgfqpoint{1.458889in}{3.084240in}}%
\pgfpathlineto{\pgfqpoint{1.463303in}{2.650760in}}%
\pgfpathlineto{\pgfqpoint{1.467718in}{2.671895in}}%
\pgfpathlineto{\pgfqpoint{1.472132in}{2.714354in}}%
\pgfpathlineto{\pgfqpoint{1.476546in}{1.106800in}}%
\pgfpathlineto{\pgfqpoint{1.480961in}{2.857620in}}%
\pgfpathlineto{\pgfqpoint{1.485375in}{3.081573in}}%
\pgfpathlineto{\pgfqpoint{1.489790in}{3.110234in}}%
\pgfpathlineto{\pgfqpoint{1.494204in}{2.689423in}}%
\pgfpathlineto{\pgfqpoint{1.498619in}{2.629692in}}%
\pgfpathlineto{\pgfqpoint{1.503033in}{3.164370in}}%
\pgfpathlineto{\pgfqpoint{1.507448in}{2.682702in}}%
\pgfpathlineto{\pgfqpoint{1.511862in}{2.975614in}}%
\pgfpathlineto{\pgfqpoint{1.520691in}{3.296311in}}%
\pgfpathlineto{\pgfqpoint{1.525105in}{2.195699in}}%
\pgfpathlineto{\pgfqpoint{1.529520in}{2.092147in}}%
\pgfpathlineto{\pgfqpoint{1.533934in}{2.803895in}}%
\pgfpathlineto{\pgfqpoint{1.538349in}{3.175449in}}%
\pgfpathlineto{\pgfqpoint{1.542763in}{3.033267in}}%
\pgfpathlineto{\pgfqpoint{1.547178in}{2.470234in}}%
\pgfpathlineto{\pgfqpoint{1.551592in}{2.905127in}}%
\pgfpathlineto{\pgfqpoint{1.556006in}{2.713544in}}%
\pgfpathlineto{\pgfqpoint{1.560421in}{2.813889in}}%
\pgfpathlineto{\pgfqpoint{1.564835in}{3.044107in}}%
\pgfpathlineto{\pgfqpoint{1.569250in}{3.112963in}}%
\pgfpathlineto{\pgfqpoint{1.573664in}{3.285690in}}%
\pgfpathlineto{\pgfqpoint{1.578079in}{2.731911in}}%
\pgfpathlineto{\pgfqpoint{1.582493in}{2.719745in}}%
\pgfpathlineto{\pgfqpoint{1.586908in}{2.089085in}}%
\pgfpathlineto{\pgfqpoint{1.591322in}{2.496026in}}%
\pgfpathlineto{\pgfqpoint{1.595737in}{3.161300in}}%
\pgfpathlineto{\pgfqpoint{1.600151in}{3.045537in}}%
\pgfpathlineto{\pgfqpoint{1.604565in}{3.049411in}}%
\pgfpathlineto{\pgfqpoint{1.608980in}{2.690644in}}%
\pgfpathlineto{\pgfqpoint{1.613394in}{3.232216in}}%
\pgfpathlineto{\pgfqpoint{1.622223in}{1.957589in}}%
\pgfpathlineto{\pgfqpoint{1.626638in}{3.099267in}}%
\pgfpathlineto{\pgfqpoint{1.631052in}{3.257855in}}%
\pgfpathlineto{\pgfqpoint{1.635467in}{3.262914in}}%
\pgfpathlineto{\pgfqpoint{1.639881in}{1.648771in}}%
\pgfpathlineto{\pgfqpoint{1.644295in}{2.883479in}}%
\pgfpathlineto{\pgfqpoint{1.648710in}{3.066067in}}%
\pgfpathlineto{\pgfqpoint{1.653124in}{3.156171in}}%
\pgfpathlineto{\pgfqpoint{1.657539in}{3.100106in}}%
\pgfpathlineto{\pgfqpoint{1.661953in}{2.847846in}}%
\pgfpathlineto{\pgfqpoint{1.666368in}{2.757444in}}%
\pgfpathlineto{\pgfqpoint{1.675197in}{3.191779in}}%
\pgfpathlineto{\pgfqpoint{1.679611in}{3.061031in}}%
\pgfpathlineto{\pgfqpoint{1.688440in}{2.064866in}}%
\pgfpathlineto{\pgfqpoint{1.692854in}{2.773191in}}%
\pgfpathlineto{\pgfqpoint{1.697269in}{2.213574in}}%
\pgfpathlineto{\pgfqpoint{1.701683in}{2.529147in}}%
\pgfpathlineto{\pgfqpoint{1.706098in}{3.199100in}}%
\pgfpathlineto{\pgfqpoint{1.710512in}{3.165917in}}%
\pgfpathlineto{\pgfqpoint{1.714927in}{2.817586in}}%
\pgfpathlineto{\pgfqpoint{1.719341in}{2.723872in}}%
\pgfpathlineto{\pgfqpoint{1.723755in}{2.582174in}}%
\pgfpathlineto{\pgfqpoint{1.728170in}{2.840806in}}%
\pgfpathlineto{\pgfqpoint{1.732584in}{2.130433in}}%
\pgfpathlineto{\pgfqpoint{1.741413in}{3.170416in}}%
\pgfpathlineto{\pgfqpoint{1.745828in}{2.214834in}}%
\pgfpathlineto{\pgfqpoint{1.754657in}{3.028971in}}%
\pgfpathlineto{\pgfqpoint{1.759071in}{3.005134in}}%
\pgfpathlineto{\pgfqpoint{1.763485in}{2.647881in}}%
\pgfpathlineto{\pgfqpoint{1.767900in}{3.051321in}}%
\pgfpathlineto{\pgfqpoint{1.772314in}{2.463442in}}%
\pgfpathlineto{\pgfqpoint{1.776729in}{3.275693in}}%
\pgfpathlineto{\pgfqpoint{1.781143in}{2.742416in}}%
\pgfpathlineto{\pgfqpoint{1.785558in}{3.117409in}}%
\pgfpathlineto{\pgfqpoint{1.789972in}{3.110375in}}%
\pgfpathlineto{\pgfqpoint{1.794387in}{3.134839in}}%
\pgfpathlineto{\pgfqpoint{1.798801in}{2.575190in}}%
\pgfpathlineto{\pgfqpoint{1.803215in}{2.738044in}}%
\pgfpathlineto{\pgfqpoint{1.807630in}{2.090537in}}%
\pgfpathlineto{\pgfqpoint{1.812044in}{2.912239in}}%
\pgfpathlineto{\pgfqpoint{1.816459in}{3.178460in}}%
\pgfpathlineto{\pgfqpoint{1.820873in}{3.169414in}}%
\pgfpathlineto{\pgfqpoint{1.829702in}{2.747681in}}%
\pgfpathlineto{\pgfqpoint{1.834117in}{1.891375in}}%
\pgfpathlineto{\pgfqpoint{1.838531in}{2.420473in}}%
\pgfpathlineto{\pgfqpoint{1.842945in}{3.192488in}}%
\pgfpathlineto{\pgfqpoint{1.847360in}{3.165231in}}%
\pgfpathlineto{\pgfqpoint{1.851774in}{2.699847in}}%
\pgfpathlineto{\pgfqpoint{1.856189in}{1.672562in}}%
\pgfpathlineto{\pgfqpoint{1.860603in}{2.658083in}}%
\pgfpathlineto{\pgfqpoint{1.865018in}{2.962667in}}%
\pgfpathlineto{\pgfqpoint{1.869432in}{2.972213in}}%
\pgfpathlineto{\pgfqpoint{1.873847in}{2.585491in}}%
\pgfpathlineto{\pgfqpoint{1.878261in}{3.107052in}}%
\pgfpathlineto{\pgfqpoint{1.882675in}{2.502784in}}%
\pgfpathlineto{\pgfqpoint{1.887090in}{3.254344in}}%
\pgfpathlineto{\pgfqpoint{1.891504in}{3.240850in}}%
\pgfpathlineto{\pgfqpoint{1.895919in}{3.130796in}}%
\pgfpathlineto{\pgfqpoint{1.900333in}{3.264301in}}%
\pgfpathlineto{\pgfqpoint{1.904748in}{3.139627in}}%
\pgfpathlineto{\pgfqpoint{1.909162in}{2.775352in}}%
\pgfpathlineto{\pgfqpoint{1.913577in}{2.211005in}}%
\pgfpathlineto{\pgfqpoint{1.917991in}{2.128109in}}%
\pgfpathlineto{\pgfqpoint{1.922405in}{3.015148in}}%
\pgfpathlineto{\pgfqpoint{1.926820in}{3.113591in}}%
\pgfpathlineto{\pgfqpoint{1.931234in}{2.646849in}}%
\pgfpathlineto{\pgfqpoint{1.935649in}{3.183960in}}%
\pgfpathlineto{\pgfqpoint{1.940063in}{3.292808in}}%
\pgfpathlineto{\pgfqpoint{1.944478in}{2.666425in}}%
\pgfpathlineto{\pgfqpoint{1.948892in}{2.745317in}}%
\pgfpathlineto{\pgfqpoint{1.953307in}{3.215931in}}%
\pgfpathlineto{\pgfqpoint{1.957721in}{3.199171in}}%
\pgfpathlineto{\pgfqpoint{1.962135in}{2.715190in}}%
\pgfpathlineto{\pgfqpoint{1.966550in}{2.715460in}}%
\pgfpathlineto{\pgfqpoint{1.970964in}{1.646574in}}%
\pgfpathlineto{\pgfqpoint{1.975379in}{2.730329in}}%
\pgfpathlineto{\pgfqpoint{1.979793in}{2.963288in}}%
\pgfpathlineto{\pgfqpoint{1.984208in}{3.120017in}}%
\pgfpathlineto{\pgfqpoint{1.988622in}{3.178761in}}%
\pgfpathlineto{\pgfqpoint{1.993037in}{3.150322in}}%
\pgfpathlineto{\pgfqpoint{1.997451in}{2.633178in}}%
\pgfpathlineto{\pgfqpoint{2.001865in}{2.713172in}}%
\pgfpathlineto{\pgfqpoint{2.006280in}{2.661820in}}%
\pgfpathlineto{\pgfqpoint{2.010694in}{2.909440in}}%
\pgfpathlineto{\pgfqpoint{2.015109in}{3.023600in}}%
\pgfpathlineto{\pgfqpoint{2.019523in}{2.202784in}}%
\pgfpathlineto{\pgfqpoint{2.023938in}{2.724992in}}%
\pgfpathlineto{\pgfqpoint{2.028352in}{2.711929in}}%
\pgfpathlineto{\pgfqpoint{2.032767in}{3.092264in}}%
\pgfpathlineto{\pgfqpoint{2.037181in}{2.842579in}}%
\pgfpathlineto{\pgfqpoint{2.041595in}{3.146670in}}%
\pgfpathlineto{\pgfqpoint{2.046010in}{3.231301in}}%
\pgfpathlineto{\pgfqpoint{2.050424in}{2.720856in}}%
\pgfpathlineto{\pgfqpoint{2.054839in}{2.691674in}}%
\pgfpathlineto{\pgfqpoint{2.059253in}{2.203732in}}%
\pgfpathlineto{\pgfqpoint{2.063668in}{2.622942in}}%
\pgfpathlineto{\pgfqpoint{2.068082in}{3.176026in}}%
\pgfpathlineto{\pgfqpoint{2.072497in}{3.272959in}}%
\pgfpathlineto{\pgfqpoint{2.076911in}{2.213261in}}%
\pgfpathlineto{\pgfqpoint{2.081325in}{2.116565in}}%
\pgfpathlineto{\pgfqpoint{2.085740in}{2.875931in}}%
\pgfpathlineto{\pgfqpoint{2.090154in}{3.073053in}}%
\pgfpathlineto{\pgfqpoint{2.094569in}{3.179991in}}%
\pgfpathlineto{\pgfqpoint{2.098983in}{3.191748in}}%
\pgfpathlineto{\pgfqpoint{2.103398in}{3.025026in}}%
\pgfpathlineto{\pgfqpoint{2.107812in}{2.653942in}}%
\pgfpathlineto{\pgfqpoint{2.112227in}{2.681205in}}%
\pgfpathlineto{\pgfqpoint{2.116641in}{2.661890in}}%
\pgfpathlineto{\pgfqpoint{2.121055in}{2.844135in}}%
\pgfpathlineto{\pgfqpoint{2.125470in}{2.716281in}}%
\pgfpathlineto{\pgfqpoint{2.129884in}{3.257867in}}%
\pgfpathlineto{\pgfqpoint{2.134299in}{2.707683in}}%
\pgfpathlineto{\pgfqpoint{2.138713in}{3.027561in}}%
\pgfpathlineto{\pgfqpoint{2.143128in}{3.108341in}}%
\pgfpathlineto{\pgfqpoint{2.147542in}{3.060902in}}%
\pgfpathlineto{\pgfqpoint{2.151957in}{3.140435in}}%
\pgfpathlineto{\pgfqpoint{2.156371in}{3.181597in}}%
\pgfpathlineto{\pgfqpoint{2.165200in}{2.193952in}}%
\pgfpathlineto{\pgfqpoint{2.169614in}{1.113167in}}%
\pgfpathlineto{\pgfqpoint{2.174029in}{3.037408in}}%
\pgfpathlineto{\pgfqpoint{2.178443in}{2.669568in}}%
\pgfpathlineto{\pgfqpoint{2.182858in}{2.600028in}}%
\pgfpathlineto{\pgfqpoint{2.187272in}{1.644064in}}%
\pgfpathlineto{\pgfqpoint{2.191687in}{3.176977in}}%
\pgfpathlineto{\pgfqpoint{2.196101in}{3.034924in}}%
\pgfpathlineto{\pgfqpoint{2.200515in}{3.189087in}}%
\pgfpathlineto{\pgfqpoint{2.204930in}{2.631439in}}%
\pgfpathlineto{\pgfqpoint{2.209344in}{3.175599in}}%
\pgfpathlineto{\pgfqpoint{2.213759in}{3.113439in}}%
\pgfpathlineto{\pgfqpoint{2.218173in}{2.759104in}}%
\pgfpathlineto{\pgfqpoint{2.222588in}{2.654538in}}%
\pgfpathlineto{\pgfqpoint{2.227002in}{2.987541in}}%
\pgfpathlineto{\pgfqpoint{2.231417in}{3.189028in}}%
\pgfpathlineto{\pgfqpoint{2.235831in}{2.653702in}}%
\pgfpathlineto{\pgfqpoint{2.240246in}{2.725212in}}%
\pgfpathlineto{\pgfqpoint{2.244660in}{2.723462in}}%
\pgfpathlineto{\pgfqpoint{2.249074in}{2.208658in}}%
\pgfpathlineto{\pgfqpoint{2.253489in}{3.218274in}}%
\pgfpathlineto{\pgfqpoint{2.257903in}{3.117462in}}%
\pgfpathlineto{\pgfqpoint{2.262318in}{2.643737in}}%
\pgfpathlineto{\pgfqpoint{2.266732in}{3.110713in}}%
\pgfpathlineto{\pgfqpoint{2.271147in}{2.758988in}}%
\pgfpathlineto{\pgfqpoint{2.275561in}{2.185677in}}%
\pgfpathlineto{\pgfqpoint{2.279976in}{2.112227in}}%
\pgfpathlineto{\pgfqpoint{2.284390in}{2.819319in}}%
\pgfpathlineto{\pgfqpoint{2.288804in}{2.725243in}}%
\pgfpathlineto{\pgfqpoint{2.293219in}{3.170146in}}%
\pgfpathlineto{\pgfqpoint{2.297633in}{2.699380in}}%
\pgfpathlineto{\pgfqpoint{2.302048in}{3.035748in}}%
\pgfpathlineto{\pgfqpoint{2.306462in}{3.169274in}}%
\pgfpathlineto{\pgfqpoint{2.310877in}{3.109584in}}%
\pgfpathlineto{\pgfqpoint{2.315291in}{3.101597in}}%
\pgfpathlineto{\pgfqpoint{2.319706in}{2.972764in}}%
\pgfpathlineto{\pgfqpoint{2.324120in}{3.242268in}}%
\pgfpathlineto{\pgfqpoint{2.328534in}{3.145043in}}%
\pgfpathlineto{\pgfqpoint{2.332949in}{2.264727in}}%
\pgfpathlineto{\pgfqpoint{2.337363in}{2.999749in}}%
\pgfpathlineto{\pgfqpoint{2.341778in}{2.742737in}}%
\pgfpathlineto{\pgfqpoint{2.346192in}{3.211919in}}%
\pgfpathlineto{\pgfqpoint{2.350607in}{3.280018in}}%
\pgfpathlineto{\pgfqpoint{2.359436in}{2.299120in}}%
\pgfpathlineto{\pgfqpoint{2.363850in}{3.224203in}}%
\pgfpathlineto{\pgfqpoint{2.368264in}{3.170354in}}%
\pgfpathlineto{\pgfqpoint{2.372679in}{2.494014in}}%
\pgfpathlineto{\pgfqpoint{2.377093in}{3.101464in}}%
\pgfpathlineto{\pgfqpoint{2.381508in}{2.740090in}}%
\pgfpathlineto{\pgfqpoint{2.385922in}{2.658412in}}%
\pgfpathlineto{\pgfqpoint{2.390337in}{3.200490in}}%
\pgfpathlineto{\pgfqpoint{2.394751in}{2.633268in}}%
\pgfpathlineto{\pgfqpoint{2.399166in}{3.262070in}}%
\pgfpathlineto{\pgfqpoint{2.403580in}{2.347780in}}%
\pgfpathlineto{\pgfqpoint{2.407994in}{2.184752in}}%
\pgfpathlineto{\pgfqpoint{2.412409in}{3.083244in}}%
\pgfpathlineto{\pgfqpoint{2.416823in}{3.181043in}}%
\pgfpathlineto{\pgfqpoint{2.421238in}{2.956201in}}%
\pgfpathlineto{\pgfqpoint{2.425652in}{3.064537in}}%
\pgfpathlineto{\pgfqpoint{2.430067in}{2.858354in}}%
\pgfpathlineto{\pgfqpoint{2.434481in}{3.146847in}}%
\pgfpathlineto{\pgfqpoint{2.438896in}{3.109612in}}%
\pgfpathlineto{\pgfqpoint{2.443310in}{2.317568in}}%
\pgfpathlineto{\pgfqpoint{2.447724in}{2.106136in}}%
\pgfpathlineto{\pgfqpoint{2.452139in}{2.736972in}}%
\pgfpathlineto{\pgfqpoint{2.456553in}{2.731050in}}%
\pgfpathlineto{\pgfqpoint{2.460968in}{2.209719in}}%
\pgfpathlineto{\pgfqpoint{2.465382in}{2.728655in}}%
\pgfpathlineto{\pgfqpoint{2.469797in}{3.042965in}}%
\pgfpathlineto{\pgfqpoint{2.474211in}{3.182655in}}%
\pgfpathlineto{\pgfqpoint{2.478626in}{3.124575in}}%
\pgfpathlineto{\pgfqpoint{2.483040in}{3.083807in}}%
\pgfpathlineto{\pgfqpoint{2.487454in}{3.179045in}}%
\pgfpathlineto{\pgfqpoint{2.491869in}{3.100657in}}%
\pgfpathlineto{\pgfqpoint{2.496283in}{2.191594in}}%
\pgfpathlineto{\pgfqpoint{2.500698in}{2.194427in}}%
\pgfpathlineto{\pgfqpoint{2.505112in}{2.512057in}}%
\pgfpathlineto{\pgfqpoint{2.509527in}{2.733292in}}%
\pgfpathlineto{\pgfqpoint{2.513941in}{2.256714in}}%
\pgfpathlineto{\pgfqpoint{2.518356in}{2.703108in}}%
\pgfpathlineto{\pgfqpoint{2.522770in}{2.565982in}}%
\pgfpathlineto{\pgfqpoint{2.527184in}{3.202907in}}%
\pgfpathlineto{\pgfqpoint{2.531599in}{2.999220in}}%
\pgfpathlineto{\pgfqpoint{2.540428in}{3.152457in}}%
\pgfpathlineto{\pgfqpoint{2.544842in}{3.115121in}}%
\pgfpathlineto{\pgfqpoint{2.549257in}{2.729097in}}%
\pgfpathlineto{\pgfqpoint{2.553671in}{2.732651in}}%
\pgfpathlineto{\pgfqpoint{2.558086in}{2.543797in}}%
\pgfpathlineto{\pgfqpoint{2.562500in}{3.289024in}}%
\pgfpathlineto{\pgfqpoint{2.566914in}{3.219729in}}%
\pgfpathlineto{\pgfqpoint{2.571329in}{2.731171in}}%
\pgfpathlineto{\pgfqpoint{2.575743in}{3.277984in}}%
\pgfpathlineto{\pgfqpoint{2.580158in}{3.016734in}}%
\pgfpathlineto{\pgfqpoint{2.584572in}{3.152513in}}%
\pgfpathlineto{\pgfqpoint{2.588987in}{3.188690in}}%
\pgfpathlineto{\pgfqpoint{2.593401in}{3.154317in}}%
\pgfpathlineto{\pgfqpoint{2.597816in}{2.971763in}}%
\pgfpathlineto{\pgfqpoint{2.602230in}{3.121019in}}%
\pgfpathlineto{\pgfqpoint{2.606644in}{2.733188in}}%
\pgfpathlineto{\pgfqpoint{2.611059in}{2.161250in}}%
\pgfpathlineto{\pgfqpoint{2.615473in}{3.244004in}}%
\pgfpathlineto{\pgfqpoint{2.619888in}{3.272362in}}%
\pgfpathlineto{\pgfqpoint{2.628717in}{2.220698in}}%
\pgfpathlineto{\pgfqpoint{2.633131in}{2.807682in}}%
\pgfpathlineto{\pgfqpoint{2.637546in}{3.059340in}}%
\pgfpathlineto{\pgfqpoint{2.641960in}{3.181794in}}%
\pgfpathlineto{\pgfqpoint{2.646374in}{3.189385in}}%
\pgfpathlineto{\pgfqpoint{2.650789in}{3.028155in}}%
\pgfpathlineto{\pgfqpoint{2.655203in}{3.036491in}}%
\pgfpathlineto{\pgfqpoint{2.659618in}{3.125985in}}%
\pgfpathlineto{\pgfqpoint{2.664032in}{2.726452in}}%
\pgfpathlineto{\pgfqpoint{2.668447in}{2.633614in}}%
\pgfpathlineto{\pgfqpoint{2.672861in}{3.290459in}}%
\pgfpathlineto{\pgfqpoint{2.677276in}{3.221848in}}%
\pgfpathlineto{\pgfqpoint{2.686104in}{2.205094in}}%
\pgfpathlineto{\pgfqpoint{2.690519in}{2.947358in}}%
\pgfpathlineto{\pgfqpoint{2.694933in}{3.142945in}}%
\pgfpathlineto{\pgfqpoint{2.699348in}{3.232905in}}%
\pgfpathlineto{\pgfqpoint{2.703762in}{3.196385in}}%
\pgfpathlineto{\pgfqpoint{2.708177in}{2.930024in}}%
\pgfpathlineto{\pgfqpoint{2.712591in}{3.259718in}}%
\pgfpathlineto{\pgfqpoint{2.717006in}{2.705469in}}%
\pgfpathlineto{\pgfqpoint{2.721420in}{2.700993in}}%
\pgfpathlineto{\pgfqpoint{2.725834in}{2.683304in}}%
\pgfpathlineto{\pgfqpoint{2.730249in}{2.739817in}}%
\pgfpathlineto{\pgfqpoint{2.739078in}{2.148249in}}%
\pgfpathlineto{\pgfqpoint{2.743492in}{3.060133in}}%
\pgfpathlineto{\pgfqpoint{2.747907in}{3.012354in}}%
\pgfpathlineto{\pgfqpoint{2.752321in}{3.189008in}}%
\pgfpathlineto{\pgfqpoint{2.756736in}{3.123877in}}%
\pgfpathlineto{\pgfqpoint{2.761150in}{2.959487in}}%
\pgfpathlineto{\pgfqpoint{2.765564in}{2.917262in}}%
\pgfpathlineto{\pgfqpoint{2.769979in}{3.111551in}}%
\pgfpathlineto{\pgfqpoint{2.774393in}{1.646967in}}%
\pgfpathlineto{\pgfqpoint{2.778808in}{2.684024in}}%
\pgfpathlineto{\pgfqpoint{2.783222in}{2.741485in}}%
\pgfpathlineto{\pgfqpoint{2.787637in}{2.738393in}}%
\pgfpathlineto{\pgfqpoint{2.792051in}{2.192089in}}%
\pgfpathlineto{\pgfqpoint{2.796466in}{2.187396in}}%
\pgfpathlineto{\pgfqpoint{2.800880in}{2.959459in}}%
\pgfpathlineto{\pgfqpoint{2.805294in}{3.198703in}}%
\pgfpathlineto{\pgfqpoint{2.809709in}{3.204117in}}%
\pgfpathlineto{\pgfqpoint{2.814123in}{3.133879in}}%
\pgfpathlineto{\pgfqpoint{2.818538in}{3.142171in}}%
\pgfpathlineto{\pgfqpoint{2.822952in}{3.255889in}}%
\pgfpathlineto{\pgfqpoint{2.827367in}{2.674843in}}%
\pgfpathlineto{\pgfqpoint{2.836196in}{2.286304in}}%
\pgfpathlineto{\pgfqpoint{2.840610in}{3.136836in}}%
\pgfpathlineto{\pgfqpoint{2.845024in}{2.752917in}}%
\pgfpathlineto{\pgfqpoint{2.849439in}{2.716582in}}%
\pgfpathlineto{\pgfqpoint{2.853853in}{2.658080in}}%
\pgfpathlineto{\pgfqpoint{2.858268in}{3.155544in}}%
\pgfpathlineto{\pgfqpoint{2.862682in}{3.170430in}}%
\pgfpathlineto{\pgfqpoint{2.867097in}{3.068062in}}%
\pgfpathlineto{\pgfqpoint{2.871511in}{3.129476in}}%
\pgfpathlineto{\pgfqpoint{2.875926in}{3.171058in}}%
\pgfpathlineto{\pgfqpoint{2.880340in}{3.260593in}}%
\pgfpathlineto{\pgfqpoint{2.884754in}{2.721776in}}%
\pgfpathlineto{\pgfqpoint{2.889169in}{3.224861in}}%
\pgfpathlineto{\pgfqpoint{2.893583in}{2.750258in}}%
\pgfpathlineto{\pgfqpoint{2.897998in}{2.751749in}}%
\pgfpathlineto{\pgfqpoint{2.902412in}{2.204911in}}%
\pgfpathlineto{\pgfqpoint{2.906827in}{2.758791in}}%
\pgfpathlineto{\pgfqpoint{2.911241in}{3.032043in}}%
\pgfpathlineto{\pgfqpoint{2.915656in}{3.117859in}}%
\pgfpathlineto{\pgfqpoint{2.920070in}{3.025791in}}%
\pgfpathlineto{\pgfqpoint{2.924485in}{3.115507in}}%
\pgfpathlineto{\pgfqpoint{2.928899in}{3.124648in}}%
\pgfpathlineto{\pgfqpoint{2.933313in}{3.234030in}}%
\pgfpathlineto{\pgfqpoint{2.937728in}{3.189790in}}%
\pgfpathlineto{\pgfqpoint{2.942142in}{2.950535in}}%
\pgfpathlineto{\pgfqpoint{2.946557in}{3.285715in}}%
\pgfpathlineto{\pgfqpoint{2.950971in}{3.148558in}}%
\pgfpathlineto{\pgfqpoint{2.955386in}{2.164593in}}%
\pgfpathlineto{\pgfqpoint{2.959800in}{2.220520in}}%
\pgfpathlineto{\pgfqpoint{2.964215in}{2.490486in}}%
\pgfpathlineto{\pgfqpoint{2.973043in}{3.151489in}}%
\pgfpathlineto{\pgfqpoint{2.977458in}{3.040135in}}%
\pgfpathlineto{\pgfqpoint{2.981872in}{3.108704in}}%
\pgfpathlineto{\pgfqpoint{2.986287in}{3.046299in}}%
\pgfpathlineto{\pgfqpoint{2.990701in}{3.163095in}}%
\pgfpathlineto{\pgfqpoint{2.995116in}{2.703854in}}%
\pgfpathlineto{\pgfqpoint{2.999530in}{2.743418in}}%
\pgfpathlineto{\pgfqpoint{3.003945in}{3.185975in}}%
\pgfpathlineto{\pgfqpoint{3.008359in}{2.760131in}}%
\pgfpathlineto{\pgfqpoint{3.012773in}{1.665264in}}%
\pgfpathlineto{\pgfqpoint{3.017188in}{2.758229in}}%
\pgfpathlineto{\pgfqpoint{3.021602in}{2.481246in}}%
\pgfpathlineto{\pgfqpoint{3.026017in}{2.666254in}}%
\pgfpathlineto{\pgfqpoint{3.030431in}{3.110608in}}%
\pgfpathlineto{\pgfqpoint{3.034846in}{3.128429in}}%
\pgfpathlineto{\pgfqpoint{3.039260in}{2.977376in}}%
\pgfpathlineto{\pgfqpoint{3.043675in}{3.249975in}}%
\pgfpathlineto{\pgfqpoint{3.048089in}{2.694153in}}%
\pgfpathlineto{\pgfqpoint{3.052503in}{2.633223in}}%
\pgfpathlineto{\pgfqpoint{3.056918in}{2.665812in}}%
\pgfpathlineto{\pgfqpoint{3.061332in}{3.042107in}}%
\pgfpathlineto{\pgfqpoint{3.065747in}{2.721925in}}%
\pgfpathlineto{\pgfqpoint{3.070161in}{2.018010in}}%
\pgfpathlineto{\pgfqpoint{3.074576in}{2.931312in}}%
\pgfpathlineto{\pgfqpoint{3.078990in}{3.061639in}}%
\pgfpathlineto{\pgfqpoint{3.083405in}{3.116494in}}%
\pgfpathlineto{\pgfqpoint{3.087819in}{2.975029in}}%
\pgfpathlineto{\pgfqpoint{3.092233in}{3.070302in}}%
\pgfpathlineto{\pgfqpoint{3.096648in}{3.068709in}}%
\pgfpathlineto{\pgfqpoint{3.101062in}{3.150375in}}%
\pgfpathlineto{\pgfqpoint{3.105477in}{2.159852in}}%
\pgfpathlineto{\pgfqpoint{3.109891in}{2.429331in}}%
\pgfpathlineto{\pgfqpoint{3.114306in}{3.231580in}}%
\pgfpathlineto{\pgfqpoint{3.118720in}{2.450753in}}%
\pgfpathlineto{\pgfqpoint{3.123135in}{2.191068in}}%
\pgfpathlineto{\pgfqpoint{3.127549in}{2.726039in}}%
\pgfpathlineto{\pgfqpoint{3.131963in}{3.068774in}}%
\pgfpathlineto{\pgfqpoint{3.136378in}{2.735293in}}%
\pgfpathlineto{\pgfqpoint{3.140792in}{3.143088in}}%
\pgfpathlineto{\pgfqpoint{3.145207in}{3.150783in}}%
\pgfpathlineto{\pgfqpoint{3.149621in}{3.106982in}}%
\pgfpathlineto{\pgfqpoint{3.154036in}{3.239807in}}%
\pgfpathlineto{\pgfqpoint{3.158450in}{2.683301in}}%
\pgfpathlineto{\pgfqpoint{3.162865in}{2.706817in}}%
\pgfpathlineto{\pgfqpoint{3.167279in}{2.630876in}}%
\pgfpathlineto{\pgfqpoint{3.171693in}{3.086080in}}%
\pgfpathlineto{\pgfqpoint{3.176108in}{2.752331in}}%
\pgfpathlineto{\pgfqpoint{3.180522in}{2.539360in}}%
\pgfpathlineto{\pgfqpoint{3.184937in}{2.554494in}}%
\pgfpathlineto{\pgfqpoint{3.189351in}{3.080166in}}%
\pgfpathlineto{\pgfqpoint{3.193766in}{3.174586in}}%
\pgfpathlineto{\pgfqpoint{3.198180in}{3.097390in}}%
\pgfpathlineto{\pgfqpoint{3.202595in}{3.146245in}}%
\pgfpathlineto{\pgfqpoint{3.207009in}{3.223395in}}%
\pgfpathlineto{\pgfqpoint{3.211423in}{3.266786in}}%
\pgfpathlineto{\pgfqpoint{3.215838in}{3.165430in}}%
\pgfpathlineto{\pgfqpoint{3.220252in}{3.026236in}}%
\pgfpathlineto{\pgfqpoint{3.224667in}{3.114359in}}%
\pgfpathlineto{\pgfqpoint{3.229081in}{3.268657in}}%
\pgfpathlineto{\pgfqpoint{3.233496in}{3.269172in}}%
\pgfpathlineto{\pgfqpoint{3.237910in}{2.719250in}}%
\pgfpathlineto{\pgfqpoint{3.242325in}{3.191025in}}%
\pgfpathlineto{\pgfqpoint{3.246739in}{3.022345in}}%
\pgfpathlineto{\pgfqpoint{3.251153in}{3.022961in}}%
\pgfpathlineto{\pgfqpoint{3.255568in}{3.255743in}}%
\pgfpathlineto{\pgfqpoint{3.259982in}{3.093418in}}%
\pgfpathlineto{\pgfqpoint{3.264397in}{3.285282in}}%
\pgfpathlineto{\pgfqpoint{3.268811in}{3.154503in}}%
\pgfpathlineto{\pgfqpoint{3.273226in}{3.206171in}}%
\pgfpathlineto{\pgfqpoint{3.277640in}{3.135145in}}%
\pgfpathlineto{\pgfqpoint{3.282055in}{3.247485in}}%
\pgfpathlineto{\pgfqpoint{3.286469in}{2.119466in}}%
\pgfpathlineto{\pgfqpoint{3.290883in}{2.734463in}}%
\pgfpathlineto{\pgfqpoint{3.295298in}{2.518360in}}%
\pgfpathlineto{\pgfqpoint{3.299712in}{2.624118in}}%
\pgfpathlineto{\pgfqpoint{3.304127in}{3.209184in}}%
\pgfpathlineto{\pgfqpoint{3.308541in}{3.161466in}}%
\pgfpathlineto{\pgfqpoint{3.312956in}{3.136049in}}%
\pgfpathlineto{\pgfqpoint{3.317370in}{3.073768in}}%
\pgfpathlineto{\pgfqpoint{3.326199in}{3.200859in}}%
\pgfpathlineto{\pgfqpoint{3.330613in}{2.595687in}}%
\pgfpathlineto{\pgfqpoint{3.335028in}{3.154987in}}%
\pgfpathlineto{\pgfqpoint{3.343857in}{2.195719in}}%
\pgfpathlineto{\pgfqpoint{3.348271in}{2.205352in}}%
\pgfpathlineto{\pgfqpoint{3.352686in}{3.218922in}}%
\pgfpathlineto{\pgfqpoint{3.357100in}{3.080945in}}%
\pgfpathlineto{\pgfqpoint{3.361515in}{3.219836in}}%
\pgfpathlineto{\pgfqpoint{3.365929in}{3.273589in}}%
\pgfpathlineto{\pgfqpoint{3.370343in}{2.822192in}}%
\pgfpathlineto{\pgfqpoint{3.374758in}{3.287713in}}%
\pgfpathlineto{\pgfqpoint{3.379172in}{2.972818in}}%
\pgfpathlineto{\pgfqpoint{3.383587in}{3.048367in}}%
\pgfpathlineto{\pgfqpoint{3.388001in}{2.661544in}}%
\pgfpathlineto{\pgfqpoint{3.396830in}{3.187705in}}%
\pgfpathlineto{\pgfqpoint{3.401245in}{2.215529in}}%
\pgfpathlineto{\pgfqpoint{3.405659in}{2.915649in}}%
\pgfpathlineto{\pgfqpoint{3.410073in}{3.134639in}}%
\pgfpathlineto{\pgfqpoint{3.414488in}{3.182030in}}%
\pgfpathlineto{\pgfqpoint{3.418902in}{3.057525in}}%
\pgfpathlineto{\pgfqpoint{3.423317in}{2.641480in}}%
\pgfpathlineto{\pgfqpoint{3.427731in}{2.968133in}}%
\pgfpathlineto{\pgfqpoint{3.432146in}{3.002022in}}%
\pgfpathlineto{\pgfqpoint{3.436560in}{3.141656in}}%
\pgfpathlineto{\pgfqpoint{3.440975in}{2.505133in}}%
\pgfpathlineto{\pgfqpoint{3.445389in}{3.299307in}}%
\pgfpathlineto{\pgfqpoint{3.449803in}{3.180894in}}%
\pgfpathlineto{\pgfqpoint{3.454218in}{2.707298in}}%
\pgfpathlineto{\pgfqpoint{3.458632in}{2.681166in}}%
\pgfpathlineto{\pgfqpoint{3.463047in}{3.169111in}}%
\pgfpathlineto{\pgfqpoint{3.467461in}{3.217886in}}%
\pgfpathlineto{\pgfqpoint{3.471876in}{3.197626in}}%
\pgfpathlineto{\pgfqpoint{3.476290in}{3.162420in}}%
\pgfpathlineto{\pgfqpoint{3.480705in}{3.149815in}}%
\pgfpathlineto{\pgfqpoint{3.485119in}{3.101675in}}%
\pgfpathlineto{\pgfqpoint{3.489533in}{2.313621in}}%
\pgfpathlineto{\pgfqpoint{3.498362in}{2.715409in}}%
\pgfpathlineto{\pgfqpoint{3.502777in}{3.211744in}}%
\pgfpathlineto{\pgfqpoint{3.507191in}{3.278484in}}%
\pgfpathlineto{\pgfqpoint{3.511606in}{2.674782in}}%
\pgfpathlineto{\pgfqpoint{3.516020in}{3.020662in}}%
\pgfpathlineto{\pgfqpoint{3.520435in}{3.084499in}}%
\pgfpathlineto{\pgfqpoint{3.524849in}{3.165779in}}%
\pgfpathlineto{\pgfqpoint{3.529263in}{2.962228in}}%
\pgfpathlineto{\pgfqpoint{3.533678in}{2.541965in}}%
\pgfpathlineto{\pgfqpoint{3.538092in}{3.056028in}}%
\pgfpathlineto{\pgfqpoint{3.542507in}{3.134161in}}%
\pgfpathlineto{\pgfqpoint{3.546921in}{3.190952in}}%
\pgfpathlineto{\pgfqpoint{3.551336in}{2.581484in}}%
\pgfpathlineto{\pgfqpoint{3.555750in}{3.270874in}}%
\pgfpathlineto{\pgfqpoint{3.560165in}{2.729061in}}%
\pgfpathlineto{\pgfqpoint{3.564579in}{2.700261in}}%
\pgfpathlineto{\pgfqpoint{3.568994in}{2.727277in}}%
\pgfpathlineto{\pgfqpoint{3.573408in}{3.123540in}}%
\pgfpathlineto{\pgfqpoint{3.577822in}{3.187379in}}%
\pgfpathlineto{\pgfqpoint{3.582237in}{3.032791in}}%
\pgfpathlineto{\pgfqpoint{3.586651in}{3.032057in}}%
\pgfpathlineto{\pgfqpoint{3.591066in}{3.167158in}}%
\pgfpathlineto{\pgfqpoint{3.595480in}{3.207893in}}%
\pgfpathlineto{\pgfqpoint{3.599895in}{2.469795in}}%
\pgfpathlineto{\pgfqpoint{3.604309in}{2.731415in}}%
\pgfpathlineto{\pgfqpoint{3.608724in}{3.204252in}}%
\pgfpathlineto{\pgfqpoint{3.613138in}{3.220289in}}%
\pgfpathlineto{\pgfqpoint{3.617552in}{3.281546in}}%
\pgfpathlineto{\pgfqpoint{3.621967in}{2.679964in}}%
\pgfpathlineto{\pgfqpoint{3.626381in}{2.894188in}}%
\pgfpathlineto{\pgfqpoint{3.630796in}{2.592021in}}%
\pgfpathlineto{\pgfqpoint{3.639625in}{3.049301in}}%
\pgfpathlineto{\pgfqpoint{3.648454in}{3.235015in}}%
\pgfpathlineto{\pgfqpoint{3.652868in}{3.132824in}}%
\pgfpathlineto{\pgfqpoint{3.657282in}{3.282274in}}%
\pgfpathlineto{\pgfqpoint{3.661697in}{3.218919in}}%
\pgfpathlineto{\pgfqpoint{3.666111in}{3.208891in}}%
\pgfpathlineto{\pgfqpoint{3.674940in}{2.190066in}}%
\pgfpathlineto{\pgfqpoint{3.679355in}{2.165513in}}%
\pgfpathlineto{\pgfqpoint{3.683769in}{3.133519in}}%
\pgfpathlineto{\pgfqpoint{3.688184in}{2.815189in}}%
\pgfpathlineto{\pgfqpoint{3.697012in}{3.149919in}}%
\pgfpathlineto{\pgfqpoint{3.701427in}{3.120597in}}%
\pgfpathlineto{\pgfqpoint{3.705841in}{3.251308in}}%
\pgfpathlineto{\pgfqpoint{3.710256in}{2.891672in}}%
\pgfpathlineto{\pgfqpoint{3.714670in}{3.094495in}}%
\pgfpathlineto{\pgfqpoint{3.719085in}{2.685076in}}%
\pgfpathlineto{\pgfqpoint{3.723499in}{3.193104in}}%
\pgfpathlineto{\pgfqpoint{3.727914in}{2.740216in}}%
\pgfpathlineto{\pgfqpoint{3.732328in}{2.714064in}}%
\pgfpathlineto{\pgfqpoint{3.736742in}{3.095362in}}%
\pgfpathlineto{\pgfqpoint{3.741157in}{2.617954in}}%
\pgfpathlineto{\pgfqpoint{3.745571in}{2.976926in}}%
\pgfpathlineto{\pgfqpoint{3.749986in}{3.152935in}}%
\pgfpathlineto{\pgfqpoint{3.754400in}{3.180348in}}%
\pgfpathlineto{\pgfqpoint{3.758815in}{3.196545in}}%
\pgfpathlineto{\pgfqpoint{3.763229in}{3.128275in}}%
\pgfpathlineto{\pgfqpoint{3.767644in}{3.184233in}}%
\pgfpathlineto{\pgfqpoint{3.772058in}{2.744878in}}%
\pgfpathlineto{\pgfqpoint{3.776472in}{3.122529in}}%
\pgfpathlineto{\pgfqpoint{3.780887in}{2.773078in}}%
\pgfpathlineto{\pgfqpoint{3.785301in}{3.223955in}}%
\pgfpathlineto{\pgfqpoint{3.789716in}{2.722966in}}%
\pgfpathlineto{\pgfqpoint{3.794130in}{3.177987in}}%
\pgfpathlineto{\pgfqpoint{3.798545in}{2.849576in}}%
\pgfpathlineto{\pgfqpoint{3.807374in}{3.207434in}}%
\pgfpathlineto{\pgfqpoint{3.811788in}{2.999282in}}%
\pgfpathlineto{\pgfqpoint{3.816202in}{3.123407in}}%
\pgfpathlineto{\pgfqpoint{3.820617in}{2.405832in}}%
\pgfpathlineto{\pgfqpoint{3.825031in}{2.100455in}}%
\pgfpathlineto{\pgfqpoint{3.829446in}{3.241858in}}%
\pgfpathlineto{\pgfqpoint{3.833860in}{3.178606in}}%
\pgfpathlineto{\pgfqpoint{3.838275in}{2.754864in}}%
\pgfpathlineto{\pgfqpoint{3.842689in}{2.212015in}}%
\pgfpathlineto{\pgfqpoint{3.847104in}{3.134788in}}%
\pgfpathlineto{\pgfqpoint{3.851518in}{3.093547in}}%
\pgfpathlineto{\pgfqpoint{3.860347in}{3.226932in}}%
\pgfpathlineto{\pgfqpoint{3.864761in}{3.169966in}}%
\pgfpathlineto{\pgfqpoint{3.869176in}{3.225941in}}%
\pgfpathlineto{\pgfqpoint{3.873590in}{3.204933in}}%
\pgfpathlineto{\pgfqpoint{3.878005in}{3.192882in}}%
\pgfpathlineto{\pgfqpoint{3.882419in}{2.335563in}}%
\pgfpathlineto{\pgfqpoint{3.886834in}{3.270325in}}%
\pgfpathlineto{\pgfqpoint{3.891248in}{2.200327in}}%
\pgfpathlineto{\pgfqpoint{3.895662in}{2.213917in}}%
\pgfpathlineto{\pgfqpoint{3.900077in}{2.745331in}}%
\pgfpathlineto{\pgfqpoint{3.904491in}{3.120912in}}%
\pgfpathlineto{\pgfqpoint{3.908906in}{3.182368in}}%
\pgfpathlineto{\pgfqpoint{3.913320in}{3.108324in}}%
\pgfpathlineto{\pgfqpoint{3.917735in}{3.054211in}}%
\pgfpathlineto{\pgfqpoint{3.922149in}{3.123936in}}%
\pgfpathlineto{\pgfqpoint{3.926564in}{2.670677in}}%
\pgfpathlineto{\pgfqpoint{3.930978in}{2.739147in}}%
\pgfpathlineto{\pgfqpoint{3.935392in}{2.258180in}}%
\pgfpathlineto{\pgfqpoint{3.939807in}{3.116860in}}%
\pgfpathlineto{\pgfqpoint{3.944221in}{3.217726in}}%
\pgfpathlineto{\pgfqpoint{3.948636in}{3.258081in}}%
\pgfpathlineto{\pgfqpoint{3.953050in}{2.161090in}}%
\pgfpathlineto{\pgfqpoint{3.957465in}{3.170025in}}%
\pgfpathlineto{\pgfqpoint{3.961879in}{3.111149in}}%
\pgfpathlineto{\pgfqpoint{3.966294in}{2.694243in}}%
\pgfpathlineto{\pgfqpoint{3.970708in}{3.228156in}}%
\pgfpathlineto{\pgfqpoint{3.975122in}{3.230069in}}%
\pgfpathlineto{\pgfqpoint{3.979537in}{3.052180in}}%
\pgfpathlineto{\pgfqpoint{3.983951in}{3.257639in}}%
\pgfpathlineto{\pgfqpoint{3.988366in}{2.808746in}}%
\pgfpathlineto{\pgfqpoint{3.992780in}{3.242004in}}%
\pgfpathlineto{\pgfqpoint{3.997195in}{3.194430in}}%
\pgfpathlineto{\pgfqpoint{4.001609in}{3.265252in}}%
\pgfpathlineto{\pgfqpoint{4.006024in}{2.154844in}}%
\pgfpathlineto{\pgfqpoint{4.014852in}{3.156160in}}%
\pgfpathlineto{\pgfqpoint{4.019267in}{3.210205in}}%
\pgfpathlineto{\pgfqpoint{4.023681in}{3.156348in}}%
\pgfpathlineto{\pgfqpoint{4.028096in}{3.059166in}}%
\pgfpathlineto{\pgfqpoint{4.032510in}{3.160310in}}%
\pgfpathlineto{\pgfqpoint{4.036925in}{2.739423in}}%
\pgfpathlineto{\pgfqpoint{4.041339in}{2.998275in}}%
\pgfpathlineto{\pgfqpoint{4.045754in}{2.918269in}}%
\pgfpathlineto{\pgfqpoint{4.050168in}{2.478320in}}%
\pgfpathlineto{\pgfqpoint{4.054582in}{3.237086in}}%
\pgfpathlineto{\pgfqpoint{4.058997in}{2.634745in}}%
\pgfpathlineto{\pgfqpoint{4.063411in}{2.712238in}}%
\pgfpathlineto{\pgfqpoint{4.067826in}{3.141361in}}%
\pgfpathlineto{\pgfqpoint{4.072240in}{3.143434in}}%
\pgfpathlineto{\pgfqpoint{4.076655in}{2.643925in}}%
\pgfpathlineto{\pgfqpoint{4.081069in}{3.086600in}}%
\pgfpathlineto{\pgfqpoint{4.085484in}{3.166801in}}%
\pgfpathlineto{\pgfqpoint{4.089898in}{3.067781in}}%
\pgfpathlineto{\pgfqpoint{4.094312in}{3.280310in}}%
\pgfpathlineto{\pgfqpoint{4.098727in}{2.445582in}}%
\pgfpathlineto{\pgfqpoint{4.103141in}{2.642507in}}%
\pgfpathlineto{\pgfqpoint{4.107556in}{3.172276in}}%
\pgfpathlineto{\pgfqpoint{4.111970in}{3.110887in}}%
\pgfpathlineto{\pgfqpoint{4.116385in}{3.239593in}}%
\pgfpathlineto{\pgfqpoint{4.120799in}{2.160451in}}%
\pgfpathlineto{\pgfqpoint{4.125214in}{2.164463in}}%
\pgfpathlineto{\pgfqpoint{4.129628in}{3.260154in}}%
\pgfpathlineto{\pgfqpoint{4.134042in}{3.132715in}}%
\pgfpathlineto{\pgfqpoint{4.138457in}{3.115476in}}%
\pgfpathlineto{\pgfqpoint{4.142871in}{3.088874in}}%
\pgfpathlineto{\pgfqpoint{4.147286in}{3.258750in}}%
\pgfpathlineto{\pgfqpoint{4.156115in}{2.391063in}}%
\pgfpathlineto{\pgfqpoint{4.160529in}{2.957020in}}%
\pgfpathlineto{\pgfqpoint{4.164944in}{3.166438in}}%
\pgfpathlineto{\pgfqpoint{4.169358in}{2.574729in}}%
\pgfpathlineto{\pgfqpoint{4.173772in}{2.187250in}}%
\pgfpathlineto{\pgfqpoint{4.178187in}{3.086263in}}%
\pgfpathlineto{\pgfqpoint{4.182601in}{3.166401in}}%
\pgfpathlineto{\pgfqpoint{4.187016in}{3.017010in}}%
\pgfpathlineto{\pgfqpoint{4.191430in}{2.996983in}}%
\pgfpathlineto{\pgfqpoint{4.195845in}{3.113827in}}%
\pgfpathlineto{\pgfqpoint{4.200259in}{2.705491in}}%
\pgfpathlineto{\pgfqpoint{4.204674in}{3.255782in}}%
\pgfpathlineto{\pgfqpoint{4.209088in}{2.749540in}}%
\pgfpathlineto{\pgfqpoint{4.213503in}{2.580066in}}%
\pgfpathlineto{\pgfqpoint{4.217917in}{3.149475in}}%
\pgfpathlineto{\pgfqpoint{4.222331in}{3.093356in}}%
\pgfpathlineto{\pgfqpoint{4.226746in}{2.755595in}}%
\pgfpathlineto{\pgfqpoint{4.231160in}{1.632430in}}%
\pgfpathlineto{\pgfqpoint{4.235575in}{2.702070in}}%
\pgfpathlineto{\pgfqpoint{4.239989in}{3.178311in}}%
\pgfpathlineto{\pgfqpoint{4.244404in}{3.134813in}}%
\pgfpathlineto{\pgfqpoint{4.248818in}{2.958238in}}%
\pgfpathlineto{\pgfqpoint{4.253233in}{3.138949in}}%
\pgfpathlineto{\pgfqpoint{4.257647in}{3.247949in}}%
\pgfpathlineto{\pgfqpoint{4.262061in}{2.662391in}}%
\pgfpathlineto{\pgfqpoint{4.266476in}{2.725876in}}%
\pgfpathlineto{\pgfqpoint{4.275305in}{3.220855in}}%
\pgfpathlineto{\pgfqpoint{4.279719in}{2.690028in}}%
\pgfpathlineto{\pgfqpoint{4.284134in}{3.110153in}}%
\pgfpathlineto{\pgfqpoint{4.288548in}{3.198906in}}%
\pgfpathlineto{\pgfqpoint{4.292963in}{3.117012in}}%
\pgfpathlineto{\pgfqpoint{4.297377in}{2.602085in}}%
\pgfpathlineto{\pgfqpoint{4.301791in}{3.097435in}}%
\pgfpathlineto{\pgfqpoint{4.306206in}{3.121238in}}%
\pgfpathlineto{\pgfqpoint{4.310620in}{3.228668in}}%
\pgfpathlineto{\pgfqpoint{4.315035in}{3.171828in}}%
\pgfpathlineto{\pgfqpoint{4.319449in}{2.983352in}}%
\pgfpathlineto{\pgfqpoint{4.323864in}{3.111531in}}%
\pgfpathlineto{\pgfqpoint{4.323864in}{3.111531in}}%
\pgfusepath{stroke}%
\end{pgfscope}%
\begin{pgfscope}%
\pgfpathrectangle{\pgfqpoint{0.625000in}{0.440000in}}{\pgfqpoint{3.875000in}{3.080000in}} %
\pgfusepath{clip}%
\pgfsetrectcap%
\pgfsetroundjoin%
\pgfsetlinewidth{1.505625pt}%
\definecolor{currentstroke}{rgb}{0.000000,0.000000,0.000000}%
\pgfsetstrokecolor{currentstroke}%
\pgfsetdash{}{0pt}%
\pgfpathmoveto{\pgfqpoint{0.840866in}{3.228541in}}%
\pgfpathlineto{\pgfqpoint{0.885011in}{3.221983in}}%
\pgfpathlineto{\pgfqpoint{0.929155in}{3.237567in}}%
\pgfpathlineto{\pgfqpoint{0.973300in}{2.965669in}}%
\pgfpathlineto{\pgfqpoint{1.017444in}{3.244182in}}%
\pgfpathlineto{\pgfqpoint{1.061589in}{3.228007in}}%
\pgfpathlineto{\pgfqpoint{1.105733in}{3.232871in}}%
\pgfpathlineto{\pgfqpoint{1.149878in}{3.214614in}}%
\pgfpathlineto{\pgfqpoint{1.194022in}{3.076916in}}%
\pgfpathlineto{\pgfqpoint{1.238166in}{3.246697in}}%
\pgfpathlineto{\pgfqpoint{1.282311in}{3.230699in}}%
\pgfpathlineto{\pgfqpoint{1.326455in}{3.242795in}}%
\pgfpathlineto{\pgfqpoint{1.370600in}{3.241635in}}%
\pgfpathlineto{\pgfqpoint{1.414744in}{3.079699in}}%
\pgfpathlineto{\pgfqpoint{1.458889in}{3.244601in}}%
\pgfpathlineto{\pgfqpoint{1.503033in}{3.237924in}}%
\pgfpathlineto{\pgfqpoint{1.547178in}{3.258500in}}%
\pgfpathlineto{\pgfqpoint{1.591322in}{3.236284in}}%
\pgfpathlineto{\pgfqpoint{1.635467in}{3.227236in}}%
\pgfpathlineto{\pgfqpoint{1.679611in}{3.265818in}}%
\pgfpathlineto{\pgfqpoint{1.723755in}{3.271465in}}%
\pgfpathlineto{\pgfqpoint{1.767900in}{3.300605in}}%
\pgfpathlineto{\pgfqpoint{1.812044in}{3.294659in}}%
\pgfpathlineto{\pgfqpoint{1.856189in}{3.287389in}}%
\pgfpathlineto{\pgfqpoint{1.900333in}{3.301105in}}%
\pgfpathlineto{\pgfqpoint{1.944478in}{3.315494in}}%
\pgfpathlineto{\pgfqpoint{1.988622in}{3.317137in}}%
\pgfpathlineto{\pgfqpoint{2.032767in}{3.308795in}}%
\pgfpathlineto{\pgfqpoint{2.076911in}{3.315376in}}%
\pgfpathlineto{\pgfqpoint{2.121055in}{3.312886in}}%
\pgfpathlineto{\pgfqpoint{2.165200in}{3.309225in}}%
\pgfpathlineto{\pgfqpoint{2.209344in}{3.314579in}}%
\pgfpathlineto{\pgfqpoint{2.253489in}{3.312886in}}%
\pgfpathlineto{\pgfqpoint{2.297633in}{3.318462in}}%
\pgfpathlineto{\pgfqpoint{2.341778in}{3.316850in}}%
\pgfpathlineto{\pgfqpoint{2.385922in}{3.306870in}}%
\pgfpathlineto{\pgfqpoint{2.430067in}{3.321051in}}%
\pgfpathlineto{\pgfqpoint{2.474211in}{3.313184in}}%
\pgfpathlineto{\pgfqpoint{2.518356in}{3.322787in}}%
\pgfpathlineto{\pgfqpoint{2.562500in}{3.319441in}}%
\pgfpathlineto{\pgfqpoint{2.606644in}{3.290425in}}%
\pgfpathlineto{\pgfqpoint{2.650789in}{3.328079in}}%
\pgfpathlineto{\pgfqpoint{2.694933in}{3.324157in}}%
\pgfpathlineto{\pgfqpoint{2.739078in}{3.326461in}}%
\pgfpathlineto{\pgfqpoint{2.783222in}{3.309960in}}%
\pgfpathlineto{\pgfqpoint{2.827367in}{3.290076in}}%
\pgfpathlineto{\pgfqpoint{2.871511in}{3.326717in}}%
\pgfpathlineto{\pgfqpoint{2.915656in}{3.329584in}}%
\pgfpathlineto{\pgfqpoint{2.959800in}{3.327620in}}%
\pgfpathlineto{\pgfqpoint{3.003945in}{3.301893in}}%
\pgfpathlineto{\pgfqpoint{3.048089in}{3.281273in}}%
\pgfpathlineto{\pgfqpoint{3.092233in}{3.328346in}}%
\pgfpathlineto{\pgfqpoint{3.136378in}{3.329750in}}%
\pgfpathlineto{\pgfqpoint{3.180522in}{3.328107in}}%
\pgfpathlineto{\pgfqpoint{3.224667in}{3.301412in}}%
\pgfpathlineto{\pgfqpoint{3.268811in}{3.287434in}}%
\pgfpathlineto{\pgfqpoint{3.312956in}{3.329919in}}%
\pgfpathlineto{\pgfqpoint{3.357100in}{3.329539in}}%
\pgfpathlineto{\pgfqpoint{3.401245in}{3.329607in}}%
\pgfpathlineto{\pgfqpoint{3.445389in}{3.302765in}}%
\pgfpathlineto{\pgfqpoint{3.489533in}{3.322817in}}%
\pgfpathlineto{\pgfqpoint{3.533678in}{3.330926in}}%
\pgfpathlineto{\pgfqpoint{3.577822in}{3.312869in}}%
\pgfpathlineto{\pgfqpoint{3.621967in}{3.327662in}}%
\pgfpathlineto{\pgfqpoint{3.666111in}{3.307109in}}%
\pgfpathlineto{\pgfqpoint{3.710256in}{3.321951in}}%
\pgfpathlineto{\pgfqpoint{3.754400in}{3.330634in}}%
\pgfpathlineto{\pgfqpoint{3.798545in}{3.310874in}}%
\pgfpathlineto{\pgfqpoint{3.842689in}{3.325094in}}%
\pgfpathlineto{\pgfqpoint{3.886834in}{3.306268in}}%
\pgfpathlineto{\pgfqpoint{3.930978in}{3.321453in}}%
\pgfpathlineto{\pgfqpoint{3.975122in}{3.330422in}}%
\pgfpathlineto{\pgfqpoint{4.019267in}{3.310705in}}%
\pgfpathlineto{\pgfqpoint{4.063411in}{3.323200in}}%
\pgfpathlineto{\pgfqpoint{4.107556in}{3.306465in}}%
\pgfpathlineto{\pgfqpoint{4.151700in}{3.318510in}}%
\pgfpathlineto{\pgfqpoint{4.195845in}{3.328523in}}%
\pgfpathlineto{\pgfqpoint{4.239989in}{3.309310in}}%
\pgfpathlineto{\pgfqpoint{4.284134in}{3.321762in}}%
\pgfusepath{stroke}%
\end{pgfscope}%
\begin{pgfscope}%
\pgfpathrectangle{\pgfqpoint{0.625000in}{0.440000in}}{\pgfqpoint{3.875000in}{3.080000in}} %
\pgfusepath{clip}%
\pgfsetrectcap%
\pgfsetroundjoin%
\pgfsetlinewidth{1.505625pt}%
\definecolor{currentstroke}{rgb}{1.000000,0.000000,0.000000}%
\pgfsetstrokecolor{currentstroke}%
\pgfsetdash{}{0pt}%
\pgfpathmoveto{\pgfqpoint{0.840866in}{2.068246in}}%
\pgfpathlineto{\pgfqpoint{0.885011in}{2.160113in}}%
\pgfpathlineto{\pgfqpoint{0.929155in}{2.146310in}}%
\pgfpathlineto{\pgfqpoint{0.973300in}{1.957518in}}%
\pgfpathlineto{\pgfqpoint{1.017444in}{2.135644in}}%
\pgfpathlineto{\pgfqpoint{1.061589in}{2.164553in}}%
\pgfpathlineto{\pgfqpoint{1.105733in}{2.222161in}}%
\pgfpathlineto{\pgfqpoint{1.149878in}{2.202922in}}%
\pgfpathlineto{\pgfqpoint{1.194022in}{2.102244in}}%
\pgfpathlineto{\pgfqpoint{1.238166in}{2.197393in}}%
\pgfpathlineto{\pgfqpoint{1.282311in}{2.221018in}}%
\pgfpathlineto{\pgfqpoint{1.326455in}{2.278969in}}%
\pgfpathlineto{\pgfqpoint{1.370600in}{2.176747in}}%
\pgfpathlineto{\pgfqpoint{1.414744in}{2.116940in}}%
\pgfpathlineto{\pgfqpoint{1.458889in}{2.253726in}}%
\pgfpathlineto{\pgfqpoint{1.503033in}{2.287336in}}%
\pgfpathlineto{\pgfqpoint{1.547178in}{2.370637in}}%
\pgfpathlineto{\pgfqpoint{1.591322in}{2.177431in}}%
\pgfpathlineto{\pgfqpoint{1.635467in}{2.359335in}}%
\pgfpathlineto{\pgfqpoint{1.679611in}{2.359588in}}%
\pgfpathlineto{\pgfqpoint{1.723755in}{2.508518in}}%
\pgfpathlineto{\pgfqpoint{1.767900in}{2.533755in}}%
\pgfpathlineto{\pgfqpoint{1.812044in}{2.446533in}}%
\pgfpathlineto{\pgfqpoint{1.856189in}{2.510470in}}%
\pgfpathlineto{\pgfqpoint{1.900333in}{2.536960in}}%
\pgfpathlineto{\pgfqpoint{1.944478in}{2.634514in}}%
\pgfpathlineto{\pgfqpoint{1.988622in}{2.590159in}}%
\pgfpathlineto{\pgfqpoint{2.032767in}{2.566423in}}%
\pgfpathlineto{\pgfqpoint{2.076911in}{2.633943in}}%
\pgfpathlineto{\pgfqpoint{2.121055in}{2.622196in}}%
\pgfpathlineto{\pgfqpoint{2.165200in}{2.641281in}}%
\pgfpathlineto{\pgfqpoint{2.209344in}{2.609614in}}%
\pgfpathlineto{\pgfqpoint{2.253489in}{2.565686in}}%
\pgfpathlineto{\pgfqpoint{2.297633in}{2.656868in}}%
\pgfpathlineto{\pgfqpoint{2.341778in}{2.651584in}}%
\pgfpathlineto{\pgfqpoint{2.385922in}{2.660950in}}%
\pgfpathlineto{\pgfqpoint{2.430067in}{2.634793in}}%
\pgfpathlineto{\pgfqpoint{2.474211in}{2.614943in}}%
\pgfpathlineto{\pgfqpoint{2.518356in}{2.682595in}}%
\pgfpathlineto{\pgfqpoint{2.562500in}{2.680015in}}%
\pgfpathlineto{\pgfqpoint{2.606644in}{2.641413in}}%
\pgfpathlineto{\pgfqpoint{2.650789in}{2.680949in}}%
\pgfpathlineto{\pgfqpoint{2.694933in}{2.688537in}}%
\pgfpathlineto{\pgfqpoint{2.739078in}{2.693053in}}%
\pgfpathlineto{\pgfqpoint{2.783222in}{2.656873in}}%
\pgfpathlineto{\pgfqpoint{2.827367in}{2.634517in}}%
\pgfpathlineto{\pgfqpoint{2.871511in}{2.682190in}}%
\pgfpathlineto{\pgfqpoint{2.915656in}{2.704155in}}%
\pgfpathlineto{\pgfqpoint{2.959800in}{2.695723in}}%
\pgfpathlineto{\pgfqpoint{3.003945in}{2.642744in}}%
\pgfpathlineto{\pgfqpoint{3.048089in}{2.618187in}}%
\pgfpathlineto{\pgfqpoint{3.092233in}{2.680411in}}%
\pgfpathlineto{\pgfqpoint{3.136378in}{2.710606in}}%
\pgfpathlineto{\pgfqpoint{3.180522in}{2.705053in}}%
\pgfpathlineto{\pgfqpoint{3.224667in}{2.649938in}}%
\pgfpathlineto{\pgfqpoint{3.268811in}{2.634734in}}%
\pgfpathlineto{\pgfqpoint{3.312956in}{2.682086in}}%
\pgfpathlineto{\pgfqpoint{3.357100in}{2.714610in}}%
\pgfpathlineto{\pgfqpoint{3.401245in}{2.719154in}}%
\pgfpathlineto{\pgfqpoint{3.445389in}{2.664650in}}%
\pgfpathlineto{\pgfqpoint{3.489533in}{2.738008in}}%
\pgfpathlineto{\pgfqpoint{3.533678in}{2.709214in}}%
\pgfpathlineto{\pgfqpoint{3.577822in}{2.670015in}}%
\pgfpathlineto{\pgfqpoint{3.621967in}{2.721422in}}%
\pgfpathlineto{\pgfqpoint{3.666111in}{2.697585in}}%
\pgfpathlineto{\pgfqpoint{3.710256in}{2.733799in}}%
\pgfpathlineto{\pgfqpoint{3.754400in}{2.699282in}}%
\pgfpathlineto{\pgfqpoint{3.798545in}{2.667967in}}%
\pgfpathlineto{\pgfqpoint{3.842689in}{2.716588in}}%
\pgfpathlineto{\pgfqpoint{3.886834in}{2.700897in}}%
\pgfpathlineto{\pgfqpoint{3.930978in}{2.732921in}}%
\pgfpathlineto{\pgfqpoint{3.975122in}{2.702360in}}%
\pgfpathlineto{\pgfqpoint{4.019267in}{2.672328in}}%
\pgfpathlineto{\pgfqpoint{4.063411in}{2.707227in}}%
\pgfpathlineto{\pgfqpoint{4.107556in}{2.703868in}}%
\pgfpathlineto{\pgfqpoint{4.151700in}{2.722865in}}%
\pgfpathlineto{\pgfqpoint{4.195845in}{2.685715in}}%
\pgfpathlineto{\pgfqpoint{4.239989in}{2.669641in}}%
\pgfpathlineto{\pgfqpoint{4.284134in}{2.697104in}}%
\pgfusepath{stroke}%
\end{pgfscope}%
\begin{pgfscope}%
\pgfpathrectangle{\pgfqpoint{0.625000in}{0.440000in}}{\pgfqpoint{3.875000in}{3.080000in}} %
\pgfusepath{clip}%
\pgfsetrectcap%
\pgfsetroundjoin%
\pgfsetlinewidth{1.505625pt}%
\definecolor{currentstroke}{rgb}{0.000000,0.501961,0.000000}%
\pgfsetstrokecolor{currentstroke}%
\pgfsetdash{}{0pt}%
\pgfpathmoveto{\pgfqpoint{0.840866in}{2.461880in}}%
\pgfpathlineto{\pgfqpoint{0.885011in}{2.642181in}}%
\pgfpathlineto{\pgfqpoint{0.929155in}{2.683852in}}%
\pgfpathlineto{\pgfqpoint{0.973300in}{2.762545in}}%
\pgfpathlineto{\pgfqpoint{1.017444in}{2.779963in}}%
\pgfpathlineto{\pgfqpoint{1.061589in}{2.802469in}}%
\pgfpathlineto{\pgfqpoint{1.105733in}{2.848974in}}%
\pgfpathlineto{\pgfqpoint{1.149878in}{2.845685in}}%
\pgfpathlineto{\pgfqpoint{1.194022in}{2.803814in}}%
\pgfpathlineto{\pgfqpoint{1.238166in}{2.781097in}}%
\pgfpathlineto{\pgfqpoint{1.282311in}{2.850507in}}%
\pgfpathlineto{\pgfqpoint{1.326455in}{2.886464in}}%
\pgfpathlineto{\pgfqpoint{1.370600in}{2.753263in}}%
\pgfpathlineto{\pgfqpoint{1.414744in}{2.846656in}}%
\pgfpathlineto{\pgfqpoint{1.458889in}{2.776317in}}%
\pgfpathlineto{\pgfqpoint{1.503033in}{2.865869in}}%
\pgfpathlineto{\pgfqpoint{1.547178in}{2.859632in}}%
\pgfpathlineto{\pgfqpoint{1.591322in}{2.732718in}}%
\pgfpathlineto{\pgfqpoint{1.635467in}{2.836102in}}%
\pgfpathlineto{\pgfqpoint{1.679611in}{2.785056in}}%
\pgfpathlineto{\pgfqpoint{1.723755in}{2.873024in}}%
\pgfpathlineto{\pgfqpoint{1.767900in}{2.869524in}}%
\pgfpathlineto{\pgfqpoint{1.812044in}{2.792717in}}%
\pgfpathlineto{\pgfqpoint{1.856189in}{2.857623in}}%
\pgfpathlineto{\pgfqpoint{1.900333in}{2.860979in}}%
\pgfpathlineto{\pgfqpoint{1.944478in}{2.858290in}}%
\pgfpathlineto{\pgfqpoint{1.988622in}{2.851582in}}%
\pgfpathlineto{\pgfqpoint{2.032767in}{2.843442in}}%
\pgfpathlineto{\pgfqpoint{2.076911in}{2.867771in}}%
\pgfpathlineto{\pgfqpoint{2.121055in}{2.849010in}}%
\pgfpathlineto{\pgfqpoint{2.165200in}{2.871460in}}%
\pgfpathlineto{\pgfqpoint{2.209344in}{2.857223in}}%
\pgfpathlineto{\pgfqpoint{2.253489in}{2.792484in}}%
\pgfpathlineto{\pgfqpoint{2.297633in}{2.872591in}}%
\pgfpathlineto{\pgfqpoint{2.341778in}{2.852938in}}%
\pgfpathlineto{\pgfqpoint{2.385922in}{2.878843in}}%
\pgfpathlineto{\pgfqpoint{2.430067in}{2.835688in}}%
\pgfpathlineto{\pgfqpoint{2.474211in}{2.837523in}}%
\pgfpathlineto{\pgfqpoint{2.518356in}{2.869549in}}%
\pgfpathlineto{\pgfqpoint{2.562500in}{2.858723in}}%
\pgfpathlineto{\pgfqpoint{2.606644in}{2.902291in}}%
\pgfpathlineto{\pgfqpoint{2.650789in}{2.845924in}}%
\pgfpathlineto{\pgfqpoint{2.694933in}{2.870216in}}%
\pgfpathlineto{\pgfqpoint{2.739078in}{2.858149in}}%
\pgfpathlineto{\pgfqpoint{2.783222in}{2.873154in}}%
\pgfpathlineto{\pgfqpoint{2.827367in}{2.891444in}}%
\pgfpathlineto{\pgfqpoint{2.871511in}{2.853220in}}%
\pgfpathlineto{\pgfqpoint{2.915656in}{2.868531in}}%
\pgfpathlineto{\pgfqpoint{2.959800in}{2.862043in}}%
\pgfpathlineto{\pgfqpoint{3.003945in}{2.881625in}}%
\pgfpathlineto{\pgfqpoint{3.048089in}{2.905706in}}%
\pgfpathlineto{\pgfqpoint{3.092233in}{2.831884in}}%
\pgfpathlineto{\pgfqpoint{3.136378in}{2.872535in}}%
\pgfpathlineto{\pgfqpoint{3.180522in}{2.866221in}}%
\pgfpathlineto{\pgfqpoint{3.224667in}{2.881169in}}%
\pgfpathlineto{\pgfqpoint{3.268811in}{2.906787in}}%
\pgfpathlineto{\pgfqpoint{3.312956in}{2.823241in}}%
\pgfpathlineto{\pgfqpoint{3.357100in}{2.875252in}}%
\pgfpathlineto{\pgfqpoint{3.401245in}{2.860352in}}%
\pgfpathlineto{\pgfqpoint{3.445389in}{2.890522in}}%
\pgfpathlineto{\pgfqpoint{3.489533in}{2.913958in}}%
\pgfpathlineto{\pgfqpoint{3.533678in}{2.835238in}}%
\pgfpathlineto{\pgfqpoint{3.577822in}{2.887666in}}%
\pgfpathlineto{\pgfqpoint{3.621967in}{2.824910in}}%
\pgfpathlineto{\pgfqpoint{3.666111in}{2.900735in}}%
\pgfpathlineto{\pgfqpoint{3.710256in}{2.913247in}}%
\pgfpathlineto{\pgfqpoint{3.754400in}{2.820135in}}%
\pgfpathlineto{\pgfqpoint{3.798545in}{2.892114in}}%
\pgfpathlineto{\pgfqpoint{3.842689in}{2.819885in}}%
\pgfpathlineto{\pgfqpoint{3.886834in}{2.904114in}}%
\pgfpathlineto{\pgfqpoint{3.930978in}{2.915382in}}%
\pgfpathlineto{\pgfqpoint{3.975122in}{2.823556in}}%
\pgfpathlineto{\pgfqpoint{4.019267in}{2.891895in}}%
\pgfpathlineto{\pgfqpoint{4.063411in}{2.808569in}}%
\pgfpathlineto{\pgfqpoint{4.107556in}{2.909232in}}%
\pgfpathlineto{\pgfqpoint{4.151700in}{2.913466in}}%
\pgfpathlineto{\pgfqpoint{4.195845in}{2.810932in}}%
\pgfpathlineto{\pgfqpoint{4.239989in}{2.891036in}}%
\pgfpathlineto{\pgfqpoint{4.284134in}{2.797576in}}%
\pgfusepath{stroke}%
\end{pgfscope}%
\begin{pgfscope}%
\pgfsetrectcap%
\pgfsetmiterjoin%
\pgfsetlinewidth{0.803000pt}%
\definecolor{currentstroke}{rgb}{0.000000,0.000000,0.000000}%
\pgfsetstrokecolor{currentstroke}%
\pgfsetdash{}{0pt}%
\pgfpathmoveto{\pgfqpoint{0.625000in}{0.440000in}}%
\pgfpathlineto{\pgfqpoint{0.625000in}{3.520000in}}%
\pgfusepath{stroke}%
\end{pgfscope}%
\begin{pgfscope}%
\pgfsetrectcap%
\pgfsetmiterjoin%
\pgfsetlinewidth{0.803000pt}%
\definecolor{currentstroke}{rgb}{0.000000,0.000000,0.000000}%
\pgfsetstrokecolor{currentstroke}%
\pgfsetdash{}{0pt}%
\pgfpathmoveto{\pgfqpoint{4.500000in}{0.440000in}}%
\pgfpathlineto{\pgfqpoint{4.500000in}{3.520000in}}%
\pgfusepath{stroke}%
\end{pgfscope}%
\begin{pgfscope}%
\pgfsetrectcap%
\pgfsetmiterjoin%
\pgfsetlinewidth{0.803000pt}%
\definecolor{currentstroke}{rgb}{0.000000,0.000000,0.000000}%
\pgfsetstrokecolor{currentstroke}%
\pgfsetdash{}{0pt}%
\pgfpathmoveto{\pgfqpoint{0.625000in}{0.440000in}}%
\pgfpathlineto{\pgfqpoint{4.500000in}{0.440000in}}%
\pgfusepath{stroke}%
\end{pgfscope}%
\begin{pgfscope}%
\pgfsetrectcap%
\pgfsetmiterjoin%
\pgfsetlinewidth{0.803000pt}%
\definecolor{currentstroke}{rgb}{0.000000,0.000000,0.000000}%
\pgfsetstrokecolor{currentstroke}%
\pgfsetdash{}{0pt}%
\pgfpathmoveto{\pgfqpoint{0.625000in}{3.520000in}}%
\pgfpathlineto{\pgfqpoint{4.500000in}{3.520000in}}%
\pgfusepath{stroke}%
\end{pgfscope}%
\end{pgfpicture}%
\makeatother%
\endgroup%
}
		\caption{\textbf{Unet\_F1\_3}}
	\end {subfigure}\hspace{1.3cm}
	\begin {subfigure}[b]{0.4\linewidth}
		\scalebox{0.675}{%% Creator: Matplotlib, PGF backend
%%
%% To include the figure in your LaTeX document, write
%%   \input{<filename>.pgf}
%%
%% Make sure the required packages are loaded in your preamble
%%   \usepackage{pgf}
%%
%% Figures using additional raster images can only be included by \input if
%% they are in the same directory as the main LaTeX file. For loading figures
%% from other directories you can use the `import` package
%%   \usepackage{import}
%% and then include the figures with
%%   \import{<path to file>}{<filename>.pgf}
%%
%% Matplotlib used the following preamble
%%   \usepackage{fontspec}
%%   \setmainfont{DejaVu Serif}
%%   \setsansfont{DejaVu Sans}
%%   \setmonofont{DejaVu Sans Mono}
%%
\begingroup%
\makeatletter%
\begin{pgfpicture}%
\pgfpathrectangle{\pgfpointorigin}{\pgfqpoint{5.000000in}{4.000000in}}%
\pgfusepath{use as bounding box, clip}%
\begin{pgfscope}%
\pgfsetbuttcap%
\pgfsetmiterjoin%
\definecolor{currentfill}{rgb}{1.000000,1.000000,1.000000}%
\pgfsetfillcolor{currentfill}%
\pgfsetlinewidth{0.000000pt}%
\definecolor{currentstroke}{rgb}{1.000000,1.000000,1.000000}%
\pgfsetstrokecolor{currentstroke}%
\pgfsetdash{}{0pt}%
\pgfpathmoveto{\pgfqpoint{0.000000in}{0.000000in}}%
\pgfpathlineto{\pgfqpoint{5.000000in}{0.000000in}}%
\pgfpathlineto{\pgfqpoint{5.000000in}{4.000000in}}%
\pgfpathlineto{\pgfqpoint{0.000000in}{4.000000in}}%
\pgfpathclose%
\pgfusepath{fill}%
\end{pgfscope}%
\begin{pgfscope}%
\pgfsetbuttcap%
\pgfsetmiterjoin%
\definecolor{currentfill}{rgb}{1.000000,1.000000,1.000000}%
\pgfsetfillcolor{currentfill}%
\pgfsetlinewidth{0.000000pt}%
\definecolor{currentstroke}{rgb}{0.000000,0.000000,0.000000}%
\pgfsetstrokecolor{currentstroke}%
\pgfsetstrokeopacity{0.000000}%
\pgfsetdash{}{0pt}%
\pgfpathmoveto{\pgfqpoint{0.625000in}{0.440000in}}%
\pgfpathlineto{\pgfqpoint{4.500000in}{0.440000in}}%
\pgfpathlineto{\pgfqpoint{4.500000in}{3.520000in}}%
\pgfpathlineto{\pgfqpoint{0.625000in}{3.520000in}}%
\pgfpathclose%
\pgfusepath{fill}%
\end{pgfscope}%
\begin{pgfscope}%
\pgfsetbuttcap%
\pgfsetroundjoin%
\definecolor{currentfill}{rgb}{0.000000,0.000000,0.000000}%
\pgfsetfillcolor{currentfill}%
\pgfsetlinewidth{0.803000pt}%
\definecolor{currentstroke}{rgb}{0.000000,0.000000,0.000000}%
\pgfsetstrokecolor{currentstroke}%
\pgfsetdash{}{0pt}%
\pgfsys@defobject{currentmarker}{\pgfqpoint{0.000000in}{-0.048611in}}{\pgfqpoint{0.000000in}{0.000000in}}{%
\pgfpathmoveto{\pgfqpoint{0.000000in}{0.000000in}}%
\pgfpathlineto{\pgfqpoint{0.000000in}{-0.048611in}}%
\pgfusepath{stroke,fill}%
}%
\begin{pgfscope}%
\pgfsys@transformshift{0.796722in}{0.440000in}%
\pgfsys@useobject{currentmarker}{}%
\end{pgfscope}%
\end{pgfscope}%
\begin{pgfscope}%
\pgftext[x=0.796722in,y=0.342778in,,top]{\sffamily\fontsize{10.000000}{12.000000}\selectfont 0}%
\end{pgfscope}%
\begin{pgfscope}%
\pgfsetbuttcap%
\pgfsetroundjoin%
\definecolor{currentfill}{rgb}{0.000000,0.000000,0.000000}%
\pgfsetfillcolor{currentfill}%
\pgfsetlinewidth{0.803000pt}%
\definecolor{currentstroke}{rgb}{0.000000,0.000000,0.000000}%
\pgfsetstrokecolor{currentstroke}%
\pgfsetdash{}{0pt}%
\pgfsys@defobject{currentmarker}{\pgfqpoint{0.000000in}{-0.048611in}}{\pgfqpoint{0.000000in}{0.000000in}}{%
\pgfpathmoveto{\pgfqpoint{0.000000in}{0.000000in}}%
\pgfpathlineto{\pgfqpoint{0.000000in}{-0.048611in}}%
\pgfusepath{stroke,fill}%
}%
\begin{pgfscope}%
\pgfsys@transformshift{1.385300in}{0.440000in}%
\pgfsys@useobject{currentmarker}{}%
\end{pgfscope}%
\end{pgfscope}%
\begin{pgfscope}%
\pgftext[x=1.385300in,y=0.342778in,,top]{\sffamily\fontsize{10.000000}{12.000000}\selectfont 5}%
\end{pgfscope}%
\begin{pgfscope}%
\pgfsetbuttcap%
\pgfsetroundjoin%
\definecolor{currentfill}{rgb}{0.000000,0.000000,0.000000}%
\pgfsetfillcolor{currentfill}%
\pgfsetlinewidth{0.803000pt}%
\definecolor{currentstroke}{rgb}{0.000000,0.000000,0.000000}%
\pgfsetstrokecolor{currentstroke}%
\pgfsetdash{}{0pt}%
\pgfsys@defobject{currentmarker}{\pgfqpoint{0.000000in}{-0.048611in}}{\pgfqpoint{0.000000in}{0.000000in}}{%
\pgfpathmoveto{\pgfqpoint{0.000000in}{0.000000in}}%
\pgfpathlineto{\pgfqpoint{0.000000in}{-0.048611in}}%
\pgfusepath{stroke,fill}%
}%
\begin{pgfscope}%
\pgfsys@transformshift{1.973878in}{0.440000in}%
\pgfsys@useobject{currentmarker}{}%
\end{pgfscope}%
\end{pgfscope}%
\begin{pgfscope}%
\pgftext[x=1.973878in,y=0.342778in,,top]{\sffamily\fontsize{10.000000}{12.000000}\selectfont 10}%
\end{pgfscope}%
\begin{pgfscope}%
\pgfsetbuttcap%
\pgfsetroundjoin%
\definecolor{currentfill}{rgb}{0.000000,0.000000,0.000000}%
\pgfsetfillcolor{currentfill}%
\pgfsetlinewidth{0.803000pt}%
\definecolor{currentstroke}{rgb}{0.000000,0.000000,0.000000}%
\pgfsetstrokecolor{currentstroke}%
\pgfsetdash{}{0pt}%
\pgfsys@defobject{currentmarker}{\pgfqpoint{0.000000in}{-0.048611in}}{\pgfqpoint{0.000000in}{0.000000in}}{%
\pgfpathmoveto{\pgfqpoint{0.000000in}{0.000000in}}%
\pgfpathlineto{\pgfqpoint{0.000000in}{-0.048611in}}%
\pgfusepath{stroke,fill}%
}%
\begin{pgfscope}%
\pgfsys@transformshift{2.562456in}{0.440000in}%
\pgfsys@useobject{currentmarker}{}%
\end{pgfscope}%
\end{pgfscope}%
\begin{pgfscope}%
\pgftext[x=2.562456in,y=0.342778in,,top]{\sffamily\fontsize{10.000000}{12.000000}\selectfont 15}%
\end{pgfscope}%
\begin{pgfscope}%
\pgfsetbuttcap%
\pgfsetroundjoin%
\definecolor{currentfill}{rgb}{0.000000,0.000000,0.000000}%
\pgfsetfillcolor{currentfill}%
\pgfsetlinewidth{0.803000pt}%
\definecolor{currentstroke}{rgb}{0.000000,0.000000,0.000000}%
\pgfsetstrokecolor{currentstroke}%
\pgfsetdash{}{0pt}%
\pgfsys@defobject{currentmarker}{\pgfqpoint{0.000000in}{-0.048611in}}{\pgfqpoint{0.000000in}{0.000000in}}{%
\pgfpathmoveto{\pgfqpoint{0.000000in}{0.000000in}}%
\pgfpathlineto{\pgfqpoint{0.000000in}{-0.048611in}}%
\pgfusepath{stroke,fill}%
}%
\begin{pgfscope}%
\pgfsys@transformshift{3.151034in}{0.440000in}%
\pgfsys@useobject{currentmarker}{}%
\end{pgfscope}%
\end{pgfscope}%
\begin{pgfscope}%
\pgftext[x=3.151034in,y=0.342778in,,top]{\sffamily\fontsize{10.000000}{12.000000}\selectfont 20}%
\end{pgfscope}%
\begin{pgfscope}%
\pgfsetbuttcap%
\pgfsetroundjoin%
\definecolor{currentfill}{rgb}{0.000000,0.000000,0.000000}%
\pgfsetfillcolor{currentfill}%
\pgfsetlinewidth{0.803000pt}%
\definecolor{currentstroke}{rgb}{0.000000,0.000000,0.000000}%
\pgfsetstrokecolor{currentstroke}%
\pgfsetdash{}{0pt}%
\pgfsys@defobject{currentmarker}{\pgfqpoint{0.000000in}{-0.048611in}}{\pgfqpoint{0.000000in}{0.000000in}}{%
\pgfpathmoveto{\pgfqpoint{0.000000in}{0.000000in}}%
\pgfpathlineto{\pgfqpoint{0.000000in}{-0.048611in}}%
\pgfusepath{stroke,fill}%
}%
\begin{pgfscope}%
\pgfsys@transformshift{3.739612in}{0.440000in}%
\pgfsys@useobject{currentmarker}{}%
\end{pgfscope}%
\end{pgfscope}%
\begin{pgfscope}%
\pgftext[x=3.739612in,y=0.342778in,,top]{\sffamily\fontsize{10.000000}{12.000000}\selectfont 25}%
\end{pgfscope}%
\begin{pgfscope}%
\pgfsetbuttcap%
\pgfsetroundjoin%
\definecolor{currentfill}{rgb}{0.000000,0.000000,0.000000}%
\pgfsetfillcolor{currentfill}%
\pgfsetlinewidth{0.803000pt}%
\definecolor{currentstroke}{rgb}{0.000000,0.000000,0.000000}%
\pgfsetstrokecolor{currentstroke}%
\pgfsetdash{}{0pt}%
\pgfsys@defobject{currentmarker}{\pgfqpoint{0.000000in}{-0.048611in}}{\pgfqpoint{0.000000in}{0.000000in}}{%
\pgfpathmoveto{\pgfqpoint{0.000000in}{0.000000in}}%
\pgfpathlineto{\pgfqpoint{0.000000in}{-0.048611in}}%
\pgfusepath{stroke,fill}%
}%
\begin{pgfscope}%
\pgfsys@transformshift{4.328190in}{0.440000in}%
\pgfsys@useobject{currentmarker}{}%
\end{pgfscope}%
\end{pgfscope}%
\begin{pgfscope}%
\pgftext[x=4.328190in,y=0.342778in,,top]{\sffamily\fontsize{10.000000}{12.000000}\selectfont 30}%
\end{pgfscope}%
\begin{pgfscope}%
\pgftext[x=2.562500in,y=0.152809in,,top]{\sffamily\fontsize{10.000000}{12.000000}\selectfont Epochs}%
\end{pgfscope}%
\begin{pgfscope}%
\pgfsetbuttcap%
\pgfsetroundjoin%
\definecolor{currentfill}{rgb}{0.000000,0.000000,0.000000}%
\pgfsetfillcolor{currentfill}%
\pgfsetlinewidth{0.803000pt}%
\definecolor{currentstroke}{rgb}{0.000000,0.000000,0.000000}%
\pgfsetstrokecolor{currentstroke}%
\pgfsetdash{}{0pt}%
\pgfsys@defobject{currentmarker}{\pgfqpoint{-0.048611in}{0.000000in}}{\pgfqpoint{0.000000in}{0.000000in}}{%
\pgfpathmoveto{\pgfqpoint{0.000000in}{0.000000in}}%
\pgfpathlineto{\pgfqpoint{-0.048611in}{0.000000in}}%
\pgfusepath{stroke,fill}%
}%
\begin{pgfscope}%
\pgfsys@transformshift{0.625000in}{0.580000in}%
\pgfsys@useobject{currentmarker}{}%
\end{pgfscope}%
\end{pgfscope}%
\begin{pgfscope}%
\pgftext[x=0.306898in,y=0.527238in,left,base]{\sffamily\fontsize{10.000000}{12.000000}\selectfont 0.0}%
\end{pgfscope}%
\begin{pgfscope}%
\pgfsetbuttcap%
\pgfsetroundjoin%
\definecolor{currentfill}{rgb}{0.000000,0.000000,0.000000}%
\pgfsetfillcolor{currentfill}%
\pgfsetlinewidth{0.803000pt}%
\definecolor{currentstroke}{rgb}{0.000000,0.000000,0.000000}%
\pgfsetstrokecolor{currentstroke}%
\pgfsetdash{}{0pt}%
\pgfsys@defobject{currentmarker}{\pgfqpoint{-0.048611in}{0.000000in}}{\pgfqpoint{0.000000in}{0.000000in}}{%
\pgfpathmoveto{\pgfqpoint{0.000000in}{0.000000in}}%
\pgfpathlineto{\pgfqpoint{-0.048611in}{0.000000in}}%
\pgfusepath{stroke,fill}%
}%
\begin{pgfscope}%
\pgfsys@transformshift{0.625000in}{1.142317in}%
\pgfsys@useobject{currentmarker}{}%
\end{pgfscope}%
\end{pgfscope}%
\begin{pgfscope}%
\pgftext[x=0.306898in,y=1.089556in,left,base]{\sffamily\fontsize{10.000000}{12.000000}\selectfont 0.2}%
\end{pgfscope}%
\begin{pgfscope}%
\pgfsetbuttcap%
\pgfsetroundjoin%
\definecolor{currentfill}{rgb}{0.000000,0.000000,0.000000}%
\pgfsetfillcolor{currentfill}%
\pgfsetlinewidth{0.803000pt}%
\definecolor{currentstroke}{rgb}{0.000000,0.000000,0.000000}%
\pgfsetstrokecolor{currentstroke}%
\pgfsetdash{}{0pt}%
\pgfsys@defobject{currentmarker}{\pgfqpoint{-0.048611in}{0.000000in}}{\pgfqpoint{0.000000in}{0.000000in}}{%
\pgfpathmoveto{\pgfqpoint{0.000000in}{0.000000in}}%
\pgfpathlineto{\pgfqpoint{-0.048611in}{0.000000in}}%
\pgfusepath{stroke,fill}%
}%
\begin{pgfscope}%
\pgfsys@transformshift{0.625000in}{1.704635in}%
\pgfsys@useobject{currentmarker}{}%
\end{pgfscope}%
\end{pgfscope}%
\begin{pgfscope}%
\pgftext[x=0.306898in,y=1.651873in,left,base]{\sffamily\fontsize{10.000000}{12.000000}\selectfont 0.4}%
\end{pgfscope}%
\begin{pgfscope}%
\pgfsetbuttcap%
\pgfsetroundjoin%
\definecolor{currentfill}{rgb}{0.000000,0.000000,0.000000}%
\pgfsetfillcolor{currentfill}%
\pgfsetlinewidth{0.803000pt}%
\definecolor{currentstroke}{rgb}{0.000000,0.000000,0.000000}%
\pgfsetstrokecolor{currentstroke}%
\pgfsetdash{}{0pt}%
\pgfsys@defobject{currentmarker}{\pgfqpoint{-0.048611in}{0.000000in}}{\pgfqpoint{0.000000in}{0.000000in}}{%
\pgfpathmoveto{\pgfqpoint{0.000000in}{0.000000in}}%
\pgfpathlineto{\pgfqpoint{-0.048611in}{0.000000in}}%
\pgfusepath{stroke,fill}%
}%
\begin{pgfscope}%
\pgfsys@transformshift{0.625000in}{2.266952in}%
\pgfsys@useobject{currentmarker}{}%
\end{pgfscope}%
\end{pgfscope}%
\begin{pgfscope}%
\pgftext[x=0.306898in,y=2.214190in,left,base]{\sffamily\fontsize{10.000000}{12.000000}\selectfont 0.6}%
\end{pgfscope}%
\begin{pgfscope}%
\pgfsetbuttcap%
\pgfsetroundjoin%
\definecolor{currentfill}{rgb}{0.000000,0.000000,0.000000}%
\pgfsetfillcolor{currentfill}%
\pgfsetlinewidth{0.803000pt}%
\definecolor{currentstroke}{rgb}{0.000000,0.000000,0.000000}%
\pgfsetstrokecolor{currentstroke}%
\pgfsetdash{}{0pt}%
\pgfsys@defobject{currentmarker}{\pgfqpoint{-0.048611in}{0.000000in}}{\pgfqpoint{0.000000in}{0.000000in}}{%
\pgfpathmoveto{\pgfqpoint{0.000000in}{0.000000in}}%
\pgfpathlineto{\pgfqpoint{-0.048611in}{0.000000in}}%
\pgfusepath{stroke,fill}%
}%
\begin{pgfscope}%
\pgfsys@transformshift{0.625000in}{2.829269in}%
\pgfsys@useobject{currentmarker}{}%
\end{pgfscope}%
\end{pgfscope}%
\begin{pgfscope}%
\pgftext[x=0.306898in,y=2.776508in,left,base]{\sffamily\fontsize{10.000000}{12.000000}\selectfont 0.8}%
\end{pgfscope}%
\begin{pgfscope}%
\pgfsetbuttcap%
\pgfsetroundjoin%
\definecolor{currentfill}{rgb}{0.000000,0.000000,0.000000}%
\pgfsetfillcolor{currentfill}%
\pgfsetlinewidth{0.803000pt}%
\definecolor{currentstroke}{rgb}{0.000000,0.000000,0.000000}%
\pgfsetstrokecolor{currentstroke}%
\pgfsetdash{}{0pt}%
\pgfsys@defobject{currentmarker}{\pgfqpoint{-0.048611in}{0.000000in}}{\pgfqpoint{0.000000in}{0.000000in}}{%
\pgfpathmoveto{\pgfqpoint{0.000000in}{0.000000in}}%
\pgfpathlineto{\pgfqpoint{-0.048611in}{0.000000in}}%
\pgfusepath{stroke,fill}%
}%
\begin{pgfscope}%
\pgfsys@transformshift{0.625000in}{3.391587in}%
\pgfsys@useobject{currentmarker}{}%
\end{pgfscope}%
\end{pgfscope}%
\begin{pgfscope}%
\pgftext[x=0.306898in,y=3.338825in,left,base]{\sffamily\fontsize{10.000000}{12.000000}\selectfont 1.0}%
\end{pgfscope}%
\begin{pgfscope}%
\pgftext[x=0.251343in,y=1.980000in,,bottom,rotate=90.000000]{\sffamily\fontsize{10.000000}{12.000000}\selectfont F-Measure score}%
\end{pgfscope}%
\begin{pgfscope}%
\pgfpathrectangle{\pgfqpoint{0.625000in}{0.440000in}}{\pgfqpoint{3.875000in}{3.080000in}} %
\pgfusepath{clip}%
\pgfsetrectcap%
\pgfsetroundjoin%
\pgfsetlinewidth{1.505625pt}%
\definecolor{currentstroke}{rgb}{0.752941,0.752941,0.752941}%
\pgfsetstrokecolor{currentstroke}%
\pgfsetdash{}{0pt}%
\pgfpathmoveto{\pgfqpoint{0.801136in}{3.193907in}}%
\pgfpathlineto{\pgfqpoint{0.809965in}{3.266238in}}%
\pgfpathlineto{\pgfqpoint{0.814380in}{3.158357in}}%
\pgfpathlineto{\pgfqpoint{0.818794in}{3.222110in}}%
\pgfpathlineto{\pgfqpoint{0.823209in}{3.263567in}}%
\pgfpathlineto{\pgfqpoint{0.827623in}{3.261747in}}%
\pgfpathlineto{\pgfqpoint{0.832037in}{3.283928in}}%
\pgfpathlineto{\pgfqpoint{0.836452in}{3.313242in}}%
\pgfpathlineto{\pgfqpoint{0.840866in}{3.274993in}}%
\pgfpathlineto{\pgfqpoint{0.845281in}{3.301962in}}%
\pgfpathlineto{\pgfqpoint{0.849695in}{3.282455in}}%
\pgfpathlineto{\pgfqpoint{0.854110in}{3.278201in}}%
\pgfpathlineto{\pgfqpoint{0.858524in}{3.296555in}}%
\pgfpathlineto{\pgfqpoint{0.862939in}{3.255244in}}%
\pgfpathlineto{\pgfqpoint{0.867353in}{2.637519in}}%
\pgfpathlineto{\pgfqpoint{0.871767in}{2.823534in}}%
\pgfpathlineto{\pgfqpoint{0.876182in}{3.280071in}}%
\pgfpathlineto{\pgfqpoint{0.880596in}{3.099415in}}%
\pgfpathlineto{\pgfqpoint{0.885011in}{3.275800in}}%
\pgfpathlineto{\pgfqpoint{0.889425in}{3.299513in}}%
\pgfpathlineto{\pgfqpoint{0.893840in}{3.284052in}}%
\pgfpathlineto{\pgfqpoint{0.898254in}{3.282618in}}%
\pgfpathlineto{\pgfqpoint{0.902669in}{3.325582in}}%
\pgfpathlineto{\pgfqpoint{0.907083in}{3.177625in}}%
\pgfpathlineto{\pgfqpoint{0.911497in}{3.277034in}}%
\pgfpathlineto{\pgfqpoint{0.915912in}{3.289262in}}%
\pgfpathlineto{\pgfqpoint{0.920326in}{3.213706in}}%
\pgfpathlineto{\pgfqpoint{0.924741in}{2.480877in}}%
\pgfpathlineto{\pgfqpoint{0.929155in}{3.304832in}}%
\pgfpathlineto{\pgfqpoint{0.933570in}{3.297733in}}%
\pgfpathlineto{\pgfqpoint{0.937984in}{3.218022in}}%
\pgfpathlineto{\pgfqpoint{0.946813in}{3.322503in}}%
\pgfpathlineto{\pgfqpoint{0.951228in}{3.281457in}}%
\pgfpathlineto{\pgfqpoint{0.955642in}{3.330432in}}%
\pgfpathlineto{\pgfqpoint{0.960056in}{3.310593in}}%
\pgfpathlineto{\pgfqpoint{0.964471in}{3.248797in}}%
\pgfpathlineto{\pgfqpoint{0.968885in}{3.307380in}}%
\pgfpathlineto{\pgfqpoint{0.973300in}{3.258925in}}%
\pgfpathlineto{\pgfqpoint{0.977714in}{2.131307in}}%
\pgfpathlineto{\pgfqpoint{0.982129in}{2.577874in}}%
\pgfpathlineto{\pgfqpoint{0.986543in}{3.178679in}}%
\pgfpathlineto{\pgfqpoint{0.990958in}{3.151067in}}%
\pgfpathlineto{\pgfqpoint{0.995372in}{3.329709in}}%
\pgfpathlineto{\pgfqpoint{0.999786in}{3.288677in}}%
\pgfpathlineto{\pgfqpoint{1.004201in}{3.301416in}}%
\pgfpathlineto{\pgfqpoint{1.008615in}{3.354845in}}%
\pgfpathlineto{\pgfqpoint{1.013030in}{3.308439in}}%
\pgfpathlineto{\pgfqpoint{1.017444in}{3.314313in}}%
\pgfpathlineto{\pgfqpoint{1.021859in}{3.162870in}}%
\pgfpathlineto{\pgfqpoint{1.026273in}{3.286731in}}%
\pgfpathlineto{\pgfqpoint{1.030688in}{3.179635in}}%
\pgfpathlineto{\pgfqpoint{1.035102in}{3.105465in}}%
\pgfpathlineto{\pgfqpoint{1.039516in}{3.187131in}}%
\pgfpathlineto{\pgfqpoint{1.043931in}{3.214454in}}%
\pgfpathlineto{\pgfqpoint{1.048345in}{3.138957in}}%
\pgfpathlineto{\pgfqpoint{1.052760in}{3.311993in}}%
\pgfpathlineto{\pgfqpoint{1.057174in}{3.319956in}}%
\pgfpathlineto{\pgfqpoint{1.061589in}{3.320366in}}%
\pgfpathlineto{\pgfqpoint{1.066003in}{3.330198in}}%
\pgfpathlineto{\pgfqpoint{1.070418in}{3.290634in}}%
\pgfpathlineto{\pgfqpoint{1.074832in}{3.208041in}}%
\pgfpathlineto{\pgfqpoint{1.079246in}{3.217383in}}%
\pgfpathlineto{\pgfqpoint{1.083661in}{3.198577in}}%
\pgfpathlineto{\pgfqpoint{1.088075in}{2.185759in}}%
\pgfpathlineto{\pgfqpoint{1.092490in}{2.552522in}}%
\pgfpathlineto{\pgfqpoint{1.096904in}{3.213866in}}%
\pgfpathlineto{\pgfqpoint{1.101319in}{3.282604in}}%
\pgfpathlineto{\pgfqpoint{1.105733in}{3.310455in}}%
\pgfpathlineto{\pgfqpoint{1.110148in}{3.251412in}}%
\pgfpathlineto{\pgfqpoint{1.114562in}{3.291517in}}%
\pgfpathlineto{\pgfqpoint{1.118976in}{3.349643in}}%
\pgfpathlineto{\pgfqpoint{1.123391in}{3.329223in}}%
\pgfpathlineto{\pgfqpoint{1.127805in}{3.291913in}}%
\pgfpathlineto{\pgfqpoint{1.132220in}{3.187269in}}%
\pgfpathlineto{\pgfqpoint{1.136634in}{3.231461in}}%
\pgfpathlineto{\pgfqpoint{1.141049in}{3.214083in}}%
\pgfpathlineto{\pgfqpoint{1.145463in}{3.306994in}}%
\pgfpathlineto{\pgfqpoint{1.149878in}{3.100087in}}%
\pgfpathlineto{\pgfqpoint{1.154292in}{3.250979in}}%
\pgfpathlineto{\pgfqpoint{1.158706in}{3.211743in}}%
\pgfpathlineto{\pgfqpoint{1.163121in}{3.321103in}}%
\pgfpathlineto{\pgfqpoint{1.171950in}{3.354490in}}%
\pgfpathlineto{\pgfqpoint{1.176364in}{3.328317in}}%
\pgfpathlineto{\pgfqpoint{1.180779in}{3.278476in}}%
\pgfpathlineto{\pgfqpoint{1.185193in}{3.215938in}}%
\pgfpathlineto{\pgfqpoint{1.189608in}{3.219914in}}%
\pgfpathlineto{\pgfqpoint{1.194022in}{3.232468in}}%
\pgfpathlineto{\pgfqpoint{1.198436in}{3.027143in}}%
\pgfpathlineto{\pgfqpoint{1.202851in}{2.977738in}}%
\pgfpathlineto{\pgfqpoint{1.207265in}{3.251668in}}%
\pgfpathlineto{\pgfqpoint{1.211680in}{3.270157in}}%
\pgfpathlineto{\pgfqpoint{1.216094in}{3.279913in}}%
\pgfpathlineto{\pgfqpoint{1.220509in}{3.278049in}}%
\pgfpathlineto{\pgfqpoint{1.224923in}{3.264289in}}%
\pgfpathlineto{\pgfqpoint{1.229338in}{3.327412in}}%
\pgfpathlineto{\pgfqpoint{1.233752in}{3.341237in}}%
\pgfpathlineto{\pgfqpoint{1.238166in}{3.214707in}}%
\pgfpathlineto{\pgfqpoint{1.242581in}{3.192470in}}%
\pgfpathlineto{\pgfqpoint{1.246995in}{3.330429in}}%
\pgfpathlineto{\pgfqpoint{1.251410in}{3.277369in}}%
\pgfpathlineto{\pgfqpoint{1.255824in}{3.289273in}}%
\pgfpathlineto{\pgfqpoint{1.260239in}{3.292557in}}%
\pgfpathlineto{\pgfqpoint{1.264653in}{3.270475in}}%
\pgfpathlineto{\pgfqpoint{1.269068in}{3.279154in}}%
\pgfpathlineto{\pgfqpoint{1.273482in}{3.337669in}}%
\pgfpathlineto{\pgfqpoint{1.277896in}{3.322017in}}%
\pgfpathlineto{\pgfqpoint{1.282311in}{3.331568in}}%
\pgfpathlineto{\pgfqpoint{1.291140in}{3.331483in}}%
\pgfpathlineto{\pgfqpoint{1.295554in}{3.255267in}}%
\pgfpathlineto{\pgfqpoint{1.299969in}{3.277925in}}%
\pgfpathlineto{\pgfqpoint{1.304383in}{3.215986in}}%
\pgfpathlineto{\pgfqpoint{1.308798in}{3.249242in}}%
\pgfpathlineto{\pgfqpoint{1.313212in}{2.990637in}}%
\pgfpathlineto{\pgfqpoint{1.317626in}{3.253352in}}%
\pgfpathlineto{\pgfqpoint{1.322041in}{3.251629in}}%
\pgfpathlineto{\pgfqpoint{1.330870in}{3.347509in}}%
\pgfpathlineto{\pgfqpoint{1.335284in}{3.298115in}}%
\pgfpathlineto{\pgfqpoint{1.339699in}{3.330322in}}%
\pgfpathlineto{\pgfqpoint{1.344113in}{3.306924in}}%
\pgfpathlineto{\pgfqpoint{1.348528in}{3.220142in}}%
\pgfpathlineto{\pgfqpoint{1.352942in}{3.188376in}}%
\pgfpathlineto{\pgfqpoint{1.357356in}{3.227978in}}%
\pgfpathlineto{\pgfqpoint{1.361771in}{3.328554in}}%
\pgfpathlineto{\pgfqpoint{1.366185in}{3.285354in}}%
\pgfpathlineto{\pgfqpoint{1.370600in}{3.283113in}}%
\pgfpathlineto{\pgfqpoint{1.375014in}{3.276728in}}%
\pgfpathlineto{\pgfqpoint{1.379429in}{3.228967in}}%
\pgfpathlineto{\pgfqpoint{1.383843in}{3.320830in}}%
\pgfpathlineto{\pgfqpoint{1.388258in}{3.333131in}}%
\pgfpathlineto{\pgfqpoint{1.392672in}{3.316340in}}%
\pgfpathlineto{\pgfqpoint{1.397086in}{3.321047in}}%
\pgfpathlineto{\pgfqpoint{1.401501in}{3.240216in}}%
\pgfpathlineto{\pgfqpoint{1.405915in}{3.273818in}}%
\pgfpathlineto{\pgfqpoint{1.410330in}{3.273084in}}%
\pgfpathlineto{\pgfqpoint{1.414744in}{3.341259in}}%
\pgfpathlineto{\pgfqpoint{1.419159in}{3.264970in}}%
\pgfpathlineto{\pgfqpoint{1.423573in}{3.249197in}}%
\pgfpathlineto{\pgfqpoint{1.427988in}{3.276494in}}%
\pgfpathlineto{\pgfqpoint{1.432402in}{3.289402in}}%
\pgfpathlineto{\pgfqpoint{1.436816in}{3.350861in}}%
\pgfpathlineto{\pgfqpoint{1.441231in}{3.342058in}}%
\pgfpathlineto{\pgfqpoint{1.445645in}{3.351932in}}%
\pgfpathlineto{\pgfqpoint{1.454474in}{3.341779in}}%
\pgfpathlineto{\pgfqpoint{1.458889in}{3.245373in}}%
\pgfpathlineto{\pgfqpoint{1.463303in}{3.278822in}}%
\pgfpathlineto{\pgfqpoint{1.467718in}{3.261927in}}%
\pgfpathlineto{\pgfqpoint{1.472132in}{3.321007in}}%
\pgfpathlineto{\pgfqpoint{1.476546in}{3.327834in}}%
\pgfpathlineto{\pgfqpoint{1.485375in}{3.268566in}}%
\pgfpathlineto{\pgfqpoint{1.489790in}{3.259532in}}%
\pgfpathlineto{\pgfqpoint{1.494204in}{3.333148in}}%
\pgfpathlineto{\pgfqpoint{1.498619in}{3.364092in}}%
\pgfpathlineto{\pgfqpoint{1.503033in}{3.329895in}}%
\pgfpathlineto{\pgfqpoint{1.507448in}{3.337112in}}%
\pgfpathlineto{\pgfqpoint{1.511862in}{3.139252in}}%
\pgfpathlineto{\pgfqpoint{1.516276in}{3.292835in}}%
\pgfpathlineto{\pgfqpoint{1.520691in}{3.259195in}}%
\pgfpathlineto{\pgfqpoint{1.525105in}{3.325194in}}%
\pgfpathlineto{\pgfqpoint{1.529520in}{3.316396in}}%
\pgfpathlineto{\pgfqpoint{1.533934in}{3.320929in}}%
\pgfpathlineto{\pgfqpoint{1.538349in}{3.280875in}}%
\pgfpathlineto{\pgfqpoint{1.542763in}{3.288536in}}%
\pgfpathlineto{\pgfqpoint{1.547178in}{3.350627in}}%
\pgfpathlineto{\pgfqpoint{1.551592in}{3.348555in}}%
\pgfpathlineto{\pgfqpoint{1.556006in}{3.358219in}}%
\pgfpathlineto{\pgfqpoint{1.560421in}{3.341029in}}%
\pgfpathlineto{\pgfqpoint{1.564835in}{3.350166in}}%
\pgfpathlineto{\pgfqpoint{1.573664in}{3.301287in}}%
\pgfpathlineto{\pgfqpoint{1.578079in}{3.303325in}}%
\pgfpathlineto{\pgfqpoint{1.582493in}{3.255697in}}%
\pgfpathlineto{\pgfqpoint{1.586908in}{3.310424in}}%
\pgfpathlineto{\pgfqpoint{1.591322in}{3.311982in}}%
\pgfpathlineto{\pgfqpoint{1.595737in}{3.315111in}}%
\pgfpathlineto{\pgfqpoint{1.600151in}{3.319835in}}%
\pgfpathlineto{\pgfqpoint{1.604565in}{3.357066in}}%
\pgfpathlineto{\pgfqpoint{1.608980in}{3.353104in}}%
\pgfpathlineto{\pgfqpoint{1.617809in}{3.362464in}}%
\pgfpathlineto{\pgfqpoint{1.626638in}{3.274155in}}%
\pgfpathlineto{\pgfqpoint{1.631052in}{3.261950in}}%
\pgfpathlineto{\pgfqpoint{1.635467in}{3.298346in}}%
\pgfpathlineto{\pgfqpoint{1.639881in}{3.304472in}}%
\pgfpathlineto{\pgfqpoint{1.644295in}{3.280841in}}%
\pgfpathlineto{\pgfqpoint{1.648710in}{3.286757in}}%
\pgfpathlineto{\pgfqpoint{1.653124in}{3.327491in}}%
\pgfpathlineto{\pgfqpoint{1.657539in}{3.346612in}}%
\pgfpathlineto{\pgfqpoint{1.661953in}{3.350765in}}%
\pgfpathlineto{\pgfqpoint{1.666368in}{3.367660in}}%
\pgfpathlineto{\pgfqpoint{1.670782in}{3.341644in}}%
\pgfpathlineto{\pgfqpoint{1.675197in}{3.349846in}}%
\pgfpathlineto{\pgfqpoint{1.679611in}{3.320456in}}%
\pgfpathlineto{\pgfqpoint{1.684025in}{3.324148in}}%
\pgfpathlineto{\pgfqpoint{1.688440in}{3.338442in}}%
\pgfpathlineto{\pgfqpoint{1.692854in}{3.265150in}}%
\pgfpathlineto{\pgfqpoint{1.697269in}{3.289945in}}%
\pgfpathlineto{\pgfqpoint{1.701683in}{3.280161in}}%
\pgfpathlineto{\pgfqpoint{1.706098in}{3.328658in}}%
\pgfpathlineto{\pgfqpoint{1.710512in}{3.329110in}}%
\pgfpathlineto{\pgfqpoint{1.714927in}{3.370151in}}%
\pgfpathlineto{\pgfqpoint{1.719341in}{3.359459in}}%
\pgfpathlineto{\pgfqpoint{1.723755in}{3.367081in}}%
\pgfpathlineto{\pgfqpoint{1.728170in}{3.340508in}}%
\pgfpathlineto{\pgfqpoint{1.732584in}{3.360035in}}%
\pgfpathlineto{\pgfqpoint{1.736999in}{3.319669in}}%
\pgfpathlineto{\pgfqpoint{1.741413in}{3.295745in}}%
\pgfpathlineto{\pgfqpoint{1.745828in}{3.320985in}}%
\pgfpathlineto{\pgfqpoint{1.750242in}{3.279013in}}%
\pgfpathlineto{\pgfqpoint{1.763485in}{3.326451in}}%
\pgfpathlineto{\pgfqpoint{1.767900in}{3.355390in}}%
\pgfpathlineto{\pgfqpoint{1.772314in}{3.363909in}}%
\pgfpathlineto{\pgfqpoint{1.776729in}{3.354794in}}%
\pgfpathlineto{\pgfqpoint{1.789972in}{3.307787in}}%
\pgfpathlineto{\pgfqpoint{1.794387in}{3.281223in}}%
\pgfpathlineto{\pgfqpoint{1.798801in}{3.325076in}}%
\pgfpathlineto{\pgfqpoint{1.803215in}{3.325123in}}%
\pgfpathlineto{\pgfqpoint{1.807630in}{3.328146in}}%
\pgfpathlineto{\pgfqpoint{1.812044in}{3.297230in}}%
\pgfpathlineto{\pgfqpoint{1.816459in}{3.347138in}}%
\pgfpathlineto{\pgfqpoint{1.820873in}{3.306182in}}%
\pgfpathlineto{\pgfqpoint{1.829702in}{3.358072in}}%
\pgfpathlineto{\pgfqpoint{1.834117in}{3.352888in}}%
\pgfpathlineto{\pgfqpoint{1.838531in}{3.334188in}}%
\pgfpathlineto{\pgfqpoint{1.842945in}{3.337607in}}%
\pgfpathlineto{\pgfqpoint{1.847360in}{3.326394in}}%
\pgfpathlineto{\pgfqpoint{1.851774in}{3.284488in}}%
\pgfpathlineto{\pgfqpoint{1.856189in}{3.319045in}}%
\pgfpathlineto{\pgfqpoint{1.860603in}{3.285154in}}%
\pgfpathlineto{\pgfqpoint{1.865018in}{3.335001in}}%
\pgfpathlineto{\pgfqpoint{1.869432in}{3.333980in}}%
\pgfpathlineto{\pgfqpoint{1.873847in}{3.352435in}}%
\pgfpathlineto{\pgfqpoint{1.878261in}{3.357513in}}%
\pgfpathlineto{\pgfqpoint{1.882675in}{3.353076in}}%
\pgfpathlineto{\pgfqpoint{1.887090in}{3.354541in}}%
\pgfpathlineto{\pgfqpoint{1.891504in}{3.348207in}}%
\pgfpathlineto{\pgfqpoint{1.895919in}{3.305147in}}%
\pgfpathlineto{\pgfqpoint{1.900333in}{3.309429in}}%
\pgfpathlineto{\pgfqpoint{1.904748in}{3.226535in}}%
\pgfpathlineto{\pgfqpoint{1.909162in}{3.321946in}}%
\pgfpathlineto{\pgfqpoint{1.913577in}{3.303176in}}%
\pgfpathlineto{\pgfqpoint{1.917991in}{3.335355in}}%
\pgfpathlineto{\pgfqpoint{1.922405in}{3.305735in}}%
\pgfpathlineto{\pgfqpoint{1.926820in}{3.364578in}}%
\pgfpathlineto{\pgfqpoint{1.931234in}{3.332358in}}%
\pgfpathlineto{\pgfqpoint{1.935649in}{3.330887in}}%
\pgfpathlineto{\pgfqpoint{1.940063in}{3.345454in}}%
\pgfpathlineto{\pgfqpoint{1.944478in}{3.352933in}}%
\pgfpathlineto{\pgfqpoint{1.948892in}{3.352570in}}%
\pgfpathlineto{\pgfqpoint{1.953307in}{3.344439in}}%
\pgfpathlineto{\pgfqpoint{1.957721in}{3.315263in}}%
\pgfpathlineto{\pgfqpoint{1.962135in}{3.302127in}}%
\pgfpathlineto{\pgfqpoint{1.966550in}{3.297919in}}%
\pgfpathlineto{\pgfqpoint{1.970964in}{3.310498in}}%
\pgfpathlineto{\pgfqpoint{1.975379in}{3.334975in}}%
\pgfpathlineto{\pgfqpoint{1.979793in}{3.309722in}}%
\pgfpathlineto{\pgfqpoint{1.984208in}{3.330991in}}%
\pgfpathlineto{\pgfqpoint{1.988622in}{3.338748in}}%
\pgfpathlineto{\pgfqpoint{1.993037in}{3.363223in}}%
\pgfpathlineto{\pgfqpoint{1.997451in}{3.363431in}}%
\pgfpathlineto{\pgfqpoint{2.001865in}{3.327252in}}%
\pgfpathlineto{\pgfqpoint{2.010694in}{3.353017in}}%
\pgfpathlineto{\pgfqpoint{2.015109in}{3.259892in}}%
\pgfpathlineto{\pgfqpoint{2.019523in}{3.356231in}}%
\pgfpathlineto{\pgfqpoint{2.023938in}{3.298079in}}%
\pgfpathlineto{\pgfqpoint{2.028352in}{3.279039in}}%
\pgfpathlineto{\pgfqpoint{2.032767in}{3.334464in}}%
\pgfpathlineto{\pgfqpoint{2.037181in}{3.348451in}}%
\pgfpathlineto{\pgfqpoint{2.041595in}{3.322478in}}%
\pgfpathlineto{\pgfqpoint{2.046010in}{3.352480in}}%
\pgfpathlineto{\pgfqpoint{2.050424in}{3.354049in}}%
\pgfpathlineto{\pgfqpoint{2.054839in}{3.342193in}}%
\pgfpathlineto{\pgfqpoint{2.059253in}{3.344954in}}%
\pgfpathlineto{\pgfqpoint{2.063668in}{3.323223in}}%
\pgfpathlineto{\pgfqpoint{2.068082in}{3.332692in}}%
\pgfpathlineto{\pgfqpoint{2.072497in}{3.295672in}}%
\pgfpathlineto{\pgfqpoint{2.076911in}{3.311150in}}%
\pgfpathlineto{\pgfqpoint{2.081325in}{3.314006in}}%
\pgfpathlineto{\pgfqpoint{2.085740in}{3.312370in}}%
\pgfpathlineto{\pgfqpoint{2.090154in}{3.266797in}}%
\pgfpathlineto{\pgfqpoint{2.094569in}{3.334396in}}%
\pgfpathlineto{\pgfqpoint{2.098983in}{3.339347in}}%
\pgfpathlineto{\pgfqpoint{2.103398in}{3.365897in}}%
\pgfpathlineto{\pgfqpoint{2.107812in}{3.359844in}}%
\pgfpathlineto{\pgfqpoint{2.112227in}{3.342482in}}%
\pgfpathlineto{\pgfqpoint{2.116641in}{3.337534in}}%
\pgfpathlineto{\pgfqpoint{2.121055in}{3.338622in}}%
\pgfpathlineto{\pgfqpoint{2.125470in}{3.288486in}}%
\pgfpathlineto{\pgfqpoint{2.129884in}{3.310146in}}%
\pgfpathlineto{\pgfqpoint{2.134299in}{3.348879in}}%
\pgfpathlineto{\pgfqpoint{2.138713in}{3.244077in}}%
\pgfpathlineto{\pgfqpoint{2.143128in}{3.305695in}}%
\pgfpathlineto{\pgfqpoint{2.147542in}{3.302052in}}%
\pgfpathlineto{\pgfqpoint{2.151957in}{3.322708in}}%
\pgfpathlineto{\pgfqpoint{2.156371in}{3.328298in}}%
\pgfpathlineto{\pgfqpoint{2.160785in}{3.357170in}}%
\pgfpathlineto{\pgfqpoint{2.165200in}{3.348656in}}%
\pgfpathlineto{\pgfqpoint{2.169614in}{3.356751in}}%
\pgfpathlineto{\pgfqpoint{2.174029in}{3.345727in}}%
\pgfpathlineto{\pgfqpoint{2.178443in}{3.344760in}}%
\pgfpathlineto{\pgfqpoint{2.182858in}{3.307773in}}%
\pgfpathlineto{\pgfqpoint{2.187272in}{3.360873in}}%
\pgfpathlineto{\pgfqpoint{2.191687in}{3.287142in}}%
\pgfpathlineto{\pgfqpoint{2.196101in}{3.323448in}}%
\pgfpathlineto{\pgfqpoint{2.200515in}{3.301897in}}%
\pgfpathlineto{\pgfqpoint{2.204930in}{3.332012in}}%
\pgfpathlineto{\pgfqpoint{2.209344in}{3.321480in}}%
\pgfpathlineto{\pgfqpoint{2.213759in}{3.322399in}}%
\pgfpathlineto{\pgfqpoint{2.218173in}{3.362256in}}%
\pgfpathlineto{\pgfqpoint{2.222588in}{3.342089in}}%
\pgfpathlineto{\pgfqpoint{2.227002in}{3.305172in}}%
\pgfpathlineto{\pgfqpoint{2.231417in}{3.343067in}}%
\pgfpathlineto{\pgfqpoint{2.235831in}{3.314605in}}%
\pgfpathlineto{\pgfqpoint{2.240246in}{3.334894in}}%
\pgfpathlineto{\pgfqpoint{2.244660in}{3.294171in}}%
\pgfpathlineto{\pgfqpoint{2.249074in}{3.334739in}}%
\pgfpathlineto{\pgfqpoint{2.253489in}{3.282576in}}%
\pgfpathlineto{\pgfqpoint{2.257903in}{3.312297in}}%
\pgfpathlineto{\pgfqpoint{2.262318in}{3.356447in}}%
\pgfpathlineto{\pgfqpoint{2.266732in}{3.320473in}}%
\pgfpathlineto{\pgfqpoint{2.271147in}{3.357386in}}%
\pgfpathlineto{\pgfqpoint{2.275561in}{3.371765in}}%
\pgfpathlineto{\pgfqpoint{2.279976in}{3.362751in}}%
\pgfpathlineto{\pgfqpoint{2.284390in}{3.371855in}}%
\pgfpathlineto{\pgfqpoint{2.288804in}{3.341079in}}%
\pgfpathlineto{\pgfqpoint{2.293219in}{3.301860in}}%
\pgfpathlineto{\pgfqpoint{2.297633in}{3.301425in}}%
\pgfpathlineto{\pgfqpoint{2.302048in}{3.273978in}}%
\pgfpathlineto{\pgfqpoint{2.306462in}{3.305732in}}%
\pgfpathlineto{\pgfqpoint{2.315291in}{3.326510in}}%
\pgfpathlineto{\pgfqpoint{2.319706in}{3.302892in}}%
\pgfpathlineto{\pgfqpoint{2.324120in}{3.357097in}}%
\pgfpathlineto{\pgfqpoint{2.328534in}{3.355115in}}%
\pgfpathlineto{\pgfqpoint{2.332949in}{3.312258in}}%
\pgfpathlineto{\pgfqpoint{2.337363in}{3.335138in}}%
\pgfpathlineto{\pgfqpoint{2.341778in}{3.307905in}}%
\pgfpathlineto{\pgfqpoint{2.346192in}{3.296302in}}%
\pgfpathlineto{\pgfqpoint{2.350607in}{3.319461in}}%
\pgfpathlineto{\pgfqpoint{2.359436in}{3.335150in}}%
\pgfpathlineto{\pgfqpoint{2.363850in}{3.305288in}}%
\pgfpathlineto{\pgfqpoint{2.368264in}{3.343236in}}%
\pgfpathlineto{\pgfqpoint{2.372679in}{3.341537in}}%
\pgfpathlineto{\pgfqpoint{2.381508in}{3.348789in}}%
\pgfpathlineto{\pgfqpoint{2.385922in}{3.358191in}}%
\pgfpathlineto{\pgfqpoint{2.390337in}{3.353172in}}%
\pgfpathlineto{\pgfqpoint{2.394751in}{3.366055in}}%
\pgfpathlineto{\pgfqpoint{2.403580in}{3.285961in}}%
\pgfpathlineto{\pgfqpoint{2.407994in}{3.325436in}}%
\pgfpathlineto{\pgfqpoint{2.412409in}{3.299687in}}%
\pgfpathlineto{\pgfqpoint{2.416823in}{3.283391in}}%
\pgfpathlineto{\pgfqpoint{2.421238in}{3.309851in}}%
\pgfpathlineto{\pgfqpoint{2.425652in}{3.349014in}}%
\pgfpathlineto{\pgfqpoint{2.430067in}{3.334756in}}%
\pgfpathlineto{\pgfqpoint{2.434481in}{3.362700in}}%
\pgfpathlineto{\pgfqpoint{2.438896in}{3.351741in}}%
\pgfpathlineto{\pgfqpoint{2.443310in}{3.333339in}}%
\pgfpathlineto{\pgfqpoint{2.447724in}{3.339732in}}%
\pgfpathlineto{\pgfqpoint{2.452139in}{3.332400in}}%
\pgfpathlineto{\pgfqpoint{2.456553in}{3.265113in}}%
\pgfpathlineto{\pgfqpoint{2.460968in}{3.343098in}}%
\pgfpathlineto{\pgfqpoint{2.465382in}{3.336696in}}%
\pgfpathlineto{\pgfqpoint{2.469797in}{3.303455in}}%
\pgfpathlineto{\pgfqpoint{2.478626in}{3.308875in}}%
\pgfpathlineto{\pgfqpoint{2.487454in}{3.355989in}}%
\pgfpathlineto{\pgfqpoint{2.491869in}{3.348072in}}%
\pgfpathlineto{\pgfqpoint{2.496283in}{3.345572in}}%
\pgfpathlineto{\pgfqpoint{2.505112in}{3.364519in}}%
\pgfpathlineto{\pgfqpoint{2.509527in}{3.333013in}}%
\pgfpathlineto{\pgfqpoint{2.513941in}{3.342617in}}%
\pgfpathlineto{\pgfqpoint{2.518356in}{3.341526in}}%
\pgfpathlineto{\pgfqpoint{2.522770in}{3.328180in}}%
\pgfpathlineto{\pgfqpoint{2.527184in}{3.305583in}}%
\pgfpathlineto{\pgfqpoint{2.531599in}{3.346677in}}%
\pgfpathlineto{\pgfqpoint{2.536013in}{3.359549in}}%
\pgfpathlineto{\pgfqpoint{2.544842in}{3.325267in}}%
\pgfpathlineto{\pgfqpoint{2.549257in}{3.361134in}}%
\pgfpathlineto{\pgfqpoint{2.553671in}{3.356782in}}%
\pgfpathlineto{\pgfqpoint{2.558086in}{3.222312in}}%
\pgfpathlineto{\pgfqpoint{2.562500in}{3.317299in}}%
\pgfpathlineto{\pgfqpoint{2.566914in}{3.278648in}}%
\pgfpathlineto{\pgfqpoint{2.571329in}{3.354027in}}%
\pgfpathlineto{\pgfqpoint{2.575743in}{3.322944in}}%
\pgfpathlineto{\pgfqpoint{2.584572in}{3.310751in}}%
\pgfpathlineto{\pgfqpoint{2.588987in}{3.308321in}}%
\pgfpathlineto{\pgfqpoint{2.593401in}{3.359534in}}%
\pgfpathlineto{\pgfqpoint{2.597816in}{3.357361in}}%
\pgfpathlineto{\pgfqpoint{2.602230in}{3.338279in}}%
\pgfpathlineto{\pgfqpoint{2.611059in}{3.366577in}}%
\pgfpathlineto{\pgfqpoint{2.615473in}{3.367609in}}%
\pgfpathlineto{\pgfqpoint{2.619888in}{3.322883in}}%
\pgfpathlineto{\pgfqpoint{2.624302in}{3.306393in}}%
\pgfpathlineto{\pgfqpoint{2.628717in}{3.363603in}}%
\pgfpathlineto{\pgfqpoint{2.633131in}{3.294500in}}%
\pgfpathlineto{\pgfqpoint{2.637546in}{3.337784in}}%
\pgfpathlineto{\pgfqpoint{2.641960in}{3.340618in}}%
\pgfpathlineto{\pgfqpoint{2.646374in}{3.322481in}}%
\pgfpathlineto{\pgfqpoint{2.650789in}{3.341889in}}%
\pgfpathlineto{\pgfqpoint{2.655203in}{3.313483in}}%
\pgfpathlineto{\pgfqpoint{2.659618in}{3.345148in}}%
\pgfpathlineto{\pgfqpoint{2.664032in}{3.347543in}}%
\pgfpathlineto{\pgfqpoint{2.668447in}{3.270449in}}%
\pgfpathlineto{\pgfqpoint{2.672861in}{3.326448in}}%
\pgfpathlineto{\pgfqpoint{2.677276in}{3.307427in}}%
\pgfpathlineto{\pgfqpoint{2.681690in}{3.347568in}}%
\pgfpathlineto{\pgfqpoint{2.686104in}{3.345634in}}%
\pgfpathlineto{\pgfqpoint{2.690519in}{3.324440in}}%
\pgfpathlineto{\pgfqpoint{2.694933in}{3.346475in}}%
\pgfpathlineto{\pgfqpoint{2.699348in}{3.352992in}}%
\pgfpathlineto{\pgfqpoint{2.703762in}{3.342434in}}%
\pgfpathlineto{\pgfqpoint{2.708177in}{3.327606in}}%
\pgfpathlineto{\pgfqpoint{2.712591in}{3.359335in}}%
\pgfpathlineto{\pgfqpoint{2.717006in}{3.362051in}}%
\pgfpathlineto{\pgfqpoint{2.721420in}{3.330772in}}%
\pgfpathlineto{\pgfqpoint{2.725834in}{3.362740in}}%
\pgfpathlineto{\pgfqpoint{2.730249in}{3.306632in}}%
\pgfpathlineto{\pgfqpoint{2.734663in}{3.344498in}}%
\pgfpathlineto{\pgfqpoint{2.739078in}{3.338630in}}%
\pgfpathlineto{\pgfqpoint{2.743492in}{3.296338in}}%
\pgfpathlineto{\pgfqpoint{2.747907in}{3.312075in}}%
\pgfpathlineto{\pgfqpoint{2.752321in}{3.310062in}}%
\pgfpathlineto{\pgfqpoint{2.756736in}{3.317923in}}%
\pgfpathlineto{\pgfqpoint{2.761150in}{3.355691in}}%
\pgfpathlineto{\pgfqpoint{2.765564in}{3.342775in}}%
\pgfpathlineto{\pgfqpoint{2.769979in}{3.340992in}}%
\pgfpathlineto{\pgfqpoint{2.774393in}{3.344096in}}%
\pgfpathlineto{\pgfqpoint{2.778808in}{3.353163in}}%
\pgfpathlineto{\pgfqpoint{2.783222in}{3.337421in}}%
\pgfpathlineto{\pgfqpoint{2.787637in}{3.289872in}}%
\pgfpathlineto{\pgfqpoint{2.792051in}{3.337556in}}%
\pgfpathlineto{\pgfqpoint{2.796466in}{3.335709in}}%
\pgfpathlineto{\pgfqpoint{2.800880in}{3.324893in}}%
\pgfpathlineto{\pgfqpoint{2.805294in}{3.310931in}}%
\pgfpathlineto{\pgfqpoint{2.809709in}{3.281940in}}%
\pgfpathlineto{\pgfqpoint{2.814123in}{3.317951in}}%
\pgfpathlineto{\pgfqpoint{2.818538in}{3.316441in}}%
\pgfpathlineto{\pgfqpoint{2.822952in}{3.358550in}}%
\pgfpathlineto{\pgfqpoint{2.827367in}{3.340868in}}%
\pgfpathlineto{\pgfqpoint{2.831781in}{3.330780in}}%
\pgfpathlineto{\pgfqpoint{2.836196in}{3.361444in}}%
\pgfpathlineto{\pgfqpoint{2.840610in}{3.311251in}}%
\pgfpathlineto{\pgfqpoint{2.845024in}{3.296845in}}%
\pgfpathlineto{\pgfqpoint{2.849439in}{3.332526in}}%
\pgfpathlineto{\pgfqpoint{2.853853in}{3.338116in}}%
\pgfpathlineto{\pgfqpoint{2.858268in}{3.282188in}}%
\pgfpathlineto{\pgfqpoint{2.862682in}{3.327662in}}%
\pgfpathlineto{\pgfqpoint{2.867097in}{3.354406in}}%
\pgfpathlineto{\pgfqpoint{2.871511in}{3.313261in}}%
\pgfpathlineto{\pgfqpoint{2.875926in}{3.357496in}}%
\pgfpathlineto{\pgfqpoint{2.880340in}{3.328236in}}%
\pgfpathlineto{\pgfqpoint{2.884754in}{3.354732in}}%
\pgfpathlineto{\pgfqpoint{2.889169in}{3.339181in}}%
\pgfpathlineto{\pgfqpoint{2.893583in}{3.297241in}}%
\pgfpathlineto{\pgfqpoint{2.897998in}{3.337902in}}%
\pgfpathlineto{\pgfqpoint{2.902412in}{3.346317in}}%
\pgfpathlineto{\pgfqpoint{2.906827in}{3.319194in}}%
\pgfpathlineto{\pgfqpoint{2.911241in}{3.320768in}}%
\pgfpathlineto{\pgfqpoint{2.915656in}{3.316886in}}%
\pgfpathlineto{\pgfqpoint{2.920070in}{3.288744in}}%
\pgfpathlineto{\pgfqpoint{2.928899in}{3.348983in}}%
\pgfpathlineto{\pgfqpoint{2.933313in}{3.341883in}}%
\pgfpathlineto{\pgfqpoint{2.937728in}{3.355792in}}%
\pgfpathlineto{\pgfqpoint{2.942142in}{3.361258in}}%
\pgfpathlineto{\pgfqpoint{2.946557in}{3.352972in}}%
\pgfpathlineto{\pgfqpoint{2.950971in}{3.316022in}}%
\pgfpathlineto{\pgfqpoint{2.959800in}{3.344270in}}%
\pgfpathlineto{\pgfqpoint{2.968629in}{3.325832in}}%
\pgfpathlineto{\pgfqpoint{2.973043in}{3.330181in}}%
\pgfpathlineto{\pgfqpoint{2.977458in}{3.345910in}}%
\pgfpathlineto{\pgfqpoint{2.981872in}{3.341414in}}%
\pgfpathlineto{\pgfqpoint{2.986287in}{3.354187in}}%
\pgfpathlineto{\pgfqpoint{2.990701in}{3.355134in}}%
\pgfpathlineto{\pgfqpoint{2.995116in}{3.358219in}}%
\pgfpathlineto{\pgfqpoint{2.999530in}{3.339136in}}%
\pgfpathlineto{\pgfqpoint{3.003945in}{3.331635in}}%
\pgfpathlineto{\pgfqpoint{3.008359in}{3.293260in}}%
\pgfpathlineto{\pgfqpoint{3.012773in}{3.380000in}}%
\pgfpathlineto{\pgfqpoint{3.017188in}{3.297654in}}%
\pgfpathlineto{\pgfqpoint{3.021602in}{3.351614in}}%
\pgfpathlineto{\pgfqpoint{3.026017in}{3.349989in}}%
\pgfpathlineto{\pgfqpoint{3.030431in}{3.328978in}}%
\pgfpathlineto{\pgfqpoint{3.039260in}{3.362239in}}%
\pgfpathlineto{\pgfqpoint{3.043675in}{3.357864in}}%
\pgfpathlineto{\pgfqpoint{3.052503in}{3.355202in}}%
\pgfpathlineto{\pgfqpoint{3.056918in}{3.348325in}}%
\pgfpathlineto{\pgfqpoint{3.061332in}{3.348595in}}%
\pgfpathlineto{\pgfqpoint{3.065747in}{3.336347in}}%
\pgfpathlineto{\pgfqpoint{3.070161in}{3.351294in}}%
\pgfpathlineto{\pgfqpoint{3.074576in}{3.325613in}}%
\pgfpathlineto{\pgfqpoint{3.078990in}{3.326898in}}%
\pgfpathlineto{\pgfqpoint{3.083405in}{3.308321in}}%
\pgfpathlineto{\pgfqpoint{3.087819in}{3.308276in}}%
\pgfpathlineto{\pgfqpoint{3.092233in}{3.354929in}}%
\pgfpathlineto{\pgfqpoint{3.096648in}{3.354822in}}%
\pgfpathlineto{\pgfqpoint{3.101062in}{3.342426in}}%
\pgfpathlineto{\pgfqpoint{3.105477in}{3.359470in}}%
\pgfpathlineto{\pgfqpoint{3.109891in}{3.319011in}}%
\pgfpathlineto{\pgfqpoint{3.114306in}{3.343508in}}%
\pgfpathlineto{\pgfqpoint{3.118720in}{3.309246in}}%
\pgfpathlineto{\pgfqpoint{3.123135in}{3.328793in}}%
\pgfpathlineto{\pgfqpoint{3.127549in}{3.318423in}}%
\pgfpathlineto{\pgfqpoint{3.131963in}{3.301045in}}%
\pgfpathlineto{\pgfqpoint{3.136378in}{3.346784in}}%
\pgfpathlineto{\pgfqpoint{3.140792in}{3.320504in}}%
\pgfpathlineto{\pgfqpoint{3.145207in}{3.286574in}}%
\pgfpathlineto{\pgfqpoint{3.149621in}{3.351229in}}%
\pgfpathlineto{\pgfqpoint{3.154036in}{3.367114in}}%
\pgfpathlineto{\pgfqpoint{3.158450in}{3.357929in}}%
\pgfpathlineto{\pgfqpoint{3.162865in}{3.332560in}}%
\pgfpathlineto{\pgfqpoint{3.167279in}{3.354088in}}%
\pgfpathlineto{\pgfqpoint{3.171693in}{3.309677in}}%
\pgfpathlineto{\pgfqpoint{3.176108in}{3.318024in}}%
\pgfpathlineto{\pgfqpoint{3.180522in}{3.341619in}}%
\pgfpathlineto{\pgfqpoint{3.184937in}{3.333173in}}%
\pgfpathlineto{\pgfqpoint{3.189351in}{3.327651in}}%
\pgfpathlineto{\pgfqpoint{3.193766in}{3.291761in}}%
\pgfpathlineto{\pgfqpoint{3.202595in}{3.339347in}}%
\pgfpathlineto{\pgfqpoint{3.207009in}{3.317650in}}%
\pgfpathlineto{\pgfqpoint{3.215838in}{3.354766in}}%
\pgfpathlineto{\pgfqpoint{3.220252in}{3.351339in}}%
\pgfpathlineto{\pgfqpoint{3.224667in}{3.343312in}}%
\pgfpathlineto{\pgfqpoint{3.229081in}{3.281027in}}%
\pgfpathlineto{\pgfqpoint{3.237910in}{3.348001in}}%
\pgfpathlineto{\pgfqpoint{3.242325in}{3.292548in}}%
\pgfpathlineto{\pgfqpoint{3.246739in}{3.325511in}}%
\pgfpathlineto{\pgfqpoint{3.251153in}{3.313992in}}%
\pgfpathlineto{\pgfqpoint{3.255568in}{3.265329in}}%
\pgfpathlineto{\pgfqpoint{3.259982in}{3.354153in}}%
\pgfpathlineto{\pgfqpoint{3.264397in}{3.361924in}}%
\pgfpathlineto{\pgfqpoint{3.268811in}{3.352792in}}%
\pgfpathlineto{\pgfqpoint{3.273226in}{3.349834in}}%
\pgfpathlineto{\pgfqpoint{3.277640in}{3.349331in}}%
\pgfpathlineto{\pgfqpoint{3.282055in}{3.295891in}}%
\pgfpathlineto{\pgfqpoint{3.286469in}{3.350782in}}%
\pgfpathlineto{\pgfqpoint{3.295298in}{3.322050in}}%
\pgfpathlineto{\pgfqpoint{3.299712in}{3.347667in}}%
\pgfpathlineto{\pgfqpoint{3.304127in}{3.302029in}}%
\pgfpathlineto{\pgfqpoint{3.312956in}{3.355534in}}%
\pgfpathlineto{\pgfqpoint{3.317370in}{3.340146in}}%
\pgfpathlineto{\pgfqpoint{3.321785in}{3.369620in}}%
\pgfpathlineto{\pgfqpoint{3.326199in}{3.356377in}}%
\pgfpathlineto{\pgfqpoint{3.330613in}{3.374169in}}%
\pgfpathlineto{\pgfqpoint{3.335028in}{3.307813in}}%
\pgfpathlineto{\pgfqpoint{3.339442in}{3.285539in}}%
\pgfpathlineto{\pgfqpoint{3.343857in}{3.372403in}}%
\pgfpathlineto{\pgfqpoint{3.348271in}{3.342401in}}%
\pgfpathlineto{\pgfqpoint{3.352686in}{3.337042in}}%
\pgfpathlineto{\pgfqpoint{3.357100in}{3.341937in}}%
\pgfpathlineto{\pgfqpoint{3.361515in}{3.330114in}}%
\pgfpathlineto{\pgfqpoint{3.365929in}{3.335793in}}%
\pgfpathlineto{\pgfqpoint{3.370343in}{3.355964in}}%
\pgfpathlineto{\pgfqpoint{3.379172in}{3.366901in}}%
\pgfpathlineto{\pgfqpoint{3.383587in}{3.363684in}}%
\pgfpathlineto{\pgfqpoint{3.388001in}{3.356332in}}%
\pgfpathlineto{\pgfqpoint{3.392416in}{3.312154in}}%
\pgfpathlineto{\pgfqpoint{3.396830in}{3.301096in}}%
\pgfpathlineto{\pgfqpoint{3.401245in}{3.283869in}}%
\pgfpathlineto{\pgfqpoint{3.405659in}{3.307419in}}%
\pgfpathlineto{\pgfqpoint{3.410073in}{3.338453in}}%
\pgfpathlineto{\pgfqpoint{3.414488in}{3.297840in}}%
\pgfpathlineto{\pgfqpoint{3.418902in}{3.324156in}}%
\pgfpathlineto{\pgfqpoint{3.423317in}{3.366108in}}%
\pgfpathlineto{\pgfqpoint{3.427731in}{3.364337in}}%
\pgfpathlineto{\pgfqpoint{3.432146in}{3.367275in}}%
\pgfpathlineto{\pgfqpoint{3.436560in}{3.355255in}}%
\pgfpathlineto{\pgfqpoint{3.440975in}{3.364753in}}%
\pgfpathlineto{\pgfqpoint{3.445389in}{3.323580in}}%
\pgfpathlineto{\pgfqpoint{3.449803in}{3.312044in}}%
\pgfpathlineto{\pgfqpoint{3.454218in}{3.344237in}}%
\pgfpathlineto{\pgfqpoint{3.458632in}{3.357266in}}%
\pgfpathlineto{\pgfqpoint{3.463047in}{3.292349in}}%
\pgfpathlineto{\pgfqpoint{3.467461in}{3.352331in}}%
\pgfpathlineto{\pgfqpoint{3.471876in}{3.339648in}}%
\pgfpathlineto{\pgfqpoint{3.476290in}{3.305004in}}%
\pgfpathlineto{\pgfqpoint{3.480705in}{3.345727in}}%
\pgfpathlineto{\pgfqpoint{3.485119in}{3.361677in}}%
\pgfpathlineto{\pgfqpoint{3.489533in}{3.370089in}}%
\pgfpathlineto{\pgfqpoint{3.493948in}{3.364179in}}%
\pgfpathlineto{\pgfqpoint{3.498362in}{3.339291in}}%
\pgfpathlineto{\pgfqpoint{3.507191in}{3.228332in}}%
\pgfpathlineto{\pgfqpoint{3.511606in}{3.360195in}}%
\pgfpathlineto{\pgfqpoint{3.516020in}{3.312921in}}%
\pgfpathlineto{\pgfqpoint{3.520435in}{3.339777in}}%
\pgfpathlineto{\pgfqpoint{3.524849in}{3.292560in}}%
\pgfpathlineto{\pgfqpoint{3.529263in}{3.315440in}}%
\pgfpathlineto{\pgfqpoint{3.533678in}{3.355711in}}%
\pgfpathlineto{\pgfqpoint{3.538092in}{3.355567in}}%
\pgfpathlineto{\pgfqpoint{3.542507in}{3.360774in}}%
\pgfpathlineto{\pgfqpoint{3.546921in}{3.357766in}}%
\pgfpathlineto{\pgfqpoint{3.551336in}{3.362984in}}%
\pgfpathlineto{\pgfqpoint{3.555750in}{3.281108in}}%
\pgfpathlineto{\pgfqpoint{3.560165in}{3.342904in}}%
\pgfpathlineto{\pgfqpoint{3.564579in}{3.361002in}}%
\pgfpathlineto{\pgfqpoint{3.568994in}{3.335852in}}%
\pgfpathlineto{\pgfqpoint{3.573408in}{3.296687in}}%
\pgfpathlineto{\pgfqpoint{3.577822in}{3.334441in}}%
\pgfpathlineto{\pgfqpoint{3.582237in}{3.324637in}}%
\pgfpathlineto{\pgfqpoint{3.586651in}{3.295217in}}%
\pgfpathlineto{\pgfqpoint{3.591066in}{3.357237in}}%
\pgfpathlineto{\pgfqpoint{3.595480in}{3.363035in}}%
\pgfpathlineto{\pgfqpoint{3.599895in}{3.362976in}}%
\pgfpathlineto{\pgfqpoint{3.604309in}{3.346469in}}%
\pgfpathlineto{\pgfqpoint{3.608724in}{3.321617in}}%
\pgfpathlineto{\pgfqpoint{3.613138in}{3.327058in}}%
\pgfpathlineto{\pgfqpoint{3.617552in}{3.262833in}}%
\pgfpathlineto{\pgfqpoint{3.621967in}{3.255042in}}%
\pgfpathlineto{\pgfqpoint{3.626381in}{3.350939in}}%
\pgfpathlineto{\pgfqpoint{3.630796in}{3.365065in}}%
\pgfpathlineto{\pgfqpoint{3.635210in}{3.334899in}}%
\pgfpathlineto{\pgfqpoint{3.644039in}{3.354347in}}%
\pgfpathlineto{\pgfqpoint{3.648454in}{3.355995in}}%
\pgfpathlineto{\pgfqpoint{3.652868in}{3.355430in}}%
\pgfpathlineto{\pgfqpoint{3.657282in}{3.358086in}}%
\pgfpathlineto{\pgfqpoint{3.661697in}{3.366999in}}%
\pgfpathlineto{\pgfqpoint{3.666111in}{3.315350in}}%
\pgfpathlineto{\pgfqpoint{3.670526in}{3.334064in}}%
\pgfpathlineto{\pgfqpoint{3.674940in}{3.360091in}}%
\pgfpathlineto{\pgfqpoint{3.679355in}{3.354696in}}%
\pgfpathlineto{\pgfqpoint{3.683769in}{3.334244in}}%
\pgfpathlineto{\pgfqpoint{3.688184in}{3.343601in}}%
\pgfpathlineto{\pgfqpoint{3.692598in}{3.344686in}}%
\pgfpathlineto{\pgfqpoint{3.697012in}{3.339963in}}%
\pgfpathlineto{\pgfqpoint{3.701427in}{3.360594in}}%
\pgfpathlineto{\pgfqpoint{3.705841in}{3.346691in}}%
\pgfpathlineto{\pgfqpoint{3.710256in}{3.361817in}}%
\pgfpathlineto{\pgfqpoint{3.714670in}{3.349466in}}%
\pgfpathlineto{\pgfqpoint{3.719085in}{3.342060in}}%
\pgfpathlineto{\pgfqpoint{3.723499in}{3.344532in}}%
\pgfpathlineto{\pgfqpoint{3.727914in}{3.306570in}}%
\pgfpathlineto{\pgfqpoint{3.732328in}{3.238622in}}%
\pgfpathlineto{\pgfqpoint{3.741157in}{3.337511in}}%
\pgfpathlineto{\pgfqpoint{3.749986in}{3.308603in}}%
\pgfpathlineto{\pgfqpoint{3.754400in}{3.356191in}}%
\pgfpathlineto{\pgfqpoint{3.758815in}{3.360032in}}%
\pgfpathlineto{\pgfqpoint{3.763229in}{3.351117in}}%
\pgfpathlineto{\pgfqpoint{3.767644in}{3.347121in}}%
\pgfpathlineto{\pgfqpoint{3.772058in}{3.359906in}}%
\pgfpathlineto{\pgfqpoint{3.776472in}{3.301652in}}%
\pgfpathlineto{\pgfqpoint{3.780887in}{3.288691in}}%
\pgfpathlineto{\pgfqpoint{3.785301in}{3.339817in}}%
\pgfpathlineto{\pgfqpoint{3.789716in}{3.343354in}}%
\pgfpathlineto{\pgfqpoint{3.794130in}{3.338478in}}%
\pgfpathlineto{\pgfqpoint{3.798545in}{3.362059in}}%
\pgfpathlineto{\pgfqpoint{3.802959in}{3.343865in}}%
\pgfpathlineto{\pgfqpoint{3.807374in}{3.314943in}}%
\pgfpathlineto{\pgfqpoint{3.811788in}{3.349326in}}%
\pgfpathlineto{\pgfqpoint{3.816202in}{3.347560in}}%
\pgfpathlineto{\pgfqpoint{3.820617in}{3.363288in}}%
\pgfpathlineto{\pgfqpoint{3.825031in}{3.361081in}}%
\pgfpathlineto{\pgfqpoint{3.829446in}{3.337427in}}%
\pgfpathlineto{\pgfqpoint{3.833860in}{3.325641in}}%
\pgfpathlineto{\pgfqpoint{3.838275in}{3.270348in}}%
\pgfpathlineto{\pgfqpoint{3.842689in}{3.320223in}}%
\pgfpathlineto{\pgfqpoint{3.847104in}{3.271400in}}%
\pgfpathlineto{\pgfqpoint{3.851518in}{3.312795in}}%
\pgfpathlineto{\pgfqpoint{3.855932in}{3.294190in}}%
\pgfpathlineto{\pgfqpoint{3.860347in}{3.308574in}}%
\pgfpathlineto{\pgfqpoint{3.864761in}{3.343368in}}%
\pgfpathlineto{\pgfqpoint{3.873590in}{3.342586in}}%
\pgfpathlineto{\pgfqpoint{3.878005in}{3.350124in}}%
\pgfpathlineto{\pgfqpoint{3.882419in}{3.360519in}}%
\pgfpathlineto{\pgfqpoint{3.886834in}{3.294663in}}%
\pgfpathlineto{\pgfqpoint{3.891248in}{3.353380in}}%
\pgfpathlineto{\pgfqpoint{3.895662in}{3.365883in}}%
\pgfpathlineto{\pgfqpoint{3.900077in}{3.329580in}}%
\pgfpathlineto{\pgfqpoint{3.904491in}{3.311543in}}%
\pgfpathlineto{\pgfqpoint{3.917735in}{3.315213in}}%
\pgfpathlineto{\pgfqpoint{3.922149in}{3.353878in}}%
\pgfpathlineto{\pgfqpoint{3.926564in}{3.356987in}}%
\pgfpathlineto{\pgfqpoint{3.930978in}{3.355334in}}%
\pgfpathlineto{\pgfqpoint{3.935392in}{3.374163in}}%
\pgfpathlineto{\pgfqpoint{3.939807in}{3.309533in}}%
\pgfpathlineto{\pgfqpoint{3.944221in}{3.310500in}}%
\pgfpathlineto{\pgfqpoint{3.948636in}{3.283371in}}%
\pgfpathlineto{\pgfqpoint{3.953050in}{3.358868in}}%
\pgfpathlineto{\pgfqpoint{3.957465in}{3.332043in}}%
\pgfpathlineto{\pgfqpoint{3.961879in}{3.328326in}}%
\pgfpathlineto{\pgfqpoint{3.966294in}{3.335422in}}%
\pgfpathlineto{\pgfqpoint{3.970708in}{3.337697in}}%
\pgfpathlineto{\pgfqpoint{3.975122in}{3.359669in}}%
\pgfpathlineto{\pgfqpoint{3.979537in}{3.335310in}}%
\pgfpathlineto{\pgfqpoint{3.983951in}{3.358733in}}%
\pgfpathlineto{\pgfqpoint{3.988366in}{3.358247in}}%
\pgfpathlineto{\pgfqpoint{3.992780in}{3.356296in}}%
\pgfpathlineto{\pgfqpoint{3.997195in}{3.336325in}}%
\pgfpathlineto{\pgfqpoint{4.001609in}{3.307939in}}%
\pgfpathlineto{\pgfqpoint{4.006024in}{3.348783in}}%
\pgfpathlineto{\pgfqpoint{4.010438in}{3.344307in}}%
\pgfpathlineto{\pgfqpoint{4.014852in}{3.329164in}}%
\pgfpathlineto{\pgfqpoint{4.019267in}{3.295754in}}%
\pgfpathlineto{\pgfqpoint{4.023681in}{3.336395in}}%
\pgfpathlineto{\pgfqpoint{4.028096in}{3.358938in}}%
\pgfpathlineto{\pgfqpoint{4.032510in}{3.362962in}}%
\pgfpathlineto{\pgfqpoint{4.036925in}{3.352044in}}%
\pgfpathlineto{\pgfqpoint{4.045754in}{3.362489in}}%
\pgfpathlineto{\pgfqpoint{4.050168in}{3.326889in}}%
\pgfpathlineto{\pgfqpoint{4.058997in}{3.288337in}}%
\pgfpathlineto{\pgfqpoint{4.063411in}{3.358070in}}%
\pgfpathlineto{\pgfqpoint{4.067826in}{3.303339in}}%
\pgfpathlineto{\pgfqpoint{4.072240in}{3.324556in}}%
\pgfpathlineto{\pgfqpoint{4.076655in}{3.355784in}}%
\pgfpathlineto{\pgfqpoint{4.081069in}{3.323130in}}%
\pgfpathlineto{\pgfqpoint{4.085484in}{3.371613in}}%
\pgfpathlineto{\pgfqpoint{4.089898in}{3.340112in}}%
\pgfpathlineto{\pgfqpoint{4.094312in}{3.364131in}}%
\pgfpathlineto{\pgfqpoint{4.098727in}{3.365113in}}%
\pgfpathlineto{\pgfqpoint{4.107556in}{3.339775in}}%
\pgfpathlineto{\pgfqpoint{4.111970in}{3.298633in}}%
\pgfpathlineto{\pgfqpoint{4.116385in}{3.335023in}}%
\pgfpathlineto{\pgfqpoint{4.120799in}{3.324207in}}%
\pgfpathlineto{\pgfqpoint{4.125214in}{3.318783in}}%
\pgfpathlineto{\pgfqpoint{4.129628in}{3.344180in}}%
\pgfpathlineto{\pgfqpoint{4.134042in}{3.359076in}}%
\pgfpathlineto{\pgfqpoint{4.138457in}{3.352736in}}%
\pgfpathlineto{\pgfqpoint{4.142871in}{3.362211in}}%
\pgfpathlineto{\pgfqpoint{4.147286in}{3.368877in}}%
\pgfpathlineto{\pgfqpoint{4.151700in}{3.351142in}}%
\pgfpathlineto{\pgfqpoint{4.156115in}{3.349674in}}%
\pgfpathlineto{\pgfqpoint{4.160529in}{3.288902in}}%
\pgfpathlineto{\pgfqpoint{4.164944in}{3.343160in}}%
\pgfpathlineto{\pgfqpoint{4.169358in}{3.355792in}}%
\pgfpathlineto{\pgfqpoint{4.173772in}{3.355261in}}%
\pgfpathlineto{\pgfqpoint{4.178187in}{3.305929in}}%
\pgfpathlineto{\pgfqpoint{4.182601in}{3.319472in}}%
\pgfpathlineto{\pgfqpoint{4.187016in}{3.328472in}}%
\pgfpathlineto{\pgfqpoint{4.191430in}{3.313616in}}%
\pgfpathlineto{\pgfqpoint{4.195845in}{3.365383in}}%
\pgfpathlineto{\pgfqpoint{4.200259in}{3.343894in}}%
\pgfpathlineto{\pgfqpoint{4.204674in}{3.341045in}}%
\pgfpathlineto{\pgfqpoint{4.209088in}{3.347380in}}%
\pgfpathlineto{\pgfqpoint{4.213503in}{3.355854in}}%
\pgfpathlineto{\pgfqpoint{4.217917in}{3.332310in}}%
\pgfpathlineto{\pgfqpoint{4.222331in}{3.316728in}}%
\pgfpathlineto{\pgfqpoint{4.226746in}{3.351448in}}%
\pgfpathlineto{\pgfqpoint{4.231160in}{3.325478in}}%
\pgfpathlineto{\pgfqpoint{4.235575in}{3.270947in}}%
\pgfpathlineto{\pgfqpoint{4.239989in}{3.330651in}}%
\pgfpathlineto{\pgfqpoint{4.244404in}{3.348457in}}%
\pgfpathlineto{\pgfqpoint{4.248818in}{3.321581in}}%
\pgfpathlineto{\pgfqpoint{4.253233in}{3.364674in}}%
\pgfpathlineto{\pgfqpoint{4.257647in}{3.360459in}}%
\pgfpathlineto{\pgfqpoint{4.262061in}{3.351190in}}%
\pgfpathlineto{\pgfqpoint{4.266476in}{3.358806in}}%
\pgfpathlineto{\pgfqpoint{4.270890in}{3.177962in}}%
\pgfpathlineto{\pgfqpoint{4.275305in}{3.339429in}}%
\pgfpathlineto{\pgfqpoint{4.279719in}{3.279924in}}%
\pgfpathlineto{\pgfqpoint{4.284134in}{3.327702in}}%
\pgfpathlineto{\pgfqpoint{4.288548in}{3.290935in}}%
\pgfpathlineto{\pgfqpoint{4.292963in}{3.340424in}}%
\pgfpathlineto{\pgfqpoint{4.297377in}{3.351935in}}%
\pgfpathlineto{\pgfqpoint{4.301791in}{3.330488in}}%
\pgfpathlineto{\pgfqpoint{4.306206in}{3.353523in}}%
\pgfpathlineto{\pgfqpoint{4.310620in}{3.343050in}}%
\pgfpathlineto{\pgfqpoint{4.315035in}{3.360668in}}%
\pgfpathlineto{\pgfqpoint{4.319449in}{3.336491in}}%
\pgfpathlineto{\pgfqpoint{4.323864in}{3.350169in}}%
\pgfpathlineto{\pgfqpoint{4.323864in}{3.350169in}}%
\pgfusepath{stroke}%
\end{pgfscope}%
\begin{pgfscope}%
\pgfpathrectangle{\pgfqpoint{0.625000in}{0.440000in}}{\pgfqpoint{3.875000in}{3.080000in}} %
\pgfusepath{clip}%
\pgfsetrectcap%
\pgfsetroundjoin%
\pgfsetlinewidth{1.505625pt}%
\definecolor{currentstroke}{rgb}{1.000000,0.894118,0.882353}%
\pgfsetstrokecolor{currentstroke}%
\pgfsetdash{}{0pt}%
\pgfpathmoveto{\pgfqpoint{0.801136in}{0.580000in}}%
\pgfpathlineto{\pgfqpoint{0.805551in}{0.580000in}}%
\pgfpathlineto{\pgfqpoint{0.809965in}{0.580000in}}%
\pgfpathlineto{\pgfqpoint{0.814380in}{0.649677in}}%
\pgfpathlineto{\pgfqpoint{0.818794in}{1.169567in}}%
\pgfpathlineto{\pgfqpoint{0.823209in}{2.002711in}}%
\pgfpathlineto{\pgfqpoint{0.827623in}{2.120378in}}%
\pgfpathlineto{\pgfqpoint{0.832037in}{1.381977in}}%
\pgfpathlineto{\pgfqpoint{0.840866in}{1.729214in}}%
\pgfpathlineto{\pgfqpoint{0.845281in}{1.558924in}}%
\pgfpathlineto{\pgfqpoint{0.849695in}{1.132865in}}%
\pgfpathlineto{\pgfqpoint{0.854110in}{0.951889in}}%
\pgfpathlineto{\pgfqpoint{0.858524in}{1.617239in}}%
\pgfpathlineto{\pgfqpoint{0.862939in}{1.735824in}}%
\pgfpathlineto{\pgfqpoint{0.867353in}{1.594089in}}%
\pgfpathlineto{\pgfqpoint{0.871767in}{1.974260in}}%
\pgfpathlineto{\pgfqpoint{0.876182in}{1.944879in}}%
\pgfpathlineto{\pgfqpoint{0.880596in}{1.474478in}}%
\pgfpathlineto{\pgfqpoint{0.885011in}{1.815493in}}%
\pgfpathlineto{\pgfqpoint{0.889425in}{1.552519in}}%
\pgfpathlineto{\pgfqpoint{0.893840in}{1.977434in}}%
\pgfpathlineto{\pgfqpoint{0.898254in}{1.315202in}}%
\pgfpathlineto{\pgfqpoint{0.902669in}{1.109039in}}%
\pgfpathlineto{\pgfqpoint{0.907083in}{1.028167in}}%
\pgfpathlineto{\pgfqpoint{0.911497in}{1.119327in}}%
\pgfpathlineto{\pgfqpoint{0.915912in}{1.392096in}}%
\pgfpathlineto{\pgfqpoint{0.920326in}{2.107051in}}%
\pgfpathlineto{\pgfqpoint{0.924741in}{2.031999in}}%
\pgfpathlineto{\pgfqpoint{0.929155in}{2.013752in}}%
\pgfpathlineto{\pgfqpoint{0.933570in}{2.297286in}}%
\pgfpathlineto{\pgfqpoint{0.937984in}{1.863709in}}%
\pgfpathlineto{\pgfqpoint{0.942399in}{2.079307in}}%
\pgfpathlineto{\pgfqpoint{0.946813in}{1.734929in}}%
\pgfpathlineto{\pgfqpoint{0.951228in}{1.899222in}}%
\pgfpathlineto{\pgfqpoint{0.955642in}{1.937097in}}%
\pgfpathlineto{\pgfqpoint{0.960056in}{1.107440in}}%
\pgfpathlineto{\pgfqpoint{0.964471in}{0.910713in}}%
\pgfpathlineto{\pgfqpoint{0.968885in}{1.697507in}}%
\pgfpathlineto{\pgfqpoint{0.973300in}{1.709347in}}%
\pgfpathlineto{\pgfqpoint{0.977714in}{1.588913in}}%
\pgfpathlineto{\pgfqpoint{0.982129in}{1.919994in}}%
\pgfpathlineto{\pgfqpoint{0.986543in}{1.904035in}}%
\pgfpathlineto{\pgfqpoint{0.990958in}{1.897585in}}%
\pgfpathlineto{\pgfqpoint{0.995372in}{2.115672in}}%
\pgfpathlineto{\pgfqpoint{0.999786in}{1.743912in}}%
\pgfpathlineto{\pgfqpoint{1.004201in}{1.855091in}}%
\pgfpathlineto{\pgfqpoint{1.008615in}{1.004457in}}%
\pgfpathlineto{\pgfqpoint{1.013030in}{1.425961in}}%
\pgfpathlineto{\pgfqpoint{1.017444in}{1.147274in}}%
\pgfpathlineto{\pgfqpoint{1.021859in}{1.244212in}}%
\pgfpathlineto{\pgfqpoint{1.026273in}{1.690284in}}%
\pgfpathlineto{\pgfqpoint{1.030688in}{1.940113in}}%
\pgfpathlineto{\pgfqpoint{1.035102in}{1.319945in}}%
\pgfpathlineto{\pgfqpoint{1.039516in}{1.639321in}}%
\pgfpathlineto{\pgfqpoint{1.043931in}{1.621355in}}%
\pgfpathlineto{\pgfqpoint{1.048345in}{1.973082in}}%
\pgfpathlineto{\pgfqpoint{1.052760in}{1.999916in}}%
\pgfpathlineto{\pgfqpoint{1.057174in}{2.110884in}}%
\pgfpathlineto{\pgfqpoint{1.061589in}{1.633012in}}%
\pgfpathlineto{\pgfqpoint{1.066003in}{1.534286in}}%
\pgfpathlineto{\pgfqpoint{1.070418in}{1.259698in}}%
\pgfpathlineto{\pgfqpoint{1.074832in}{1.338485in}}%
\pgfpathlineto{\pgfqpoint{1.079246in}{1.931111in}}%
\pgfpathlineto{\pgfqpoint{1.083661in}{1.842934in}}%
\pgfpathlineto{\pgfqpoint{1.088075in}{1.950536in}}%
\pgfpathlineto{\pgfqpoint{1.092490in}{1.924852in}}%
\pgfpathlineto{\pgfqpoint{1.096904in}{2.500997in}}%
\pgfpathlineto{\pgfqpoint{1.101319in}{2.463277in}}%
\pgfpathlineto{\pgfqpoint{1.105733in}{1.738641in}}%
\pgfpathlineto{\pgfqpoint{1.110148in}{1.684672in}}%
\pgfpathlineto{\pgfqpoint{1.114562in}{1.660099in}}%
\pgfpathlineto{\pgfqpoint{1.118976in}{1.459883in}}%
\pgfpathlineto{\pgfqpoint{1.123391in}{1.519244in}}%
\pgfpathlineto{\pgfqpoint{1.127805in}{1.281418in}}%
\pgfpathlineto{\pgfqpoint{1.132220in}{1.414619in}}%
\pgfpathlineto{\pgfqpoint{1.136634in}{1.642521in}}%
\pgfpathlineto{\pgfqpoint{1.141049in}{1.629484in}}%
\pgfpathlineto{\pgfqpoint{1.145463in}{0.792572in}}%
\pgfpathlineto{\pgfqpoint{1.149878in}{1.869320in}}%
\pgfpathlineto{\pgfqpoint{1.154292in}{2.057944in}}%
\pgfpathlineto{\pgfqpoint{1.158706in}{2.391131in}}%
\pgfpathlineto{\pgfqpoint{1.163121in}{2.012090in}}%
\pgfpathlineto{\pgfqpoint{1.167535in}{1.907285in}}%
\pgfpathlineto{\pgfqpoint{1.171950in}{1.485053in}}%
\pgfpathlineto{\pgfqpoint{1.176364in}{1.702756in}}%
\pgfpathlineto{\pgfqpoint{1.180779in}{1.298532in}}%
\pgfpathlineto{\pgfqpoint{1.185193in}{1.302178in}}%
\pgfpathlineto{\pgfqpoint{1.189608in}{2.016600in}}%
\pgfpathlineto{\pgfqpoint{1.194022in}{1.983814in}}%
\pgfpathlineto{\pgfqpoint{1.198436in}{1.854425in}}%
\pgfpathlineto{\pgfqpoint{1.202851in}{1.883288in}}%
\pgfpathlineto{\pgfqpoint{1.207265in}{2.293971in}}%
\pgfpathlineto{\pgfqpoint{1.211680in}{1.905286in}}%
\pgfpathlineto{\pgfqpoint{1.216094in}{1.683489in}}%
\pgfpathlineto{\pgfqpoint{1.220509in}{1.995448in}}%
\pgfpathlineto{\pgfqpoint{1.224923in}{1.936236in}}%
\pgfpathlineto{\pgfqpoint{1.229338in}{1.928971in}}%
\pgfpathlineto{\pgfqpoint{1.238166in}{0.972883in}}%
\pgfpathlineto{\pgfqpoint{1.242581in}{1.707244in}}%
\pgfpathlineto{\pgfqpoint{1.246995in}{1.395922in}}%
\pgfpathlineto{\pgfqpoint{1.251410in}{2.194315in}}%
\pgfpathlineto{\pgfqpoint{1.255824in}{0.822874in}}%
\pgfpathlineto{\pgfqpoint{1.260239in}{2.442493in}}%
\pgfpathlineto{\pgfqpoint{1.264653in}{2.172612in}}%
\pgfpathlineto{\pgfqpoint{1.269068in}{2.503429in}}%
\pgfpathlineto{\pgfqpoint{1.273482in}{2.210062in}}%
\pgfpathlineto{\pgfqpoint{1.282311in}{1.794783in}}%
\pgfpathlineto{\pgfqpoint{1.286725in}{2.253746in}}%
\pgfpathlineto{\pgfqpoint{1.291140in}{1.591398in}}%
\pgfpathlineto{\pgfqpoint{1.295554in}{1.099927in}}%
\pgfpathlineto{\pgfqpoint{1.304383in}{2.012444in}}%
\pgfpathlineto{\pgfqpoint{1.308798in}{1.213585in}}%
\pgfpathlineto{\pgfqpoint{1.313212in}{1.742082in}}%
\pgfpathlineto{\pgfqpoint{1.317626in}{2.006129in}}%
\pgfpathlineto{\pgfqpoint{1.322041in}{1.729233in}}%
\pgfpathlineto{\pgfqpoint{1.326455in}{2.459065in}}%
\pgfpathlineto{\pgfqpoint{1.335284in}{1.928060in}}%
\pgfpathlineto{\pgfqpoint{1.344113in}{1.647883in}}%
\pgfpathlineto{\pgfqpoint{1.348528in}{0.979692in}}%
\pgfpathlineto{\pgfqpoint{1.352942in}{1.222006in}}%
\pgfpathlineto{\pgfqpoint{1.361771in}{1.883612in}}%
\pgfpathlineto{\pgfqpoint{1.366185in}{0.764448in}}%
\pgfpathlineto{\pgfqpoint{1.370600in}{1.878143in}}%
\pgfpathlineto{\pgfqpoint{1.375014in}{1.975427in}}%
\pgfpathlineto{\pgfqpoint{1.379429in}{1.972315in}}%
\pgfpathlineto{\pgfqpoint{1.383843in}{2.119667in}}%
\pgfpathlineto{\pgfqpoint{1.388258in}{2.439378in}}%
\pgfpathlineto{\pgfqpoint{1.392672in}{1.763177in}}%
\pgfpathlineto{\pgfqpoint{1.397086in}{1.985709in}}%
\pgfpathlineto{\pgfqpoint{1.401501in}{1.456301in}}%
\pgfpathlineto{\pgfqpoint{1.405915in}{1.441583in}}%
\pgfpathlineto{\pgfqpoint{1.410330in}{2.168175in}}%
\pgfpathlineto{\pgfqpoint{1.414744in}{2.215073in}}%
\pgfpathlineto{\pgfqpoint{1.419159in}{0.833464in}}%
\pgfpathlineto{\pgfqpoint{1.423573in}{2.134765in}}%
\pgfpathlineto{\pgfqpoint{1.427988in}{2.211344in}}%
\pgfpathlineto{\pgfqpoint{1.432402in}{2.207518in}}%
\pgfpathlineto{\pgfqpoint{1.436816in}{2.347273in}}%
\pgfpathlineto{\pgfqpoint{1.441231in}{2.117449in}}%
\pgfpathlineto{\pgfqpoint{1.445645in}{2.312933in}}%
\pgfpathlineto{\pgfqpoint{1.454474in}{1.641886in}}%
\pgfpathlineto{\pgfqpoint{1.458889in}{1.607598in}}%
\pgfpathlineto{\pgfqpoint{1.463303in}{1.629562in}}%
\pgfpathlineto{\pgfqpoint{1.467718in}{1.985417in}}%
\pgfpathlineto{\pgfqpoint{1.472132in}{1.514113in}}%
\pgfpathlineto{\pgfqpoint{1.476546in}{0.693511in}}%
\pgfpathlineto{\pgfqpoint{1.480961in}{2.214448in}}%
\pgfpathlineto{\pgfqpoint{1.485375in}{1.569080in}}%
\pgfpathlineto{\pgfqpoint{1.489790in}{2.170599in}}%
\pgfpathlineto{\pgfqpoint{1.498619in}{2.564997in}}%
\pgfpathlineto{\pgfqpoint{1.503033in}{1.858493in}}%
\pgfpathlineto{\pgfqpoint{1.507448in}{2.107982in}}%
\pgfpathlineto{\pgfqpoint{1.511862in}{1.400699in}}%
\pgfpathlineto{\pgfqpoint{1.516276in}{2.088335in}}%
\pgfpathlineto{\pgfqpoint{1.520691in}{2.519944in}}%
\pgfpathlineto{\pgfqpoint{1.525105in}{2.080572in}}%
\pgfpathlineto{\pgfqpoint{1.529520in}{0.762323in}}%
\pgfpathlineto{\pgfqpoint{1.533934in}{2.438029in}}%
\pgfpathlineto{\pgfqpoint{1.538349in}{2.436462in}}%
\pgfpathlineto{\pgfqpoint{1.542763in}{2.549353in}}%
\pgfpathlineto{\pgfqpoint{1.547178in}{2.193901in}}%
\pgfpathlineto{\pgfqpoint{1.551592in}{2.138131in}}%
\pgfpathlineto{\pgfqpoint{1.556006in}{2.538278in}}%
\pgfpathlineto{\pgfqpoint{1.560421in}{2.039326in}}%
\pgfpathlineto{\pgfqpoint{1.564835in}{1.767066in}}%
\pgfpathlineto{\pgfqpoint{1.569250in}{2.180116in}}%
\pgfpathlineto{\pgfqpoint{1.573664in}{2.438220in}}%
\pgfpathlineto{\pgfqpoint{1.578079in}{2.198422in}}%
\pgfpathlineto{\pgfqpoint{1.582493in}{2.144353in}}%
\pgfpathlineto{\pgfqpoint{1.586908in}{0.783415in}}%
\pgfpathlineto{\pgfqpoint{1.591322in}{2.519382in}}%
\pgfpathlineto{\pgfqpoint{1.595737in}{2.193623in}}%
\pgfpathlineto{\pgfqpoint{1.600151in}{2.687987in}}%
\pgfpathlineto{\pgfqpoint{1.604565in}{2.816603in}}%
\pgfpathlineto{\pgfqpoint{1.613394in}{2.070970in}}%
\pgfpathlineto{\pgfqpoint{1.617809in}{2.279683in}}%
\pgfpathlineto{\pgfqpoint{1.622223in}{1.582750in}}%
\pgfpathlineto{\pgfqpoint{1.626638in}{2.176630in}}%
\pgfpathlineto{\pgfqpoint{1.631052in}{2.564044in}}%
\pgfpathlineto{\pgfqpoint{1.635467in}{2.413303in}}%
\pgfpathlineto{\pgfqpoint{1.639881in}{0.779722in}}%
\pgfpathlineto{\pgfqpoint{1.644295in}{2.283422in}}%
\pgfpathlineto{\pgfqpoint{1.648710in}{2.477408in}}%
\pgfpathlineto{\pgfqpoint{1.653124in}{2.595446in}}%
\pgfpathlineto{\pgfqpoint{1.657539in}{2.268082in}}%
\pgfpathlineto{\pgfqpoint{1.661953in}{2.642943in}}%
\pgfpathlineto{\pgfqpoint{1.666368in}{2.622117in}}%
\pgfpathlineto{\pgfqpoint{1.670782in}{1.807362in}}%
\pgfpathlineto{\pgfqpoint{1.679611in}{2.449680in}}%
\pgfpathlineto{\pgfqpoint{1.684025in}{2.364354in}}%
\pgfpathlineto{\pgfqpoint{1.688440in}{2.495188in}}%
\pgfpathlineto{\pgfqpoint{1.692854in}{2.511692in}}%
\pgfpathlineto{\pgfqpoint{1.697269in}{1.048571in}}%
\pgfpathlineto{\pgfqpoint{1.701683in}{2.455494in}}%
\pgfpathlineto{\pgfqpoint{1.706098in}{2.611796in}}%
\pgfpathlineto{\pgfqpoint{1.710512in}{2.541307in}}%
\pgfpathlineto{\pgfqpoint{1.714927in}{2.793129in}}%
\pgfpathlineto{\pgfqpoint{1.719341in}{2.744126in}}%
\pgfpathlineto{\pgfqpoint{1.723755in}{1.704871in}}%
\pgfpathlineto{\pgfqpoint{1.728170in}{1.822345in}}%
\pgfpathlineto{\pgfqpoint{1.732584in}{1.731513in}}%
\pgfpathlineto{\pgfqpoint{1.736999in}{2.108283in}}%
\pgfpathlineto{\pgfqpoint{1.741413in}{2.369940in}}%
\pgfpathlineto{\pgfqpoint{1.750242in}{0.938036in}}%
\pgfpathlineto{\pgfqpoint{1.754657in}{2.546320in}}%
\pgfpathlineto{\pgfqpoint{1.759071in}{2.556970in}}%
\pgfpathlineto{\pgfqpoint{1.763485in}{2.670150in}}%
\pgfpathlineto{\pgfqpoint{1.767900in}{2.328737in}}%
\pgfpathlineto{\pgfqpoint{1.772314in}{2.871643in}}%
\pgfpathlineto{\pgfqpoint{1.776729in}{2.538669in}}%
\pgfpathlineto{\pgfqpoint{1.781143in}{2.335015in}}%
\pgfpathlineto{\pgfqpoint{1.785558in}{1.361177in}}%
\pgfpathlineto{\pgfqpoint{1.794387in}{2.538942in}}%
\pgfpathlineto{\pgfqpoint{1.798801in}{2.453332in}}%
\pgfpathlineto{\pgfqpoint{1.803215in}{2.455809in}}%
\pgfpathlineto{\pgfqpoint{1.807630in}{0.961875in}}%
\pgfpathlineto{\pgfqpoint{1.812044in}{2.598612in}}%
\pgfpathlineto{\pgfqpoint{1.816459in}{2.545501in}}%
\pgfpathlineto{\pgfqpoint{1.820873in}{2.104259in}}%
\pgfpathlineto{\pgfqpoint{1.825288in}{2.329690in}}%
\pgfpathlineto{\pgfqpoint{1.829702in}{2.409511in}}%
\pgfpathlineto{\pgfqpoint{1.834117in}{1.704016in}}%
\pgfpathlineto{\pgfqpoint{1.838531in}{2.083622in}}%
\pgfpathlineto{\pgfqpoint{1.842945in}{1.965038in}}%
\pgfpathlineto{\pgfqpoint{1.847360in}{2.357235in}}%
\pgfpathlineto{\pgfqpoint{1.851774in}{2.155340in}}%
\pgfpathlineto{\pgfqpoint{1.856189in}{1.802647in}}%
\pgfpathlineto{\pgfqpoint{1.860603in}{0.774909in}}%
\pgfpathlineto{\pgfqpoint{1.865018in}{2.590914in}}%
\pgfpathlineto{\pgfqpoint{1.869432in}{2.556866in}}%
\pgfpathlineto{\pgfqpoint{1.873847in}{2.832365in}}%
\pgfpathlineto{\pgfqpoint{1.878261in}{2.389262in}}%
\pgfpathlineto{\pgfqpoint{1.882675in}{2.655876in}}%
\pgfpathlineto{\pgfqpoint{1.887090in}{2.666532in}}%
\pgfpathlineto{\pgfqpoint{1.891504in}{2.310585in}}%
\pgfpathlineto{\pgfqpoint{1.895919in}{1.760962in}}%
\pgfpathlineto{\pgfqpoint{1.900333in}{2.352686in}}%
\pgfpathlineto{\pgfqpoint{1.909162in}{2.684301in}}%
\pgfpathlineto{\pgfqpoint{1.913577in}{2.680767in}}%
\pgfpathlineto{\pgfqpoint{1.917991in}{1.368543in}}%
\pgfpathlineto{\pgfqpoint{1.922405in}{2.751065in}}%
\pgfpathlineto{\pgfqpoint{1.926820in}{2.535936in}}%
\pgfpathlineto{\pgfqpoint{1.931234in}{2.414417in}}%
\pgfpathlineto{\pgfqpoint{1.935649in}{2.191545in}}%
\pgfpathlineto{\pgfqpoint{1.940063in}{2.241802in}}%
\pgfpathlineto{\pgfqpoint{1.944478in}{1.990601in}}%
\pgfpathlineto{\pgfqpoint{1.948892in}{2.209247in}}%
\pgfpathlineto{\pgfqpoint{1.953307in}{2.065080in}}%
\pgfpathlineto{\pgfqpoint{1.957721in}{2.566349in}}%
\pgfpathlineto{\pgfqpoint{1.962135in}{2.041764in}}%
\pgfpathlineto{\pgfqpoint{1.966550in}{1.888555in}}%
\pgfpathlineto{\pgfqpoint{1.970964in}{0.937060in}}%
\pgfpathlineto{\pgfqpoint{1.975379in}{2.495396in}}%
\pgfpathlineto{\pgfqpoint{1.979793in}{2.349686in}}%
\pgfpathlineto{\pgfqpoint{1.984208in}{2.258242in}}%
\pgfpathlineto{\pgfqpoint{1.988622in}{2.453782in}}%
\pgfpathlineto{\pgfqpoint{1.993037in}{2.528438in}}%
\pgfpathlineto{\pgfqpoint{1.997451in}{2.814714in}}%
\pgfpathlineto{\pgfqpoint{2.001865in}{2.068988in}}%
\pgfpathlineto{\pgfqpoint{2.006280in}{1.997363in}}%
\pgfpathlineto{\pgfqpoint{2.010694in}{1.982456in}}%
\pgfpathlineto{\pgfqpoint{2.015109in}{2.058523in}}%
\pgfpathlineto{\pgfqpoint{2.019523in}{2.771339in}}%
\pgfpathlineto{\pgfqpoint{2.023938in}{2.291922in}}%
\pgfpathlineto{\pgfqpoint{2.032767in}{2.616876in}}%
\pgfpathlineto{\pgfqpoint{2.037181in}{2.534019in}}%
\pgfpathlineto{\pgfqpoint{2.041595in}{2.245753in}}%
\pgfpathlineto{\pgfqpoint{2.046010in}{2.366274in}}%
\pgfpathlineto{\pgfqpoint{2.050424in}{2.708458in}}%
\pgfpathlineto{\pgfqpoint{2.054839in}{1.959935in}}%
\pgfpathlineto{\pgfqpoint{2.059253in}{1.724991in}}%
\pgfpathlineto{\pgfqpoint{2.063668in}{1.972205in}}%
\pgfpathlineto{\pgfqpoint{2.068082in}{2.626883in}}%
\pgfpathlineto{\pgfqpoint{2.072497in}{2.148733in}}%
\pgfpathlineto{\pgfqpoint{2.076911in}{2.010454in}}%
\pgfpathlineto{\pgfqpoint{2.081325in}{0.974139in}}%
\pgfpathlineto{\pgfqpoint{2.085740in}{2.522823in}}%
\pgfpathlineto{\pgfqpoint{2.094569in}{2.219934in}}%
\pgfpathlineto{\pgfqpoint{2.098983in}{2.688538in}}%
\pgfpathlineto{\pgfqpoint{2.103398in}{2.412249in}}%
\pgfpathlineto{\pgfqpoint{2.112227in}{2.070532in}}%
\pgfpathlineto{\pgfqpoint{2.116641in}{1.568956in}}%
\pgfpathlineto{\pgfqpoint{2.121055in}{2.206927in}}%
\pgfpathlineto{\pgfqpoint{2.125470in}{1.977423in}}%
\pgfpathlineto{\pgfqpoint{2.129884in}{2.698927in}}%
\pgfpathlineto{\pgfqpoint{2.134299in}{2.298861in}}%
\pgfpathlineto{\pgfqpoint{2.143128in}{2.587200in}}%
\pgfpathlineto{\pgfqpoint{2.147542in}{2.443258in}}%
\pgfpathlineto{\pgfqpoint{2.151957in}{2.061318in}}%
\pgfpathlineto{\pgfqpoint{2.156371in}{2.325312in}}%
\pgfpathlineto{\pgfqpoint{2.160785in}{2.520130in}}%
\pgfpathlineto{\pgfqpoint{2.169614in}{1.532627in}}%
\pgfpathlineto{\pgfqpoint{2.178443in}{2.632070in}}%
\pgfpathlineto{\pgfqpoint{2.182858in}{2.016021in}}%
\pgfpathlineto{\pgfqpoint{2.187272in}{1.904791in}}%
\pgfpathlineto{\pgfqpoint{2.191687in}{1.466372in}}%
\pgfpathlineto{\pgfqpoint{2.196101in}{2.637013in}}%
\pgfpathlineto{\pgfqpoint{2.200515in}{2.591088in}}%
\pgfpathlineto{\pgfqpoint{2.204930in}{2.505911in}}%
\pgfpathlineto{\pgfqpoint{2.209344in}{2.529259in}}%
\pgfpathlineto{\pgfqpoint{2.213759in}{2.080167in}}%
\pgfpathlineto{\pgfqpoint{2.218173in}{2.433229in}}%
\pgfpathlineto{\pgfqpoint{2.222588in}{2.023795in}}%
\pgfpathlineto{\pgfqpoint{2.227002in}{1.405814in}}%
\pgfpathlineto{\pgfqpoint{2.231417in}{2.490687in}}%
\pgfpathlineto{\pgfqpoint{2.235831in}{2.785378in}}%
\pgfpathlineto{\pgfqpoint{2.244660in}{2.406364in}}%
\pgfpathlineto{\pgfqpoint{2.249074in}{2.778863in}}%
\pgfpathlineto{\pgfqpoint{2.253489in}{2.610134in}}%
\pgfpathlineto{\pgfqpoint{2.257903in}{2.696779in}}%
\pgfpathlineto{\pgfqpoint{2.262318in}{2.603133in}}%
\pgfpathlineto{\pgfqpoint{2.266732in}{2.217642in}}%
\pgfpathlineto{\pgfqpoint{2.271147in}{2.091852in}}%
\pgfpathlineto{\pgfqpoint{2.275561in}{1.805208in}}%
\pgfpathlineto{\pgfqpoint{2.279976in}{1.678152in}}%
\pgfpathlineto{\pgfqpoint{2.284390in}{1.852980in}}%
\pgfpathlineto{\pgfqpoint{2.288804in}{2.544070in}}%
\pgfpathlineto{\pgfqpoint{2.293219in}{2.041862in}}%
\pgfpathlineto{\pgfqpoint{2.297633in}{1.948582in}}%
\pgfpathlineto{\pgfqpoint{2.302048in}{2.463400in}}%
\pgfpathlineto{\pgfqpoint{2.306462in}{2.513517in}}%
\pgfpathlineto{\pgfqpoint{2.310877in}{2.637851in}}%
\pgfpathlineto{\pgfqpoint{2.319706in}{2.186299in}}%
\pgfpathlineto{\pgfqpoint{2.328534in}{2.635419in}}%
\pgfpathlineto{\pgfqpoint{2.332949in}{1.722989in}}%
\pgfpathlineto{\pgfqpoint{2.337363in}{1.694943in}}%
\pgfpathlineto{\pgfqpoint{2.341778in}{2.628469in}}%
\pgfpathlineto{\pgfqpoint{2.346192in}{2.259085in}}%
\pgfpathlineto{\pgfqpoint{2.350607in}{2.494977in}}%
\pgfpathlineto{\pgfqpoint{2.355021in}{2.070256in}}%
\pgfpathlineto{\pgfqpoint{2.359436in}{2.804519in}}%
\pgfpathlineto{\pgfqpoint{2.363850in}{2.680840in}}%
\pgfpathlineto{\pgfqpoint{2.368264in}{2.980198in}}%
\pgfpathlineto{\pgfqpoint{2.372679in}{2.474219in}}%
\pgfpathlineto{\pgfqpoint{2.377093in}{2.266820in}}%
\pgfpathlineto{\pgfqpoint{2.381508in}{2.588187in}}%
\pgfpathlineto{\pgfqpoint{2.385922in}{1.823430in}}%
\pgfpathlineto{\pgfqpoint{2.390337in}{1.901150in}}%
\pgfpathlineto{\pgfqpoint{2.394751in}{1.642425in}}%
\pgfpathlineto{\pgfqpoint{2.399166in}{2.483593in}}%
\pgfpathlineto{\pgfqpoint{2.403580in}{2.555539in}}%
\pgfpathlineto{\pgfqpoint{2.407994in}{2.064616in}}%
\pgfpathlineto{\pgfqpoint{2.412409in}{2.534171in}}%
\pgfpathlineto{\pgfqpoint{2.416823in}{2.610820in}}%
\pgfpathlineto{\pgfqpoint{2.421238in}{2.659045in}}%
\pgfpathlineto{\pgfqpoint{2.425652in}{2.265583in}}%
\pgfpathlineto{\pgfqpoint{2.430067in}{2.111637in}}%
\pgfpathlineto{\pgfqpoint{2.434481in}{2.510576in}}%
\pgfpathlineto{\pgfqpoint{2.438896in}{2.075862in}}%
\pgfpathlineto{\pgfqpoint{2.447724in}{1.626366in}}%
\pgfpathlineto{\pgfqpoint{2.452139in}{2.651217in}}%
\pgfpathlineto{\pgfqpoint{2.456553in}{1.656045in}}%
\pgfpathlineto{\pgfqpoint{2.460968in}{2.206922in}}%
\pgfpathlineto{\pgfqpoint{2.465382in}{2.082051in}}%
\pgfpathlineto{\pgfqpoint{2.469797in}{2.686578in}}%
\pgfpathlineto{\pgfqpoint{2.474211in}{2.594018in}}%
\pgfpathlineto{\pgfqpoint{2.478626in}{2.590844in}}%
\pgfpathlineto{\pgfqpoint{2.483040in}{2.287831in}}%
\pgfpathlineto{\pgfqpoint{2.487454in}{2.160935in}}%
\pgfpathlineto{\pgfqpoint{2.491869in}{2.366375in}}%
\pgfpathlineto{\pgfqpoint{2.500698in}{1.593169in}}%
\pgfpathlineto{\pgfqpoint{2.505112in}{1.793829in}}%
\pgfpathlineto{\pgfqpoint{2.509527in}{2.477146in}}%
\pgfpathlineto{\pgfqpoint{2.513941in}{2.254800in}}%
\pgfpathlineto{\pgfqpoint{2.518356in}{2.614875in}}%
\pgfpathlineto{\pgfqpoint{2.522770in}{2.667665in}}%
\pgfpathlineto{\pgfqpoint{2.527184in}{2.825167in}}%
\pgfpathlineto{\pgfqpoint{2.531599in}{2.650972in}}%
\pgfpathlineto{\pgfqpoint{2.540428in}{2.236576in}}%
\pgfpathlineto{\pgfqpoint{2.544842in}{2.410683in}}%
\pgfpathlineto{\pgfqpoint{2.549257in}{2.034141in}}%
\pgfpathlineto{\pgfqpoint{2.553671in}{1.916954in}}%
\pgfpathlineto{\pgfqpoint{2.558086in}{1.391975in}}%
\pgfpathlineto{\pgfqpoint{2.562500in}{2.712172in}}%
\pgfpathlineto{\pgfqpoint{2.566914in}{2.215590in}}%
\pgfpathlineto{\pgfqpoint{2.571329in}{2.540882in}}%
\pgfpathlineto{\pgfqpoint{2.575743in}{2.547481in}}%
\pgfpathlineto{\pgfqpoint{2.580158in}{2.637376in}}%
\pgfpathlineto{\pgfqpoint{2.584572in}{2.638360in}}%
\pgfpathlineto{\pgfqpoint{2.588987in}{2.438914in}}%
\pgfpathlineto{\pgfqpoint{2.593401in}{2.489090in}}%
\pgfpathlineto{\pgfqpoint{2.597816in}{1.824293in}}%
\pgfpathlineto{\pgfqpoint{2.602230in}{2.045852in}}%
\pgfpathlineto{\pgfqpoint{2.611059in}{1.539420in}}%
\pgfpathlineto{\pgfqpoint{2.615473in}{2.349008in}}%
\pgfpathlineto{\pgfqpoint{2.619888in}{2.508335in}}%
\pgfpathlineto{\pgfqpoint{2.624302in}{2.070911in}}%
\pgfpathlineto{\pgfqpoint{2.628717in}{2.362647in}}%
\pgfpathlineto{\pgfqpoint{2.633131in}{2.457850in}}%
\pgfpathlineto{\pgfqpoint{2.637546in}{2.889004in}}%
\pgfpathlineto{\pgfqpoint{2.641960in}{2.580916in}}%
\pgfpathlineto{\pgfqpoint{2.646374in}{2.404441in}}%
\pgfpathlineto{\pgfqpoint{2.650789in}{2.177614in}}%
\pgfpathlineto{\pgfqpoint{2.659618in}{2.499352in}}%
\pgfpathlineto{\pgfqpoint{2.668447in}{1.844722in}}%
\pgfpathlineto{\pgfqpoint{2.672861in}{2.587779in}}%
\pgfpathlineto{\pgfqpoint{2.677276in}{2.188660in}}%
\pgfpathlineto{\pgfqpoint{2.681690in}{2.537443in}}%
\pgfpathlineto{\pgfqpoint{2.686104in}{2.314347in}}%
\pgfpathlineto{\pgfqpoint{2.690519in}{2.741733in}}%
\pgfpathlineto{\pgfqpoint{2.694933in}{2.761850in}}%
\pgfpathlineto{\pgfqpoint{2.699348in}{2.760402in}}%
\pgfpathlineto{\pgfqpoint{2.703762in}{2.754408in}}%
\pgfpathlineto{\pgfqpoint{2.708177in}{1.960773in}}%
\pgfpathlineto{\pgfqpoint{2.712591in}{2.454780in}}%
\pgfpathlineto{\pgfqpoint{2.717006in}{2.014306in}}%
\pgfpathlineto{\pgfqpoint{2.721420in}{2.148089in}}%
\pgfpathlineto{\pgfqpoint{2.725834in}{2.156499in}}%
\pgfpathlineto{\pgfqpoint{2.730249in}{2.632644in}}%
\pgfpathlineto{\pgfqpoint{2.734663in}{2.390063in}}%
\pgfpathlineto{\pgfqpoint{2.739078in}{2.020961in}}%
\pgfpathlineto{\pgfqpoint{2.743492in}{2.593776in}}%
\pgfpathlineto{\pgfqpoint{2.747907in}{2.564159in}}%
\pgfpathlineto{\pgfqpoint{2.752321in}{2.681717in}}%
\pgfpathlineto{\pgfqpoint{2.756736in}{2.447234in}}%
\pgfpathlineto{\pgfqpoint{2.761150in}{2.004102in}}%
\pgfpathlineto{\pgfqpoint{2.769979in}{2.417518in}}%
\pgfpathlineto{\pgfqpoint{2.774393in}{1.863526in}}%
\pgfpathlineto{\pgfqpoint{2.778808in}{2.013884in}}%
\pgfpathlineto{\pgfqpoint{2.783222in}{2.528556in}}%
\pgfpathlineto{\pgfqpoint{2.787637in}{2.632090in}}%
\pgfpathlineto{\pgfqpoint{2.792051in}{2.117645in}}%
\pgfpathlineto{\pgfqpoint{2.796466in}{1.971778in}}%
\pgfpathlineto{\pgfqpoint{2.800880in}{2.814362in}}%
\pgfpathlineto{\pgfqpoint{2.805294in}{2.531160in}}%
\pgfpathlineto{\pgfqpoint{2.809709in}{2.489267in}}%
\pgfpathlineto{\pgfqpoint{2.814123in}{2.775056in}}%
\pgfpathlineto{\pgfqpoint{2.818538in}{2.255756in}}%
\pgfpathlineto{\pgfqpoint{2.822952in}{2.564885in}}%
\pgfpathlineto{\pgfqpoint{2.827367in}{1.947238in}}%
\pgfpathlineto{\pgfqpoint{2.831781in}{2.082278in}}%
\pgfpathlineto{\pgfqpoint{2.836196in}{1.677733in}}%
\pgfpathlineto{\pgfqpoint{2.840610in}{2.189040in}}%
\pgfpathlineto{\pgfqpoint{2.845024in}{2.476485in}}%
\pgfpathlineto{\pgfqpoint{2.849439in}{1.734266in}}%
\pgfpathlineto{\pgfqpoint{2.853853in}{2.692722in}}%
\pgfpathlineto{\pgfqpoint{2.858268in}{2.422047in}}%
\pgfpathlineto{\pgfqpoint{2.862682in}{2.895401in}}%
\pgfpathlineto{\pgfqpoint{2.867097in}{2.813682in}}%
\pgfpathlineto{\pgfqpoint{2.871511in}{2.060222in}}%
\pgfpathlineto{\pgfqpoint{2.875926in}{2.325672in}}%
\pgfpathlineto{\pgfqpoint{2.880340in}{2.381271in}}%
\pgfpathlineto{\pgfqpoint{2.884754in}{2.044898in}}%
\pgfpathlineto{\pgfqpoint{2.889169in}{2.067478in}}%
\pgfpathlineto{\pgfqpoint{2.893583in}{2.741458in}}%
\pgfpathlineto{\pgfqpoint{2.897998in}{2.596031in}}%
\pgfpathlineto{\pgfqpoint{2.902412in}{1.657487in}}%
\pgfpathlineto{\pgfqpoint{2.911241in}{2.836470in}}%
\pgfpathlineto{\pgfqpoint{2.915656in}{2.555041in}}%
\pgfpathlineto{\pgfqpoint{2.920070in}{2.829446in}}%
\pgfpathlineto{\pgfqpoint{2.924485in}{2.920840in}}%
\pgfpathlineto{\pgfqpoint{2.928899in}{1.952864in}}%
\pgfpathlineto{\pgfqpoint{2.933313in}{2.356866in}}%
\pgfpathlineto{\pgfqpoint{2.937728in}{1.821661in}}%
\pgfpathlineto{\pgfqpoint{2.942142in}{1.872388in}}%
\pgfpathlineto{\pgfqpoint{2.950971in}{2.293949in}}%
\pgfpathlineto{\pgfqpoint{2.955386in}{2.607297in}}%
\pgfpathlineto{\pgfqpoint{2.959800in}{2.217257in}}%
\pgfpathlineto{\pgfqpoint{2.964215in}{2.617380in}}%
\pgfpathlineto{\pgfqpoint{2.968629in}{2.796433in}}%
\pgfpathlineto{\pgfqpoint{2.973043in}{2.783693in}}%
\pgfpathlineto{\pgfqpoint{2.977458in}{2.562860in}}%
\pgfpathlineto{\pgfqpoint{2.981872in}{1.949161in}}%
\pgfpathlineto{\pgfqpoint{2.986287in}{2.157980in}}%
\pgfpathlineto{\pgfqpoint{2.990701in}{2.622173in}}%
\pgfpathlineto{\pgfqpoint{2.995116in}{2.222613in}}%
\pgfpathlineto{\pgfqpoint{2.999530in}{2.072303in}}%
\pgfpathlineto{\pgfqpoint{3.003945in}{2.336058in}}%
\pgfpathlineto{\pgfqpoint{3.008359in}{2.741174in}}%
\pgfpathlineto{\pgfqpoint{3.012773in}{1.903346in}}%
\pgfpathlineto{\pgfqpoint{3.017188in}{2.113608in}}%
\pgfpathlineto{\pgfqpoint{3.021602in}{2.734187in}}%
\pgfpathlineto{\pgfqpoint{3.026017in}{2.818147in}}%
\pgfpathlineto{\pgfqpoint{3.030431in}{2.818450in}}%
\pgfpathlineto{\pgfqpoint{3.034846in}{2.902699in}}%
\pgfpathlineto{\pgfqpoint{3.039260in}{2.275963in}}%
\pgfpathlineto{\pgfqpoint{3.043675in}{2.602717in}}%
\pgfpathlineto{\pgfqpoint{3.048089in}{1.860973in}}%
\pgfpathlineto{\pgfqpoint{3.052503in}{1.834274in}}%
\pgfpathlineto{\pgfqpoint{3.056918in}{2.031020in}}%
\pgfpathlineto{\pgfqpoint{3.061332in}{1.948872in}}%
\pgfpathlineto{\pgfqpoint{3.065747in}{2.629385in}}%
\pgfpathlineto{\pgfqpoint{3.070161in}{2.495264in}}%
\pgfpathlineto{\pgfqpoint{3.074576in}{2.552359in}}%
\pgfpathlineto{\pgfqpoint{3.078990in}{2.714205in}}%
\pgfpathlineto{\pgfqpoint{3.083405in}{2.607129in}}%
\pgfpathlineto{\pgfqpoint{3.087819in}{1.994498in}}%
\pgfpathlineto{\pgfqpoint{3.092233in}{2.101929in}}%
\pgfpathlineto{\pgfqpoint{3.096648in}{2.252633in}}%
\pgfpathlineto{\pgfqpoint{3.101062in}{2.474512in}}%
\pgfpathlineto{\pgfqpoint{3.105477in}{1.640379in}}%
\pgfpathlineto{\pgfqpoint{3.114306in}{2.379323in}}%
\pgfpathlineto{\pgfqpoint{3.118720in}{2.177006in}}%
\pgfpathlineto{\pgfqpoint{3.123135in}{2.159485in}}%
\pgfpathlineto{\pgfqpoint{3.127549in}{1.863495in}}%
\pgfpathlineto{\pgfqpoint{3.131963in}{2.617565in}}%
\pgfpathlineto{\pgfqpoint{3.136378in}{2.760998in}}%
\pgfpathlineto{\pgfqpoint{3.140792in}{2.281938in}}%
\pgfpathlineto{\pgfqpoint{3.145207in}{2.267992in}}%
\pgfpathlineto{\pgfqpoint{3.149621in}{2.734280in}}%
\pgfpathlineto{\pgfqpoint{3.154036in}{2.577984in}}%
\pgfpathlineto{\pgfqpoint{3.158450in}{2.093342in}}%
\pgfpathlineto{\pgfqpoint{3.162865in}{2.351176in}}%
\pgfpathlineto{\pgfqpoint{3.167279in}{1.560021in}}%
\pgfpathlineto{\pgfqpoint{3.171693in}{2.031591in}}%
\pgfpathlineto{\pgfqpoint{3.176108in}{2.143774in}}%
\pgfpathlineto{\pgfqpoint{3.184937in}{2.533057in}}%
\pgfpathlineto{\pgfqpoint{3.189351in}{2.592362in}}%
\pgfpathlineto{\pgfqpoint{3.193766in}{2.685128in}}%
\pgfpathlineto{\pgfqpoint{3.198180in}{2.388528in}}%
\pgfpathlineto{\pgfqpoint{3.202595in}{2.444605in}}%
\pgfpathlineto{\pgfqpoint{3.207009in}{2.167796in}}%
\pgfpathlineto{\pgfqpoint{3.211423in}{2.391460in}}%
\pgfpathlineto{\pgfqpoint{3.215838in}{1.807823in}}%
\pgfpathlineto{\pgfqpoint{3.220252in}{2.181612in}}%
\pgfpathlineto{\pgfqpoint{3.224667in}{1.837918in}}%
\pgfpathlineto{\pgfqpoint{3.229081in}{2.386073in}}%
\pgfpathlineto{\pgfqpoint{3.233496in}{2.329749in}}%
\pgfpathlineto{\pgfqpoint{3.237910in}{2.174372in}}%
\pgfpathlineto{\pgfqpoint{3.242325in}{2.731581in}}%
\pgfpathlineto{\pgfqpoint{3.246739in}{2.620562in}}%
\pgfpathlineto{\pgfqpoint{3.251153in}{2.079962in}}%
\pgfpathlineto{\pgfqpoint{3.255568in}{2.213849in}}%
\pgfpathlineto{\pgfqpoint{3.259982in}{2.312036in}}%
\pgfpathlineto{\pgfqpoint{3.264397in}{2.599172in}}%
\pgfpathlineto{\pgfqpoint{3.268811in}{2.541801in}}%
\pgfpathlineto{\pgfqpoint{3.273226in}{2.551141in}}%
\pgfpathlineto{\pgfqpoint{3.277640in}{2.076869in}}%
\pgfpathlineto{\pgfqpoint{3.282055in}{2.509957in}}%
\pgfpathlineto{\pgfqpoint{3.286469in}{1.832281in}}%
\pgfpathlineto{\pgfqpoint{3.290883in}{2.702835in}}%
\pgfpathlineto{\pgfqpoint{3.295298in}{2.638554in}}%
\pgfpathlineto{\pgfqpoint{3.299712in}{2.606358in}}%
\pgfpathlineto{\pgfqpoint{3.304127in}{2.674643in}}%
\pgfpathlineto{\pgfqpoint{3.308541in}{2.845908in}}%
\pgfpathlineto{\pgfqpoint{3.312956in}{2.424246in}}%
\pgfpathlineto{\pgfqpoint{3.317370in}{2.237574in}}%
\pgfpathlineto{\pgfqpoint{3.321785in}{2.221019in}}%
\pgfpathlineto{\pgfqpoint{3.326199in}{2.154297in}}%
\pgfpathlineto{\pgfqpoint{3.330613in}{1.791968in}}%
\pgfpathlineto{\pgfqpoint{3.335028in}{2.115559in}}%
\pgfpathlineto{\pgfqpoint{3.339442in}{2.782454in}}%
\pgfpathlineto{\pgfqpoint{3.348271in}{1.785510in}}%
\pgfpathlineto{\pgfqpoint{3.352686in}{2.918377in}}%
\pgfpathlineto{\pgfqpoint{3.361515in}{2.448305in}}%
\pgfpathlineto{\pgfqpoint{3.365929in}{2.844705in}}%
\pgfpathlineto{\pgfqpoint{3.370343in}{2.336508in}}%
\pgfpathlineto{\pgfqpoint{3.374758in}{2.485266in}}%
\pgfpathlineto{\pgfqpoint{3.379172in}{2.214173in}}%
\pgfpathlineto{\pgfqpoint{3.383587in}{2.245232in}}%
\pgfpathlineto{\pgfqpoint{3.388001in}{2.214631in}}%
\pgfpathlineto{\pgfqpoint{3.392416in}{2.323808in}}%
\pgfpathlineto{\pgfqpoint{3.396830in}{2.202915in}}%
\pgfpathlineto{\pgfqpoint{3.401245in}{2.701876in}}%
\pgfpathlineto{\pgfqpoint{3.405659in}{2.742495in}}%
\pgfpathlineto{\pgfqpoint{3.410073in}{2.462363in}}%
\pgfpathlineto{\pgfqpoint{3.414488in}{2.621833in}}%
\pgfpathlineto{\pgfqpoint{3.418902in}{2.901367in}}%
\pgfpathlineto{\pgfqpoint{3.423317in}{2.749013in}}%
\pgfpathlineto{\pgfqpoint{3.427731in}{2.302980in}}%
\pgfpathlineto{\pgfqpoint{3.432146in}{2.183836in}}%
\pgfpathlineto{\pgfqpoint{3.436560in}{2.269921in}}%
\pgfpathlineto{\pgfqpoint{3.440975in}{1.542159in}}%
\pgfpathlineto{\pgfqpoint{3.445389in}{2.802031in}}%
\pgfpathlineto{\pgfqpoint{3.449803in}{2.500811in}}%
\pgfpathlineto{\pgfqpoint{3.454218in}{2.655980in}}%
\pgfpathlineto{\pgfqpoint{3.458632in}{2.069362in}}%
\pgfpathlineto{\pgfqpoint{3.463047in}{2.660349in}}%
\pgfpathlineto{\pgfqpoint{3.467461in}{2.819370in}}%
\pgfpathlineto{\pgfqpoint{3.476290in}{2.580407in}}%
\pgfpathlineto{\pgfqpoint{3.480705in}{2.132077in}}%
\pgfpathlineto{\pgfqpoint{3.485119in}{2.352354in}}%
\pgfpathlineto{\pgfqpoint{3.489533in}{1.845737in}}%
\pgfpathlineto{\pgfqpoint{3.493948in}{2.204858in}}%
\pgfpathlineto{\pgfqpoint{3.498362in}{1.757422in}}%
\pgfpathlineto{\pgfqpoint{3.502777in}{2.625185in}}%
\pgfpathlineto{\pgfqpoint{3.507191in}{2.445609in}}%
\pgfpathlineto{\pgfqpoint{3.511606in}{2.405681in}}%
\pgfpathlineto{\pgfqpoint{3.516020in}{2.855451in}}%
\pgfpathlineto{\pgfqpoint{3.520435in}{2.560875in}}%
\pgfpathlineto{\pgfqpoint{3.524849in}{2.544399in}}%
\pgfpathlineto{\pgfqpoint{3.529263in}{2.714731in}}%
\pgfpathlineto{\pgfqpoint{3.533678in}{2.457693in}}%
\pgfpathlineto{\pgfqpoint{3.538092in}{2.409452in}}%
\pgfpathlineto{\pgfqpoint{3.542507in}{2.415938in}}%
\pgfpathlineto{\pgfqpoint{3.546921in}{2.579612in}}%
\pgfpathlineto{\pgfqpoint{3.551336in}{1.721293in}}%
\pgfpathlineto{\pgfqpoint{3.555750in}{2.698286in}}%
\pgfpathlineto{\pgfqpoint{3.560165in}{2.889918in}}%
\pgfpathlineto{\pgfqpoint{3.564579in}{2.146293in}}%
\pgfpathlineto{\pgfqpoint{3.568994in}{2.043251in}}%
\pgfpathlineto{\pgfqpoint{3.573408in}{2.617993in}}%
\pgfpathlineto{\pgfqpoint{3.577822in}{2.765210in}}%
\pgfpathlineto{\pgfqpoint{3.582237in}{2.710766in}}%
\pgfpathlineto{\pgfqpoint{3.586651in}{2.296535in}}%
\pgfpathlineto{\pgfqpoint{3.591066in}{2.215410in}}%
\pgfpathlineto{\pgfqpoint{3.595480in}{2.547270in}}%
\pgfpathlineto{\pgfqpoint{3.599895in}{2.024017in}}%
\pgfpathlineto{\pgfqpoint{3.604309in}{2.522964in}}%
\pgfpathlineto{\pgfqpoint{3.608724in}{2.129676in}}%
\pgfpathlineto{\pgfqpoint{3.613138in}{2.219102in}}%
\pgfpathlineto{\pgfqpoint{3.617552in}{2.362931in}}%
\pgfpathlineto{\pgfqpoint{3.621967in}{2.022746in}}%
\pgfpathlineto{\pgfqpoint{3.626381in}{2.743150in}}%
\pgfpathlineto{\pgfqpoint{3.630796in}{2.774502in}}%
\pgfpathlineto{\pgfqpoint{3.635210in}{2.646212in}}%
\pgfpathlineto{\pgfqpoint{3.639625in}{2.797473in}}%
\pgfpathlineto{\pgfqpoint{3.644039in}{2.047319in}}%
\pgfpathlineto{\pgfqpoint{3.648454in}{2.288103in}}%
\pgfpathlineto{\pgfqpoint{3.652868in}{2.332209in}}%
\pgfpathlineto{\pgfqpoint{3.657282in}{2.525938in}}%
\pgfpathlineto{\pgfqpoint{3.661697in}{2.383501in}}%
\pgfpathlineto{\pgfqpoint{3.666111in}{2.391311in}}%
\pgfpathlineto{\pgfqpoint{3.670526in}{2.611503in}}%
\pgfpathlineto{\pgfqpoint{3.674940in}{2.353158in}}%
\pgfpathlineto{\pgfqpoint{3.679355in}{1.783280in}}%
\pgfpathlineto{\pgfqpoint{3.683769in}{2.705796in}}%
\pgfpathlineto{\pgfqpoint{3.688184in}{2.928257in}}%
\pgfpathlineto{\pgfqpoint{3.692598in}{2.810103in}}%
\pgfpathlineto{\pgfqpoint{3.697012in}{2.479156in}}%
\pgfpathlineto{\pgfqpoint{3.701427in}{2.599889in}}%
\pgfpathlineto{\pgfqpoint{3.705841in}{2.572403in}}%
\pgfpathlineto{\pgfqpoint{3.710256in}{2.251603in}}%
\pgfpathlineto{\pgfqpoint{3.714670in}{2.259020in}}%
\pgfpathlineto{\pgfqpoint{3.719085in}{2.104985in}}%
\pgfpathlineto{\pgfqpoint{3.723499in}{2.320774in}}%
\pgfpathlineto{\pgfqpoint{3.727914in}{1.985970in}}%
\pgfpathlineto{\pgfqpoint{3.732328in}{2.122195in}}%
\pgfpathlineto{\pgfqpoint{3.736742in}{2.557361in}}%
\pgfpathlineto{\pgfqpoint{3.741157in}{2.523239in}}%
\pgfpathlineto{\pgfqpoint{3.745571in}{2.422823in}}%
\pgfpathlineto{\pgfqpoint{3.749986in}{2.399057in}}%
\pgfpathlineto{\pgfqpoint{3.754400in}{2.259476in}}%
\pgfpathlineto{\pgfqpoint{3.758815in}{2.300981in}}%
\pgfpathlineto{\pgfqpoint{3.763229in}{2.457004in}}%
\pgfpathlineto{\pgfqpoint{3.767644in}{2.277034in}}%
\pgfpathlineto{\pgfqpoint{3.772058in}{2.291244in}}%
\pgfpathlineto{\pgfqpoint{3.776472in}{2.208659in}}%
\pgfpathlineto{\pgfqpoint{3.780887in}{2.043327in}}%
\pgfpathlineto{\pgfqpoint{3.785301in}{2.512859in}}%
\pgfpathlineto{\pgfqpoint{3.789716in}{2.249301in}}%
\pgfpathlineto{\pgfqpoint{3.794130in}{2.665005in}}%
\pgfpathlineto{\pgfqpoint{3.798545in}{2.881126in}}%
\pgfpathlineto{\pgfqpoint{3.802959in}{2.457010in}}%
\pgfpathlineto{\pgfqpoint{3.807374in}{2.261748in}}%
\pgfpathlineto{\pgfqpoint{3.811788in}{2.178283in}}%
\pgfpathlineto{\pgfqpoint{3.816202in}{2.284358in}}%
\pgfpathlineto{\pgfqpoint{3.825031in}{1.892977in}}%
\pgfpathlineto{\pgfqpoint{3.829446in}{2.010344in}}%
\pgfpathlineto{\pgfqpoint{3.833860in}{2.023722in}}%
\pgfpathlineto{\pgfqpoint{3.838275in}{2.571902in}}%
\pgfpathlineto{\pgfqpoint{3.842689in}{2.164697in}}%
\pgfpathlineto{\pgfqpoint{3.847104in}{2.651327in}}%
\pgfpathlineto{\pgfqpoint{3.851518in}{2.219585in}}%
\pgfpathlineto{\pgfqpoint{3.860347in}{2.488134in}}%
\pgfpathlineto{\pgfqpoint{3.864761in}{2.450565in}}%
\pgfpathlineto{\pgfqpoint{3.869176in}{2.529565in}}%
\pgfpathlineto{\pgfqpoint{3.873590in}{2.503988in}}%
\pgfpathlineto{\pgfqpoint{3.878005in}{2.344254in}}%
\pgfpathlineto{\pgfqpoint{3.882419in}{1.953528in}}%
\pgfpathlineto{\pgfqpoint{3.886834in}{2.592323in}}%
\pgfpathlineto{\pgfqpoint{3.891248in}{1.691800in}}%
\pgfpathlineto{\pgfqpoint{3.895662in}{2.356369in}}%
\pgfpathlineto{\pgfqpoint{3.904491in}{2.667820in}}%
\pgfpathlineto{\pgfqpoint{3.908906in}{2.417150in}}%
\pgfpathlineto{\pgfqpoint{3.913320in}{2.090289in}}%
\pgfpathlineto{\pgfqpoint{3.917735in}{2.346565in}}%
\pgfpathlineto{\pgfqpoint{3.922149in}{2.049881in}}%
\pgfpathlineto{\pgfqpoint{3.926564in}{2.412482in}}%
\pgfpathlineto{\pgfqpoint{3.930978in}{2.166308in}}%
\pgfpathlineto{\pgfqpoint{3.935392in}{2.183799in}}%
\pgfpathlineto{\pgfqpoint{3.939807in}{1.985776in}}%
\pgfpathlineto{\pgfqpoint{3.944221in}{2.284564in}}%
\pgfpathlineto{\pgfqpoint{3.948636in}{2.301886in}}%
\pgfpathlineto{\pgfqpoint{3.953050in}{2.304667in}}%
\pgfpathlineto{\pgfqpoint{3.957465in}{2.875804in}}%
\pgfpathlineto{\pgfqpoint{3.961879in}{2.513981in}}%
\pgfpathlineto{\pgfqpoint{3.966294in}{2.665309in}}%
\pgfpathlineto{\pgfqpoint{3.970708in}{2.769003in}}%
\pgfpathlineto{\pgfqpoint{3.975122in}{2.737477in}}%
\pgfpathlineto{\pgfqpoint{3.979537in}{2.487139in}}%
\pgfpathlineto{\pgfqpoint{3.983951in}{2.470904in}}%
\pgfpathlineto{\pgfqpoint{3.988366in}{2.074429in}}%
\pgfpathlineto{\pgfqpoint{3.992780in}{2.135600in}}%
\pgfpathlineto{\pgfqpoint{3.997195in}{2.251103in}}%
\pgfpathlineto{\pgfqpoint{4.001609in}{2.308937in}}%
\pgfpathlineto{\pgfqpoint{4.010438in}{1.974479in}}%
\pgfpathlineto{\pgfqpoint{4.014852in}{2.574062in}}%
\pgfpathlineto{\pgfqpoint{4.019267in}{2.382688in}}%
\pgfpathlineto{\pgfqpoint{4.023681in}{2.563560in}}%
\pgfpathlineto{\pgfqpoint{4.028096in}{2.825176in}}%
\pgfpathlineto{\pgfqpoint{4.032510in}{2.479750in}}%
\pgfpathlineto{\pgfqpoint{4.036925in}{2.686100in}}%
\pgfpathlineto{\pgfqpoint{4.045754in}{2.282874in}}%
\pgfpathlineto{\pgfqpoint{4.050168in}{1.546846in}}%
\pgfpathlineto{\pgfqpoint{4.054582in}{2.516157in}}%
\pgfpathlineto{\pgfqpoint{4.058997in}{2.689432in}}%
\pgfpathlineto{\pgfqpoint{4.063411in}{2.485238in}}%
\pgfpathlineto{\pgfqpoint{4.067826in}{2.722868in}}%
\pgfpathlineto{\pgfqpoint{4.072240in}{2.729534in}}%
\pgfpathlineto{\pgfqpoint{4.076655in}{2.743656in}}%
\pgfpathlineto{\pgfqpoint{4.081069in}{2.604629in}}%
\pgfpathlineto{\pgfqpoint{4.085484in}{2.774263in}}%
\pgfpathlineto{\pgfqpoint{4.089898in}{2.291463in}}%
\pgfpathlineto{\pgfqpoint{4.094312in}{2.293569in}}%
\pgfpathlineto{\pgfqpoint{4.098727in}{2.106228in}}%
\pgfpathlineto{\pgfqpoint{4.103141in}{2.195833in}}%
\pgfpathlineto{\pgfqpoint{4.107556in}{2.262805in}}%
\pgfpathlineto{\pgfqpoint{4.111970in}{2.756882in}}%
\pgfpathlineto{\pgfqpoint{4.120799in}{1.290867in}}%
\pgfpathlineto{\pgfqpoint{4.125214in}{1.493853in}}%
\pgfpathlineto{\pgfqpoint{4.129628in}{2.836939in}}%
\pgfpathlineto{\pgfqpoint{4.134042in}{2.972593in}}%
\pgfpathlineto{\pgfqpoint{4.138457in}{2.687666in}}%
\pgfpathlineto{\pgfqpoint{4.142871in}{2.737128in}}%
\pgfpathlineto{\pgfqpoint{4.147286in}{2.769447in}}%
\pgfpathlineto{\pgfqpoint{4.151700in}{2.025791in}}%
\pgfpathlineto{\pgfqpoint{4.156115in}{2.171108in}}%
\pgfpathlineto{\pgfqpoint{4.160529in}{1.570525in}}%
\pgfpathlineto{\pgfqpoint{4.169358in}{2.522444in}}%
\pgfpathlineto{\pgfqpoint{4.173772in}{2.481639in}}%
\pgfpathlineto{\pgfqpoint{4.178187in}{2.695047in}}%
\pgfpathlineto{\pgfqpoint{4.182601in}{2.741787in}}%
\pgfpathlineto{\pgfqpoint{4.187016in}{2.600125in}}%
\pgfpathlineto{\pgfqpoint{4.191430in}{2.296665in}}%
\pgfpathlineto{\pgfqpoint{4.195845in}{2.728972in}}%
\pgfpathlineto{\pgfqpoint{4.200259in}{2.272364in}}%
\pgfpathlineto{\pgfqpoint{4.204674in}{2.398042in}}%
\pgfpathlineto{\pgfqpoint{4.209088in}{2.616334in}}%
\pgfpathlineto{\pgfqpoint{4.213503in}{1.662067in}}%
\pgfpathlineto{\pgfqpoint{4.217917in}{2.068490in}}%
\pgfpathlineto{\pgfqpoint{4.222331in}{2.359065in}}%
\pgfpathlineto{\pgfqpoint{4.226746in}{2.320091in}}%
\pgfpathlineto{\pgfqpoint{4.231160in}{0.901735in}}%
\pgfpathlineto{\pgfqpoint{4.235575in}{1.048871in}}%
\pgfpathlineto{\pgfqpoint{4.239989in}{2.689075in}}%
\pgfpathlineto{\pgfqpoint{4.244404in}{2.715102in}}%
\pgfpathlineto{\pgfqpoint{4.248818in}{2.419132in}}%
\pgfpathlineto{\pgfqpoint{4.253233in}{2.761459in}}%
\pgfpathlineto{\pgfqpoint{4.257647in}{2.645037in}}%
\pgfpathlineto{\pgfqpoint{4.262061in}{2.162830in}}%
\pgfpathlineto{\pgfqpoint{4.266476in}{2.376078in}}%
\pgfpathlineto{\pgfqpoint{4.270890in}{1.642018in}}%
\pgfpathlineto{\pgfqpoint{4.275305in}{2.376545in}}%
\pgfpathlineto{\pgfqpoint{4.279719in}{2.224131in}}%
\pgfpathlineto{\pgfqpoint{4.284134in}{2.617391in}}%
\pgfpathlineto{\pgfqpoint{4.288548in}{2.088247in}}%
\pgfpathlineto{\pgfqpoint{4.292963in}{2.914084in}}%
\pgfpathlineto{\pgfqpoint{4.297377in}{2.534151in}}%
\pgfpathlineto{\pgfqpoint{4.301791in}{2.429754in}}%
\pgfpathlineto{\pgfqpoint{4.306206in}{2.233075in}}%
\pgfpathlineto{\pgfqpoint{4.310620in}{2.631544in}}%
\pgfpathlineto{\pgfqpoint{4.315035in}{2.694482in}}%
\pgfpathlineto{\pgfqpoint{4.319449in}{2.445603in}}%
\pgfpathlineto{\pgfqpoint{4.323864in}{2.078680in}}%
\pgfpathlineto{\pgfqpoint{4.323864in}{2.078680in}}%
\pgfusepath{stroke}%
\end{pgfscope}%
\begin{pgfscope}%
\pgfpathrectangle{\pgfqpoint{0.625000in}{0.440000in}}{\pgfqpoint{3.875000in}{3.080000in}} %
\pgfusepath{clip}%
\pgfsetrectcap%
\pgfsetroundjoin%
\pgfsetlinewidth{1.505625pt}%
\definecolor{currentstroke}{rgb}{0.941176,1.000000,0.941176}%
\pgfsetstrokecolor{currentstroke}%
\pgfsetdash{}{0pt}%
\pgfpathmoveto{\pgfqpoint{0.801136in}{2.450006in}}%
\pgfpathlineto{\pgfqpoint{0.805551in}{2.532934in}}%
\pgfpathlineto{\pgfqpoint{0.809965in}{2.532180in}}%
\pgfpathlineto{\pgfqpoint{0.814380in}{2.735604in}}%
\pgfpathlineto{\pgfqpoint{0.823209in}{2.976262in}}%
\pgfpathlineto{\pgfqpoint{0.827623in}{2.919263in}}%
\pgfpathlineto{\pgfqpoint{0.832037in}{2.964771in}}%
\pgfpathlineto{\pgfqpoint{0.836452in}{2.403111in}}%
\pgfpathlineto{\pgfqpoint{0.840866in}{2.566515in}}%
\pgfpathlineto{\pgfqpoint{0.845281in}{2.597409in}}%
\pgfpathlineto{\pgfqpoint{0.849695in}{3.088829in}}%
\pgfpathlineto{\pgfqpoint{0.854110in}{2.561111in}}%
\pgfpathlineto{\pgfqpoint{0.858524in}{2.418525in}}%
\pgfpathlineto{\pgfqpoint{0.862939in}{2.356453in}}%
\pgfpathlineto{\pgfqpoint{0.867353in}{2.888925in}}%
\pgfpathlineto{\pgfqpoint{0.871767in}{2.900012in}}%
\pgfpathlineto{\pgfqpoint{0.876182in}{2.667462in}}%
\pgfpathlineto{\pgfqpoint{0.880596in}{3.070003in}}%
\pgfpathlineto{\pgfqpoint{0.885011in}{2.834617in}}%
\pgfpathlineto{\pgfqpoint{0.889425in}{2.855917in}}%
\pgfpathlineto{\pgfqpoint{0.893840in}{2.912070in}}%
\pgfpathlineto{\pgfqpoint{0.898254in}{2.470699in}}%
\pgfpathlineto{\pgfqpoint{0.902669in}{2.448887in}}%
\pgfpathlineto{\pgfqpoint{0.907083in}{2.409997in}}%
\pgfpathlineto{\pgfqpoint{0.911497in}{2.436100in}}%
\pgfpathlineto{\pgfqpoint{0.915912in}{2.348747in}}%
\pgfpathlineto{\pgfqpoint{0.924741in}{3.005356in}}%
\pgfpathlineto{\pgfqpoint{0.929155in}{3.009928in}}%
\pgfpathlineto{\pgfqpoint{0.933570in}{2.909636in}}%
\pgfpathlineto{\pgfqpoint{0.937984in}{2.775610in}}%
\pgfpathlineto{\pgfqpoint{0.942399in}{3.028346in}}%
\pgfpathlineto{\pgfqpoint{0.946813in}{2.602447in}}%
\pgfpathlineto{\pgfqpoint{0.951228in}{3.225560in}}%
\pgfpathlineto{\pgfqpoint{0.960056in}{2.643263in}}%
\pgfpathlineto{\pgfqpoint{0.968885in}{3.134405in}}%
\pgfpathlineto{\pgfqpoint{0.973300in}{2.569124in}}%
\pgfpathlineto{\pgfqpoint{0.982129in}{2.986727in}}%
\pgfpathlineto{\pgfqpoint{0.986543in}{3.007462in}}%
\pgfpathlineto{\pgfqpoint{0.990958in}{2.982526in}}%
\pgfpathlineto{\pgfqpoint{0.995372in}{2.777185in}}%
\pgfpathlineto{\pgfqpoint{0.999786in}{3.046737in}}%
\pgfpathlineto{\pgfqpoint{1.004201in}{3.039073in}}%
\pgfpathlineto{\pgfqpoint{1.008615in}{2.040327in}}%
\pgfpathlineto{\pgfqpoint{1.013030in}{2.573851in}}%
\pgfpathlineto{\pgfqpoint{1.017444in}{2.569782in}}%
\pgfpathlineto{\pgfqpoint{1.021859in}{3.209820in}}%
\pgfpathlineto{\pgfqpoint{1.026273in}{3.243787in}}%
\pgfpathlineto{\pgfqpoint{1.030688in}{2.173250in}}%
\pgfpathlineto{\pgfqpoint{1.035102in}{2.059727in}}%
\pgfpathlineto{\pgfqpoint{1.039516in}{3.110015in}}%
\pgfpathlineto{\pgfqpoint{1.043931in}{3.180945in}}%
\pgfpathlineto{\pgfqpoint{1.048345in}{2.922889in}}%
\pgfpathlineto{\pgfqpoint{1.052760in}{2.552123in}}%
\pgfpathlineto{\pgfqpoint{1.057174in}{3.122945in}}%
\pgfpathlineto{\pgfqpoint{1.066003in}{2.732348in}}%
\pgfpathlineto{\pgfqpoint{1.070418in}{3.206309in}}%
\pgfpathlineto{\pgfqpoint{1.074832in}{2.851860in}}%
\pgfpathlineto{\pgfqpoint{1.079246in}{2.726764in}}%
\pgfpathlineto{\pgfqpoint{1.083661in}{2.728013in}}%
\pgfpathlineto{\pgfqpoint{1.092490in}{2.982866in}}%
\pgfpathlineto{\pgfqpoint{1.096904in}{2.556787in}}%
\pgfpathlineto{\pgfqpoint{1.101319in}{2.960933in}}%
\pgfpathlineto{\pgfqpoint{1.105733in}{2.508850in}}%
\pgfpathlineto{\pgfqpoint{1.110148in}{3.159245in}}%
\pgfpathlineto{\pgfqpoint{1.114562in}{2.711998in}}%
\pgfpathlineto{\pgfqpoint{1.118976in}{2.610682in}}%
\pgfpathlineto{\pgfqpoint{1.123391in}{2.324530in}}%
\pgfpathlineto{\pgfqpoint{1.127805in}{3.140889in}}%
\pgfpathlineto{\pgfqpoint{1.132220in}{2.675186in}}%
\pgfpathlineto{\pgfqpoint{1.141049in}{2.730242in}}%
\pgfpathlineto{\pgfqpoint{1.145463in}{1.608473in}}%
\pgfpathlineto{\pgfqpoint{1.149878in}{3.028130in}}%
\pgfpathlineto{\pgfqpoint{1.154292in}{3.249854in}}%
\pgfpathlineto{\pgfqpoint{1.158706in}{3.011533in}}%
\pgfpathlineto{\pgfqpoint{1.163121in}{3.052709in}}%
\pgfpathlineto{\pgfqpoint{1.167535in}{2.615246in}}%
\pgfpathlineto{\pgfqpoint{1.171950in}{2.343579in}}%
\pgfpathlineto{\pgfqpoint{1.180779in}{2.945961in}}%
\pgfpathlineto{\pgfqpoint{1.185193in}{2.997042in}}%
\pgfpathlineto{\pgfqpoint{1.189608in}{2.721501in}}%
\pgfpathlineto{\pgfqpoint{1.194022in}{2.219959in}}%
\pgfpathlineto{\pgfqpoint{1.198436in}{2.865932in}}%
\pgfpathlineto{\pgfqpoint{1.202851in}{3.128594in}}%
\pgfpathlineto{\pgfqpoint{1.207265in}{2.513910in}}%
\pgfpathlineto{\pgfqpoint{1.211680in}{2.915264in}}%
\pgfpathlineto{\pgfqpoint{1.216094in}{2.522531in}}%
\pgfpathlineto{\pgfqpoint{1.220509in}{3.157111in}}%
\pgfpathlineto{\pgfqpoint{1.224923in}{3.199406in}}%
\pgfpathlineto{\pgfqpoint{1.229338in}{3.150501in}}%
\pgfpathlineto{\pgfqpoint{1.233752in}{2.802911in}}%
\pgfpathlineto{\pgfqpoint{1.238166in}{3.040602in}}%
\pgfpathlineto{\pgfqpoint{1.242581in}{3.167306in}}%
\pgfpathlineto{\pgfqpoint{1.246995in}{2.640148in}}%
\pgfpathlineto{\pgfqpoint{1.251410in}{2.723703in}}%
\pgfpathlineto{\pgfqpoint{1.255824in}{2.587686in}}%
\pgfpathlineto{\pgfqpoint{1.260239in}{3.039708in}}%
\pgfpathlineto{\pgfqpoint{1.264653in}{2.643707in}}%
\pgfpathlineto{\pgfqpoint{1.269068in}{2.974001in}}%
\pgfpathlineto{\pgfqpoint{1.273482in}{2.746412in}}%
\pgfpathlineto{\pgfqpoint{1.277896in}{2.651549in}}%
\pgfpathlineto{\pgfqpoint{1.282311in}{2.706080in}}%
\pgfpathlineto{\pgfqpoint{1.286725in}{3.247394in}}%
\pgfpathlineto{\pgfqpoint{1.291140in}{2.399361in}}%
\pgfpathlineto{\pgfqpoint{1.295554in}{3.108685in}}%
\pgfpathlineto{\pgfqpoint{1.299969in}{2.205749in}}%
\pgfpathlineto{\pgfqpoint{1.304383in}{2.134034in}}%
\pgfpathlineto{\pgfqpoint{1.308798in}{2.541163in}}%
\pgfpathlineto{\pgfqpoint{1.313212in}{3.068743in}}%
\pgfpathlineto{\pgfqpoint{1.317626in}{3.006323in}}%
\pgfpathlineto{\pgfqpoint{1.322041in}{2.979380in}}%
\pgfpathlineto{\pgfqpoint{1.326455in}{2.577922in}}%
\pgfpathlineto{\pgfqpoint{1.330870in}{3.118626in}}%
\pgfpathlineto{\pgfqpoint{1.335284in}{3.178724in}}%
\pgfpathlineto{\pgfqpoint{1.339699in}{2.944668in}}%
\pgfpathlineto{\pgfqpoint{1.344113in}{3.102241in}}%
\pgfpathlineto{\pgfqpoint{1.348528in}{3.022257in}}%
\pgfpathlineto{\pgfqpoint{1.352942in}{2.206832in}}%
\pgfpathlineto{\pgfqpoint{1.357356in}{3.255233in}}%
\pgfpathlineto{\pgfqpoint{1.361771in}{2.201914in}}%
\pgfpathlineto{\pgfqpoint{1.366185in}{2.135519in}}%
\pgfpathlineto{\pgfqpoint{1.370600in}{2.859505in}}%
\pgfpathlineto{\pgfqpoint{1.375014in}{3.039258in}}%
\pgfpathlineto{\pgfqpoint{1.379429in}{3.112286in}}%
\pgfpathlineto{\pgfqpoint{1.383843in}{2.350431in}}%
\pgfpathlineto{\pgfqpoint{1.388258in}{2.617391in}}%
\pgfpathlineto{\pgfqpoint{1.392672in}{3.189886in}}%
\pgfpathlineto{\pgfqpoint{1.397086in}{2.956356in}}%
\pgfpathlineto{\pgfqpoint{1.405915in}{3.093156in}}%
\pgfpathlineto{\pgfqpoint{1.410330in}{2.623284in}}%
\pgfpathlineto{\pgfqpoint{1.414744in}{2.684422in}}%
\pgfpathlineto{\pgfqpoint{1.419159in}{2.082453in}}%
\pgfpathlineto{\pgfqpoint{1.423573in}{3.032808in}}%
\pgfpathlineto{\pgfqpoint{1.427988in}{3.144558in}}%
\pgfpathlineto{\pgfqpoint{1.432402in}{3.146574in}}%
\pgfpathlineto{\pgfqpoint{1.436816in}{2.079326in}}%
\pgfpathlineto{\pgfqpoint{1.441231in}{2.898341in}}%
\pgfpathlineto{\pgfqpoint{1.445645in}{2.706844in}}%
\pgfpathlineto{\pgfqpoint{1.450060in}{2.902863in}}%
\pgfpathlineto{\pgfqpoint{1.454474in}{2.683396in}}%
\pgfpathlineto{\pgfqpoint{1.458889in}{2.984896in}}%
\pgfpathlineto{\pgfqpoint{1.463303in}{2.687335in}}%
\pgfpathlineto{\pgfqpoint{1.467718in}{2.733926in}}%
\pgfpathlineto{\pgfqpoint{1.472132in}{2.725513in}}%
\pgfpathlineto{\pgfqpoint{1.476546in}{1.602726in}}%
\pgfpathlineto{\pgfqpoint{1.480961in}{2.968198in}}%
\pgfpathlineto{\pgfqpoint{1.485375in}{3.081840in}}%
\pgfpathlineto{\pgfqpoint{1.489790in}{3.226189in}}%
\pgfpathlineto{\pgfqpoint{1.494204in}{2.685653in}}%
\pgfpathlineto{\pgfqpoint{1.498619in}{2.626849in}}%
\pgfpathlineto{\pgfqpoint{1.503033in}{3.137976in}}%
\pgfpathlineto{\pgfqpoint{1.507448in}{2.715361in}}%
\pgfpathlineto{\pgfqpoint{1.511862in}{3.026707in}}%
\pgfpathlineto{\pgfqpoint{1.516276in}{3.205266in}}%
\pgfpathlineto{\pgfqpoint{1.520691in}{3.284634in}}%
\pgfpathlineto{\pgfqpoint{1.525105in}{2.215030in}}%
\pgfpathlineto{\pgfqpoint{1.529520in}{2.192881in}}%
\pgfpathlineto{\pgfqpoint{1.533934in}{2.890399in}}%
\pgfpathlineto{\pgfqpoint{1.538349in}{3.213264in}}%
\pgfpathlineto{\pgfqpoint{1.542763in}{3.099260in}}%
\pgfpathlineto{\pgfqpoint{1.547178in}{2.487732in}}%
\pgfpathlineto{\pgfqpoint{1.551592in}{2.892552in}}%
\pgfpathlineto{\pgfqpoint{1.556006in}{2.731344in}}%
\pgfpathlineto{\pgfqpoint{1.560421in}{2.796545in}}%
\pgfpathlineto{\pgfqpoint{1.564835in}{3.022197in}}%
\pgfpathlineto{\pgfqpoint{1.569250in}{3.073068in}}%
\pgfpathlineto{\pgfqpoint{1.573664in}{3.284814in}}%
\pgfpathlineto{\pgfqpoint{1.578079in}{2.741250in}}%
\pgfpathlineto{\pgfqpoint{1.582493in}{2.750019in}}%
\pgfpathlineto{\pgfqpoint{1.586908in}{2.159263in}}%
\pgfpathlineto{\pgfqpoint{1.591322in}{2.571565in}}%
\pgfpathlineto{\pgfqpoint{1.595737in}{3.171799in}}%
\pgfpathlineto{\pgfqpoint{1.600151in}{3.142047in}}%
\pgfpathlineto{\pgfqpoint{1.604565in}{3.156757in}}%
\pgfpathlineto{\pgfqpoint{1.608980in}{2.715600in}}%
\pgfpathlineto{\pgfqpoint{1.613394in}{3.265071in}}%
\pgfpathlineto{\pgfqpoint{1.617809in}{2.717348in}}%
\pgfpathlineto{\pgfqpoint{1.622223in}{2.004364in}}%
\pgfpathlineto{\pgfqpoint{1.626638in}{3.200199in}}%
\pgfpathlineto{\pgfqpoint{1.631052in}{3.239736in}}%
\pgfpathlineto{\pgfqpoint{1.635467in}{3.294764in}}%
\pgfpathlineto{\pgfqpoint{1.639881in}{1.642046in}}%
\pgfpathlineto{\pgfqpoint{1.644295in}{2.944094in}}%
\pgfpathlineto{\pgfqpoint{1.648710in}{3.190136in}}%
\pgfpathlineto{\pgfqpoint{1.653124in}{3.176922in}}%
\pgfpathlineto{\pgfqpoint{1.657539in}{3.190294in}}%
\pgfpathlineto{\pgfqpoint{1.661953in}{3.031512in}}%
\pgfpathlineto{\pgfqpoint{1.666368in}{2.752909in}}%
\pgfpathlineto{\pgfqpoint{1.670782in}{2.783693in}}%
\pgfpathlineto{\pgfqpoint{1.675197in}{2.646536in}}%
\pgfpathlineto{\pgfqpoint{1.679611in}{3.180982in}}%
\pgfpathlineto{\pgfqpoint{1.684025in}{2.562197in}}%
\pgfpathlineto{\pgfqpoint{1.688440in}{2.105558in}}%
\pgfpathlineto{\pgfqpoint{1.692854in}{2.771868in}}%
\pgfpathlineto{\pgfqpoint{1.697269in}{2.200967in}}%
\pgfpathlineto{\pgfqpoint{1.706098in}{2.948081in}}%
\pgfpathlineto{\pgfqpoint{1.710512in}{3.167214in}}%
\pgfpathlineto{\pgfqpoint{1.714927in}{2.797628in}}%
\pgfpathlineto{\pgfqpoint{1.719341in}{2.717593in}}%
\pgfpathlineto{\pgfqpoint{1.723755in}{2.391455in}}%
\pgfpathlineto{\pgfqpoint{1.728170in}{2.403893in}}%
\pgfpathlineto{\pgfqpoint{1.732584in}{2.053319in}}%
\pgfpathlineto{\pgfqpoint{1.736999in}{2.722426in}}%
\pgfpathlineto{\pgfqpoint{1.741413in}{3.140422in}}%
\pgfpathlineto{\pgfqpoint{1.745828in}{2.216965in}}%
\pgfpathlineto{\pgfqpoint{1.750242in}{2.779080in}}%
\pgfpathlineto{\pgfqpoint{1.754657in}{3.058087in}}%
\pgfpathlineto{\pgfqpoint{1.759071in}{3.173801in}}%
\pgfpathlineto{\pgfqpoint{1.763485in}{2.654619in}}%
\pgfpathlineto{\pgfqpoint{1.767900in}{3.046858in}}%
\pgfpathlineto{\pgfqpoint{1.772314in}{2.463561in}}%
\pgfpathlineto{\pgfqpoint{1.776729in}{3.255345in}}%
\pgfpathlineto{\pgfqpoint{1.781143in}{2.733841in}}%
\pgfpathlineto{\pgfqpoint{1.785558in}{2.624729in}}%
\pgfpathlineto{\pgfqpoint{1.789972in}{3.160927in}}%
\pgfpathlineto{\pgfqpoint{1.794387in}{3.149000in}}%
\pgfpathlineto{\pgfqpoint{1.798801in}{2.652676in}}%
\pgfpathlineto{\pgfqpoint{1.803215in}{2.751920in}}%
\pgfpathlineto{\pgfqpoint{1.807630in}{2.086248in}}%
\pgfpathlineto{\pgfqpoint{1.812044in}{3.048874in}}%
\pgfpathlineto{\pgfqpoint{1.816459in}{3.080040in}}%
\pgfpathlineto{\pgfqpoint{1.820873in}{3.179958in}}%
\pgfpathlineto{\pgfqpoint{1.825288in}{2.857379in}}%
\pgfpathlineto{\pgfqpoint{1.829702in}{2.739346in}}%
\pgfpathlineto{\pgfqpoint{1.834117in}{1.893416in}}%
\pgfpathlineto{\pgfqpoint{1.838531in}{2.428033in}}%
\pgfpathlineto{\pgfqpoint{1.842945in}{3.101456in}}%
\pgfpathlineto{\pgfqpoint{1.847360in}{3.214704in}}%
\pgfpathlineto{\pgfqpoint{1.851774in}{2.662283in}}%
\pgfpathlineto{\pgfqpoint{1.856189in}{1.667302in}}%
\pgfpathlineto{\pgfqpoint{1.860603in}{2.655016in}}%
\pgfpathlineto{\pgfqpoint{1.865018in}{3.011061in}}%
\pgfpathlineto{\pgfqpoint{1.869432in}{3.074740in}}%
\pgfpathlineto{\pgfqpoint{1.873847in}{2.604033in}}%
\pgfpathlineto{\pgfqpoint{1.878261in}{3.120609in}}%
\pgfpathlineto{\pgfqpoint{1.882675in}{2.617844in}}%
\pgfpathlineto{\pgfqpoint{1.887090in}{3.251266in}}%
\pgfpathlineto{\pgfqpoint{1.891504in}{3.200879in}}%
\pgfpathlineto{\pgfqpoint{1.895919in}{3.094320in}}%
\pgfpathlineto{\pgfqpoint{1.900333in}{3.251758in}}%
\pgfpathlineto{\pgfqpoint{1.904748in}{3.049326in}}%
\pgfpathlineto{\pgfqpoint{1.909162in}{2.765460in}}%
\pgfpathlineto{\pgfqpoint{1.913577in}{2.216956in}}%
\pgfpathlineto{\pgfqpoint{1.917991in}{2.267939in}}%
\pgfpathlineto{\pgfqpoint{1.922405in}{3.020533in}}%
\pgfpathlineto{\pgfqpoint{1.926820in}{3.068530in}}%
\pgfpathlineto{\pgfqpoint{1.931234in}{2.676007in}}%
\pgfpathlineto{\pgfqpoint{1.935649in}{3.195208in}}%
\pgfpathlineto{\pgfqpoint{1.940063in}{3.277121in}}%
\pgfpathlineto{\pgfqpoint{1.944478in}{2.669411in}}%
\pgfpathlineto{\pgfqpoint{1.948892in}{2.738430in}}%
\pgfpathlineto{\pgfqpoint{1.953307in}{2.838210in}}%
\pgfpathlineto{\pgfqpoint{1.957721in}{3.253091in}}%
\pgfpathlineto{\pgfqpoint{1.962135in}{2.707443in}}%
\pgfpathlineto{\pgfqpoint{1.966550in}{2.728004in}}%
\pgfpathlineto{\pgfqpoint{1.970964in}{1.630707in}}%
\pgfpathlineto{\pgfqpoint{1.975379in}{2.949071in}}%
\pgfpathlineto{\pgfqpoint{1.984208in}{3.165169in}}%
\pgfpathlineto{\pgfqpoint{1.988622in}{3.184024in}}%
\pgfpathlineto{\pgfqpoint{1.993037in}{3.115950in}}%
\pgfpathlineto{\pgfqpoint{1.997451in}{2.625418in}}%
\pgfpathlineto{\pgfqpoint{2.001865in}{2.713707in}}%
\pgfpathlineto{\pgfqpoint{2.006280in}{2.662365in}}%
\pgfpathlineto{\pgfqpoint{2.010694in}{3.249630in}}%
\pgfpathlineto{\pgfqpoint{2.015109in}{2.995873in}}%
\pgfpathlineto{\pgfqpoint{2.019523in}{2.167236in}}%
\pgfpathlineto{\pgfqpoint{2.023938in}{2.753224in}}%
\pgfpathlineto{\pgfqpoint{2.028352in}{2.800299in}}%
\pgfpathlineto{\pgfqpoint{2.032767in}{2.973506in}}%
\pgfpathlineto{\pgfqpoint{2.037181in}{3.012438in}}%
\pgfpathlineto{\pgfqpoint{2.041595in}{3.149149in}}%
\pgfpathlineto{\pgfqpoint{2.046010in}{3.240835in}}%
\pgfpathlineto{\pgfqpoint{2.050424in}{2.738500in}}%
\pgfpathlineto{\pgfqpoint{2.054839in}{2.693827in}}%
\pgfpathlineto{\pgfqpoint{2.059253in}{2.191818in}}%
\pgfpathlineto{\pgfqpoint{2.063668in}{2.566518in}}%
\pgfpathlineto{\pgfqpoint{2.068082in}{3.219442in}}%
\pgfpathlineto{\pgfqpoint{2.072497in}{3.247535in}}%
\pgfpathlineto{\pgfqpoint{2.076911in}{2.208260in}}%
\pgfpathlineto{\pgfqpoint{2.081325in}{2.112188in}}%
\pgfpathlineto{\pgfqpoint{2.085740in}{2.994200in}}%
\pgfpathlineto{\pgfqpoint{2.094569in}{3.194230in}}%
\pgfpathlineto{\pgfqpoint{2.098983in}{3.148334in}}%
\pgfpathlineto{\pgfqpoint{2.103398in}{3.011828in}}%
\pgfpathlineto{\pgfqpoint{2.107812in}{2.649901in}}%
\pgfpathlineto{\pgfqpoint{2.112227in}{2.719882in}}%
\pgfpathlineto{\pgfqpoint{2.116641in}{2.682209in}}%
\pgfpathlineto{\pgfqpoint{2.121055in}{3.232667in}}%
\pgfpathlineto{\pgfqpoint{2.125470in}{2.705402in}}%
\pgfpathlineto{\pgfqpoint{2.129884in}{3.269741in}}%
\pgfpathlineto{\pgfqpoint{2.134299in}{2.728547in}}%
\pgfpathlineto{\pgfqpoint{2.138713in}{3.133216in}}%
\pgfpathlineto{\pgfqpoint{2.143128in}{3.091908in}}%
\pgfpathlineto{\pgfqpoint{2.147542in}{3.004563in}}%
\pgfpathlineto{\pgfqpoint{2.151957in}{3.140149in}}%
\pgfpathlineto{\pgfqpoint{2.156371in}{3.212399in}}%
\pgfpathlineto{\pgfqpoint{2.160785in}{2.627687in}}%
\pgfpathlineto{\pgfqpoint{2.165200in}{2.209610in}}%
\pgfpathlineto{\pgfqpoint{2.169614in}{1.121208in}}%
\pgfpathlineto{\pgfqpoint{2.174029in}{3.035257in}}%
\pgfpathlineto{\pgfqpoint{2.178443in}{2.685049in}}%
\pgfpathlineto{\pgfqpoint{2.182858in}{2.581023in}}%
\pgfpathlineto{\pgfqpoint{2.187272in}{1.625407in}}%
\pgfpathlineto{\pgfqpoint{2.191687in}{3.150471in}}%
\pgfpathlineto{\pgfqpoint{2.196101in}{3.025352in}}%
\pgfpathlineto{\pgfqpoint{2.200515in}{3.175893in}}%
\pgfpathlineto{\pgfqpoint{2.204930in}{2.626649in}}%
\pgfpathlineto{\pgfqpoint{2.209344in}{3.208403in}}%
\pgfpathlineto{\pgfqpoint{2.213759in}{3.110223in}}%
\pgfpathlineto{\pgfqpoint{2.218173in}{2.750927in}}%
\pgfpathlineto{\pgfqpoint{2.222588in}{2.626692in}}%
\pgfpathlineto{\pgfqpoint{2.227002in}{2.974386in}}%
\pgfpathlineto{\pgfqpoint{2.231417in}{3.206885in}}%
\pgfpathlineto{\pgfqpoint{2.235831in}{2.726458in}}%
\pgfpathlineto{\pgfqpoint{2.240246in}{2.688690in}}%
\pgfpathlineto{\pgfqpoint{2.244660in}{2.713404in}}%
\pgfpathlineto{\pgfqpoint{2.249074in}{2.362650in}}%
\pgfpathlineto{\pgfqpoint{2.253489in}{3.221640in}}%
\pgfpathlineto{\pgfqpoint{2.257903in}{3.127250in}}%
\pgfpathlineto{\pgfqpoint{2.262318in}{2.653649in}}%
\pgfpathlineto{\pgfqpoint{2.266732in}{3.104973in}}%
\pgfpathlineto{\pgfqpoint{2.271147in}{2.762660in}}%
\pgfpathlineto{\pgfqpoint{2.275561in}{2.198060in}}%
\pgfpathlineto{\pgfqpoint{2.279976in}{2.115154in}}%
\pgfpathlineto{\pgfqpoint{2.284390in}{2.911421in}}%
\pgfpathlineto{\pgfqpoint{2.288804in}{2.703057in}}%
\pgfpathlineto{\pgfqpoint{2.293219in}{3.141884in}}%
\pgfpathlineto{\pgfqpoint{2.297633in}{2.704820in}}%
\pgfpathlineto{\pgfqpoint{2.302048in}{3.016676in}}%
\pgfpathlineto{\pgfqpoint{2.306462in}{3.084691in}}%
\pgfpathlineto{\pgfqpoint{2.310877in}{3.068069in}}%
\pgfpathlineto{\pgfqpoint{2.315291in}{2.962856in}}%
\pgfpathlineto{\pgfqpoint{2.319706in}{3.012720in}}%
\pgfpathlineto{\pgfqpoint{2.324120in}{3.230969in}}%
\pgfpathlineto{\pgfqpoint{2.328534in}{3.143312in}}%
\pgfpathlineto{\pgfqpoint{2.332949in}{2.155295in}}%
\pgfpathlineto{\pgfqpoint{2.337363in}{3.006039in}}%
\pgfpathlineto{\pgfqpoint{2.341778in}{2.693697in}}%
\pgfpathlineto{\pgfqpoint{2.346192in}{3.255424in}}%
\pgfpathlineto{\pgfqpoint{2.350607in}{3.256535in}}%
\pgfpathlineto{\pgfqpoint{2.359436in}{2.343388in}}%
\pgfpathlineto{\pgfqpoint{2.363850in}{3.212280in}}%
\pgfpathlineto{\pgfqpoint{2.368264in}{3.172257in}}%
\pgfpathlineto{\pgfqpoint{2.372679in}{2.557586in}}%
\pgfpathlineto{\pgfqpoint{2.377093in}{3.094840in}}%
\pgfpathlineto{\pgfqpoint{2.381508in}{2.739653in}}%
\pgfpathlineto{\pgfqpoint{2.385922in}{2.666009in}}%
\pgfpathlineto{\pgfqpoint{2.390337in}{3.198450in}}%
\pgfpathlineto{\pgfqpoint{2.394751in}{2.698508in}}%
\pgfpathlineto{\pgfqpoint{2.399166in}{3.251997in}}%
\pgfpathlineto{\pgfqpoint{2.403580in}{2.383371in}}%
\pgfpathlineto{\pgfqpoint{2.407994in}{2.184747in}}%
\pgfpathlineto{\pgfqpoint{2.412409in}{3.080636in}}%
\pgfpathlineto{\pgfqpoint{2.416823in}{3.127688in}}%
\pgfpathlineto{\pgfqpoint{2.421238in}{2.909456in}}%
\pgfpathlineto{\pgfqpoint{2.425652in}{3.027315in}}%
\pgfpathlineto{\pgfqpoint{2.430067in}{2.850086in}}%
\pgfpathlineto{\pgfqpoint{2.434481in}{3.189442in}}%
\pgfpathlineto{\pgfqpoint{2.438896in}{3.130396in}}%
\pgfpathlineto{\pgfqpoint{2.443310in}{2.246141in}}%
\pgfpathlineto{\pgfqpoint{2.447724in}{2.567370in}}%
\pgfpathlineto{\pgfqpoint{2.452139in}{2.697600in}}%
\pgfpathlineto{\pgfqpoint{2.456553in}{2.732756in}}%
\pgfpathlineto{\pgfqpoint{2.460968in}{2.206033in}}%
\pgfpathlineto{\pgfqpoint{2.465382in}{2.709670in}}%
\pgfpathlineto{\pgfqpoint{2.469797in}{3.051877in}}%
\pgfpathlineto{\pgfqpoint{2.474211in}{3.179109in}}%
\pgfpathlineto{\pgfqpoint{2.478626in}{3.097123in}}%
\pgfpathlineto{\pgfqpoint{2.483040in}{3.065265in}}%
\pgfpathlineto{\pgfqpoint{2.487454in}{3.207790in}}%
\pgfpathlineto{\pgfqpoint{2.491869in}{3.109705in}}%
\pgfpathlineto{\pgfqpoint{2.496283in}{2.197452in}}%
\pgfpathlineto{\pgfqpoint{2.500698in}{2.205100in}}%
\pgfpathlineto{\pgfqpoint{2.505112in}{2.523506in}}%
\pgfpathlineto{\pgfqpoint{2.509527in}{2.721504in}}%
\pgfpathlineto{\pgfqpoint{2.513941in}{2.249663in}}%
\pgfpathlineto{\pgfqpoint{2.518356in}{2.695888in}}%
\pgfpathlineto{\pgfqpoint{2.522770in}{2.618993in}}%
\pgfpathlineto{\pgfqpoint{2.527184in}{3.152276in}}%
\pgfpathlineto{\pgfqpoint{2.531599in}{2.881520in}}%
\pgfpathlineto{\pgfqpoint{2.536013in}{3.209877in}}%
\pgfpathlineto{\pgfqpoint{2.540428in}{3.193378in}}%
\pgfpathlineto{\pgfqpoint{2.544842in}{3.213801in}}%
\pgfpathlineto{\pgfqpoint{2.549257in}{2.722735in}}%
\pgfpathlineto{\pgfqpoint{2.553671in}{2.733847in}}%
\pgfpathlineto{\pgfqpoint{2.558086in}{2.775627in}}%
\pgfpathlineto{\pgfqpoint{2.562500in}{3.249239in}}%
\pgfpathlineto{\pgfqpoint{2.566914in}{3.218890in}}%
\pgfpathlineto{\pgfqpoint{2.571329in}{2.730751in}}%
\pgfpathlineto{\pgfqpoint{2.575743in}{3.270669in}}%
\pgfpathlineto{\pgfqpoint{2.580158in}{3.069980in}}%
\pgfpathlineto{\pgfqpoint{2.584572in}{3.161947in}}%
\pgfpathlineto{\pgfqpoint{2.588987in}{3.155008in}}%
\pgfpathlineto{\pgfqpoint{2.593401in}{3.183358in}}%
\pgfpathlineto{\pgfqpoint{2.597816in}{3.058636in}}%
\pgfpathlineto{\pgfqpoint{2.602230in}{3.138206in}}%
\pgfpathlineto{\pgfqpoint{2.606644in}{2.728682in}}%
\pgfpathlineto{\pgfqpoint{2.611059in}{2.150859in}}%
\pgfpathlineto{\pgfqpoint{2.615473in}{3.245058in}}%
\pgfpathlineto{\pgfqpoint{2.619888in}{3.261143in}}%
\pgfpathlineto{\pgfqpoint{2.628717in}{2.200528in}}%
\pgfpathlineto{\pgfqpoint{2.633131in}{2.919277in}}%
\pgfpathlineto{\pgfqpoint{2.637546in}{2.943864in}}%
\pgfpathlineto{\pgfqpoint{2.641960in}{3.114508in}}%
\pgfpathlineto{\pgfqpoint{2.646374in}{3.209629in}}%
\pgfpathlineto{\pgfqpoint{2.650789in}{3.068035in}}%
\pgfpathlineto{\pgfqpoint{2.655203in}{3.072944in}}%
\pgfpathlineto{\pgfqpoint{2.659618in}{3.118646in}}%
\pgfpathlineto{\pgfqpoint{2.664032in}{2.720424in}}%
\pgfpathlineto{\pgfqpoint{2.668447in}{2.655016in}}%
\pgfpathlineto{\pgfqpoint{2.672861in}{3.270804in}}%
\pgfpathlineto{\pgfqpoint{2.677276in}{3.198169in}}%
\pgfpathlineto{\pgfqpoint{2.686104in}{2.206340in}}%
\pgfpathlineto{\pgfqpoint{2.690519in}{2.972759in}}%
\pgfpathlineto{\pgfqpoint{2.699348in}{3.218486in}}%
\pgfpathlineto{\pgfqpoint{2.703762in}{3.217606in}}%
\pgfpathlineto{\pgfqpoint{2.708177in}{3.011972in}}%
\pgfpathlineto{\pgfqpoint{2.712591in}{3.261475in}}%
\pgfpathlineto{\pgfqpoint{2.717006in}{2.695410in}}%
\pgfpathlineto{\pgfqpoint{2.725834in}{2.673960in}}%
\pgfpathlineto{\pgfqpoint{2.730249in}{2.738036in}}%
\pgfpathlineto{\pgfqpoint{2.734663in}{2.658898in}}%
\pgfpathlineto{\pgfqpoint{2.739078in}{2.133351in}}%
\pgfpathlineto{\pgfqpoint{2.743492in}{3.122225in}}%
\pgfpathlineto{\pgfqpoint{2.747907in}{3.024809in}}%
\pgfpathlineto{\pgfqpoint{2.752321in}{3.201844in}}%
\pgfpathlineto{\pgfqpoint{2.756736in}{3.180375in}}%
\pgfpathlineto{\pgfqpoint{2.761150in}{2.975989in}}%
\pgfpathlineto{\pgfqpoint{2.765564in}{2.936700in}}%
\pgfpathlineto{\pgfqpoint{2.769979in}{3.105255in}}%
\pgfpathlineto{\pgfqpoint{2.774393in}{1.638709in}}%
\pgfpathlineto{\pgfqpoint{2.778808in}{2.667929in}}%
\pgfpathlineto{\pgfqpoint{2.783222in}{2.717492in}}%
\pgfpathlineto{\pgfqpoint{2.787637in}{2.724178in}}%
\pgfpathlineto{\pgfqpoint{2.792051in}{2.194655in}}%
\pgfpathlineto{\pgfqpoint{2.796466in}{2.189720in}}%
\pgfpathlineto{\pgfqpoint{2.800880in}{2.956156in}}%
\pgfpathlineto{\pgfqpoint{2.805294in}{3.162850in}}%
\pgfpathlineto{\pgfqpoint{2.809709in}{3.197146in}}%
\pgfpathlineto{\pgfqpoint{2.814123in}{3.217867in}}%
\pgfpathlineto{\pgfqpoint{2.818538in}{3.182208in}}%
\pgfpathlineto{\pgfqpoint{2.822952in}{3.221713in}}%
\pgfpathlineto{\pgfqpoint{2.827367in}{2.688575in}}%
\pgfpathlineto{\pgfqpoint{2.831781in}{2.607387in}}%
\pgfpathlineto{\pgfqpoint{2.836196in}{2.289093in}}%
\pgfpathlineto{\pgfqpoint{2.840610in}{3.222486in}}%
\pgfpathlineto{\pgfqpoint{2.845024in}{2.769548in}}%
\pgfpathlineto{\pgfqpoint{2.849439in}{2.694150in}}%
\pgfpathlineto{\pgfqpoint{2.853853in}{2.753283in}}%
\pgfpathlineto{\pgfqpoint{2.858268in}{3.174968in}}%
\pgfpathlineto{\pgfqpoint{2.862682in}{3.149217in}}%
\pgfpathlineto{\pgfqpoint{2.867097in}{3.157336in}}%
\pgfpathlineto{\pgfqpoint{2.871511in}{3.110003in}}%
\pgfpathlineto{\pgfqpoint{2.880340in}{3.268521in}}%
\pgfpathlineto{\pgfqpoint{2.884754in}{2.687652in}}%
\pgfpathlineto{\pgfqpoint{2.889169in}{3.219959in}}%
\pgfpathlineto{\pgfqpoint{2.893583in}{2.750005in}}%
\pgfpathlineto{\pgfqpoint{2.897998in}{2.754962in}}%
\pgfpathlineto{\pgfqpoint{2.902412in}{2.197776in}}%
\pgfpathlineto{\pgfqpoint{2.906827in}{2.750840in}}%
\pgfpathlineto{\pgfqpoint{2.915656in}{3.180940in}}%
\pgfpathlineto{\pgfqpoint{2.920070in}{3.056586in}}%
\pgfpathlineto{\pgfqpoint{2.924485in}{3.136812in}}%
\pgfpathlineto{\pgfqpoint{2.928899in}{3.113060in}}%
\pgfpathlineto{\pgfqpoint{2.933313in}{3.236823in}}%
\pgfpathlineto{\pgfqpoint{2.937728in}{3.165431in}}%
\pgfpathlineto{\pgfqpoint{2.942142in}{3.075159in}}%
\pgfpathlineto{\pgfqpoint{2.946557in}{3.288098in}}%
\pgfpathlineto{\pgfqpoint{2.950971in}{3.230803in}}%
\pgfpathlineto{\pgfqpoint{2.955386in}{2.182036in}}%
\pgfpathlineto{\pgfqpoint{2.959800in}{2.225532in}}%
\pgfpathlineto{\pgfqpoint{2.968629in}{2.869815in}}%
\pgfpathlineto{\pgfqpoint{2.973043in}{3.110931in}}%
\pgfpathlineto{\pgfqpoint{2.977458in}{3.093724in}}%
\pgfpathlineto{\pgfqpoint{2.981872in}{3.166072in}}%
\pgfpathlineto{\pgfqpoint{2.986287in}{3.056639in}}%
\pgfpathlineto{\pgfqpoint{2.990701in}{3.159631in}}%
\pgfpathlineto{\pgfqpoint{2.995116in}{2.675110in}}%
\pgfpathlineto{\pgfqpoint{2.999530in}{2.748152in}}%
\pgfpathlineto{\pgfqpoint{3.003945in}{3.214667in}}%
\pgfpathlineto{\pgfqpoint{3.008359in}{2.755808in}}%
\pgfpathlineto{\pgfqpoint{3.012773in}{1.669777in}}%
\pgfpathlineto{\pgfqpoint{3.017188in}{2.752963in}}%
\pgfpathlineto{\pgfqpoint{3.021602in}{2.365057in}}%
\pgfpathlineto{\pgfqpoint{3.030431in}{3.165130in}}%
\pgfpathlineto{\pgfqpoint{3.034846in}{3.125689in}}%
\pgfpathlineto{\pgfqpoint{3.039260in}{3.036320in}}%
\pgfpathlineto{\pgfqpoint{3.043675in}{3.234273in}}%
\pgfpathlineto{\pgfqpoint{3.048089in}{2.687413in}}%
\pgfpathlineto{\pgfqpoint{3.052503in}{2.629346in}}%
\pgfpathlineto{\pgfqpoint{3.056918in}{2.648768in}}%
\pgfpathlineto{\pgfqpoint{3.061332in}{3.175994in}}%
\pgfpathlineto{\pgfqpoint{3.065747in}{2.748802in}}%
\pgfpathlineto{\pgfqpoint{3.070161in}{1.988602in}}%
\pgfpathlineto{\pgfqpoint{3.074576in}{2.937338in}}%
\pgfpathlineto{\pgfqpoint{3.078990in}{2.996202in}}%
\pgfpathlineto{\pgfqpoint{3.083405in}{3.167427in}}%
\pgfpathlineto{\pgfqpoint{3.087819in}{3.083735in}}%
\pgfpathlineto{\pgfqpoint{3.092233in}{3.102499in}}%
\pgfpathlineto{\pgfqpoint{3.096648in}{3.058872in}}%
\pgfpathlineto{\pgfqpoint{3.101062in}{3.141226in}}%
\pgfpathlineto{\pgfqpoint{3.105477in}{2.144370in}}%
\pgfpathlineto{\pgfqpoint{3.109891in}{2.414228in}}%
\pgfpathlineto{\pgfqpoint{3.114306in}{3.223411in}}%
\pgfpathlineto{\pgfqpoint{3.118720in}{2.340050in}}%
\pgfpathlineto{\pgfqpoint{3.123135in}{2.160907in}}%
\pgfpathlineto{\pgfqpoint{3.127549in}{2.732511in}}%
\pgfpathlineto{\pgfqpoint{3.131963in}{3.054019in}}%
\pgfpathlineto{\pgfqpoint{3.136378in}{2.763295in}}%
\pgfpathlineto{\pgfqpoint{3.140792in}{3.145058in}}%
\pgfpathlineto{\pgfqpoint{3.154036in}{3.219734in}}%
\pgfpathlineto{\pgfqpoint{3.158450in}{2.730225in}}%
\pgfpathlineto{\pgfqpoint{3.162865in}{2.684906in}}%
\pgfpathlineto{\pgfqpoint{3.167279in}{2.593498in}}%
\pgfpathlineto{\pgfqpoint{3.171693in}{3.209278in}}%
\pgfpathlineto{\pgfqpoint{3.176108in}{2.756562in}}%
\pgfpathlineto{\pgfqpoint{3.180522in}{2.509075in}}%
\pgfpathlineto{\pgfqpoint{3.184937in}{2.559689in}}%
\pgfpathlineto{\pgfqpoint{3.189351in}{3.081474in}}%
\pgfpathlineto{\pgfqpoint{3.193766in}{3.226963in}}%
\pgfpathlineto{\pgfqpoint{3.198180in}{3.148008in}}%
\pgfpathlineto{\pgfqpoint{3.207009in}{3.175359in}}%
\pgfpathlineto{\pgfqpoint{3.211423in}{3.270981in}}%
\pgfpathlineto{\pgfqpoint{3.220252in}{3.021798in}}%
\pgfpathlineto{\pgfqpoint{3.224667in}{3.222031in}}%
\pgfpathlineto{\pgfqpoint{3.229081in}{3.266358in}}%
\pgfpathlineto{\pgfqpoint{3.233496in}{3.248049in}}%
\pgfpathlineto{\pgfqpoint{3.237910in}{2.716685in}}%
\pgfpathlineto{\pgfqpoint{3.242325in}{3.203953in}}%
\pgfpathlineto{\pgfqpoint{3.246739in}{3.051162in}}%
\pgfpathlineto{\pgfqpoint{3.251153in}{3.059105in}}%
\pgfpathlineto{\pgfqpoint{3.255568in}{3.229175in}}%
\pgfpathlineto{\pgfqpoint{3.259982in}{3.186875in}}%
\pgfpathlineto{\pgfqpoint{3.264397in}{3.299822in}}%
\pgfpathlineto{\pgfqpoint{3.268811in}{3.171895in}}%
\pgfpathlineto{\pgfqpoint{3.273226in}{3.171349in}}%
\pgfpathlineto{\pgfqpoint{3.277640in}{3.109402in}}%
\pgfpathlineto{\pgfqpoint{3.282055in}{3.275066in}}%
\pgfpathlineto{\pgfqpoint{3.286469in}{2.145865in}}%
\pgfpathlineto{\pgfqpoint{3.290883in}{2.734046in}}%
\pgfpathlineto{\pgfqpoint{3.295298in}{2.553363in}}%
\pgfpathlineto{\pgfqpoint{3.299712in}{2.608959in}}%
\pgfpathlineto{\pgfqpoint{3.304127in}{3.233204in}}%
\pgfpathlineto{\pgfqpoint{3.308541in}{3.184392in}}%
\pgfpathlineto{\pgfqpoint{3.312956in}{3.193578in}}%
\pgfpathlineto{\pgfqpoint{3.317370in}{3.050485in}}%
\pgfpathlineto{\pgfqpoint{3.321785in}{3.139784in}}%
\pgfpathlineto{\pgfqpoint{3.326199in}{3.191224in}}%
\pgfpathlineto{\pgfqpoint{3.330613in}{2.650107in}}%
\pgfpathlineto{\pgfqpoint{3.335028in}{3.285041in}}%
\pgfpathlineto{\pgfqpoint{3.339442in}{2.620919in}}%
\pgfpathlineto{\pgfqpoint{3.343857in}{2.203202in}}%
\pgfpathlineto{\pgfqpoint{3.348271in}{2.198315in}}%
\pgfpathlineto{\pgfqpoint{3.352686in}{3.218635in}}%
\pgfpathlineto{\pgfqpoint{3.357100in}{3.134948in}}%
\pgfpathlineto{\pgfqpoint{3.365929in}{3.243124in}}%
\pgfpathlineto{\pgfqpoint{3.370343in}{2.839970in}}%
\pgfpathlineto{\pgfqpoint{3.374758in}{3.299277in}}%
\pgfpathlineto{\pgfqpoint{3.379172in}{2.958065in}}%
\pgfpathlineto{\pgfqpoint{3.383587in}{3.033784in}}%
\pgfpathlineto{\pgfqpoint{3.388001in}{2.660490in}}%
\pgfpathlineto{\pgfqpoint{3.392416in}{3.169589in}}%
\pgfpathlineto{\pgfqpoint{3.396830in}{3.220279in}}%
\pgfpathlineto{\pgfqpoint{3.401245in}{2.199491in}}%
\pgfpathlineto{\pgfqpoint{3.405659in}{2.976546in}}%
\pgfpathlineto{\pgfqpoint{3.410073in}{3.145578in}}%
\pgfpathlineto{\pgfqpoint{3.414488in}{3.194936in}}%
\pgfpathlineto{\pgfqpoint{3.418902in}{3.154885in}}%
\pgfpathlineto{\pgfqpoint{3.423317in}{2.626821in}}%
\pgfpathlineto{\pgfqpoint{3.427731in}{2.974749in}}%
\pgfpathlineto{\pgfqpoint{3.432146in}{3.080440in}}%
\pgfpathlineto{\pgfqpoint{3.436560in}{3.101512in}}%
\pgfpathlineto{\pgfqpoint{3.440975in}{2.368639in}}%
\pgfpathlineto{\pgfqpoint{3.445389in}{3.315432in}}%
\pgfpathlineto{\pgfqpoint{3.449803in}{3.100570in}}%
\pgfpathlineto{\pgfqpoint{3.454218in}{2.706543in}}%
\pgfpathlineto{\pgfqpoint{3.458632in}{2.684824in}}%
\pgfpathlineto{\pgfqpoint{3.463047in}{3.167481in}}%
\pgfpathlineto{\pgfqpoint{3.467461in}{3.208673in}}%
\pgfpathlineto{\pgfqpoint{3.471876in}{3.202426in}}%
\pgfpathlineto{\pgfqpoint{3.476290in}{3.184283in}}%
\pgfpathlineto{\pgfqpoint{3.480705in}{3.199086in}}%
\pgfpathlineto{\pgfqpoint{3.485119in}{3.142064in}}%
\pgfpathlineto{\pgfqpoint{3.489533in}{2.306663in}}%
\pgfpathlineto{\pgfqpoint{3.498362in}{2.711571in}}%
\pgfpathlineto{\pgfqpoint{3.502777in}{3.271040in}}%
\pgfpathlineto{\pgfqpoint{3.507191in}{3.268436in}}%
\pgfpathlineto{\pgfqpoint{3.511606in}{2.638512in}}%
\pgfpathlineto{\pgfqpoint{3.516020in}{3.096221in}}%
\pgfpathlineto{\pgfqpoint{3.520435in}{3.119785in}}%
\pgfpathlineto{\pgfqpoint{3.524849in}{3.205384in}}%
\pgfpathlineto{\pgfqpoint{3.529263in}{3.022175in}}%
\pgfpathlineto{\pgfqpoint{3.533678in}{2.559197in}}%
\pgfpathlineto{\pgfqpoint{3.538092in}{3.093814in}}%
\pgfpathlineto{\pgfqpoint{3.542507in}{3.223982in}}%
\pgfpathlineto{\pgfqpoint{3.546921in}{3.164483in}}%
\pgfpathlineto{\pgfqpoint{3.551336in}{2.539757in}}%
\pgfpathlineto{\pgfqpoint{3.555750in}{3.300283in}}%
\pgfpathlineto{\pgfqpoint{3.560165in}{2.745731in}}%
\pgfpathlineto{\pgfqpoint{3.564579in}{2.697144in}}%
\pgfpathlineto{\pgfqpoint{3.568994in}{2.714601in}}%
\pgfpathlineto{\pgfqpoint{3.573408in}{3.127385in}}%
\pgfpathlineto{\pgfqpoint{3.577822in}{3.201560in}}%
\pgfpathlineto{\pgfqpoint{3.582237in}{3.076711in}}%
\pgfpathlineto{\pgfqpoint{3.586651in}{3.056111in}}%
\pgfpathlineto{\pgfqpoint{3.591066in}{3.204006in}}%
\pgfpathlineto{\pgfqpoint{3.595480in}{3.262484in}}%
\pgfpathlineto{\pgfqpoint{3.599895in}{2.461699in}}%
\pgfpathlineto{\pgfqpoint{3.604309in}{2.735165in}}%
\pgfpathlineto{\pgfqpoint{3.608724in}{3.206716in}}%
\pgfpathlineto{\pgfqpoint{3.613138in}{3.271672in}}%
\pgfpathlineto{\pgfqpoint{3.617552in}{3.281423in}}%
\pgfpathlineto{\pgfqpoint{3.621967in}{2.664029in}}%
\pgfpathlineto{\pgfqpoint{3.626381in}{2.975753in}}%
\pgfpathlineto{\pgfqpoint{3.630796in}{2.551136in}}%
\pgfpathlineto{\pgfqpoint{3.635210in}{3.080872in}}%
\pgfpathlineto{\pgfqpoint{3.639625in}{3.087263in}}%
\pgfpathlineto{\pgfqpoint{3.644039in}{3.181255in}}%
\pgfpathlineto{\pgfqpoint{3.648454in}{3.222318in}}%
\pgfpathlineto{\pgfqpoint{3.652868in}{3.101498in}}%
\pgfpathlineto{\pgfqpoint{3.657282in}{3.280276in}}%
\pgfpathlineto{\pgfqpoint{3.661697in}{3.230460in}}%
\pgfpathlineto{\pgfqpoint{3.666111in}{3.223822in}}%
\pgfpathlineto{\pgfqpoint{3.674940in}{2.192636in}}%
\pgfpathlineto{\pgfqpoint{3.679355in}{2.143563in}}%
\pgfpathlineto{\pgfqpoint{3.683769in}{2.907454in}}%
\pgfpathlineto{\pgfqpoint{3.688184in}{2.886775in}}%
\pgfpathlineto{\pgfqpoint{3.692598in}{2.986406in}}%
\pgfpathlineto{\pgfqpoint{3.697012in}{3.181260in}}%
\pgfpathlineto{\pgfqpoint{3.701427in}{3.081308in}}%
\pgfpathlineto{\pgfqpoint{3.705841in}{3.258286in}}%
\pgfpathlineto{\pgfqpoint{3.710256in}{2.909236in}}%
\pgfpathlineto{\pgfqpoint{3.714670in}{3.094883in}}%
\pgfpathlineto{\pgfqpoint{3.719085in}{2.678979in}}%
\pgfpathlineto{\pgfqpoint{3.723499in}{3.267899in}}%
\pgfpathlineto{\pgfqpoint{3.727914in}{2.722735in}}%
\pgfpathlineto{\pgfqpoint{3.732328in}{2.739552in}}%
\pgfpathlineto{\pgfqpoint{3.736742in}{3.130120in}}%
\pgfpathlineto{\pgfqpoint{3.741157in}{2.602605in}}%
\pgfpathlineto{\pgfqpoint{3.745571in}{3.073841in}}%
\pgfpathlineto{\pgfqpoint{3.749986in}{3.202895in}}%
\pgfpathlineto{\pgfqpoint{3.754400in}{3.171392in}}%
\pgfpathlineto{\pgfqpoint{3.758815in}{3.194458in}}%
\pgfpathlineto{\pgfqpoint{3.763229in}{3.132909in}}%
\pgfpathlineto{\pgfqpoint{3.767644in}{3.178198in}}%
\pgfpathlineto{\pgfqpoint{3.772058in}{2.763813in}}%
\pgfpathlineto{\pgfqpoint{3.776472in}{3.255776in}}%
\pgfpathlineto{\pgfqpoint{3.780887in}{2.754121in}}%
\pgfpathlineto{\pgfqpoint{3.785301in}{3.199375in}}%
\pgfpathlineto{\pgfqpoint{3.789716in}{2.693799in}}%
\pgfpathlineto{\pgfqpoint{3.794130in}{3.121787in}}%
\pgfpathlineto{\pgfqpoint{3.798545in}{2.883637in}}%
\pgfpathlineto{\pgfqpoint{3.802959in}{3.010170in}}%
\pgfpathlineto{\pgfqpoint{3.807374in}{3.211611in}}%
\pgfpathlineto{\pgfqpoint{3.811788in}{3.043734in}}%
\pgfpathlineto{\pgfqpoint{3.816202in}{3.128234in}}%
\pgfpathlineto{\pgfqpoint{3.820617in}{2.559357in}}%
\pgfpathlineto{\pgfqpoint{3.825031in}{2.281252in}}%
\pgfpathlineto{\pgfqpoint{3.829446in}{3.230859in}}%
\pgfpathlineto{\pgfqpoint{3.833860in}{3.245066in}}%
\pgfpathlineto{\pgfqpoint{3.842689in}{2.222023in}}%
\pgfpathlineto{\pgfqpoint{3.847104in}{3.134973in}}%
\pgfpathlineto{\pgfqpoint{3.851518in}{3.145739in}}%
\pgfpathlineto{\pgfqpoint{3.855932in}{3.220201in}}%
\pgfpathlineto{\pgfqpoint{3.860347in}{3.225430in}}%
\pgfpathlineto{\pgfqpoint{3.864761in}{3.145081in}}%
\pgfpathlineto{\pgfqpoint{3.869176in}{3.240166in}}%
\pgfpathlineto{\pgfqpoint{3.873590in}{3.201523in}}%
\pgfpathlineto{\pgfqpoint{3.878005in}{3.190311in}}%
\pgfpathlineto{\pgfqpoint{3.882419in}{2.360662in}}%
\pgfpathlineto{\pgfqpoint{3.886834in}{3.304880in}}%
\pgfpathlineto{\pgfqpoint{3.891248in}{2.186929in}}%
\pgfpathlineto{\pgfqpoint{3.895662in}{2.213343in}}%
\pgfpathlineto{\pgfqpoint{3.904491in}{3.157216in}}%
\pgfpathlineto{\pgfqpoint{3.908906in}{3.194244in}}%
\pgfpathlineto{\pgfqpoint{3.913320in}{3.119774in}}%
\pgfpathlineto{\pgfqpoint{3.917735in}{3.079571in}}%
\pgfpathlineto{\pgfqpoint{3.922149in}{3.155121in}}%
\pgfpathlineto{\pgfqpoint{3.926564in}{2.732306in}}%
\pgfpathlineto{\pgfqpoint{3.930978in}{2.873495in}}%
\pgfpathlineto{\pgfqpoint{3.935392in}{2.456951in}}%
\pgfpathlineto{\pgfqpoint{3.939807in}{3.148542in}}%
\pgfpathlineto{\pgfqpoint{3.944221in}{3.271113in}}%
\pgfpathlineto{\pgfqpoint{3.948636in}{3.238023in}}%
\pgfpathlineto{\pgfqpoint{3.953050in}{2.144988in}}%
\pgfpathlineto{\pgfqpoint{3.957465in}{3.131383in}}%
\pgfpathlineto{\pgfqpoint{3.961879in}{3.144808in}}%
\pgfpathlineto{\pgfqpoint{3.966294in}{2.969435in}}%
\pgfpathlineto{\pgfqpoint{3.970708in}{3.209300in}}%
\pgfpathlineto{\pgfqpoint{3.975122in}{3.212531in}}%
\pgfpathlineto{\pgfqpoint{3.979537in}{3.056190in}}%
\pgfpathlineto{\pgfqpoint{3.983951in}{3.236904in}}%
\pgfpathlineto{\pgfqpoint{3.988366in}{2.811261in}}%
\pgfpathlineto{\pgfqpoint{3.992780in}{3.243076in}}%
\pgfpathlineto{\pgfqpoint{3.997195in}{3.262504in}}%
\pgfpathlineto{\pgfqpoint{4.001609in}{3.249528in}}%
\pgfpathlineto{\pgfqpoint{4.006024in}{2.115036in}}%
\pgfpathlineto{\pgfqpoint{4.014852in}{3.192956in}}%
\pgfpathlineto{\pgfqpoint{4.019267in}{3.235037in}}%
\pgfpathlineto{\pgfqpoint{4.023681in}{3.081331in}}%
\pgfpathlineto{\pgfqpoint{4.028096in}{3.076003in}}%
\pgfpathlineto{\pgfqpoint{4.032510in}{3.128000in}}%
\pgfpathlineto{\pgfqpoint{4.036925in}{2.744989in}}%
\pgfpathlineto{\pgfqpoint{4.041339in}{3.021340in}}%
\pgfpathlineto{\pgfqpoint{4.045754in}{2.929795in}}%
\pgfpathlineto{\pgfqpoint{4.050168in}{2.492556in}}%
\pgfpathlineto{\pgfqpoint{4.054582in}{3.260134in}}%
\pgfpathlineto{\pgfqpoint{4.058997in}{2.670004in}}%
\pgfpathlineto{\pgfqpoint{4.063411in}{2.701345in}}%
\pgfpathlineto{\pgfqpoint{4.067826in}{3.146897in}}%
\pgfpathlineto{\pgfqpoint{4.072240in}{3.157648in}}%
\pgfpathlineto{\pgfqpoint{4.076655in}{2.963143in}}%
\pgfpathlineto{\pgfqpoint{4.081069in}{3.121399in}}%
\pgfpathlineto{\pgfqpoint{4.085484in}{3.209480in}}%
\pgfpathlineto{\pgfqpoint{4.089898in}{3.073832in}}%
\pgfpathlineto{\pgfqpoint{4.094312in}{3.285826in}}%
\pgfpathlineto{\pgfqpoint{4.098727in}{2.485556in}}%
\pgfpathlineto{\pgfqpoint{4.103141in}{2.627305in}}%
\pgfpathlineto{\pgfqpoint{4.107556in}{3.234416in}}%
\pgfpathlineto{\pgfqpoint{4.111970in}{3.191764in}}%
\pgfpathlineto{\pgfqpoint{4.116385in}{3.200258in}}%
\pgfpathlineto{\pgfqpoint{4.120799in}{2.153639in}}%
\pgfpathlineto{\pgfqpoint{4.125214in}{2.184958in}}%
\pgfpathlineto{\pgfqpoint{4.129628in}{3.254536in}}%
\pgfpathlineto{\pgfqpoint{4.134042in}{3.067818in}}%
\pgfpathlineto{\pgfqpoint{4.138457in}{3.139719in}}%
\pgfpathlineto{\pgfqpoint{4.142871in}{3.035687in}}%
\pgfpathlineto{\pgfqpoint{4.147286in}{3.258539in}}%
\pgfpathlineto{\pgfqpoint{4.156115in}{2.419680in}}%
\pgfpathlineto{\pgfqpoint{4.160529in}{3.033992in}}%
\pgfpathlineto{\pgfqpoint{4.164944in}{3.205077in}}%
\pgfpathlineto{\pgfqpoint{4.169358in}{2.627049in}}%
\pgfpathlineto{\pgfqpoint{4.173772in}{2.204971in}}%
\pgfpathlineto{\pgfqpoint{4.178187in}{3.137214in}}%
\pgfpathlineto{\pgfqpoint{4.182601in}{3.172198in}}%
\pgfpathlineto{\pgfqpoint{4.187016in}{3.062201in}}%
\pgfpathlineto{\pgfqpoint{4.191430in}{3.080535in}}%
\pgfpathlineto{\pgfqpoint{4.195845in}{3.109312in}}%
\pgfpathlineto{\pgfqpoint{4.200259in}{2.717627in}}%
\pgfpathlineto{\pgfqpoint{4.204674in}{3.284291in}}%
\pgfpathlineto{\pgfqpoint{4.209088in}{2.751132in}}%
\pgfpathlineto{\pgfqpoint{4.213503in}{2.536547in}}%
\pgfpathlineto{\pgfqpoint{4.217917in}{3.178541in}}%
\pgfpathlineto{\pgfqpoint{4.222331in}{3.153111in}}%
\pgfpathlineto{\pgfqpoint{4.226746in}{2.747379in}}%
\pgfpathlineto{\pgfqpoint{4.231160in}{1.620990in}}%
\pgfpathlineto{\pgfqpoint{4.235575in}{2.690090in}}%
\pgfpathlineto{\pgfqpoint{4.239989in}{3.186307in}}%
\pgfpathlineto{\pgfqpoint{4.244404in}{3.156752in}}%
\pgfpathlineto{\pgfqpoint{4.248818in}{3.019234in}}%
\pgfpathlineto{\pgfqpoint{4.253233in}{3.092467in}}%
\pgfpathlineto{\pgfqpoint{4.257647in}{3.288770in}}%
\pgfpathlineto{\pgfqpoint{4.262061in}{2.670052in}}%
\pgfpathlineto{\pgfqpoint{4.266476in}{2.743164in}}%
\pgfpathlineto{\pgfqpoint{4.275305in}{3.242168in}}%
\pgfpathlineto{\pgfqpoint{4.279719in}{2.688620in}}%
\pgfpathlineto{\pgfqpoint{4.284134in}{3.257536in}}%
\pgfpathlineto{\pgfqpoint{4.288548in}{3.204796in}}%
\pgfpathlineto{\pgfqpoint{4.292963in}{3.068994in}}%
\pgfpathlineto{\pgfqpoint{4.297377in}{3.018230in}}%
\pgfpathlineto{\pgfqpoint{4.301791in}{3.156290in}}%
\pgfpathlineto{\pgfqpoint{4.306206in}{3.139820in}}%
\pgfpathlineto{\pgfqpoint{4.310620in}{3.220139in}}%
\pgfpathlineto{\pgfqpoint{4.315035in}{3.170677in}}%
\pgfpathlineto{\pgfqpoint{4.319449in}{3.011575in}}%
\pgfpathlineto{\pgfqpoint{4.323864in}{3.051098in}}%
\pgfpathlineto{\pgfqpoint{4.323864in}{3.051098in}}%
\pgfusepath{stroke}%
\end{pgfscope}%
\begin{pgfscope}%
\pgfpathrectangle{\pgfqpoint{0.625000in}{0.440000in}}{\pgfqpoint{3.875000in}{3.080000in}} %
\pgfusepath{clip}%
\pgfsetrectcap%
\pgfsetroundjoin%
\pgfsetlinewidth{1.505625pt}%
\definecolor{currentstroke}{rgb}{0.901961,0.901961,0.980392}%
\pgfsetstrokecolor{currentstroke}%
\pgfsetdash{}{0pt}%
\pgfpathmoveto{\pgfqpoint{0.801136in}{0.580000in}}%
\pgfpathlineto{\pgfqpoint{0.805551in}{0.580000in}}%
\pgfpathlineto{\pgfqpoint{0.809965in}{1.277895in}}%
\pgfpathlineto{\pgfqpoint{0.814380in}{1.206700in}}%
\pgfpathlineto{\pgfqpoint{0.818794in}{1.203438in}}%
\pgfpathlineto{\pgfqpoint{0.827623in}{1.919209in}}%
\pgfpathlineto{\pgfqpoint{0.832037in}{1.677075in}}%
\pgfpathlineto{\pgfqpoint{0.836452in}{1.628078in}}%
\pgfpathlineto{\pgfqpoint{0.840866in}{1.838882in}}%
\pgfpathlineto{\pgfqpoint{0.845281in}{1.776052in}}%
\pgfpathlineto{\pgfqpoint{0.849695in}{1.873302in}}%
\pgfpathlineto{\pgfqpoint{0.854110in}{1.209787in}}%
\pgfpathlineto{\pgfqpoint{0.858524in}{1.992586in}}%
\pgfpathlineto{\pgfqpoint{0.862939in}{1.383338in}}%
\pgfpathlineto{\pgfqpoint{0.867353in}{2.138583in}}%
\pgfpathlineto{\pgfqpoint{0.871767in}{1.964557in}}%
\pgfpathlineto{\pgfqpoint{0.876182in}{1.861215in}}%
\pgfpathlineto{\pgfqpoint{0.880596in}{2.019350in}}%
\pgfpathlineto{\pgfqpoint{0.889425in}{1.684169in}}%
\pgfpathlineto{\pgfqpoint{0.893840in}{1.838081in}}%
\pgfpathlineto{\pgfqpoint{0.898254in}{1.520563in}}%
\pgfpathlineto{\pgfqpoint{0.902669in}{1.612566in}}%
\pgfpathlineto{\pgfqpoint{0.907083in}{1.512449in}}%
\pgfpathlineto{\pgfqpoint{0.911497in}{1.453321in}}%
\pgfpathlineto{\pgfqpoint{0.915912in}{1.934594in}}%
\pgfpathlineto{\pgfqpoint{0.924741in}{1.770566in}}%
\pgfpathlineto{\pgfqpoint{0.929155in}{2.195009in}}%
\pgfpathlineto{\pgfqpoint{0.933570in}{2.133472in}}%
\pgfpathlineto{\pgfqpoint{0.937984in}{2.159015in}}%
\pgfpathlineto{\pgfqpoint{0.942399in}{1.919966in}}%
\pgfpathlineto{\pgfqpoint{0.946813in}{2.000723in}}%
\pgfpathlineto{\pgfqpoint{0.951228in}{2.397024in}}%
\pgfpathlineto{\pgfqpoint{0.955642in}{2.245924in}}%
\pgfpathlineto{\pgfqpoint{0.960056in}{1.731828in}}%
\pgfpathlineto{\pgfqpoint{0.964471in}{1.549075in}}%
\pgfpathlineto{\pgfqpoint{0.968885in}{2.092732in}}%
\pgfpathlineto{\pgfqpoint{0.973300in}{1.556546in}}%
\pgfpathlineto{\pgfqpoint{0.977714in}{2.172604in}}%
\pgfpathlineto{\pgfqpoint{0.982129in}{2.080316in}}%
\pgfpathlineto{\pgfqpoint{0.986543in}{2.066649in}}%
\pgfpathlineto{\pgfqpoint{0.995372in}{1.796110in}}%
\pgfpathlineto{\pgfqpoint{0.999786in}{1.880856in}}%
\pgfpathlineto{\pgfqpoint{1.004201in}{2.271096in}}%
\pgfpathlineto{\pgfqpoint{1.008615in}{1.328641in}}%
\pgfpathlineto{\pgfqpoint{1.013030in}{1.840595in}}%
\pgfpathlineto{\pgfqpoint{1.017444in}{1.490305in}}%
\pgfpathlineto{\pgfqpoint{1.021859in}{1.965359in}}%
\pgfpathlineto{\pgfqpoint{1.026273in}{1.971952in}}%
\pgfpathlineto{\pgfqpoint{1.030688in}{1.638872in}}%
\pgfpathlineto{\pgfqpoint{1.035102in}{1.212975in}}%
\pgfpathlineto{\pgfqpoint{1.039516in}{2.089431in}}%
\pgfpathlineto{\pgfqpoint{1.043931in}{2.235308in}}%
\pgfpathlineto{\pgfqpoint{1.048345in}{2.078511in}}%
\pgfpathlineto{\pgfqpoint{1.052760in}{2.126704in}}%
\pgfpathlineto{\pgfqpoint{1.057174in}{2.428303in}}%
\pgfpathlineto{\pgfqpoint{1.061589in}{1.894869in}}%
\pgfpathlineto{\pgfqpoint{1.066003in}{1.572999in}}%
\pgfpathlineto{\pgfqpoint{1.074832in}{1.896497in}}%
\pgfpathlineto{\pgfqpoint{1.079246in}{2.469161in}}%
\pgfpathlineto{\pgfqpoint{1.083661in}{1.934943in}}%
\pgfpathlineto{\pgfqpoint{1.088075in}{2.083010in}}%
\pgfpathlineto{\pgfqpoint{1.092490in}{2.102134in}}%
\pgfpathlineto{\pgfqpoint{1.096904in}{2.112199in}}%
\pgfpathlineto{\pgfqpoint{1.101319in}{2.139534in}}%
\pgfpathlineto{\pgfqpoint{1.105733in}{1.980271in}}%
\pgfpathlineto{\pgfqpoint{1.110148in}{2.323313in}}%
\pgfpathlineto{\pgfqpoint{1.114562in}{2.254634in}}%
\pgfpathlineto{\pgfqpoint{1.118976in}{2.099207in}}%
\pgfpathlineto{\pgfqpoint{1.123391in}{2.014187in}}%
\pgfpathlineto{\pgfqpoint{1.127805in}{1.846949in}}%
\pgfpathlineto{\pgfqpoint{1.132220in}{2.042753in}}%
\pgfpathlineto{\pgfqpoint{1.136634in}{2.501621in}}%
\pgfpathlineto{\pgfqpoint{1.145463in}{0.933931in}}%
\pgfpathlineto{\pgfqpoint{1.149878in}{2.012354in}}%
\pgfpathlineto{\pgfqpoint{1.154292in}{2.346778in}}%
\pgfpathlineto{\pgfqpoint{1.158706in}{1.865581in}}%
\pgfpathlineto{\pgfqpoint{1.163121in}{1.957675in}}%
\pgfpathlineto{\pgfqpoint{1.167535in}{2.296755in}}%
\pgfpathlineto{\pgfqpoint{1.171950in}{1.677019in}}%
\pgfpathlineto{\pgfqpoint{1.176364in}{2.014955in}}%
\pgfpathlineto{\pgfqpoint{1.180779in}{1.902376in}}%
\pgfpathlineto{\pgfqpoint{1.185193in}{1.859868in}}%
\pgfpathlineto{\pgfqpoint{1.189608in}{2.459959in}}%
\pgfpathlineto{\pgfqpoint{1.194022in}{1.679957in}}%
\pgfpathlineto{\pgfqpoint{1.198436in}{1.838753in}}%
\pgfpathlineto{\pgfqpoint{1.202851in}{2.203053in}}%
\pgfpathlineto{\pgfqpoint{1.207265in}{2.063998in}}%
\pgfpathlineto{\pgfqpoint{1.211680in}{2.033902in}}%
\pgfpathlineto{\pgfqpoint{1.216094in}{2.039798in}}%
\pgfpathlineto{\pgfqpoint{1.220509in}{2.238378in}}%
\pgfpathlineto{\pgfqpoint{1.224923in}{2.077535in}}%
\pgfpathlineto{\pgfqpoint{1.229338in}{2.097292in}}%
\pgfpathlineto{\pgfqpoint{1.238166in}{1.876158in}}%
\pgfpathlineto{\pgfqpoint{1.246995in}{2.104594in}}%
\pgfpathlineto{\pgfqpoint{1.251410in}{1.874955in}}%
\pgfpathlineto{\pgfqpoint{1.255824in}{0.872607in}}%
\pgfpathlineto{\pgfqpoint{1.260239in}{2.203351in}}%
\pgfpathlineto{\pgfqpoint{1.264653in}{2.346615in}}%
\pgfpathlineto{\pgfqpoint{1.273482in}{1.861819in}}%
\pgfpathlineto{\pgfqpoint{1.277896in}{2.221520in}}%
\pgfpathlineto{\pgfqpoint{1.282311in}{2.340118in}}%
\pgfpathlineto{\pgfqpoint{1.286725in}{2.392177in}}%
\pgfpathlineto{\pgfqpoint{1.291140in}{1.946378in}}%
\pgfpathlineto{\pgfqpoint{1.295554in}{1.680722in}}%
\pgfpathlineto{\pgfqpoint{1.299969in}{1.734066in}}%
\pgfpathlineto{\pgfqpoint{1.304383in}{1.709665in}}%
\pgfpathlineto{\pgfqpoint{1.308798in}{1.474394in}}%
\pgfpathlineto{\pgfqpoint{1.313212in}{2.044755in}}%
\pgfpathlineto{\pgfqpoint{1.317626in}{2.069148in}}%
\pgfpathlineto{\pgfqpoint{1.322041in}{1.887652in}}%
\pgfpathlineto{\pgfqpoint{1.326455in}{2.163067in}}%
\pgfpathlineto{\pgfqpoint{1.330870in}{2.093286in}}%
\pgfpathlineto{\pgfqpoint{1.335284in}{2.096620in}}%
\pgfpathlineto{\pgfqpoint{1.339699in}{1.698885in}}%
\pgfpathlineto{\pgfqpoint{1.344113in}{1.989063in}}%
\pgfpathlineto{\pgfqpoint{1.348528in}{1.581844in}}%
\pgfpathlineto{\pgfqpoint{1.352942in}{2.071007in}}%
\pgfpathlineto{\pgfqpoint{1.357356in}{2.079458in}}%
\pgfpathlineto{\pgfqpoint{1.361771in}{2.158498in}}%
\pgfpathlineto{\pgfqpoint{1.366185in}{0.836889in}}%
\pgfpathlineto{\pgfqpoint{1.370600in}{2.143430in}}%
\pgfpathlineto{\pgfqpoint{1.375014in}{1.985310in}}%
\pgfpathlineto{\pgfqpoint{1.379429in}{1.902185in}}%
\pgfpathlineto{\pgfqpoint{1.383843in}{2.007150in}}%
\pgfpathlineto{\pgfqpoint{1.388258in}{1.995243in}}%
\pgfpathlineto{\pgfqpoint{1.392672in}{2.088605in}}%
\pgfpathlineto{\pgfqpoint{1.397086in}{2.019445in}}%
\pgfpathlineto{\pgfqpoint{1.401501in}{2.094877in}}%
\pgfpathlineto{\pgfqpoint{1.405915in}{1.588246in}}%
\pgfpathlineto{\pgfqpoint{1.410330in}{2.060300in}}%
\pgfpathlineto{\pgfqpoint{1.414744in}{2.246964in}}%
\pgfpathlineto{\pgfqpoint{1.419159in}{1.101493in}}%
\pgfpathlineto{\pgfqpoint{1.423573in}{1.952257in}}%
\pgfpathlineto{\pgfqpoint{1.427988in}{2.233452in}}%
\pgfpathlineto{\pgfqpoint{1.432402in}{2.157958in}}%
\pgfpathlineto{\pgfqpoint{1.436816in}{2.150235in}}%
\pgfpathlineto{\pgfqpoint{1.441231in}{1.850770in}}%
\pgfpathlineto{\pgfqpoint{1.445645in}{2.318443in}}%
\pgfpathlineto{\pgfqpoint{1.450060in}{1.845161in}}%
\pgfpathlineto{\pgfqpoint{1.454474in}{1.833102in}}%
\pgfpathlineto{\pgfqpoint{1.458889in}{1.907963in}}%
\pgfpathlineto{\pgfqpoint{1.463303in}{2.259647in}}%
\pgfpathlineto{\pgfqpoint{1.467718in}{2.413604in}}%
\pgfpathlineto{\pgfqpoint{1.472132in}{2.347659in}}%
\pgfpathlineto{\pgfqpoint{1.476546in}{0.690752in}}%
\pgfpathlineto{\pgfqpoint{1.480961in}{2.069834in}}%
\pgfpathlineto{\pgfqpoint{1.485375in}{1.899514in}}%
\pgfpathlineto{\pgfqpoint{1.489790in}{2.119712in}}%
\pgfpathlineto{\pgfqpoint{1.494204in}{2.039658in}}%
\pgfpathlineto{\pgfqpoint{1.498619in}{2.062282in}}%
\pgfpathlineto{\pgfqpoint{1.503033in}{1.693577in}}%
\pgfpathlineto{\pgfqpoint{1.507448in}{1.893514in}}%
\pgfpathlineto{\pgfqpoint{1.511862in}{2.023429in}}%
\pgfpathlineto{\pgfqpoint{1.516276in}{1.882006in}}%
\pgfpathlineto{\pgfqpoint{1.520691in}{2.528978in}}%
\pgfpathlineto{\pgfqpoint{1.525105in}{1.973225in}}%
\pgfpathlineto{\pgfqpoint{1.529520in}{1.158450in}}%
\pgfpathlineto{\pgfqpoint{1.533934in}{2.320932in}}%
\pgfpathlineto{\pgfqpoint{1.538349in}{2.286417in}}%
\pgfpathlineto{\pgfqpoint{1.542763in}{2.358407in}}%
\pgfpathlineto{\pgfqpoint{1.547178in}{2.005862in}}%
\pgfpathlineto{\pgfqpoint{1.551592in}{1.950643in}}%
\pgfpathlineto{\pgfqpoint{1.556006in}{2.396546in}}%
\pgfpathlineto{\pgfqpoint{1.560421in}{1.790616in}}%
\pgfpathlineto{\pgfqpoint{1.564835in}{2.056839in}}%
\pgfpathlineto{\pgfqpoint{1.569250in}{1.677711in}}%
\pgfpathlineto{\pgfqpoint{1.573664in}{2.182239in}}%
\pgfpathlineto{\pgfqpoint{1.578079in}{2.013035in}}%
\pgfpathlineto{\pgfqpoint{1.582493in}{2.044853in}}%
\pgfpathlineto{\pgfqpoint{1.586908in}{1.068983in}}%
\pgfpathlineto{\pgfqpoint{1.591322in}{2.205952in}}%
\pgfpathlineto{\pgfqpoint{1.600151in}{2.361750in}}%
\pgfpathlineto{\pgfqpoint{1.604565in}{2.193853in}}%
\pgfpathlineto{\pgfqpoint{1.608980in}{2.507896in}}%
\pgfpathlineto{\pgfqpoint{1.613394in}{2.046256in}}%
\pgfpathlineto{\pgfqpoint{1.617809in}{2.360952in}}%
\pgfpathlineto{\pgfqpoint{1.622223in}{1.881430in}}%
\pgfpathlineto{\pgfqpoint{1.626638in}{2.327823in}}%
\pgfpathlineto{\pgfqpoint{1.631052in}{2.469423in}}%
\pgfpathlineto{\pgfqpoint{1.635467in}{2.430631in}}%
\pgfpathlineto{\pgfqpoint{1.639881in}{1.046186in}}%
\pgfpathlineto{\pgfqpoint{1.644295in}{2.339182in}}%
\pgfpathlineto{\pgfqpoint{1.648710in}{2.247040in}}%
\pgfpathlineto{\pgfqpoint{1.653124in}{2.337863in}}%
\pgfpathlineto{\pgfqpoint{1.657539in}{2.146231in}}%
\pgfpathlineto{\pgfqpoint{1.661953in}{2.204130in}}%
\pgfpathlineto{\pgfqpoint{1.666368in}{2.011753in}}%
\pgfpathlineto{\pgfqpoint{1.670782in}{1.191475in}}%
\pgfpathlineto{\pgfqpoint{1.675197in}{2.158962in}}%
\pgfpathlineto{\pgfqpoint{1.679611in}{2.123688in}}%
\pgfpathlineto{\pgfqpoint{1.684025in}{2.299566in}}%
\pgfpathlineto{\pgfqpoint{1.688440in}{1.995479in}}%
\pgfpathlineto{\pgfqpoint{1.692854in}{2.396572in}}%
\pgfpathlineto{\pgfqpoint{1.697269in}{1.401905in}}%
\pgfpathlineto{\pgfqpoint{1.701683in}{2.460476in}}%
\pgfpathlineto{\pgfqpoint{1.706098in}{2.524072in}}%
\pgfpathlineto{\pgfqpoint{1.714927in}{2.306362in}}%
\pgfpathlineto{\pgfqpoint{1.719341in}{2.442578in}}%
\pgfpathlineto{\pgfqpoint{1.723755in}{1.650925in}}%
\pgfpathlineto{\pgfqpoint{1.728170in}{1.650717in}}%
\pgfpathlineto{\pgfqpoint{1.732584in}{2.118424in}}%
\pgfpathlineto{\pgfqpoint{1.736999in}{1.762927in}}%
\pgfpathlineto{\pgfqpoint{1.741413in}{2.212202in}}%
\pgfpathlineto{\pgfqpoint{1.745828in}{1.805745in}}%
\pgfpathlineto{\pgfqpoint{1.750242in}{1.619652in}}%
\pgfpathlineto{\pgfqpoint{1.754657in}{2.168217in}}%
\pgfpathlineto{\pgfqpoint{1.759071in}{2.324674in}}%
\pgfpathlineto{\pgfqpoint{1.763485in}{2.356436in}}%
\pgfpathlineto{\pgfqpoint{1.767900in}{2.244903in}}%
\pgfpathlineto{\pgfqpoint{1.772314in}{2.475021in}}%
\pgfpathlineto{\pgfqpoint{1.776729in}{2.485362in}}%
\pgfpathlineto{\pgfqpoint{1.781143in}{1.988476in}}%
\pgfpathlineto{\pgfqpoint{1.785558in}{1.908941in}}%
\pgfpathlineto{\pgfqpoint{1.789972in}{1.965834in}}%
\pgfpathlineto{\pgfqpoint{1.794387in}{2.419070in}}%
\pgfpathlineto{\pgfqpoint{1.798801in}{2.453076in}}%
\pgfpathlineto{\pgfqpoint{1.803215in}{1.957517in}}%
\pgfpathlineto{\pgfqpoint{1.807630in}{1.133832in}}%
\pgfpathlineto{\pgfqpoint{1.812044in}{2.348502in}}%
\pgfpathlineto{\pgfqpoint{1.816459in}{2.300879in}}%
\pgfpathlineto{\pgfqpoint{1.820873in}{2.085495in}}%
\pgfpathlineto{\pgfqpoint{1.825288in}{2.216802in}}%
\pgfpathlineto{\pgfqpoint{1.829702in}{2.404540in}}%
\pgfpathlineto{\pgfqpoint{1.834117in}{1.308673in}}%
\pgfpathlineto{\pgfqpoint{1.838531in}{2.064149in}}%
\pgfpathlineto{\pgfqpoint{1.842945in}{2.409764in}}%
\pgfpathlineto{\pgfqpoint{1.847360in}{2.056985in}}%
\pgfpathlineto{\pgfqpoint{1.851774in}{2.125431in}}%
\pgfpathlineto{\pgfqpoint{1.856189in}{1.844556in}}%
\pgfpathlineto{\pgfqpoint{1.860603in}{1.333297in}}%
\pgfpathlineto{\pgfqpoint{1.865018in}{2.327409in}}%
\pgfpathlineto{\pgfqpoint{1.869432in}{2.417411in}}%
\pgfpathlineto{\pgfqpoint{1.873847in}{2.553354in}}%
\pgfpathlineto{\pgfqpoint{1.882675in}{2.237779in}}%
\pgfpathlineto{\pgfqpoint{1.887090in}{2.473123in}}%
\pgfpathlineto{\pgfqpoint{1.895919in}{2.115750in}}%
\pgfpathlineto{\pgfqpoint{1.900333in}{2.351322in}}%
\pgfpathlineto{\pgfqpoint{1.904748in}{2.371073in}}%
\pgfpathlineto{\pgfqpoint{1.909162in}{2.073835in}}%
\pgfpathlineto{\pgfqpoint{1.913577in}{1.695404in}}%
\pgfpathlineto{\pgfqpoint{1.917991in}{1.654645in}}%
\pgfpathlineto{\pgfqpoint{1.922405in}{2.242626in}}%
\pgfpathlineto{\pgfqpoint{1.926820in}{2.145570in}}%
\pgfpathlineto{\pgfqpoint{1.931234in}{2.430103in}}%
\pgfpathlineto{\pgfqpoint{1.935649in}{2.359383in}}%
\pgfpathlineto{\pgfqpoint{1.940063in}{2.406598in}}%
\pgfpathlineto{\pgfqpoint{1.944478in}{1.996938in}}%
\pgfpathlineto{\pgfqpoint{1.948892in}{2.536184in}}%
\pgfpathlineto{\pgfqpoint{1.953307in}{2.345305in}}%
\pgfpathlineto{\pgfqpoint{1.957721in}{2.302752in}}%
\pgfpathlineto{\pgfqpoint{1.962135in}{2.034676in}}%
\pgfpathlineto{\pgfqpoint{1.966550in}{2.564162in}}%
\pgfpathlineto{\pgfqpoint{1.970964in}{1.208364in}}%
\pgfpathlineto{\pgfqpoint{1.975379in}{2.235479in}}%
\pgfpathlineto{\pgfqpoint{1.979793in}{2.364103in}}%
\pgfpathlineto{\pgfqpoint{1.984208in}{2.081868in}}%
\pgfpathlineto{\pgfqpoint{1.988622in}{2.348654in}}%
\pgfpathlineto{\pgfqpoint{1.993037in}{2.171262in}}%
\pgfpathlineto{\pgfqpoint{1.997451in}{2.194168in}}%
\pgfpathlineto{\pgfqpoint{2.001865in}{2.009962in}}%
\pgfpathlineto{\pgfqpoint{2.006280in}{2.033728in}}%
\pgfpathlineto{\pgfqpoint{2.010694in}{1.955375in}}%
\pgfpathlineto{\pgfqpoint{2.015109in}{2.443320in}}%
\pgfpathlineto{\pgfqpoint{2.019523in}{1.721465in}}%
\pgfpathlineto{\pgfqpoint{2.023938in}{1.945574in}}%
\pgfpathlineto{\pgfqpoint{2.028352in}{2.542507in}}%
\pgfpathlineto{\pgfqpoint{2.032767in}{2.498672in}}%
\pgfpathlineto{\pgfqpoint{2.041595in}{2.108280in}}%
\pgfpathlineto{\pgfqpoint{2.046010in}{2.437326in}}%
\pgfpathlineto{\pgfqpoint{2.050424in}{2.575838in}}%
\pgfpathlineto{\pgfqpoint{2.054839in}{2.217507in}}%
\pgfpathlineto{\pgfqpoint{2.059253in}{1.954214in}}%
\pgfpathlineto{\pgfqpoint{2.068082in}{2.232445in}}%
\pgfpathlineto{\pgfqpoint{2.072497in}{2.160266in}}%
\pgfpathlineto{\pgfqpoint{2.076911in}{2.020674in}}%
\pgfpathlineto{\pgfqpoint{2.081325in}{1.324013in}}%
\pgfpathlineto{\pgfqpoint{2.085740in}{2.259237in}}%
\pgfpathlineto{\pgfqpoint{2.090154in}{2.114899in}}%
\pgfpathlineto{\pgfqpoint{2.094569in}{2.237273in}}%
\pgfpathlineto{\pgfqpoint{2.098983in}{2.458531in}}%
\pgfpathlineto{\pgfqpoint{2.103398in}{2.263868in}}%
\pgfpathlineto{\pgfqpoint{2.107812in}{2.310309in}}%
\pgfpathlineto{\pgfqpoint{2.112227in}{2.051140in}}%
\pgfpathlineto{\pgfqpoint{2.116641in}{1.598635in}}%
\pgfpathlineto{\pgfqpoint{2.121055in}{1.968016in}}%
\pgfpathlineto{\pgfqpoint{2.125470in}{2.102814in}}%
\pgfpathlineto{\pgfqpoint{2.129884in}{2.543024in}}%
\pgfpathlineto{\pgfqpoint{2.134299in}{1.945838in}}%
\pgfpathlineto{\pgfqpoint{2.138713in}{2.541922in}}%
\pgfpathlineto{\pgfqpoint{2.143128in}{2.430775in}}%
\pgfpathlineto{\pgfqpoint{2.147542in}{2.172879in}}%
\pgfpathlineto{\pgfqpoint{2.151957in}{1.994563in}}%
\pgfpathlineto{\pgfqpoint{2.156371in}{2.516700in}}%
\pgfpathlineto{\pgfqpoint{2.160785in}{2.551158in}}%
\pgfpathlineto{\pgfqpoint{2.169614in}{1.723655in}}%
\pgfpathlineto{\pgfqpoint{2.174029in}{2.322332in}}%
\pgfpathlineto{\pgfqpoint{2.178443in}{2.275004in}}%
\pgfpathlineto{\pgfqpoint{2.182858in}{2.284569in}}%
\pgfpathlineto{\pgfqpoint{2.187272in}{1.924144in}}%
\pgfpathlineto{\pgfqpoint{2.191687in}{1.808556in}}%
\pgfpathlineto{\pgfqpoint{2.196101in}{2.426203in}}%
\pgfpathlineto{\pgfqpoint{2.200515in}{2.387923in}}%
\pgfpathlineto{\pgfqpoint{2.204930in}{2.396934in}}%
\pgfpathlineto{\pgfqpoint{2.209344in}{2.437790in}}%
\pgfpathlineto{\pgfqpoint{2.213759in}{2.264112in}}%
\pgfpathlineto{\pgfqpoint{2.218173in}{2.413270in}}%
\pgfpathlineto{\pgfqpoint{2.222588in}{2.310430in}}%
\pgfpathlineto{\pgfqpoint{2.227002in}{2.056842in}}%
\pgfpathlineto{\pgfqpoint{2.231417in}{2.130202in}}%
\pgfpathlineto{\pgfqpoint{2.235831in}{2.091590in}}%
\pgfpathlineto{\pgfqpoint{2.240246in}{2.159226in}}%
\pgfpathlineto{\pgfqpoint{2.244660in}{2.153890in}}%
\pgfpathlineto{\pgfqpoint{2.249074in}{2.672090in}}%
\pgfpathlineto{\pgfqpoint{2.257903in}{2.299243in}}%
\pgfpathlineto{\pgfqpoint{2.262318in}{2.459607in}}%
\pgfpathlineto{\pgfqpoint{2.266732in}{2.386756in}}%
\pgfpathlineto{\pgfqpoint{2.271147in}{2.286057in}}%
\pgfpathlineto{\pgfqpoint{2.275561in}{2.326484in}}%
\pgfpathlineto{\pgfqpoint{2.279976in}{2.107549in}}%
\pgfpathlineto{\pgfqpoint{2.284390in}{2.331697in}}%
\pgfpathlineto{\pgfqpoint{2.288804in}{2.199654in}}%
\pgfpathlineto{\pgfqpoint{2.293219in}{2.387552in}}%
\pgfpathlineto{\pgfqpoint{2.297633in}{2.191312in}}%
\pgfpathlineto{\pgfqpoint{2.302048in}{2.459689in}}%
\pgfpathlineto{\pgfqpoint{2.306462in}{2.460400in}}%
\pgfpathlineto{\pgfqpoint{2.310877in}{2.382053in}}%
\pgfpathlineto{\pgfqpoint{2.315291in}{2.105637in}}%
\pgfpathlineto{\pgfqpoint{2.319706in}{2.280737in}}%
\pgfpathlineto{\pgfqpoint{2.324120in}{2.525148in}}%
\pgfpathlineto{\pgfqpoint{2.328534in}{2.626048in}}%
\pgfpathlineto{\pgfqpoint{2.332949in}{1.753632in}}%
\pgfpathlineto{\pgfqpoint{2.337363in}{2.200118in}}%
\pgfpathlineto{\pgfqpoint{2.341778in}{2.509153in}}%
\pgfpathlineto{\pgfqpoint{2.346192in}{2.533611in}}%
\pgfpathlineto{\pgfqpoint{2.350607in}{2.579665in}}%
\pgfpathlineto{\pgfqpoint{2.355021in}{2.347698in}}%
\pgfpathlineto{\pgfqpoint{2.359436in}{2.576440in}}%
\pgfpathlineto{\pgfqpoint{2.363850in}{2.239803in}}%
\pgfpathlineto{\pgfqpoint{2.368264in}{2.430235in}}%
\pgfpathlineto{\pgfqpoint{2.372679in}{2.236460in}}%
\pgfpathlineto{\pgfqpoint{2.377093in}{2.329487in}}%
\pgfpathlineto{\pgfqpoint{2.381508in}{2.577421in}}%
\pgfpathlineto{\pgfqpoint{2.385922in}{2.402068in}}%
\pgfpathlineto{\pgfqpoint{2.390337in}{2.339494in}}%
\pgfpathlineto{\pgfqpoint{2.394751in}{1.947986in}}%
\pgfpathlineto{\pgfqpoint{2.399166in}{2.137942in}}%
\pgfpathlineto{\pgfqpoint{2.403580in}{2.464337in}}%
\pgfpathlineto{\pgfqpoint{2.407994in}{2.238951in}}%
\pgfpathlineto{\pgfqpoint{2.412409in}{2.435827in}}%
\pgfpathlineto{\pgfqpoint{2.416823in}{2.457980in}}%
\pgfpathlineto{\pgfqpoint{2.421238in}{2.134307in}}%
\pgfpathlineto{\pgfqpoint{2.425652in}{1.939759in}}%
\pgfpathlineto{\pgfqpoint{2.434481in}{2.544014in}}%
\pgfpathlineto{\pgfqpoint{2.438896in}{2.216785in}}%
\pgfpathlineto{\pgfqpoint{2.447724in}{1.777786in}}%
\pgfpathlineto{\pgfqpoint{2.452139in}{2.470952in}}%
\pgfpathlineto{\pgfqpoint{2.456553in}{2.363544in}}%
\pgfpathlineto{\pgfqpoint{2.460968in}{1.925884in}}%
\pgfpathlineto{\pgfqpoint{2.469797in}{2.488800in}}%
\pgfpathlineto{\pgfqpoint{2.478626in}{2.328180in}}%
\pgfpathlineto{\pgfqpoint{2.483040in}{2.089364in}}%
\pgfpathlineto{\pgfqpoint{2.487454in}{2.541318in}}%
\pgfpathlineto{\pgfqpoint{2.491869in}{2.560307in}}%
\pgfpathlineto{\pgfqpoint{2.496283in}{2.176588in}}%
\pgfpathlineto{\pgfqpoint{2.500698in}{2.191722in}}%
\pgfpathlineto{\pgfqpoint{2.505112in}{2.101513in}}%
\pgfpathlineto{\pgfqpoint{2.509527in}{2.496403in}}%
\pgfpathlineto{\pgfqpoint{2.513941in}{2.287471in}}%
\pgfpathlineto{\pgfqpoint{2.518356in}{2.526861in}}%
\pgfpathlineto{\pgfqpoint{2.522770in}{2.615797in}}%
\pgfpathlineto{\pgfqpoint{2.527184in}{2.423830in}}%
\pgfpathlineto{\pgfqpoint{2.531599in}{2.422056in}}%
\pgfpathlineto{\pgfqpoint{2.536013in}{2.377273in}}%
\pgfpathlineto{\pgfqpoint{2.540428in}{2.498883in}}%
\pgfpathlineto{\pgfqpoint{2.544842in}{2.424721in}}%
\pgfpathlineto{\pgfqpoint{2.549257in}{2.086681in}}%
\pgfpathlineto{\pgfqpoint{2.553671in}{2.536108in}}%
\pgfpathlineto{\pgfqpoint{2.558086in}{1.464854in}}%
\pgfpathlineto{\pgfqpoint{2.562500in}{2.469732in}}%
\pgfpathlineto{\pgfqpoint{2.566914in}{2.421952in}}%
\pgfpathlineto{\pgfqpoint{2.571329in}{1.949417in}}%
\pgfpathlineto{\pgfqpoint{2.575743in}{2.437894in}}%
\pgfpathlineto{\pgfqpoint{2.580158in}{2.638107in}}%
\pgfpathlineto{\pgfqpoint{2.584572in}{2.465765in}}%
\pgfpathlineto{\pgfqpoint{2.588987in}{2.346849in}}%
\pgfpathlineto{\pgfqpoint{2.593401in}{2.419396in}}%
\pgfpathlineto{\pgfqpoint{2.597816in}{2.676848in}}%
\pgfpathlineto{\pgfqpoint{2.602230in}{2.178519in}}%
\pgfpathlineto{\pgfqpoint{2.606644in}{2.286110in}}%
\pgfpathlineto{\pgfqpoint{2.611059in}{2.109318in}}%
\pgfpathlineto{\pgfqpoint{2.615473in}{2.534311in}}%
\pgfpathlineto{\pgfqpoint{2.624302in}{2.294539in}}%
\pgfpathlineto{\pgfqpoint{2.628717in}{1.972320in}}%
\pgfpathlineto{\pgfqpoint{2.633131in}{2.599939in}}%
\pgfpathlineto{\pgfqpoint{2.641960in}{2.431154in}}%
\pgfpathlineto{\pgfqpoint{2.646374in}{2.481037in}}%
\pgfpathlineto{\pgfqpoint{2.650789in}{2.430620in}}%
\pgfpathlineto{\pgfqpoint{2.655203in}{2.352317in}}%
\pgfpathlineto{\pgfqpoint{2.659618in}{2.562095in}}%
\pgfpathlineto{\pgfqpoint{2.664032in}{2.424851in}}%
\pgfpathlineto{\pgfqpoint{2.668447in}{2.063972in}}%
\pgfpathlineto{\pgfqpoint{2.672861in}{2.459166in}}%
\pgfpathlineto{\pgfqpoint{2.677276in}{2.201389in}}%
\pgfpathlineto{\pgfqpoint{2.681690in}{2.100492in}}%
\pgfpathlineto{\pgfqpoint{2.686104in}{2.135243in}}%
\pgfpathlineto{\pgfqpoint{2.690519in}{2.563920in}}%
\pgfpathlineto{\pgfqpoint{2.694933in}{2.495677in}}%
\pgfpathlineto{\pgfqpoint{2.699348in}{2.577666in}}%
\pgfpathlineto{\pgfqpoint{2.703762in}{2.537545in}}%
\pgfpathlineto{\pgfqpoint{2.708177in}{2.534629in}}%
\pgfpathlineto{\pgfqpoint{2.712591in}{2.159316in}}%
\pgfpathlineto{\pgfqpoint{2.717006in}{2.197666in}}%
\pgfpathlineto{\pgfqpoint{2.721420in}{2.193876in}}%
\pgfpathlineto{\pgfqpoint{2.725834in}{2.158087in}}%
\pgfpathlineto{\pgfqpoint{2.730249in}{2.506741in}}%
\pgfpathlineto{\pgfqpoint{2.734663in}{2.356833in}}%
\pgfpathlineto{\pgfqpoint{2.739078in}{1.990565in}}%
\pgfpathlineto{\pgfqpoint{2.743492in}{2.592857in}}%
\pgfpathlineto{\pgfqpoint{2.747907in}{2.451004in}}%
\pgfpathlineto{\pgfqpoint{2.752321in}{2.369431in}}%
\pgfpathlineto{\pgfqpoint{2.756736in}{2.398624in}}%
\pgfpathlineto{\pgfqpoint{2.761150in}{2.481735in}}%
\pgfpathlineto{\pgfqpoint{2.765564in}{2.308631in}}%
\pgfpathlineto{\pgfqpoint{2.769979in}{2.579361in}}%
\pgfpathlineto{\pgfqpoint{2.774393in}{1.865275in}}%
\pgfpathlineto{\pgfqpoint{2.778808in}{2.315083in}}%
\pgfpathlineto{\pgfqpoint{2.783222in}{2.436721in}}%
\pgfpathlineto{\pgfqpoint{2.787637in}{2.646145in}}%
\pgfpathlineto{\pgfqpoint{2.792051in}{1.735011in}}%
\pgfpathlineto{\pgfqpoint{2.796466in}{1.667330in}}%
\pgfpathlineto{\pgfqpoint{2.800880in}{2.501919in}}%
\pgfpathlineto{\pgfqpoint{2.805294in}{2.593318in}}%
\pgfpathlineto{\pgfqpoint{2.809709in}{2.336075in}}%
\pgfpathlineto{\pgfqpoint{2.814123in}{2.476618in}}%
\pgfpathlineto{\pgfqpoint{2.818538in}{2.535593in}}%
\pgfpathlineto{\pgfqpoint{2.822952in}{2.655176in}}%
\pgfpathlineto{\pgfqpoint{2.827367in}{2.316062in}}%
\pgfpathlineto{\pgfqpoint{2.831781in}{2.176742in}}%
\pgfpathlineto{\pgfqpoint{2.836196in}{2.108381in}}%
\pgfpathlineto{\pgfqpoint{2.840610in}{2.148129in}}%
\pgfpathlineto{\pgfqpoint{2.845024in}{2.118078in}}%
\pgfpathlineto{\pgfqpoint{2.853853in}{2.710696in}}%
\pgfpathlineto{\pgfqpoint{2.858268in}{2.466333in}}%
\pgfpathlineto{\pgfqpoint{2.862682in}{2.299161in}}%
\pgfpathlineto{\pgfqpoint{2.867097in}{2.506536in}}%
\pgfpathlineto{\pgfqpoint{2.871511in}{2.450970in}}%
\pgfpathlineto{\pgfqpoint{2.875926in}{2.548094in}}%
\pgfpathlineto{\pgfqpoint{2.880340in}{2.557203in}}%
\pgfpathlineto{\pgfqpoint{2.884754in}{2.131431in}}%
\pgfpathlineto{\pgfqpoint{2.889169in}{2.258736in}}%
\pgfpathlineto{\pgfqpoint{2.893583in}{2.617731in}}%
\pgfpathlineto{\pgfqpoint{2.897998in}{2.603114in}}%
\pgfpathlineto{\pgfqpoint{2.902412in}{1.680950in}}%
\pgfpathlineto{\pgfqpoint{2.906827in}{2.174617in}}%
\pgfpathlineto{\pgfqpoint{2.911241in}{2.440961in}}%
\pgfpathlineto{\pgfqpoint{2.915656in}{2.541447in}}%
\pgfpathlineto{\pgfqpoint{2.920070in}{2.319492in}}%
\pgfpathlineto{\pgfqpoint{2.928899in}{2.574953in}}%
\pgfpathlineto{\pgfqpoint{2.933313in}{2.616069in}}%
\pgfpathlineto{\pgfqpoint{2.937728in}{2.544655in}}%
\pgfpathlineto{\pgfqpoint{2.946557in}{2.133129in}}%
\pgfpathlineto{\pgfqpoint{2.950971in}{2.086139in}}%
\pgfpathlineto{\pgfqpoint{2.955386in}{2.122524in}}%
\pgfpathlineto{\pgfqpoint{2.964215in}{2.468576in}}%
\pgfpathlineto{\pgfqpoint{2.968629in}{2.539108in}}%
\pgfpathlineto{\pgfqpoint{2.973043in}{2.410438in}}%
\pgfpathlineto{\pgfqpoint{2.977458in}{2.410177in}}%
\pgfpathlineto{\pgfqpoint{2.981872in}{2.347611in}}%
\pgfpathlineto{\pgfqpoint{2.986287in}{2.412165in}}%
\pgfpathlineto{\pgfqpoint{2.990701in}{2.642602in}}%
\pgfpathlineto{\pgfqpoint{2.995116in}{2.180642in}}%
\pgfpathlineto{\pgfqpoint{2.999530in}{2.323541in}}%
\pgfpathlineto{\pgfqpoint{3.003945in}{1.790363in}}%
\pgfpathlineto{\pgfqpoint{3.008359in}{2.190645in}}%
\pgfpathlineto{\pgfqpoint{3.012773in}{1.499771in}}%
\pgfpathlineto{\pgfqpoint{3.017188in}{2.400890in}}%
\pgfpathlineto{\pgfqpoint{3.021602in}{2.541239in}}%
\pgfpathlineto{\pgfqpoint{3.026017in}{2.461905in}}%
\pgfpathlineto{\pgfqpoint{3.030431in}{2.660490in}}%
\pgfpathlineto{\pgfqpoint{3.034846in}{2.520596in}}%
\pgfpathlineto{\pgfqpoint{3.039260in}{2.558429in}}%
\pgfpathlineto{\pgfqpoint{3.043675in}{2.544748in}}%
\pgfpathlineto{\pgfqpoint{3.048089in}{2.578709in}}%
\pgfpathlineto{\pgfqpoint{3.052503in}{2.571579in}}%
\pgfpathlineto{\pgfqpoint{3.056918in}{2.052259in}}%
\pgfpathlineto{\pgfqpoint{3.061332in}{1.767437in}}%
\pgfpathlineto{\pgfqpoint{3.065747in}{2.579710in}}%
\pgfpathlineto{\pgfqpoint{3.070161in}{2.155113in}}%
\pgfpathlineto{\pgfqpoint{3.074576in}{2.445656in}}%
\pgfpathlineto{\pgfqpoint{3.078990in}{2.369474in}}%
\pgfpathlineto{\pgfqpoint{3.083405in}{2.505734in}}%
\pgfpathlineto{\pgfqpoint{3.087819in}{2.371180in}}%
\pgfpathlineto{\pgfqpoint{3.092233in}{2.420602in}}%
\pgfpathlineto{\pgfqpoint{3.096648in}{2.408996in}}%
\pgfpathlineto{\pgfqpoint{3.101062in}{2.655350in}}%
\pgfpathlineto{\pgfqpoint{3.105477in}{2.047387in}}%
\pgfpathlineto{\pgfqpoint{3.109891in}{2.048717in}}%
\pgfpathlineto{\pgfqpoint{3.114306in}{1.750573in}}%
\pgfpathlineto{\pgfqpoint{3.118720in}{2.010710in}}%
\pgfpathlineto{\pgfqpoint{3.123135in}{1.858583in}}%
\pgfpathlineto{\pgfqpoint{3.127549in}{2.577129in}}%
\pgfpathlineto{\pgfqpoint{3.136378in}{2.393428in}}%
\pgfpathlineto{\pgfqpoint{3.140792in}{2.498624in}}%
\pgfpathlineto{\pgfqpoint{3.145207in}{2.499538in}}%
\pgfpathlineto{\pgfqpoint{3.149621in}{2.257201in}}%
\pgfpathlineto{\pgfqpoint{3.154036in}{2.596523in}}%
\pgfpathlineto{\pgfqpoint{3.162865in}{2.486635in}}%
\pgfpathlineto{\pgfqpoint{3.167279in}{1.688035in}}%
\pgfpathlineto{\pgfqpoint{3.171693in}{1.832264in}}%
\pgfpathlineto{\pgfqpoint{3.176108in}{2.404559in}}%
\pgfpathlineto{\pgfqpoint{3.180522in}{2.298529in}}%
\pgfpathlineto{\pgfqpoint{3.184937in}{2.606631in}}%
\pgfpathlineto{\pgfqpoint{3.189351in}{2.464741in}}%
\pgfpathlineto{\pgfqpoint{3.193766in}{2.579451in}}%
\pgfpathlineto{\pgfqpoint{3.198180in}{2.463999in}}%
\pgfpathlineto{\pgfqpoint{3.202595in}{2.171619in}}%
\pgfpathlineto{\pgfqpoint{3.207009in}{2.511023in}}%
\pgfpathlineto{\pgfqpoint{3.211423in}{2.600010in}}%
\pgfpathlineto{\pgfqpoint{3.215838in}{2.470331in}}%
\pgfpathlineto{\pgfqpoint{3.220252in}{2.218264in}}%
\pgfpathlineto{\pgfqpoint{3.224667in}{1.782906in}}%
\pgfpathlineto{\pgfqpoint{3.229081in}{2.497606in}}%
\pgfpathlineto{\pgfqpoint{3.233496in}{2.492607in}}%
\pgfpathlineto{\pgfqpoint{3.237910in}{2.682223in}}%
\pgfpathlineto{\pgfqpoint{3.242325in}{2.528789in}}%
\pgfpathlineto{\pgfqpoint{3.246739in}{2.496501in}}%
\pgfpathlineto{\pgfqpoint{3.251153in}{2.421010in}}%
\pgfpathlineto{\pgfqpoint{3.255568in}{2.516565in}}%
\pgfpathlineto{\pgfqpoint{3.259982in}{2.116757in}}%
\pgfpathlineto{\pgfqpoint{3.264397in}{2.712394in}}%
\pgfpathlineto{\pgfqpoint{3.268811in}{2.596369in}}%
\pgfpathlineto{\pgfqpoint{3.273226in}{2.629961in}}%
\pgfpathlineto{\pgfqpoint{3.277640in}{2.285359in}}%
\pgfpathlineto{\pgfqpoint{3.282055in}{2.254677in}}%
\pgfpathlineto{\pgfqpoint{3.286469in}{2.302451in}}%
\pgfpathlineto{\pgfqpoint{3.290883in}{1.905269in}}%
\pgfpathlineto{\pgfqpoint{3.295298in}{2.582463in}}%
\pgfpathlineto{\pgfqpoint{3.299712in}{2.453866in}}%
\pgfpathlineto{\pgfqpoint{3.304127in}{2.658125in}}%
\pgfpathlineto{\pgfqpoint{3.308541in}{2.471689in}}%
\pgfpathlineto{\pgfqpoint{3.312956in}{2.150977in}}%
\pgfpathlineto{\pgfqpoint{3.317370in}{2.401230in}}%
\pgfpathlineto{\pgfqpoint{3.326199in}{2.524015in}}%
\pgfpathlineto{\pgfqpoint{3.330613in}{2.310664in}}%
\pgfpathlineto{\pgfqpoint{3.335028in}{2.195841in}}%
\pgfpathlineto{\pgfqpoint{3.339442in}{2.181989in}}%
\pgfpathlineto{\pgfqpoint{3.343857in}{2.201886in}}%
\pgfpathlineto{\pgfqpoint{3.348271in}{2.599127in}}%
\pgfpathlineto{\pgfqpoint{3.352686in}{2.683556in}}%
\pgfpathlineto{\pgfqpoint{3.357100in}{2.554510in}}%
\pgfpathlineto{\pgfqpoint{3.361515in}{2.506946in}}%
\pgfpathlineto{\pgfqpoint{3.365929in}{2.497637in}}%
\pgfpathlineto{\pgfqpoint{3.374758in}{2.319737in}}%
\pgfpathlineto{\pgfqpoint{3.379172in}{2.277529in}}%
\pgfpathlineto{\pgfqpoint{3.383587in}{2.324480in}}%
\pgfpathlineto{\pgfqpoint{3.388001in}{2.241785in}}%
\pgfpathlineto{\pgfqpoint{3.392416in}{1.865781in}}%
\pgfpathlineto{\pgfqpoint{3.396830in}{2.505403in}}%
\pgfpathlineto{\pgfqpoint{3.401245in}{2.086656in}}%
\pgfpathlineto{\pgfqpoint{3.405659in}{2.507351in}}%
\pgfpathlineto{\pgfqpoint{3.410073in}{2.435554in}}%
\pgfpathlineto{\pgfqpoint{3.414488in}{2.640381in}}%
\pgfpathlineto{\pgfqpoint{3.418902in}{2.422298in}}%
\pgfpathlineto{\pgfqpoint{3.423317in}{2.437146in}}%
\pgfpathlineto{\pgfqpoint{3.427731in}{2.502602in}}%
\pgfpathlineto{\pgfqpoint{3.432146in}{2.510975in}}%
\pgfpathlineto{\pgfqpoint{3.436560in}{2.465723in}}%
\pgfpathlineto{\pgfqpoint{3.440975in}{1.881576in}}%
\pgfpathlineto{\pgfqpoint{3.445389in}{2.658218in}}%
\pgfpathlineto{\pgfqpoint{3.449803in}{2.454083in}}%
\pgfpathlineto{\pgfqpoint{3.454218in}{2.037906in}}%
\pgfpathlineto{\pgfqpoint{3.458632in}{2.231290in}}%
\pgfpathlineto{\pgfqpoint{3.463047in}{2.510553in}}%
\pgfpathlineto{\pgfqpoint{3.471876in}{2.666897in}}%
\pgfpathlineto{\pgfqpoint{3.476290in}{2.368020in}}%
\pgfpathlineto{\pgfqpoint{3.480705in}{2.146405in}}%
\pgfpathlineto{\pgfqpoint{3.485119in}{2.349638in}}%
\pgfpathlineto{\pgfqpoint{3.489533in}{2.044426in}}%
\pgfpathlineto{\pgfqpoint{3.493948in}{2.147727in}}%
\pgfpathlineto{\pgfqpoint{3.498362in}{2.061172in}}%
\pgfpathlineto{\pgfqpoint{3.502777in}{2.126724in}}%
\pgfpathlineto{\pgfqpoint{3.507191in}{2.620287in}}%
\pgfpathlineto{\pgfqpoint{3.511606in}{2.087415in}}%
\pgfpathlineto{\pgfqpoint{3.516020in}{2.537171in}}%
\pgfpathlineto{\pgfqpoint{3.520435in}{2.365473in}}%
\pgfpathlineto{\pgfqpoint{3.524849in}{2.521080in}}%
\pgfpathlineto{\pgfqpoint{3.529263in}{2.502169in}}%
\pgfpathlineto{\pgfqpoint{3.533678in}{2.105404in}}%
\pgfpathlineto{\pgfqpoint{3.538092in}{2.671744in}}%
\pgfpathlineto{\pgfqpoint{3.542507in}{2.641135in}}%
\pgfpathlineto{\pgfqpoint{3.546921in}{2.593630in}}%
\pgfpathlineto{\pgfqpoint{3.551336in}{1.695564in}}%
\pgfpathlineto{\pgfqpoint{3.555750in}{2.222737in}}%
\pgfpathlineto{\pgfqpoint{3.560165in}{2.585578in}}%
\pgfpathlineto{\pgfqpoint{3.564579in}{2.385770in}}%
\pgfpathlineto{\pgfqpoint{3.568994in}{2.327550in}}%
\pgfpathlineto{\pgfqpoint{3.573408in}{2.341473in}}%
\pgfpathlineto{\pgfqpoint{3.577822in}{2.587237in}}%
\pgfpathlineto{\pgfqpoint{3.582237in}{2.364784in}}%
\pgfpathlineto{\pgfqpoint{3.586651in}{2.335108in}}%
\pgfpathlineto{\pgfqpoint{3.591066in}{2.134602in}}%
\pgfpathlineto{\pgfqpoint{3.595480in}{2.578518in}}%
\pgfpathlineto{\pgfqpoint{3.599895in}{2.415449in}}%
\pgfpathlineto{\pgfqpoint{3.604309in}{2.630397in}}%
\pgfpathlineto{\pgfqpoint{3.608724in}{2.433884in}}%
\pgfpathlineto{\pgfqpoint{3.613138in}{2.022125in}}%
\pgfpathlineto{\pgfqpoint{3.617552in}{2.526731in}}%
\pgfpathlineto{\pgfqpoint{3.621967in}{1.805348in}}%
\pgfpathlineto{\pgfqpoint{3.626381in}{2.682569in}}%
\pgfpathlineto{\pgfqpoint{3.630796in}{2.432422in}}%
\pgfpathlineto{\pgfqpoint{3.635210in}{2.517436in}}%
\pgfpathlineto{\pgfqpoint{3.639625in}{2.759432in}}%
\pgfpathlineto{\pgfqpoint{3.644039in}{2.102168in}}%
\pgfpathlineto{\pgfqpoint{3.648454in}{2.403896in}}%
\pgfpathlineto{\pgfqpoint{3.652868in}{2.354896in}}%
\pgfpathlineto{\pgfqpoint{3.657282in}{2.346435in}}%
\pgfpathlineto{\pgfqpoint{3.661697in}{2.525255in}}%
\pgfpathlineto{\pgfqpoint{3.666111in}{2.134909in}}%
\pgfpathlineto{\pgfqpoint{3.670526in}{2.324221in}}%
\pgfpathlineto{\pgfqpoint{3.674940in}{2.383861in}}%
\pgfpathlineto{\pgfqpoint{3.679355in}{2.265063in}}%
\pgfpathlineto{\pgfqpoint{3.683769in}{2.492703in}}%
\pgfpathlineto{\pgfqpoint{3.688184in}{2.401402in}}%
\pgfpathlineto{\pgfqpoint{3.692598in}{2.443500in}}%
\pgfpathlineto{\pgfqpoint{3.697012in}{2.730785in}}%
\pgfpathlineto{\pgfqpoint{3.701427in}{2.246073in}}%
\pgfpathlineto{\pgfqpoint{3.705841in}{2.211600in}}%
\pgfpathlineto{\pgfqpoint{3.710256in}{2.164003in}}%
\pgfpathlineto{\pgfqpoint{3.714670in}{2.231622in}}%
\pgfpathlineto{\pgfqpoint{3.719085in}{2.253380in}}%
\pgfpathlineto{\pgfqpoint{3.723499in}{1.881964in}}%
\pgfpathlineto{\pgfqpoint{3.727914in}{2.102373in}}%
\pgfpathlineto{\pgfqpoint{3.732328in}{2.144443in}}%
\pgfpathlineto{\pgfqpoint{3.736742in}{2.478389in}}%
\pgfpathlineto{\pgfqpoint{3.741157in}{2.396603in}}%
\pgfpathlineto{\pgfqpoint{3.745571in}{2.478974in}}%
\pgfpathlineto{\pgfqpoint{3.754400in}{2.325059in}}%
\pgfpathlineto{\pgfqpoint{3.758815in}{2.354010in}}%
\pgfpathlineto{\pgfqpoint{3.763229in}{2.414498in}}%
\pgfpathlineto{\pgfqpoint{3.767644in}{2.082017in}}%
\pgfpathlineto{\pgfqpoint{3.772058in}{2.145328in}}%
\pgfpathlineto{\pgfqpoint{3.776472in}{1.904072in}}%
\pgfpathlineto{\pgfqpoint{3.780887in}{2.122979in}}%
\pgfpathlineto{\pgfqpoint{3.785301in}{2.456737in}}%
\pgfpathlineto{\pgfqpoint{3.789716in}{2.478664in}}%
\pgfpathlineto{\pgfqpoint{3.794130in}{2.611214in}}%
\pgfpathlineto{\pgfqpoint{3.802959in}{2.301279in}}%
\pgfpathlineto{\pgfqpoint{3.807374in}{2.569824in}}%
\pgfpathlineto{\pgfqpoint{3.811788in}{2.035842in}}%
\pgfpathlineto{\pgfqpoint{3.816202in}{2.296994in}}%
\pgfpathlineto{\pgfqpoint{3.820617in}{1.908924in}}%
\pgfpathlineto{\pgfqpoint{3.825031in}{2.199502in}}%
\pgfpathlineto{\pgfqpoint{3.829446in}{2.150139in}}%
\pgfpathlineto{\pgfqpoint{3.833860in}{2.205353in}}%
\pgfpathlineto{\pgfqpoint{3.842689in}{2.073655in}}%
\pgfpathlineto{\pgfqpoint{3.847104in}{2.374436in}}%
\pgfpathlineto{\pgfqpoint{3.855932in}{2.501972in}}%
\pgfpathlineto{\pgfqpoint{3.860347in}{2.315941in}}%
\pgfpathlineto{\pgfqpoint{3.864761in}{2.389717in}}%
\pgfpathlineto{\pgfqpoint{3.869176in}{2.564741in}}%
\pgfpathlineto{\pgfqpoint{3.873590in}{2.603142in}}%
\pgfpathlineto{\pgfqpoint{3.878005in}{2.480394in}}%
\pgfpathlineto{\pgfqpoint{3.882419in}{2.018576in}}%
\pgfpathlineto{\pgfqpoint{3.886834in}{2.430994in}}%
\pgfpathlineto{\pgfqpoint{3.891248in}{2.306992in}}%
\pgfpathlineto{\pgfqpoint{3.895662in}{2.426172in}}%
\pgfpathlineto{\pgfqpoint{3.900077in}{2.511892in}}%
\pgfpathlineto{\pgfqpoint{3.904491in}{2.549938in}}%
\pgfpathlineto{\pgfqpoint{3.908906in}{2.418617in}}%
\pgfpathlineto{\pgfqpoint{3.913320in}{2.016892in}}%
\pgfpathlineto{\pgfqpoint{3.917735in}{2.305777in}}%
\pgfpathlineto{\pgfqpoint{3.922149in}{2.165057in}}%
\pgfpathlineto{\pgfqpoint{3.926564in}{2.816089in}}%
\pgfpathlineto{\pgfqpoint{3.930978in}{2.242857in}}%
\pgfpathlineto{\pgfqpoint{3.935392in}{2.585673in}}%
\pgfpathlineto{\pgfqpoint{3.939807in}{2.385511in}}%
\pgfpathlineto{\pgfqpoint{3.944221in}{1.894844in}}%
\pgfpathlineto{\pgfqpoint{3.948636in}{2.390080in}}%
\pgfpathlineto{\pgfqpoint{3.953050in}{1.995111in}}%
\pgfpathlineto{\pgfqpoint{3.957465in}{2.471042in}}%
\pgfpathlineto{\pgfqpoint{3.961879in}{2.394542in}}%
\pgfpathlineto{\pgfqpoint{3.966294in}{2.376590in}}%
\pgfpathlineto{\pgfqpoint{3.970708in}{2.582235in}}%
\pgfpathlineto{\pgfqpoint{3.975122in}{2.517819in}}%
\pgfpathlineto{\pgfqpoint{3.979537in}{2.561783in}}%
\pgfpathlineto{\pgfqpoint{3.983951in}{2.396662in}}%
\pgfpathlineto{\pgfqpoint{3.988366in}{1.797282in}}%
\pgfpathlineto{\pgfqpoint{4.001609in}{2.464455in}}%
\pgfpathlineto{\pgfqpoint{4.006024in}{2.416045in}}%
\pgfpathlineto{\pgfqpoint{4.014852in}{2.710809in}}%
\pgfpathlineto{\pgfqpoint{4.019267in}{2.404770in}}%
\pgfpathlineto{\pgfqpoint{4.023681in}{2.236024in}}%
\pgfpathlineto{\pgfqpoint{4.028096in}{2.607525in}}%
\pgfpathlineto{\pgfqpoint{4.032510in}{2.339974in}}%
\pgfpathlineto{\pgfqpoint{4.036925in}{2.779842in}}%
\pgfpathlineto{\pgfqpoint{4.041339in}{2.535321in}}%
\pgfpathlineto{\pgfqpoint{4.050168in}{1.953322in}}%
\pgfpathlineto{\pgfqpoint{4.054582in}{2.192923in}}%
\pgfpathlineto{\pgfqpoint{4.058997in}{2.215745in}}%
\pgfpathlineto{\pgfqpoint{4.063411in}{2.490667in}}%
\pgfpathlineto{\pgfqpoint{4.067826in}{2.527001in}}%
\pgfpathlineto{\pgfqpoint{4.072240in}{2.349464in}}%
\pgfpathlineto{\pgfqpoint{4.076655in}{2.267691in}}%
\pgfpathlineto{\pgfqpoint{4.081069in}{2.227584in}}%
\pgfpathlineto{\pgfqpoint{4.085484in}{2.483399in}}%
\pgfpathlineto{\pgfqpoint{4.089898in}{2.535979in}}%
\pgfpathlineto{\pgfqpoint{4.094312in}{2.450976in}}%
\pgfpathlineto{\pgfqpoint{4.098727in}{1.933472in}}%
\pgfpathlineto{\pgfqpoint{4.103141in}{2.038370in}}%
\pgfpathlineto{\pgfqpoint{4.107556in}{1.965986in}}%
\pgfpathlineto{\pgfqpoint{4.111970in}{2.541695in}}%
\pgfpathlineto{\pgfqpoint{4.116385in}{2.464502in}}%
\pgfpathlineto{\pgfqpoint{4.120799in}{2.048441in}}%
\pgfpathlineto{\pgfqpoint{4.125214in}{2.208687in}}%
\pgfpathlineto{\pgfqpoint{4.129628in}{2.570665in}}%
\pgfpathlineto{\pgfqpoint{4.134042in}{2.550391in}}%
\pgfpathlineto{\pgfqpoint{4.138457in}{2.665868in}}%
\pgfpathlineto{\pgfqpoint{4.142871in}{2.514116in}}%
\pgfpathlineto{\pgfqpoint{4.147286in}{2.310200in}}%
\pgfpathlineto{\pgfqpoint{4.151700in}{1.773069in}}%
\pgfpathlineto{\pgfqpoint{4.156115in}{2.000588in}}%
\pgfpathlineto{\pgfqpoint{4.160529in}{1.895977in}}%
\pgfpathlineto{\pgfqpoint{4.164944in}{2.122687in}}%
\pgfpathlineto{\pgfqpoint{4.169358in}{1.916083in}}%
\pgfpathlineto{\pgfqpoint{4.173772in}{2.063227in}}%
\pgfpathlineto{\pgfqpoint{4.178187in}{2.535832in}}%
\pgfpathlineto{\pgfqpoint{4.191430in}{2.192459in}}%
\pgfpathlineto{\pgfqpoint{4.195845in}{2.519739in}}%
\pgfpathlineto{\pgfqpoint{4.200259in}{2.641939in}}%
\pgfpathlineto{\pgfqpoint{4.204674in}{2.667805in}}%
\pgfpathlineto{\pgfqpoint{4.209088in}{2.649516in}}%
\pgfpathlineto{\pgfqpoint{4.213503in}{1.600685in}}%
\pgfpathlineto{\pgfqpoint{4.222331in}{2.504073in}}%
\pgfpathlineto{\pgfqpoint{4.226746in}{2.034676in}}%
\pgfpathlineto{\pgfqpoint{4.231160in}{2.133303in}}%
\pgfpathlineto{\pgfqpoint{4.235575in}{2.002846in}}%
\pgfpathlineto{\pgfqpoint{4.239989in}{2.417681in}}%
\pgfpathlineto{\pgfqpoint{4.244404in}{2.350554in}}%
\pgfpathlineto{\pgfqpoint{4.248818in}{2.401919in}}%
\pgfpathlineto{\pgfqpoint{4.257647in}{2.684492in}}%
\pgfpathlineto{\pgfqpoint{4.262061in}{2.066087in}}%
\pgfpathlineto{\pgfqpoint{4.266476in}{2.339820in}}%
\pgfpathlineto{\pgfqpoint{4.270890in}{2.022631in}}%
\pgfpathlineto{\pgfqpoint{4.279719in}{2.410427in}}%
\pgfpathlineto{\pgfqpoint{4.284134in}{2.453248in}}%
\pgfpathlineto{\pgfqpoint{4.288548in}{2.536257in}}%
\pgfpathlineto{\pgfqpoint{4.292963in}{2.376311in}}%
\pgfpathlineto{\pgfqpoint{4.297377in}{2.562433in}}%
\pgfpathlineto{\pgfqpoint{4.301791in}{2.517827in}}%
\pgfpathlineto{\pgfqpoint{4.306206in}{2.281794in}}%
\pgfpathlineto{\pgfqpoint{4.310620in}{2.294657in}}%
\pgfpathlineto{\pgfqpoint{4.315035in}{2.333359in}}%
\pgfpathlineto{\pgfqpoint{4.319449in}{2.450768in}}%
\pgfpathlineto{\pgfqpoint{4.323864in}{2.173081in}}%
\pgfpathlineto{\pgfqpoint{4.323864in}{2.173081in}}%
\pgfusepath{stroke}%
\end{pgfscope}%
\begin{pgfscope}%
\pgfpathrectangle{\pgfqpoint{0.625000in}{0.440000in}}{\pgfqpoint{3.875000in}{3.080000in}} %
\pgfusepath{clip}%
\pgfsetrectcap%
\pgfsetroundjoin%
\pgfsetlinewidth{1.505625pt}%
\definecolor{currentstroke}{rgb}{0.000000,0.000000,0.000000}%
\pgfsetstrokecolor{currentstroke}%
\pgfsetdash{}{0pt}%
\pgfpathmoveto{\pgfqpoint{0.840866in}{3.226175in}}%
\pgfpathlineto{\pgfqpoint{0.885011in}{3.209190in}}%
\pgfpathlineto{\pgfqpoint{0.929155in}{3.221843in}}%
\pgfpathlineto{\pgfqpoint{0.973300in}{3.034470in}}%
\pgfpathlineto{\pgfqpoint{1.017444in}{3.244929in}}%
\pgfpathlineto{\pgfqpoint{1.061589in}{3.229122in}}%
\pgfpathlineto{\pgfqpoint{1.105733in}{3.227083in}}%
\pgfpathlineto{\pgfqpoint{1.149878in}{3.179905in}}%
\pgfpathlineto{\pgfqpoint{1.194022in}{3.051750in}}%
\pgfpathlineto{\pgfqpoint{1.238166in}{3.238687in}}%
\pgfpathlineto{\pgfqpoint{1.282311in}{3.229549in}}%
\pgfpathlineto{\pgfqpoint{1.326455in}{3.218744in}}%
\pgfpathlineto{\pgfqpoint{1.370600in}{3.281836in}}%
\pgfpathlineto{\pgfqpoint{1.414744in}{3.199398in}}%
\pgfpathlineto{\pgfqpoint{1.458889in}{3.267584in}}%
\pgfpathlineto{\pgfqpoint{1.503033in}{3.305479in}}%
\pgfpathlineto{\pgfqpoint{1.547178in}{3.303151in}}%
\pgfpathlineto{\pgfqpoint{1.591322in}{3.292529in}}%
\pgfpathlineto{\pgfqpoint{1.635467in}{3.292374in}}%
\pgfpathlineto{\pgfqpoint{1.679611in}{3.301540in}}%
\pgfpathlineto{\pgfqpoint{1.723755in}{3.308136in}}%
\pgfpathlineto{\pgfqpoint{1.767900in}{3.299631in}}%
\pgfpathlineto{\pgfqpoint{1.812044in}{3.322393in}}%
\pgfpathlineto{\pgfqpoint{1.856189in}{3.314150in}}%
\pgfpathlineto{\pgfqpoint{1.900333in}{3.320993in}}%
\pgfpathlineto{\pgfqpoint{1.944478in}{3.311192in}}%
\pgfpathlineto{\pgfqpoint{1.988622in}{3.302979in}}%
\pgfpathlineto{\pgfqpoint{2.032767in}{3.315516in}}%
\pgfpathlineto{\pgfqpoint{2.076911in}{3.316228in}}%
\pgfpathlineto{\pgfqpoint{2.121055in}{3.323937in}}%
\pgfpathlineto{\pgfqpoint{2.165200in}{3.316323in}}%
\pgfpathlineto{\pgfqpoint{2.209344in}{3.300851in}}%
\pgfpathlineto{\pgfqpoint{2.253489in}{3.318123in}}%
\pgfpathlineto{\pgfqpoint{2.297633in}{3.316922in}}%
\pgfpathlineto{\pgfqpoint{2.341778in}{3.324404in}}%
\pgfpathlineto{\pgfqpoint{2.385922in}{3.320240in}}%
\pgfpathlineto{\pgfqpoint{2.430067in}{3.292177in}}%
\pgfpathlineto{\pgfqpoint{2.474211in}{3.317437in}}%
\pgfpathlineto{\pgfqpoint{2.518356in}{3.313928in}}%
\pgfpathlineto{\pgfqpoint{2.562500in}{3.324879in}}%
\pgfpathlineto{\pgfqpoint{2.606644in}{3.306595in}}%
\pgfpathlineto{\pgfqpoint{2.650789in}{3.314262in}}%
\pgfpathlineto{\pgfqpoint{2.694933in}{3.307728in}}%
\pgfpathlineto{\pgfqpoint{2.739078in}{3.321392in}}%
\pgfpathlineto{\pgfqpoint{2.783222in}{3.311771in}}%
\pgfpathlineto{\pgfqpoint{2.827367in}{3.302960in}}%
\pgfpathlineto{\pgfqpoint{2.871511in}{3.313725in}}%
\pgfpathlineto{\pgfqpoint{2.915656in}{3.312632in}}%
\pgfpathlineto{\pgfqpoint{2.959800in}{3.320023in}}%
\pgfpathlineto{\pgfqpoint{3.003945in}{3.312319in}}%
\pgfpathlineto{\pgfqpoint{3.048089in}{3.303618in}}%
\pgfpathlineto{\pgfqpoint{3.092233in}{3.314268in}}%
\pgfpathlineto{\pgfqpoint{3.136378in}{3.313514in}}%
\pgfpathlineto{\pgfqpoint{3.180522in}{3.319104in}}%
\pgfpathlineto{\pgfqpoint{3.224667in}{3.308324in}}%
\pgfpathlineto{\pgfqpoint{3.268811in}{3.306218in}}%
\pgfpathlineto{\pgfqpoint{3.312956in}{3.310753in}}%
\pgfpathlineto{\pgfqpoint{3.357100in}{3.310374in}}%
\pgfpathlineto{\pgfqpoint{3.401245in}{3.318446in}}%
\pgfpathlineto{\pgfqpoint{3.445389in}{3.309932in}}%
\pgfpathlineto{\pgfqpoint{3.489533in}{3.311667in}}%
\pgfpathlineto{\pgfqpoint{3.533678in}{3.319765in}}%
\pgfpathlineto{\pgfqpoint{3.577822in}{3.307141in}}%
\pgfpathlineto{\pgfqpoint{3.621967in}{3.319607in}}%
\pgfpathlineto{\pgfqpoint{3.666111in}{3.296220in}}%
\pgfpathlineto{\pgfqpoint{3.710256in}{3.315002in}}%
\pgfpathlineto{\pgfqpoint{3.754400in}{3.318806in}}%
\pgfpathlineto{\pgfqpoint{3.798545in}{3.306618in}}%
\pgfpathlineto{\pgfqpoint{3.842689in}{3.318432in}}%
\pgfpathlineto{\pgfqpoint{3.886834in}{3.298180in}}%
\pgfpathlineto{\pgfqpoint{3.930978in}{3.317110in}}%
\pgfpathlineto{\pgfqpoint{3.975122in}{3.317341in}}%
\pgfpathlineto{\pgfqpoint{4.019267in}{3.305026in}}%
\pgfpathlineto{\pgfqpoint{4.063411in}{3.318109in}}%
\pgfpathlineto{\pgfqpoint{4.107556in}{3.304129in}}%
\pgfpathlineto{\pgfqpoint{4.151700in}{3.315747in}}%
\pgfpathlineto{\pgfqpoint{4.195845in}{3.317746in}}%
\pgfpathlineto{\pgfqpoint{4.239989in}{3.302769in}}%
\pgfpathlineto{\pgfqpoint{4.284134in}{3.319219in}}%
\pgfusepath{stroke}%
\end{pgfscope}%
\begin{pgfscope}%
\pgfpathrectangle{\pgfqpoint{0.625000in}{0.440000in}}{\pgfqpoint{3.875000in}{3.080000in}} %
\pgfusepath{clip}%
\pgfsetrectcap%
\pgfsetroundjoin%
\pgfsetlinewidth{1.505625pt}%
\definecolor{currentstroke}{rgb}{1.000000,0.000000,0.000000}%
\pgfsetstrokecolor{currentstroke}%
\pgfsetdash{}{0pt}%
\pgfpathmoveto{\pgfqpoint{0.840866in}{1.570637in}}%
\pgfpathlineto{\pgfqpoint{0.885011in}{1.642577in}}%
\pgfpathlineto{\pgfqpoint{0.929155in}{1.528573in}}%
\pgfpathlineto{\pgfqpoint{0.973300in}{1.460665in}}%
\pgfpathlineto{\pgfqpoint{1.017444in}{1.487333in}}%
\pgfpathlineto{\pgfqpoint{1.061589in}{1.492284in}}%
\pgfpathlineto{\pgfqpoint{1.105733in}{1.507455in}}%
\pgfpathlineto{\pgfqpoint{1.149878in}{1.657386in}}%
\pgfpathlineto{\pgfqpoint{1.194022in}{1.528269in}}%
\pgfpathlineto{\pgfqpoint{1.238166in}{1.547616in}}%
\pgfpathlineto{\pgfqpoint{1.282311in}{1.666124in}}%
\pgfpathlineto{\pgfqpoint{1.326455in}{1.585915in}}%
\pgfpathlineto{\pgfqpoint{1.370600in}{1.839703in}}%
\pgfpathlineto{\pgfqpoint{1.414744in}{1.659607in}}%
\pgfpathlineto{\pgfqpoint{1.458889in}{1.722429in}}%
\pgfpathlineto{\pgfqpoint{1.503033in}{2.044946in}}%
\pgfpathlineto{\pgfqpoint{1.547178in}{2.001100in}}%
\pgfpathlineto{\pgfqpoint{1.591322in}{1.873352in}}%
\pgfpathlineto{\pgfqpoint{1.635467in}{1.868730in}}%
\pgfpathlineto{\pgfqpoint{1.679611in}{1.924565in}}%
\pgfpathlineto{\pgfqpoint{1.723755in}{2.110636in}}%
\pgfpathlineto{\pgfqpoint{1.767900in}{1.942337in}}%
\pgfpathlineto{\pgfqpoint{1.812044in}{1.966593in}}%
\pgfpathlineto{\pgfqpoint{1.856189in}{2.019667in}}%
\pgfpathlineto{\pgfqpoint{1.900333in}{2.064349in}}%
\pgfpathlineto{\pgfqpoint{1.944478in}{2.088146in}}%
\pgfpathlineto{\pgfqpoint{1.988622in}{1.917556in}}%
\pgfpathlineto{\pgfqpoint{2.032767in}{1.852471in}}%
\pgfpathlineto{\pgfqpoint{2.076911in}{2.053066in}}%
\pgfpathlineto{\pgfqpoint{2.121055in}{2.102277in}}%
\pgfpathlineto{\pgfqpoint{2.165200in}{2.144997in}}%
\pgfpathlineto{\pgfqpoint{2.209344in}{1.909537in}}%
\pgfpathlineto{\pgfqpoint{2.253489in}{1.850269in}}%
\pgfpathlineto{\pgfqpoint{2.297633in}{2.031678in}}%
\pgfpathlineto{\pgfqpoint{2.341778in}{2.102972in}}%
\pgfpathlineto{\pgfqpoint{2.385922in}{2.183726in}}%
\pgfpathlineto{\pgfqpoint{2.430067in}{1.812968in}}%
\pgfpathlineto{\pgfqpoint{2.474211in}{1.820036in}}%
\pgfpathlineto{\pgfqpoint{2.518356in}{2.035865in}}%
\pgfpathlineto{\pgfqpoint{2.562500in}{2.110827in}}%
\pgfpathlineto{\pgfqpoint{2.606644in}{2.101175in}}%
\pgfpathlineto{\pgfqpoint{2.650789in}{1.936751in}}%
\pgfpathlineto{\pgfqpoint{2.694933in}{1.898066in}}%
\pgfpathlineto{\pgfqpoint{2.739078in}{2.071631in}}%
\pgfpathlineto{\pgfqpoint{2.783222in}{1.998502in}}%
\pgfpathlineto{\pgfqpoint{2.827367in}{2.056538in}}%
\pgfpathlineto{\pgfqpoint{2.871511in}{1.930059in}}%
\pgfpathlineto{\pgfqpoint{2.915656in}{1.946951in}}%
\pgfpathlineto{\pgfqpoint{2.959800in}{2.077429in}}%
\pgfpathlineto{\pgfqpoint{3.003945in}{2.007187in}}%
\pgfpathlineto{\pgfqpoint{3.048089in}{2.068313in}}%
\pgfpathlineto{\pgfqpoint{3.092233in}{1.933742in}}%
\pgfpathlineto{\pgfqpoint{3.136378in}{1.955439in}}%
\pgfpathlineto{\pgfqpoint{3.180522in}{2.064509in}}%
\pgfpathlineto{\pgfqpoint{3.224667in}{2.001819in}}%
\pgfpathlineto{\pgfqpoint{3.268811in}{2.059246in}}%
\pgfpathlineto{\pgfqpoint{3.312956in}{1.942779in}}%
\pgfpathlineto{\pgfqpoint{3.357100in}{1.936804in}}%
\pgfpathlineto{\pgfqpoint{3.401245in}{2.066432in}}%
\pgfpathlineto{\pgfqpoint{3.445389in}{2.015706in}}%
\pgfpathlineto{\pgfqpoint{3.489533in}{2.130475in}}%
\pgfpathlineto{\pgfqpoint{3.533678in}{2.022077in}}%
\pgfpathlineto{\pgfqpoint{3.577822in}{1.938511in}}%
\pgfpathlineto{\pgfqpoint{3.621967in}{1.982830in}}%
\pgfpathlineto{\pgfqpoint{3.666111in}{1.961698in}}%
\pgfpathlineto{\pgfqpoint{3.710256in}{2.125099in}}%
\pgfpathlineto{\pgfqpoint{3.754400in}{2.000627in}}%
\pgfpathlineto{\pgfqpoint{3.798545in}{1.943119in}}%
\pgfpathlineto{\pgfqpoint{3.842689in}{1.963014in}}%
\pgfpathlineto{\pgfqpoint{3.886834in}{1.961830in}}%
\pgfpathlineto{\pgfqpoint{3.930978in}{2.127137in}}%
\pgfpathlineto{\pgfqpoint{3.975122in}{1.987326in}}%
\pgfpathlineto{\pgfqpoint{4.019267in}{1.922625in}}%
\pgfpathlineto{\pgfqpoint{4.063411in}{1.962761in}}%
\pgfpathlineto{\pgfqpoint{4.107556in}{2.008536in}}%
\pgfpathlineto{\pgfqpoint{4.151700in}{2.107633in}}%
\pgfpathlineto{\pgfqpoint{4.195845in}{1.989491in}}%
\pgfpathlineto{\pgfqpoint{4.239989in}{1.906540in}}%
\pgfpathlineto{\pgfqpoint{4.284134in}{1.975562in}}%
\pgfusepath{stroke}%
\end{pgfscope}%
\begin{pgfscope}%
\pgfpathrectangle{\pgfqpoint{0.625000in}{0.440000in}}{\pgfqpoint{3.875000in}{3.080000in}} %
\pgfusepath{clip}%
\pgfsetrectcap%
\pgfsetroundjoin%
\pgfsetlinewidth{1.505625pt}%
\definecolor{currentstroke}{rgb}{0.000000,0.501961,0.000000}%
\pgfsetstrokecolor{currentstroke}%
\pgfsetdash{}{0pt}%
\pgfpathmoveto{\pgfqpoint{0.840866in}{2.479331in}}%
\pgfpathlineto{\pgfqpoint{0.885011in}{2.663861in}}%
\pgfpathlineto{\pgfqpoint{0.929155in}{2.721813in}}%
\pgfpathlineto{\pgfqpoint{0.973300in}{2.785167in}}%
\pgfpathlineto{\pgfqpoint{1.017444in}{2.763695in}}%
\pgfpathlineto{\pgfqpoint{1.061589in}{2.812464in}}%
\pgfpathlineto{\pgfqpoint{1.105733in}{2.741798in}}%
\pgfpathlineto{\pgfqpoint{1.149878in}{2.817815in}}%
\pgfpathlineto{\pgfqpoint{1.194022in}{2.815192in}}%
\pgfpathlineto{\pgfqpoint{1.238166in}{2.809313in}}%
\pgfpathlineto{\pgfqpoint{1.282311in}{2.823666in}}%
\pgfpathlineto{\pgfqpoint{1.326455in}{2.775174in}}%
\pgfpathlineto{\pgfqpoint{1.370600in}{2.854006in}}%
\pgfpathlineto{\pgfqpoint{1.414744in}{2.838846in}}%
\pgfpathlineto{\pgfqpoint{1.458889in}{2.791448in}}%
\pgfpathlineto{\pgfqpoint{1.503033in}{2.783336in}}%
\pgfpathlineto{\pgfqpoint{1.547178in}{2.836214in}}%
\pgfpathlineto{\pgfqpoint{1.591322in}{2.787377in}}%
\pgfpathlineto{\pgfqpoint{1.635467in}{2.844710in}}%
\pgfpathlineto{\pgfqpoint{1.679611in}{2.818566in}}%
\pgfpathlineto{\pgfqpoint{1.723755in}{2.858726in}}%
\pgfpathlineto{\pgfqpoint{1.767900in}{2.838910in}}%
\pgfpathlineto{\pgfqpoint{1.812044in}{2.784756in}}%
\pgfpathlineto{\pgfqpoint{1.856189in}{2.853564in}}%
\pgfpathlineto{\pgfqpoint{1.900333in}{2.831369in}}%
\pgfpathlineto{\pgfqpoint{1.944478in}{2.874092in}}%
\pgfpathlineto{\pgfqpoint{1.988622in}{2.831223in}}%
\pgfpathlineto{\pgfqpoint{2.032767in}{2.766352in}}%
\pgfpathlineto{\pgfqpoint{2.076911in}{2.856398in}}%
\pgfpathlineto{\pgfqpoint{2.121055in}{2.838145in}}%
\pgfpathlineto{\pgfqpoint{2.165200in}{2.871328in}}%
\pgfpathlineto{\pgfqpoint{2.209344in}{2.821554in}}%
\pgfpathlineto{\pgfqpoint{2.253489in}{2.752538in}}%
\pgfpathlineto{\pgfqpoint{2.297633in}{2.856823in}}%
\pgfpathlineto{\pgfqpoint{2.341778in}{2.842770in}}%
\pgfpathlineto{\pgfqpoint{2.385922in}{2.880089in}}%
\pgfpathlineto{\pgfqpoint{2.430067in}{2.792769in}}%
\pgfpathlineto{\pgfqpoint{2.474211in}{2.730031in}}%
\pgfpathlineto{\pgfqpoint{2.518356in}{2.852054in}}%
\pgfpathlineto{\pgfqpoint{2.562500in}{2.855909in}}%
\pgfpathlineto{\pgfqpoint{2.606644in}{2.900436in}}%
\pgfpathlineto{\pgfqpoint{2.650789in}{2.797754in}}%
\pgfpathlineto{\pgfqpoint{2.694933in}{2.858965in}}%
\pgfpathlineto{\pgfqpoint{2.739078in}{2.850876in}}%
\pgfpathlineto{\pgfqpoint{2.783222in}{2.857416in}}%
\pgfpathlineto{\pgfqpoint{2.827367in}{2.902179in}}%
\pgfpathlineto{\pgfqpoint{2.871511in}{2.779605in}}%
\pgfpathlineto{\pgfqpoint{2.915656in}{2.857498in}}%
\pgfpathlineto{\pgfqpoint{2.959800in}{2.852934in}}%
\pgfpathlineto{\pgfqpoint{3.003945in}{2.868359in}}%
\pgfpathlineto{\pgfqpoint{3.048089in}{2.904353in}}%
\pgfpathlineto{\pgfqpoint{3.092233in}{2.778970in}}%
\pgfpathlineto{\pgfqpoint{3.136378in}{2.860382in}}%
\pgfpathlineto{\pgfqpoint{3.180522in}{2.854152in}}%
\pgfpathlineto{\pgfqpoint{3.224667in}{2.873122in}}%
\pgfpathlineto{\pgfqpoint{3.268811in}{2.903315in}}%
\pgfpathlineto{\pgfqpoint{3.312956in}{2.780696in}}%
\pgfpathlineto{\pgfqpoint{3.357100in}{2.859395in}}%
\pgfpathlineto{\pgfqpoint{3.401245in}{2.845557in}}%
\pgfpathlineto{\pgfqpoint{3.445389in}{2.882408in}}%
\pgfpathlineto{\pgfqpoint{3.489533in}{2.902840in}}%
\pgfpathlineto{\pgfqpoint{3.533678in}{2.799874in}}%
\pgfpathlineto{\pgfqpoint{3.577822in}{2.872675in}}%
\pgfpathlineto{\pgfqpoint{3.621967in}{2.825957in}}%
\pgfpathlineto{\pgfqpoint{3.666111in}{2.873546in}}%
\pgfpathlineto{\pgfqpoint{3.710256in}{2.900515in}}%
\pgfpathlineto{\pgfqpoint{3.754400in}{2.791850in}}%
\pgfpathlineto{\pgfqpoint{3.798545in}{2.875478in}}%
\pgfpathlineto{\pgfqpoint{3.842689in}{2.821349in}}%
\pgfpathlineto{\pgfqpoint{3.886834in}{2.873015in}}%
\pgfpathlineto{\pgfqpoint{3.930978in}{2.897937in}}%
\pgfpathlineto{\pgfqpoint{3.975122in}{2.787503in}}%
\pgfpathlineto{\pgfqpoint{4.019267in}{2.870512in}}%
\pgfpathlineto{\pgfqpoint{4.063411in}{2.820511in}}%
\pgfpathlineto{\pgfqpoint{4.107556in}{2.884416in}}%
\pgfpathlineto{\pgfqpoint{4.151700in}{2.894175in}}%
\pgfpathlineto{\pgfqpoint{4.195845in}{2.787680in}}%
\pgfpathlineto{\pgfqpoint{4.239989in}{2.866635in}}%
\pgfpathlineto{\pgfqpoint{4.284134in}{2.819336in}}%
\pgfusepath{stroke}%
\end{pgfscope}%
\begin{pgfscope}%
\pgfpathrectangle{\pgfqpoint{0.625000in}{0.440000in}}{\pgfqpoint{3.875000in}{3.080000in}} %
\pgfusepath{clip}%
\pgfsetrectcap%
\pgfsetroundjoin%
\pgfsetlinewidth{1.505625pt}%
\definecolor{currentstroke}{rgb}{0.000000,0.000000,1.000000}%
\pgfsetstrokecolor{currentstroke}%
\pgfsetdash{}{0pt}%
\pgfpathmoveto{\pgfqpoint{0.840866in}{1.622702in}}%
\pgfpathlineto{\pgfqpoint{0.885011in}{1.784652in}}%
\pgfpathlineto{\pgfqpoint{0.929155in}{1.938522in}}%
\pgfpathlineto{\pgfqpoint{0.973300in}{1.872959in}}%
\pgfpathlineto{\pgfqpoint{1.017444in}{1.927633in}}%
\pgfpathlineto{\pgfqpoint{1.061589in}{1.966011in}}%
\pgfpathlineto{\pgfqpoint{1.105733in}{1.768278in}}%
\pgfpathlineto{\pgfqpoint{1.149878in}{2.022068in}}%
\pgfpathlineto{\pgfqpoint{1.194022in}{2.030284in}}%
\pgfpathlineto{\pgfqpoint{1.238166in}{2.009801in}}%
\pgfpathlineto{\pgfqpoint{1.282311in}{2.049748in}}%
\pgfpathlineto{\pgfqpoint{1.326455in}{1.904406in}}%
\pgfpathlineto{\pgfqpoint{1.370600in}{2.070754in}}%
\pgfpathlineto{\pgfqpoint{1.414744in}{2.051925in}}%
\pgfpathlineto{\pgfqpoint{1.458889in}{1.991262in}}%
\pgfpathlineto{\pgfqpoint{1.503033in}{2.127984in}}%
\pgfpathlineto{\pgfqpoint{1.547178in}{2.030908in}}%
\pgfpathlineto{\pgfqpoint{1.591322in}{1.924669in}}%
\pgfpathlineto{\pgfqpoint{1.635467in}{2.096604in}}%
\pgfpathlineto{\pgfqpoint{1.679611in}{2.095648in}}%
\pgfpathlineto{\pgfqpoint{1.723755in}{2.178289in}}%
\pgfpathlineto{\pgfqpoint{1.767900in}{1.992623in}}%
\pgfpathlineto{\pgfqpoint{1.812044in}{1.932733in}}%
\pgfpathlineto{\pgfqpoint{1.856189in}{2.167655in}}%
\pgfpathlineto{\pgfqpoint{1.900333in}{2.130045in}}%
\pgfpathlineto{\pgfqpoint{1.944478in}{2.185495in}}%
\pgfpathlineto{\pgfqpoint{1.988622in}{1.980710in}}%
\pgfpathlineto{\pgfqpoint{2.032767in}{1.868584in}}%
\pgfpathlineto{\pgfqpoint{2.076911in}{2.194188in}}%
\pgfpathlineto{\pgfqpoint{2.121055in}{2.161037in}}%
\pgfpathlineto{\pgfqpoint{2.165200in}{2.214654in}}%
\pgfpathlineto{\pgfqpoint{2.209344in}{1.985155in}}%
\pgfpathlineto{\pgfqpoint{2.253489in}{1.862396in}}%
\pgfpathlineto{\pgfqpoint{2.297633in}{2.182059in}}%
\pgfpathlineto{\pgfqpoint{2.341778in}{2.174473in}}%
\pgfpathlineto{\pgfqpoint{2.385922in}{2.241603in}}%
\pgfpathlineto{\pgfqpoint{2.430067in}{1.906852in}}%
\pgfpathlineto{\pgfqpoint{2.474211in}{1.827782in}}%
\pgfpathlineto{\pgfqpoint{2.518356in}{2.195636in}}%
\pgfpathlineto{\pgfqpoint{2.562500in}{2.199055in}}%
\pgfpathlineto{\pgfqpoint{2.606644in}{2.248154in}}%
\pgfpathlineto{\pgfqpoint{2.650789in}{1.971443in}}%
\pgfpathlineto{\pgfqpoint{2.694933in}{2.109306in}}%
\pgfpathlineto{\pgfqpoint{2.739078in}{2.152647in}}%
\pgfpathlineto{\pgfqpoint{2.783222in}{2.162338in}}%
\pgfpathlineto{\pgfqpoint{2.827367in}{2.230398in}}%
\pgfpathlineto{\pgfqpoint{2.871511in}{1.945748in}}%
\pgfpathlineto{\pgfqpoint{2.915656in}{2.132617in}}%
\pgfpathlineto{\pgfqpoint{2.959800in}{2.166036in}}%
\pgfpathlineto{\pgfqpoint{3.003945in}{2.178890in}}%
\pgfpathlineto{\pgfqpoint{3.048089in}{2.242199in}}%
\pgfpathlineto{\pgfqpoint{3.092233in}{1.952102in}}%
\pgfpathlineto{\pgfqpoint{3.136378in}{2.134374in}}%
\pgfpathlineto{\pgfqpoint{3.180522in}{2.162842in}}%
\pgfpathlineto{\pgfqpoint{3.224667in}{2.190578in}}%
\pgfpathlineto{\pgfqpoint{3.268811in}{2.235513in}}%
\pgfpathlineto{\pgfqpoint{3.312956in}{1.969435in}}%
\pgfpathlineto{\pgfqpoint{3.357100in}{2.114415in}}%
\pgfpathlineto{\pgfqpoint{3.401245in}{2.172081in}}%
\pgfpathlineto{\pgfqpoint{3.445389in}{2.192563in}}%
\pgfpathlineto{\pgfqpoint{3.489533in}{2.263013in}}%
\pgfpathlineto{\pgfqpoint{3.533678in}{2.040639in}}%
\pgfpathlineto{\pgfqpoint{3.577822in}{2.144769in}}%
\pgfpathlineto{\pgfqpoint{3.621967in}{2.107406in}}%
\pgfpathlineto{\pgfqpoint{3.666111in}{2.168085in}}%
\pgfpathlineto{\pgfqpoint{3.710256in}{2.247504in}}%
\pgfpathlineto{\pgfqpoint{3.754400in}{2.025754in}}%
\pgfpathlineto{\pgfqpoint{3.798545in}{2.146200in}}%
\pgfpathlineto{\pgfqpoint{3.842689in}{2.098063in}}%
\pgfpathlineto{\pgfqpoint{3.886834in}{2.156710in}}%
\pgfpathlineto{\pgfqpoint{3.930978in}{2.241411in}}%
\pgfpathlineto{\pgfqpoint{3.975122in}{2.017769in}}%
\pgfpathlineto{\pgfqpoint{4.019267in}{2.129339in}}%
\pgfpathlineto{\pgfqpoint{4.063411in}{2.098940in}}%
\pgfpathlineto{\pgfqpoint{4.107556in}{2.182183in}}%
\pgfpathlineto{\pgfqpoint{4.151700in}{2.229682in}}%
\pgfpathlineto{\pgfqpoint{4.195845in}{2.021326in}}%
\pgfpathlineto{\pgfqpoint{4.239989in}{2.109239in}}%
\pgfpathlineto{\pgfqpoint{4.284134in}{2.100984in}}%
\pgfusepath{stroke}%
\end{pgfscope}%
\begin{pgfscope}%
\pgfsetrectcap%
\pgfsetmiterjoin%
\pgfsetlinewidth{0.803000pt}%
\definecolor{currentstroke}{rgb}{0.000000,0.000000,0.000000}%
\pgfsetstrokecolor{currentstroke}%
\pgfsetdash{}{0pt}%
\pgfpathmoveto{\pgfqpoint{0.625000in}{0.440000in}}%
\pgfpathlineto{\pgfqpoint{0.625000in}{3.520000in}}%
\pgfusepath{stroke}%
\end{pgfscope}%
\begin{pgfscope}%
\pgfsetrectcap%
\pgfsetmiterjoin%
\pgfsetlinewidth{0.803000pt}%
\definecolor{currentstroke}{rgb}{0.000000,0.000000,0.000000}%
\pgfsetstrokecolor{currentstroke}%
\pgfsetdash{}{0pt}%
\pgfpathmoveto{\pgfqpoint{4.500000in}{0.440000in}}%
\pgfpathlineto{\pgfqpoint{4.500000in}{3.520000in}}%
\pgfusepath{stroke}%
\end{pgfscope}%
\begin{pgfscope}%
\pgfsetrectcap%
\pgfsetmiterjoin%
\pgfsetlinewidth{0.803000pt}%
\definecolor{currentstroke}{rgb}{0.000000,0.000000,0.000000}%
\pgfsetstrokecolor{currentstroke}%
\pgfsetdash{}{0pt}%
\pgfpathmoveto{\pgfqpoint{0.625000in}{0.440000in}}%
\pgfpathlineto{\pgfqpoint{4.500000in}{0.440000in}}%
\pgfusepath{stroke}%
\end{pgfscope}%
\begin{pgfscope}%
\pgfsetrectcap%
\pgfsetmiterjoin%
\pgfsetlinewidth{0.803000pt}%
\definecolor{currentstroke}{rgb}{0.000000,0.000000,0.000000}%
\pgfsetstrokecolor{currentstroke}%
\pgfsetdash{}{0pt}%
\pgfpathmoveto{\pgfqpoint{0.625000in}{3.520000in}}%
\pgfpathlineto{\pgfqpoint{4.500000in}{3.520000in}}%
\pgfusepath{stroke}%
\end{pgfscope}%
\end{pgfpicture}%
\makeatother%
\endgroup%
}
		\caption{\textbf{Unet\_F1\_4}}
	\end {subfigure}

		\caption[Training progress of the first collection of networks.]{Training progress of the first collection of networks. The training loss is shown in muted colors, while the validation loss is denoted by bright colors. For Unet\_F1, the Multi-class F-Measure score of each class is shown in the respective color. Repetitions in the validation progress in later epochs happen because shuffling was not used.}
		\label{fig:weighted_f1_training}
\end {figure}

\noindent The results of testing the networks on the validation set are shown in Table \textbf{\ref{tab:results1}}. They indicate that both networks perform similarly well, although using a Cross-Entropy loss function beats the F-Measure slightly. The Cross-Entropy networks achieved Macro F-Measure scores of $\approx$\textbf{0.877} for 3 classes and $\approx$\textbf{0.746} for 4 classes. Based on these results, for the following tests, Cross-Entropy was used as the loss function.\\ 

\begin {table}
	\begin{flushleft}
		\begin {tabular}[!ht]{|l|c|c|c|c|}
			\hline\multicolumn{5}{|l|}{\textbf{3-class Micro F-Measure Scores}} \\ \hline
			\textbf{Network}& \textbf{Class 1}& \textbf{Class 2}& \textbf{Class 3}& \textbf{Overall} \\ \hline
			Unet\_Weighted\_3& \cellcolor{green!25}0.936013& \cellcolor{green!25}0.979221& \cellcolor{green!25}0.848447& \cellcolor{green!25}0.959854 \\ \hline
			Unet\_F1\_3& 0.933723& 0.968166& 0.786455&  0.941755\\ \hline
			\multicolumn{5}{|l|}{\textbf{3-class Macro F-Measure Scores}} \\ \hline
			\textbf{Network}& \textbf{Class 1}& \textbf{Class 2}& \textbf{Class 3}& \textbf{Overall} \\ \hline
			Unet\_Weighted\_3& \cellcolor{green!25}0.841432& \cellcolor{green!25}0.977186& \cellcolor{green!25}0.803589& \cellcolor{green!25}0.876880 \\ \hline
			Unet\_F1\_3& 0.837715& 0.964669& 0.761217& 0.858995 \\ \hline
		\end {tabular}
		\vspace{0.5cm}\\
		\begin {tabular}[!ht]{|l|c|c|c|c|c|}
			\hline\multicolumn{6}{|l|}{\textbf{4-class Micro F-Measure Scores}} \\ \hline
			\textbf{Network}& \textbf{Class 1}& \textbf{Class 2}& \textbf{Class 3}& \textbf{Class 4}& \textbf{Overall} \\ \hline
			Unet\_Weighted\_4& \cellcolor{green!25}0.63248& 0.978377& \cellcolor{green!25}0.660174& 0.927164& 0.934388 \\ \hline
			Unet\_F1\_4& 0.632356& \cellcolor{green!25}0.978707& 0.641875& \cellcolor{green!25}0.928546& \cellcolor{green!25}0.935324 \\ \hline
			\multicolumn{6}{|l|}{\textbf{4-class Macro F-Measure Scores}} \\ \hline
			\textbf{Network}& \textbf{Class 1}& \textbf{Class 2}& \textbf{Class 3}& \textbf{Class 4}& \textbf{Overall} \\ \hline
			Unet\_Weighted\_4& \cellcolor{green!25}0.59455& 0.975988& 0.565911& \cellcolor{green!25}0.827576& \cellcolor{green!25}0.746051 \\ \hline
			Unet\_F1\_4& 0.592614& \cellcolor{green!25}0.976762& \cellcolor{green!25}0.57009& 0.823417& 0.742489 \\ \hline
		\end {tabular}
	\end {flushleft}

\caption[Micro and Macro F-Measure scores for Unet\_Weighted and Unet\_F1.]{Micro and Macro F-Measure scores achieved by Unet\_Weighted and Unet\_F1 when segmenting the validation set images into 3 and 4 classes. In the 4-class dataset, \textbf{class 1} is the background, \textbf{class 2} is the cell proper, \textbf{class 3} are the Filopodia and \textbf{class 4} are the Lamellopodia, while in the 3-class dataset, \textbf{class 3} represents both Filopodia and Lamellopodia. The best scores in each category, per class, are marked in green.}
\label{tab:results1}
\end {table}

\noindent The second training case tested whether using Batch Normalization provides benefits, either in convergence speed or validation score. Therefore, the \textbf{Unet\_Weighted} networks were modified to perform Batch Normalization before each ReLU activation, implemented in Caffe as a ``Batch Normalization'' layer that normalizes its input according to the mini-batch statistics, followed by a ``Scale'' layer that applies the affine transformation. Also, all Dropout layers were removed, as advised in \cite{batchnorm}.

Again, the network was trained on the 3- and 4-class datasets, this time, as advised in \cite{batchnorm}, with a higher initial learning rate of 0.005, and a faster step learning rate decay that reduces the learning rate by $\zeta = 0.1$ every 7,500 iterations. Momentum was kept at $\gamma = 0.99$. However, the network parameters didn't fit into memory with a mini-batch size of 5 because of the added Batch Normalization layers, and therefore the mini-batch size had to be lowered to 2. To be able to still compare the training to previous results, the number of iterations was raised to 200,000 so that the network was trained for $\approx$30 epochs as before. The results of the training are shown in Table \textbf{\ref{tab:results2}}.


\begin {table}
	\begin{flushleft}
		\begin {tabular}[!ht]{|l|c|c|c|c|}
			\hline\multicolumn{5}{|l|}{\textbf{3-class Micro F-Measure Scores}} \\ \hline
			\textbf{Network}& \textbf{Class 1}& \textbf{Class 2}& \textbf{Class 3}& \textbf{Overall} \\ \hline
			Unet\_Weighted\_3& 0.936013& 0.979221& 0.848447& 0.959854 \\ \hline
			Unet\_Weighted\_Batchnorm\_3& & & & \\ \hline
			\multicolumn{5}{|l|}{\textbf{3-class Macro F-Measure Scores}} \\ \hline
			\textbf{Network}& \textbf{Class 1}& \textbf{Class 2}& \textbf{Class 3}& \textbf{Overall} \\ \hline
			Unet\_Weighted\_3& 0.841432& 0.977186&0.803589& 0.876880 \\ \hline
			Unet\_Weighted\_Batchnorm\_3& & & & \\ \hline
		\end {tabular}
		\vspace{0.5cm}\\
		\begin {tabular}[!ht]{|l|c|c|c|c|c|}
			\hline\multicolumn{6}{|l|}{\textbf{4-class Micro F-Measure Scores}} \\ \hline
			\textbf{Network}& \textbf{Class 1}& \textbf{Class 2}& \textbf{Class 3}& \textbf{Class 4}& \textbf{Overall} \\ \hline
			Unet\_Weighted\_4& 0.63248& 0.978377& 0.660174& 0.927164& 0.934388 \\ \hline
			Unet\_Weighted\_Batchnorm\_4& & & & & \\ \hline
			\multicolumn{6}{|l|}{\textbf{4-class Macro F-Measure Scores}} \\ \hline
			\textbf{Network}& \textbf{Class 1}& \textbf{Class 2}& \textbf{Class 3}& \textbf{Class 4}& \textbf{Overall} \\ \hline
			Unet\_Weighted\_4& 0.59455& 0.975988& 0.565911& 0.827576& 0.746051 \\ \hline
			Unet\_Weighted\_Batchnorm\_4& & & & & \\ \hline
		\end {tabular}
	\end {flushleft}

\caption[Micro and Macro F-Measure scores for a network with and without Batch Normalization.]{Micro and Macro F-Measure scores of 3 and 4-class segmenations using a weighted Cross-Entropy loss with or without batch normalization.}
\label{tab:results2}
\end {table}

\noindent It is evident that \textbf{TODO}\\


\begin {table}
	\centering
	\begin {tabular}[!ht]{|l|c|c|}
		\hline
		\textbf{Activation}& \textbf{3 classes}& \textbf{4 classes}\\ \hline
		ReLU& & \\ \hline
		LReLU& & \\ \hline
		PReLU& & \\ \hline
		ELU& & \\ \hline
	\end {tabular}
\caption[Multi-Class F-Measure scores for networks with different activation functions.]{Multi-class F-Measure scores of 3 and 4-class segmentations for the \textbf{TODO} network, using different activations functions.}
\end {table}

\noindent Then, the effect of the weight initialization on the best network was compared, using the Xavier and the MSRA initializations.

\begin {table}
	\centering
	\begin {tabular}[!ht]{|l|c|c|}
		\hline
		\textbf{Init method}& \textbf{3 classes}& \textbf{4 classes}\\ \hline
		Xavier& & \\ \hline
		MSRA& & \\ \hline
	\end {tabular}
\caption[Multi-Class F-Measure scores for networks with Xavier and MSRA weight initialization.]{Multi-class F-Measure scores of 3 and 4-class segmentations for the \textbf{TODO} network, using \textbf{TODO} activations and either Xavier or MSRA weight initialization.}
\end {table}



\noindent Because Otsu thresholding, K-Means and Gaussian Mixture Models are all unsupervised methods, i.e. they do not depend on ground truth images, the labels they output have no direct relation to the ground truth labels used in the CNN training. Therefore, all combinations of matching the output labels with the ground truth labels are evaluated and for each, a multiclass F-Measure score is calculated. The assignment with the highest score is then assumed to be the correct one, which is then used for the overall evaluation using the Macro F-Measure over all segmentations.\\

\textbf{TODO: Show activation maps of different layers!}\\
\textbf{TODO: Enable shuffling in tests https://valserb.wordpress.com/2016/05/15/hdf5-shuffle-caffe/}\\


\begin {table}
	\centering
	\begin {tabular}[!ht]{|l|c|c|}
		\hline
		\textbf{Method}& \textbf{3 classes}& \textbf{4 classes}\\ \hline
		Otsu& & \\ \hline
		K-Means& & \\ \hline
		GMM& & \\ \hline
		\textbf{TODO-network}& & \\ \hline
	\end {tabular}
\caption[Multi-class F-Measure scores of the best network in comparison to the unsupervised methods.]{Multi-class F-Measure scores of 3 and 4-class segmentations. The best network is compared to the outputs of unsupervised methods.}
\end {table}