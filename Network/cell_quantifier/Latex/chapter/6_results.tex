\chapter {Results}
\label{chap:results}

	\section{Hardware}
All variants of the U-Net were trained on a NVIDIA TITAN Xp GPU (12 GB VRAM) using Caffe's CUDA/cuDNN support.

	\section {Segmentation quality}

To evaluate the quality of the segmentations, the average multi-class F-Measure (see Section \ref{subsec:fmeasure}) of the network outputs for all validation images compared to their ground truth was calculated.\\

\begin {table}
	\centering
	\begin {tabular}[!ht]{|l|c|c|}
		\hline
		\textbf{CNN variant}& \textbf{3 classes}& \textbf{4 classes}\\ \hline
		Unet& & \\ \hline
		Unet\_Batchnorm& & \\ \hline
		Unet\_Weighted& & \\ \hline
		Unet\_Weighted\_Batchnorm& & \\ \hline
		Unet\_F1& & \\ \hline
		Unet\_F1\_Batchnorm & & \\ \hline
	\end {tabular}
\caption[]{Multi-class F-Measure scores of 3 and 4-class CNN segmentations.}
\end {table}

\textbf{TODO: Batchnorm: lower learning rate, higher decay rate, no dropout?}
\noindent Among the CNNs, the best network \textbf{TODO} was selected and its ReLU activation functions were replaced by LReLUs, PReLUs and ELUs to evaluate the differences in performance with respect to the activation function used.

\begin {table}
	\centering
	\begin {tabular}[!ht]{|l|c|c|}
		\hline
		\textbf{Activation}& \textbf{3 classes}& \textbf{4 classes}\\ \hline
		ReLU& & \\ \hline
		LReLU& & \\ \hline
		PReLU& & \\ \hline
		ELU& & \\ \hline
	\end {tabular}
\caption[]{Multi-class F-Measure scores of 3 and 4-class segmentations for the \textbf{TODO} network, using different activations functions.}
\end {table}

\noindent Then, the effect of the weight initialization on the best network was compared, using the Xavier and the MSRA initializations.

\begin {table}
	\centering
	\begin {tabular}[!ht]{|l|c|c|}
		\hline
		\textbf{Init method}& \textbf{3 classes}& \textbf{4 classes}\\ \hline
		Xavier& & \\ \hline
		MSRA& & \\ \hline
	\end {tabular}
\caption[]{Multi-class F-Measure scores of 3 and 4-class segmentations for the \textbf{TODO} network, using \textbf{TODO} activations and either Xavier or MSRA weight initialization.}
\end {table}

\textbf{TODO: Evaluation of training progress, i.e. validation loss progession}


To compare the CNN results with the results of the other methods, the problem of matching the output labels with those of the ground truth data has to be dealt with. Because Otsu thresholding, K-Means and Gaussian Mixture Models are all unsupervised methods, i.e. they do not depend on ground truth images, the labels they output have no direct relation to the ground truth labels used in the CNN training. Therefore, all combinations of matching the output labels with the ground truth labels are evaluated and for each, an F-Measure score is calculated. The assignment with the highest score is then assumed to be the correct one.


\begin {table}
	\centering
	\begin {tabular}[!ht]{|l|c|c|}
		\hline
		\textbf{Method}& \textbf{3 classes}& \textbf{4 classes}\\ \hline
		Otsu& & \\ \hline
		K-Means& & \\ \hline
		GMM& & \\ \hline
		\textbf{TODO-network}& & \\ \hline
	\end {tabular}
\caption[]{Multi-class F-Measure scores of 3 and 4-class segmentations. The best network is compared to the outputs of unsupervised methods.}
\end {table}