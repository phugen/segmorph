\chapter{Basic Concepts}

This section explains an array of concepts that can be used to segment an image into multiple areas. In the ``Results'' chapter (Ch. \ref{chapter_results}), the performance of all the mentioned algorithms is compared, using the same data set each time.
	
	\section{Thresholding}
\textbf{TODO: Mehr über Otsu, Unterschied zwischen Global/Local thresholding, FIGURE}\\
Thresholding is the simplest segmentation algorithm there is: Given an input image with dimensions $x \times y$ and intensity values $z$ (for instance, $[0, 255]$ at 8-bit color depth), defined as a function 

\[I(x, y) \to z, \text{ for } x, y, z \in \mathbb{N}_0,\]

\noindent the thresholding function is defined as follows:

\[ T(I(x, y), \theta) =  \begin{cases}
				1 \text{ if } I(x, y) \, \geq \, \theta \\
			           0 \text{ otherwise}
			     \end{cases}
\]


\noindent This yields a binary segmentation of the image into two classes, given that $\theta$ is chosen properly. The choice of $\theta$ is therefore crucial for the success of the segmentation. One way to find a suitable threshold parameter is (automated) image histogram analysis, such as in Otsu's method\cite{Otsu}.

Otsu's method iterates through all possible values for $\theta$ and calculates the ``between-class variance'' ${\sigma^{2}}_B$ for each $\theta$, defined as

\[ {\sigma^2}_{B} = W_b W_f (\mu_b - \mu_f)^2,\]

\noindent where $W_b$ and $W_f$ are the weights - the sum of pixels in all bins belonging to either the background or foreground class, as determined by $\theta$, divided by the total number of pixels - and $\mu_b$ and $\mu_f$ are the statistical mean values for the background and foreground classes. The threshold with the maximum \textbf{between}-class variance corresponds to a segmentation in which the pixels of each class have minimum \textbf{within}-class variance: Intuitively, this means that the pixels of each class are very much alike.

However, this approach doesn't work as well for images whose histograms are not bimodal at all - for example, due to excessive noise - and also, while possible, doesn't perform well for cases in which a large number of classes is to be segmented. 


	\section{K-Means}
K-Means\cite{kmeans} is a general-purpose data clustering algorithm whose aim is to create $k$ data clusters from all $n$-dimensional data points $d = (f_1, f_2, \dots, f_n)$ so that the squared distance from each data point in the cluster to the cluster mean is minimized overall. Mathematically, this means calculating

\[ \argmin \limits_{C} \sum \limits_{i=1}^{k} \sum \limits_{d \in C_i} || d - \mu_i||^2 ,\]

\noindent where $C_{i \dots k}$ are the $k$ clusters and $\mu_{i \dots k}$ is the mean of the respective cluster.

The algorithm is initialized with $k$ either random or differently selected cluster centers, which are chosen from the given data points. Then, the algorithm executes the following two steps alternatingly until either a set number of iterations is reached or the overall difference between the current and the last iteration falls below a threshold $\epsilon$:

\textbf{TODO: make this a code environment}\\
\begin {enumerate}
	\item Assign each data point to exactly one of the $k$ clusters by selecting the cluster that has the closest mean distance as defined above.
	\item Calculate new cluster centers by recalculating the mean of all elements assigned to each cluster.
\end {enumerate}

	\section{Gaussian Mixture Models}

	\section{Graph Cuts}

	\section{Multi-layer Perceptrons}

		\subsection {Gradient Descent}

		\subsection{Backpropagation}