\chapter{Basic Concepts}
\label{chap:concepts}

Image segmentation is one of the main challenges in image processing and has been an active field of study for years, which results in many segmentation methods being available. In the following pages, a selection of methods that can perform multi-class segmentation, i.e. segmentation with more than two classes, is presented. As the number of images to be segmented is usually high, and computation time can be a crucial factor, human interaction should remain at a minimum level. Therefore, only methods that, once the segmentation process has been started, work without human interaction were chosen. All methods described here are later used to segment a set of images and compared to the ground truth to measure the quality of the segmentations (see Chapter \ref{chap:results}).
	
	\section{Thresholding}
\label{sec:thresholding}

Thresholding is the most basic segmentation algorithm there is: Given an input image with dimensions $x \times y$ and intensity values $z$ (for instance, $[0, 255]$ at 8-bit color depth), defined as a function 

\[I(x, y) \to z, \text{ for } x, y, z \in \mathbb{N}_0,\]

\noindent the thresholding function is defined as follows:

\[ T(I(x, y), \theta) =  \begin{cases}
				1 \text{ if } I(x, y) \, \geq \, \theta \\
			           0 \text{ otherwise}
			     \end{cases}
\]


\noindent This yields a binary segmentation of the image into two classes, given that $\theta$ is chosen properly. The choice of $\theta$ is therefore crucial for the success of the segmentation. One way to find a suitable threshold parameter is (automated) image histogram analysis, such as in \textit{Otsu thresholding}.\\

\noindent The Otsu thresholding algorithm \cite{Otsu} iterates through all possible values for $\theta$ and calculates the ``between-class variance'' ${\sigma^{2}}_B$ for each $\theta$, defined as

\[ {\sigma^2}_{B} = w_0 \, w_1 (\mu_0 - \mu_1)^2,\]

\noindent where $w_0$ and $w_1$ are the weights, defined as

\begin {align}
	w_0 &= \sum \limits_{i=1}^{\theta} p_i \\
	w_1 &= \sum \limits_{i=\theta+1}^{k} p_i
\end {align}

\noindent for a histogram of the image consisting of $k$ grayscale intensity bins which each contain the probability for a pixel being in this bin, $p_i$.\footnote{Here, it is assumed that Otsu's algorithm is applied to a grayscale image. For RGB images, each channel can be thresholded separately using the same algorithm, or the image can be converted to grayscale and then be segmented.} $\mu_0$ and $\mu_1$ are the statistical mean values, defined as

\begin {align}
	\mu_0 &= \sum \limits_{i=1}^{\theta} i \, p_i \\
	\mu_1 &= \sum \limits_{i=\theta + 1}^{k} i \, p_i\\
\end {align}

\noindent The threshold with the maximum \textbf{between}-class variance corresponds to a segmentation in which the pixels of each class have minimum \textbf{within}-class variance: This means that the pixels of each class are very much alike.

\begin {figure}[!ht]
	\begin {subfigure}[t]{0.3\linewidth}
		\scalebox{0.65}{\inputpgf{img}{fig_otsu_histogram.pgf}}
	\end {subfigure}
	\hspace{5.5cm}
	\begin {subfigure}[b]{0.3\linewidth}
		\scalebox{0.65}{\inputpgf{img}{fig_otsu.pgf}}
	\end {subfigure}
		\caption[]{\textbf{Left:} Histogram of an 8-bit grayscale input image. Optimal thresholds found by Otsu's algorithm are marked in orange. \textbf{Right:} Original image (top) and four-way segmentation result.}
		\label{fig:otsu}
	\end {figure}

\noindent To extend the normal Otsu algorithm to $n$-class segmentation, the weight and mean formulas are changed to correspond to $n$ classes defined by a set of thresholds $\theta = \{t_1 \dots t_{n-1}\}$. The between-class variance formula is changed to respect the variances of all classes, so that the best $\theta$ is the one that minimizes the overall within-class variance by maximizing the overall between-class variance:

\[ \sigma_B^2 = \sum \limits_{i=1}^{n} w_i (\mu_i - \mu_t)^2 \]

\noindent Here, $w_i$ is the weight of the $i$th class, $\mu_i$ is the mean of that class and $\mu_t$ is the total mean of all classes, i.e. $\sum_{i=1}^{n} w_i \mu_i$. The thresholds can then be applied using

\[ T(I(x, y), \theta) =  \begin{cases}
				n \text{ if } I(x, y) \, \geq \, t_{n-1} \\
				n-1 \text { if } I(x, y) \, \geq t_{n-2} \land I(x, y) < t_{n-1}\\
				\dots\\
			           0 \text{ if } I(x, y) \, < t_1
			     \end{cases}
\]

\noindent However, this approach doesn't work as well for images whose histograms do not exhibit distinct histogram peaks - for example, due to noise - and also doesn't perform well for cases in which $n$ is large, neither considering the time needed to perform a segmentation\footnote{While variants of Otsu exist that are optimized for speed, such as \cite{otsu_fast}, only the original Otsu algorithm is evaluated here because it is the most common.} nor considering the segmentation quality, due to the method inspecting only the histogram data. \cite{Otsu}\\

%\noindent The Otsu algorithm belongs to a family of thresholding mechanisms that is called ``global thresholding'' because the same threshold is applied to the entire image. However, there are also local or ``adaptive'' thresholding algorithms which segment different parts of the image with different thresholds. The need for such algorithms often arises when the illumination of the input image is highly irregular, for example when cast shadows overlay part of the image (see \ref{fig:illumination}). One approach is to calculate a threshold for each pixel of the image while examining a neighborhood of size $L \times L$ around the pixel in question, using the mean, the mean of the maximum and minimum values or the median of the resulting local pixel intensity distribution to determine a local threshold.

%\textbf{TODO: are there adaptive thresholds for multi-label purposes?}\\
%Local thresholding, however, is much slower than global thresholding because it has to process the neighborhoods of each pixel, while it also depends on choosing $L$ so that the neighborhoods contain enough pixels of either class, or otherwise the thresholds won't be chosen well. \cite[pp.~84--93]{machine_vision}


\begin{comment}
	\section{Canny Edge Detection}
\label{sec:canny}
The \textit{Canny Edge Detector} \cite{canny} is one of the most famous edge-detection algorithms and actually is a composite algorithm that returns a binary segmentation of an image into edges and non-edges.\\

\noindent First, a Gaussian convolution filter is applied to try and subdue noise in the image. A \textit{convolution} is a matrix operation often used in image processing: A matrix $A$, describing the pixel values of an image, is convolved pixel-wise with a convolution matrix $B$ with dimensions $n \times n$ - that is, for each pixel $p$ of $A$, the pixel's $n \times n$-neighborhood pixel values are summed up while being weighted according to corresponding value in $B$. The resulting sum is then assigned to the output matrix in place of the previous value of the center pixel, which intuitively assumes the distance-weighted average value of its neighborhood. In the case of pixels that lie on the edge of the matrix to be convoluted, out-of-bounds considerations have to be made: A constant value such as zero can be assumed for the pixels in the neighborhood that would lie ``outside'' of the image, the existing image values can be mirrored or clamped to provide a torus-like out-of-bounds handling, or the convolution can be done only on those pixels whose neighborhood fully lies inside of the image, resulting in smaller output images.

The aforementioned ``Gaussian filter'' or ``kernel'' is a matrix whose values are defined so that performing a convolution using that matrix approximates the behavior of the two-dimensional Gaussian function with uniform variances for its $x$- and $y$-dimensions:

\[ f(x, y) = \exp \left(- \left( \frac{\left(x - p_x \right)^2}{2\sigma^2} + \frac{ \left(y - p_y \right)^2}{2\sigma^2} \right) \right), \]

\noindent where $p$ is the center pixel of the current neighborhood. Subsequently, $p_x$ and $p_y$ are the coordinates of this pixel within the matrix and $\sigma$, the standard variance, acts as the smoothing constant. The higher this constant is, the stronger the blur effect becomes.

In Gaussian filters, the $\sigma$ constant is expressed through the dimensions of the matrix - the larger the filter matrix, the stronger the blur effect. An example for a $3 \times 3$ Gaussian filter is the following matrix:

\[ \frac{1}{16} \left [ \begin{tabular}{ccc}
				1& 2& 1\\
				2& 4& 2\\
				1& 2& 1 
			   \end{tabular} \right ]\]

\noindent The coefficient $\frac{1}{16}$ is equal to the sum of the matrix values and ensures that the convolution does not change the average image value. \cite[p. 41]{machine_vision}\\

\noindent As the second step, an gradient-based edge detector filter is applied to the smoothed image. The most famous of these is the \textit{Sobel filter} \cite{sobel}, which approximates the local partial derivatives $\frac{\partial I}{\partial x}$ and $\frac{\partial I}{\partial y}$ of each pixel of the image function $I(x, y)$, using the $3 \times 3$ neighborhood of that pixel. It is given by the following matrices \cite[pp. 113 -- 114]{machine_vision}:

\[ \text{Sobel}_x = \left [ \begin{tabular}{ccc}
				-1& 0& 1\\
				-2& 0& 2\\
				-1& 0& 1 
			   \end{tabular} \right ] \text{ and } 
\text{Sobel}_y = \left [ \begin{tabular}{ccc}
				1& 2& 1\\
				0& 0& 0\\
				-1& -2& 1 
			   \end{tabular} \right ] 
\]

\noindent The result of these convolutions are two images, $G_x$ and $G_y$, which represent the partial local derivatives of each pixel. The gradient image of the original input image $I$ is then defined as

\[G = |\nabla I| = \sqrt{{G_x}^2 + {G_y}^2}.\]

\noindent Additionally, the gradient direction of each pixel can be calculated from the derivative images by measuring the angle between the x-axis and the gradient pixel coordinates by employing the atan2 function:

\[G_\phi = \text{atan2}(G_y, G_x) \]

\noindent Using $G_\phi$, edge thinning via non-maximum suppression is applied to the gradient image as the third step in the algorithm: For each pixel, the gradient direction acts as a criterion to decide which two neighboring pixels, that are each on opposite sides (positive and negative direction of the gradient), should be compared to the current pixel. If the value of the current pixel is not larger than the two neighbors' values, the pixel's value is not a local maximum and is set to zero. The gradient direction angles can either be rounded so that each angle represents one of the north-south, west-east directions and so forth, or linear interpolation can be used.

In the final step, a hysteresis threshold is applied. This process consists of defining two thresholds, $\theta_{high}$ and $\theta_{low}$. The definition for the thresholding function as given in \ref{sec:thresholding} is slightly altered:

\[ T(I(x, y), \theta_{high}, \theta_{low}) =  \begin{cases}
							2 \text{ if } I(x, y) \, \geq \, \theta_{high} \\
							1 \text{ if } I(x, y) \, \geq \, \theta_{low} \text{ and } < \theta_{high} \\
			          				0 \text{ otherwise}
			   			        \end{cases}
\]

\noindent Pixels that have a value of $2$ are called strong pixels because they had values larger than the high threshold, whereas pixels with a value of $1$ are called weak pixels. Finally, the algorithm checks for each pixel if an 8-connected path between that pixel and a strong pixel exists - if not, then the pixel is dropped. This can be done with the help of connected component-finding algorithms by dropping each ``$1$''-component which is not connected to at least one ``2''-value.\\

\noindent The thresholds $\theta_{high}$ and $\theta_{low}$ have to be set by the user, although there exists the possibility to set these by using the Otsu threshold described in \ref{sec:thresholding} . Using this combination approach, $\theta_{high}$ is set to the Otsu threshold value for $I(x, y)$, and $\theta_{low}$ is set to $0.5 * \theta_{high}$. \cite{otsu_combine}

\end {comment}

	\section{K-Means}
K-Means \cite{kmeans} is a general-purpose data clustering algorithm whose aim is to create $k$ data clusters from all $n$-dimensional data points $d = (f_1, f_2, \dots, f_n)$ so that the squared distance from each data point in the cluster to the cluster mean is minimized overall. Mathematically, this means calculating

\[ \argmin \limits_{C} \sum \limits_{i=1}^{k} \sum \limits_{d \in C_i} || d - \mu_i||^2 ,\]

\noindent where $C_{i \dots k}$ are the $k$ clusters and $\mu_{i \dots k}$ is the mean of the respective cluster. Viewed graphically, this is equal to computing a higher-order Voronoi diagram for the data, using the $k$ cluster centroids as the Voronoi cell centers.

The algorithm is initialized with $k$ either random or differently selected cluster centers. Then, the algorithm executes the two steps described in \ref{alg:kmeans_pseudo} alternatingly until either a set number of iterations is reached or the overall difference between the current and the last centroid positions falls below a threshold $\epsilon$  (see Figure \ref{fig:kmeans}).

\begin {algorithm}[!ht]
	\caption{K-Means ($\epsilon$, iter\_max)}\label{alg:kmeans_pseudo}
	\begin {algorithmic}[1]
		\State iter = 0
		\State Assign each data point to exactly one of the $k$ clusters by selecting the cluster that has the closest mean distance as defined above.
		\State Calculate new cluster centers by recalculating the mean of all elements assigned to each cluster and calculate difference $\Delta$ compared to previous iteration.
		\If {$\Delta < \epsilon$ \textbf{or} iter == iter\_max}
			\State \textbf{end}
		\Else
			\State iter += 1
			\State \textbf{goto} step 2
		\EndIf
	\end{algorithmic}
\end{algorithm}

\begin {figure}[!ht]
	\begin {center}
		\includegraphics[scale=0.8]{img/fig_kmeans}
	\end{center}
	\caption{Application of the K-Means algorithm to a test image. \textbf{From left to right}: Input image,  K-Means results with $k=4$, results with $k=6$.}
	\label{fig:kmeans}
\end {figure}

\noindent Obviously, this algorithm can also be used to segment images into $k$ different classes, but since K-Means assigns classes to data by using the distance from the mean, the algorithm output favors segmentations in which the classes have roughly the same size, which isn't necessarily the correct way to classify pixels in arbitrary image data.




	\section{Gaussian Mixture Models}
Gaussian Mixtures Models (GMMs) are a subclass of mixture models, that is, probabilistic models which are combined with other models of the same distribution type to form a more complex model that is able to model the distribution of a data set more accurately than a simple model could. In the case of GMMs, the base distributions are often $n$-multivariate Gaussian distributions:

\[ pdf(x) = \mathcal{N}_n (x\,|\,\mu,\, \Sigma) \]

\noindent Here, $pdf(x)$ denotes the probability, or density, of an $n$-dimensional piece of data $x$, while $\mu$ and $\Sigma$ are the $n$-dimensional mean vector and the $n \times n$ covariance matrix of the distribution. Since the shape such a distribution can take is limited, a single Gaussian cannot accurately model a multimodal distribution (see Figure \ref{fig:normal_vs_gmm}). Instead, a weighted combination of multiple Gaussians can be used instead:

\[ pdf_m (x) = \sum \limits_{k=1}^{K} \pi_k \, \mathcal{N}_n (x\,|\, \mu_k, \Sigma_k) \]

\noindent A mixture model consists of $K$ models that each have different parameters $\mu$ and $\Sigma$ and are weighted by weights $\pi_k \in \mathbb{R}$, $1 > \pi_k > 0$, with $\sum_{i=1}^{k} \pi_i = 1$. \cite[pp. 430]{bishop_pattern}

\begin {figure}[!ht]
	\begin{center}
		\includegraphics[scale=0.75]{img/fig_normal_vs_gmm}
	\end{center}
	\caption{Comparison of a single Gaussian fit (upper) with a GMM fit, using $K=3$, on a two-dimensional dataset randomly sampled from three Gaussian distributions. The single Gaussian fails to capture the structure of the data, while the GMM succeeds.}
	\label{fig:normal_vs_gmm}
\end {figure}

\noindent The parameters of such a mixture model can be calculated by applying the Expectation-Maximization \cite{em_algorithm} algorithm, which can additionally be combined with a K-Means initialization to avoid shallow local minima. GMMs can also be used for segmenting images: Images are intepreted as numerical data in which each pixel position corresponds to an intensity value. If the number of components to be found in the image is known beforehand - as it is in the case of cell segmentation - the image can be segmented according to the fitted GMM (see Figure \ref{fig:gmm_vs_gt}). To do this, for each pixel in the image the posterior probabilities of that pixel in respect to each of the Gaussians is calculated and the Gaussian with the highest probability is chosen as the source distribution of the pixel, thus yielding a class with which the pixel is labelled:

\[ label_x = \argmax \limits_{k} \, p(\mu_k, \Sigma_k\, | \, x) \]

\begin {figure}[!ht]
	\centering
	\includegraphics[scale=0.55]{img/fig_gmm_vs_gt}
	\caption{\textbf{Left:} Input image. \textbf{Right:} Image labels in pseudocolor, assigned by a GMM with $K=4$, fitted to the input image.}
	\label{fig:gmm_vs_gt}
\end {figure}


	\section{Multi-layer Perceptrons}
\textit{Multi-layer Perceptrons} (MLPs) or \textit{Feedforward Artificial Neuronal Networks} (ANNs) are mathematical functions that try to emulate the way neurons in the brain process information, typically to classify data or perform regression on it. ANNs have recently achieved impressing results on various tasks, such as image classification, sound analysis and regression, typically achieved by very deep networks that used lots of artificial neurons and were fed large datasets exceeding one million samples. In this section, the concepts behind optimization of functions via Gradient Descent, neuronal networks and training of those networks using the Backpropagation algorithm are explained. 



	\subsection {Gradient Descent}
\label{subsec:grad_desc}

\textit{Gradient Descent} or is a standard, analytic first-order technique that iteratively optimizes at least once-differentiable, continuous functions, that is, that calculates a vector of parameters $\Theta$ for a function $f$ so that

\[ \Theta = \argmin \limits_{i} f(i) \]

\noindent To find values for $\Theta$, the approach takes advantage of the fact that the negative gradient vector $\nabla_f = [ \frac{\partial f}{\partial \Theta_0}, \dots, \frac{\partial f}{\partial \Theta_n} ]$, the vector of all first partial derivatives of $f$, points into the direction of the fastest change, or put graphically, the ``steepest slope''. By taking steps into the direction of this gradient, the function value is minimized step by step, while maximization works the same, only that the function is negated and then minimized, which results in parameters that maximize the original function. The parameter $\eta$ depicts the step size and is typically a value in the range $(0.0, 1.0]$. \cite[pp. 40--42]{optimization_book}

The algorithm starts with a guess for the parameters in $\Theta$ and then iteratively evaluates the gradient and updates the parameters accordingly until convergence within arbitrary precision is reached (see algorithm \ref{alg:grad_desc}).

\begin {algorithm}
	\begin {algorithmic}[1]
		\State $\Theta_0$ = random
		\While {$|f(\Theta_i) - f(\Theta_{i - 1})| > \epsilon$}
			\State $\Theta_i = \Theta_{i-1} - \eta \nabla_f$ 
		\EndWhile
	\end{algorithmic}
	\caption{Gradient Descent scheme for optimizing differentiable functions.}
	\label{alg:grad_desc}
\end{algorithm}

\noindent For convex functions, Gradient Descent always finds the global minimum. For non-convex functions however, the algorithm may get stuck in a local minimum or at saddle points, called ``false minima' (see Figure \ref{fig:grad_desc}). Therefore, running the algorithm multiple times with random values or even informed guesses given knowledge about the form of the function to be optimized is advised. The step size $\eta$ has to be chosen depending on the shape of the function to minimize; if $\eta$ is too large, the algorithm will overshoot the minimum and will never converge even if the minimum exists, and if $\eta$ is too small, convergence will take a long time.

\begin {figure}[!ht]
	\begin{center}
		\includegraphics[scale=0.9]{img/fig_grad_desc}
	\end{center}
	\caption{\textbf{Left:} A contour plot of Gradient Descent minimization of $f(a, b) = \sin(a) + \cos^2(b)$. The blue line shows the search path of the algorithm along the negative gradient while estimating the optimal $\Theta$. \textbf{Right}: Effect of the initial guess for $\Theta$ on the optimization. The function $f(a) = \frac{1}{40}a^2 * \frac{1}{2} \cos(a)$ has multiple local minima in the interval [0, 25]. Only the third guess $a=25$ leads to finding the global minimum, while the other two optimization runs get trapped in local minima.}
	\label{fig:grad_desc}
\end {figure}

\noindent However, a more interesting application of the Gradient Descent algorithm exists in the context of machine learning, where the function to optimize often has fixed parameters in addition to $\Theta$, such as data from a dataset that is related to the function is some way. In such cases, one can consider one of the subtypes of Gradient Descent; either, evaluating the average gradient using all available data samples at once, called \textit{Batch Gradient Descent}, or using single data points only - as an approximation of the entire dataset - to perform the updates. This is called \textit{Stochastic Gradient Descent} (SGD) and is preferable if working with datasets that are too large to fit into memory. Alternatively, one can use the average gradient of a subset of data points - this method is often referred to as \textit{Mini-Batch} Gradient Descent. For example, considering the function

\[ f(\Theta) =  \theta_1 x_1 + \theta_2 x_2\]

\noindent with fixed $x_1$, $x_2$ from some dataset $X$. To minimize this function $f$ with respect to the variable parameters $\Theta$ using the dataset and SGD, the algorithm is modified to only take into consideration a single datapoint $x_i = [x_{i1}, x_{i2}]$ at a time chosen at random from $X$. Then, updating $\Theta$ is done by the formula

\[ \Theta_i = \Theta_{i-1} - \eta \nabla_{f_i} \]

\noindent where $f_i$ is the function $f$ evaluated for $x_i$ and $\Theta_{i-1}$, while the number of steps that it takes to consider all datapoints in the dataset is called an \textit{epoch}. Similarily, to use Mini-Batch SGD with a batch size of $n$, an average over the gradients for $n$ datapoints is used:

\[ \Theta_i = \Theta_{i-1} - \eta \left ( \frac{1}{n} \sum \limits_{i=1}^{n} \nabla_{f_i} \right ) \]

\noindent The price that is paid for the relaxed memory cost of SGD is that the traversion towards the minimum becomes noisy for small $n$, as the algorithm approximates the gradient by using only a part of the dataset. For functions with minima in shallow ``valleys'', SGD starts moving in a wild zig-zag pattern near the optimum, further increasing the number of steps the algorithm takes until covergence. To address these problems, a number of optimizations were proposed.


	\subsubsection {Momentum}
\text{Momentum} is an optimization for the Gradient Descent that adds an additional factor to the update step which is dependent on the steepness of the traversed function in previous updates. Using the analogy of a marble rolling down a slope, momentum refers to the inertia of the marble that allows it to jump over saddle points or small holes (local minima) on the bowl's surface rather than stopping abruptly once such a location is reached. This changes the SGD update formula to

\begin {align}
	& v_i = \gamma v_ {i-1} + \eta \nabla_f\\ 
	& \Theta_i = \Theta_{i-1} - v_i
\end {align}

\noindent where $v$ is the momentum vector with one entry per gradient element and $\gamma$ is a modifier that changes the influence of the momentum on the update.

	\begin {figure}[!ht]
		\begin{center}
			\scalebox{0.75}{%% Creator: Matplotlib, PGF backend
%%
%% To include the figure in your LaTeX document, write
%%   \input{<filename>.pgf}
%%
%% Make sure the required packages are loaded in your preamble
%%   \usepackage{pgf}
%%
%% Figures using additional raster images can only be included by \input if
%% they are in the same directory as the main LaTeX file. For loading figures
%% from other directories you can use the `import` package
%%   \usepackage{import}
%% and then include the figures with
%%   \import{<path to file>}{<filename>.pgf}
%%
%% Matplotlib used the following preamble
%%   \usepackage{fontspec}
%%   \setmainfont{DejaVu Serif}
%%   \setsansfont{DejaVu Sans}
%%   \setmonofont{DejaVu Sans Mono}
%%
\begingroup%
\makeatletter%
\begin{pgfpicture}%
\pgfpathrectangle{\pgfpointorigin}{\pgfqpoint{6.400000in}{4.800000in}}%
\pgfusepath{use as bounding box, clip}%
\begin{pgfscope}%
\pgfsetbuttcap%
\pgfsetmiterjoin%
\definecolor{currentfill}{rgb}{1.000000,1.000000,1.000000}%
\pgfsetfillcolor{currentfill}%
\pgfsetlinewidth{0.000000pt}%
\definecolor{currentstroke}{rgb}{1.000000,1.000000,1.000000}%
\pgfsetstrokecolor{currentstroke}%
\pgfsetdash{}{0pt}%
\pgfpathmoveto{\pgfqpoint{0.000000in}{0.000000in}}%
\pgfpathlineto{\pgfqpoint{6.400000in}{0.000000in}}%
\pgfpathlineto{\pgfqpoint{6.400000in}{4.800000in}}%
\pgfpathlineto{\pgfqpoint{0.000000in}{4.800000in}}%
\pgfpathclose%
\pgfusepath{fill}%
\end{pgfscope}%
\begin{pgfscope}%
\pgfsetbuttcap%
\pgfsetmiterjoin%
\definecolor{currentfill}{rgb}{1.000000,1.000000,1.000000}%
\pgfsetfillcolor{currentfill}%
\pgfsetlinewidth{0.000000pt}%
\definecolor{currentstroke}{rgb}{0.000000,0.000000,0.000000}%
\pgfsetstrokecolor{currentstroke}%
\pgfsetstrokeopacity{0.000000}%
\pgfsetdash{}{0pt}%
\pgfpathmoveto{\pgfqpoint{0.800000in}{0.528000in}}%
\pgfpathlineto{\pgfqpoint{2.763636in}{0.528000in}}%
\pgfpathlineto{\pgfqpoint{2.763636in}{4.224000in}}%
\pgfpathlineto{\pgfqpoint{0.800000in}{4.224000in}}%
\pgfpathclose%
\pgfusepath{fill}%
\end{pgfscope}%
\begin{pgfscope}%
\pgfpathrectangle{\pgfqpoint{0.800000in}{0.528000in}}{\pgfqpoint{1.963636in}{3.696000in}} %
\pgfusepath{clip}%
\pgfsetbuttcap%
\pgfsetroundjoin%
\definecolor{currentfill}{rgb}{0.050383,0.029803,0.527975}%
\pgfsetfillcolor{currentfill}%
\pgfsetlinewidth{0.000000pt}%
\definecolor{currentstroke}{rgb}{0.000000,0.000000,0.000000}%
\pgfsetstrokecolor{currentstroke}%
\pgfsetdash{}{0pt}%
\pgfpathmoveto{\pgfqpoint{1.315702in}{2.586655in}}%
\pgfpathlineto{\pgfqpoint{1.315708in}{2.586667in}}%
\pgfpathlineto{\pgfqpoint{1.315702in}{2.586679in}}%
\pgfpathlineto{\pgfqpoint{1.315697in}{2.586667in}}%
\pgfpathclose%
\pgfusepath{fill}%
\end{pgfscope}%
\begin{pgfscope}%
\pgfpathrectangle{\pgfqpoint{0.800000in}{0.528000in}}{\pgfqpoint{1.963636in}{3.696000in}} %
\pgfusepath{clip}%
\pgfsetbuttcap%
\pgfsetroundjoin%
\definecolor{currentfill}{rgb}{0.050383,0.029803,0.527975}%
\pgfsetfillcolor{currentfill}%
\pgfsetlinewidth{0.000000pt}%
\definecolor{currentstroke}{rgb}{0.000000,0.000000,0.000000}%
\pgfsetstrokecolor{currentstroke}%
\pgfsetdash{}{0pt}%
\pgfpathmoveto{\pgfqpoint{1.266116in}{2.693321in}}%
\pgfpathlineto{\pgfqpoint{1.266121in}{2.693333in}}%
\pgfpathlineto{\pgfqpoint{1.266116in}{2.693345in}}%
\pgfpathlineto{\pgfqpoint{1.266110in}{2.693333in}}%
\pgfpathclose%
\pgfusepath{fill}%
\end{pgfscope}%
\begin{pgfscope}%
\pgfpathrectangle{\pgfqpoint{0.800000in}{0.528000in}}{\pgfqpoint{1.963636in}{3.696000in}} %
\pgfusepath{clip}%
\pgfsetbuttcap%
\pgfsetroundjoin%
\definecolor{currentfill}{rgb}{0.050383,0.029803,0.527975}%
\pgfsetfillcolor{currentfill}%
\pgfsetlinewidth{0.000000pt}%
\definecolor{currentstroke}{rgb}{0.000000,0.000000,0.000000}%
\pgfsetstrokecolor{currentstroke}%
\pgfsetdash{}{0pt}%
\pgfpathmoveto{\pgfqpoint{1.365289in}{2.479880in}}%
\pgfpathlineto{\pgfqpoint{1.365354in}{2.480000in}}%
\pgfpathlineto{\pgfqpoint{1.365289in}{2.480134in}}%
\pgfpathlineto{\pgfqpoint{1.365231in}{2.480000in}}%
\pgfpathclose%
\pgfusepath{fill}%
\end{pgfscope}%
\begin{pgfscope}%
\pgfpathrectangle{\pgfqpoint{0.800000in}{0.528000in}}{\pgfqpoint{1.963636in}{3.696000in}} %
\pgfusepath{clip}%
\pgfsetbuttcap%
\pgfsetroundjoin%
\definecolor{currentfill}{rgb}{0.050383,0.029803,0.527975}%
\pgfsetfillcolor{currentfill}%
\pgfsetlinewidth{0.000000pt}%
\definecolor{currentstroke}{rgb}{0.000000,0.000000,0.000000}%
\pgfsetstrokecolor{currentstroke}%
\pgfsetdash{}{0pt}%
\pgfpathmoveto{\pgfqpoint{1.315702in}{2.586470in}}%
\pgfpathlineto{\pgfqpoint{1.315801in}{2.586667in}}%
\pgfpathlineto{\pgfqpoint{1.315702in}{2.586870in}}%
\pgfpathlineto{\pgfqpoint{1.315608in}{2.586667in}}%
\pgfpathclose%
\pgfpathmoveto{\pgfqpoint{1.315697in}{2.586667in}}%
\pgfpathlineto{\pgfqpoint{1.315702in}{2.586679in}}%
\pgfpathlineto{\pgfqpoint{1.315708in}{2.586667in}}%
\pgfpathlineto{\pgfqpoint{1.315702in}{2.586655in}}%
\pgfpathclose%
\pgfusepath{fill}%
\end{pgfscope}%
\begin{pgfscope}%
\pgfpathrectangle{\pgfqpoint{0.800000in}{0.528000in}}{\pgfqpoint{1.963636in}{3.696000in}} %
\pgfusepath{clip}%
\pgfsetbuttcap%
\pgfsetroundjoin%
\definecolor{currentfill}{rgb}{0.050383,0.029803,0.527975}%
\pgfsetfillcolor{currentfill}%
\pgfsetlinewidth{0.000000pt}%
\definecolor{currentstroke}{rgb}{0.000000,0.000000,0.000000}%
\pgfsetstrokecolor{currentstroke}%
\pgfsetdash{}{0pt}%
\pgfpathmoveto{\pgfqpoint{1.266116in}{2.693130in}}%
\pgfpathlineto{\pgfqpoint{1.266210in}{2.693333in}}%
\pgfpathlineto{\pgfqpoint{1.266116in}{2.693530in}}%
\pgfpathlineto{\pgfqpoint{1.266017in}{2.693333in}}%
\pgfpathclose%
\pgfpathmoveto{\pgfqpoint{1.266110in}{2.693333in}}%
\pgfpathlineto{\pgfqpoint{1.266116in}{2.693345in}}%
\pgfpathlineto{\pgfqpoint{1.266121in}{2.693333in}}%
\pgfpathlineto{\pgfqpoint{1.266116in}{2.693321in}}%
\pgfpathclose%
\pgfusepath{fill}%
\end{pgfscope}%
\begin{pgfscope}%
\pgfpathrectangle{\pgfqpoint{0.800000in}{0.528000in}}{\pgfqpoint{1.963636in}{3.696000in}} %
\pgfusepath{clip}%
\pgfsetbuttcap%
\pgfsetroundjoin%
\definecolor{currentfill}{rgb}{0.050383,0.029803,0.527975}%
\pgfsetfillcolor{currentfill}%
\pgfsetlinewidth{0.000000pt}%
\definecolor{currentstroke}{rgb}{0.000000,0.000000,0.000000}%
\pgfsetstrokecolor{currentstroke}%
\pgfsetdash{}{0pt}%
\pgfpathmoveto{\pgfqpoint{1.216529in}{2.799866in}}%
\pgfpathlineto{\pgfqpoint{1.216587in}{2.800000in}}%
\pgfpathlineto{\pgfqpoint{1.216529in}{2.800120in}}%
\pgfpathlineto{\pgfqpoint{1.216464in}{2.800000in}}%
\pgfpathclose%
\pgfusepath{fill}%
\end{pgfscope}%
\begin{pgfscope}%
\pgfpathrectangle{\pgfqpoint{0.800000in}{0.528000in}}{\pgfqpoint{1.963636in}{3.696000in}} %
\pgfusepath{clip}%
\pgfsetbuttcap%
\pgfsetroundjoin%
\definecolor{currentfill}{rgb}{0.050383,0.029803,0.527975}%
\pgfsetfillcolor{currentfill}%
\pgfsetlinewidth{0.000000pt}%
\definecolor{currentstroke}{rgb}{0.000000,0.000000,0.000000}%
\pgfsetstrokecolor{currentstroke}%
\pgfsetdash{}{0pt}%
\pgfpathmoveto{\pgfqpoint{3.745455in}{-2.534940in}}%
\pgfpathlineto{\pgfqpoint{3.745455in}{-2.533333in}}%
\pgfpathlineto{\pgfqpoint{3.745455in}{-2.532190in}}%
\pgfpathlineto{\pgfqpoint{3.744676in}{-2.533333in}}%
\pgfpathclose%
\pgfusepath{fill}%
\end{pgfscope}%
\begin{pgfscope}%
\pgfpathrectangle{\pgfqpoint{0.800000in}{0.528000in}}{\pgfqpoint{1.963636in}{3.696000in}} %
\pgfusepath{clip}%
\pgfsetbuttcap%
\pgfsetroundjoin%
\definecolor{currentfill}{rgb}{0.050383,0.029803,0.527975}%
\pgfsetfillcolor{currentfill}%
\pgfsetlinewidth{0.000000pt}%
\definecolor{currentstroke}{rgb}{0.000000,0.000000,0.000000}%
\pgfsetstrokecolor{currentstroke}%
\pgfsetdash{}{0pt}%
\pgfpathmoveto{\pgfqpoint{3.695868in}{-2.427884in}}%
\pgfpathlineto{\pgfqpoint{3.696254in}{-2.426667in}}%
\pgfpathlineto{\pgfqpoint{3.695868in}{-2.425864in}}%
\pgfpathlineto{\pgfqpoint{3.695277in}{-2.426667in}}%
\pgfpathclose%
\pgfusepath{fill}%
\end{pgfscope}%
\begin{pgfscope}%
\pgfpathrectangle{\pgfqpoint{0.800000in}{0.528000in}}{\pgfqpoint{1.963636in}{3.696000in}} %
\pgfusepath{clip}%
\pgfsetbuttcap%
\pgfsetroundjoin%
\definecolor{currentfill}{rgb}{0.050383,0.029803,0.527975}%
\pgfsetfillcolor{currentfill}%
\pgfsetlinewidth{0.000000pt}%
\definecolor{currentstroke}{rgb}{0.000000,0.000000,0.000000}%
\pgfsetstrokecolor{currentstroke}%
\pgfsetdash{}{0pt}%
\pgfpathmoveto{\pgfqpoint{3.646281in}{-2.320694in}}%
\pgfpathlineto{\pgfqpoint{3.646485in}{-2.320000in}}%
\pgfpathlineto{\pgfqpoint{3.646281in}{-2.319577in}}%
\pgfpathlineto{\pgfqpoint{3.645944in}{-2.320000in}}%
\pgfpathclose%
\pgfusepath{fill}%
\end{pgfscope}%
\begin{pgfscope}%
\pgfpathrectangle{\pgfqpoint{0.800000in}{0.528000in}}{\pgfqpoint{1.963636in}{3.696000in}} %
\pgfusepath{clip}%
\pgfsetbuttcap%
\pgfsetroundjoin%
\definecolor{currentfill}{rgb}{0.050383,0.029803,0.527975}%
\pgfsetfillcolor{currentfill}%
\pgfsetlinewidth{0.000000pt}%
\definecolor{currentstroke}{rgb}{0.000000,0.000000,0.000000}%
\pgfsetstrokecolor{currentstroke}%
\pgfsetdash{}{0pt}%
\pgfpathmoveto{\pgfqpoint{3.596694in}{-2.213349in}}%
\pgfpathlineto{\pgfqpoint{3.596698in}{-2.213333in}}%
\pgfpathlineto{\pgfqpoint{3.596694in}{-2.213325in}}%
\pgfpathlineto{\pgfqpoint{3.596687in}{-2.213333in}}%
\pgfpathclose%
\pgfusepath{fill}%
\end{pgfscope}%
\begin{pgfscope}%
\pgfpathrectangle{\pgfqpoint{0.800000in}{0.528000in}}{\pgfqpoint{1.963636in}{3.696000in}} %
\pgfusepath{clip}%
\pgfsetbuttcap%
\pgfsetroundjoin%
\definecolor{currentfill}{rgb}{0.050383,0.029803,0.527975}%
\pgfsetfillcolor{currentfill}%
\pgfsetlinewidth{0.000000pt}%
\definecolor{currentstroke}{rgb}{0.000000,0.000000,0.000000}%
\pgfsetstrokecolor{currentstroke}%
\pgfsetdash{}{0pt}%
\pgfpathmoveto{\pgfqpoint{1.662810in}{1.839971in}}%
\pgfpathlineto{\pgfqpoint{1.662835in}{1.840000in}}%
\pgfpathlineto{\pgfqpoint{1.662810in}{1.840051in}}%
\pgfpathlineto{\pgfqpoint{1.662796in}{1.840000in}}%
\pgfpathclose%
\pgfusepath{fill}%
\end{pgfscope}%
\begin{pgfscope}%
\pgfpathrectangle{\pgfqpoint{0.800000in}{0.528000in}}{\pgfqpoint{1.963636in}{3.696000in}} %
\pgfusepath{clip}%
\pgfsetbuttcap%
\pgfsetroundjoin%
\definecolor{currentfill}{rgb}{0.050383,0.029803,0.527975}%
\pgfsetfillcolor{currentfill}%
\pgfsetlinewidth{0.000000pt}%
\definecolor{currentstroke}{rgb}{0.000000,0.000000,0.000000}%
\pgfsetstrokecolor{currentstroke}%
\pgfsetdash{}{0pt}%
\pgfpathmoveto{\pgfqpoint{1.613223in}{1.946225in}}%
\pgfpathlineto{\pgfqpoint{1.613573in}{1.946667in}}%
\pgfpathlineto{\pgfqpoint{1.613223in}{1.947388in}}%
\pgfpathlineto{\pgfqpoint{1.613011in}{1.946667in}}%
\pgfpathclose%
\pgfusepath{fill}%
\end{pgfscope}%
\begin{pgfscope}%
\pgfpathrectangle{\pgfqpoint{0.800000in}{0.528000in}}{\pgfqpoint{1.963636in}{3.696000in}} %
\pgfusepath{clip}%
\pgfsetbuttcap%
\pgfsetroundjoin%
\definecolor{currentfill}{rgb}{0.050383,0.029803,0.527975}%
\pgfsetfillcolor{currentfill}%
\pgfsetlinewidth{0.000000pt}%
\definecolor{currentstroke}{rgb}{0.000000,0.000000,0.000000}%
\pgfsetstrokecolor{currentstroke}%
\pgfsetdash{}{0pt}%
\pgfpathmoveto{\pgfqpoint{1.563636in}{2.052514in}}%
\pgfpathlineto{\pgfqpoint{1.564237in}{2.053333in}}%
\pgfpathlineto{\pgfqpoint{1.563636in}{2.054572in}}%
\pgfpathlineto{\pgfqpoint{1.563242in}{2.053333in}}%
\pgfpathclose%
\pgfusepath{fill}%
\end{pgfscope}%
\begin{pgfscope}%
\pgfpathrectangle{\pgfqpoint{0.800000in}{0.528000in}}{\pgfqpoint{1.963636in}{3.696000in}} %
\pgfusepath{clip}%
\pgfsetbuttcap%
\pgfsetroundjoin%
\definecolor{currentfill}{rgb}{0.050383,0.029803,0.527975}%
\pgfsetfillcolor{currentfill}%
\pgfsetlinewidth{0.000000pt}%
\definecolor{currentstroke}{rgb}{0.000000,0.000000,0.000000}%
\pgfsetstrokecolor{currentstroke}%
\pgfsetdash{}{0pt}%
\pgfpathmoveto{\pgfqpoint{1.514050in}{2.158842in}}%
\pgfpathlineto{\pgfqpoint{1.514835in}{2.160000in}}%
\pgfpathlineto{\pgfqpoint{1.514050in}{2.161621in}}%
\pgfpathlineto{\pgfqpoint{1.513492in}{2.160000in}}%
\pgfpathclose%
\pgfusepath{fill}%
\end{pgfscope}%
\begin{pgfscope}%
\pgfpathrectangle{\pgfqpoint{0.800000in}{0.528000in}}{\pgfqpoint{1.963636in}{3.696000in}} %
\pgfusepath{clip}%
\pgfsetbuttcap%
\pgfsetroundjoin%
\definecolor{currentfill}{rgb}{0.050383,0.029803,0.527975}%
\pgfsetfillcolor{currentfill}%
\pgfsetlinewidth{0.000000pt}%
\definecolor{currentstroke}{rgb}{0.000000,0.000000,0.000000}%
\pgfsetstrokecolor{currentstroke}%
\pgfsetdash{}{0pt}%
\pgfpathmoveto{\pgfqpoint{1.464463in}{2.265212in}}%
\pgfpathlineto{\pgfqpoint{1.465377in}{2.266667in}}%
\pgfpathlineto{\pgfqpoint{1.464463in}{2.268555in}}%
\pgfpathlineto{\pgfqpoint{1.463762in}{2.266667in}}%
\pgfpathclose%
\pgfusepath{fill}%
\end{pgfscope}%
\begin{pgfscope}%
\pgfpathrectangle{\pgfqpoint{0.800000in}{0.528000in}}{\pgfqpoint{1.963636in}{3.696000in}} %
\pgfusepath{clip}%
\pgfsetbuttcap%
\pgfsetroundjoin%
\definecolor{currentfill}{rgb}{0.050383,0.029803,0.527975}%
\pgfsetfillcolor{currentfill}%
\pgfsetlinewidth{0.000000pt}%
\definecolor{currentstroke}{rgb}{0.000000,0.000000,0.000000}%
\pgfsetstrokecolor{currentstroke}%
\pgfsetdash{}{0pt}%
\pgfpathmoveto{\pgfqpoint{1.414876in}{2.371628in}}%
\pgfpathlineto{\pgfqpoint{1.415868in}{2.373333in}}%
\pgfpathlineto{\pgfqpoint{1.414876in}{2.375386in}}%
\pgfpathlineto{\pgfqpoint{1.414055in}{2.373333in}}%
\pgfpathclose%
\pgfusepath{fill}%
\end{pgfscope}%
\begin{pgfscope}%
\pgfpathrectangle{\pgfqpoint{0.800000in}{0.528000in}}{\pgfqpoint{1.963636in}{3.696000in}} %
\pgfusepath{clip}%
\pgfsetbuttcap%
\pgfsetroundjoin%
\definecolor{currentfill}{rgb}{0.050383,0.029803,0.527975}%
\pgfsetfillcolor{currentfill}%
\pgfsetlinewidth{0.000000pt}%
\definecolor{currentstroke}{rgb}{0.000000,0.000000,0.000000}%
\pgfsetstrokecolor{currentstroke}%
\pgfsetdash{}{0pt}%
\pgfpathmoveto{\pgfqpoint{1.365289in}{2.478097in}}%
\pgfpathlineto{\pgfqpoint{1.366317in}{2.480000in}}%
\pgfpathlineto{\pgfqpoint{1.365289in}{2.482127in}}%
\pgfpathlineto{\pgfqpoint{1.364372in}{2.480000in}}%
\pgfpathclose%
\pgfpathmoveto{\pgfqpoint{1.365231in}{2.480000in}}%
\pgfpathlineto{\pgfqpoint{1.365289in}{2.480134in}}%
\pgfpathlineto{\pgfqpoint{1.365354in}{2.480000in}}%
\pgfpathlineto{\pgfqpoint{1.365289in}{2.479880in}}%
\pgfpathclose%
\pgfusepath{fill}%
\end{pgfscope}%
\begin{pgfscope}%
\pgfpathrectangle{\pgfqpoint{0.800000in}{0.528000in}}{\pgfqpoint{1.963636in}{3.696000in}} %
\pgfusepath{clip}%
\pgfsetbuttcap%
\pgfsetroundjoin%
\definecolor{currentfill}{rgb}{0.050383,0.029803,0.527975}%
\pgfsetfillcolor{currentfill}%
\pgfsetlinewidth{0.000000pt}%
\definecolor{currentstroke}{rgb}{0.000000,0.000000,0.000000}%
\pgfsetstrokecolor{currentstroke}%
\pgfsetdash{}{0pt}%
\pgfpathmoveto{\pgfqpoint{1.315702in}{2.584623in}}%
\pgfpathlineto{\pgfqpoint{1.316726in}{2.586667in}}%
\pgfpathlineto{\pgfqpoint{1.315702in}{2.588788in}}%
\pgfpathlineto{\pgfqpoint{1.314716in}{2.586667in}}%
\pgfpathclose%
\pgfpathmoveto{\pgfqpoint{1.315608in}{2.586667in}}%
\pgfpathlineto{\pgfqpoint{1.315702in}{2.586870in}}%
\pgfpathlineto{\pgfqpoint{1.315801in}{2.586667in}}%
\pgfpathlineto{\pgfqpoint{1.315702in}{2.586470in}}%
\pgfpathclose%
\pgfusepath{fill}%
\end{pgfscope}%
\begin{pgfscope}%
\pgfpathrectangle{\pgfqpoint{0.800000in}{0.528000in}}{\pgfqpoint{1.963636in}{3.696000in}} %
\pgfusepath{clip}%
\pgfsetbuttcap%
\pgfsetroundjoin%
\definecolor{currentfill}{rgb}{0.050383,0.029803,0.527975}%
\pgfsetfillcolor{currentfill}%
\pgfsetlinewidth{0.000000pt}%
\definecolor{currentstroke}{rgb}{0.000000,0.000000,0.000000}%
\pgfsetstrokecolor{currentstroke}%
\pgfsetdash{}{0pt}%
\pgfpathmoveto{\pgfqpoint{1.266116in}{2.691212in}}%
\pgfpathlineto{\pgfqpoint{1.267102in}{2.693333in}}%
\pgfpathlineto{\pgfqpoint{1.266116in}{2.695377in}}%
\pgfpathlineto{\pgfqpoint{1.265092in}{2.693333in}}%
\pgfpathclose%
\pgfpathmoveto{\pgfqpoint{1.266017in}{2.693333in}}%
\pgfpathlineto{\pgfqpoint{1.266116in}{2.693530in}}%
\pgfpathlineto{\pgfqpoint{1.266210in}{2.693333in}}%
\pgfpathlineto{\pgfqpoint{1.266116in}{2.693130in}}%
\pgfpathclose%
\pgfusepath{fill}%
\end{pgfscope}%
\begin{pgfscope}%
\pgfpathrectangle{\pgfqpoint{0.800000in}{0.528000in}}{\pgfqpoint{1.963636in}{3.696000in}} %
\pgfusepath{clip}%
\pgfsetbuttcap%
\pgfsetroundjoin%
\definecolor{currentfill}{rgb}{0.050383,0.029803,0.527975}%
\pgfsetfillcolor{currentfill}%
\pgfsetlinewidth{0.000000pt}%
\definecolor{currentstroke}{rgb}{0.000000,0.000000,0.000000}%
\pgfsetstrokecolor{currentstroke}%
\pgfsetdash{}{0pt}%
\pgfpathmoveto{\pgfqpoint{1.216529in}{2.797873in}}%
\pgfpathlineto{\pgfqpoint{1.217446in}{2.800000in}}%
\pgfpathlineto{\pgfqpoint{1.216529in}{2.801903in}}%
\pgfpathlineto{\pgfqpoint{1.215501in}{2.800000in}}%
\pgfpathclose%
\pgfpathmoveto{\pgfqpoint{1.216464in}{2.800000in}}%
\pgfpathlineto{\pgfqpoint{1.216529in}{2.800120in}}%
\pgfpathlineto{\pgfqpoint{1.216587in}{2.800000in}}%
\pgfpathlineto{\pgfqpoint{1.216529in}{2.799866in}}%
\pgfpathclose%
\pgfusepath{fill}%
\end{pgfscope}%
\begin{pgfscope}%
\pgfpathrectangle{\pgfqpoint{0.800000in}{0.528000in}}{\pgfqpoint{1.963636in}{3.696000in}} %
\pgfusepath{clip}%
\pgfsetbuttcap%
\pgfsetroundjoin%
\definecolor{currentfill}{rgb}{0.050383,0.029803,0.527975}%
\pgfsetfillcolor{currentfill}%
\pgfsetlinewidth{0.000000pt}%
\definecolor{currentstroke}{rgb}{0.000000,0.000000,0.000000}%
\pgfsetstrokecolor{currentstroke}%
\pgfsetdash{}{0pt}%
\pgfpathmoveto{\pgfqpoint{1.166942in}{2.904614in}}%
\pgfpathlineto{\pgfqpoint{1.167764in}{2.906667in}}%
\pgfpathlineto{\pgfqpoint{1.166942in}{2.908372in}}%
\pgfpathlineto{\pgfqpoint{1.165950in}{2.906667in}}%
\pgfpathclose%
\pgfusepath{fill}%
\end{pgfscope}%
\begin{pgfscope}%
\pgfpathrectangle{\pgfqpoint{0.800000in}{0.528000in}}{\pgfqpoint{1.963636in}{3.696000in}} %
\pgfusepath{clip}%
\pgfsetbuttcap%
\pgfsetroundjoin%
\definecolor{currentfill}{rgb}{0.050383,0.029803,0.527975}%
\pgfsetfillcolor{currentfill}%
\pgfsetlinewidth{0.000000pt}%
\definecolor{currentstroke}{rgb}{0.000000,0.000000,0.000000}%
\pgfsetstrokecolor{currentstroke}%
\pgfsetdash{}{0pt}%
\pgfpathmoveto{\pgfqpoint{1.117355in}{3.011445in}}%
\pgfpathlineto{\pgfqpoint{1.118056in}{3.013333in}}%
\pgfpathlineto{\pgfqpoint{1.117355in}{3.014788in}}%
\pgfpathlineto{\pgfqpoint{1.116442in}{3.013333in}}%
\pgfpathclose%
\pgfusepath{fill}%
\end{pgfscope}%
\begin{pgfscope}%
\pgfpathrectangle{\pgfqpoint{0.800000in}{0.528000in}}{\pgfqpoint{1.963636in}{3.696000in}} %
\pgfusepath{clip}%
\pgfsetbuttcap%
\pgfsetroundjoin%
\definecolor{currentfill}{rgb}{0.050383,0.029803,0.527975}%
\pgfsetfillcolor{currentfill}%
\pgfsetlinewidth{0.000000pt}%
\definecolor{currentstroke}{rgb}{0.000000,0.000000,0.000000}%
\pgfsetstrokecolor{currentstroke}%
\pgfsetdash{}{0pt}%
\pgfpathmoveto{\pgfqpoint{1.067769in}{3.118379in}}%
\pgfpathlineto{\pgfqpoint{1.068326in}{3.120000in}}%
\pgfpathlineto{\pgfqpoint{1.067769in}{3.121158in}}%
\pgfpathlineto{\pgfqpoint{1.066983in}{3.120000in}}%
\pgfpathclose%
\pgfusepath{fill}%
\end{pgfscope}%
\begin{pgfscope}%
\pgfpathrectangle{\pgfqpoint{0.800000in}{0.528000in}}{\pgfqpoint{1.963636in}{3.696000in}} %
\pgfusepath{clip}%
\pgfsetbuttcap%
\pgfsetroundjoin%
\definecolor{currentfill}{rgb}{0.050383,0.029803,0.527975}%
\pgfsetfillcolor{currentfill}%
\pgfsetlinewidth{0.000000pt}%
\definecolor{currentstroke}{rgb}{0.000000,0.000000,0.000000}%
\pgfsetstrokecolor{currentstroke}%
\pgfsetdash{}{0pt}%
\pgfpathmoveto{\pgfqpoint{1.018182in}{3.225428in}}%
\pgfpathlineto{\pgfqpoint{1.018576in}{3.226667in}}%
\pgfpathlineto{\pgfqpoint{1.018182in}{3.227486in}}%
\pgfpathlineto{\pgfqpoint{1.017581in}{3.226667in}}%
\pgfpathclose%
\pgfusepath{fill}%
\end{pgfscope}%
\begin{pgfscope}%
\pgfpathrectangle{\pgfqpoint{0.800000in}{0.528000in}}{\pgfqpoint{1.963636in}{3.696000in}} %
\pgfusepath{clip}%
\pgfsetbuttcap%
\pgfsetroundjoin%
\definecolor{currentfill}{rgb}{0.050383,0.029803,0.527975}%
\pgfsetfillcolor{currentfill}%
\pgfsetlinewidth{0.000000pt}%
\definecolor{currentstroke}{rgb}{0.000000,0.000000,0.000000}%
\pgfsetstrokecolor{currentstroke}%
\pgfsetdash{}{0pt}%
\pgfpathmoveto{\pgfqpoint{0.968595in}{3.332612in}}%
\pgfpathlineto{\pgfqpoint{0.968807in}{3.333333in}}%
\pgfpathlineto{\pgfqpoint{0.968595in}{3.333775in}}%
\pgfpathlineto{\pgfqpoint{0.968245in}{3.333333in}}%
\pgfpathclose%
\pgfusepath{fill}%
\end{pgfscope}%
\begin{pgfscope}%
\pgfpathrectangle{\pgfqpoint{0.800000in}{0.528000in}}{\pgfqpoint{1.963636in}{3.696000in}} %
\pgfusepath{clip}%
\pgfsetbuttcap%
\pgfsetroundjoin%
\definecolor{currentfill}{rgb}{0.050383,0.029803,0.527975}%
\pgfsetfillcolor{currentfill}%
\pgfsetlinewidth{0.000000pt}%
\definecolor{currentstroke}{rgb}{0.000000,0.000000,0.000000}%
\pgfsetstrokecolor{currentstroke}%
\pgfsetdash{}{0pt}%
\pgfpathmoveto{\pgfqpoint{0.919008in}{3.439949in}}%
\pgfpathlineto{\pgfqpoint{0.919022in}{3.440000in}}%
\pgfpathlineto{\pgfqpoint{0.919008in}{3.440029in}}%
\pgfpathlineto{\pgfqpoint{0.918984in}{3.440000in}}%
\pgfpathclose%
\pgfusepath{fill}%
\end{pgfscope}%
\begin{pgfscope}%
\pgfpathrectangle{\pgfqpoint{0.800000in}{0.528000in}}{\pgfqpoint{1.963636in}{3.696000in}} %
\pgfusepath{clip}%
\pgfsetbuttcap%
\pgfsetroundjoin%
\definecolor{currentfill}{rgb}{0.050383,0.029803,0.527975}%
\pgfsetfillcolor{currentfill}%
\pgfsetlinewidth{0.000000pt}%
\definecolor{currentstroke}{rgb}{0.000000,0.000000,0.000000}%
\pgfsetstrokecolor{currentstroke}%
\pgfsetdash{}{0pt}%
\pgfpathmoveto{\pgfqpoint{-1.014876in}{7.493325in}}%
\pgfpathlineto{\pgfqpoint{-1.014868in}{7.493333in}}%
\pgfpathlineto{\pgfqpoint{-1.014876in}{7.493349in}}%
\pgfpathlineto{\pgfqpoint{-1.014880in}{7.493333in}}%
\pgfpathclose%
\pgfusepath{fill}%
\end{pgfscope}%
\begin{pgfscope}%
\pgfpathrectangle{\pgfqpoint{0.800000in}{0.528000in}}{\pgfqpoint{1.963636in}{3.696000in}} %
\pgfusepath{clip}%
\pgfsetbuttcap%
\pgfsetroundjoin%
\definecolor{currentfill}{rgb}{0.050383,0.029803,0.527975}%
\pgfsetfillcolor{currentfill}%
\pgfsetlinewidth{0.000000pt}%
\definecolor{currentstroke}{rgb}{0.000000,0.000000,0.000000}%
\pgfsetstrokecolor{currentstroke}%
\pgfsetdash{}{0pt}%
\pgfpathmoveto{\pgfqpoint{-1.064463in}{7.599577in}}%
\pgfpathlineto{\pgfqpoint{-1.064126in}{7.600000in}}%
\pgfpathlineto{\pgfqpoint{-1.064463in}{7.600694in}}%
\pgfpathlineto{\pgfqpoint{-1.064666in}{7.600000in}}%
\pgfpathclose%
\pgfusepath{fill}%
\end{pgfscope}%
\begin{pgfscope}%
\pgfpathrectangle{\pgfqpoint{0.800000in}{0.528000in}}{\pgfqpoint{1.963636in}{3.696000in}} %
\pgfusepath{clip}%
\pgfsetbuttcap%
\pgfsetroundjoin%
\definecolor{currentfill}{rgb}{0.050383,0.029803,0.527975}%
\pgfsetfillcolor{currentfill}%
\pgfsetlinewidth{0.000000pt}%
\definecolor{currentstroke}{rgb}{0.000000,0.000000,0.000000}%
\pgfsetstrokecolor{currentstroke}%
\pgfsetdash{}{0pt}%
\pgfpathmoveto{\pgfqpoint{-1.114050in}{7.705864in}}%
\pgfpathlineto{\pgfqpoint{-1.113459in}{7.706667in}}%
\pgfpathlineto{\pgfqpoint{-1.114050in}{7.707884in}}%
\pgfpathlineto{\pgfqpoint{-1.114436in}{7.706667in}}%
\pgfpathclose%
\pgfusepath{fill}%
\end{pgfscope}%
\begin{pgfscope}%
\pgfpathrectangle{\pgfqpoint{0.800000in}{0.528000in}}{\pgfqpoint{1.963636in}{3.696000in}} %
\pgfusepath{clip}%
\pgfsetbuttcap%
\pgfsetroundjoin%
\definecolor{currentfill}{rgb}{0.050383,0.029803,0.527975}%
\pgfsetfillcolor{currentfill}%
\pgfsetlinewidth{0.000000pt}%
\definecolor{currentstroke}{rgb}{0.000000,0.000000,0.000000}%
\pgfsetstrokecolor{currentstroke}%
\pgfsetdash{}{0pt}%
\pgfpathmoveto{\pgfqpoint{-1.162858in}{7.813333in}}%
\pgfpathlineto{\pgfqpoint{-1.163636in}{7.814940in}}%
\pgfpathlineto{\pgfqpoint{-1.163636in}{7.813333in}}%
\pgfpathlineto{\pgfqpoint{-1.163636in}{7.812190in}}%
\pgfpathclose%
\pgfusepath{fill}%
\end{pgfscope}%
\begin{pgfscope}%
\pgfpathrectangle{\pgfqpoint{0.800000in}{0.528000in}}{\pgfqpoint{1.963636in}{3.696000in}} %
\pgfusepath{clip}%
\pgfsetbuttcap%
\pgfsetroundjoin%
\definecolor{currentfill}{rgb}{0.050383,0.029803,0.527975}%
\pgfsetfillcolor{currentfill}%
\pgfsetlinewidth{0.000000pt}%
\definecolor{currentstroke}{rgb}{0.000000,0.000000,0.000000}%
\pgfsetstrokecolor{currentstroke}%
\pgfsetdash{}{0pt}%
\pgfpathmoveto{\pgfqpoint{3.745455in}{-2.557564in}}%
\pgfpathlineto{\pgfqpoint{3.745455in}{-2.534940in}}%
\pgfpathlineto{\pgfqpoint{3.744676in}{-2.533333in}}%
\pgfpathlineto{\pgfqpoint{3.745455in}{-2.532190in}}%
\pgfpathlineto{\pgfqpoint{3.745455in}{-2.516082in}}%
\pgfpathlineto{\pgfqpoint{3.733717in}{-2.533333in}}%
\pgfpathclose%
\pgfusepath{fill}%
\end{pgfscope}%
\begin{pgfscope}%
\pgfpathrectangle{\pgfqpoint{0.800000in}{0.528000in}}{\pgfqpoint{1.963636in}{3.696000in}} %
\pgfusepath{clip}%
\pgfsetbuttcap%
\pgfsetroundjoin%
\definecolor{currentfill}{rgb}{0.050383,0.029803,0.527975}%
\pgfsetfillcolor{currentfill}%
\pgfsetlinewidth{0.000000pt}%
\definecolor{currentstroke}{rgb}{0.000000,0.000000,0.000000}%
\pgfsetstrokecolor{currentstroke}%
\pgfsetdash{}{0pt}%
\pgfpathmoveto{\pgfqpoint{3.695868in}{-2.451562in}}%
\pgfpathlineto{\pgfqpoint{3.703764in}{-2.426667in}}%
\pgfpathlineto{\pgfqpoint{3.695868in}{-2.410251in}}%
\pgfpathlineto{\pgfqpoint{3.683795in}{-2.426667in}}%
\pgfpathclose%
\pgfpathmoveto{\pgfqpoint{3.695277in}{-2.426667in}}%
\pgfpathlineto{\pgfqpoint{3.695868in}{-2.425864in}}%
\pgfpathlineto{\pgfqpoint{3.696254in}{-2.426667in}}%
\pgfpathlineto{\pgfqpoint{3.695868in}{-2.427884in}}%
\pgfpathclose%
\pgfusepath{fill}%
\end{pgfscope}%
\begin{pgfscope}%
\pgfpathrectangle{\pgfqpoint{0.800000in}{0.528000in}}{\pgfqpoint{1.963636in}{3.696000in}} %
\pgfusepath{clip}%
\pgfsetbuttcap%
\pgfsetroundjoin%
\definecolor{currentfill}{rgb}{0.050383,0.029803,0.527975}%
\pgfsetfillcolor{currentfill}%
\pgfsetlinewidth{0.000000pt}%
\definecolor{currentstroke}{rgb}{0.000000,0.000000,0.000000}%
\pgfsetstrokecolor{currentstroke}%
\pgfsetdash{}{0pt}%
\pgfpathmoveto{\pgfqpoint{3.646281in}{-2.345528in}}%
\pgfpathlineto{\pgfqpoint{3.653768in}{-2.320000in}}%
\pgfpathlineto{\pgfqpoint{3.646281in}{-2.304428in}}%
\pgfpathlineto{\pgfqpoint{3.633887in}{-2.320000in}}%
\pgfpathclose%
\pgfpathmoveto{\pgfqpoint{3.645944in}{-2.320000in}}%
\pgfpathlineto{\pgfqpoint{3.646281in}{-2.319577in}}%
\pgfpathlineto{\pgfqpoint{3.646485in}{-2.320000in}}%
\pgfpathlineto{\pgfqpoint{3.646281in}{-2.320694in}}%
\pgfpathclose%
\pgfusepath{fill}%
\end{pgfscope}%
\begin{pgfscope}%
\pgfpathrectangle{\pgfqpoint{0.800000in}{0.528000in}}{\pgfqpoint{1.963636in}{3.696000in}} %
\pgfusepath{clip}%
\pgfsetbuttcap%
\pgfsetroundjoin%
\definecolor{currentfill}{rgb}{0.050383,0.029803,0.527975}%
\pgfsetfillcolor{currentfill}%
\pgfsetlinewidth{0.000000pt}%
\definecolor{currentstroke}{rgb}{0.000000,0.000000,0.000000}%
\pgfsetstrokecolor{currentstroke}%
\pgfsetdash{}{0pt}%
\pgfpathmoveto{\pgfqpoint{3.596694in}{-2.239458in}}%
\pgfpathlineto{\pgfqpoint{3.603768in}{-2.213333in}}%
\pgfpathlineto{\pgfqpoint{3.596694in}{-2.198615in}}%
\pgfpathlineto{\pgfqpoint{3.583994in}{-2.213333in}}%
\pgfpathclose%
\pgfpathmoveto{\pgfqpoint{3.596687in}{-2.213333in}}%
\pgfpathlineto{\pgfqpoint{3.596694in}{-2.213325in}}%
\pgfpathlineto{\pgfqpoint{3.596698in}{-2.213333in}}%
\pgfpathlineto{\pgfqpoint{3.596694in}{-2.213349in}}%
\pgfpathclose%
\pgfusepath{fill}%
\end{pgfscope}%
\begin{pgfscope}%
\pgfpathrectangle{\pgfqpoint{0.800000in}{0.528000in}}{\pgfqpoint{1.963636in}{3.696000in}} %
\pgfusepath{clip}%
\pgfsetbuttcap%
\pgfsetroundjoin%
\definecolor{currentfill}{rgb}{0.050383,0.029803,0.527975}%
\pgfsetfillcolor{currentfill}%
\pgfsetlinewidth{0.000000pt}%
\definecolor{currentstroke}{rgb}{0.000000,0.000000,0.000000}%
\pgfsetstrokecolor{currentstroke}%
\pgfsetdash{}{0pt}%
\pgfpathmoveto{\pgfqpoint{3.547107in}{-2.133346in}}%
\pgfpathlineto{\pgfqpoint{3.553765in}{-2.106667in}}%
\pgfpathlineto{\pgfqpoint{3.547107in}{-2.092808in}}%
\pgfpathlineto{\pgfqpoint{3.534120in}{-2.106667in}}%
\pgfpathclose%
\pgfusepath{fill}%
\end{pgfscope}%
\begin{pgfscope}%
\pgfpathrectangle{\pgfqpoint{0.800000in}{0.528000in}}{\pgfqpoint{1.963636in}{3.696000in}} %
\pgfusepath{clip}%
\pgfsetbuttcap%
\pgfsetroundjoin%
\definecolor{currentfill}{rgb}{0.050383,0.029803,0.527975}%
\pgfsetfillcolor{currentfill}%
\pgfsetlinewidth{0.000000pt}%
\definecolor{currentstroke}{rgb}{0.000000,0.000000,0.000000}%
\pgfsetstrokecolor{currentstroke}%
\pgfsetdash{}{0pt}%
\pgfpathmoveto{\pgfqpoint{3.497521in}{-2.027184in}}%
\pgfpathlineto{\pgfqpoint{3.503759in}{-2.000000in}}%
\pgfpathlineto{\pgfqpoint{3.497521in}{-1.987009in}}%
\pgfpathlineto{\pgfqpoint{3.484266in}{-2.000000in}}%
\pgfpathclose%
\pgfusepath{fill}%
\end{pgfscope}%
\begin{pgfscope}%
\pgfpathrectangle{\pgfqpoint{0.800000in}{0.528000in}}{\pgfqpoint{1.963636in}{3.696000in}} %
\pgfusepath{clip}%
\pgfsetbuttcap%
\pgfsetroundjoin%
\definecolor{currentfill}{rgb}{0.050383,0.029803,0.527975}%
\pgfsetfillcolor{currentfill}%
\pgfsetlinewidth{0.000000pt}%
\definecolor{currentstroke}{rgb}{0.000000,0.000000,0.000000}%
\pgfsetstrokecolor{currentstroke}%
\pgfsetdash{}{0pt}%
\pgfpathmoveto{\pgfqpoint{3.447934in}{-1.920964in}}%
\pgfpathlineto{\pgfqpoint{3.453751in}{-1.893333in}}%
\pgfpathlineto{\pgfqpoint{3.447934in}{-1.881216in}}%
\pgfpathlineto{\pgfqpoint{3.434437in}{-1.893333in}}%
\pgfpathclose%
\pgfusepath{fill}%
\end{pgfscope}%
\begin{pgfscope}%
\pgfpathrectangle{\pgfqpoint{0.800000in}{0.528000in}}{\pgfqpoint{1.963636in}{3.696000in}} %
\pgfusepath{clip}%
\pgfsetbuttcap%
\pgfsetroundjoin%
\definecolor{currentfill}{rgb}{0.050383,0.029803,0.527975}%
\pgfsetfillcolor{currentfill}%
\pgfsetlinewidth{0.000000pt}%
\definecolor{currentstroke}{rgb}{0.000000,0.000000,0.000000}%
\pgfsetstrokecolor{currentstroke}%
\pgfsetdash{}{0pt}%
\pgfpathmoveto{\pgfqpoint{3.398347in}{-1.814674in}}%
\pgfpathlineto{\pgfqpoint{3.403740in}{-1.786667in}}%
\pgfpathlineto{\pgfqpoint{3.398347in}{-1.775428in}}%
\pgfpathlineto{\pgfqpoint{3.384639in}{-1.786667in}}%
\pgfpathclose%
\pgfusepath{fill}%
\end{pgfscope}%
\begin{pgfscope}%
\pgfpathrectangle{\pgfqpoint{0.800000in}{0.528000in}}{\pgfqpoint{1.963636in}{3.696000in}} %
\pgfusepath{clip}%
\pgfsetbuttcap%
\pgfsetroundjoin%
\definecolor{currentfill}{rgb}{0.050383,0.029803,0.527975}%
\pgfsetfillcolor{currentfill}%
\pgfsetlinewidth{0.000000pt}%
\definecolor{currentstroke}{rgb}{0.000000,0.000000,0.000000}%
\pgfsetstrokecolor{currentstroke}%
\pgfsetdash{}{0pt}%
\pgfpathmoveto{\pgfqpoint{3.348760in}{-1.708300in}}%
\pgfpathlineto{\pgfqpoint{3.353727in}{-1.680000in}}%
\pgfpathlineto{\pgfqpoint{3.348760in}{-1.669646in}}%
\pgfpathlineto{\pgfqpoint{3.334877in}{-1.680000in}}%
\pgfpathclose%
\pgfusepath{fill}%
\end{pgfscope}%
\begin{pgfscope}%
\pgfpathrectangle{\pgfqpoint{0.800000in}{0.528000in}}{\pgfqpoint{1.963636in}{3.696000in}} %
\pgfusepath{clip}%
\pgfsetbuttcap%
\pgfsetroundjoin%
\definecolor{currentfill}{rgb}{0.050383,0.029803,0.527975}%
\pgfsetfillcolor{currentfill}%
\pgfsetlinewidth{0.000000pt}%
\definecolor{currentstroke}{rgb}{0.000000,0.000000,0.000000}%
\pgfsetstrokecolor{currentstroke}%
\pgfsetdash{}{0pt}%
\pgfpathmoveto{\pgfqpoint{3.299174in}{-1.601822in}}%
\pgfpathlineto{\pgfqpoint{3.303712in}{-1.573333in}}%
\pgfpathlineto{\pgfqpoint{3.299174in}{-1.563870in}}%
\pgfpathlineto{\pgfqpoint{3.285160in}{-1.573333in}}%
\pgfpathclose%
\pgfusepath{fill}%
\end{pgfscope}%
\begin{pgfscope}%
\pgfpathrectangle{\pgfqpoint{0.800000in}{0.528000in}}{\pgfqpoint{1.963636in}{3.696000in}} %
\pgfusepath{clip}%
\pgfsetbuttcap%
\pgfsetroundjoin%
\definecolor{currentfill}{rgb}{0.050383,0.029803,0.527975}%
\pgfsetfillcolor{currentfill}%
\pgfsetlinewidth{0.000000pt}%
\definecolor{currentstroke}{rgb}{0.000000,0.000000,0.000000}%
\pgfsetstrokecolor{currentstroke}%
\pgfsetdash{}{0pt}%
\pgfpathmoveto{\pgfqpoint{3.249587in}{-1.495217in}}%
\pgfpathlineto{\pgfqpoint{3.253695in}{-1.466667in}}%
\pgfpathlineto{\pgfqpoint{3.249587in}{-1.458098in}}%
\pgfpathlineto{\pgfqpoint{3.235500in}{-1.466667in}}%
\pgfpathclose%
\pgfusepath{fill}%
\end{pgfscope}%
\begin{pgfscope}%
\pgfpathrectangle{\pgfqpoint{0.800000in}{0.528000in}}{\pgfqpoint{1.963636in}{3.696000in}} %
\pgfusepath{clip}%
\pgfsetbuttcap%
\pgfsetroundjoin%
\definecolor{currentfill}{rgb}{0.050383,0.029803,0.527975}%
\pgfsetfillcolor{currentfill}%
\pgfsetlinewidth{0.000000pt}%
\definecolor{currentstroke}{rgb}{0.000000,0.000000,0.000000}%
\pgfsetstrokecolor{currentstroke}%
\pgfsetdash{}{0pt}%
\pgfpathmoveto{\pgfqpoint{3.200000in}{-1.388451in}}%
\pgfpathlineto{\pgfqpoint{3.203676in}{-1.360000in}}%
\pgfpathlineto{\pgfqpoint{3.200000in}{-1.352330in}}%
\pgfpathlineto{\pgfqpoint{3.185911in}{-1.360000in}}%
\pgfpathclose%
\pgfusepath{fill}%
\end{pgfscope}%
\begin{pgfscope}%
\pgfpathrectangle{\pgfqpoint{0.800000in}{0.528000in}}{\pgfqpoint{1.963636in}{3.696000in}} %
\pgfusepath{clip}%
\pgfsetbuttcap%
\pgfsetroundjoin%
\definecolor{currentfill}{rgb}{0.050383,0.029803,0.527975}%
\pgfsetfillcolor{currentfill}%
\pgfsetlinewidth{0.000000pt}%
\definecolor{currentstroke}{rgb}{0.000000,0.000000,0.000000}%
\pgfsetstrokecolor{currentstroke}%
\pgfsetdash{}{0pt}%
\pgfpathmoveto{\pgfqpoint{3.150413in}{-1.281477in}}%
\pgfpathlineto{\pgfqpoint{3.153656in}{-1.253333in}}%
\pgfpathlineto{\pgfqpoint{3.150413in}{-1.246566in}}%
\pgfpathlineto{\pgfqpoint{3.136415in}{-1.253333in}}%
\pgfpathclose%
\pgfusepath{fill}%
\end{pgfscope}%
\begin{pgfscope}%
\pgfpathrectangle{\pgfqpoint{0.800000in}{0.528000in}}{\pgfqpoint{1.963636in}{3.696000in}} %
\pgfusepath{clip}%
\pgfsetbuttcap%
\pgfsetroundjoin%
\definecolor{currentfill}{rgb}{0.050383,0.029803,0.527975}%
\pgfsetfillcolor{currentfill}%
\pgfsetlinewidth{0.000000pt}%
\definecolor{currentstroke}{rgb}{0.000000,0.000000,0.000000}%
\pgfsetstrokecolor{currentstroke}%
\pgfsetdash{}{0pt}%
\pgfpathmoveto{\pgfqpoint{3.100826in}{-1.174230in}}%
\pgfpathlineto{\pgfqpoint{3.103634in}{-1.146667in}}%
\pgfpathlineto{\pgfqpoint{3.100826in}{-1.140806in}}%
\pgfpathlineto{\pgfqpoint{3.087044in}{-1.146667in}}%
\pgfpathclose%
\pgfusepath{fill}%
\end{pgfscope}%
\begin{pgfscope}%
\pgfpathrectangle{\pgfqpoint{0.800000in}{0.528000in}}{\pgfqpoint{1.963636in}{3.696000in}} %
\pgfusepath{clip}%
\pgfsetbuttcap%
\pgfsetroundjoin%
\definecolor{currentfill}{rgb}{0.050383,0.029803,0.527975}%
\pgfsetfillcolor{currentfill}%
\pgfsetlinewidth{0.000000pt}%
\definecolor{currentstroke}{rgb}{0.000000,0.000000,0.000000}%
\pgfsetstrokecolor{currentstroke}%
\pgfsetdash{}{0pt}%
\pgfpathmoveto{\pgfqpoint{3.051240in}{-1.066613in}}%
\pgfpathlineto{\pgfqpoint{3.053610in}{-1.040000in}}%
\pgfpathlineto{\pgfqpoint{3.051240in}{-1.035050in}}%
\pgfpathlineto{\pgfqpoint{3.037845in}{-1.040000in}}%
\pgfpathclose%
\pgfusepath{fill}%
\end{pgfscope}%
\begin{pgfscope}%
\pgfpathrectangle{\pgfqpoint{0.800000in}{0.528000in}}{\pgfqpoint{1.963636in}{3.696000in}} %
\pgfusepath{clip}%
\pgfsetbuttcap%
\pgfsetroundjoin%
\definecolor{currentfill}{rgb}{0.050383,0.029803,0.527975}%
\pgfsetfillcolor{currentfill}%
\pgfsetlinewidth{0.000000pt}%
\definecolor{currentstroke}{rgb}{0.000000,0.000000,0.000000}%
\pgfsetstrokecolor{currentstroke}%
\pgfsetdash{}{0pt}%
\pgfpathmoveto{\pgfqpoint{3.001653in}{-0.958478in}}%
\pgfpathlineto{\pgfqpoint{3.003586in}{-0.933333in}}%
\pgfpathlineto{\pgfqpoint{3.001653in}{-0.929296in}}%
\pgfpathlineto{\pgfqpoint{2.988890in}{-0.933333in}}%
\pgfpathclose%
\pgfusepath{fill}%
\end{pgfscope}%
\begin{pgfscope}%
\pgfpathrectangle{\pgfqpoint{0.800000in}{0.528000in}}{\pgfqpoint{1.963636in}{3.696000in}} %
\pgfusepath{clip}%
\pgfsetbuttcap%
\pgfsetroundjoin%
\definecolor{currentfill}{rgb}{0.050383,0.029803,0.527975}%
\pgfsetfillcolor{currentfill}%
\pgfsetlinewidth{0.000000pt}%
\definecolor{currentstroke}{rgb}{0.000000,0.000000,0.000000}%
\pgfsetstrokecolor{currentstroke}%
\pgfsetdash{}{0pt}%
\pgfpathmoveto{\pgfqpoint{2.952066in}{-0.849585in}}%
\pgfpathlineto{\pgfqpoint{2.953560in}{-0.826667in}}%
\pgfpathlineto{\pgfqpoint{2.952066in}{-0.823546in}}%
\pgfpathlineto{\pgfqpoint{2.940303in}{-0.826667in}}%
\pgfpathclose%
\pgfusepath{fill}%
\end{pgfscope}%
\begin{pgfscope}%
\pgfpathrectangle{\pgfqpoint{0.800000in}{0.528000in}}{\pgfqpoint{1.963636in}{3.696000in}} %
\pgfusepath{clip}%
\pgfsetbuttcap%
\pgfsetroundjoin%
\definecolor{currentfill}{rgb}{0.050383,0.029803,0.527975}%
\pgfsetfillcolor{currentfill}%
\pgfsetlinewidth{0.000000pt}%
\definecolor{currentstroke}{rgb}{0.000000,0.000000,0.000000}%
\pgfsetstrokecolor{currentstroke}%
\pgfsetdash{}{0pt}%
\pgfpathmoveto{\pgfqpoint{2.902479in}{-0.739519in}}%
\pgfpathlineto{\pgfqpoint{2.903533in}{-0.720000in}}%
\pgfpathlineto{\pgfqpoint{2.902479in}{-0.717799in}}%
\pgfpathlineto{\pgfqpoint{2.892306in}{-0.720000in}}%
\pgfpathclose%
\pgfusepath{fill}%
\end{pgfscope}%
\begin{pgfscope}%
\pgfpathrectangle{\pgfqpoint{0.800000in}{0.528000in}}{\pgfqpoint{1.963636in}{3.696000in}} %
\pgfusepath{clip}%
\pgfsetbuttcap%
\pgfsetroundjoin%
\definecolor{currentfill}{rgb}{0.050383,0.029803,0.527975}%
\pgfsetfillcolor{currentfill}%
\pgfsetlinewidth{0.000000pt}%
\definecolor{currentstroke}{rgb}{0.000000,0.000000,0.000000}%
\pgfsetstrokecolor{currentstroke}%
\pgfsetdash{}{0pt}%
\pgfpathmoveto{\pgfqpoint{2.852893in}{-0.627492in}}%
\pgfpathlineto{\pgfqpoint{2.853504in}{-0.613333in}}%
\pgfpathlineto{\pgfqpoint{2.852893in}{-0.612055in}}%
\pgfpathlineto{\pgfqpoint{2.845343in}{-0.613333in}}%
\pgfpathclose%
\pgfusepath{fill}%
\end{pgfscope}%
\begin{pgfscope}%
\pgfpathrectangle{\pgfqpoint{0.800000in}{0.528000in}}{\pgfqpoint{1.963636in}{3.696000in}} %
\pgfusepath{clip}%
\pgfsetbuttcap%
\pgfsetroundjoin%
\definecolor{currentfill}{rgb}{0.050383,0.029803,0.527975}%
\pgfsetfillcolor{currentfill}%
\pgfsetlinewidth{0.000000pt}%
\definecolor{currentstroke}{rgb}{0.000000,0.000000,0.000000}%
\pgfsetstrokecolor{currentstroke}%
\pgfsetdash{}{0pt}%
\pgfpathmoveto{\pgfqpoint{2.803306in}{-0.511816in}}%
\pgfpathlineto{\pgfqpoint{2.803475in}{-0.506667in}}%
\pgfpathlineto{\pgfqpoint{2.803306in}{-0.506313in}}%
\pgfpathlineto{\pgfqpoint{2.800457in}{-0.506667in}}%
\pgfpathclose%
\pgfusepath{fill}%
\end{pgfscope}%
\begin{pgfscope}%
\pgfpathrectangle{\pgfqpoint{0.800000in}{0.528000in}}{\pgfqpoint{1.963636in}{3.696000in}} %
\pgfusepath{clip}%
\pgfsetbuttcap%
\pgfsetroundjoin%
\definecolor{currentfill}{rgb}{0.050383,0.029803,0.527975}%
\pgfsetfillcolor{currentfill}%
\pgfsetlinewidth{0.000000pt}%
\definecolor{currentstroke}{rgb}{0.000000,0.000000,0.000000}%
\pgfsetstrokecolor{currentstroke}%
\pgfsetdash{}{0pt}%
\pgfpathmoveto{\pgfqpoint{2.456198in}{0.132936in}}%
\pgfpathlineto{\pgfqpoint{2.459341in}{0.133333in}}%
\pgfpathlineto{\pgfqpoint{2.456198in}{0.139027in}}%
\pgfpathlineto{\pgfqpoint{2.456008in}{0.133333in}}%
\pgfpathclose%
\pgfusepath{fill}%
\end{pgfscope}%
\begin{pgfscope}%
\pgfpathrectangle{\pgfqpoint{0.800000in}{0.528000in}}{\pgfqpoint{1.963636in}{3.696000in}} %
\pgfusepath{clip}%
\pgfsetbuttcap%
\pgfsetroundjoin%
\definecolor{currentfill}{rgb}{0.050383,0.029803,0.527975}%
\pgfsetfillcolor{currentfill}%
\pgfsetlinewidth{0.000000pt}%
\definecolor{currentstroke}{rgb}{0.000000,0.000000,0.000000}%
\pgfsetstrokecolor{currentstroke}%
\pgfsetdash{}{0pt}%
\pgfpathmoveto{\pgfqpoint{2.406612in}{0.238678in}}%
\pgfpathlineto{\pgfqpoint{2.414317in}{0.240000in}}%
\pgfpathlineto{\pgfqpoint{2.406612in}{0.254471in}}%
\pgfpathlineto{\pgfqpoint{2.405979in}{0.240000in}}%
\pgfpathclose%
\pgfusepath{fill}%
\end{pgfscope}%
\begin{pgfscope}%
\pgfpathrectangle{\pgfqpoint{0.800000in}{0.528000in}}{\pgfqpoint{1.963636in}{3.696000in}} %
\pgfusepath{clip}%
\pgfsetbuttcap%
\pgfsetroundjoin%
\definecolor{currentfill}{rgb}{0.050383,0.029803,0.527975}%
\pgfsetfillcolor{currentfill}%
\pgfsetlinewidth{0.000000pt}%
\definecolor{currentstroke}{rgb}{0.000000,0.000000,0.000000}%
\pgfsetstrokecolor{currentstroke}%
\pgfsetdash{}{0pt}%
\pgfpathmoveto{\pgfqpoint{2.357025in}{0.344423in}}%
\pgfpathlineto{\pgfqpoint{2.367290in}{0.346667in}}%
\pgfpathlineto{\pgfqpoint{2.357025in}{0.366378in}}%
\pgfpathlineto{\pgfqpoint{2.355951in}{0.346667in}}%
\pgfpathclose%
\pgfusepath{fill}%
\end{pgfscope}%
\begin{pgfscope}%
\pgfpathrectangle{\pgfqpoint{0.800000in}{0.528000in}}{\pgfqpoint{1.963636in}{3.696000in}} %
\pgfusepath{clip}%
\pgfsetbuttcap%
\pgfsetroundjoin%
\definecolor{currentfill}{rgb}{0.050383,0.029803,0.527975}%
\pgfsetfillcolor{currentfill}%
\pgfsetlinewidth{0.000000pt}%
\definecolor{currentstroke}{rgb}{0.000000,0.000000,0.000000}%
\pgfsetstrokecolor{currentstroke}%
\pgfsetdash{}{0pt}%
\pgfpathmoveto{\pgfqpoint{2.307438in}{0.450170in}}%
\pgfpathlineto{\pgfqpoint{2.319258in}{0.453333in}}%
\pgfpathlineto{\pgfqpoint{2.307438in}{0.476377in}}%
\pgfpathlineto{\pgfqpoint{2.305924in}{0.453333in}}%
\pgfpathclose%
\pgfusepath{fill}%
\end{pgfscope}%
\begin{pgfscope}%
\pgfpathrectangle{\pgfqpoint{0.800000in}{0.528000in}}{\pgfqpoint{1.963636in}{3.696000in}} %
\pgfusepath{clip}%
\pgfsetbuttcap%
\pgfsetroundjoin%
\definecolor{currentfill}{rgb}{0.050383,0.029803,0.527975}%
\pgfsetfillcolor{currentfill}%
\pgfsetlinewidth{0.000000pt}%
\definecolor{currentstroke}{rgb}{0.000000,0.000000,0.000000}%
\pgfsetstrokecolor{currentstroke}%
\pgfsetdash{}{0pt}%
\pgfpathmoveto{\pgfqpoint{2.257851in}{0.555921in}}%
\pgfpathlineto{\pgfqpoint{2.270651in}{0.560000in}}%
\pgfpathlineto{\pgfqpoint{2.257851in}{0.585227in}}%
\pgfpathlineto{\pgfqpoint{2.255898in}{0.560000in}}%
\pgfpathclose%
\pgfusepath{fill}%
\end{pgfscope}%
\begin{pgfscope}%
\pgfpathrectangle{\pgfqpoint{0.800000in}{0.528000in}}{\pgfqpoint{1.963636in}{3.696000in}} %
\pgfusepath{clip}%
\pgfsetbuttcap%
\pgfsetroundjoin%
\definecolor{currentfill}{rgb}{0.050383,0.029803,0.527975}%
\pgfsetfillcolor{currentfill}%
\pgfsetlinewidth{0.000000pt}%
\definecolor{currentstroke}{rgb}{0.000000,0.000000,0.000000}%
\pgfsetstrokecolor{currentstroke}%
\pgfsetdash{}{0pt}%
\pgfpathmoveto{\pgfqpoint{2.208264in}{0.661674in}}%
\pgfpathlineto{\pgfqpoint{2.221682in}{0.666667in}}%
\pgfpathlineto{\pgfqpoint{2.208264in}{0.693334in}}%
\pgfpathlineto{\pgfqpoint{2.205873in}{0.666667in}}%
\pgfpathclose%
\pgfusepath{fill}%
\end{pgfscope}%
\begin{pgfscope}%
\pgfpathrectangle{\pgfqpoint{0.800000in}{0.528000in}}{\pgfqpoint{1.963636in}{3.696000in}} %
\pgfusepath{clip}%
\pgfsetbuttcap%
\pgfsetroundjoin%
\definecolor{currentfill}{rgb}{0.050383,0.029803,0.527975}%
\pgfsetfillcolor{currentfill}%
\pgfsetlinewidth{0.000000pt}%
\definecolor{currentstroke}{rgb}{0.000000,0.000000,0.000000}%
\pgfsetstrokecolor{currentstroke}%
\pgfsetdash{}{0pt}%
\pgfpathmoveto{\pgfqpoint{2.158678in}{0.767430in}}%
\pgfpathlineto{\pgfqpoint{2.172473in}{0.773333in}}%
\pgfpathlineto{\pgfqpoint{2.158678in}{0.800931in}}%
\pgfpathlineto{\pgfqpoint{2.155850in}{0.773333in}}%
\pgfpathclose%
\pgfusepath{fill}%
\end{pgfscope}%
\begin{pgfscope}%
\pgfpathrectangle{\pgfqpoint{0.800000in}{0.528000in}}{\pgfqpoint{1.963636in}{3.696000in}} %
\pgfusepath{clip}%
\pgfsetbuttcap%
\pgfsetroundjoin%
\definecolor{currentfill}{rgb}{0.050383,0.029803,0.527975}%
\pgfsetfillcolor{currentfill}%
\pgfsetlinewidth{0.000000pt}%
\definecolor{currentstroke}{rgb}{0.000000,0.000000,0.000000}%
\pgfsetstrokecolor{currentstroke}%
\pgfsetdash{}{0pt}%
\pgfpathmoveto{\pgfqpoint{2.109091in}{0.873191in}}%
\pgfpathlineto{\pgfqpoint{2.123095in}{0.880000in}}%
\pgfpathlineto{\pgfqpoint{2.109091in}{0.908163in}}%
\pgfpathlineto{\pgfqpoint{2.105828in}{0.880000in}}%
\pgfpathclose%
\pgfusepath{fill}%
\end{pgfscope}%
\begin{pgfscope}%
\pgfpathrectangle{\pgfqpoint{0.800000in}{0.528000in}}{\pgfqpoint{1.963636in}{3.696000in}} %
\pgfusepath{clip}%
\pgfsetbuttcap%
\pgfsetroundjoin%
\definecolor{currentfill}{rgb}{0.050383,0.029803,0.527975}%
\pgfsetfillcolor{currentfill}%
\pgfsetlinewidth{0.000000pt}%
\definecolor{currentstroke}{rgb}{0.000000,0.000000,0.000000}%
\pgfsetstrokecolor{currentstroke}%
\pgfsetdash{}{0pt}%
\pgfpathmoveto{\pgfqpoint{2.059504in}{0.978955in}}%
\pgfpathlineto{\pgfqpoint{2.073595in}{0.986667in}}%
\pgfpathlineto{\pgfqpoint{2.059504in}{1.015126in}}%
\pgfpathlineto{\pgfqpoint{2.055808in}{0.986667in}}%
\pgfpathclose%
\pgfusepath{fill}%
\end{pgfscope}%
\begin{pgfscope}%
\pgfpathrectangle{\pgfqpoint{0.800000in}{0.528000in}}{\pgfqpoint{1.963636in}{3.696000in}} %
\pgfusepath{clip}%
\pgfsetbuttcap%
\pgfsetroundjoin%
\definecolor{currentfill}{rgb}{0.050383,0.029803,0.527975}%
\pgfsetfillcolor{currentfill}%
\pgfsetlinewidth{0.000000pt}%
\definecolor{currentstroke}{rgb}{0.000000,0.000000,0.000000}%
\pgfsetstrokecolor{currentstroke}%
\pgfsetdash{}{0pt}%
\pgfpathmoveto{\pgfqpoint{2.009917in}{1.084723in}}%
\pgfpathlineto{\pgfqpoint{2.024002in}{1.093333in}}%
\pgfpathlineto{\pgfqpoint{2.009917in}{1.121884in}}%
\pgfpathlineto{\pgfqpoint{2.005789in}{1.093333in}}%
\pgfpathclose%
\pgfusepath{fill}%
\end{pgfscope}%
\begin{pgfscope}%
\pgfpathrectangle{\pgfqpoint{0.800000in}{0.528000in}}{\pgfqpoint{1.963636in}{3.696000in}} %
\pgfusepath{clip}%
\pgfsetbuttcap%
\pgfsetroundjoin%
\definecolor{currentfill}{rgb}{0.050383,0.029803,0.527975}%
\pgfsetfillcolor{currentfill}%
\pgfsetlinewidth{0.000000pt}%
\definecolor{currentstroke}{rgb}{0.000000,0.000000,0.000000}%
\pgfsetstrokecolor{currentstroke}%
\pgfsetdash{}{0pt}%
\pgfpathmoveto{\pgfqpoint{1.960331in}{1.190495in}}%
\pgfpathlineto{\pgfqpoint{1.974339in}{1.200000in}}%
\pgfpathlineto{\pgfqpoint{1.960331in}{1.228483in}}%
\pgfpathlineto{\pgfqpoint{1.955772in}{1.200000in}}%
\pgfpathclose%
\pgfusepath{fill}%
\end{pgfscope}%
\begin{pgfscope}%
\pgfpathrectangle{\pgfqpoint{0.800000in}{0.528000in}}{\pgfqpoint{1.963636in}{3.696000in}} %
\pgfusepath{clip}%
\pgfsetbuttcap%
\pgfsetroundjoin%
\definecolor{currentfill}{rgb}{0.050383,0.029803,0.527975}%
\pgfsetfillcolor{currentfill}%
\pgfsetlinewidth{0.000000pt}%
\definecolor{currentstroke}{rgb}{0.000000,0.000000,0.000000}%
\pgfsetstrokecolor{currentstroke}%
\pgfsetdash{}{0pt}%
\pgfpathmoveto{\pgfqpoint{1.910744in}{1.296272in}}%
\pgfpathlineto{\pgfqpoint{1.924620in}{1.306667in}}%
\pgfpathlineto{\pgfqpoint{1.910744in}{1.334955in}}%
\pgfpathlineto{\pgfqpoint{1.905757in}{1.306667in}}%
\pgfpathclose%
\pgfusepath{fill}%
\end{pgfscope}%
\begin{pgfscope}%
\pgfpathrectangle{\pgfqpoint{0.800000in}{0.528000in}}{\pgfqpoint{1.963636in}{3.696000in}} %
\pgfusepath{clip}%
\pgfsetbuttcap%
\pgfsetroundjoin%
\definecolor{currentfill}{rgb}{0.050383,0.029803,0.527975}%
\pgfsetfillcolor{currentfill}%
\pgfsetlinewidth{0.000000pt}%
\definecolor{currentstroke}{rgb}{0.000000,0.000000,0.000000}%
\pgfsetstrokecolor{currentstroke}%
\pgfsetdash{}{0pt}%
\pgfpathmoveto{\pgfqpoint{1.861157in}{1.402054in}}%
\pgfpathlineto{\pgfqpoint{1.874856in}{1.413333in}}%
\pgfpathlineto{\pgfqpoint{1.861157in}{1.441325in}}%
\pgfpathlineto{\pgfqpoint{1.855744in}{1.413333in}}%
\pgfpathclose%
\pgfusepath{fill}%
\end{pgfscope}%
\begin{pgfscope}%
\pgfpathrectangle{\pgfqpoint{0.800000in}{0.528000in}}{\pgfqpoint{1.963636in}{3.696000in}} %
\pgfusepath{clip}%
\pgfsetbuttcap%
\pgfsetroundjoin%
\definecolor{currentfill}{rgb}{0.050383,0.029803,0.527975}%
\pgfsetfillcolor{currentfill}%
\pgfsetlinewidth{0.000000pt}%
\definecolor{currentstroke}{rgb}{0.000000,0.000000,0.000000}%
\pgfsetstrokecolor{currentstroke}%
\pgfsetdash{}{0pt}%
\pgfpathmoveto{\pgfqpoint{1.811570in}{1.507841in}}%
\pgfpathlineto{\pgfqpoint{1.825056in}{1.520000in}}%
\pgfpathlineto{\pgfqpoint{1.811570in}{1.547611in}}%
\pgfpathlineto{\pgfqpoint{1.805734in}{1.520000in}}%
\pgfpathclose%
\pgfusepath{fill}%
\end{pgfscope}%
\begin{pgfscope}%
\pgfpathrectangle{\pgfqpoint{0.800000in}{0.528000in}}{\pgfqpoint{1.963636in}{3.696000in}} %
\pgfusepath{clip}%
\pgfsetbuttcap%
\pgfsetroundjoin%
\definecolor{currentfill}{rgb}{0.050383,0.029803,0.527975}%
\pgfsetfillcolor{currentfill}%
\pgfsetlinewidth{0.000000pt}%
\definecolor{currentstroke}{rgb}{0.000000,0.000000,0.000000}%
\pgfsetstrokecolor{currentstroke}%
\pgfsetdash{}{0pt}%
\pgfpathmoveto{\pgfqpoint{1.761983in}{1.613635in}}%
\pgfpathlineto{\pgfqpoint{1.775226in}{1.626667in}}%
\pgfpathlineto{\pgfqpoint{1.761983in}{1.653829in}}%
\pgfpathlineto{\pgfqpoint{1.755725in}{1.626667in}}%
\pgfpathclose%
\pgfusepath{fill}%
\end{pgfscope}%
\begin{pgfscope}%
\pgfpathrectangle{\pgfqpoint{0.800000in}{0.528000in}}{\pgfqpoint{1.963636in}{3.696000in}} %
\pgfusepath{clip}%
\pgfsetbuttcap%
\pgfsetroundjoin%
\definecolor{currentfill}{rgb}{0.050383,0.029803,0.527975}%
\pgfsetfillcolor{currentfill}%
\pgfsetlinewidth{0.000000pt}%
\definecolor{currentstroke}{rgb}{0.000000,0.000000,0.000000}%
\pgfsetstrokecolor{currentstroke}%
\pgfsetdash{}{0pt}%
\pgfpathmoveto{\pgfqpoint{1.712397in}{1.719435in}}%
\pgfpathlineto{\pgfqpoint{1.725372in}{1.733333in}}%
\pgfpathlineto{\pgfqpoint{1.712397in}{1.759988in}}%
\pgfpathlineto{\pgfqpoint{1.705720in}{1.733333in}}%
\pgfpathclose%
\pgfusepath{fill}%
\end{pgfscope}%
\begin{pgfscope}%
\pgfpathrectangle{\pgfqpoint{0.800000in}{0.528000in}}{\pgfqpoint{1.963636in}{3.696000in}} %
\pgfusepath{clip}%
\pgfsetbuttcap%
\pgfsetroundjoin%
\definecolor{currentfill}{rgb}{0.050383,0.029803,0.527975}%
\pgfsetfillcolor{currentfill}%
\pgfsetlinewidth{0.000000pt}%
\definecolor{currentstroke}{rgb}{0.000000,0.000000,0.000000}%
\pgfsetstrokecolor{currentstroke}%
\pgfsetdash{}{0pt}%
\pgfpathmoveto{\pgfqpoint{1.662810in}{1.825241in}}%
\pgfpathlineto{\pgfqpoint{1.675496in}{1.840000in}}%
\pgfpathlineto{\pgfqpoint{1.662810in}{1.866098in}}%
\pgfpathlineto{\pgfqpoint{1.655717in}{1.840000in}}%
\pgfpathclose%
\pgfpathmoveto{\pgfqpoint{1.662796in}{1.840000in}}%
\pgfpathlineto{\pgfqpoint{1.662810in}{1.840051in}}%
\pgfpathlineto{\pgfqpoint{1.662835in}{1.840000in}}%
\pgfpathlineto{\pgfqpoint{1.662810in}{1.839971in}}%
\pgfpathclose%
\pgfusepath{fill}%
\end{pgfscope}%
\begin{pgfscope}%
\pgfpathrectangle{\pgfqpoint{0.800000in}{0.528000in}}{\pgfqpoint{1.963636in}{3.696000in}} %
\pgfusepath{clip}%
\pgfsetbuttcap%
\pgfsetroundjoin%
\definecolor{currentfill}{rgb}{0.050383,0.029803,0.527975}%
\pgfsetfillcolor{currentfill}%
\pgfsetlinewidth{0.000000pt}%
\definecolor{currentstroke}{rgb}{0.000000,0.000000,0.000000}%
\pgfsetstrokecolor{currentstroke}%
\pgfsetdash{}{0pt}%
\pgfpathmoveto{\pgfqpoint{1.613223in}{1.931056in}}%
\pgfpathlineto{\pgfqpoint{1.625602in}{1.946667in}}%
\pgfpathlineto{\pgfqpoint{1.613223in}{1.972166in}}%
\pgfpathlineto{\pgfqpoint{1.605717in}{1.946667in}}%
\pgfpathclose%
\pgfpathmoveto{\pgfqpoint{1.613011in}{1.946667in}}%
\pgfpathlineto{\pgfqpoint{1.613223in}{1.947388in}}%
\pgfpathlineto{\pgfqpoint{1.613573in}{1.946667in}}%
\pgfpathlineto{\pgfqpoint{1.613223in}{1.946225in}}%
\pgfpathclose%
\pgfusepath{fill}%
\end{pgfscope}%
\begin{pgfscope}%
\pgfpathrectangle{\pgfqpoint{0.800000in}{0.528000in}}{\pgfqpoint{1.963636in}{3.696000in}} %
\pgfusepath{clip}%
\pgfsetbuttcap%
\pgfsetroundjoin%
\definecolor{currentfill}{rgb}{0.050383,0.029803,0.527975}%
\pgfsetfillcolor{currentfill}%
\pgfsetlinewidth{0.000000pt}%
\definecolor{currentstroke}{rgb}{0.000000,0.000000,0.000000}%
\pgfsetstrokecolor{currentstroke}%
\pgfsetdash{}{0pt}%
\pgfpathmoveto{\pgfqpoint{1.563636in}{2.036878in}}%
\pgfpathlineto{\pgfqpoint{1.575693in}{2.053333in}}%
\pgfpathlineto{\pgfqpoint{1.563636in}{2.078198in}}%
\pgfpathlineto{\pgfqpoint{1.555721in}{2.053333in}}%
\pgfpathclose%
\pgfpathmoveto{\pgfqpoint{1.563242in}{2.053333in}}%
\pgfpathlineto{\pgfqpoint{1.563636in}{2.054572in}}%
\pgfpathlineto{\pgfqpoint{1.564237in}{2.053333in}}%
\pgfpathlineto{\pgfqpoint{1.563636in}{2.052514in}}%
\pgfpathclose%
\pgfusepath{fill}%
\end{pgfscope}%
\begin{pgfscope}%
\pgfpathrectangle{\pgfqpoint{0.800000in}{0.528000in}}{\pgfqpoint{1.963636in}{3.696000in}} %
\pgfusepath{clip}%
\pgfsetbuttcap%
\pgfsetroundjoin%
\definecolor{currentfill}{rgb}{0.050383,0.029803,0.527975}%
\pgfsetfillcolor{currentfill}%
\pgfsetlinewidth{0.000000pt}%
\definecolor{currentstroke}{rgb}{0.000000,0.000000,0.000000}%
\pgfsetstrokecolor{currentstroke}%
\pgfsetdash{}{0pt}%
\pgfpathmoveto{\pgfqpoint{1.514050in}{2.142710in}}%
\pgfpathlineto{\pgfqpoint{1.525771in}{2.160000in}}%
\pgfpathlineto{\pgfqpoint{1.514050in}{2.184199in}}%
\pgfpathlineto{\pgfqpoint{1.505728in}{2.160000in}}%
\pgfpathclose%
\pgfpathmoveto{\pgfqpoint{1.513492in}{2.160000in}}%
\pgfpathlineto{\pgfqpoint{1.514050in}{2.161621in}}%
\pgfpathlineto{\pgfqpoint{1.514835in}{2.160000in}}%
\pgfpathlineto{\pgfqpoint{1.514050in}{2.158842in}}%
\pgfpathclose%
\pgfusepath{fill}%
\end{pgfscope}%
\begin{pgfscope}%
\pgfpathrectangle{\pgfqpoint{0.800000in}{0.528000in}}{\pgfqpoint{1.963636in}{3.696000in}} %
\pgfusepath{clip}%
\pgfsetbuttcap%
\pgfsetroundjoin%
\definecolor{currentfill}{rgb}{0.050383,0.029803,0.527975}%
\pgfsetfillcolor{currentfill}%
\pgfsetlinewidth{0.000000pt}%
\definecolor{currentstroke}{rgb}{0.000000,0.000000,0.000000}%
\pgfsetstrokecolor{currentstroke}%
\pgfsetdash{}{0pt}%
\pgfpathmoveto{\pgfqpoint{1.464463in}{2.248551in}}%
\pgfpathlineto{\pgfqpoint{1.475838in}{2.266667in}}%
\pgfpathlineto{\pgfqpoint{1.464463in}{2.290173in}}%
\pgfpathlineto{\pgfqpoint{1.455740in}{2.266667in}}%
\pgfpathclose%
\pgfpathmoveto{\pgfqpoint{1.463762in}{2.266667in}}%
\pgfpathlineto{\pgfqpoint{1.464463in}{2.268555in}}%
\pgfpathlineto{\pgfqpoint{1.465377in}{2.266667in}}%
\pgfpathlineto{\pgfqpoint{1.464463in}{2.265212in}}%
\pgfpathclose%
\pgfusepath{fill}%
\end{pgfscope}%
\begin{pgfscope}%
\pgfpathrectangle{\pgfqpoint{0.800000in}{0.528000in}}{\pgfqpoint{1.963636in}{3.696000in}} %
\pgfusepath{clip}%
\pgfsetbuttcap%
\pgfsetroundjoin%
\definecolor{currentfill}{rgb}{0.050383,0.029803,0.527975}%
\pgfsetfillcolor{currentfill}%
\pgfsetlinewidth{0.000000pt}%
\definecolor{currentstroke}{rgb}{0.000000,0.000000,0.000000}%
\pgfsetstrokecolor{currentstroke}%
\pgfsetdash{}{0pt}%
\pgfpathmoveto{\pgfqpoint{1.414876in}{2.354404in}}%
\pgfpathlineto{\pgfqpoint{1.425894in}{2.373333in}}%
\pgfpathlineto{\pgfqpoint{1.414876in}{2.396122in}}%
\pgfpathlineto{\pgfqpoint{1.405756in}{2.373333in}}%
\pgfpathclose%
\pgfpathmoveto{\pgfqpoint{1.414055in}{2.373333in}}%
\pgfpathlineto{\pgfqpoint{1.414876in}{2.375386in}}%
\pgfpathlineto{\pgfqpoint{1.415868in}{2.373333in}}%
\pgfpathlineto{\pgfqpoint{1.414876in}{2.371628in}}%
\pgfpathclose%
\pgfusepath{fill}%
\end{pgfscope}%
\begin{pgfscope}%
\pgfpathrectangle{\pgfqpoint{0.800000in}{0.528000in}}{\pgfqpoint{1.963636in}{3.696000in}} %
\pgfusepath{clip}%
\pgfsetbuttcap%
\pgfsetroundjoin%
\definecolor{currentfill}{rgb}{0.050383,0.029803,0.527975}%
\pgfsetfillcolor{currentfill}%
\pgfsetlinewidth{0.000000pt}%
\definecolor{currentstroke}{rgb}{0.000000,0.000000,0.000000}%
\pgfsetstrokecolor{currentstroke}%
\pgfsetdash{}{0pt}%
\pgfpathmoveto{\pgfqpoint{1.365289in}{2.460269in}}%
\pgfpathlineto{\pgfqpoint{1.375942in}{2.480000in}}%
\pgfpathlineto{\pgfqpoint{1.365289in}{2.502051in}}%
\pgfpathlineto{\pgfqpoint{1.355777in}{2.480000in}}%
\pgfpathclose%
\pgfpathmoveto{\pgfqpoint{1.364372in}{2.480000in}}%
\pgfpathlineto{\pgfqpoint{1.365289in}{2.482127in}}%
\pgfpathlineto{\pgfqpoint{1.366317in}{2.480000in}}%
\pgfpathlineto{\pgfqpoint{1.365289in}{2.478097in}}%
\pgfpathclose%
\pgfusepath{fill}%
\end{pgfscope}%
\begin{pgfscope}%
\pgfpathrectangle{\pgfqpoint{0.800000in}{0.528000in}}{\pgfqpoint{1.963636in}{3.696000in}} %
\pgfusepath{clip}%
\pgfsetbuttcap%
\pgfsetroundjoin%
\definecolor{currentfill}{rgb}{0.050383,0.029803,0.527975}%
\pgfsetfillcolor{currentfill}%
\pgfsetlinewidth{0.000000pt}%
\definecolor{currentstroke}{rgb}{0.000000,0.000000,0.000000}%
\pgfsetstrokecolor{currentstroke}%
\pgfsetdash{}{0pt}%
\pgfpathmoveto{\pgfqpoint{1.315702in}{2.566147in}}%
\pgfpathlineto{\pgfqpoint{1.325982in}{2.586667in}}%
\pgfpathlineto{\pgfqpoint{1.315702in}{2.607960in}}%
\pgfpathlineto{\pgfqpoint{1.305804in}{2.586667in}}%
\pgfpathclose%
\pgfpathmoveto{\pgfqpoint{1.314716in}{2.586667in}}%
\pgfpathlineto{\pgfqpoint{1.315702in}{2.588788in}}%
\pgfpathlineto{\pgfqpoint{1.316726in}{2.586667in}}%
\pgfpathlineto{\pgfqpoint{1.315702in}{2.584623in}}%
\pgfpathclose%
\pgfusepath{fill}%
\end{pgfscope}%
\begin{pgfscope}%
\pgfpathrectangle{\pgfqpoint{0.800000in}{0.528000in}}{\pgfqpoint{1.963636in}{3.696000in}} %
\pgfusepath{clip}%
\pgfsetbuttcap%
\pgfsetroundjoin%
\definecolor{currentfill}{rgb}{0.050383,0.029803,0.527975}%
\pgfsetfillcolor{currentfill}%
\pgfsetlinewidth{0.000000pt}%
\definecolor{currentstroke}{rgb}{0.000000,0.000000,0.000000}%
\pgfsetstrokecolor{currentstroke}%
\pgfsetdash{}{0pt}%
\pgfpathmoveto{\pgfqpoint{1.266116in}{2.672040in}}%
\pgfpathlineto{\pgfqpoint{1.276015in}{2.693333in}}%
\pgfpathlineto{\pgfqpoint{1.266116in}{2.713853in}}%
\pgfpathlineto{\pgfqpoint{1.255836in}{2.693333in}}%
\pgfpathclose%
\pgfpathmoveto{\pgfqpoint{1.265092in}{2.693333in}}%
\pgfpathlineto{\pgfqpoint{1.266116in}{2.695377in}}%
\pgfpathlineto{\pgfqpoint{1.267102in}{2.693333in}}%
\pgfpathlineto{\pgfqpoint{1.266116in}{2.691212in}}%
\pgfpathclose%
\pgfusepath{fill}%
\end{pgfscope}%
\begin{pgfscope}%
\pgfpathrectangle{\pgfqpoint{0.800000in}{0.528000in}}{\pgfqpoint{1.963636in}{3.696000in}} %
\pgfusepath{clip}%
\pgfsetbuttcap%
\pgfsetroundjoin%
\definecolor{currentfill}{rgb}{0.050383,0.029803,0.527975}%
\pgfsetfillcolor{currentfill}%
\pgfsetlinewidth{0.000000pt}%
\definecolor{currentstroke}{rgb}{0.000000,0.000000,0.000000}%
\pgfsetstrokecolor{currentstroke}%
\pgfsetdash{}{0pt}%
\pgfpathmoveto{\pgfqpoint{1.216529in}{2.777949in}}%
\pgfpathlineto{\pgfqpoint{1.226041in}{2.800000in}}%
\pgfpathlineto{\pgfqpoint{1.216529in}{2.819731in}}%
\pgfpathlineto{\pgfqpoint{1.205876in}{2.800000in}}%
\pgfpathclose%
\pgfpathmoveto{\pgfqpoint{1.215501in}{2.800000in}}%
\pgfpathlineto{\pgfqpoint{1.216529in}{2.801903in}}%
\pgfpathlineto{\pgfqpoint{1.217446in}{2.800000in}}%
\pgfpathlineto{\pgfqpoint{1.216529in}{2.797873in}}%
\pgfpathclose%
\pgfusepath{fill}%
\end{pgfscope}%
\begin{pgfscope}%
\pgfpathrectangle{\pgfqpoint{0.800000in}{0.528000in}}{\pgfqpoint{1.963636in}{3.696000in}} %
\pgfusepath{clip}%
\pgfsetbuttcap%
\pgfsetroundjoin%
\definecolor{currentfill}{rgb}{0.050383,0.029803,0.527975}%
\pgfsetfillcolor{currentfill}%
\pgfsetlinewidth{0.000000pt}%
\definecolor{currentstroke}{rgb}{0.000000,0.000000,0.000000}%
\pgfsetstrokecolor{currentstroke}%
\pgfsetdash{}{0pt}%
\pgfpathmoveto{\pgfqpoint{1.166942in}{2.883878in}}%
\pgfpathlineto{\pgfqpoint{1.176062in}{2.906667in}}%
\pgfpathlineto{\pgfqpoint{1.166942in}{2.925596in}}%
\pgfpathlineto{\pgfqpoint{1.155924in}{2.906667in}}%
\pgfpathclose%
\pgfpathmoveto{\pgfqpoint{1.165950in}{2.906667in}}%
\pgfpathlineto{\pgfqpoint{1.166942in}{2.908372in}}%
\pgfpathlineto{\pgfqpoint{1.167764in}{2.906667in}}%
\pgfpathlineto{\pgfqpoint{1.166942in}{2.904614in}}%
\pgfpathclose%
\pgfusepath{fill}%
\end{pgfscope}%
\begin{pgfscope}%
\pgfpathrectangle{\pgfqpoint{0.800000in}{0.528000in}}{\pgfqpoint{1.963636in}{3.696000in}} %
\pgfusepath{clip}%
\pgfsetbuttcap%
\pgfsetroundjoin%
\definecolor{currentfill}{rgb}{0.050383,0.029803,0.527975}%
\pgfsetfillcolor{currentfill}%
\pgfsetlinewidth{0.000000pt}%
\definecolor{currentstroke}{rgb}{0.000000,0.000000,0.000000}%
\pgfsetstrokecolor{currentstroke}%
\pgfsetdash{}{0pt}%
\pgfpathmoveto{\pgfqpoint{1.117355in}{2.989827in}}%
\pgfpathlineto{\pgfqpoint{1.126078in}{3.013333in}}%
\pgfpathlineto{\pgfqpoint{1.117355in}{3.031449in}}%
\pgfpathlineto{\pgfqpoint{1.105980in}{3.013333in}}%
\pgfpathclose%
\pgfpathmoveto{\pgfqpoint{1.116442in}{3.013333in}}%
\pgfpathlineto{\pgfqpoint{1.117355in}{3.014788in}}%
\pgfpathlineto{\pgfqpoint{1.118056in}{3.013333in}}%
\pgfpathlineto{\pgfqpoint{1.117355in}{3.011445in}}%
\pgfpathclose%
\pgfusepath{fill}%
\end{pgfscope}%
\begin{pgfscope}%
\pgfpathrectangle{\pgfqpoint{0.800000in}{0.528000in}}{\pgfqpoint{1.963636in}{3.696000in}} %
\pgfusepath{clip}%
\pgfsetbuttcap%
\pgfsetroundjoin%
\definecolor{currentfill}{rgb}{0.050383,0.029803,0.527975}%
\pgfsetfillcolor{currentfill}%
\pgfsetlinewidth{0.000000pt}%
\definecolor{currentstroke}{rgb}{0.000000,0.000000,0.000000}%
\pgfsetstrokecolor{currentstroke}%
\pgfsetdash{}{0pt}%
\pgfpathmoveto{\pgfqpoint{1.067769in}{3.095801in}}%
\pgfpathlineto{\pgfqpoint{1.076090in}{3.120000in}}%
\pgfpathlineto{\pgfqpoint{1.067769in}{3.137290in}}%
\pgfpathlineto{\pgfqpoint{1.056047in}{3.120000in}}%
\pgfpathclose%
\pgfpathmoveto{\pgfqpoint{1.066983in}{3.120000in}}%
\pgfpathlineto{\pgfqpoint{1.067769in}{3.121158in}}%
\pgfpathlineto{\pgfqpoint{1.068326in}{3.120000in}}%
\pgfpathlineto{\pgfqpoint{1.067769in}{3.118379in}}%
\pgfpathclose%
\pgfusepath{fill}%
\end{pgfscope}%
\begin{pgfscope}%
\pgfpathrectangle{\pgfqpoint{0.800000in}{0.528000in}}{\pgfqpoint{1.963636in}{3.696000in}} %
\pgfusepath{clip}%
\pgfsetbuttcap%
\pgfsetroundjoin%
\definecolor{currentfill}{rgb}{0.050383,0.029803,0.527975}%
\pgfsetfillcolor{currentfill}%
\pgfsetlinewidth{0.000000pt}%
\definecolor{currentstroke}{rgb}{0.000000,0.000000,0.000000}%
\pgfsetstrokecolor{currentstroke}%
\pgfsetdash{}{0pt}%
\pgfpathmoveto{\pgfqpoint{1.018182in}{3.201802in}}%
\pgfpathlineto{\pgfqpoint{1.026097in}{3.226667in}}%
\pgfpathlineto{\pgfqpoint{1.018182in}{3.243122in}}%
\pgfpathlineto{\pgfqpoint{1.006125in}{3.226667in}}%
\pgfpathclose%
\pgfpathmoveto{\pgfqpoint{1.017581in}{3.226667in}}%
\pgfpathlineto{\pgfqpoint{1.018182in}{3.227486in}}%
\pgfpathlineto{\pgfqpoint{1.018576in}{3.226667in}}%
\pgfpathlineto{\pgfqpoint{1.018182in}{3.225428in}}%
\pgfpathclose%
\pgfusepath{fill}%
\end{pgfscope}%
\begin{pgfscope}%
\pgfpathrectangle{\pgfqpoint{0.800000in}{0.528000in}}{\pgfqpoint{1.963636in}{3.696000in}} %
\pgfusepath{clip}%
\pgfsetbuttcap%
\pgfsetroundjoin%
\definecolor{currentfill}{rgb}{0.050383,0.029803,0.527975}%
\pgfsetfillcolor{currentfill}%
\pgfsetlinewidth{0.000000pt}%
\definecolor{currentstroke}{rgb}{0.000000,0.000000,0.000000}%
\pgfsetstrokecolor{currentstroke}%
\pgfsetdash{}{0pt}%
\pgfpathmoveto{\pgfqpoint{0.968595in}{3.307834in}}%
\pgfpathlineto{\pgfqpoint{0.976101in}{3.333333in}}%
\pgfpathlineto{\pgfqpoint{0.968595in}{3.348944in}}%
\pgfpathlineto{\pgfqpoint{0.956216in}{3.333333in}}%
\pgfpathclose%
\pgfpathmoveto{\pgfqpoint{0.968245in}{3.333333in}}%
\pgfpathlineto{\pgfqpoint{0.968595in}{3.333775in}}%
\pgfpathlineto{\pgfqpoint{0.968807in}{3.333333in}}%
\pgfpathlineto{\pgfqpoint{0.968595in}{3.332612in}}%
\pgfpathclose%
\pgfusepath{fill}%
\end{pgfscope}%
\begin{pgfscope}%
\pgfpathrectangle{\pgfqpoint{0.800000in}{0.528000in}}{\pgfqpoint{1.963636in}{3.696000in}} %
\pgfusepath{clip}%
\pgfsetbuttcap%
\pgfsetroundjoin%
\definecolor{currentfill}{rgb}{0.050383,0.029803,0.527975}%
\pgfsetfillcolor{currentfill}%
\pgfsetlinewidth{0.000000pt}%
\definecolor{currentstroke}{rgb}{0.000000,0.000000,0.000000}%
\pgfsetstrokecolor{currentstroke}%
\pgfsetdash{}{0pt}%
\pgfpathmoveto{\pgfqpoint{0.919008in}{3.413902in}}%
\pgfpathlineto{\pgfqpoint{0.926101in}{3.440000in}}%
\pgfpathlineto{\pgfqpoint{0.919008in}{3.454759in}}%
\pgfpathlineto{\pgfqpoint{0.906322in}{3.440000in}}%
\pgfpathclose%
\pgfpathmoveto{\pgfqpoint{0.918984in}{3.440000in}}%
\pgfpathlineto{\pgfqpoint{0.919008in}{3.440029in}}%
\pgfpathlineto{\pgfqpoint{0.919022in}{3.440000in}}%
\pgfpathlineto{\pgfqpoint{0.919008in}{3.439949in}}%
\pgfpathclose%
\pgfusepath{fill}%
\end{pgfscope}%
\begin{pgfscope}%
\pgfpathrectangle{\pgfqpoint{0.800000in}{0.528000in}}{\pgfqpoint{1.963636in}{3.696000in}} %
\pgfusepath{clip}%
\pgfsetbuttcap%
\pgfsetroundjoin%
\definecolor{currentfill}{rgb}{0.050383,0.029803,0.527975}%
\pgfsetfillcolor{currentfill}%
\pgfsetlinewidth{0.000000pt}%
\definecolor{currentstroke}{rgb}{0.000000,0.000000,0.000000}%
\pgfsetstrokecolor{currentstroke}%
\pgfsetdash{}{0pt}%
\pgfpathmoveto{\pgfqpoint{0.869421in}{3.520012in}}%
\pgfpathlineto{\pgfqpoint{0.876098in}{3.546667in}}%
\pgfpathlineto{\pgfqpoint{0.869421in}{3.560565in}}%
\pgfpathlineto{\pgfqpoint{0.856447in}{3.546667in}}%
\pgfpathclose%
\pgfusepath{fill}%
\end{pgfscope}%
\begin{pgfscope}%
\pgfpathrectangle{\pgfqpoint{0.800000in}{0.528000in}}{\pgfqpoint{1.963636in}{3.696000in}} %
\pgfusepath{clip}%
\pgfsetbuttcap%
\pgfsetroundjoin%
\definecolor{currentfill}{rgb}{0.050383,0.029803,0.527975}%
\pgfsetfillcolor{currentfill}%
\pgfsetlinewidth{0.000000pt}%
\definecolor{currentstroke}{rgb}{0.000000,0.000000,0.000000}%
\pgfsetstrokecolor{currentstroke}%
\pgfsetdash{}{0pt}%
\pgfpathmoveto{\pgfqpoint{0.819835in}{3.626171in}}%
\pgfpathlineto{\pgfqpoint{0.826093in}{3.653333in}}%
\pgfpathlineto{\pgfqpoint{0.819835in}{3.666365in}}%
\pgfpathlineto{\pgfqpoint{0.806592in}{3.653333in}}%
\pgfpathclose%
\pgfusepath{fill}%
\end{pgfscope}%
\begin{pgfscope}%
\pgfpathrectangle{\pgfqpoint{0.800000in}{0.528000in}}{\pgfqpoint{1.963636in}{3.696000in}} %
\pgfusepath{clip}%
\pgfsetbuttcap%
\pgfsetroundjoin%
\definecolor{currentfill}{rgb}{0.050383,0.029803,0.527975}%
\pgfsetfillcolor{currentfill}%
\pgfsetlinewidth{0.000000pt}%
\definecolor{currentstroke}{rgb}{0.000000,0.000000,0.000000}%
\pgfsetstrokecolor{currentstroke}%
\pgfsetdash{}{0pt}%
\pgfpathmoveto{\pgfqpoint{0.770248in}{3.732389in}}%
\pgfpathlineto{\pgfqpoint{0.776084in}{3.760000in}}%
\pgfpathlineto{\pgfqpoint{0.770248in}{3.772159in}}%
\pgfpathlineto{\pgfqpoint{0.756762in}{3.760000in}}%
\pgfpathclose%
\pgfusepath{fill}%
\end{pgfscope}%
\begin{pgfscope}%
\pgfpathrectangle{\pgfqpoint{0.800000in}{0.528000in}}{\pgfqpoint{1.963636in}{3.696000in}} %
\pgfusepath{clip}%
\pgfsetbuttcap%
\pgfsetroundjoin%
\definecolor{currentfill}{rgb}{0.050383,0.029803,0.527975}%
\pgfsetfillcolor{currentfill}%
\pgfsetlinewidth{0.000000pt}%
\definecolor{currentstroke}{rgb}{0.000000,0.000000,0.000000}%
\pgfsetstrokecolor{currentstroke}%
\pgfsetdash{}{0pt}%
\pgfpathmoveto{\pgfqpoint{0.720661in}{3.838675in}}%
\pgfpathlineto{\pgfqpoint{0.726074in}{3.866667in}}%
\pgfpathlineto{\pgfqpoint{0.720661in}{3.877946in}}%
\pgfpathlineto{\pgfqpoint{0.706962in}{3.866667in}}%
\pgfpathclose%
\pgfusepath{fill}%
\end{pgfscope}%
\begin{pgfscope}%
\pgfpathrectangle{\pgfqpoint{0.800000in}{0.528000in}}{\pgfqpoint{1.963636in}{3.696000in}} %
\pgfusepath{clip}%
\pgfsetbuttcap%
\pgfsetroundjoin%
\definecolor{currentfill}{rgb}{0.050383,0.029803,0.527975}%
\pgfsetfillcolor{currentfill}%
\pgfsetlinewidth{0.000000pt}%
\definecolor{currentstroke}{rgb}{0.000000,0.000000,0.000000}%
\pgfsetstrokecolor{currentstroke}%
\pgfsetdash{}{0pt}%
\pgfpathmoveto{\pgfqpoint{0.671074in}{3.945045in}}%
\pgfpathlineto{\pgfqpoint{0.676061in}{3.973333in}}%
\pgfpathlineto{\pgfqpoint{0.671074in}{3.983728in}}%
\pgfpathlineto{\pgfqpoint{0.657198in}{3.973333in}}%
\pgfpathclose%
\pgfusepath{fill}%
\end{pgfscope}%
\begin{pgfscope}%
\pgfpathrectangle{\pgfqpoint{0.800000in}{0.528000in}}{\pgfqpoint{1.963636in}{3.696000in}} %
\pgfusepath{clip}%
\pgfsetbuttcap%
\pgfsetroundjoin%
\definecolor{currentfill}{rgb}{0.050383,0.029803,0.527975}%
\pgfsetfillcolor{currentfill}%
\pgfsetlinewidth{0.000000pt}%
\definecolor{currentstroke}{rgb}{0.000000,0.000000,0.000000}%
\pgfsetstrokecolor{currentstroke}%
\pgfsetdash{}{0pt}%
\pgfpathmoveto{\pgfqpoint{0.621488in}{4.051517in}}%
\pgfpathlineto{\pgfqpoint{0.626046in}{4.080000in}}%
\pgfpathlineto{\pgfqpoint{0.621488in}{4.089505in}}%
\pgfpathlineto{\pgfqpoint{0.607479in}{4.080000in}}%
\pgfpathclose%
\pgfusepath{fill}%
\end{pgfscope}%
\begin{pgfscope}%
\pgfpathrectangle{\pgfqpoint{0.800000in}{0.528000in}}{\pgfqpoint{1.963636in}{3.696000in}} %
\pgfusepath{clip}%
\pgfsetbuttcap%
\pgfsetroundjoin%
\definecolor{currentfill}{rgb}{0.050383,0.029803,0.527975}%
\pgfsetfillcolor{currentfill}%
\pgfsetlinewidth{0.000000pt}%
\definecolor{currentstroke}{rgb}{0.000000,0.000000,0.000000}%
\pgfsetstrokecolor{currentstroke}%
\pgfsetdash{}{0pt}%
\pgfpathmoveto{\pgfqpoint{0.571901in}{4.158116in}}%
\pgfpathlineto{\pgfqpoint{0.576029in}{4.186667in}}%
\pgfpathlineto{\pgfqpoint{0.571901in}{4.195277in}}%
\pgfpathlineto{\pgfqpoint{0.557816in}{4.186667in}}%
\pgfpathclose%
\pgfusepath{fill}%
\end{pgfscope}%
\begin{pgfscope}%
\pgfpathrectangle{\pgfqpoint{0.800000in}{0.528000in}}{\pgfqpoint{1.963636in}{3.696000in}} %
\pgfusepath{clip}%
\pgfsetbuttcap%
\pgfsetroundjoin%
\definecolor{currentfill}{rgb}{0.050383,0.029803,0.527975}%
\pgfsetfillcolor{currentfill}%
\pgfsetlinewidth{0.000000pt}%
\definecolor{currentstroke}{rgb}{0.000000,0.000000,0.000000}%
\pgfsetstrokecolor{currentstroke}%
\pgfsetdash{}{0pt}%
\pgfpathmoveto{\pgfqpoint{0.522314in}{4.264874in}}%
\pgfpathlineto{\pgfqpoint{0.526010in}{4.293333in}}%
\pgfpathlineto{\pgfqpoint{0.522314in}{4.301045in}}%
\pgfpathlineto{\pgfqpoint{0.508223in}{4.293333in}}%
\pgfpathclose%
\pgfusepath{fill}%
\end{pgfscope}%
\begin{pgfscope}%
\pgfpathrectangle{\pgfqpoint{0.800000in}{0.528000in}}{\pgfqpoint{1.963636in}{3.696000in}} %
\pgfusepath{clip}%
\pgfsetbuttcap%
\pgfsetroundjoin%
\definecolor{currentfill}{rgb}{0.050383,0.029803,0.527975}%
\pgfsetfillcolor{currentfill}%
\pgfsetlinewidth{0.000000pt}%
\definecolor{currentstroke}{rgb}{0.000000,0.000000,0.000000}%
\pgfsetstrokecolor{currentstroke}%
\pgfsetdash{}{0pt}%
\pgfpathmoveto{\pgfqpoint{0.472727in}{4.371837in}}%
\pgfpathlineto{\pgfqpoint{0.475990in}{4.400000in}}%
\pgfpathlineto{\pgfqpoint{0.472727in}{4.406809in}}%
\pgfpathlineto{\pgfqpoint{0.458723in}{4.400000in}}%
\pgfpathclose%
\pgfusepath{fill}%
\end{pgfscope}%
\begin{pgfscope}%
\pgfpathrectangle{\pgfqpoint{0.800000in}{0.528000in}}{\pgfqpoint{1.963636in}{3.696000in}} %
\pgfusepath{clip}%
\pgfsetbuttcap%
\pgfsetroundjoin%
\definecolor{currentfill}{rgb}{0.050383,0.029803,0.527975}%
\pgfsetfillcolor{currentfill}%
\pgfsetlinewidth{0.000000pt}%
\definecolor{currentstroke}{rgb}{0.000000,0.000000,0.000000}%
\pgfsetstrokecolor{currentstroke}%
\pgfsetdash{}{0pt}%
\pgfpathmoveto{\pgfqpoint{0.423140in}{4.479069in}}%
\pgfpathlineto{\pgfqpoint{0.425968in}{4.506667in}}%
\pgfpathlineto{\pgfqpoint{0.423140in}{4.512570in}}%
\pgfpathlineto{\pgfqpoint{0.409345in}{4.506667in}}%
\pgfpathclose%
\pgfusepath{fill}%
\end{pgfscope}%
\begin{pgfscope}%
\pgfpathrectangle{\pgfqpoint{0.800000in}{0.528000in}}{\pgfqpoint{1.963636in}{3.696000in}} %
\pgfusepath{clip}%
\pgfsetbuttcap%
\pgfsetroundjoin%
\definecolor{currentfill}{rgb}{0.050383,0.029803,0.527975}%
\pgfsetfillcolor{currentfill}%
\pgfsetlinewidth{0.000000pt}%
\definecolor{currentstroke}{rgb}{0.000000,0.000000,0.000000}%
\pgfsetstrokecolor{currentstroke}%
\pgfsetdash{}{0pt}%
\pgfpathmoveto{\pgfqpoint{0.373554in}{4.586666in}}%
\pgfpathlineto{\pgfqpoint{0.375945in}{4.613333in}}%
\pgfpathlineto{\pgfqpoint{0.373554in}{4.618326in}}%
\pgfpathlineto{\pgfqpoint{0.360136in}{4.613333in}}%
\pgfpathclose%
\pgfusepath{fill}%
\end{pgfscope}%
\begin{pgfscope}%
\pgfpathrectangle{\pgfqpoint{0.800000in}{0.528000in}}{\pgfqpoint{1.963636in}{3.696000in}} %
\pgfusepath{clip}%
\pgfsetbuttcap%
\pgfsetroundjoin%
\definecolor{currentfill}{rgb}{0.050383,0.029803,0.527975}%
\pgfsetfillcolor{currentfill}%
\pgfsetlinewidth{0.000000pt}%
\definecolor{currentstroke}{rgb}{0.000000,0.000000,0.000000}%
\pgfsetstrokecolor{currentstroke}%
\pgfsetdash{}{0pt}%
\pgfpathmoveto{\pgfqpoint{0.323967in}{4.694773in}}%
\pgfpathlineto{\pgfqpoint{0.325920in}{4.720000in}}%
\pgfpathlineto{\pgfqpoint{0.323967in}{4.724079in}}%
\pgfpathlineto{\pgfqpoint{0.311168in}{4.720000in}}%
\pgfpathclose%
\pgfusepath{fill}%
\end{pgfscope}%
\begin{pgfscope}%
\pgfpathrectangle{\pgfqpoint{0.800000in}{0.528000in}}{\pgfqpoint{1.963636in}{3.696000in}} %
\pgfusepath{clip}%
\pgfsetbuttcap%
\pgfsetroundjoin%
\definecolor{currentfill}{rgb}{0.050383,0.029803,0.527975}%
\pgfsetfillcolor{currentfill}%
\pgfsetlinewidth{0.000000pt}%
\definecolor{currentstroke}{rgb}{0.000000,0.000000,0.000000}%
\pgfsetstrokecolor{currentstroke}%
\pgfsetdash{}{0pt}%
\pgfpathmoveto{\pgfqpoint{0.274380in}{4.803623in}}%
\pgfpathlineto{\pgfqpoint{0.275894in}{4.826667in}}%
\pgfpathlineto{\pgfqpoint{0.274380in}{4.829830in}}%
\pgfpathlineto{\pgfqpoint{0.262560in}{4.826667in}}%
\pgfpathclose%
\pgfusepath{fill}%
\end{pgfscope}%
\begin{pgfscope}%
\pgfpathrectangle{\pgfqpoint{0.800000in}{0.528000in}}{\pgfqpoint{1.963636in}{3.696000in}} %
\pgfusepath{clip}%
\pgfsetbuttcap%
\pgfsetroundjoin%
\definecolor{currentfill}{rgb}{0.050383,0.029803,0.527975}%
\pgfsetfillcolor{currentfill}%
\pgfsetlinewidth{0.000000pt}%
\definecolor{currentstroke}{rgb}{0.000000,0.000000,0.000000}%
\pgfsetstrokecolor{currentstroke}%
\pgfsetdash{}{0pt}%
\pgfpathmoveto{\pgfqpoint{0.224793in}{4.913622in}}%
\pgfpathlineto{\pgfqpoint{0.225867in}{4.933333in}}%
\pgfpathlineto{\pgfqpoint{0.224793in}{4.935577in}}%
\pgfpathlineto{\pgfqpoint{0.214528in}{4.933333in}}%
\pgfpathclose%
\pgfusepath{fill}%
\end{pgfscope}%
\begin{pgfscope}%
\pgfpathrectangle{\pgfqpoint{0.800000in}{0.528000in}}{\pgfqpoint{1.963636in}{3.696000in}} %
\pgfusepath{clip}%
\pgfsetbuttcap%
\pgfsetroundjoin%
\definecolor{currentfill}{rgb}{0.050383,0.029803,0.527975}%
\pgfsetfillcolor{currentfill}%
\pgfsetlinewidth{0.000000pt}%
\definecolor{currentstroke}{rgb}{0.000000,0.000000,0.000000}%
\pgfsetstrokecolor{currentstroke}%
\pgfsetdash{}{0pt}%
\pgfpathmoveto{\pgfqpoint{0.175207in}{5.025529in}}%
\pgfpathlineto{\pgfqpoint{0.175839in}{5.040000in}}%
\pgfpathlineto{\pgfqpoint{0.175207in}{5.041322in}}%
\pgfpathlineto{\pgfqpoint{0.167501in}{5.040000in}}%
\pgfpathclose%
\pgfusepath{fill}%
\end{pgfscope}%
\begin{pgfscope}%
\pgfpathrectangle{\pgfqpoint{0.800000in}{0.528000in}}{\pgfqpoint{1.963636in}{3.696000in}} %
\pgfusepath{clip}%
\pgfsetbuttcap%
\pgfsetroundjoin%
\definecolor{currentfill}{rgb}{0.050383,0.029803,0.527975}%
\pgfsetfillcolor{currentfill}%
\pgfsetlinewidth{0.000000pt}%
\definecolor{currentstroke}{rgb}{0.000000,0.000000,0.000000}%
\pgfsetstrokecolor{currentstroke}%
\pgfsetdash{}{0pt}%
\pgfpathmoveto{\pgfqpoint{0.125620in}{5.140973in}}%
\pgfpathlineto{\pgfqpoint{0.125810in}{5.146667in}}%
\pgfpathlineto{\pgfqpoint{0.125620in}{5.147064in}}%
\pgfpathlineto{\pgfqpoint{0.122477in}{5.146667in}}%
\pgfpathclose%
\pgfusepath{fill}%
\end{pgfscope}%
\begin{pgfscope}%
\pgfpathrectangle{\pgfqpoint{0.800000in}{0.528000in}}{\pgfqpoint{1.963636in}{3.696000in}} %
\pgfusepath{clip}%
\pgfsetbuttcap%
\pgfsetroundjoin%
\definecolor{currentfill}{rgb}{0.050383,0.029803,0.527975}%
\pgfsetfillcolor{currentfill}%
\pgfsetlinewidth{0.000000pt}%
\definecolor{currentstroke}{rgb}{0.000000,0.000000,0.000000}%
\pgfsetstrokecolor{currentstroke}%
\pgfsetdash{}{0pt}%
\pgfpathmoveto{\pgfqpoint{-0.221488in}{5.786313in}}%
\pgfpathlineto{\pgfqpoint{-0.218639in}{5.786667in}}%
\pgfpathlineto{\pgfqpoint{-0.221488in}{5.791816in}}%
\pgfpathlineto{\pgfqpoint{-0.221657in}{5.786667in}}%
\pgfpathclose%
\pgfusepath{fill}%
\end{pgfscope}%
\begin{pgfscope}%
\pgfpathrectangle{\pgfqpoint{0.800000in}{0.528000in}}{\pgfqpoint{1.963636in}{3.696000in}} %
\pgfusepath{clip}%
\pgfsetbuttcap%
\pgfsetroundjoin%
\definecolor{currentfill}{rgb}{0.050383,0.029803,0.527975}%
\pgfsetfillcolor{currentfill}%
\pgfsetlinewidth{0.000000pt}%
\definecolor{currentstroke}{rgb}{0.000000,0.000000,0.000000}%
\pgfsetstrokecolor{currentstroke}%
\pgfsetdash{}{0pt}%
\pgfpathmoveto{\pgfqpoint{-0.271074in}{5.892055in}}%
\pgfpathlineto{\pgfqpoint{-0.263525in}{5.893333in}}%
\pgfpathlineto{\pgfqpoint{-0.271074in}{5.907492in}}%
\pgfpathlineto{\pgfqpoint{-0.271686in}{5.893333in}}%
\pgfpathclose%
\pgfusepath{fill}%
\end{pgfscope}%
\begin{pgfscope}%
\pgfpathrectangle{\pgfqpoint{0.800000in}{0.528000in}}{\pgfqpoint{1.963636in}{3.696000in}} %
\pgfusepath{clip}%
\pgfsetbuttcap%
\pgfsetroundjoin%
\definecolor{currentfill}{rgb}{0.050383,0.029803,0.527975}%
\pgfsetfillcolor{currentfill}%
\pgfsetlinewidth{0.000000pt}%
\definecolor{currentstroke}{rgb}{0.000000,0.000000,0.000000}%
\pgfsetstrokecolor{currentstroke}%
\pgfsetdash{}{0pt}%
\pgfpathmoveto{\pgfqpoint{-0.320661in}{5.997799in}}%
\pgfpathlineto{\pgfqpoint{-0.310488in}{6.000000in}}%
\pgfpathlineto{\pgfqpoint{-0.320661in}{6.019519in}}%
\pgfpathlineto{\pgfqpoint{-0.321714in}{6.000000in}}%
\pgfpathclose%
\pgfusepath{fill}%
\end{pgfscope}%
\begin{pgfscope}%
\pgfpathrectangle{\pgfqpoint{0.800000in}{0.528000in}}{\pgfqpoint{1.963636in}{3.696000in}} %
\pgfusepath{clip}%
\pgfsetbuttcap%
\pgfsetroundjoin%
\definecolor{currentfill}{rgb}{0.050383,0.029803,0.527975}%
\pgfsetfillcolor{currentfill}%
\pgfsetlinewidth{0.000000pt}%
\definecolor{currentstroke}{rgb}{0.000000,0.000000,0.000000}%
\pgfsetstrokecolor{currentstroke}%
\pgfsetdash{}{0pt}%
\pgfpathmoveto{\pgfqpoint{-0.370248in}{6.103546in}}%
\pgfpathlineto{\pgfqpoint{-0.358485in}{6.106667in}}%
\pgfpathlineto{\pgfqpoint{-0.370248in}{6.129585in}}%
\pgfpathlineto{\pgfqpoint{-0.371742in}{6.106667in}}%
\pgfpathclose%
\pgfusepath{fill}%
\end{pgfscope}%
\begin{pgfscope}%
\pgfpathrectangle{\pgfqpoint{0.800000in}{0.528000in}}{\pgfqpoint{1.963636in}{3.696000in}} %
\pgfusepath{clip}%
\pgfsetbuttcap%
\pgfsetroundjoin%
\definecolor{currentfill}{rgb}{0.050383,0.029803,0.527975}%
\pgfsetfillcolor{currentfill}%
\pgfsetlinewidth{0.000000pt}%
\definecolor{currentstroke}{rgb}{0.000000,0.000000,0.000000}%
\pgfsetstrokecolor{currentstroke}%
\pgfsetdash{}{0pt}%
\pgfpathmoveto{\pgfqpoint{-0.419835in}{6.209296in}}%
\pgfpathlineto{\pgfqpoint{-0.407072in}{6.213333in}}%
\pgfpathlineto{\pgfqpoint{-0.419835in}{6.238478in}}%
\pgfpathlineto{\pgfqpoint{-0.421768in}{6.213333in}}%
\pgfpathclose%
\pgfusepath{fill}%
\end{pgfscope}%
\begin{pgfscope}%
\pgfpathrectangle{\pgfqpoint{0.800000in}{0.528000in}}{\pgfqpoint{1.963636in}{3.696000in}} %
\pgfusepath{clip}%
\pgfsetbuttcap%
\pgfsetroundjoin%
\definecolor{currentfill}{rgb}{0.050383,0.029803,0.527975}%
\pgfsetfillcolor{currentfill}%
\pgfsetlinewidth{0.000000pt}%
\definecolor{currentstroke}{rgb}{0.000000,0.000000,0.000000}%
\pgfsetstrokecolor{currentstroke}%
\pgfsetdash{}{0pt}%
\pgfpathmoveto{\pgfqpoint{-0.469421in}{6.315050in}}%
\pgfpathlineto{\pgfqpoint{-0.456027in}{6.320000in}}%
\pgfpathlineto{\pgfqpoint{-0.469421in}{6.346613in}}%
\pgfpathlineto{\pgfqpoint{-0.471792in}{6.320000in}}%
\pgfpathclose%
\pgfusepath{fill}%
\end{pgfscope}%
\begin{pgfscope}%
\pgfpathrectangle{\pgfqpoint{0.800000in}{0.528000in}}{\pgfqpoint{1.963636in}{3.696000in}} %
\pgfusepath{clip}%
\pgfsetbuttcap%
\pgfsetroundjoin%
\definecolor{currentfill}{rgb}{0.050383,0.029803,0.527975}%
\pgfsetfillcolor{currentfill}%
\pgfsetlinewidth{0.000000pt}%
\definecolor{currentstroke}{rgb}{0.000000,0.000000,0.000000}%
\pgfsetstrokecolor{currentstroke}%
\pgfsetdash{}{0pt}%
\pgfpathmoveto{\pgfqpoint{-0.519008in}{6.420806in}}%
\pgfpathlineto{\pgfqpoint{-0.505226in}{6.426667in}}%
\pgfpathlineto{\pgfqpoint{-0.519008in}{6.454230in}}%
\pgfpathlineto{\pgfqpoint{-0.521816in}{6.426667in}}%
\pgfpathclose%
\pgfusepath{fill}%
\end{pgfscope}%
\begin{pgfscope}%
\pgfpathrectangle{\pgfqpoint{0.800000in}{0.528000in}}{\pgfqpoint{1.963636in}{3.696000in}} %
\pgfusepath{clip}%
\pgfsetbuttcap%
\pgfsetroundjoin%
\definecolor{currentfill}{rgb}{0.050383,0.029803,0.527975}%
\pgfsetfillcolor{currentfill}%
\pgfsetlinewidth{0.000000pt}%
\definecolor{currentstroke}{rgb}{0.000000,0.000000,0.000000}%
\pgfsetstrokecolor{currentstroke}%
\pgfsetdash{}{0pt}%
\pgfpathmoveto{\pgfqpoint{-0.568595in}{6.526566in}}%
\pgfpathlineto{\pgfqpoint{-0.554597in}{6.533333in}}%
\pgfpathlineto{\pgfqpoint{-0.568595in}{6.561477in}}%
\pgfpathlineto{\pgfqpoint{-0.571838in}{6.533333in}}%
\pgfpathclose%
\pgfusepath{fill}%
\end{pgfscope}%
\begin{pgfscope}%
\pgfpathrectangle{\pgfqpoint{0.800000in}{0.528000in}}{\pgfqpoint{1.963636in}{3.696000in}} %
\pgfusepath{clip}%
\pgfsetbuttcap%
\pgfsetroundjoin%
\definecolor{currentfill}{rgb}{0.050383,0.029803,0.527975}%
\pgfsetfillcolor{currentfill}%
\pgfsetlinewidth{0.000000pt}%
\definecolor{currentstroke}{rgb}{0.000000,0.000000,0.000000}%
\pgfsetstrokecolor{currentstroke}%
\pgfsetdash{}{0pt}%
\pgfpathmoveto{\pgfqpoint{-0.618182in}{6.632330in}}%
\pgfpathlineto{\pgfqpoint{-0.604093in}{6.640000in}}%
\pgfpathlineto{\pgfqpoint{-0.618182in}{6.668451in}}%
\pgfpathlineto{\pgfqpoint{-0.621858in}{6.640000in}}%
\pgfpathclose%
\pgfusepath{fill}%
\end{pgfscope}%
\begin{pgfscope}%
\pgfpathrectangle{\pgfqpoint{0.800000in}{0.528000in}}{\pgfqpoint{1.963636in}{3.696000in}} %
\pgfusepath{clip}%
\pgfsetbuttcap%
\pgfsetroundjoin%
\definecolor{currentfill}{rgb}{0.050383,0.029803,0.527975}%
\pgfsetfillcolor{currentfill}%
\pgfsetlinewidth{0.000000pt}%
\definecolor{currentstroke}{rgb}{0.000000,0.000000,0.000000}%
\pgfsetstrokecolor{currentstroke}%
\pgfsetdash{}{0pt}%
\pgfpathmoveto{\pgfqpoint{-0.667769in}{6.738098in}}%
\pgfpathlineto{\pgfqpoint{-0.653681in}{6.746667in}}%
\pgfpathlineto{\pgfqpoint{-0.667769in}{6.775217in}}%
\pgfpathlineto{\pgfqpoint{-0.671877in}{6.746667in}}%
\pgfpathclose%
\pgfusepath{fill}%
\end{pgfscope}%
\begin{pgfscope}%
\pgfpathrectangle{\pgfqpoint{0.800000in}{0.528000in}}{\pgfqpoint{1.963636in}{3.696000in}} %
\pgfusepath{clip}%
\pgfsetbuttcap%
\pgfsetroundjoin%
\definecolor{currentfill}{rgb}{0.050383,0.029803,0.527975}%
\pgfsetfillcolor{currentfill}%
\pgfsetlinewidth{0.000000pt}%
\definecolor{currentstroke}{rgb}{0.000000,0.000000,0.000000}%
\pgfsetstrokecolor{currentstroke}%
\pgfsetdash{}{0pt}%
\pgfpathmoveto{\pgfqpoint{-0.717355in}{6.843870in}}%
\pgfpathlineto{\pgfqpoint{-0.703342in}{6.853333in}}%
\pgfpathlineto{\pgfqpoint{-0.717355in}{6.881822in}}%
\pgfpathlineto{\pgfqpoint{-0.721894in}{6.853333in}}%
\pgfpathclose%
\pgfusepath{fill}%
\end{pgfscope}%
\begin{pgfscope}%
\pgfpathrectangle{\pgfqpoint{0.800000in}{0.528000in}}{\pgfqpoint{1.963636in}{3.696000in}} %
\pgfusepath{clip}%
\pgfsetbuttcap%
\pgfsetroundjoin%
\definecolor{currentfill}{rgb}{0.050383,0.029803,0.527975}%
\pgfsetfillcolor{currentfill}%
\pgfsetlinewidth{0.000000pt}%
\definecolor{currentstroke}{rgb}{0.000000,0.000000,0.000000}%
\pgfsetstrokecolor{currentstroke}%
\pgfsetdash{}{0pt}%
\pgfpathmoveto{\pgfqpoint{-0.766942in}{6.949646in}}%
\pgfpathlineto{\pgfqpoint{-0.753059in}{6.960000in}}%
\pgfpathlineto{\pgfqpoint{-0.766942in}{6.988300in}}%
\pgfpathlineto{\pgfqpoint{-0.771909in}{6.960000in}}%
\pgfpathclose%
\pgfusepath{fill}%
\end{pgfscope}%
\begin{pgfscope}%
\pgfpathrectangle{\pgfqpoint{0.800000in}{0.528000in}}{\pgfqpoint{1.963636in}{3.696000in}} %
\pgfusepath{clip}%
\pgfsetbuttcap%
\pgfsetroundjoin%
\definecolor{currentfill}{rgb}{0.050383,0.029803,0.527975}%
\pgfsetfillcolor{currentfill}%
\pgfsetlinewidth{0.000000pt}%
\definecolor{currentstroke}{rgb}{0.000000,0.000000,0.000000}%
\pgfsetstrokecolor{currentstroke}%
\pgfsetdash{}{0pt}%
\pgfpathmoveto{\pgfqpoint{-0.816529in}{7.055428in}}%
\pgfpathlineto{\pgfqpoint{-0.802821in}{7.066667in}}%
\pgfpathlineto{\pgfqpoint{-0.816529in}{7.094674in}}%
\pgfpathlineto{\pgfqpoint{-0.821922in}{7.066667in}}%
\pgfpathclose%
\pgfusepath{fill}%
\end{pgfscope}%
\begin{pgfscope}%
\pgfpathrectangle{\pgfqpoint{0.800000in}{0.528000in}}{\pgfqpoint{1.963636in}{3.696000in}} %
\pgfusepath{clip}%
\pgfsetbuttcap%
\pgfsetroundjoin%
\definecolor{currentfill}{rgb}{0.050383,0.029803,0.527975}%
\pgfsetfillcolor{currentfill}%
\pgfsetlinewidth{0.000000pt}%
\definecolor{currentstroke}{rgb}{0.000000,0.000000,0.000000}%
\pgfsetstrokecolor{currentstroke}%
\pgfsetdash{}{0pt}%
\pgfpathmoveto{\pgfqpoint{-0.866116in}{7.161216in}}%
\pgfpathlineto{\pgfqpoint{-0.852619in}{7.173333in}}%
\pgfpathlineto{\pgfqpoint{-0.866116in}{7.200964in}}%
\pgfpathlineto{\pgfqpoint{-0.871933in}{7.173333in}}%
\pgfpathclose%
\pgfusepath{fill}%
\end{pgfscope}%
\begin{pgfscope}%
\pgfpathrectangle{\pgfqpoint{0.800000in}{0.528000in}}{\pgfqpoint{1.963636in}{3.696000in}} %
\pgfusepath{clip}%
\pgfsetbuttcap%
\pgfsetroundjoin%
\definecolor{currentfill}{rgb}{0.050383,0.029803,0.527975}%
\pgfsetfillcolor{currentfill}%
\pgfsetlinewidth{0.000000pt}%
\definecolor{currentstroke}{rgb}{0.000000,0.000000,0.000000}%
\pgfsetstrokecolor{currentstroke}%
\pgfsetdash{}{0pt}%
\pgfpathmoveto{\pgfqpoint{-0.915702in}{7.267009in}}%
\pgfpathlineto{\pgfqpoint{-0.902448in}{7.280000in}}%
\pgfpathlineto{\pgfqpoint{-0.915702in}{7.307184in}}%
\pgfpathlineto{\pgfqpoint{-0.921941in}{7.280000in}}%
\pgfpathclose%
\pgfusepath{fill}%
\end{pgfscope}%
\begin{pgfscope}%
\pgfpathrectangle{\pgfqpoint{0.800000in}{0.528000in}}{\pgfqpoint{1.963636in}{3.696000in}} %
\pgfusepath{clip}%
\pgfsetbuttcap%
\pgfsetroundjoin%
\definecolor{currentfill}{rgb}{0.050383,0.029803,0.527975}%
\pgfsetfillcolor{currentfill}%
\pgfsetlinewidth{0.000000pt}%
\definecolor{currentstroke}{rgb}{0.000000,0.000000,0.000000}%
\pgfsetstrokecolor{currentstroke}%
\pgfsetdash{}{0pt}%
\pgfpathmoveto{\pgfqpoint{-0.965289in}{7.372808in}}%
\pgfpathlineto{\pgfqpoint{-0.952301in}{7.386667in}}%
\pgfpathlineto{\pgfqpoint{-0.965289in}{7.413346in}}%
\pgfpathlineto{\pgfqpoint{-0.971947in}{7.386667in}}%
\pgfpathclose%
\pgfusepath{fill}%
\end{pgfscope}%
\begin{pgfscope}%
\pgfpathrectangle{\pgfqpoint{0.800000in}{0.528000in}}{\pgfqpoint{1.963636in}{3.696000in}} %
\pgfusepath{clip}%
\pgfsetbuttcap%
\pgfsetroundjoin%
\definecolor{currentfill}{rgb}{0.050383,0.029803,0.527975}%
\pgfsetfillcolor{currentfill}%
\pgfsetlinewidth{0.000000pt}%
\definecolor{currentstroke}{rgb}{0.000000,0.000000,0.000000}%
\pgfsetstrokecolor{currentstroke}%
\pgfsetdash{}{0pt}%
\pgfpathmoveto{\pgfqpoint{-1.014876in}{7.478615in}}%
\pgfpathlineto{\pgfqpoint{-1.002176in}{7.493333in}}%
\pgfpathlineto{\pgfqpoint{-1.014876in}{7.519458in}}%
\pgfpathlineto{\pgfqpoint{-1.021950in}{7.493333in}}%
\pgfpathclose%
\pgfpathmoveto{\pgfqpoint{-1.014880in}{7.493333in}}%
\pgfpathlineto{\pgfqpoint{-1.014876in}{7.493349in}}%
\pgfpathlineto{\pgfqpoint{-1.014868in}{7.493333in}}%
\pgfpathlineto{\pgfqpoint{-1.014876in}{7.493325in}}%
\pgfpathclose%
\pgfusepath{fill}%
\end{pgfscope}%
\begin{pgfscope}%
\pgfpathrectangle{\pgfqpoint{0.800000in}{0.528000in}}{\pgfqpoint{1.963636in}{3.696000in}} %
\pgfusepath{clip}%
\pgfsetbuttcap%
\pgfsetroundjoin%
\definecolor{currentfill}{rgb}{0.050383,0.029803,0.527975}%
\pgfsetfillcolor{currentfill}%
\pgfsetlinewidth{0.000000pt}%
\definecolor{currentstroke}{rgb}{0.000000,0.000000,0.000000}%
\pgfsetstrokecolor{currentstroke}%
\pgfsetdash{}{0pt}%
\pgfpathmoveto{\pgfqpoint{-1.064463in}{7.584428in}}%
\pgfpathlineto{\pgfqpoint{-1.052069in}{7.600000in}}%
\pgfpathlineto{\pgfqpoint{-1.064463in}{7.625528in}}%
\pgfpathlineto{\pgfqpoint{-1.071950in}{7.600000in}}%
\pgfpathclose%
\pgfpathmoveto{\pgfqpoint{-1.064666in}{7.600000in}}%
\pgfpathlineto{\pgfqpoint{-1.064463in}{7.600694in}}%
\pgfpathlineto{\pgfqpoint{-1.064126in}{7.600000in}}%
\pgfpathlineto{\pgfqpoint{-1.064463in}{7.599577in}}%
\pgfpathclose%
\pgfusepath{fill}%
\end{pgfscope}%
\begin{pgfscope}%
\pgfpathrectangle{\pgfqpoint{0.800000in}{0.528000in}}{\pgfqpoint{1.963636in}{3.696000in}} %
\pgfusepath{clip}%
\pgfsetbuttcap%
\pgfsetroundjoin%
\definecolor{currentfill}{rgb}{0.050383,0.029803,0.527975}%
\pgfsetfillcolor{currentfill}%
\pgfsetlinewidth{0.000000pt}%
\definecolor{currentstroke}{rgb}{0.000000,0.000000,0.000000}%
\pgfsetstrokecolor{currentstroke}%
\pgfsetdash{}{0pt}%
\pgfpathmoveto{\pgfqpoint{-1.114050in}{7.690251in}}%
\pgfpathlineto{\pgfqpoint{-1.101977in}{7.706667in}}%
\pgfpathlineto{\pgfqpoint{-1.114050in}{7.731562in}}%
\pgfpathlineto{\pgfqpoint{-1.121946in}{7.706667in}}%
\pgfpathclose%
\pgfpathmoveto{\pgfqpoint{-1.114436in}{7.706667in}}%
\pgfpathlineto{\pgfqpoint{-1.114050in}{7.707884in}}%
\pgfpathlineto{\pgfqpoint{-1.113459in}{7.706667in}}%
\pgfpathlineto{\pgfqpoint{-1.114050in}{7.705864in}}%
\pgfpathclose%
\pgfusepath{fill}%
\end{pgfscope}%
\begin{pgfscope}%
\pgfpathrectangle{\pgfqpoint{0.800000in}{0.528000in}}{\pgfqpoint{1.963636in}{3.696000in}} %
\pgfusepath{clip}%
\pgfsetbuttcap%
\pgfsetroundjoin%
\definecolor{currentfill}{rgb}{0.050383,0.029803,0.527975}%
\pgfsetfillcolor{currentfill}%
\pgfsetlinewidth{0.000000pt}%
\definecolor{currentstroke}{rgb}{0.000000,0.000000,0.000000}%
\pgfsetstrokecolor{currentstroke}%
\pgfsetdash{}{0pt}%
\pgfpathmoveto{\pgfqpoint{-1.151899in}{7.813333in}}%
\pgfpathlineto{\pgfqpoint{-1.163636in}{7.837564in}}%
\pgfpathlineto{\pgfqpoint{-1.163636in}{7.814940in}}%
\pgfpathlineto{\pgfqpoint{-1.162858in}{7.813333in}}%
\pgfpathlineto{\pgfqpoint{-1.163636in}{7.812190in}}%
\pgfpathlineto{\pgfqpoint{-1.163636in}{7.796082in}}%
\pgfpathclose%
\pgfusepath{fill}%
\end{pgfscope}%
\begin{pgfscope}%
\pgfpathrectangle{\pgfqpoint{0.800000in}{0.528000in}}{\pgfqpoint{1.963636in}{3.696000in}} %
\pgfusepath{clip}%
\pgfsetbuttcap%
\pgfsetroundjoin%
\definecolor{currentfill}{rgb}{0.050383,0.029803,0.527975}%
\pgfsetfillcolor{currentfill}%
\pgfsetlinewidth{0.000000pt}%
\definecolor{currentstroke}{rgb}{0.000000,0.000000,0.000000}%
\pgfsetstrokecolor{currentstroke}%
\pgfsetdash{}{0pt}%
\pgfpathmoveto{\pgfqpoint{3.695868in}{-2.577413in}}%
\pgfpathlineto{\pgfqpoint{3.725317in}{-2.640000in}}%
\pgfpathlineto{\pgfqpoint{3.745455in}{-2.640000in}}%
\pgfpathlineto{\pgfqpoint{3.745455in}{-2.557564in}}%
\pgfpathlineto{\pgfqpoint{3.733717in}{-2.533333in}}%
\pgfpathlineto{\pgfqpoint{3.745455in}{-2.516082in}}%
\pgfpathlineto{\pgfqpoint{3.745455in}{-2.426667in}}%
\pgfpathlineto{\pgfqpoint{3.745455in}{-2.400242in}}%
\pgfpathlineto{\pgfqpoint{3.707670in}{-2.320000in}}%
\pgfpathlineto{\pgfqpoint{3.695868in}{-2.295227in}}%
\pgfpathlineto{\pgfqpoint{3.657302in}{-2.213333in}}%
\pgfpathlineto{\pgfqpoint{3.646281in}{-2.190196in}}%
\pgfpathlineto{\pgfqpoint{3.606943in}{-2.106667in}}%
\pgfpathlineto{\pgfqpoint{3.596694in}{-2.085151in}}%
\pgfpathlineto{\pgfqpoint{3.556589in}{-2.000000in}}%
\pgfpathlineto{\pgfqpoint{3.547107in}{-1.980092in}}%
\pgfpathlineto{\pgfqpoint{3.506243in}{-1.893333in}}%
\pgfpathlineto{\pgfqpoint{3.497521in}{-1.875020in}}%
\pgfpathlineto{\pgfqpoint{3.455903in}{-1.786667in}}%
\pgfpathlineto{\pgfqpoint{3.447934in}{-1.769933in}}%
\pgfpathlineto{\pgfqpoint{3.405569in}{-1.680000in}}%
\pgfpathlineto{\pgfqpoint{3.398347in}{-1.664835in}}%
\pgfpathlineto{\pgfqpoint{3.355241in}{-1.573333in}}%
\pgfpathlineto{\pgfqpoint{3.348760in}{-1.559723in}}%
\pgfpathlineto{\pgfqpoint{3.304919in}{-1.466667in}}%
\pgfpathlineto{\pgfqpoint{3.299174in}{-1.454600in}}%
\pgfpathlineto{\pgfqpoint{3.254603in}{-1.360000in}}%
\pgfpathlineto{\pgfqpoint{3.249587in}{-1.349465in}}%
\pgfpathlineto{\pgfqpoint{3.204292in}{-1.253333in}}%
\pgfpathlineto{\pgfqpoint{3.200000in}{-1.244319in}}%
\pgfpathlineto{\pgfqpoint{3.153986in}{-1.146667in}}%
\pgfpathlineto{\pgfqpoint{3.150413in}{-1.139162in}}%
\pgfpathlineto{\pgfqpoint{3.103686in}{-1.040000in}}%
\pgfpathlineto{\pgfqpoint{3.100826in}{-1.033994in}}%
\pgfpathlineto{\pgfqpoint{3.053390in}{-0.933333in}}%
\pgfpathlineto{\pgfqpoint{3.051240in}{-0.928816in}}%
\pgfpathlineto{\pgfqpoint{3.003099in}{-0.826667in}}%
\pgfpathlineto{\pgfqpoint{3.001653in}{-0.823628in}}%
\pgfpathlineto{\pgfqpoint{2.952813in}{-0.720000in}}%
\pgfpathlineto{\pgfqpoint{2.952066in}{-0.718430in}}%
\pgfpathlineto{\pgfqpoint{2.902532in}{-0.613333in}}%
\pgfpathlineto{\pgfqpoint{2.902479in}{-0.613223in}}%
\pgfpathlineto{\pgfqpoint{2.852893in}{-0.509468in}}%
\pgfpathlineto{\pgfqpoint{2.851579in}{-0.506667in}}%
\pgfpathlineto{\pgfqpoint{2.803306in}{-0.405754in}}%
\pgfpathlineto{\pgfqpoint{2.800607in}{-0.400000in}}%
\pgfpathlineto{\pgfqpoint{2.753719in}{-0.301965in}}%
\pgfpathlineto{\pgfqpoint{2.749670in}{-0.293333in}}%
\pgfpathlineto{\pgfqpoint{2.704132in}{-0.198103in}}%
\pgfpathlineto{\pgfqpoint{2.698766in}{-0.186667in}}%
\pgfpathlineto{\pgfqpoint{2.654545in}{-0.094174in}}%
\pgfpathlineto{\pgfqpoint{2.647894in}{-0.080000in}}%
\pgfpathlineto{\pgfqpoint{2.604959in}{0.009820in}}%
\pgfpathlineto{\pgfqpoint{2.597052in}{0.026667in}}%
\pgfpathlineto{\pgfqpoint{2.555372in}{0.113874in}}%
\pgfpathlineto{\pgfqpoint{2.546237in}{0.133333in}}%
\pgfpathlineto{\pgfqpoint{2.505785in}{0.217986in}}%
\pgfpathlineto{\pgfqpoint{2.495450in}{0.240000in}}%
\pgfpathlineto{\pgfqpoint{2.456198in}{0.322152in}}%
\pgfpathlineto{\pgfqpoint{2.444687in}{0.346667in}}%
\pgfpathlineto{\pgfqpoint{2.406612in}{0.426370in}}%
\pgfpathlineto{\pgfqpoint{2.393949in}{0.453333in}}%
\pgfpathlineto{\pgfqpoint{2.357025in}{0.530637in}}%
\pgfpathlineto{\pgfqpoint{2.343233in}{0.560000in}}%
\pgfpathlineto{\pgfqpoint{2.307438in}{0.634950in}}%
\pgfpathlineto{\pgfqpoint{2.292539in}{0.666667in}}%
\pgfpathlineto{\pgfqpoint{2.257851in}{0.739308in}}%
\pgfpathlineto{\pgfqpoint{2.241865in}{0.773333in}}%
\pgfpathlineto{\pgfqpoint{2.208264in}{0.843709in}}%
\pgfpathlineto{\pgfqpoint{2.191212in}{0.880000in}}%
\pgfpathlineto{\pgfqpoint{2.158678in}{0.948149in}}%
\pgfpathlineto{\pgfqpoint{2.140577in}{0.986667in}}%
\pgfpathlineto{\pgfqpoint{2.109091in}{1.052628in}}%
\pgfpathlineto{\pgfqpoint{2.089959in}{1.093333in}}%
\pgfpathlineto{\pgfqpoint{2.059504in}{1.157143in}}%
\pgfpathlineto{\pgfqpoint{2.039359in}{1.200000in}}%
\pgfpathlineto{\pgfqpoint{2.009917in}{1.261694in}}%
\pgfpathlineto{\pgfqpoint{1.988775in}{1.306667in}}%
\pgfpathlineto{\pgfqpoint{1.960331in}{1.366278in}}%
\pgfpathlineto{\pgfqpoint{1.938207in}{1.413333in}}%
\pgfpathlineto{\pgfqpoint{1.910744in}{1.470894in}}%
\pgfpathlineto{\pgfqpoint{1.887654in}{1.520000in}}%
\pgfpathlineto{\pgfqpoint{1.861157in}{1.575540in}}%
\pgfpathlineto{\pgfqpoint{1.837114in}{1.626667in}}%
\pgfpathlineto{\pgfqpoint{1.811570in}{1.680216in}}%
\pgfpathlineto{\pgfqpoint{1.786589in}{1.733333in}}%
\pgfpathlineto{\pgfqpoint{1.761983in}{1.784920in}}%
\pgfpathlineto{\pgfqpoint{1.736076in}{1.840000in}}%
\pgfpathlineto{\pgfqpoint{1.712397in}{1.889651in}}%
\pgfpathlineto{\pgfqpoint{1.685576in}{1.946667in}}%
\pgfpathlineto{\pgfqpoint{1.662810in}{1.994407in}}%
\pgfpathlineto{\pgfqpoint{1.635089in}{2.053333in}}%
\pgfpathlineto{\pgfqpoint{1.613223in}{2.099188in}}%
\pgfpathlineto{\pgfqpoint{1.584612in}{2.160000in}}%
\pgfpathlineto{\pgfqpoint{1.563636in}{2.203993in}}%
\pgfpathlineto{\pgfqpoint{1.534147in}{2.266667in}}%
\pgfpathlineto{\pgfqpoint{1.514050in}{2.308821in}}%
\pgfpathlineto{\pgfqpoint{1.483692in}{2.373333in}}%
\pgfpathlineto{\pgfqpoint{1.464463in}{2.413671in}}%
\pgfpathlineto{\pgfqpoint{1.433248in}{2.480000in}}%
\pgfpathlineto{\pgfqpoint{1.414876in}{2.518542in}}%
\pgfpathlineto{\pgfqpoint{1.382814in}{2.586667in}}%
\pgfpathlineto{\pgfqpoint{1.365289in}{2.623434in}}%
\pgfpathlineto{\pgfqpoint{1.332389in}{2.693333in}}%
\pgfpathlineto{\pgfqpoint{1.315702in}{2.728345in}}%
\pgfpathlineto{\pgfqpoint{1.281973in}{2.800000in}}%
\pgfpathlineto{\pgfqpoint{1.266116in}{2.833274in}}%
\pgfpathlineto{\pgfqpoint{1.231566in}{2.906667in}}%
\pgfpathlineto{\pgfqpoint{1.216529in}{2.938222in}}%
\pgfpathlineto{\pgfqpoint{1.181167in}{3.013333in}}%
\pgfpathlineto{\pgfqpoint{1.166942in}{3.043188in}}%
\pgfpathlineto{\pgfqpoint{1.130777in}{3.120000in}}%
\pgfpathlineto{\pgfqpoint{1.117355in}{3.148170in}}%
\pgfpathlineto{\pgfqpoint{1.080395in}{3.226667in}}%
\pgfpathlineto{\pgfqpoint{1.067769in}{3.253169in}}%
\pgfpathlineto{\pgfqpoint{1.030020in}{3.333333in}}%
\pgfpathlineto{\pgfqpoint{1.018182in}{3.358183in}}%
\pgfpathlineto{\pgfqpoint{0.979653in}{3.440000in}}%
\pgfpathlineto{\pgfqpoint{0.968595in}{3.463213in}}%
\pgfpathlineto{\pgfqpoint{0.929292in}{3.546667in}}%
\pgfpathlineto{\pgfqpoint{0.919008in}{3.568257in}}%
\pgfpathlineto{\pgfqpoint{0.878939in}{3.653333in}}%
\pgfpathlineto{\pgfqpoint{0.869421in}{3.673315in}}%
\pgfpathlineto{\pgfqpoint{0.828592in}{3.760000in}}%
\pgfpathlineto{\pgfqpoint{0.819835in}{3.778388in}}%
\pgfpathlineto{\pgfqpoint{0.778252in}{3.866667in}}%
\pgfpathlineto{\pgfqpoint{0.770248in}{3.883473in}}%
\pgfpathlineto{\pgfqpoint{0.727918in}{3.973333in}}%
\pgfpathlineto{\pgfqpoint{0.720661in}{3.988571in}}%
\pgfpathlineto{\pgfqpoint{0.677590in}{4.080000in}}%
\pgfpathlineto{\pgfqpoint{0.671074in}{4.093682in}}%
\pgfpathlineto{\pgfqpoint{0.627267in}{4.186667in}}%
\pgfpathlineto{\pgfqpoint{0.621488in}{4.198805in}}%
\pgfpathlineto{\pgfqpoint{0.576951in}{4.293333in}}%
\pgfpathlineto{\pgfqpoint{0.571901in}{4.303939in}}%
\pgfpathlineto{\pgfqpoint{0.526640in}{4.400000in}}%
\pgfpathlineto{\pgfqpoint{0.522314in}{4.409085in}}%
\pgfpathlineto{\pgfqpoint{0.476334in}{4.506667in}}%
\pgfpathlineto{\pgfqpoint{0.472727in}{4.514242in}}%
\pgfpathlineto{\pgfqpoint{0.426033in}{4.613333in}}%
\pgfpathlineto{\pgfqpoint{0.423140in}{4.619409in}}%
\pgfpathlineto{\pgfqpoint{0.375737in}{4.720000in}}%
\pgfpathlineto{\pgfqpoint{0.373554in}{4.724587in}}%
\pgfpathlineto{\pgfqpoint{0.325446in}{4.826667in}}%
\pgfpathlineto{\pgfqpoint{0.323967in}{4.829774in}}%
\pgfpathlineto{\pgfqpoint{0.275160in}{4.933333in}}%
\pgfpathlineto{\pgfqpoint{0.274380in}{4.934971in}}%
\pgfpathlineto{\pgfqpoint{0.224878in}{5.040000in}}%
\pgfpathlineto{\pgfqpoint{0.224793in}{5.040178in}}%
\pgfpathlineto{\pgfqpoint{0.175207in}{5.144005in}}%
\pgfpathlineto{\pgfqpoint{0.173959in}{5.146667in}}%
\pgfpathlineto{\pgfqpoint{0.125620in}{5.247715in}}%
\pgfpathlineto{\pgfqpoint{0.122985in}{5.253333in}}%
\pgfpathlineto{\pgfqpoint{0.076033in}{5.351501in}}%
\pgfpathlineto{\pgfqpoint{0.072046in}{5.360000in}}%
\pgfpathlineto{\pgfqpoint{0.026446in}{5.455359in}}%
\pgfpathlineto{\pgfqpoint{0.021141in}{5.466667in}}%
\pgfpathlineto{\pgfqpoint{-0.023140in}{5.559285in}}%
\pgfpathlineto{\pgfqpoint{-0.029733in}{5.573333in}}%
\pgfpathlineto{\pgfqpoint{-0.072727in}{5.663276in}}%
\pgfpathlineto{\pgfqpoint{-0.080576in}{5.680000in}}%
\pgfpathlineto{\pgfqpoint{-0.122314in}{5.767327in}}%
\pgfpathlineto{\pgfqpoint{-0.131392in}{5.786667in}}%
\pgfpathlineto{\pgfqpoint{-0.171901in}{5.871437in}}%
\pgfpathlineto{\pgfqpoint{-0.182181in}{5.893333in}}%
\pgfpathlineto{\pgfqpoint{-0.221488in}{5.975600in}}%
\pgfpathlineto{\pgfqpoint{-0.232945in}{6.000000in}}%
\pgfpathlineto{\pgfqpoint{-0.271074in}{6.079816in}}%
\pgfpathlineto{\pgfqpoint{-0.283684in}{6.106667in}}%
\pgfpathlineto{\pgfqpoint{-0.320661in}{6.184081in}}%
\pgfpathlineto{\pgfqpoint{-0.334401in}{6.213333in}}%
\pgfpathlineto{\pgfqpoint{-0.370248in}{6.288392in}}%
\pgfpathlineto{\pgfqpoint{-0.385096in}{6.320000in}}%
\pgfpathlineto{\pgfqpoint{-0.419835in}{6.392748in}}%
\pgfpathlineto{\pgfqpoint{-0.435770in}{6.426667in}}%
\pgfpathlineto{\pgfqpoint{-0.469421in}{6.497146in}}%
\pgfpathlineto{\pgfqpoint{-0.486425in}{6.533333in}}%
\pgfpathlineto{\pgfqpoint{-0.519008in}{6.601585in}}%
\pgfpathlineto{\pgfqpoint{-0.537061in}{6.640000in}}%
\pgfpathlineto{\pgfqpoint{-0.568595in}{6.706062in}}%
\pgfpathlineto{\pgfqpoint{-0.587679in}{6.746667in}}%
\pgfpathlineto{\pgfqpoint{-0.618182in}{6.810576in}}%
\pgfpathlineto{\pgfqpoint{-0.638280in}{6.853333in}}%
\pgfpathlineto{\pgfqpoint{-0.667769in}{6.915125in}}%
\pgfpathlineto{\pgfqpoint{-0.688865in}{6.960000in}}%
\pgfpathlineto{\pgfqpoint{-0.717355in}{7.019707in}}%
\pgfpathlineto{\pgfqpoint{-0.739434in}{7.066667in}}%
\pgfpathlineto{\pgfqpoint{-0.766942in}{7.124322in}}%
\pgfpathlineto{\pgfqpoint{-0.789988in}{7.173333in}}%
\pgfpathlineto{\pgfqpoint{-0.816529in}{7.228967in}}%
\pgfpathlineto{\pgfqpoint{-0.840528in}{7.280000in}}%
\pgfpathlineto{\pgfqpoint{-0.866116in}{7.333641in}}%
\pgfpathlineto{\pgfqpoint{-0.891054in}{7.386667in}}%
\pgfpathlineto{\pgfqpoint{-0.915702in}{7.438344in}}%
\pgfpathlineto{\pgfqpoint{-0.941567in}{7.493333in}}%
\pgfpathlineto{\pgfqpoint{-0.965289in}{7.543073in}}%
\pgfpathlineto{\pgfqpoint{-0.992067in}{7.600000in}}%
\pgfpathlineto{\pgfqpoint{-1.014876in}{7.647829in}}%
\pgfpathlineto{\pgfqpoint{-1.042556in}{7.706667in}}%
\pgfpathlineto{\pgfqpoint{-1.064463in}{7.752609in}}%
\pgfpathlineto{\pgfqpoint{-1.093033in}{7.813333in}}%
\pgfpathlineto{\pgfqpoint{-1.114050in}{7.857413in}}%
\pgfpathlineto{\pgfqpoint{-1.143498in}{7.920000in}}%
\pgfpathlineto{\pgfqpoint{-1.163636in}{7.920000in}}%
\pgfpathlineto{\pgfqpoint{-1.163636in}{7.837564in}}%
\pgfpathlineto{\pgfqpoint{-1.151899in}{7.813333in}}%
\pgfpathlineto{\pgfqpoint{-1.163636in}{7.796082in}}%
\pgfpathlineto{\pgfqpoint{-1.163636in}{7.706667in}}%
\pgfpathlineto{\pgfqpoint{-1.163636in}{7.680242in}}%
\pgfpathlineto{\pgfqpoint{-1.125851in}{7.600000in}}%
\pgfpathlineto{\pgfqpoint{-1.114050in}{7.575227in}}%
\pgfpathlineto{\pgfqpoint{-1.075484in}{7.493333in}}%
\pgfpathlineto{\pgfqpoint{-1.064463in}{7.470196in}}%
\pgfpathlineto{\pgfqpoint{-1.025124in}{7.386667in}}%
\pgfpathlineto{\pgfqpoint{-1.014876in}{7.365151in}}%
\pgfpathlineto{\pgfqpoint{-0.974771in}{7.280000in}}%
\pgfpathlineto{\pgfqpoint{-0.965289in}{7.260092in}}%
\pgfpathlineto{\pgfqpoint{-0.924425in}{7.173333in}}%
\pgfpathlineto{\pgfqpoint{-0.915702in}{7.155020in}}%
\pgfpathlineto{\pgfqpoint{-0.874085in}{7.066667in}}%
\pgfpathlineto{\pgfqpoint{-0.866116in}{7.049933in}}%
\pgfpathlineto{\pgfqpoint{-0.823751in}{6.960000in}}%
\pgfpathlineto{\pgfqpoint{-0.816529in}{6.944835in}}%
\pgfpathlineto{\pgfqpoint{-0.773423in}{6.853333in}}%
\pgfpathlineto{\pgfqpoint{-0.766942in}{6.839723in}}%
\pgfpathlineto{\pgfqpoint{-0.723101in}{6.746667in}}%
\pgfpathlineto{\pgfqpoint{-0.717355in}{6.734600in}}%
\pgfpathlineto{\pgfqpoint{-0.672785in}{6.640000in}}%
\pgfpathlineto{\pgfqpoint{-0.667769in}{6.629465in}}%
\pgfpathlineto{\pgfqpoint{-0.622474in}{6.533333in}}%
\pgfpathlineto{\pgfqpoint{-0.618182in}{6.524319in}}%
\pgfpathlineto{\pgfqpoint{-0.572168in}{6.426667in}}%
\pgfpathlineto{\pgfqpoint{-0.568595in}{6.419162in}}%
\pgfpathlineto{\pgfqpoint{-0.521868in}{6.320000in}}%
\pgfpathlineto{\pgfqpoint{-0.519008in}{6.313994in}}%
\pgfpathlineto{\pgfqpoint{-0.471572in}{6.213333in}}%
\pgfpathlineto{\pgfqpoint{-0.469421in}{6.208816in}}%
\pgfpathlineto{\pgfqpoint{-0.421281in}{6.106667in}}%
\pgfpathlineto{\pgfqpoint{-0.419835in}{6.103628in}}%
\pgfpathlineto{\pgfqpoint{-0.370995in}{6.000000in}}%
\pgfpathlineto{\pgfqpoint{-0.370248in}{5.998430in}}%
\pgfpathlineto{\pgfqpoint{-0.320714in}{5.893333in}}%
\pgfpathlineto{\pgfqpoint{-0.320661in}{5.893223in}}%
\pgfpathlineto{\pgfqpoint{-0.271074in}{5.789468in}}%
\pgfpathlineto{\pgfqpoint{-0.269761in}{5.786667in}}%
\pgfpathlineto{\pgfqpoint{-0.221488in}{5.685754in}}%
\pgfpathlineto{\pgfqpoint{-0.218789in}{5.680000in}}%
\pgfpathlineto{\pgfqpoint{-0.171901in}{5.581965in}}%
\pgfpathlineto{\pgfqpoint{-0.167852in}{5.573333in}}%
\pgfpathlineto{\pgfqpoint{-0.122314in}{5.478103in}}%
\pgfpathlineto{\pgfqpoint{-0.116948in}{5.466667in}}%
\pgfpathlineto{\pgfqpoint{-0.072727in}{5.374174in}}%
\pgfpathlineto{\pgfqpoint{-0.066076in}{5.360000in}}%
\pgfpathlineto{\pgfqpoint{-0.023140in}{5.270180in}}%
\pgfpathlineto{\pgfqpoint{-0.015234in}{5.253333in}}%
\pgfpathlineto{\pgfqpoint{0.026446in}{5.166126in}}%
\pgfpathlineto{\pgfqpoint{0.035581in}{5.146667in}}%
\pgfpathlineto{\pgfqpoint{0.076033in}{5.062014in}}%
\pgfpathlineto{\pgfqpoint{0.086369in}{5.040000in}}%
\pgfpathlineto{\pgfqpoint{0.125620in}{4.957848in}}%
\pgfpathlineto{\pgfqpoint{0.137131in}{4.933333in}}%
\pgfpathlineto{\pgfqpoint{0.175207in}{4.853630in}}%
\pgfpathlineto{\pgfqpoint{0.187870in}{4.826667in}}%
\pgfpathlineto{\pgfqpoint{0.224793in}{4.749363in}}%
\pgfpathlineto{\pgfqpoint{0.238585in}{4.720000in}}%
\pgfpathlineto{\pgfqpoint{0.274380in}{4.645050in}}%
\pgfpathlineto{\pgfqpoint{0.289279in}{4.613333in}}%
\pgfpathlineto{\pgfqpoint{0.323967in}{4.540692in}}%
\pgfpathlineto{\pgfqpoint{0.339953in}{4.506667in}}%
\pgfpathlineto{\pgfqpoint{0.373554in}{4.436291in}}%
\pgfpathlineto{\pgfqpoint{0.390607in}{4.400000in}}%
\pgfpathlineto{\pgfqpoint{0.423140in}{4.331851in}}%
\pgfpathlineto{\pgfqpoint{0.441242in}{4.293333in}}%
\pgfpathlineto{\pgfqpoint{0.472727in}{4.227372in}}%
\pgfpathlineto{\pgfqpoint{0.491859in}{4.186667in}}%
\pgfpathlineto{\pgfqpoint{0.522314in}{4.122857in}}%
\pgfpathlineto{\pgfqpoint{0.542459in}{4.080000in}}%
\pgfpathlineto{\pgfqpoint{0.571901in}{4.018306in}}%
\pgfpathlineto{\pgfqpoint{0.593043in}{3.973333in}}%
\pgfpathlineto{\pgfqpoint{0.621488in}{3.913722in}}%
\pgfpathlineto{\pgfqpoint{0.643611in}{3.866667in}}%
\pgfpathlineto{\pgfqpoint{0.671074in}{3.809106in}}%
\pgfpathlineto{\pgfqpoint{0.694165in}{3.760000in}}%
\pgfpathlineto{\pgfqpoint{0.720661in}{3.704460in}}%
\pgfpathlineto{\pgfqpoint{0.744704in}{3.653333in}}%
\pgfpathlineto{\pgfqpoint{0.770248in}{3.599784in}}%
\pgfpathlineto{\pgfqpoint{0.795229in}{3.546667in}}%
\pgfpathlineto{\pgfqpoint{0.819835in}{3.495080in}}%
\pgfpathlineto{\pgfqpoint{0.845742in}{3.440000in}}%
\pgfpathlineto{\pgfqpoint{0.869421in}{3.390349in}}%
\pgfpathlineto{\pgfqpoint{0.896242in}{3.333333in}}%
\pgfpathlineto{\pgfqpoint{0.919008in}{3.285593in}}%
\pgfpathlineto{\pgfqpoint{0.946730in}{3.226667in}}%
\pgfpathlineto{\pgfqpoint{0.968595in}{3.180812in}}%
\pgfpathlineto{\pgfqpoint{0.997206in}{3.120000in}}%
\pgfpathlineto{\pgfqpoint{1.018182in}{3.076007in}}%
\pgfpathlineto{\pgfqpoint{1.047671in}{3.013333in}}%
\pgfpathlineto{\pgfqpoint{1.067769in}{2.971179in}}%
\pgfpathlineto{\pgfqpoint{1.098126in}{2.906667in}}%
\pgfpathlineto{\pgfqpoint{1.117355in}{2.866329in}}%
\pgfpathlineto{\pgfqpoint{1.148570in}{2.800000in}}%
\pgfpathlineto{\pgfqpoint{1.166942in}{2.761458in}}%
\pgfpathlineto{\pgfqpoint{1.199005in}{2.693333in}}%
\pgfpathlineto{\pgfqpoint{1.216529in}{2.656566in}}%
\pgfpathlineto{\pgfqpoint{1.249429in}{2.586667in}}%
\pgfpathlineto{\pgfqpoint{1.266116in}{2.551655in}}%
\pgfpathlineto{\pgfqpoint{1.299845in}{2.480000in}}%
\pgfpathlineto{\pgfqpoint{1.315702in}{2.446726in}}%
\pgfpathlineto{\pgfqpoint{1.350252in}{2.373333in}}%
\pgfpathlineto{\pgfqpoint{1.365289in}{2.341778in}}%
\pgfpathlineto{\pgfqpoint{1.400651in}{2.266667in}}%
\pgfpathlineto{\pgfqpoint{1.414876in}{2.236812in}}%
\pgfpathlineto{\pgfqpoint{1.451041in}{2.160000in}}%
\pgfpathlineto{\pgfqpoint{1.464463in}{2.131830in}}%
\pgfpathlineto{\pgfqpoint{1.501423in}{2.053333in}}%
\pgfpathlineto{\pgfqpoint{1.514050in}{2.026831in}}%
\pgfpathlineto{\pgfqpoint{1.551798in}{1.946667in}}%
\pgfpathlineto{\pgfqpoint{1.563636in}{1.921817in}}%
\pgfpathlineto{\pgfqpoint{1.602166in}{1.840000in}}%
\pgfpathlineto{\pgfqpoint{1.613223in}{1.816787in}}%
\pgfpathlineto{\pgfqpoint{1.652526in}{1.733333in}}%
\pgfpathlineto{\pgfqpoint{1.662810in}{1.711743in}}%
\pgfpathlineto{\pgfqpoint{1.702879in}{1.626667in}}%
\pgfpathlineto{\pgfqpoint{1.712397in}{1.606685in}}%
\pgfpathlineto{\pgfqpoint{1.753226in}{1.520000in}}%
\pgfpathlineto{\pgfqpoint{1.761983in}{1.501612in}}%
\pgfpathlineto{\pgfqpoint{1.803566in}{1.413333in}}%
\pgfpathlineto{\pgfqpoint{1.811570in}{1.396527in}}%
\pgfpathlineto{\pgfqpoint{1.853900in}{1.306667in}}%
\pgfpathlineto{\pgfqpoint{1.861157in}{1.291429in}}%
\pgfpathlineto{\pgfqpoint{1.904229in}{1.200000in}}%
\pgfpathlineto{\pgfqpoint{1.910744in}{1.186318in}}%
\pgfpathlineto{\pgfqpoint{1.954551in}{1.093333in}}%
\pgfpathlineto{\pgfqpoint{1.960331in}{1.081195in}}%
\pgfpathlineto{\pgfqpoint{2.004868in}{0.986667in}}%
\pgfpathlineto{\pgfqpoint{2.009917in}{0.976061in}}%
\pgfpathlineto{\pgfqpoint{2.055179in}{0.880000in}}%
\pgfpathlineto{\pgfqpoint{2.059504in}{0.870915in}}%
\pgfpathlineto{\pgfqpoint{2.105485in}{0.773333in}}%
\pgfpathlineto{\pgfqpoint{2.109091in}{0.765758in}}%
\pgfpathlineto{\pgfqpoint{2.155785in}{0.666667in}}%
\pgfpathlineto{\pgfqpoint{2.158678in}{0.660591in}}%
\pgfpathlineto{\pgfqpoint{2.206081in}{0.560000in}}%
\pgfpathlineto{\pgfqpoint{2.208264in}{0.555413in}}%
\pgfpathlineto{\pgfqpoint{2.256372in}{0.453333in}}%
\pgfpathlineto{\pgfqpoint{2.257851in}{0.450226in}}%
\pgfpathlineto{\pgfqpoint{2.306658in}{0.346667in}}%
\pgfpathlineto{\pgfqpoint{2.307438in}{0.345029in}}%
\pgfpathlineto{\pgfqpoint{2.356940in}{0.240000in}}%
\pgfpathlineto{\pgfqpoint{2.357025in}{0.239822in}}%
\pgfpathlineto{\pgfqpoint{2.406612in}{0.135995in}}%
\pgfpathlineto{\pgfqpoint{2.407860in}{0.133333in}}%
\pgfpathlineto{\pgfqpoint{2.456198in}{0.032285in}}%
\pgfpathlineto{\pgfqpoint{2.458833in}{0.026667in}}%
\pgfpathlineto{\pgfqpoint{2.505785in}{-0.071501in}}%
\pgfpathlineto{\pgfqpoint{2.509772in}{-0.080000in}}%
\pgfpathlineto{\pgfqpoint{2.555372in}{-0.175359in}}%
\pgfpathlineto{\pgfqpoint{2.560677in}{-0.186667in}}%
\pgfpathlineto{\pgfqpoint{2.604959in}{-0.279285in}}%
\pgfpathlineto{\pgfqpoint{2.611551in}{-0.293333in}}%
\pgfpathlineto{\pgfqpoint{2.654545in}{-0.383276in}}%
\pgfpathlineto{\pgfqpoint{2.662395in}{-0.400000in}}%
\pgfpathlineto{\pgfqpoint{2.704132in}{-0.487327in}}%
\pgfpathlineto{\pgfqpoint{2.713210in}{-0.506667in}}%
\pgfpathlineto{\pgfqpoint{2.753719in}{-0.591437in}}%
\pgfpathlineto{\pgfqpoint{2.763999in}{-0.613333in}}%
\pgfpathlineto{\pgfqpoint{2.803306in}{-0.695600in}}%
\pgfpathlineto{\pgfqpoint{2.814763in}{-0.720000in}}%
\pgfpathlineto{\pgfqpoint{2.852893in}{-0.799816in}}%
\pgfpathlineto{\pgfqpoint{2.865502in}{-0.826667in}}%
\pgfpathlineto{\pgfqpoint{2.902479in}{-0.904081in}}%
\pgfpathlineto{\pgfqpoint{2.916219in}{-0.933333in}}%
\pgfpathlineto{\pgfqpoint{2.952066in}{-1.008392in}}%
\pgfpathlineto{\pgfqpoint{2.966914in}{-1.040000in}}%
\pgfpathlineto{\pgfqpoint{3.001653in}{-1.112748in}}%
\pgfpathlineto{\pgfqpoint{3.017589in}{-1.146667in}}%
\pgfpathlineto{\pgfqpoint{3.051240in}{-1.217146in}}%
\pgfpathlineto{\pgfqpoint{3.068243in}{-1.253333in}}%
\pgfpathlineto{\pgfqpoint{3.100826in}{-1.321585in}}%
\pgfpathlineto{\pgfqpoint{3.118879in}{-1.360000in}}%
\pgfpathlineto{\pgfqpoint{3.150413in}{-1.426062in}}%
\pgfpathlineto{\pgfqpoint{3.169497in}{-1.466667in}}%
\pgfpathlineto{\pgfqpoint{3.200000in}{-1.530576in}}%
\pgfpathlineto{\pgfqpoint{3.220098in}{-1.573333in}}%
\pgfpathlineto{\pgfqpoint{3.249587in}{-1.635125in}}%
\pgfpathlineto{\pgfqpoint{3.270683in}{-1.680000in}}%
\pgfpathlineto{\pgfqpoint{3.299174in}{-1.739707in}}%
\pgfpathlineto{\pgfqpoint{3.321252in}{-1.786667in}}%
\pgfpathlineto{\pgfqpoint{3.348760in}{-1.844322in}}%
\pgfpathlineto{\pgfqpoint{3.371806in}{-1.893333in}}%
\pgfpathlineto{\pgfqpoint{3.398347in}{-1.948967in}}%
\pgfpathlineto{\pgfqpoint{3.422346in}{-2.000000in}}%
\pgfpathlineto{\pgfqpoint{3.447934in}{-2.053641in}}%
\pgfpathlineto{\pgfqpoint{3.472872in}{-2.106667in}}%
\pgfpathlineto{\pgfqpoint{3.497521in}{-2.158344in}}%
\pgfpathlineto{\pgfqpoint{3.523385in}{-2.213333in}}%
\pgfpathlineto{\pgfqpoint{3.547107in}{-2.263073in}}%
\pgfpathlineto{\pgfqpoint{3.573886in}{-2.320000in}}%
\pgfpathlineto{\pgfqpoint{3.596694in}{-2.367829in}}%
\pgfpathlineto{\pgfqpoint{3.624374in}{-2.426667in}}%
\pgfpathlineto{\pgfqpoint{3.646281in}{-2.472609in}}%
\pgfpathlineto{\pgfqpoint{3.674851in}{-2.533333in}}%
\pgfpathclose%
\pgfpathmoveto{\pgfqpoint{3.683795in}{-2.426667in}}%
\pgfpathlineto{\pgfqpoint{3.695868in}{-2.410251in}}%
\pgfpathlineto{\pgfqpoint{3.703764in}{-2.426667in}}%
\pgfpathlineto{\pgfqpoint{3.695868in}{-2.451562in}}%
\pgfpathclose%
\pgfpathmoveto{\pgfqpoint{3.633887in}{-2.320000in}}%
\pgfpathlineto{\pgfqpoint{3.646281in}{-2.304428in}}%
\pgfpathlineto{\pgfqpoint{3.653768in}{-2.320000in}}%
\pgfpathlineto{\pgfqpoint{3.646281in}{-2.345528in}}%
\pgfpathclose%
\pgfpathmoveto{\pgfqpoint{3.583994in}{-2.213333in}}%
\pgfpathlineto{\pgfqpoint{3.596694in}{-2.198615in}}%
\pgfpathlineto{\pgfqpoint{3.603768in}{-2.213333in}}%
\pgfpathlineto{\pgfqpoint{3.596694in}{-2.239458in}}%
\pgfpathclose%
\pgfpathmoveto{\pgfqpoint{3.534120in}{-2.106667in}}%
\pgfpathlineto{\pgfqpoint{3.547107in}{-2.092808in}}%
\pgfpathlineto{\pgfqpoint{3.553765in}{-2.106667in}}%
\pgfpathlineto{\pgfqpoint{3.547107in}{-2.133346in}}%
\pgfpathclose%
\pgfpathmoveto{\pgfqpoint{3.484266in}{-2.000000in}}%
\pgfpathlineto{\pgfqpoint{3.497521in}{-1.987009in}}%
\pgfpathlineto{\pgfqpoint{3.503759in}{-2.000000in}}%
\pgfpathlineto{\pgfqpoint{3.497521in}{-2.027184in}}%
\pgfpathclose%
\pgfpathmoveto{\pgfqpoint{3.434437in}{-1.893333in}}%
\pgfpathlineto{\pgfqpoint{3.447934in}{-1.881216in}}%
\pgfpathlineto{\pgfqpoint{3.453751in}{-1.893333in}}%
\pgfpathlineto{\pgfqpoint{3.447934in}{-1.920964in}}%
\pgfpathclose%
\pgfpathmoveto{\pgfqpoint{3.384639in}{-1.786667in}}%
\pgfpathlineto{\pgfqpoint{3.398347in}{-1.775428in}}%
\pgfpathlineto{\pgfqpoint{3.403740in}{-1.786667in}}%
\pgfpathlineto{\pgfqpoint{3.398347in}{-1.814674in}}%
\pgfpathclose%
\pgfpathmoveto{\pgfqpoint{3.334877in}{-1.680000in}}%
\pgfpathlineto{\pgfqpoint{3.348760in}{-1.669646in}}%
\pgfpathlineto{\pgfqpoint{3.353727in}{-1.680000in}}%
\pgfpathlineto{\pgfqpoint{3.348760in}{-1.708300in}}%
\pgfpathclose%
\pgfpathmoveto{\pgfqpoint{3.285160in}{-1.573333in}}%
\pgfpathlineto{\pgfqpoint{3.299174in}{-1.563870in}}%
\pgfpathlineto{\pgfqpoint{3.303712in}{-1.573333in}}%
\pgfpathlineto{\pgfqpoint{3.299174in}{-1.601822in}}%
\pgfpathclose%
\pgfpathmoveto{\pgfqpoint{3.235500in}{-1.466667in}}%
\pgfpathlineto{\pgfqpoint{3.249587in}{-1.458098in}}%
\pgfpathlineto{\pgfqpoint{3.253695in}{-1.466667in}}%
\pgfpathlineto{\pgfqpoint{3.249587in}{-1.495217in}}%
\pgfpathclose%
\pgfpathmoveto{\pgfqpoint{3.185911in}{-1.360000in}}%
\pgfpathlineto{\pgfqpoint{3.200000in}{-1.352330in}}%
\pgfpathlineto{\pgfqpoint{3.203676in}{-1.360000in}}%
\pgfpathlineto{\pgfqpoint{3.200000in}{-1.388451in}}%
\pgfpathclose%
\pgfpathmoveto{\pgfqpoint{3.136415in}{-1.253333in}}%
\pgfpathlineto{\pgfqpoint{3.150413in}{-1.246566in}}%
\pgfpathlineto{\pgfqpoint{3.153656in}{-1.253333in}}%
\pgfpathlineto{\pgfqpoint{3.150413in}{-1.281477in}}%
\pgfpathclose%
\pgfpathmoveto{\pgfqpoint{3.087044in}{-1.146667in}}%
\pgfpathlineto{\pgfqpoint{3.100826in}{-1.140806in}}%
\pgfpathlineto{\pgfqpoint{3.103634in}{-1.146667in}}%
\pgfpathlineto{\pgfqpoint{3.100826in}{-1.174230in}}%
\pgfpathclose%
\pgfpathmoveto{\pgfqpoint{3.037845in}{-1.040000in}}%
\pgfpathlineto{\pgfqpoint{3.051240in}{-1.035050in}}%
\pgfpathlineto{\pgfqpoint{3.053610in}{-1.040000in}}%
\pgfpathlineto{\pgfqpoint{3.051240in}{-1.066613in}}%
\pgfpathclose%
\pgfpathmoveto{\pgfqpoint{2.988890in}{-0.933333in}}%
\pgfpathlineto{\pgfqpoint{3.001653in}{-0.929296in}}%
\pgfpathlineto{\pgfqpoint{3.003586in}{-0.933333in}}%
\pgfpathlineto{\pgfqpoint{3.001653in}{-0.958478in}}%
\pgfpathclose%
\pgfpathmoveto{\pgfqpoint{2.940303in}{-0.826667in}}%
\pgfpathlineto{\pgfqpoint{2.952066in}{-0.823546in}}%
\pgfpathlineto{\pgfqpoint{2.953560in}{-0.826667in}}%
\pgfpathlineto{\pgfqpoint{2.952066in}{-0.849585in}}%
\pgfpathclose%
\pgfpathmoveto{\pgfqpoint{2.892306in}{-0.720000in}}%
\pgfpathlineto{\pgfqpoint{2.902479in}{-0.717799in}}%
\pgfpathlineto{\pgfqpoint{2.903533in}{-0.720000in}}%
\pgfpathlineto{\pgfqpoint{2.902479in}{-0.739519in}}%
\pgfpathclose%
\pgfpathmoveto{\pgfqpoint{2.845343in}{-0.613333in}}%
\pgfpathlineto{\pgfqpoint{2.852893in}{-0.612055in}}%
\pgfpathlineto{\pgfqpoint{2.853504in}{-0.613333in}}%
\pgfpathlineto{\pgfqpoint{2.852893in}{-0.627492in}}%
\pgfpathclose%
\pgfpathmoveto{\pgfqpoint{2.800457in}{-0.506667in}}%
\pgfpathlineto{\pgfqpoint{2.803306in}{-0.506313in}}%
\pgfpathlineto{\pgfqpoint{2.803475in}{-0.506667in}}%
\pgfpathlineto{\pgfqpoint{2.803306in}{-0.511816in}}%
\pgfpathclose%
\pgfpathmoveto{\pgfqpoint{2.456008in}{0.133333in}}%
\pgfpathlineto{\pgfqpoint{2.456198in}{0.139027in}}%
\pgfpathlineto{\pgfqpoint{2.459341in}{0.133333in}}%
\pgfpathlineto{\pgfqpoint{2.456198in}{0.132936in}}%
\pgfpathclose%
\pgfpathmoveto{\pgfqpoint{2.405979in}{0.240000in}}%
\pgfpathlineto{\pgfqpoint{2.406612in}{0.254471in}}%
\pgfpathlineto{\pgfqpoint{2.414317in}{0.240000in}}%
\pgfpathlineto{\pgfqpoint{2.406612in}{0.238678in}}%
\pgfpathclose%
\pgfpathmoveto{\pgfqpoint{2.355951in}{0.346667in}}%
\pgfpathlineto{\pgfqpoint{2.357025in}{0.366378in}}%
\pgfpathlineto{\pgfqpoint{2.367290in}{0.346667in}}%
\pgfpathlineto{\pgfqpoint{2.357025in}{0.344423in}}%
\pgfpathclose%
\pgfpathmoveto{\pgfqpoint{2.305924in}{0.453333in}}%
\pgfpathlineto{\pgfqpoint{2.307438in}{0.476377in}}%
\pgfpathlineto{\pgfqpoint{2.319258in}{0.453333in}}%
\pgfpathlineto{\pgfqpoint{2.307438in}{0.450170in}}%
\pgfpathclose%
\pgfpathmoveto{\pgfqpoint{2.255898in}{0.560000in}}%
\pgfpathlineto{\pgfqpoint{2.257851in}{0.585227in}}%
\pgfpathlineto{\pgfqpoint{2.270651in}{0.560000in}}%
\pgfpathlineto{\pgfqpoint{2.257851in}{0.555921in}}%
\pgfpathclose%
\pgfpathmoveto{\pgfqpoint{2.205873in}{0.666667in}}%
\pgfpathlineto{\pgfqpoint{2.208264in}{0.693334in}}%
\pgfpathlineto{\pgfqpoint{2.221682in}{0.666667in}}%
\pgfpathlineto{\pgfqpoint{2.208264in}{0.661674in}}%
\pgfpathclose%
\pgfpathmoveto{\pgfqpoint{2.155850in}{0.773333in}}%
\pgfpathlineto{\pgfqpoint{2.158678in}{0.800931in}}%
\pgfpathlineto{\pgfqpoint{2.172473in}{0.773333in}}%
\pgfpathlineto{\pgfqpoint{2.158678in}{0.767430in}}%
\pgfpathclose%
\pgfpathmoveto{\pgfqpoint{2.105828in}{0.880000in}}%
\pgfpathlineto{\pgfqpoint{2.109091in}{0.908163in}}%
\pgfpathlineto{\pgfqpoint{2.123095in}{0.880000in}}%
\pgfpathlineto{\pgfqpoint{2.109091in}{0.873191in}}%
\pgfpathclose%
\pgfpathmoveto{\pgfqpoint{2.055808in}{0.986667in}}%
\pgfpathlineto{\pgfqpoint{2.059504in}{1.015126in}}%
\pgfpathlineto{\pgfqpoint{2.073595in}{0.986667in}}%
\pgfpathlineto{\pgfqpoint{2.059504in}{0.978955in}}%
\pgfpathclose%
\pgfpathmoveto{\pgfqpoint{2.005789in}{1.093333in}}%
\pgfpathlineto{\pgfqpoint{2.009917in}{1.121884in}}%
\pgfpathlineto{\pgfqpoint{2.024002in}{1.093333in}}%
\pgfpathlineto{\pgfqpoint{2.009917in}{1.084723in}}%
\pgfpathclose%
\pgfpathmoveto{\pgfqpoint{1.955772in}{1.200000in}}%
\pgfpathlineto{\pgfqpoint{1.960331in}{1.228483in}}%
\pgfpathlineto{\pgfqpoint{1.974339in}{1.200000in}}%
\pgfpathlineto{\pgfqpoint{1.960331in}{1.190495in}}%
\pgfpathclose%
\pgfpathmoveto{\pgfqpoint{1.905757in}{1.306667in}}%
\pgfpathlineto{\pgfqpoint{1.910744in}{1.334955in}}%
\pgfpathlineto{\pgfqpoint{1.924620in}{1.306667in}}%
\pgfpathlineto{\pgfqpoint{1.910744in}{1.296272in}}%
\pgfpathclose%
\pgfpathmoveto{\pgfqpoint{1.855744in}{1.413333in}}%
\pgfpathlineto{\pgfqpoint{1.861157in}{1.441325in}}%
\pgfpathlineto{\pgfqpoint{1.874856in}{1.413333in}}%
\pgfpathlineto{\pgfqpoint{1.861157in}{1.402054in}}%
\pgfpathclose%
\pgfpathmoveto{\pgfqpoint{1.805734in}{1.520000in}}%
\pgfpathlineto{\pgfqpoint{1.811570in}{1.547611in}}%
\pgfpathlineto{\pgfqpoint{1.825056in}{1.520000in}}%
\pgfpathlineto{\pgfqpoint{1.811570in}{1.507841in}}%
\pgfpathclose%
\pgfpathmoveto{\pgfqpoint{1.755725in}{1.626667in}}%
\pgfpathlineto{\pgfqpoint{1.761983in}{1.653829in}}%
\pgfpathlineto{\pgfqpoint{1.775226in}{1.626667in}}%
\pgfpathlineto{\pgfqpoint{1.761983in}{1.613635in}}%
\pgfpathclose%
\pgfpathmoveto{\pgfqpoint{1.705720in}{1.733333in}}%
\pgfpathlineto{\pgfqpoint{1.712397in}{1.759988in}}%
\pgfpathlineto{\pgfqpoint{1.725372in}{1.733333in}}%
\pgfpathlineto{\pgfqpoint{1.712397in}{1.719435in}}%
\pgfpathclose%
\pgfpathmoveto{\pgfqpoint{1.655717in}{1.840000in}}%
\pgfpathlineto{\pgfqpoint{1.662810in}{1.866098in}}%
\pgfpathlineto{\pgfqpoint{1.675496in}{1.840000in}}%
\pgfpathlineto{\pgfqpoint{1.662810in}{1.825241in}}%
\pgfpathclose%
\pgfpathmoveto{\pgfqpoint{1.605717in}{1.946667in}}%
\pgfpathlineto{\pgfqpoint{1.613223in}{1.972166in}}%
\pgfpathlineto{\pgfqpoint{1.625602in}{1.946667in}}%
\pgfpathlineto{\pgfqpoint{1.613223in}{1.931056in}}%
\pgfpathclose%
\pgfpathmoveto{\pgfqpoint{1.555721in}{2.053333in}}%
\pgfpathlineto{\pgfqpoint{1.563636in}{2.078198in}}%
\pgfpathlineto{\pgfqpoint{1.575693in}{2.053333in}}%
\pgfpathlineto{\pgfqpoint{1.563636in}{2.036878in}}%
\pgfpathclose%
\pgfpathmoveto{\pgfqpoint{1.505728in}{2.160000in}}%
\pgfpathlineto{\pgfqpoint{1.514050in}{2.184199in}}%
\pgfpathlineto{\pgfqpoint{1.525771in}{2.160000in}}%
\pgfpathlineto{\pgfqpoint{1.514050in}{2.142710in}}%
\pgfpathclose%
\pgfpathmoveto{\pgfqpoint{1.455740in}{2.266667in}}%
\pgfpathlineto{\pgfqpoint{1.464463in}{2.290173in}}%
\pgfpathlineto{\pgfqpoint{1.475838in}{2.266667in}}%
\pgfpathlineto{\pgfqpoint{1.464463in}{2.248551in}}%
\pgfpathclose%
\pgfpathmoveto{\pgfqpoint{1.405756in}{2.373333in}}%
\pgfpathlineto{\pgfqpoint{1.414876in}{2.396122in}}%
\pgfpathlineto{\pgfqpoint{1.425894in}{2.373333in}}%
\pgfpathlineto{\pgfqpoint{1.414876in}{2.354404in}}%
\pgfpathclose%
\pgfpathmoveto{\pgfqpoint{1.355777in}{2.480000in}}%
\pgfpathlineto{\pgfqpoint{1.365289in}{2.502051in}}%
\pgfpathlineto{\pgfqpoint{1.375942in}{2.480000in}}%
\pgfpathlineto{\pgfqpoint{1.365289in}{2.460269in}}%
\pgfpathclose%
\pgfpathmoveto{\pgfqpoint{1.305804in}{2.586667in}}%
\pgfpathlineto{\pgfqpoint{1.315702in}{2.607960in}}%
\pgfpathlineto{\pgfqpoint{1.325982in}{2.586667in}}%
\pgfpathlineto{\pgfqpoint{1.315702in}{2.566147in}}%
\pgfpathclose%
\pgfpathmoveto{\pgfqpoint{1.255836in}{2.693333in}}%
\pgfpathlineto{\pgfqpoint{1.266116in}{2.713853in}}%
\pgfpathlineto{\pgfqpoint{1.276015in}{2.693333in}}%
\pgfpathlineto{\pgfqpoint{1.266116in}{2.672040in}}%
\pgfpathclose%
\pgfpathmoveto{\pgfqpoint{1.205876in}{2.800000in}}%
\pgfpathlineto{\pgfqpoint{1.216529in}{2.819731in}}%
\pgfpathlineto{\pgfqpoint{1.226041in}{2.800000in}}%
\pgfpathlineto{\pgfqpoint{1.216529in}{2.777949in}}%
\pgfpathclose%
\pgfpathmoveto{\pgfqpoint{1.155924in}{2.906667in}}%
\pgfpathlineto{\pgfqpoint{1.166942in}{2.925596in}}%
\pgfpathlineto{\pgfqpoint{1.176062in}{2.906667in}}%
\pgfpathlineto{\pgfqpoint{1.166942in}{2.883878in}}%
\pgfpathclose%
\pgfpathmoveto{\pgfqpoint{1.105980in}{3.013333in}}%
\pgfpathlineto{\pgfqpoint{1.117355in}{3.031449in}}%
\pgfpathlineto{\pgfqpoint{1.126078in}{3.013333in}}%
\pgfpathlineto{\pgfqpoint{1.117355in}{2.989827in}}%
\pgfpathclose%
\pgfpathmoveto{\pgfqpoint{1.056047in}{3.120000in}}%
\pgfpathlineto{\pgfqpoint{1.067769in}{3.137290in}}%
\pgfpathlineto{\pgfqpoint{1.076090in}{3.120000in}}%
\pgfpathlineto{\pgfqpoint{1.067769in}{3.095801in}}%
\pgfpathclose%
\pgfpathmoveto{\pgfqpoint{1.006125in}{3.226667in}}%
\pgfpathlineto{\pgfqpoint{1.018182in}{3.243122in}}%
\pgfpathlineto{\pgfqpoint{1.026097in}{3.226667in}}%
\pgfpathlineto{\pgfqpoint{1.018182in}{3.201802in}}%
\pgfpathclose%
\pgfpathmoveto{\pgfqpoint{0.956216in}{3.333333in}}%
\pgfpathlineto{\pgfqpoint{0.968595in}{3.348944in}}%
\pgfpathlineto{\pgfqpoint{0.976101in}{3.333333in}}%
\pgfpathlineto{\pgfqpoint{0.968595in}{3.307834in}}%
\pgfpathclose%
\pgfpathmoveto{\pgfqpoint{0.906322in}{3.440000in}}%
\pgfpathlineto{\pgfqpoint{0.919008in}{3.454759in}}%
\pgfpathlineto{\pgfqpoint{0.926101in}{3.440000in}}%
\pgfpathlineto{\pgfqpoint{0.919008in}{3.413902in}}%
\pgfpathclose%
\pgfpathmoveto{\pgfqpoint{0.856447in}{3.546667in}}%
\pgfpathlineto{\pgfqpoint{0.869421in}{3.560565in}}%
\pgfpathlineto{\pgfqpoint{0.876098in}{3.546667in}}%
\pgfpathlineto{\pgfqpoint{0.869421in}{3.520012in}}%
\pgfpathclose%
\pgfpathmoveto{\pgfqpoint{0.806592in}{3.653333in}}%
\pgfpathlineto{\pgfqpoint{0.819835in}{3.666365in}}%
\pgfpathlineto{\pgfqpoint{0.826093in}{3.653333in}}%
\pgfpathlineto{\pgfqpoint{0.819835in}{3.626171in}}%
\pgfpathclose%
\pgfpathmoveto{\pgfqpoint{0.756762in}{3.760000in}}%
\pgfpathlineto{\pgfqpoint{0.770248in}{3.772159in}}%
\pgfpathlineto{\pgfqpoint{0.776084in}{3.760000in}}%
\pgfpathlineto{\pgfqpoint{0.770248in}{3.732389in}}%
\pgfpathclose%
\pgfpathmoveto{\pgfqpoint{0.706962in}{3.866667in}}%
\pgfpathlineto{\pgfqpoint{0.720661in}{3.877946in}}%
\pgfpathlineto{\pgfqpoint{0.726074in}{3.866667in}}%
\pgfpathlineto{\pgfqpoint{0.720661in}{3.838675in}}%
\pgfpathclose%
\pgfpathmoveto{\pgfqpoint{0.657198in}{3.973333in}}%
\pgfpathlineto{\pgfqpoint{0.671074in}{3.983728in}}%
\pgfpathlineto{\pgfqpoint{0.676061in}{3.973333in}}%
\pgfpathlineto{\pgfqpoint{0.671074in}{3.945045in}}%
\pgfpathclose%
\pgfpathmoveto{\pgfqpoint{0.607479in}{4.080000in}}%
\pgfpathlineto{\pgfqpoint{0.621488in}{4.089505in}}%
\pgfpathlineto{\pgfqpoint{0.626046in}{4.080000in}}%
\pgfpathlineto{\pgfqpoint{0.621488in}{4.051517in}}%
\pgfpathclose%
\pgfpathmoveto{\pgfqpoint{0.557816in}{4.186667in}}%
\pgfpathlineto{\pgfqpoint{0.571901in}{4.195277in}}%
\pgfpathlineto{\pgfqpoint{0.576029in}{4.186667in}}%
\pgfpathlineto{\pgfqpoint{0.571901in}{4.158116in}}%
\pgfpathclose%
\pgfpathmoveto{\pgfqpoint{0.508223in}{4.293333in}}%
\pgfpathlineto{\pgfqpoint{0.522314in}{4.301045in}}%
\pgfpathlineto{\pgfqpoint{0.526010in}{4.293333in}}%
\pgfpathlineto{\pgfqpoint{0.522314in}{4.264874in}}%
\pgfpathclose%
\pgfpathmoveto{\pgfqpoint{0.458723in}{4.400000in}}%
\pgfpathlineto{\pgfqpoint{0.472727in}{4.406809in}}%
\pgfpathlineto{\pgfqpoint{0.475990in}{4.400000in}}%
\pgfpathlineto{\pgfqpoint{0.472727in}{4.371837in}}%
\pgfpathclose%
\pgfpathmoveto{\pgfqpoint{0.409345in}{4.506667in}}%
\pgfpathlineto{\pgfqpoint{0.423140in}{4.512570in}}%
\pgfpathlineto{\pgfqpoint{0.425968in}{4.506667in}}%
\pgfpathlineto{\pgfqpoint{0.423140in}{4.479069in}}%
\pgfpathclose%
\pgfpathmoveto{\pgfqpoint{0.360136in}{4.613333in}}%
\pgfpathlineto{\pgfqpoint{0.373554in}{4.618326in}}%
\pgfpathlineto{\pgfqpoint{0.375945in}{4.613333in}}%
\pgfpathlineto{\pgfqpoint{0.373554in}{4.586666in}}%
\pgfpathclose%
\pgfpathmoveto{\pgfqpoint{0.311168in}{4.720000in}}%
\pgfpathlineto{\pgfqpoint{0.323967in}{4.724079in}}%
\pgfpathlineto{\pgfqpoint{0.325920in}{4.720000in}}%
\pgfpathlineto{\pgfqpoint{0.323967in}{4.694773in}}%
\pgfpathclose%
\pgfpathmoveto{\pgfqpoint{0.262560in}{4.826667in}}%
\pgfpathlineto{\pgfqpoint{0.274380in}{4.829830in}}%
\pgfpathlineto{\pgfqpoint{0.275894in}{4.826667in}}%
\pgfpathlineto{\pgfqpoint{0.274380in}{4.803623in}}%
\pgfpathclose%
\pgfpathmoveto{\pgfqpoint{0.214528in}{4.933333in}}%
\pgfpathlineto{\pgfqpoint{0.224793in}{4.935577in}}%
\pgfpathlineto{\pgfqpoint{0.225867in}{4.933333in}}%
\pgfpathlineto{\pgfqpoint{0.224793in}{4.913622in}}%
\pgfpathclose%
\pgfpathmoveto{\pgfqpoint{0.167501in}{5.040000in}}%
\pgfpathlineto{\pgfqpoint{0.175207in}{5.041322in}}%
\pgfpathlineto{\pgfqpoint{0.175839in}{5.040000in}}%
\pgfpathlineto{\pgfqpoint{0.175207in}{5.025529in}}%
\pgfpathclose%
\pgfpathmoveto{\pgfqpoint{0.122477in}{5.146667in}}%
\pgfpathlineto{\pgfqpoint{0.125620in}{5.147064in}}%
\pgfpathlineto{\pgfqpoint{0.125810in}{5.146667in}}%
\pgfpathlineto{\pgfqpoint{0.125620in}{5.140973in}}%
\pgfpathclose%
\pgfpathmoveto{\pgfqpoint{-0.221657in}{5.786667in}}%
\pgfpathlineto{\pgfqpoint{-0.221488in}{5.791816in}}%
\pgfpathlineto{\pgfqpoint{-0.218639in}{5.786667in}}%
\pgfpathlineto{\pgfqpoint{-0.221488in}{5.786313in}}%
\pgfpathclose%
\pgfpathmoveto{\pgfqpoint{-0.271686in}{5.893333in}}%
\pgfpathlineto{\pgfqpoint{-0.271074in}{5.907492in}}%
\pgfpathlineto{\pgfqpoint{-0.263525in}{5.893333in}}%
\pgfpathlineto{\pgfqpoint{-0.271074in}{5.892055in}}%
\pgfpathclose%
\pgfpathmoveto{\pgfqpoint{-0.321714in}{6.000000in}}%
\pgfpathlineto{\pgfqpoint{-0.320661in}{6.019519in}}%
\pgfpathlineto{\pgfqpoint{-0.310488in}{6.000000in}}%
\pgfpathlineto{\pgfqpoint{-0.320661in}{5.997799in}}%
\pgfpathclose%
\pgfpathmoveto{\pgfqpoint{-0.371742in}{6.106667in}}%
\pgfpathlineto{\pgfqpoint{-0.370248in}{6.129585in}}%
\pgfpathlineto{\pgfqpoint{-0.358485in}{6.106667in}}%
\pgfpathlineto{\pgfqpoint{-0.370248in}{6.103546in}}%
\pgfpathclose%
\pgfpathmoveto{\pgfqpoint{-0.421768in}{6.213333in}}%
\pgfpathlineto{\pgfqpoint{-0.419835in}{6.238478in}}%
\pgfpathlineto{\pgfqpoint{-0.407072in}{6.213333in}}%
\pgfpathlineto{\pgfqpoint{-0.419835in}{6.209296in}}%
\pgfpathclose%
\pgfpathmoveto{\pgfqpoint{-0.471792in}{6.320000in}}%
\pgfpathlineto{\pgfqpoint{-0.469421in}{6.346613in}}%
\pgfpathlineto{\pgfqpoint{-0.456027in}{6.320000in}}%
\pgfpathlineto{\pgfqpoint{-0.469421in}{6.315050in}}%
\pgfpathclose%
\pgfpathmoveto{\pgfqpoint{-0.521816in}{6.426667in}}%
\pgfpathlineto{\pgfqpoint{-0.519008in}{6.454230in}}%
\pgfpathlineto{\pgfqpoint{-0.505226in}{6.426667in}}%
\pgfpathlineto{\pgfqpoint{-0.519008in}{6.420806in}}%
\pgfpathclose%
\pgfpathmoveto{\pgfqpoint{-0.571838in}{6.533333in}}%
\pgfpathlineto{\pgfqpoint{-0.568595in}{6.561477in}}%
\pgfpathlineto{\pgfqpoint{-0.554597in}{6.533333in}}%
\pgfpathlineto{\pgfqpoint{-0.568595in}{6.526566in}}%
\pgfpathclose%
\pgfpathmoveto{\pgfqpoint{-0.621858in}{6.640000in}}%
\pgfpathlineto{\pgfqpoint{-0.618182in}{6.668451in}}%
\pgfpathlineto{\pgfqpoint{-0.604093in}{6.640000in}}%
\pgfpathlineto{\pgfqpoint{-0.618182in}{6.632330in}}%
\pgfpathclose%
\pgfpathmoveto{\pgfqpoint{-0.671877in}{6.746667in}}%
\pgfpathlineto{\pgfqpoint{-0.667769in}{6.775217in}}%
\pgfpathlineto{\pgfqpoint{-0.653681in}{6.746667in}}%
\pgfpathlineto{\pgfqpoint{-0.667769in}{6.738098in}}%
\pgfpathclose%
\pgfpathmoveto{\pgfqpoint{-0.721894in}{6.853333in}}%
\pgfpathlineto{\pgfqpoint{-0.717355in}{6.881822in}}%
\pgfpathlineto{\pgfqpoint{-0.703342in}{6.853333in}}%
\pgfpathlineto{\pgfqpoint{-0.717355in}{6.843870in}}%
\pgfpathclose%
\pgfpathmoveto{\pgfqpoint{-0.771909in}{6.960000in}}%
\pgfpathlineto{\pgfqpoint{-0.766942in}{6.988300in}}%
\pgfpathlineto{\pgfqpoint{-0.753059in}{6.960000in}}%
\pgfpathlineto{\pgfqpoint{-0.766942in}{6.949646in}}%
\pgfpathclose%
\pgfpathmoveto{\pgfqpoint{-0.821922in}{7.066667in}}%
\pgfpathlineto{\pgfqpoint{-0.816529in}{7.094674in}}%
\pgfpathlineto{\pgfqpoint{-0.802821in}{7.066667in}}%
\pgfpathlineto{\pgfqpoint{-0.816529in}{7.055428in}}%
\pgfpathclose%
\pgfpathmoveto{\pgfqpoint{-0.871933in}{7.173333in}}%
\pgfpathlineto{\pgfqpoint{-0.866116in}{7.200964in}}%
\pgfpathlineto{\pgfqpoint{-0.852619in}{7.173333in}}%
\pgfpathlineto{\pgfqpoint{-0.866116in}{7.161216in}}%
\pgfpathclose%
\pgfpathmoveto{\pgfqpoint{-0.921941in}{7.280000in}}%
\pgfpathlineto{\pgfqpoint{-0.915702in}{7.307184in}}%
\pgfpathlineto{\pgfqpoint{-0.902448in}{7.280000in}}%
\pgfpathlineto{\pgfqpoint{-0.915702in}{7.267009in}}%
\pgfpathclose%
\pgfpathmoveto{\pgfqpoint{-0.971947in}{7.386667in}}%
\pgfpathlineto{\pgfqpoint{-0.965289in}{7.413346in}}%
\pgfpathlineto{\pgfqpoint{-0.952301in}{7.386667in}}%
\pgfpathlineto{\pgfqpoint{-0.965289in}{7.372808in}}%
\pgfpathclose%
\pgfpathmoveto{\pgfqpoint{-1.021950in}{7.493333in}}%
\pgfpathlineto{\pgfqpoint{-1.014876in}{7.519458in}}%
\pgfpathlineto{\pgfqpoint{-1.002176in}{7.493333in}}%
\pgfpathlineto{\pgfqpoint{-1.014876in}{7.478615in}}%
\pgfpathclose%
\pgfpathmoveto{\pgfqpoint{-1.071950in}{7.600000in}}%
\pgfpathlineto{\pgfqpoint{-1.064463in}{7.625528in}}%
\pgfpathlineto{\pgfqpoint{-1.052069in}{7.600000in}}%
\pgfpathlineto{\pgfqpoint{-1.064463in}{7.584428in}}%
\pgfpathclose%
\pgfpathmoveto{\pgfqpoint{-1.121946in}{7.706667in}}%
\pgfpathlineto{\pgfqpoint{-1.114050in}{7.731562in}}%
\pgfpathlineto{\pgfqpoint{-1.101977in}{7.706667in}}%
\pgfpathlineto{\pgfqpoint{-1.114050in}{7.690251in}}%
\pgfpathclose%
\pgfusepath{fill}%
\end{pgfscope}%
\begin{pgfscope}%
\pgfpathrectangle{\pgfqpoint{0.800000in}{0.528000in}}{\pgfqpoint{1.963636in}{3.696000in}} %
\pgfusepath{clip}%
\pgfsetbuttcap%
\pgfsetroundjoin%
\definecolor{currentfill}{rgb}{0.423689,0.000646,0.658956}%
\pgfsetfillcolor{currentfill}%
\pgfsetlinewidth{0.000000pt}%
\definecolor{currentstroke}{rgb}{0.000000,0.000000,0.000000}%
\pgfsetstrokecolor{currentstroke}%
\pgfsetdash{}{0pt}%
\pgfpathmoveto{\pgfqpoint{3.547107in}{-2.592811in}}%
\pgfpathlineto{\pgfqpoint{3.569411in}{-2.640000in}}%
\pgfpathlineto{\pgfqpoint{3.596694in}{-2.640000in}}%
\pgfpathlineto{\pgfqpoint{3.646281in}{-2.640000in}}%
\pgfpathlineto{\pgfqpoint{3.695868in}{-2.640000in}}%
\pgfpathlineto{\pgfqpoint{3.725317in}{-2.640000in}}%
\pgfpathlineto{\pgfqpoint{3.695868in}{-2.577413in}}%
\pgfpathlineto{\pgfqpoint{3.674851in}{-2.533333in}}%
\pgfpathlineto{\pgfqpoint{3.646281in}{-2.472609in}}%
\pgfpathlineto{\pgfqpoint{3.624374in}{-2.426667in}}%
\pgfpathlineto{\pgfqpoint{3.596694in}{-2.367829in}}%
\pgfpathlineto{\pgfqpoint{3.573886in}{-2.320000in}}%
\pgfpathlineto{\pgfqpoint{3.547107in}{-2.263073in}}%
\pgfpathlineto{\pgfqpoint{3.523385in}{-2.213333in}}%
\pgfpathlineto{\pgfqpoint{3.497521in}{-2.158344in}}%
\pgfpathlineto{\pgfqpoint{3.472872in}{-2.106667in}}%
\pgfpathlineto{\pgfqpoint{3.447934in}{-2.053641in}}%
\pgfpathlineto{\pgfqpoint{3.422346in}{-2.000000in}}%
\pgfpathlineto{\pgfqpoint{3.398347in}{-1.948967in}}%
\pgfpathlineto{\pgfqpoint{3.371806in}{-1.893333in}}%
\pgfpathlineto{\pgfqpoint{3.348760in}{-1.844322in}}%
\pgfpathlineto{\pgfqpoint{3.321252in}{-1.786667in}}%
\pgfpathlineto{\pgfqpoint{3.299174in}{-1.739707in}}%
\pgfpathlineto{\pgfqpoint{3.270683in}{-1.680000in}}%
\pgfpathlineto{\pgfqpoint{3.249587in}{-1.635125in}}%
\pgfpathlineto{\pgfqpoint{3.220098in}{-1.573333in}}%
\pgfpathlineto{\pgfqpoint{3.200000in}{-1.530576in}}%
\pgfpathlineto{\pgfqpoint{3.169497in}{-1.466667in}}%
\pgfpathlineto{\pgfqpoint{3.150413in}{-1.426062in}}%
\pgfpathlineto{\pgfqpoint{3.118879in}{-1.360000in}}%
\pgfpathlineto{\pgfqpoint{3.100826in}{-1.321585in}}%
\pgfpathlineto{\pgfqpoint{3.068243in}{-1.253333in}}%
\pgfpathlineto{\pgfqpoint{3.051240in}{-1.217146in}}%
\pgfpathlineto{\pgfqpoint{3.017589in}{-1.146667in}}%
\pgfpathlineto{\pgfqpoint{3.001653in}{-1.112748in}}%
\pgfpathlineto{\pgfqpoint{2.966914in}{-1.040000in}}%
\pgfpathlineto{\pgfqpoint{2.952066in}{-1.008392in}}%
\pgfpathlineto{\pgfqpoint{2.916219in}{-0.933333in}}%
\pgfpathlineto{\pgfqpoint{2.902479in}{-0.904081in}}%
\pgfpathlineto{\pgfqpoint{2.865502in}{-0.826667in}}%
\pgfpathlineto{\pgfqpoint{2.852893in}{-0.799816in}}%
\pgfpathlineto{\pgfqpoint{2.814763in}{-0.720000in}}%
\pgfpathlineto{\pgfqpoint{2.803306in}{-0.695600in}}%
\pgfpathlineto{\pgfqpoint{2.763999in}{-0.613333in}}%
\pgfpathlineto{\pgfqpoint{2.753719in}{-0.591437in}}%
\pgfpathlineto{\pgfqpoint{2.713210in}{-0.506667in}}%
\pgfpathlineto{\pgfqpoint{2.704132in}{-0.487327in}}%
\pgfpathlineto{\pgfqpoint{2.662395in}{-0.400000in}}%
\pgfpathlineto{\pgfqpoint{2.654545in}{-0.383276in}}%
\pgfpathlineto{\pgfqpoint{2.611551in}{-0.293333in}}%
\pgfpathlineto{\pgfqpoint{2.604959in}{-0.279285in}}%
\pgfpathlineto{\pgfqpoint{2.560677in}{-0.186667in}}%
\pgfpathlineto{\pgfqpoint{2.555372in}{-0.175359in}}%
\pgfpathlineto{\pgfqpoint{2.509772in}{-0.080000in}}%
\pgfpathlineto{\pgfqpoint{2.505785in}{-0.071501in}}%
\pgfpathlineto{\pgfqpoint{2.458833in}{0.026667in}}%
\pgfpathlineto{\pgfqpoint{2.456198in}{0.032285in}}%
\pgfpathlineto{\pgfqpoint{2.407860in}{0.133333in}}%
\pgfpathlineto{\pgfqpoint{2.406612in}{0.135995in}}%
\pgfpathlineto{\pgfqpoint{2.357025in}{0.239822in}}%
\pgfpathlineto{\pgfqpoint{2.356940in}{0.240000in}}%
\pgfpathlineto{\pgfqpoint{2.307438in}{0.345029in}}%
\pgfpathlineto{\pgfqpoint{2.306658in}{0.346667in}}%
\pgfpathlineto{\pgfqpoint{2.257851in}{0.450226in}}%
\pgfpathlineto{\pgfqpoint{2.256372in}{0.453333in}}%
\pgfpathlineto{\pgfqpoint{2.208264in}{0.555413in}}%
\pgfpathlineto{\pgfqpoint{2.206081in}{0.560000in}}%
\pgfpathlineto{\pgfqpoint{2.158678in}{0.660591in}}%
\pgfpathlineto{\pgfqpoint{2.155785in}{0.666667in}}%
\pgfpathlineto{\pgfqpoint{2.109091in}{0.765758in}}%
\pgfpathlineto{\pgfqpoint{2.105485in}{0.773333in}}%
\pgfpathlineto{\pgfqpoint{2.059504in}{0.870915in}}%
\pgfpathlineto{\pgfqpoint{2.055179in}{0.880000in}}%
\pgfpathlineto{\pgfqpoint{2.009917in}{0.976061in}}%
\pgfpathlineto{\pgfqpoint{2.004868in}{0.986667in}}%
\pgfpathlineto{\pgfqpoint{1.960331in}{1.081195in}}%
\pgfpathlineto{\pgfqpoint{1.954551in}{1.093333in}}%
\pgfpathlineto{\pgfqpoint{1.910744in}{1.186318in}}%
\pgfpathlineto{\pgfqpoint{1.904229in}{1.200000in}}%
\pgfpathlineto{\pgfqpoint{1.861157in}{1.291429in}}%
\pgfpathlineto{\pgfqpoint{1.853900in}{1.306667in}}%
\pgfpathlineto{\pgfqpoint{1.811570in}{1.396527in}}%
\pgfpathlineto{\pgfqpoint{1.803566in}{1.413333in}}%
\pgfpathlineto{\pgfqpoint{1.761983in}{1.501612in}}%
\pgfpathlineto{\pgfqpoint{1.753226in}{1.520000in}}%
\pgfpathlineto{\pgfqpoint{1.712397in}{1.606685in}}%
\pgfpathlineto{\pgfqpoint{1.702879in}{1.626667in}}%
\pgfpathlineto{\pgfqpoint{1.662810in}{1.711743in}}%
\pgfpathlineto{\pgfqpoint{1.652526in}{1.733333in}}%
\pgfpathlineto{\pgfqpoint{1.613223in}{1.816787in}}%
\pgfpathlineto{\pgfqpoint{1.602166in}{1.840000in}}%
\pgfpathlineto{\pgfqpoint{1.563636in}{1.921817in}}%
\pgfpathlineto{\pgfqpoint{1.551798in}{1.946667in}}%
\pgfpathlineto{\pgfqpoint{1.514050in}{2.026831in}}%
\pgfpathlineto{\pgfqpoint{1.501423in}{2.053333in}}%
\pgfpathlineto{\pgfqpoint{1.464463in}{2.131830in}}%
\pgfpathlineto{\pgfqpoint{1.451041in}{2.160000in}}%
\pgfpathlineto{\pgfqpoint{1.414876in}{2.236812in}}%
\pgfpathlineto{\pgfqpoint{1.400651in}{2.266667in}}%
\pgfpathlineto{\pgfqpoint{1.365289in}{2.341778in}}%
\pgfpathlineto{\pgfqpoint{1.350252in}{2.373333in}}%
\pgfpathlineto{\pgfqpoint{1.315702in}{2.446726in}}%
\pgfpathlineto{\pgfqpoint{1.299845in}{2.480000in}}%
\pgfpathlineto{\pgfqpoint{1.266116in}{2.551655in}}%
\pgfpathlineto{\pgfqpoint{1.249429in}{2.586667in}}%
\pgfpathlineto{\pgfqpoint{1.216529in}{2.656566in}}%
\pgfpathlineto{\pgfqpoint{1.199005in}{2.693333in}}%
\pgfpathlineto{\pgfqpoint{1.166942in}{2.761458in}}%
\pgfpathlineto{\pgfqpoint{1.148570in}{2.800000in}}%
\pgfpathlineto{\pgfqpoint{1.117355in}{2.866329in}}%
\pgfpathlineto{\pgfqpoint{1.098126in}{2.906667in}}%
\pgfpathlineto{\pgfqpoint{1.067769in}{2.971179in}}%
\pgfpathlineto{\pgfqpoint{1.047671in}{3.013333in}}%
\pgfpathlineto{\pgfqpoint{1.018182in}{3.076007in}}%
\pgfpathlineto{\pgfqpoint{0.997206in}{3.120000in}}%
\pgfpathlineto{\pgfqpoint{0.968595in}{3.180812in}}%
\pgfpathlineto{\pgfqpoint{0.946730in}{3.226667in}}%
\pgfpathlineto{\pgfqpoint{0.919008in}{3.285593in}}%
\pgfpathlineto{\pgfqpoint{0.896242in}{3.333333in}}%
\pgfpathlineto{\pgfqpoint{0.869421in}{3.390349in}}%
\pgfpathlineto{\pgfqpoint{0.845742in}{3.440000in}}%
\pgfpathlineto{\pgfqpoint{0.819835in}{3.495080in}}%
\pgfpathlineto{\pgfqpoint{0.795229in}{3.546667in}}%
\pgfpathlineto{\pgfqpoint{0.770248in}{3.599784in}}%
\pgfpathlineto{\pgfqpoint{0.744704in}{3.653333in}}%
\pgfpathlineto{\pgfqpoint{0.720661in}{3.704460in}}%
\pgfpathlineto{\pgfqpoint{0.694165in}{3.760000in}}%
\pgfpathlineto{\pgfqpoint{0.671074in}{3.809106in}}%
\pgfpathlineto{\pgfqpoint{0.643611in}{3.866667in}}%
\pgfpathlineto{\pgfqpoint{0.621488in}{3.913722in}}%
\pgfpathlineto{\pgfqpoint{0.593043in}{3.973333in}}%
\pgfpathlineto{\pgfqpoint{0.571901in}{4.018306in}}%
\pgfpathlineto{\pgfqpoint{0.542459in}{4.080000in}}%
\pgfpathlineto{\pgfqpoint{0.522314in}{4.122857in}}%
\pgfpathlineto{\pgfqpoint{0.491859in}{4.186667in}}%
\pgfpathlineto{\pgfqpoint{0.472727in}{4.227372in}}%
\pgfpathlineto{\pgfqpoint{0.441242in}{4.293333in}}%
\pgfpathlineto{\pgfqpoint{0.423140in}{4.331851in}}%
\pgfpathlineto{\pgfqpoint{0.390607in}{4.400000in}}%
\pgfpathlineto{\pgfqpoint{0.373554in}{4.436291in}}%
\pgfpathlineto{\pgfqpoint{0.339953in}{4.506667in}}%
\pgfpathlineto{\pgfqpoint{0.323967in}{4.540692in}}%
\pgfpathlineto{\pgfqpoint{0.289279in}{4.613333in}}%
\pgfpathlineto{\pgfqpoint{0.274380in}{4.645050in}}%
\pgfpathlineto{\pgfqpoint{0.238585in}{4.720000in}}%
\pgfpathlineto{\pgfqpoint{0.224793in}{4.749363in}}%
\pgfpathlineto{\pgfqpoint{0.187870in}{4.826667in}}%
\pgfpathlineto{\pgfqpoint{0.175207in}{4.853630in}}%
\pgfpathlineto{\pgfqpoint{0.137131in}{4.933333in}}%
\pgfpathlineto{\pgfqpoint{0.125620in}{4.957848in}}%
\pgfpathlineto{\pgfqpoint{0.086369in}{5.040000in}}%
\pgfpathlineto{\pgfqpoint{0.076033in}{5.062014in}}%
\pgfpathlineto{\pgfqpoint{0.035581in}{5.146667in}}%
\pgfpathlineto{\pgfqpoint{0.026446in}{5.166126in}}%
\pgfpathlineto{\pgfqpoint{-0.015234in}{5.253333in}}%
\pgfpathlineto{\pgfqpoint{-0.023140in}{5.270180in}}%
\pgfpathlineto{\pgfqpoint{-0.066076in}{5.360000in}}%
\pgfpathlineto{\pgfqpoint{-0.072727in}{5.374174in}}%
\pgfpathlineto{\pgfqpoint{-0.116948in}{5.466667in}}%
\pgfpathlineto{\pgfqpoint{-0.122314in}{5.478103in}}%
\pgfpathlineto{\pgfqpoint{-0.167852in}{5.573333in}}%
\pgfpathlineto{\pgfqpoint{-0.171901in}{5.581965in}}%
\pgfpathlineto{\pgfqpoint{-0.218789in}{5.680000in}}%
\pgfpathlineto{\pgfqpoint{-0.221488in}{5.685754in}}%
\pgfpathlineto{\pgfqpoint{-0.269761in}{5.786667in}}%
\pgfpathlineto{\pgfqpoint{-0.271074in}{5.789468in}}%
\pgfpathlineto{\pgfqpoint{-0.320661in}{5.893223in}}%
\pgfpathlineto{\pgfqpoint{-0.320714in}{5.893333in}}%
\pgfpathlineto{\pgfqpoint{-0.370248in}{5.998430in}}%
\pgfpathlineto{\pgfqpoint{-0.370995in}{6.000000in}}%
\pgfpathlineto{\pgfqpoint{-0.419835in}{6.103628in}}%
\pgfpathlineto{\pgfqpoint{-0.421281in}{6.106667in}}%
\pgfpathlineto{\pgfqpoint{-0.469421in}{6.208816in}}%
\pgfpathlineto{\pgfqpoint{-0.471572in}{6.213333in}}%
\pgfpathlineto{\pgfqpoint{-0.519008in}{6.313994in}}%
\pgfpathlineto{\pgfqpoint{-0.521868in}{6.320000in}}%
\pgfpathlineto{\pgfqpoint{-0.568595in}{6.419162in}}%
\pgfpathlineto{\pgfqpoint{-0.572168in}{6.426667in}}%
\pgfpathlineto{\pgfqpoint{-0.618182in}{6.524319in}}%
\pgfpathlineto{\pgfqpoint{-0.622474in}{6.533333in}}%
\pgfpathlineto{\pgfqpoint{-0.667769in}{6.629465in}}%
\pgfpathlineto{\pgfqpoint{-0.672785in}{6.640000in}}%
\pgfpathlineto{\pgfqpoint{-0.717355in}{6.734600in}}%
\pgfpathlineto{\pgfqpoint{-0.723101in}{6.746667in}}%
\pgfpathlineto{\pgfqpoint{-0.766942in}{6.839723in}}%
\pgfpathlineto{\pgfqpoint{-0.773423in}{6.853333in}}%
\pgfpathlineto{\pgfqpoint{-0.816529in}{6.944835in}}%
\pgfpathlineto{\pgfqpoint{-0.823751in}{6.960000in}}%
\pgfpathlineto{\pgfqpoint{-0.866116in}{7.049933in}}%
\pgfpathlineto{\pgfqpoint{-0.874085in}{7.066667in}}%
\pgfpathlineto{\pgfqpoint{-0.915702in}{7.155020in}}%
\pgfpathlineto{\pgfqpoint{-0.924425in}{7.173333in}}%
\pgfpathlineto{\pgfqpoint{-0.965289in}{7.260092in}}%
\pgfpathlineto{\pgfqpoint{-0.974771in}{7.280000in}}%
\pgfpathlineto{\pgfqpoint{-1.014876in}{7.365151in}}%
\pgfpathlineto{\pgfqpoint{-1.025124in}{7.386667in}}%
\pgfpathlineto{\pgfqpoint{-1.064463in}{7.470196in}}%
\pgfpathlineto{\pgfqpoint{-1.075484in}{7.493333in}}%
\pgfpathlineto{\pgfqpoint{-1.114050in}{7.575227in}}%
\pgfpathlineto{\pgfqpoint{-1.125851in}{7.600000in}}%
\pgfpathlineto{\pgfqpoint{-1.163636in}{7.680242in}}%
\pgfpathlineto{\pgfqpoint{-1.163636in}{7.600000in}}%
\pgfpathlineto{\pgfqpoint{-1.163636in}{7.493333in}}%
\pgfpathlineto{\pgfqpoint{-1.163636in}{7.386667in}}%
\pgfpathlineto{\pgfqpoint{-1.163636in}{7.352672in}}%
\pgfpathlineto{\pgfqpoint{-1.129286in}{7.280000in}}%
\pgfpathlineto{\pgfqpoint{-1.114050in}{7.247897in}}%
\pgfpathlineto{\pgfqpoint{-1.078804in}{7.173333in}}%
\pgfpathlineto{\pgfqpoint{-1.064463in}{7.143115in}}%
\pgfpathlineto{\pgfqpoint{-1.028326in}{7.066667in}}%
\pgfpathlineto{\pgfqpoint{-1.014876in}{7.038326in}}%
\pgfpathlineto{\pgfqpoint{-0.977852in}{6.960000in}}%
\pgfpathlineto{\pgfqpoint{-0.965289in}{6.933529in}}%
\pgfpathlineto{\pgfqpoint{-0.927380in}{6.853333in}}%
\pgfpathlineto{\pgfqpoint{-0.915702in}{6.828726in}}%
\pgfpathlineto{\pgfqpoint{-0.876913in}{6.746667in}}%
\pgfpathlineto{\pgfqpoint{-0.866116in}{6.723916in}}%
\pgfpathlineto{\pgfqpoint{-0.826448in}{6.640000in}}%
\pgfpathlineto{\pgfqpoint{-0.816529in}{6.619099in}}%
\pgfpathlineto{\pgfqpoint{-0.775986in}{6.533333in}}%
\pgfpathlineto{\pgfqpoint{-0.766942in}{6.514275in}}%
\pgfpathlineto{\pgfqpoint{-0.725528in}{6.426667in}}%
\pgfpathlineto{\pgfqpoint{-0.717355in}{6.409445in}}%
\pgfpathlineto{\pgfqpoint{-0.675073in}{6.320000in}}%
\pgfpathlineto{\pgfqpoint{-0.667769in}{6.304608in}}%
\pgfpathlineto{\pgfqpoint{-0.624621in}{6.213333in}}%
\pgfpathlineto{\pgfqpoint{-0.618182in}{6.199764in}}%
\pgfpathlineto{\pgfqpoint{-0.574172in}{6.106667in}}%
\pgfpathlineto{\pgfqpoint{-0.568595in}{6.094915in}}%
\pgfpathlineto{\pgfqpoint{-0.523726in}{6.000000in}}%
\pgfpathlineto{\pgfqpoint{-0.519008in}{5.990058in}}%
\pgfpathlineto{\pgfqpoint{-0.473283in}{5.893333in}}%
\pgfpathlineto{\pgfqpoint{-0.469421in}{5.885196in}}%
\pgfpathlineto{\pgfqpoint{-0.422843in}{5.786667in}}%
\pgfpathlineto{\pgfqpoint{-0.419835in}{5.780327in}}%
\pgfpathlineto{\pgfqpoint{-0.372406in}{5.680000in}}%
\pgfpathlineto{\pgfqpoint{-0.370248in}{5.675452in}}%
\pgfpathlineto{\pgfqpoint{-0.321972in}{5.573333in}}%
\pgfpathlineto{\pgfqpoint{-0.320661in}{5.570571in}}%
\pgfpathlineto{\pgfqpoint{-0.271541in}{5.466667in}}%
\pgfpathlineto{\pgfqpoint{-0.271074in}{5.465684in}}%
\pgfpathlineto{\pgfqpoint{-0.221488in}{5.360992in}}%
\pgfpathlineto{\pgfqpoint{-0.221019in}{5.360000in}}%
\pgfpathlineto{\pgfqpoint{-0.171901in}{5.256539in}}%
\pgfpathlineto{\pgfqpoint{-0.170386in}{5.253333in}}%
\pgfpathlineto{\pgfqpoint{-0.122314in}{5.152075in}}%
\pgfpathlineto{\pgfqpoint{-0.119759in}{5.146667in}}%
\pgfpathlineto{\pgfqpoint{-0.072727in}{5.047599in}}%
\pgfpathlineto{\pgfqpoint{-0.069136in}{5.040000in}}%
\pgfpathlineto{\pgfqpoint{-0.023140in}{4.943113in}}%
\pgfpathlineto{\pgfqpoint{-0.018519in}{4.933333in}}%
\pgfpathlineto{\pgfqpoint{0.026446in}{4.838615in}}%
\pgfpathlineto{\pgfqpoint{0.032093in}{4.826667in}}%
\pgfpathlineto{\pgfqpoint{0.076033in}{4.734107in}}%
\pgfpathlineto{\pgfqpoint{0.082699in}{4.720000in}}%
\pgfpathlineto{\pgfqpoint{0.125620in}{4.629588in}}%
\pgfpathlineto{\pgfqpoint{0.133301in}{4.613333in}}%
\pgfpathlineto{\pgfqpoint{0.175207in}{4.525058in}}%
\pgfpathlineto{\pgfqpoint{0.183898in}{4.506667in}}%
\pgfpathlineto{\pgfqpoint{0.224793in}{4.420518in}}%
\pgfpathlineto{\pgfqpoint{0.234490in}{4.400000in}}%
\pgfpathlineto{\pgfqpoint{0.274380in}{4.315968in}}%
\pgfpathlineto{\pgfqpoint{0.285077in}{4.293333in}}%
\pgfpathlineto{\pgfqpoint{0.323967in}{4.211408in}}%
\pgfpathlineto{\pgfqpoint{0.335659in}{4.186667in}}%
\pgfpathlineto{\pgfqpoint{0.373554in}{4.106838in}}%
\pgfpathlineto{\pgfqpoint{0.386237in}{4.080000in}}%
\pgfpathlineto{\pgfqpoint{0.423140in}{4.002258in}}%
\pgfpathlineto{\pgfqpoint{0.436810in}{3.973333in}}%
\pgfpathlineto{\pgfqpoint{0.472727in}{3.897668in}}%
\pgfpathlineto{\pgfqpoint{0.487378in}{3.866667in}}%
\pgfpathlineto{\pgfqpoint{0.522314in}{3.793069in}}%
\pgfpathlineto{\pgfqpoint{0.537942in}{3.760000in}}%
\pgfpathlineto{\pgfqpoint{0.571901in}{3.688460in}}%
\pgfpathlineto{\pgfqpoint{0.588502in}{3.653333in}}%
\pgfpathlineto{\pgfqpoint{0.621488in}{3.583842in}}%
\pgfpathlineto{\pgfqpoint{0.639057in}{3.546667in}}%
\pgfpathlineto{\pgfqpoint{0.671074in}{3.479215in}}%
\pgfpathlineto{\pgfqpoint{0.689608in}{3.440000in}}%
\pgfpathlineto{\pgfqpoint{0.720661in}{3.374578in}}%
\pgfpathlineto{\pgfqpoint{0.740154in}{3.333333in}}%
\pgfpathlineto{\pgfqpoint{0.770248in}{3.269933in}}%
\pgfpathlineto{\pgfqpoint{0.790697in}{3.226667in}}%
\pgfpathlineto{\pgfqpoint{0.819835in}{3.165279in}}%
\pgfpathlineto{\pgfqpoint{0.841235in}{3.120000in}}%
\pgfpathlineto{\pgfqpoint{0.869421in}{3.060616in}}%
\pgfpathlineto{\pgfqpoint{0.891769in}{3.013333in}}%
\pgfpathlineto{\pgfqpoint{0.919008in}{2.955944in}}%
\pgfpathlineto{\pgfqpoint{0.942299in}{2.906667in}}%
\pgfpathlineto{\pgfqpoint{0.968595in}{2.851264in}}%
\pgfpathlineto{\pgfqpoint{0.992825in}{2.800000in}}%
\pgfpathlineto{\pgfqpoint{1.018182in}{2.746575in}}%
\pgfpathlineto{\pgfqpoint{1.043346in}{2.693333in}}%
\pgfpathlineto{\pgfqpoint{1.067769in}{2.641879in}}%
\pgfpathlineto{\pgfqpoint{1.093865in}{2.586667in}}%
\pgfpathlineto{\pgfqpoint{1.117355in}{2.537174in}}%
\pgfpathlineto{\pgfqpoint{1.144379in}{2.480000in}}%
\pgfpathlineto{\pgfqpoint{1.166942in}{2.432461in}}%
\pgfpathlineto{\pgfqpoint{1.194889in}{2.373333in}}%
\pgfpathlineto{\pgfqpoint{1.216529in}{2.327739in}}%
\pgfpathlineto{\pgfqpoint{1.245396in}{2.266667in}}%
\pgfpathlineto{\pgfqpoint{1.266116in}{2.223010in}}%
\pgfpathlineto{\pgfqpoint{1.295899in}{2.160000in}}%
\pgfpathlineto{\pgfqpoint{1.315702in}{2.118274in}}%
\pgfpathlineto{\pgfqpoint{1.346398in}{2.053333in}}%
\pgfpathlineto{\pgfqpoint{1.365289in}{2.013529in}}%
\pgfpathlineto{\pgfqpoint{1.396893in}{1.946667in}}%
\pgfpathlineto{\pgfqpoint{1.414876in}{1.908777in}}%
\pgfpathlineto{\pgfqpoint{1.447385in}{1.840000in}}%
\pgfpathlineto{\pgfqpoint{1.464463in}{1.804017in}}%
\pgfpathlineto{\pgfqpoint{1.497874in}{1.733333in}}%
\pgfpathlineto{\pgfqpoint{1.514050in}{1.699250in}}%
\pgfpathlineto{\pgfqpoint{1.548359in}{1.626667in}}%
\pgfpathlineto{\pgfqpoint{1.563636in}{1.594476in}}%
\pgfpathlineto{\pgfqpoint{1.598840in}{1.520000in}}%
\pgfpathlineto{\pgfqpoint{1.613223in}{1.489694in}}%
\pgfpathlineto{\pgfqpoint{1.649318in}{1.413333in}}%
\pgfpathlineto{\pgfqpoint{1.662810in}{1.384905in}}%
\pgfpathlineto{\pgfqpoint{1.699793in}{1.306667in}}%
\pgfpathlineto{\pgfqpoint{1.712397in}{1.280109in}}%
\pgfpathlineto{\pgfqpoint{1.750264in}{1.200000in}}%
\pgfpathlineto{\pgfqpoint{1.761983in}{1.175306in}}%
\pgfpathlineto{\pgfqpoint{1.800732in}{1.093333in}}%
\pgfpathlineto{\pgfqpoint{1.811570in}{1.070497in}}%
\pgfpathlineto{\pgfqpoint{1.851197in}{0.986667in}}%
\pgfpathlineto{\pgfqpoint{1.861157in}{0.965680in}}%
\pgfpathlineto{\pgfqpoint{1.901659in}{0.880000in}}%
\pgfpathlineto{\pgfqpoint{1.910744in}{0.860856in}}%
\pgfpathlineto{\pgfqpoint{1.952117in}{0.773333in}}%
\pgfpathlineto{\pgfqpoint{1.960331in}{0.756026in}}%
\pgfpathlineto{\pgfqpoint{2.002573in}{0.666667in}}%
\pgfpathlineto{\pgfqpoint{2.009917in}{0.651190in}}%
\pgfpathlineto{\pgfqpoint{2.053025in}{0.560000in}}%
\pgfpathlineto{\pgfqpoint{2.059504in}{0.546347in}}%
\pgfpathlineto{\pgfqpoint{2.103474in}{0.453333in}}%
\pgfpathlineto{\pgfqpoint{2.109091in}{0.441497in}}%
\pgfpathlineto{\pgfqpoint{2.153920in}{0.346667in}}%
\pgfpathlineto{\pgfqpoint{2.158678in}{0.336641in}}%
\pgfpathlineto{\pgfqpoint{2.204363in}{0.240000in}}%
\pgfpathlineto{\pgfqpoint{2.208264in}{0.231779in}}%
\pgfpathlineto{\pgfqpoint{2.254803in}{0.133333in}}%
\pgfpathlineto{\pgfqpoint{2.257851in}{0.126910in}}%
\pgfpathlineto{\pgfqpoint{2.305240in}{0.026667in}}%
\pgfpathlineto{\pgfqpoint{2.307438in}{0.022035in}}%
\pgfpathlineto{\pgfqpoint{2.355675in}{-0.080000in}}%
\pgfpathlineto{\pgfqpoint{2.357025in}{-0.082845in}}%
\pgfpathlineto{\pgfqpoint{2.406106in}{-0.186667in}}%
\pgfpathlineto{\pgfqpoint{2.406612in}{-0.187732in}}%
\pgfpathlineto{\pgfqpoint{2.456198in}{-0.292445in}}%
\pgfpathlineto{\pgfqpoint{2.456618in}{-0.293333in}}%
\pgfpathlineto{\pgfqpoint{2.505785in}{-0.396897in}}%
\pgfpathlineto{\pgfqpoint{2.507251in}{-0.400000in}}%
\pgfpathlineto{\pgfqpoint{2.555372in}{-0.501361in}}%
\pgfpathlineto{\pgfqpoint{2.557879in}{-0.506667in}}%
\pgfpathlineto{\pgfqpoint{2.604959in}{-0.605836in}}%
\pgfpathlineto{\pgfqpoint{2.608502in}{-0.613333in}}%
\pgfpathlineto{\pgfqpoint{2.654545in}{-0.710322in}}%
\pgfpathlineto{\pgfqpoint{2.659119in}{-0.720000in}}%
\pgfpathlineto{\pgfqpoint{2.704132in}{-0.814819in}}%
\pgfpathlineto{\pgfqpoint{2.709731in}{-0.826667in}}%
\pgfpathlineto{\pgfqpoint{2.753719in}{-0.919327in}}%
\pgfpathlineto{\pgfqpoint{2.760338in}{-0.933333in}}%
\pgfpathlineto{\pgfqpoint{2.803306in}{-1.023845in}}%
\pgfpathlineto{\pgfqpoint{2.810940in}{-1.040000in}}%
\pgfpathlineto{\pgfqpoint{2.852893in}{-1.128374in}}%
\pgfpathlineto{\pgfqpoint{2.861537in}{-1.146667in}}%
\pgfpathlineto{\pgfqpoint{2.902479in}{-1.232914in}}%
\pgfpathlineto{\pgfqpoint{2.912129in}{-1.253333in}}%
\pgfpathlineto{\pgfqpoint{2.952066in}{-1.337463in}}%
\pgfpathlineto{\pgfqpoint{2.962716in}{-1.360000in}}%
\pgfpathlineto{\pgfqpoint{3.001653in}{-1.442023in}}%
\pgfpathlineto{\pgfqpoint{3.013299in}{-1.466667in}}%
\pgfpathlineto{\pgfqpoint{3.051240in}{-1.546593in}}%
\pgfpathlineto{\pgfqpoint{3.063877in}{-1.573333in}}%
\pgfpathlineto{\pgfqpoint{3.100826in}{-1.651172in}}%
\pgfpathlineto{\pgfqpoint{3.114450in}{-1.680000in}}%
\pgfpathlineto{\pgfqpoint{3.150413in}{-1.755762in}}%
\pgfpathlineto{\pgfqpoint{3.165019in}{-1.786667in}}%
\pgfpathlineto{\pgfqpoint{3.200000in}{-1.860361in}}%
\pgfpathlineto{\pgfqpoint{3.215583in}{-1.893333in}}%
\pgfpathlineto{\pgfqpoint{3.249587in}{-1.964969in}}%
\pgfpathlineto{\pgfqpoint{3.266143in}{-2.000000in}}%
\pgfpathlineto{\pgfqpoint{3.299174in}{-2.069587in}}%
\pgfpathlineto{\pgfqpoint{3.316698in}{-2.106667in}}%
\pgfpathlineto{\pgfqpoint{3.348760in}{-2.174213in}}%
\pgfpathlineto{\pgfqpoint{3.367249in}{-2.213333in}}%
\pgfpathlineto{\pgfqpoint{3.398347in}{-2.278849in}}%
\pgfpathlineto{\pgfqpoint{3.417796in}{-2.320000in}}%
\pgfpathlineto{\pgfqpoint{3.447934in}{-2.383494in}}%
\pgfpathlineto{\pgfqpoint{3.468338in}{-2.426667in}}%
\pgfpathlineto{\pgfqpoint{3.497521in}{-2.488148in}}%
\pgfpathlineto{\pgfqpoint{3.518876in}{-2.533333in}}%
\pgfpathclose%
\pgfusepath{fill}%
\end{pgfscope}%
\begin{pgfscope}%
\pgfpathrectangle{\pgfqpoint{0.800000in}{0.528000in}}{\pgfqpoint{1.963636in}{3.696000in}} %
\pgfusepath{clip}%
\pgfsetbuttcap%
\pgfsetroundjoin%
\definecolor{currentfill}{rgb}{0.423689,0.000646,0.658956}%
\pgfsetfillcolor{currentfill}%
\pgfsetlinewidth{0.000000pt}%
\definecolor{currentstroke}{rgb}{0.000000,0.000000,0.000000}%
\pgfsetstrokecolor{currentstroke}%
\pgfsetdash{}{0pt}%
\pgfpathmoveto{\pgfqpoint{3.745455in}{-2.400242in}}%
\pgfpathlineto{\pgfqpoint{3.745455in}{-2.320000in}}%
\pgfpathlineto{\pgfqpoint{3.745455in}{-2.213333in}}%
\pgfpathlineto{\pgfqpoint{3.745455in}{-2.106667in}}%
\pgfpathlineto{\pgfqpoint{3.745455in}{-2.072672in}}%
\pgfpathlineto{\pgfqpoint{3.711104in}{-2.000000in}}%
\pgfpathlineto{\pgfqpoint{3.695868in}{-1.967897in}}%
\pgfpathlineto{\pgfqpoint{3.660622in}{-1.893333in}}%
\pgfpathlineto{\pgfqpoint{3.646281in}{-1.863115in}}%
\pgfpathlineto{\pgfqpoint{3.610144in}{-1.786667in}}%
\pgfpathlineto{\pgfqpoint{3.596694in}{-1.758326in}}%
\pgfpathlineto{\pgfqpoint{3.559670in}{-1.680000in}}%
\pgfpathlineto{\pgfqpoint{3.547107in}{-1.653529in}}%
\pgfpathlineto{\pgfqpoint{3.509199in}{-1.573333in}}%
\pgfpathlineto{\pgfqpoint{3.497521in}{-1.548726in}}%
\pgfpathlineto{\pgfqpoint{3.458731in}{-1.466667in}}%
\pgfpathlineto{\pgfqpoint{3.447934in}{-1.443916in}}%
\pgfpathlineto{\pgfqpoint{3.408266in}{-1.360000in}}%
\pgfpathlineto{\pgfqpoint{3.398347in}{-1.339099in}}%
\pgfpathlineto{\pgfqpoint{3.357805in}{-1.253333in}}%
\pgfpathlineto{\pgfqpoint{3.348760in}{-1.234275in}}%
\pgfpathlineto{\pgfqpoint{3.307346in}{-1.146667in}}%
\pgfpathlineto{\pgfqpoint{3.299174in}{-1.129445in}}%
\pgfpathlineto{\pgfqpoint{3.256891in}{-1.040000in}}%
\pgfpathlineto{\pgfqpoint{3.249587in}{-1.024608in}}%
\pgfpathlineto{\pgfqpoint{3.206439in}{-0.933333in}}%
\pgfpathlineto{\pgfqpoint{3.200000in}{-0.919764in}}%
\pgfpathlineto{\pgfqpoint{3.155990in}{-0.826667in}}%
\pgfpathlineto{\pgfqpoint{3.150413in}{-0.814915in}}%
\pgfpathlineto{\pgfqpoint{3.105544in}{-0.720000in}}%
\pgfpathlineto{\pgfqpoint{3.100826in}{-0.710058in}}%
\pgfpathlineto{\pgfqpoint{3.055101in}{-0.613333in}}%
\pgfpathlineto{\pgfqpoint{3.051240in}{-0.605196in}}%
\pgfpathlineto{\pgfqpoint{3.004661in}{-0.506667in}}%
\pgfpathlineto{\pgfqpoint{3.001653in}{-0.500327in}}%
\pgfpathlineto{\pgfqpoint{2.954224in}{-0.400000in}}%
\pgfpathlineto{\pgfqpoint{2.952066in}{-0.395452in}}%
\pgfpathlineto{\pgfqpoint{2.903790in}{-0.293333in}}%
\pgfpathlineto{\pgfqpoint{2.902479in}{-0.290571in}}%
\pgfpathlineto{\pgfqpoint{2.853359in}{-0.186667in}}%
\pgfpathlineto{\pgfqpoint{2.852893in}{-0.185684in}}%
\pgfpathlineto{\pgfqpoint{2.803306in}{-0.080992in}}%
\pgfpathlineto{\pgfqpoint{2.802837in}{-0.080000in}}%
\pgfpathlineto{\pgfqpoint{2.753719in}{0.023461in}}%
\pgfpathlineto{\pgfqpoint{2.752204in}{0.026667in}}%
\pgfpathlineto{\pgfqpoint{2.704132in}{0.127925in}}%
\pgfpathlineto{\pgfqpoint{2.701577in}{0.133333in}}%
\pgfpathlineto{\pgfqpoint{2.654545in}{0.232401in}}%
\pgfpathlineto{\pgfqpoint{2.650954in}{0.240000in}}%
\pgfpathlineto{\pgfqpoint{2.604959in}{0.336887in}}%
\pgfpathlineto{\pgfqpoint{2.600337in}{0.346667in}}%
\pgfpathlineto{\pgfqpoint{2.555372in}{0.441385in}}%
\pgfpathlineto{\pgfqpoint{2.549726in}{0.453333in}}%
\pgfpathlineto{\pgfqpoint{2.505785in}{0.545893in}}%
\pgfpathlineto{\pgfqpoint{2.499119in}{0.560000in}}%
\pgfpathlineto{\pgfqpoint{2.456198in}{0.650412in}}%
\pgfpathlineto{\pgfqpoint{2.448517in}{0.666667in}}%
\pgfpathlineto{\pgfqpoint{2.406612in}{0.754942in}}%
\pgfpathlineto{\pgfqpoint{2.397920in}{0.773333in}}%
\pgfpathlineto{\pgfqpoint{2.357025in}{0.859482in}}%
\pgfpathlineto{\pgfqpoint{2.347328in}{0.880000in}}%
\pgfpathlineto{\pgfqpoint{2.307438in}{0.964032in}}%
\pgfpathlineto{\pgfqpoint{2.296741in}{0.986667in}}%
\pgfpathlineto{\pgfqpoint{2.257851in}{1.068592in}}%
\pgfpathlineto{\pgfqpoint{2.246159in}{1.093333in}}%
\pgfpathlineto{\pgfqpoint{2.208264in}{1.173162in}}%
\pgfpathlineto{\pgfqpoint{2.195581in}{1.200000in}}%
\pgfpathlineto{\pgfqpoint{2.158678in}{1.277742in}}%
\pgfpathlineto{\pgfqpoint{2.145008in}{1.306667in}}%
\pgfpathlineto{\pgfqpoint{2.109091in}{1.382332in}}%
\pgfpathlineto{\pgfqpoint{2.094440in}{1.413333in}}%
\pgfpathlineto{\pgfqpoint{2.059504in}{1.486931in}}%
\pgfpathlineto{\pgfqpoint{2.043876in}{1.520000in}}%
\pgfpathlineto{\pgfqpoint{2.009917in}{1.591540in}}%
\pgfpathlineto{\pgfqpoint{1.993316in}{1.626667in}}%
\pgfpathlineto{\pgfqpoint{1.960331in}{1.696158in}}%
\pgfpathlineto{\pgfqpoint{1.942761in}{1.733333in}}%
\pgfpathlineto{\pgfqpoint{1.910744in}{1.800785in}}%
\pgfpathlineto{\pgfqpoint{1.892210in}{1.840000in}}%
\pgfpathlineto{\pgfqpoint{1.861157in}{1.905422in}}%
\pgfpathlineto{\pgfqpoint{1.841664in}{1.946667in}}%
\pgfpathlineto{\pgfqpoint{1.811570in}{2.010067in}}%
\pgfpathlineto{\pgfqpoint{1.791122in}{2.053333in}}%
\pgfpathlineto{\pgfqpoint{1.761983in}{2.114721in}}%
\pgfpathlineto{\pgfqpoint{1.740584in}{2.160000in}}%
\pgfpathlineto{\pgfqpoint{1.712397in}{2.219384in}}%
\pgfpathlineto{\pgfqpoint{1.690050in}{2.266667in}}%
\pgfpathlineto{\pgfqpoint{1.662810in}{2.324056in}}%
\pgfpathlineto{\pgfqpoint{1.639520in}{2.373333in}}%
\pgfpathlineto{\pgfqpoint{1.613223in}{2.428736in}}%
\pgfpathlineto{\pgfqpoint{1.588994in}{2.480000in}}%
\pgfpathlineto{\pgfqpoint{1.563636in}{2.533425in}}%
\pgfpathlineto{\pgfqpoint{1.538472in}{2.586667in}}%
\pgfpathlineto{\pgfqpoint{1.514050in}{2.638121in}}%
\pgfpathlineto{\pgfqpoint{1.487954in}{2.693333in}}%
\pgfpathlineto{\pgfqpoint{1.464463in}{2.742826in}}%
\pgfpathlineto{\pgfqpoint{1.437439in}{2.800000in}}%
\pgfpathlineto{\pgfqpoint{1.414876in}{2.847539in}}%
\pgfpathlineto{\pgfqpoint{1.386929in}{2.906667in}}%
\pgfpathlineto{\pgfqpoint{1.365289in}{2.952261in}}%
\pgfpathlineto{\pgfqpoint{1.336423in}{3.013333in}}%
\pgfpathlineto{\pgfqpoint{1.315702in}{3.056990in}}%
\pgfpathlineto{\pgfqpoint{1.285920in}{3.120000in}}%
\pgfpathlineto{\pgfqpoint{1.266116in}{3.161726in}}%
\pgfpathlineto{\pgfqpoint{1.235420in}{3.226667in}}%
\pgfpathlineto{\pgfqpoint{1.216529in}{3.266471in}}%
\pgfpathlineto{\pgfqpoint{1.184925in}{3.333333in}}%
\pgfpathlineto{\pgfqpoint{1.166942in}{3.371223in}}%
\pgfpathlineto{\pgfqpoint{1.134433in}{3.440000in}}%
\pgfpathlineto{\pgfqpoint{1.117355in}{3.475983in}}%
\pgfpathlineto{\pgfqpoint{1.083944in}{3.546667in}}%
\pgfpathlineto{\pgfqpoint{1.067769in}{3.580750in}}%
\pgfpathlineto{\pgfqpoint{1.033459in}{3.653333in}}%
\pgfpathlineto{\pgfqpoint{1.018182in}{3.685524in}}%
\pgfpathlineto{\pgfqpoint{0.982978in}{3.760000in}}%
\pgfpathlineto{\pgfqpoint{0.968595in}{3.790306in}}%
\pgfpathlineto{\pgfqpoint{0.932500in}{3.866667in}}%
\pgfpathlineto{\pgfqpoint{0.919008in}{3.895095in}}%
\pgfpathlineto{\pgfqpoint{0.882025in}{3.973333in}}%
\pgfpathlineto{\pgfqpoint{0.869421in}{3.999891in}}%
\pgfpathlineto{\pgfqpoint{0.831554in}{4.080000in}}%
\pgfpathlineto{\pgfqpoint{0.819835in}{4.104694in}}%
\pgfpathlineto{\pgfqpoint{0.781086in}{4.186667in}}%
\pgfpathlineto{\pgfqpoint{0.770248in}{4.209503in}}%
\pgfpathlineto{\pgfqpoint{0.730621in}{4.293333in}}%
\pgfpathlineto{\pgfqpoint{0.720661in}{4.314320in}}%
\pgfpathlineto{\pgfqpoint{0.680159in}{4.400000in}}%
\pgfpathlineto{\pgfqpoint{0.671074in}{4.419144in}}%
\pgfpathlineto{\pgfqpoint{0.629701in}{4.506667in}}%
\pgfpathlineto{\pgfqpoint{0.621488in}{4.523974in}}%
\pgfpathlineto{\pgfqpoint{0.579246in}{4.613333in}}%
\pgfpathlineto{\pgfqpoint{0.571901in}{4.628810in}}%
\pgfpathlineto{\pgfqpoint{0.528793in}{4.720000in}}%
\pgfpathlineto{\pgfqpoint{0.522314in}{4.733653in}}%
\pgfpathlineto{\pgfqpoint{0.478344in}{4.826667in}}%
\pgfpathlineto{\pgfqpoint{0.472727in}{4.838503in}}%
\pgfpathlineto{\pgfqpoint{0.427898in}{4.933333in}}%
\pgfpathlineto{\pgfqpoint{0.423140in}{4.943359in}}%
\pgfpathlineto{\pgfqpoint{0.377455in}{5.040000in}}%
\pgfpathlineto{\pgfqpoint{0.373554in}{5.048221in}}%
\pgfpathlineto{\pgfqpoint{0.327015in}{5.146667in}}%
\pgfpathlineto{\pgfqpoint{0.323967in}{5.153090in}}%
\pgfpathlineto{\pgfqpoint{0.276578in}{5.253333in}}%
\pgfpathlineto{\pgfqpoint{0.274380in}{5.257965in}}%
\pgfpathlineto{\pgfqpoint{0.226144in}{5.360000in}}%
\pgfpathlineto{\pgfqpoint{0.224793in}{5.362845in}}%
\pgfpathlineto{\pgfqpoint{0.175712in}{5.466667in}}%
\pgfpathlineto{\pgfqpoint{0.175207in}{5.467732in}}%
\pgfpathlineto{\pgfqpoint{0.125620in}{5.572445in}}%
\pgfpathlineto{\pgfqpoint{0.125200in}{5.573333in}}%
\pgfpathlineto{\pgfqpoint{0.076033in}{5.676897in}}%
\pgfpathlineto{\pgfqpoint{0.074567in}{5.680000in}}%
\pgfpathlineto{\pgfqpoint{0.026446in}{5.781361in}}%
\pgfpathlineto{\pgfqpoint{0.023939in}{5.786667in}}%
\pgfpathlineto{\pgfqpoint{-0.023140in}{5.885836in}}%
\pgfpathlineto{\pgfqpoint{-0.026683in}{5.893333in}}%
\pgfpathlineto{\pgfqpoint{-0.072727in}{5.990322in}}%
\pgfpathlineto{\pgfqpoint{-0.077301in}{6.000000in}}%
\pgfpathlineto{\pgfqpoint{-0.122314in}{6.094819in}}%
\pgfpathlineto{\pgfqpoint{-0.127913in}{6.106667in}}%
\pgfpathlineto{\pgfqpoint{-0.171901in}{6.199327in}}%
\pgfpathlineto{\pgfqpoint{-0.178520in}{6.213333in}}%
\pgfpathlineto{\pgfqpoint{-0.221488in}{6.303845in}}%
\pgfpathlineto{\pgfqpoint{-0.229122in}{6.320000in}}%
\pgfpathlineto{\pgfqpoint{-0.271074in}{6.408374in}}%
\pgfpathlineto{\pgfqpoint{-0.279719in}{6.426667in}}%
\pgfpathlineto{\pgfqpoint{-0.320661in}{6.512914in}}%
\pgfpathlineto{\pgfqpoint{-0.330311in}{6.533333in}}%
\pgfpathlineto{\pgfqpoint{-0.370248in}{6.617463in}}%
\pgfpathlineto{\pgfqpoint{-0.380898in}{6.640000in}}%
\pgfpathlineto{\pgfqpoint{-0.419835in}{6.722023in}}%
\pgfpathlineto{\pgfqpoint{-0.431481in}{6.746667in}}%
\pgfpathlineto{\pgfqpoint{-0.469421in}{6.826593in}}%
\pgfpathlineto{\pgfqpoint{-0.482059in}{6.853333in}}%
\pgfpathlineto{\pgfqpoint{-0.519008in}{6.931172in}}%
\pgfpathlineto{\pgfqpoint{-0.532632in}{6.960000in}}%
\pgfpathlineto{\pgfqpoint{-0.568595in}{7.035762in}}%
\pgfpathlineto{\pgfqpoint{-0.583201in}{7.066667in}}%
\pgfpathlineto{\pgfqpoint{-0.618182in}{7.140361in}}%
\pgfpathlineto{\pgfqpoint{-0.633765in}{7.173333in}}%
\pgfpathlineto{\pgfqpoint{-0.667769in}{7.244969in}}%
\pgfpathlineto{\pgfqpoint{-0.684324in}{7.280000in}}%
\pgfpathlineto{\pgfqpoint{-0.717355in}{7.349587in}}%
\pgfpathlineto{\pgfqpoint{-0.734880in}{7.386667in}}%
\pgfpathlineto{\pgfqpoint{-0.766942in}{7.454213in}}%
\pgfpathlineto{\pgfqpoint{-0.785431in}{7.493333in}}%
\pgfpathlineto{\pgfqpoint{-0.816529in}{7.558849in}}%
\pgfpathlineto{\pgfqpoint{-0.835977in}{7.600000in}}%
\pgfpathlineto{\pgfqpoint{-0.866116in}{7.663494in}}%
\pgfpathlineto{\pgfqpoint{-0.886520in}{7.706667in}}%
\pgfpathlineto{\pgfqpoint{-0.915702in}{7.768148in}}%
\pgfpathlineto{\pgfqpoint{-0.937058in}{7.813333in}}%
\pgfpathlineto{\pgfqpoint{-0.965289in}{7.872811in}}%
\pgfpathlineto{\pgfqpoint{-0.987592in}{7.920000in}}%
\pgfpathlineto{\pgfqpoint{-1.014876in}{7.920000in}}%
\pgfpathlineto{\pgfqpoint{-1.064463in}{7.920000in}}%
\pgfpathlineto{\pgfqpoint{-1.114050in}{7.920000in}}%
\pgfpathlineto{\pgfqpoint{-1.143498in}{7.920000in}}%
\pgfpathlineto{\pgfqpoint{-1.114050in}{7.857413in}}%
\pgfpathlineto{\pgfqpoint{-1.093033in}{7.813333in}}%
\pgfpathlineto{\pgfqpoint{-1.064463in}{7.752609in}}%
\pgfpathlineto{\pgfqpoint{-1.042556in}{7.706667in}}%
\pgfpathlineto{\pgfqpoint{-1.014876in}{7.647829in}}%
\pgfpathlineto{\pgfqpoint{-0.992067in}{7.600000in}}%
\pgfpathlineto{\pgfqpoint{-0.965289in}{7.543073in}}%
\pgfpathlineto{\pgfqpoint{-0.941567in}{7.493333in}}%
\pgfpathlineto{\pgfqpoint{-0.915702in}{7.438344in}}%
\pgfpathlineto{\pgfqpoint{-0.891054in}{7.386667in}}%
\pgfpathlineto{\pgfqpoint{-0.866116in}{7.333641in}}%
\pgfpathlineto{\pgfqpoint{-0.840528in}{7.280000in}}%
\pgfpathlineto{\pgfqpoint{-0.816529in}{7.228967in}}%
\pgfpathlineto{\pgfqpoint{-0.789988in}{7.173333in}}%
\pgfpathlineto{\pgfqpoint{-0.766942in}{7.124322in}}%
\pgfpathlineto{\pgfqpoint{-0.739434in}{7.066667in}}%
\pgfpathlineto{\pgfqpoint{-0.717355in}{7.019707in}}%
\pgfpathlineto{\pgfqpoint{-0.688865in}{6.960000in}}%
\pgfpathlineto{\pgfqpoint{-0.667769in}{6.915125in}}%
\pgfpathlineto{\pgfqpoint{-0.638280in}{6.853333in}}%
\pgfpathlineto{\pgfqpoint{-0.618182in}{6.810576in}}%
\pgfpathlineto{\pgfqpoint{-0.587679in}{6.746667in}}%
\pgfpathlineto{\pgfqpoint{-0.568595in}{6.706062in}}%
\pgfpathlineto{\pgfqpoint{-0.537061in}{6.640000in}}%
\pgfpathlineto{\pgfqpoint{-0.519008in}{6.601585in}}%
\pgfpathlineto{\pgfqpoint{-0.486425in}{6.533333in}}%
\pgfpathlineto{\pgfqpoint{-0.469421in}{6.497146in}}%
\pgfpathlineto{\pgfqpoint{-0.435770in}{6.426667in}}%
\pgfpathlineto{\pgfqpoint{-0.419835in}{6.392748in}}%
\pgfpathlineto{\pgfqpoint{-0.385096in}{6.320000in}}%
\pgfpathlineto{\pgfqpoint{-0.370248in}{6.288392in}}%
\pgfpathlineto{\pgfqpoint{-0.334401in}{6.213333in}}%
\pgfpathlineto{\pgfqpoint{-0.320661in}{6.184081in}}%
\pgfpathlineto{\pgfqpoint{-0.283684in}{6.106667in}}%
\pgfpathlineto{\pgfqpoint{-0.271074in}{6.079816in}}%
\pgfpathlineto{\pgfqpoint{-0.232945in}{6.000000in}}%
\pgfpathlineto{\pgfqpoint{-0.221488in}{5.975600in}}%
\pgfpathlineto{\pgfqpoint{-0.182181in}{5.893333in}}%
\pgfpathlineto{\pgfqpoint{-0.171901in}{5.871437in}}%
\pgfpathlineto{\pgfqpoint{-0.131392in}{5.786667in}}%
\pgfpathlineto{\pgfqpoint{-0.122314in}{5.767327in}}%
\pgfpathlineto{\pgfqpoint{-0.080576in}{5.680000in}}%
\pgfpathlineto{\pgfqpoint{-0.072727in}{5.663276in}}%
\pgfpathlineto{\pgfqpoint{-0.029733in}{5.573333in}}%
\pgfpathlineto{\pgfqpoint{-0.023140in}{5.559285in}}%
\pgfpathlineto{\pgfqpoint{0.021141in}{5.466667in}}%
\pgfpathlineto{\pgfqpoint{0.026446in}{5.455359in}}%
\pgfpathlineto{\pgfqpoint{0.072046in}{5.360000in}}%
\pgfpathlineto{\pgfqpoint{0.076033in}{5.351501in}}%
\pgfpathlineto{\pgfqpoint{0.122985in}{5.253333in}}%
\pgfpathlineto{\pgfqpoint{0.125620in}{5.247715in}}%
\pgfpathlineto{\pgfqpoint{0.173959in}{5.146667in}}%
\pgfpathlineto{\pgfqpoint{0.175207in}{5.144005in}}%
\pgfpathlineto{\pgfqpoint{0.224793in}{5.040178in}}%
\pgfpathlineto{\pgfqpoint{0.224878in}{5.040000in}}%
\pgfpathlineto{\pgfqpoint{0.274380in}{4.934971in}}%
\pgfpathlineto{\pgfqpoint{0.275160in}{4.933333in}}%
\pgfpathlineto{\pgfqpoint{0.323967in}{4.829774in}}%
\pgfpathlineto{\pgfqpoint{0.325446in}{4.826667in}}%
\pgfpathlineto{\pgfqpoint{0.373554in}{4.724587in}}%
\pgfpathlineto{\pgfqpoint{0.375737in}{4.720000in}}%
\pgfpathlineto{\pgfqpoint{0.423140in}{4.619409in}}%
\pgfpathlineto{\pgfqpoint{0.426033in}{4.613333in}}%
\pgfpathlineto{\pgfqpoint{0.472727in}{4.514242in}}%
\pgfpathlineto{\pgfqpoint{0.476334in}{4.506667in}}%
\pgfpathlineto{\pgfqpoint{0.522314in}{4.409085in}}%
\pgfpathlineto{\pgfqpoint{0.526640in}{4.400000in}}%
\pgfpathlineto{\pgfqpoint{0.571901in}{4.303939in}}%
\pgfpathlineto{\pgfqpoint{0.576951in}{4.293333in}}%
\pgfpathlineto{\pgfqpoint{0.621488in}{4.198805in}}%
\pgfpathlineto{\pgfqpoint{0.627267in}{4.186667in}}%
\pgfpathlineto{\pgfqpoint{0.671074in}{4.093682in}}%
\pgfpathlineto{\pgfqpoint{0.677590in}{4.080000in}}%
\pgfpathlineto{\pgfqpoint{0.720661in}{3.988571in}}%
\pgfpathlineto{\pgfqpoint{0.727918in}{3.973333in}}%
\pgfpathlineto{\pgfqpoint{0.770248in}{3.883473in}}%
\pgfpathlineto{\pgfqpoint{0.778252in}{3.866667in}}%
\pgfpathlineto{\pgfqpoint{0.819835in}{3.778388in}}%
\pgfpathlineto{\pgfqpoint{0.828592in}{3.760000in}}%
\pgfpathlineto{\pgfqpoint{0.869421in}{3.673315in}}%
\pgfpathlineto{\pgfqpoint{0.878939in}{3.653333in}}%
\pgfpathlineto{\pgfqpoint{0.919008in}{3.568257in}}%
\pgfpathlineto{\pgfqpoint{0.929292in}{3.546667in}}%
\pgfpathlineto{\pgfqpoint{0.968595in}{3.463213in}}%
\pgfpathlineto{\pgfqpoint{0.979653in}{3.440000in}}%
\pgfpathlineto{\pgfqpoint{1.018182in}{3.358183in}}%
\pgfpathlineto{\pgfqpoint{1.030020in}{3.333333in}}%
\pgfpathlineto{\pgfqpoint{1.067769in}{3.253169in}}%
\pgfpathlineto{\pgfqpoint{1.080395in}{3.226667in}}%
\pgfpathlineto{\pgfqpoint{1.117355in}{3.148170in}}%
\pgfpathlineto{\pgfqpoint{1.130777in}{3.120000in}}%
\pgfpathlineto{\pgfqpoint{1.166942in}{3.043188in}}%
\pgfpathlineto{\pgfqpoint{1.181167in}{3.013333in}}%
\pgfpathlineto{\pgfqpoint{1.216529in}{2.938222in}}%
\pgfpathlineto{\pgfqpoint{1.231566in}{2.906667in}}%
\pgfpathlineto{\pgfqpoint{1.266116in}{2.833274in}}%
\pgfpathlineto{\pgfqpoint{1.281973in}{2.800000in}}%
\pgfpathlineto{\pgfqpoint{1.315702in}{2.728345in}}%
\pgfpathlineto{\pgfqpoint{1.332389in}{2.693333in}}%
\pgfpathlineto{\pgfqpoint{1.365289in}{2.623434in}}%
\pgfpathlineto{\pgfqpoint{1.382814in}{2.586667in}}%
\pgfpathlineto{\pgfqpoint{1.414876in}{2.518542in}}%
\pgfpathlineto{\pgfqpoint{1.433248in}{2.480000in}}%
\pgfpathlineto{\pgfqpoint{1.464463in}{2.413671in}}%
\pgfpathlineto{\pgfqpoint{1.483692in}{2.373333in}}%
\pgfpathlineto{\pgfqpoint{1.514050in}{2.308821in}}%
\pgfpathlineto{\pgfqpoint{1.534147in}{2.266667in}}%
\pgfpathlineto{\pgfqpoint{1.563636in}{2.203993in}}%
\pgfpathlineto{\pgfqpoint{1.584612in}{2.160000in}}%
\pgfpathlineto{\pgfqpoint{1.613223in}{2.099188in}}%
\pgfpathlineto{\pgfqpoint{1.635089in}{2.053333in}}%
\pgfpathlineto{\pgfqpoint{1.662810in}{1.994407in}}%
\pgfpathlineto{\pgfqpoint{1.685576in}{1.946667in}}%
\pgfpathlineto{\pgfqpoint{1.712397in}{1.889651in}}%
\pgfpathlineto{\pgfqpoint{1.736076in}{1.840000in}}%
\pgfpathlineto{\pgfqpoint{1.761983in}{1.784920in}}%
\pgfpathlineto{\pgfqpoint{1.786589in}{1.733333in}}%
\pgfpathlineto{\pgfqpoint{1.811570in}{1.680216in}}%
\pgfpathlineto{\pgfqpoint{1.837114in}{1.626667in}}%
\pgfpathlineto{\pgfqpoint{1.861157in}{1.575540in}}%
\pgfpathlineto{\pgfqpoint{1.887654in}{1.520000in}}%
\pgfpathlineto{\pgfqpoint{1.910744in}{1.470894in}}%
\pgfpathlineto{\pgfqpoint{1.938207in}{1.413333in}}%
\pgfpathlineto{\pgfqpoint{1.960331in}{1.366278in}}%
\pgfpathlineto{\pgfqpoint{1.988775in}{1.306667in}}%
\pgfpathlineto{\pgfqpoint{2.009917in}{1.261694in}}%
\pgfpathlineto{\pgfqpoint{2.039359in}{1.200000in}}%
\pgfpathlineto{\pgfqpoint{2.059504in}{1.157143in}}%
\pgfpathlineto{\pgfqpoint{2.089959in}{1.093333in}}%
\pgfpathlineto{\pgfqpoint{2.109091in}{1.052628in}}%
\pgfpathlineto{\pgfqpoint{2.140577in}{0.986667in}}%
\pgfpathlineto{\pgfqpoint{2.158678in}{0.948149in}}%
\pgfpathlineto{\pgfqpoint{2.191212in}{0.880000in}}%
\pgfpathlineto{\pgfqpoint{2.208264in}{0.843709in}}%
\pgfpathlineto{\pgfqpoint{2.241865in}{0.773333in}}%
\pgfpathlineto{\pgfqpoint{2.257851in}{0.739308in}}%
\pgfpathlineto{\pgfqpoint{2.292539in}{0.666667in}}%
\pgfpathlineto{\pgfqpoint{2.307438in}{0.634950in}}%
\pgfpathlineto{\pgfqpoint{2.343233in}{0.560000in}}%
\pgfpathlineto{\pgfqpoint{2.357025in}{0.530637in}}%
\pgfpathlineto{\pgfqpoint{2.393949in}{0.453333in}}%
\pgfpathlineto{\pgfqpoint{2.406612in}{0.426370in}}%
\pgfpathlineto{\pgfqpoint{2.444687in}{0.346667in}}%
\pgfpathlineto{\pgfqpoint{2.456198in}{0.322152in}}%
\pgfpathlineto{\pgfqpoint{2.495450in}{0.240000in}}%
\pgfpathlineto{\pgfqpoint{2.505785in}{0.217986in}}%
\pgfpathlineto{\pgfqpoint{2.546237in}{0.133333in}}%
\pgfpathlineto{\pgfqpoint{2.555372in}{0.113874in}}%
\pgfpathlineto{\pgfqpoint{2.597052in}{0.026667in}}%
\pgfpathlineto{\pgfqpoint{2.604959in}{0.009820in}}%
\pgfpathlineto{\pgfqpoint{2.647894in}{-0.080000in}}%
\pgfpathlineto{\pgfqpoint{2.654545in}{-0.094174in}}%
\pgfpathlineto{\pgfqpoint{2.698766in}{-0.186667in}}%
\pgfpathlineto{\pgfqpoint{2.704132in}{-0.198103in}}%
\pgfpathlineto{\pgfqpoint{2.749670in}{-0.293333in}}%
\pgfpathlineto{\pgfqpoint{2.753719in}{-0.301965in}}%
\pgfpathlineto{\pgfqpoint{2.800607in}{-0.400000in}}%
\pgfpathlineto{\pgfqpoint{2.803306in}{-0.405754in}}%
\pgfpathlineto{\pgfqpoint{2.851579in}{-0.506667in}}%
\pgfpathlineto{\pgfqpoint{2.852893in}{-0.509468in}}%
\pgfpathlineto{\pgfqpoint{2.902479in}{-0.613223in}}%
\pgfpathlineto{\pgfqpoint{2.902532in}{-0.613333in}}%
\pgfpathlineto{\pgfqpoint{2.952066in}{-0.718430in}}%
\pgfpathlineto{\pgfqpoint{2.952813in}{-0.720000in}}%
\pgfpathlineto{\pgfqpoint{3.001653in}{-0.823628in}}%
\pgfpathlineto{\pgfqpoint{3.003099in}{-0.826667in}}%
\pgfpathlineto{\pgfqpoint{3.051240in}{-0.928816in}}%
\pgfpathlineto{\pgfqpoint{3.053390in}{-0.933333in}}%
\pgfpathlineto{\pgfqpoint{3.100826in}{-1.033994in}}%
\pgfpathlineto{\pgfqpoint{3.103686in}{-1.040000in}}%
\pgfpathlineto{\pgfqpoint{3.150413in}{-1.139162in}}%
\pgfpathlineto{\pgfqpoint{3.153986in}{-1.146667in}}%
\pgfpathlineto{\pgfqpoint{3.200000in}{-1.244319in}}%
\pgfpathlineto{\pgfqpoint{3.204292in}{-1.253333in}}%
\pgfpathlineto{\pgfqpoint{3.249587in}{-1.349465in}}%
\pgfpathlineto{\pgfqpoint{3.254603in}{-1.360000in}}%
\pgfpathlineto{\pgfqpoint{3.299174in}{-1.454600in}}%
\pgfpathlineto{\pgfqpoint{3.304919in}{-1.466667in}}%
\pgfpathlineto{\pgfqpoint{3.348760in}{-1.559723in}}%
\pgfpathlineto{\pgfqpoint{3.355241in}{-1.573333in}}%
\pgfpathlineto{\pgfqpoint{3.398347in}{-1.664835in}}%
\pgfpathlineto{\pgfqpoint{3.405569in}{-1.680000in}}%
\pgfpathlineto{\pgfqpoint{3.447934in}{-1.769933in}}%
\pgfpathlineto{\pgfqpoint{3.455903in}{-1.786667in}}%
\pgfpathlineto{\pgfqpoint{3.497521in}{-1.875020in}}%
\pgfpathlineto{\pgfqpoint{3.506243in}{-1.893333in}}%
\pgfpathlineto{\pgfqpoint{3.547107in}{-1.980092in}}%
\pgfpathlineto{\pgfqpoint{3.556589in}{-2.000000in}}%
\pgfpathlineto{\pgfqpoint{3.596694in}{-2.085151in}}%
\pgfpathlineto{\pgfqpoint{3.606943in}{-2.106667in}}%
\pgfpathlineto{\pgfqpoint{3.646281in}{-2.190196in}}%
\pgfpathlineto{\pgfqpoint{3.657302in}{-2.213333in}}%
\pgfpathlineto{\pgfqpoint{3.695868in}{-2.295227in}}%
\pgfpathlineto{\pgfqpoint{3.707670in}{-2.320000in}}%
\pgfpathclose%
\pgfusepath{fill}%
\end{pgfscope}%
\begin{pgfscope}%
\pgfpathrectangle{\pgfqpoint{0.800000in}{0.528000in}}{\pgfqpoint{1.963636in}{3.696000in}} %
\pgfusepath{clip}%
\pgfsetbuttcap%
\pgfsetroundjoin%
\definecolor{currentfill}{rgb}{0.744232,0.218288,0.520524}%
\pgfsetfillcolor{currentfill}%
\pgfsetlinewidth{0.000000pt}%
\definecolor{currentstroke}{rgb}{0.000000,0.000000,0.000000}%
\pgfsetstrokecolor{currentstroke}%
\pgfsetdash{}{0pt}%
\pgfpathmoveto{\pgfqpoint{3.051240in}{-2.569163in}}%
\pgfpathlineto{\pgfqpoint{3.084769in}{-2.640000in}}%
\pgfpathlineto{\pgfqpoint{3.100826in}{-2.640000in}}%
\pgfpathlineto{\pgfqpoint{3.150413in}{-2.640000in}}%
\pgfpathlineto{\pgfqpoint{3.200000in}{-2.640000in}}%
\pgfpathlineto{\pgfqpoint{3.249587in}{-2.640000in}}%
\pgfpathlineto{\pgfqpoint{3.299174in}{-2.640000in}}%
\pgfpathlineto{\pgfqpoint{3.348760in}{-2.640000in}}%
\pgfpathlineto{\pgfqpoint{3.398347in}{-2.640000in}}%
\pgfpathlineto{\pgfqpoint{3.447934in}{-2.640000in}}%
\pgfpathlineto{\pgfqpoint{3.497521in}{-2.640000in}}%
\pgfpathlineto{\pgfqpoint{3.547107in}{-2.640000in}}%
\pgfpathlineto{\pgfqpoint{3.569411in}{-2.640000in}}%
\pgfpathlineto{\pgfqpoint{3.547107in}{-2.592811in}}%
\pgfpathlineto{\pgfqpoint{3.518876in}{-2.533333in}}%
\pgfpathlineto{\pgfqpoint{3.497521in}{-2.488148in}}%
\pgfpathlineto{\pgfqpoint{3.468338in}{-2.426667in}}%
\pgfpathlineto{\pgfqpoint{3.447934in}{-2.383494in}}%
\pgfpathlineto{\pgfqpoint{3.417796in}{-2.320000in}}%
\pgfpathlineto{\pgfqpoint{3.398347in}{-2.278849in}}%
\pgfpathlineto{\pgfqpoint{3.367249in}{-2.213333in}}%
\pgfpathlineto{\pgfqpoint{3.348760in}{-2.174213in}}%
\pgfpathlineto{\pgfqpoint{3.316698in}{-2.106667in}}%
\pgfpathlineto{\pgfqpoint{3.299174in}{-2.069587in}}%
\pgfpathlineto{\pgfqpoint{3.266143in}{-2.000000in}}%
\pgfpathlineto{\pgfqpoint{3.249587in}{-1.964969in}}%
\pgfpathlineto{\pgfqpoint{3.215583in}{-1.893333in}}%
\pgfpathlineto{\pgfqpoint{3.200000in}{-1.860361in}}%
\pgfpathlineto{\pgfqpoint{3.165019in}{-1.786667in}}%
\pgfpathlineto{\pgfqpoint{3.150413in}{-1.755762in}}%
\pgfpathlineto{\pgfqpoint{3.114450in}{-1.680000in}}%
\pgfpathlineto{\pgfqpoint{3.100826in}{-1.651172in}}%
\pgfpathlineto{\pgfqpoint{3.063877in}{-1.573333in}}%
\pgfpathlineto{\pgfqpoint{3.051240in}{-1.546593in}}%
\pgfpathlineto{\pgfqpoint{3.013299in}{-1.466667in}}%
\pgfpathlineto{\pgfqpoint{3.001653in}{-1.442023in}}%
\pgfpathlineto{\pgfqpoint{2.962716in}{-1.360000in}}%
\pgfpathlineto{\pgfqpoint{2.952066in}{-1.337463in}}%
\pgfpathlineto{\pgfqpoint{2.912129in}{-1.253333in}}%
\pgfpathlineto{\pgfqpoint{2.902479in}{-1.232914in}}%
\pgfpathlineto{\pgfqpoint{2.861537in}{-1.146667in}}%
\pgfpathlineto{\pgfqpoint{2.852893in}{-1.128374in}}%
\pgfpathlineto{\pgfqpoint{2.810940in}{-1.040000in}}%
\pgfpathlineto{\pgfqpoint{2.803306in}{-1.023845in}}%
\pgfpathlineto{\pgfqpoint{2.760338in}{-0.933333in}}%
\pgfpathlineto{\pgfqpoint{2.753719in}{-0.919327in}}%
\pgfpathlineto{\pgfqpoint{2.709731in}{-0.826667in}}%
\pgfpathlineto{\pgfqpoint{2.704132in}{-0.814819in}}%
\pgfpathlineto{\pgfqpoint{2.659119in}{-0.720000in}}%
\pgfpathlineto{\pgfqpoint{2.654545in}{-0.710322in}}%
\pgfpathlineto{\pgfqpoint{2.608502in}{-0.613333in}}%
\pgfpathlineto{\pgfqpoint{2.604959in}{-0.605836in}}%
\pgfpathlineto{\pgfqpoint{2.557879in}{-0.506667in}}%
\pgfpathlineto{\pgfqpoint{2.555372in}{-0.501361in}}%
\pgfpathlineto{\pgfqpoint{2.507251in}{-0.400000in}}%
\pgfpathlineto{\pgfqpoint{2.505785in}{-0.396897in}}%
\pgfpathlineto{\pgfqpoint{2.456618in}{-0.293333in}}%
\pgfpathlineto{\pgfqpoint{2.456198in}{-0.292445in}}%
\pgfpathlineto{\pgfqpoint{2.406612in}{-0.187732in}}%
\pgfpathlineto{\pgfqpoint{2.406106in}{-0.186667in}}%
\pgfpathlineto{\pgfqpoint{2.357025in}{-0.082845in}}%
\pgfpathlineto{\pgfqpoint{2.355675in}{-0.080000in}}%
\pgfpathlineto{\pgfqpoint{2.307438in}{0.022035in}}%
\pgfpathlineto{\pgfqpoint{2.305240in}{0.026667in}}%
\pgfpathlineto{\pgfqpoint{2.257851in}{0.126910in}}%
\pgfpathlineto{\pgfqpoint{2.254803in}{0.133333in}}%
\pgfpathlineto{\pgfqpoint{2.208264in}{0.231779in}}%
\pgfpathlineto{\pgfqpoint{2.204363in}{0.240000in}}%
\pgfpathlineto{\pgfqpoint{2.158678in}{0.336641in}}%
\pgfpathlineto{\pgfqpoint{2.153920in}{0.346667in}}%
\pgfpathlineto{\pgfqpoint{2.109091in}{0.441497in}}%
\pgfpathlineto{\pgfqpoint{2.103474in}{0.453333in}}%
\pgfpathlineto{\pgfqpoint{2.059504in}{0.546347in}}%
\pgfpathlineto{\pgfqpoint{2.053025in}{0.560000in}}%
\pgfpathlineto{\pgfqpoint{2.009917in}{0.651190in}}%
\pgfpathlineto{\pgfqpoint{2.002573in}{0.666667in}}%
\pgfpathlineto{\pgfqpoint{1.960331in}{0.756026in}}%
\pgfpathlineto{\pgfqpoint{1.952117in}{0.773333in}}%
\pgfpathlineto{\pgfqpoint{1.910744in}{0.860856in}}%
\pgfpathlineto{\pgfqpoint{1.901659in}{0.880000in}}%
\pgfpathlineto{\pgfqpoint{1.861157in}{0.965680in}}%
\pgfpathlineto{\pgfqpoint{1.851197in}{0.986667in}}%
\pgfpathlineto{\pgfqpoint{1.811570in}{1.070497in}}%
\pgfpathlineto{\pgfqpoint{1.800732in}{1.093333in}}%
\pgfpathlineto{\pgfqpoint{1.761983in}{1.175306in}}%
\pgfpathlineto{\pgfqpoint{1.750264in}{1.200000in}}%
\pgfpathlineto{\pgfqpoint{1.712397in}{1.280109in}}%
\pgfpathlineto{\pgfqpoint{1.699793in}{1.306667in}}%
\pgfpathlineto{\pgfqpoint{1.662810in}{1.384905in}}%
\pgfpathlineto{\pgfqpoint{1.649318in}{1.413333in}}%
\pgfpathlineto{\pgfqpoint{1.613223in}{1.489694in}}%
\pgfpathlineto{\pgfqpoint{1.598840in}{1.520000in}}%
\pgfpathlineto{\pgfqpoint{1.563636in}{1.594476in}}%
\pgfpathlineto{\pgfqpoint{1.548359in}{1.626667in}}%
\pgfpathlineto{\pgfqpoint{1.514050in}{1.699250in}}%
\pgfpathlineto{\pgfqpoint{1.497874in}{1.733333in}}%
\pgfpathlineto{\pgfqpoint{1.464463in}{1.804017in}}%
\pgfpathlineto{\pgfqpoint{1.447385in}{1.840000in}}%
\pgfpathlineto{\pgfqpoint{1.414876in}{1.908777in}}%
\pgfpathlineto{\pgfqpoint{1.396893in}{1.946667in}}%
\pgfpathlineto{\pgfqpoint{1.365289in}{2.013529in}}%
\pgfpathlineto{\pgfqpoint{1.346398in}{2.053333in}}%
\pgfpathlineto{\pgfqpoint{1.315702in}{2.118274in}}%
\pgfpathlineto{\pgfqpoint{1.295899in}{2.160000in}}%
\pgfpathlineto{\pgfqpoint{1.266116in}{2.223010in}}%
\pgfpathlineto{\pgfqpoint{1.245396in}{2.266667in}}%
\pgfpathlineto{\pgfqpoint{1.216529in}{2.327739in}}%
\pgfpathlineto{\pgfqpoint{1.194889in}{2.373333in}}%
\pgfpathlineto{\pgfqpoint{1.166942in}{2.432461in}}%
\pgfpathlineto{\pgfqpoint{1.144379in}{2.480000in}}%
\pgfpathlineto{\pgfqpoint{1.117355in}{2.537174in}}%
\pgfpathlineto{\pgfqpoint{1.093865in}{2.586667in}}%
\pgfpathlineto{\pgfqpoint{1.067769in}{2.641879in}}%
\pgfpathlineto{\pgfqpoint{1.043346in}{2.693333in}}%
\pgfpathlineto{\pgfqpoint{1.018182in}{2.746575in}}%
\pgfpathlineto{\pgfqpoint{0.992825in}{2.800000in}}%
\pgfpathlineto{\pgfqpoint{0.968595in}{2.851264in}}%
\pgfpathlineto{\pgfqpoint{0.942299in}{2.906667in}}%
\pgfpathlineto{\pgfqpoint{0.919008in}{2.955944in}}%
\pgfpathlineto{\pgfqpoint{0.891769in}{3.013333in}}%
\pgfpathlineto{\pgfqpoint{0.869421in}{3.060616in}}%
\pgfpathlineto{\pgfqpoint{0.841235in}{3.120000in}}%
\pgfpathlineto{\pgfqpoint{0.819835in}{3.165279in}}%
\pgfpathlineto{\pgfqpoint{0.790697in}{3.226667in}}%
\pgfpathlineto{\pgfqpoint{0.770248in}{3.269933in}}%
\pgfpathlineto{\pgfqpoint{0.740154in}{3.333333in}}%
\pgfpathlineto{\pgfqpoint{0.720661in}{3.374578in}}%
\pgfpathlineto{\pgfqpoint{0.689608in}{3.440000in}}%
\pgfpathlineto{\pgfqpoint{0.671074in}{3.479215in}}%
\pgfpathlineto{\pgfqpoint{0.639057in}{3.546667in}}%
\pgfpathlineto{\pgfqpoint{0.621488in}{3.583842in}}%
\pgfpathlineto{\pgfqpoint{0.588502in}{3.653333in}}%
\pgfpathlineto{\pgfqpoint{0.571901in}{3.688460in}}%
\pgfpathlineto{\pgfqpoint{0.537942in}{3.760000in}}%
\pgfpathlineto{\pgfqpoint{0.522314in}{3.793069in}}%
\pgfpathlineto{\pgfqpoint{0.487378in}{3.866667in}}%
\pgfpathlineto{\pgfqpoint{0.472727in}{3.897668in}}%
\pgfpathlineto{\pgfqpoint{0.436810in}{3.973333in}}%
\pgfpathlineto{\pgfqpoint{0.423140in}{4.002258in}}%
\pgfpathlineto{\pgfqpoint{0.386237in}{4.080000in}}%
\pgfpathlineto{\pgfqpoint{0.373554in}{4.106838in}}%
\pgfpathlineto{\pgfqpoint{0.335659in}{4.186667in}}%
\pgfpathlineto{\pgfqpoint{0.323967in}{4.211408in}}%
\pgfpathlineto{\pgfqpoint{0.285077in}{4.293333in}}%
\pgfpathlineto{\pgfqpoint{0.274380in}{4.315968in}}%
\pgfpathlineto{\pgfqpoint{0.234490in}{4.400000in}}%
\pgfpathlineto{\pgfqpoint{0.224793in}{4.420518in}}%
\pgfpathlineto{\pgfqpoint{0.183898in}{4.506667in}}%
\pgfpathlineto{\pgfqpoint{0.175207in}{4.525058in}}%
\pgfpathlineto{\pgfqpoint{0.133301in}{4.613333in}}%
\pgfpathlineto{\pgfqpoint{0.125620in}{4.629588in}}%
\pgfpathlineto{\pgfqpoint{0.082699in}{4.720000in}}%
\pgfpathlineto{\pgfqpoint{0.076033in}{4.734107in}}%
\pgfpathlineto{\pgfqpoint{0.032093in}{4.826667in}}%
\pgfpathlineto{\pgfqpoint{0.026446in}{4.838615in}}%
\pgfpathlineto{\pgfqpoint{-0.018519in}{4.933333in}}%
\pgfpathlineto{\pgfqpoint{-0.023140in}{4.943113in}}%
\pgfpathlineto{\pgfqpoint{-0.069136in}{5.040000in}}%
\pgfpathlineto{\pgfqpoint{-0.072727in}{5.047599in}}%
\pgfpathlineto{\pgfqpoint{-0.119759in}{5.146667in}}%
\pgfpathlineto{\pgfqpoint{-0.122314in}{5.152075in}}%
\pgfpathlineto{\pgfqpoint{-0.170386in}{5.253333in}}%
\pgfpathlineto{\pgfqpoint{-0.171901in}{5.256539in}}%
\pgfpathlineto{\pgfqpoint{-0.221019in}{5.360000in}}%
\pgfpathlineto{\pgfqpoint{-0.221488in}{5.360992in}}%
\pgfpathlineto{\pgfqpoint{-0.271074in}{5.465684in}}%
\pgfpathlineto{\pgfqpoint{-0.271541in}{5.466667in}}%
\pgfpathlineto{\pgfqpoint{-0.320661in}{5.570571in}}%
\pgfpathlineto{\pgfqpoint{-0.321972in}{5.573333in}}%
\pgfpathlineto{\pgfqpoint{-0.370248in}{5.675452in}}%
\pgfpathlineto{\pgfqpoint{-0.372406in}{5.680000in}}%
\pgfpathlineto{\pgfqpoint{-0.419835in}{5.780327in}}%
\pgfpathlineto{\pgfqpoint{-0.422843in}{5.786667in}}%
\pgfpathlineto{\pgfqpoint{-0.469421in}{5.885196in}}%
\pgfpathlineto{\pgfqpoint{-0.473283in}{5.893333in}}%
\pgfpathlineto{\pgfqpoint{-0.519008in}{5.990058in}}%
\pgfpathlineto{\pgfqpoint{-0.523726in}{6.000000in}}%
\pgfpathlineto{\pgfqpoint{-0.568595in}{6.094915in}}%
\pgfpathlineto{\pgfqpoint{-0.574172in}{6.106667in}}%
\pgfpathlineto{\pgfqpoint{-0.618182in}{6.199764in}}%
\pgfpathlineto{\pgfqpoint{-0.624621in}{6.213333in}}%
\pgfpathlineto{\pgfqpoint{-0.667769in}{6.304608in}}%
\pgfpathlineto{\pgfqpoint{-0.675073in}{6.320000in}}%
\pgfpathlineto{\pgfqpoint{-0.717355in}{6.409445in}}%
\pgfpathlineto{\pgfqpoint{-0.725528in}{6.426667in}}%
\pgfpathlineto{\pgfqpoint{-0.766942in}{6.514275in}}%
\pgfpathlineto{\pgfqpoint{-0.775986in}{6.533333in}}%
\pgfpathlineto{\pgfqpoint{-0.816529in}{6.619099in}}%
\pgfpathlineto{\pgfqpoint{-0.826448in}{6.640000in}}%
\pgfpathlineto{\pgfqpoint{-0.866116in}{6.723916in}}%
\pgfpathlineto{\pgfqpoint{-0.876913in}{6.746667in}}%
\pgfpathlineto{\pgfqpoint{-0.915702in}{6.828726in}}%
\pgfpathlineto{\pgfqpoint{-0.927380in}{6.853333in}}%
\pgfpathlineto{\pgfqpoint{-0.965289in}{6.933529in}}%
\pgfpathlineto{\pgfqpoint{-0.977852in}{6.960000in}}%
\pgfpathlineto{\pgfqpoint{-1.014876in}{7.038326in}}%
\pgfpathlineto{\pgfqpoint{-1.028326in}{7.066667in}}%
\pgfpathlineto{\pgfqpoint{-1.064463in}{7.143115in}}%
\pgfpathlineto{\pgfqpoint{-1.078804in}{7.173333in}}%
\pgfpathlineto{\pgfqpoint{-1.114050in}{7.247897in}}%
\pgfpathlineto{\pgfqpoint{-1.129286in}{7.280000in}}%
\pgfpathlineto{\pgfqpoint{-1.163636in}{7.352672in}}%
\pgfpathlineto{\pgfqpoint{-1.163636in}{7.280000in}}%
\pgfpathlineto{\pgfqpoint{-1.163636in}{7.173333in}}%
\pgfpathlineto{\pgfqpoint{-1.163636in}{7.066667in}}%
\pgfpathlineto{\pgfqpoint{-1.163636in}{6.960000in}}%
\pgfpathlineto{\pgfqpoint{-1.163636in}{6.853333in}}%
\pgfpathlineto{\pgfqpoint{-1.163636in}{6.746667in}}%
\pgfpathlineto{\pgfqpoint{-1.163636in}{6.640000in}}%
\pgfpathlineto{\pgfqpoint{-1.163636in}{6.533333in}}%
\pgfpathlineto{\pgfqpoint{-1.163636in}{6.426667in}}%
\pgfpathlineto{\pgfqpoint{-1.163636in}{6.329197in}}%
\pgfpathlineto{\pgfqpoint{-1.159283in}{6.320000in}}%
\pgfpathlineto{\pgfqpoint{-1.114050in}{6.224563in}}%
\pgfpathlineto{\pgfqpoint{-1.108735in}{6.213333in}}%
\pgfpathlineto{\pgfqpoint{-1.064463in}{6.119925in}}%
\pgfpathlineto{\pgfqpoint{-1.058187in}{6.106667in}}%
\pgfpathlineto{\pgfqpoint{-1.014876in}{6.015285in}}%
\pgfpathlineto{\pgfqpoint{-1.007641in}{6.000000in}}%
\pgfpathlineto{\pgfqpoint{-0.965289in}{5.910643in}}%
\pgfpathlineto{\pgfqpoint{-0.957096in}{5.893333in}}%
\pgfpathlineto{\pgfqpoint{-0.915702in}{5.805997in}}%
\pgfpathlineto{\pgfqpoint{-0.906553in}{5.786667in}}%
\pgfpathlineto{\pgfqpoint{-0.866116in}{5.701348in}}%
\pgfpathlineto{\pgfqpoint{-0.856011in}{5.680000in}}%
\pgfpathlineto{\pgfqpoint{-0.816529in}{5.596697in}}%
\pgfpathlineto{\pgfqpoint{-0.805470in}{5.573333in}}%
\pgfpathlineto{\pgfqpoint{-0.766942in}{5.492043in}}%
\pgfpathlineto{\pgfqpoint{-0.754931in}{5.466667in}}%
\pgfpathlineto{\pgfqpoint{-0.717355in}{5.387386in}}%
\pgfpathlineto{\pgfqpoint{-0.704393in}{5.360000in}}%
\pgfpathlineto{\pgfqpoint{-0.667769in}{5.282727in}}%
\pgfpathlineto{\pgfqpoint{-0.653856in}{5.253333in}}%
\pgfpathlineto{\pgfqpoint{-0.618182in}{5.178065in}}%
\pgfpathlineto{\pgfqpoint{-0.603320in}{5.146667in}}%
\pgfpathlineto{\pgfqpoint{-0.568595in}{5.073400in}}%
\pgfpathlineto{\pgfqpoint{-0.552786in}{5.040000in}}%
\pgfpathlineto{\pgfqpoint{-0.519008in}{4.968732in}}%
\pgfpathlineto{\pgfqpoint{-0.502253in}{4.933333in}}%
\pgfpathlineto{\pgfqpoint{-0.469421in}{4.864062in}}%
\pgfpathlineto{\pgfqpoint{-0.451722in}{4.826667in}}%
\pgfpathlineto{\pgfqpoint{-0.419835in}{4.759388in}}%
\pgfpathlineto{\pgfqpoint{-0.401191in}{4.720000in}}%
\pgfpathlineto{\pgfqpoint{-0.370248in}{4.654713in}}%
\pgfpathlineto{\pgfqpoint{-0.350662in}{4.613333in}}%
\pgfpathlineto{\pgfqpoint{-0.320661in}{4.550034in}}%
\pgfpathlineto{\pgfqpoint{-0.300134in}{4.506667in}}%
\pgfpathlineto{\pgfqpoint{-0.271074in}{4.445353in}}%
\pgfpathlineto{\pgfqpoint{-0.249608in}{4.400000in}}%
\pgfpathlineto{\pgfqpoint{-0.221488in}{4.340669in}}%
\pgfpathlineto{\pgfqpoint{-0.199082in}{4.293333in}}%
\pgfpathlineto{\pgfqpoint{-0.171901in}{4.235983in}}%
\pgfpathlineto{\pgfqpoint{-0.148558in}{4.186667in}}%
\pgfpathlineto{\pgfqpoint{-0.122314in}{4.131294in}}%
\pgfpathlineto{\pgfqpoint{-0.098035in}{4.080000in}}%
\pgfpathlineto{\pgfqpoint{-0.072727in}{4.026602in}}%
\pgfpathlineto{\pgfqpoint{-0.047514in}{3.973333in}}%
\pgfpathlineto{\pgfqpoint{-0.023140in}{3.921908in}}%
\pgfpathlineto{\pgfqpoint{0.003007in}{3.866667in}}%
\pgfpathlineto{\pgfqpoint{0.026446in}{3.817211in}}%
\pgfpathlineto{\pgfqpoint{0.053526in}{3.760000in}}%
\pgfpathlineto{\pgfqpoint{0.076033in}{3.712512in}}%
\pgfpathlineto{\pgfqpoint{0.104044in}{3.653333in}}%
\pgfpathlineto{\pgfqpoint{0.125620in}{3.607810in}}%
\pgfpathlineto{\pgfqpoint{0.154560in}{3.546667in}}%
\pgfpathlineto{\pgfqpoint{0.175207in}{3.503105in}}%
\pgfpathlineto{\pgfqpoint{0.205076in}{3.440000in}}%
\pgfpathlineto{\pgfqpoint{0.224793in}{3.398398in}}%
\pgfpathlineto{\pgfqpoint{0.255590in}{3.333333in}}%
\pgfpathlineto{\pgfqpoint{0.274380in}{3.293688in}}%
\pgfpathlineto{\pgfqpoint{0.306103in}{3.226667in}}%
\pgfpathlineto{\pgfqpoint{0.323967in}{3.188976in}}%
\pgfpathlineto{\pgfqpoint{0.356615in}{3.120000in}}%
\pgfpathlineto{\pgfqpoint{0.373554in}{3.084261in}}%
\pgfpathlineto{\pgfqpoint{0.407126in}{3.013333in}}%
\pgfpathlineto{\pgfqpoint{0.423140in}{2.979544in}}%
\pgfpathlineto{\pgfqpoint{0.457635in}{2.906667in}}%
\pgfpathlineto{\pgfqpoint{0.472727in}{2.874824in}}%
\pgfpathlineto{\pgfqpoint{0.508144in}{2.800000in}}%
\pgfpathlineto{\pgfqpoint{0.522314in}{2.770102in}}%
\pgfpathlineto{\pgfqpoint{0.558651in}{2.693333in}}%
\pgfpathlineto{\pgfqpoint{0.571901in}{2.665377in}}%
\pgfpathlineto{\pgfqpoint{0.609157in}{2.586667in}}%
\pgfpathlineto{\pgfqpoint{0.621488in}{2.560649in}}%
\pgfpathlineto{\pgfqpoint{0.659662in}{2.480000in}}%
\pgfpathlineto{\pgfqpoint{0.671074in}{2.455920in}}%
\pgfpathlineto{\pgfqpoint{0.710165in}{2.373333in}}%
\pgfpathlineto{\pgfqpoint{0.720661in}{2.351187in}}%
\pgfpathlineto{\pgfqpoint{0.760668in}{2.266667in}}%
\pgfpathlineto{\pgfqpoint{0.770248in}{2.246453in}}%
\pgfpathlineto{\pgfqpoint{0.811169in}{2.160000in}}%
\pgfpathlineto{\pgfqpoint{0.819835in}{2.141716in}}%
\pgfpathlineto{\pgfqpoint{0.861669in}{2.053333in}}%
\pgfpathlineto{\pgfqpoint{0.869421in}{2.036976in}}%
\pgfpathlineto{\pgfqpoint{0.912168in}{1.946667in}}%
\pgfpathlineto{\pgfqpoint{0.919008in}{1.932234in}}%
\pgfpathlineto{\pgfqpoint{0.962666in}{1.840000in}}%
\pgfpathlineto{\pgfqpoint{0.968595in}{1.827490in}}%
\pgfpathlineto{\pgfqpoint{1.013163in}{1.733333in}}%
\pgfpathlineto{\pgfqpoint{1.018182in}{1.722743in}}%
\pgfpathlineto{\pgfqpoint{1.063658in}{1.626667in}}%
\pgfpathlineto{\pgfqpoint{1.067769in}{1.617994in}}%
\pgfpathlineto{\pgfqpoint{1.114153in}{1.520000in}}%
\pgfpathlineto{\pgfqpoint{1.117355in}{1.513242in}}%
\pgfpathlineto{\pgfqpoint{1.164646in}{1.413333in}}%
\pgfpathlineto{\pgfqpoint{1.166942in}{1.408488in}}%
\pgfpathlineto{\pgfqpoint{1.215138in}{1.306667in}}%
\pgfpathlineto{\pgfqpoint{1.216529in}{1.303732in}}%
\pgfpathlineto{\pgfqpoint{1.265629in}{1.200000in}}%
\pgfpathlineto{\pgfqpoint{1.266116in}{1.198973in}}%
\pgfpathlineto{\pgfqpoint{1.315702in}{1.094277in}}%
\pgfpathlineto{\pgfqpoint{1.316149in}{1.093333in}}%
\pgfpathlineto{\pgfqpoint{1.365289in}{0.989655in}}%
\pgfpathlineto{\pgfqpoint{1.366703in}{0.986667in}}%
\pgfpathlineto{\pgfqpoint{1.414876in}{0.885029in}}%
\pgfpathlineto{\pgfqpoint{1.417256in}{0.880000in}}%
\pgfpathlineto{\pgfqpoint{1.464463in}{0.780400in}}%
\pgfpathlineto{\pgfqpoint{1.467808in}{0.773333in}}%
\pgfpathlineto{\pgfqpoint{1.514050in}{0.675769in}}%
\pgfpathlineto{\pgfqpoint{1.518358in}{0.666667in}}%
\pgfpathlineto{\pgfqpoint{1.563636in}{0.571135in}}%
\pgfpathlineto{\pgfqpoint{1.568907in}{0.560000in}}%
\pgfpathlineto{\pgfqpoint{1.613223in}{0.466498in}}%
\pgfpathlineto{\pgfqpoint{1.619454in}{0.453333in}}%
\pgfpathlineto{\pgfqpoint{1.662810in}{0.361858in}}%
\pgfpathlineto{\pgfqpoint{1.670000in}{0.346667in}}%
\pgfpathlineto{\pgfqpoint{1.712397in}{0.257215in}}%
\pgfpathlineto{\pgfqpoint{1.720545in}{0.240000in}}%
\pgfpathlineto{\pgfqpoint{1.761983in}{0.152570in}}%
\pgfpathlineto{\pgfqpoint{1.771088in}{0.133333in}}%
\pgfpathlineto{\pgfqpoint{1.811570in}{0.047921in}}%
\pgfpathlineto{\pgfqpoint{1.821630in}{0.026667in}}%
\pgfpathlineto{\pgfqpoint{1.861157in}{-0.056730in}}%
\pgfpathlineto{\pgfqpoint{1.872171in}{-0.080000in}}%
\pgfpathlineto{\pgfqpoint{1.910744in}{-0.161384in}}%
\pgfpathlineto{\pgfqpoint{1.922711in}{-0.186667in}}%
\pgfpathlineto{\pgfqpoint{1.960331in}{-0.266040in}}%
\pgfpathlineto{\pgfqpoint{1.973249in}{-0.293333in}}%
\pgfpathlineto{\pgfqpoint{2.009917in}{-0.370700in}}%
\pgfpathlineto{\pgfqpoint{2.023786in}{-0.400000in}}%
\pgfpathlineto{\pgfqpoint{2.059504in}{-0.475362in}}%
\pgfpathlineto{\pgfqpoint{2.074321in}{-0.506667in}}%
\pgfpathlineto{\pgfqpoint{2.109091in}{-0.580027in}}%
\pgfpathlineto{\pgfqpoint{2.124856in}{-0.613333in}}%
\pgfpathlineto{\pgfqpoint{2.158678in}{-0.684694in}}%
\pgfpathlineto{\pgfqpoint{2.175389in}{-0.720000in}}%
\pgfpathlineto{\pgfqpoint{2.208264in}{-0.789365in}}%
\pgfpathlineto{\pgfqpoint{2.225920in}{-0.826667in}}%
\pgfpathlineto{\pgfqpoint{2.257851in}{-0.894038in}}%
\pgfpathlineto{\pgfqpoint{2.276451in}{-0.933333in}}%
\pgfpathlineto{\pgfqpoint{2.307438in}{-0.998713in}}%
\pgfpathlineto{\pgfqpoint{2.326980in}{-1.040000in}}%
\pgfpathlineto{\pgfqpoint{2.357025in}{-1.103392in}}%
\pgfpathlineto{\pgfqpoint{2.377508in}{-1.146667in}}%
\pgfpathlineto{\pgfqpoint{2.406612in}{-1.208073in}}%
\pgfpathlineto{\pgfqpoint{2.428035in}{-1.253333in}}%
\pgfpathlineto{\pgfqpoint{2.456198in}{-1.312756in}}%
\pgfpathlineto{\pgfqpoint{2.478560in}{-1.360000in}}%
\pgfpathlineto{\pgfqpoint{2.505785in}{-1.417442in}}%
\pgfpathlineto{\pgfqpoint{2.529084in}{-1.466667in}}%
\pgfpathlineto{\pgfqpoint{2.555372in}{-1.522131in}}%
\pgfpathlineto{\pgfqpoint{2.579607in}{-1.573333in}}%
\pgfpathlineto{\pgfqpoint{2.604959in}{-1.626823in}}%
\pgfpathlineto{\pgfqpoint{2.630129in}{-1.680000in}}%
\pgfpathlineto{\pgfqpoint{2.654545in}{-1.731517in}}%
\pgfpathlineto{\pgfqpoint{2.680649in}{-1.786667in}}%
\pgfpathlineto{\pgfqpoint{2.704132in}{-1.836214in}}%
\pgfpathlineto{\pgfqpoint{2.731168in}{-1.893333in}}%
\pgfpathlineto{\pgfqpoint{2.753719in}{-1.940913in}}%
\pgfpathlineto{\pgfqpoint{2.781686in}{-2.000000in}}%
\pgfpathlineto{\pgfqpoint{2.803306in}{-2.045615in}}%
\pgfpathlineto{\pgfqpoint{2.832203in}{-2.106667in}}%
\pgfpathlineto{\pgfqpoint{2.852893in}{-2.150320in}}%
\pgfpathlineto{\pgfqpoint{2.882719in}{-2.213333in}}%
\pgfpathlineto{\pgfqpoint{2.902479in}{-2.255027in}}%
\pgfpathlineto{\pgfqpoint{2.933233in}{-2.320000in}}%
\pgfpathlineto{\pgfqpoint{2.952066in}{-2.359736in}}%
\pgfpathlineto{\pgfqpoint{2.983746in}{-2.426667in}}%
\pgfpathlineto{\pgfqpoint{3.001653in}{-2.464448in}}%
\pgfpathlineto{\pgfqpoint{3.034258in}{-2.533333in}}%
\pgfpathclose%
\pgfusepath{fill}%
\end{pgfscope}%
\begin{pgfscope}%
\pgfpathrectangle{\pgfqpoint{0.800000in}{0.528000in}}{\pgfqpoint{1.963636in}{3.696000in}} %
\pgfusepath{clip}%
\pgfsetbuttcap%
\pgfsetroundjoin%
\definecolor{currentfill}{rgb}{0.744232,0.218288,0.520524}%
\pgfsetfillcolor{currentfill}%
\pgfsetlinewidth{0.000000pt}%
\definecolor{currentstroke}{rgb}{0.000000,0.000000,0.000000}%
\pgfsetstrokecolor{currentstroke}%
\pgfsetdash{}{0pt}%
\pgfpathmoveto{\pgfqpoint{3.745455in}{-2.072672in}}%
\pgfpathlineto{\pgfqpoint{3.745455in}{-2.000000in}}%
\pgfpathlineto{\pgfqpoint{3.745455in}{-1.893333in}}%
\pgfpathlineto{\pgfqpoint{3.745455in}{-1.786667in}}%
\pgfpathlineto{\pgfqpoint{3.745455in}{-1.680000in}}%
\pgfpathlineto{\pgfqpoint{3.745455in}{-1.573333in}}%
\pgfpathlineto{\pgfqpoint{3.745455in}{-1.466667in}}%
\pgfpathlineto{\pgfqpoint{3.745455in}{-1.360000in}}%
\pgfpathlineto{\pgfqpoint{3.745455in}{-1.253333in}}%
\pgfpathlineto{\pgfqpoint{3.745455in}{-1.146667in}}%
\pgfpathlineto{\pgfqpoint{3.745455in}{-1.049197in}}%
\pgfpathlineto{\pgfqpoint{3.741101in}{-1.040000in}}%
\pgfpathlineto{\pgfqpoint{3.695868in}{-0.944563in}}%
\pgfpathlineto{\pgfqpoint{3.690553in}{-0.933333in}}%
\pgfpathlineto{\pgfqpoint{3.646281in}{-0.839925in}}%
\pgfpathlineto{\pgfqpoint{3.640005in}{-0.826667in}}%
\pgfpathlineto{\pgfqpoint{3.596694in}{-0.735285in}}%
\pgfpathlineto{\pgfqpoint{3.589459in}{-0.720000in}}%
\pgfpathlineto{\pgfqpoint{3.547107in}{-0.630643in}}%
\pgfpathlineto{\pgfqpoint{3.538915in}{-0.613333in}}%
\pgfpathlineto{\pgfqpoint{3.497521in}{-0.525997in}}%
\pgfpathlineto{\pgfqpoint{3.488371in}{-0.506667in}}%
\pgfpathlineto{\pgfqpoint{3.447934in}{-0.421348in}}%
\pgfpathlineto{\pgfqpoint{3.437829in}{-0.400000in}}%
\pgfpathlineto{\pgfqpoint{3.398347in}{-0.316697in}}%
\pgfpathlineto{\pgfqpoint{3.387289in}{-0.293333in}}%
\pgfpathlineto{\pgfqpoint{3.348760in}{-0.212043in}}%
\pgfpathlineto{\pgfqpoint{3.336749in}{-0.186667in}}%
\pgfpathlineto{\pgfqpoint{3.299174in}{-0.107386in}}%
\pgfpathlineto{\pgfqpoint{3.286211in}{-0.080000in}}%
\pgfpathlineto{\pgfqpoint{3.249587in}{-0.002727in}}%
\pgfpathlineto{\pgfqpoint{3.235674in}{0.026667in}}%
\pgfpathlineto{\pgfqpoint{3.200000in}{0.101935in}}%
\pgfpathlineto{\pgfqpoint{3.185139in}{0.133333in}}%
\pgfpathlineto{\pgfqpoint{3.150413in}{0.206600in}}%
\pgfpathlineto{\pgfqpoint{3.134604in}{0.240000in}}%
\pgfpathlineto{\pgfqpoint{3.100826in}{0.311268in}}%
\pgfpathlineto{\pgfqpoint{3.084071in}{0.346667in}}%
\pgfpathlineto{\pgfqpoint{3.051240in}{0.415938in}}%
\pgfpathlineto{\pgfqpoint{3.033540in}{0.453333in}}%
\pgfpathlineto{\pgfqpoint{3.001653in}{0.520612in}}%
\pgfpathlineto{\pgfqpoint{2.983009in}{0.560000in}}%
\pgfpathlineto{\pgfqpoint{2.952066in}{0.625287in}}%
\pgfpathlineto{\pgfqpoint{2.932480in}{0.666667in}}%
\pgfpathlineto{\pgfqpoint{2.902479in}{0.729966in}}%
\pgfpathlineto{\pgfqpoint{2.881952in}{0.773333in}}%
\pgfpathlineto{\pgfqpoint{2.852893in}{0.834647in}}%
\pgfpathlineto{\pgfqpoint{2.831426in}{0.880000in}}%
\pgfpathlineto{\pgfqpoint{2.803306in}{0.939331in}}%
\pgfpathlineto{\pgfqpoint{2.780900in}{0.986667in}}%
\pgfpathlineto{\pgfqpoint{2.753719in}{1.044017in}}%
\pgfpathlineto{\pgfqpoint{2.730376in}{1.093333in}}%
\pgfpathlineto{\pgfqpoint{2.704132in}{1.148706in}}%
\pgfpathlineto{\pgfqpoint{2.679854in}{1.200000in}}%
\pgfpathlineto{\pgfqpoint{2.654545in}{1.253398in}}%
\pgfpathlineto{\pgfqpoint{2.629332in}{1.306667in}}%
\pgfpathlineto{\pgfqpoint{2.604959in}{1.358092in}}%
\pgfpathlineto{\pgfqpoint{2.578811in}{1.413333in}}%
\pgfpathlineto{\pgfqpoint{2.555372in}{1.462789in}}%
\pgfpathlineto{\pgfqpoint{2.528292in}{1.520000in}}%
\pgfpathlineto{\pgfqpoint{2.505785in}{1.567488in}}%
\pgfpathlineto{\pgfqpoint{2.477774in}{1.626667in}}%
\pgfpathlineto{\pgfqpoint{2.456198in}{1.672190in}}%
\pgfpathlineto{\pgfqpoint{2.427258in}{1.733333in}}%
\pgfpathlineto{\pgfqpoint{2.406612in}{1.776895in}}%
\pgfpathlineto{\pgfqpoint{2.376742in}{1.840000in}}%
\pgfpathlineto{\pgfqpoint{2.357025in}{1.881602in}}%
\pgfpathlineto{\pgfqpoint{2.326228in}{1.946667in}}%
\pgfpathlineto{\pgfqpoint{2.307438in}{1.986312in}}%
\pgfpathlineto{\pgfqpoint{2.275715in}{2.053333in}}%
\pgfpathlineto{\pgfqpoint{2.257851in}{2.091024in}}%
\pgfpathlineto{\pgfqpoint{2.225203in}{2.160000in}}%
\pgfpathlineto{\pgfqpoint{2.208264in}{2.195739in}}%
\pgfpathlineto{\pgfqpoint{2.174692in}{2.266667in}}%
\pgfpathlineto{\pgfqpoint{2.158678in}{2.300456in}}%
\pgfpathlineto{\pgfqpoint{2.124183in}{2.373333in}}%
\pgfpathlineto{\pgfqpoint{2.109091in}{2.405176in}}%
\pgfpathlineto{\pgfqpoint{2.073674in}{2.480000in}}%
\pgfpathlineto{\pgfqpoint{2.059504in}{2.509898in}}%
\pgfpathlineto{\pgfqpoint{2.023167in}{2.586667in}}%
\pgfpathlineto{\pgfqpoint{2.009917in}{2.614623in}}%
\pgfpathlineto{\pgfqpoint{1.972661in}{2.693333in}}%
\pgfpathlineto{\pgfqpoint{1.960331in}{2.719351in}}%
\pgfpathlineto{\pgfqpoint{1.922157in}{2.800000in}}%
\pgfpathlineto{\pgfqpoint{1.910744in}{2.824080in}}%
\pgfpathlineto{\pgfqpoint{1.871653in}{2.906667in}}%
\pgfpathlineto{\pgfqpoint{1.861157in}{2.928813in}}%
\pgfpathlineto{\pgfqpoint{1.821151in}{3.013333in}}%
\pgfpathlineto{\pgfqpoint{1.811570in}{3.033547in}}%
\pgfpathlineto{\pgfqpoint{1.770649in}{3.120000in}}%
\pgfpathlineto{\pgfqpoint{1.761983in}{3.138284in}}%
\pgfpathlineto{\pgfqpoint{1.720149in}{3.226667in}}%
\pgfpathlineto{\pgfqpoint{1.712397in}{3.243024in}}%
\pgfpathlineto{\pgfqpoint{1.669650in}{3.333333in}}%
\pgfpathlineto{\pgfqpoint{1.662810in}{3.347766in}}%
\pgfpathlineto{\pgfqpoint{1.619152in}{3.440000in}}%
\pgfpathlineto{\pgfqpoint{1.613223in}{3.452510in}}%
\pgfpathlineto{\pgfqpoint{1.568656in}{3.546667in}}%
\pgfpathlineto{\pgfqpoint{1.563636in}{3.557257in}}%
\pgfpathlineto{\pgfqpoint{1.518160in}{3.653333in}}%
\pgfpathlineto{\pgfqpoint{1.514050in}{3.662006in}}%
\pgfpathlineto{\pgfqpoint{1.467666in}{3.760000in}}%
\pgfpathlineto{\pgfqpoint{1.464463in}{3.766758in}}%
\pgfpathlineto{\pgfqpoint{1.417172in}{3.866667in}}%
\pgfpathlineto{\pgfqpoint{1.414876in}{3.871512in}}%
\pgfpathlineto{\pgfqpoint{1.366680in}{3.973333in}}%
\pgfpathlineto{\pgfqpoint{1.365289in}{3.976268in}}%
\pgfpathlineto{\pgfqpoint{1.316189in}{4.080000in}}%
\pgfpathlineto{\pgfqpoint{1.315702in}{4.081027in}}%
\pgfpathlineto{\pgfqpoint{1.266116in}{4.185723in}}%
\pgfpathlineto{\pgfqpoint{1.265669in}{4.186667in}}%
\pgfpathlineto{\pgfqpoint{1.216529in}{4.290345in}}%
\pgfpathlineto{\pgfqpoint{1.215115in}{4.293333in}}%
\pgfpathlineto{\pgfqpoint{1.166942in}{4.394971in}}%
\pgfpathlineto{\pgfqpoint{1.164562in}{4.400000in}}%
\pgfpathlineto{\pgfqpoint{1.117355in}{4.499600in}}%
\pgfpathlineto{\pgfqpoint{1.114010in}{4.506667in}}%
\pgfpathlineto{\pgfqpoint{1.067769in}{4.604231in}}%
\pgfpathlineto{\pgfqpoint{1.063460in}{4.613333in}}%
\pgfpathlineto{\pgfqpoint{1.018182in}{4.708865in}}%
\pgfpathlineto{\pgfqpoint{1.012911in}{4.720000in}}%
\pgfpathlineto{\pgfqpoint{0.968595in}{4.813502in}}%
\pgfpathlineto{\pgfqpoint{0.962364in}{4.826667in}}%
\pgfpathlineto{\pgfqpoint{0.919008in}{4.918142in}}%
\pgfpathlineto{\pgfqpoint{0.911818in}{4.933333in}}%
\pgfpathlineto{\pgfqpoint{0.869421in}{5.022785in}}%
\pgfpathlineto{\pgfqpoint{0.861273in}{5.040000in}}%
\pgfpathlineto{\pgfqpoint{0.819835in}{5.127430in}}%
\pgfpathlineto{\pgfqpoint{0.810730in}{5.146667in}}%
\pgfpathlineto{\pgfqpoint{0.770248in}{5.232079in}}%
\pgfpathlineto{\pgfqpoint{0.760188in}{5.253333in}}%
\pgfpathlineto{\pgfqpoint{0.720661in}{5.336730in}}%
\pgfpathlineto{\pgfqpoint{0.709647in}{5.360000in}}%
\pgfpathlineto{\pgfqpoint{0.671074in}{5.441384in}}%
\pgfpathlineto{\pgfqpoint{0.659107in}{5.466667in}}%
\pgfpathlineto{\pgfqpoint{0.621488in}{5.546040in}}%
\pgfpathlineto{\pgfqpoint{0.608569in}{5.573333in}}%
\pgfpathlineto{\pgfqpoint{0.571901in}{5.650700in}}%
\pgfpathlineto{\pgfqpoint{0.558032in}{5.680000in}}%
\pgfpathlineto{\pgfqpoint{0.522314in}{5.755362in}}%
\pgfpathlineto{\pgfqpoint{0.507497in}{5.786667in}}%
\pgfpathlineto{\pgfqpoint{0.472727in}{5.860027in}}%
\pgfpathlineto{\pgfqpoint{0.456962in}{5.893333in}}%
\pgfpathlineto{\pgfqpoint{0.423140in}{5.964694in}}%
\pgfpathlineto{\pgfqpoint{0.406429in}{6.000000in}}%
\pgfpathlineto{\pgfqpoint{0.373554in}{6.069365in}}%
\pgfpathlineto{\pgfqpoint{0.355898in}{6.106667in}}%
\pgfpathlineto{\pgfqpoint{0.323967in}{6.174038in}}%
\pgfpathlineto{\pgfqpoint{0.305367in}{6.213333in}}%
\pgfpathlineto{\pgfqpoint{0.274380in}{6.278713in}}%
\pgfpathlineto{\pgfqpoint{0.254838in}{6.320000in}}%
\pgfpathlineto{\pgfqpoint{0.224793in}{6.383392in}}%
\pgfpathlineto{\pgfqpoint{0.204310in}{6.426667in}}%
\pgfpathlineto{\pgfqpoint{0.175207in}{6.488073in}}%
\pgfpathlineto{\pgfqpoint{0.153784in}{6.533333in}}%
\pgfpathlineto{\pgfqpoint{0.125620in}{6.592756in}}%
\pgfpathlineto{\pgfqpoint{0.103258in}{6.640000in}}%
\pgfpathlineto{\pgfqpoint{0.076033in}{6.697442in}}%
\pgfpathlineto{\pgfqpoint{0.052734in}{6.746667in}}%
\pgfpathlineto{\pgfqpoint{0.026446in}{6.802131in}}%
\pgfpathlineto{\pgfqpoint{0.002211in}{6.853333in}}%
\pgfpathlineto{\pgfqpoint{-0.023140in}{6.906823in}}%
\pgfpathlineto{\pgfqpoint{-0.048311in}{6.960000in}}%
\pgfpathlineto{\pgfqpoint{-0.072727in}{7.011517in}}%
\pgfpathlineto{\pgfqpoint{-0.098831in}{7.066667in}}%
\pgfpathlineto{\pgfqpoint{-0.122314in}{7.116214in}}%
\pgfpathlineto{\pgfqpoint{-0.149350in}{7.173333in}}%
\pgfpathlineto{\pgfqpoint{-0.171901in}{7.220913in}}%
\pgfpathlineto{\pgfqpoint{-0.199868in}{7.280000in}}%
\pgfpathlineto{\pgfqpoint{-0.221488in}{7.325615in}}%
\pgfpathlineto{\pgfqpoint{-0.250385in}{7.386667in}}%
\pgfpathlineto{\pgfqpoint{-0.271074in}{7.430320in}}%
\pgfpathlineto{\pgfqpoint{-0.300901in}{7.493333in}}%
\pgfpathlineto{\pgfqpoint{-0.320661in}{7.535027in}}%
\pgfpathlineto{\pgfqpoint{-0.351415in}{7.600000in}}%
\pgfpathlineto{\pgfqpoint{-0.370248in}{7.639736in}}%
\pgfpathlineto{\pgfqpoint{-0.401928in}{7.706667in}}%
\pgfpathlineto{\pgfqpoint{-0.419835in}{7.744448in}}%
\pgfpathlineto{\pgfqpoint{-0.452440in}{7.813333in}}%
\pgfpathlineto{\pgfqpoint{-0.469421in}{7.849163in}}%
\pgfpathlineto{\pgfqpoint{-0.502951in}{7.920000in}}%
\pgfpathlineto{\pgfqpoint{-0.519008in}{7.920000in}}%
\pgfpathlineto{\pgfqpoint{-0.568595in}{7.920000in}}%
\pgfpathlineto{\pgfqpoint{-0.618182in}{7.920000in}}%
\pgfpathlineto{\pgfqpoint{-0.667769in}{7.920000in}}%
\pgfpathlineto{\pgfqpoint{-0.717355in}{7.920000in}}%
\pgfpathlineto{\pgfqpoint{-0.766942in}{7.920000in}}%
\pgfpathlineto{\pgfqpoint{-0.816529in}{7.920000in}}%
\pgfpathlineto{\pgfqpoint{-0.866116in}{7.920000in}}%
\pgfpathlineto{\pgfqpoint{-0.915702in}{7.920000in}}%
\pgfpathlineto{\pgfqpoint{-0.965289in}{7.920000in}}%
\pgfpathlineto{\pgfqpoint{-0.987592in}{7.920000in}}%
\pgfpathlineto{\pgfqpoint{-0.965289in}{7.872811in}}%
\pgfpathlineto{\pgfqpoint{-0.937058in}{7.813333in}}%
\pgfpathlineto{\pgfqpoint{-0.915702in}{7.768148in}}%
\pgfpathlineto{\pgfqpoint{-0.886520in}{7.706667in}}%
\pgfpathlineto{\pgfqpoint{-0.866116in}{7.663494in}}%
\pgfpathlineto{\pgfqpoint{-0.835977in}{7.600000in}}%
\pgfpathlineto{\pgfqpoint{-0.816529in}{7.558849in}}%
\pgfpathlineto{\pgfqpoint{-0.785431in}{7.493333in}}%
\pgfpathlineto{\pgfqpoint{-0.766942in}{7.454213in}}%
\pgfpathlineto{\pgfqpoint{-0.734880in}{7.386667in}}%
\pgfpathlineto{\pgfqpoint{-0.717355in}{7.349587in}}%
\pgfpathlineto{\pgfqpoint{-0.684324in}{7.280000in}}%
\pgfpathlineto{\pgfqpoint{-0.667769in}{7.244969in}}%
\pgfpathlineto{\pgfqpoint{-0.633765in}{7.173333in}}%
\pgfpathlineto{\pgfqpoint{-0.618182in}{7.140361in}}%
\pgfpathlineto{\pgfqpoint{-0.583201in}{7.066667in}}%
\pgfpathlineto{\pgfqpoint{-0.568595in}{7.035762in}}%
\pgfpathlineto{\pgfqpoint{-0.532632in}{6.960000in}}%
\pgfpathlineto{\pgfqpoint{-0.519008in}{6.931172in}}%
\pgfpathlineto{\pgfqpoint{-0.482059in}{6.853333in}}%
\pgfpathlineto{\pgfqpoint{-0.469421in}{6.826593in}}%
\pgfpathlineto{\pgfqpoint{-0.431481in}{6.746667in}}%
\pgfpathlineto{\pgfqpoint{-0.419835in}{6.722023in}}%
\pgfpathlineto{\pgfqpoint{-0.380898in}{6.640000in}}%
\pgfpathlineto{\pgfqpoint{-0.370248in}{6.617463in}}%
\pgfpathlineto{\pgfqpoint{-0.330311in}{6.533333in}}%
\pgfpathlineto{\pgfqpoint{-0.320661in}{6.512914in}}%
\pgfpathlineto{\pgfqpoint{-0.279719in}{6.426667in}}%
\pgfpathlineto{\pgfqpoint{-0.271074in}{6.408374in}}%
\pgfpathlineto{\pgfqpoint{-0.229122in}{6.320000in}}%
\pgfpathlineto{\pgfqpoint{-0.221488in}{6.303845in}}%
\pgfpathlineto{\pgfqpoint{-0.178520in}{6.213333in}}%
\pgfpathlineto{\pgfqpoint{-0.171901in}{6.199327in}}%
\pgfpathlineto{\pgfqpoint{-0.127913in}{6.106667in}}%
\pgfpathlineto{\pgfqpoint{-0.122314in}{6.094819in}}%
\pgfpathlineto{\pgfqpoint{-0.077301in}{6.000000in}}%
\pgfpathlineto{\pgfqpoint{-0.072727in}{5.990322in}}%
\pgfpathlineto{\pgfqpoint{-0.026683in}{5.893333in}}%
\pgfpathlineto{\pgfqpoint{-0.023140in}{5.885836in}}%
\pgfpathlineto{\pgfqpoint{0.023939in}{5.786667in}}%
\pgfpathlineto{\pgfqpoint{0.026446in}{5.781361in}}%
\pgfpathlineto{\pgfqpoint{0.074567in}{5.680000in}}%
\pgfpathlineto{\pgfqpoint{0.076033in}{5.676897in}}%
\pgfpathlineto{\pgfqpoint{0.125200in}{5.573333in}}%
\pgfpathlineto{\pgfqpoint{0.125620in}{5.572445in}}%
\pgfpathlineto{\pgfqpoint{0.175207in}{5.467732in}}%
\pgfpathlineto{\pgfqpoint{0.175712in}{5.466667in}}%
\pgfpathlineto{\pgfqpoint{0.224793in}{5.362845in}}%
\pgfpathlineto{\pgfqpoint{0.226144in}{5.360000in}}%
\pgfpathlineto{\pgfqpoint{0.274380in}{5.257965in}}%
\pgfpathlineto{\pgfqpoint{0.276578in}{5.253333in}}%
\pgfpathlineto{\pgfqpoint{0.323967in}{5.153090in}}%
\pgfpathlineto{\pgfqpoint{0.327015in}{5.146667in}}%
\pgfpathlineto{\pgfqpoint{0.373554in}{5.048221in}}%
\pgfpathlineto{\pgfqpoint{0.377455in}{5.040000in}}%
\pgfpathlineto{\pgfqpoint{0.423140in}{4.943359in}}%
\pgfpathlineto{\pgfqpoint{0.427898in}{4.933333in}}%
\pgfpathlineto{\pgfqpoint{0.472727in}{4.838503in}}%
\pgfpathlineto{\pgfqpoint{0.478344in}{4.826667in}}%
\pgfpathlineto{\pgfqpoint{0.522314in}{4.733653in}}%
\pgfpathlineto{\pgfqpoint{0.528793in}{4.720000in}}%
\pgfpathlineto{\pgfqpoint{0.571901in}{4.628810in}}%
\pgfpathlineto{\pgfqpoint{0.579246in}{4.613333in}}%
\pgfpathlineto{\pgfqpoint{0.621488in}{4.523974in}}%
\pgfpathlineto{\pgfqpoint{0.629701in}{4.506667in}}%
\pgfpathlineto{\pgfqpoint{0.671074in}{4.419144in}}%
\pgfpathlineto{\pgfqpoint{0.680159in}{4.400000in}}%
\pgfpathlineto{\pgfqpoint{0.720661in}{4.314320in}}%
\pgfpathlineto{\pgfqpoint{0.730621in}{4.293333in}}%
\pgfpathlineto{\pgfqpoint{0.770248in}{4.209503in}}%
\pgfpathlineto{\pgfqpoint{0.781086in}{4.186667in}}%
\pgfpathlineto{\pgfqpoint{0.819835in}{4.104694in}}%
\pgfpathlineto{\pgfqpoint{0.831554in}{4.080000in}}%
\pgfpathlineto{\pgfqpoint{0.869421in}{3.999891in}}%
\pgfpathlineto{\pgfqpoint{0.882025in}{3.973333in}}%
\pgfpathlineto{\pgfqpoint{0.919008in}{3.895095in}}%
\pgfpathlineto{\pgfqpoint{0.932500in}{3.866667in}}%
\pgfpathlineto{\pgfqpoint{0.968595in}{3.790306in}}%
\pgfpathlineto{\pgfqpoint{0.982978in}{3.760000in}}%
\pgfpathlineto{\pgfqpoint{1.018182in}{3.685524in}}%
\pgfpathlineto{\pgfqpoint{1.033459in}{3.653333in}}%
\pgfpathlineto{\pgfqpoint{1.067769in}{3.580750in}}%
\pgfpathlineto{\pgfqpoint{1.083944in}{3.546667in}}%
\pgfpathlineto{\pgfqpoint{1.117355in}{3.475983in}}%
\pgfpathlineto{\pgfqpoint{1.134433in}{3.440000in}}%
\pgfpathlineto{\pgfqpoint{1.166942in}{3.371223in}}%
\pgfpathlineto{\pgfqpoint{1.184925in}{3.333333in}}%
\pgfpathlineto{\pgfqpoint{1.216529in}{3.266471in}}%
\pgfpathlineto{\pgfqpoint{1.235420in}{3.226667in}}%
\pgfpathlineto{\pgfqpoint{1.266116in}{3.161726in}}%
\pgfpathlineto{\pgfqpoint{1.285920in}{3.120000in}}%
\pgfpathlineto{\pgfqpoint{1.315702in}{3.056990in}}%
\pgfpathlineto{\pgfqpoint{1.336423in}{3.013333in}}%
\pgfpathlineto{\pgfqpoint{1.365289in}{2.952261in}}%
\pgfpathlineto{\pgfqpoint{1.386929in}{2.906667in}}%
\pgfpathlineto{\pgfqpoint{1.414876in}{2.847539in}}%
\pgfpathlineto{\pgfqpoint{1.437439in}{2.800000in}}%
\pgfpathlineto{\pgfqpoint{1.464463in}{2.742826in}}%
\pgfpathlineto{\pgfqpoint{1.487954in}{2.693333in}}%
\pgfpathlineto{\pgfqpoint{1.514050in}{2.638121in}}%
\pgfpathlineto{\pgfqpoint{1.538472in}{2.586667in}}%
\pgfpathlineto{\pgfqpoint{1.563636in}{2.533425in}}%
\pgfpathlineto{\pgfqpoint{1.588994in}{2.480000in}}%
\pgfpathlineto{\pgfqpoint{1.613223in}{2.428736in}}%
\pgfpathlineto{\pgfqpoint{1.639520in}{2.373333in}}%
\pgfpathlineto{\pgfqpoint{1.662810in}{2.324056in}}%
\pgfpathlineto{\pgfqpoint{1.690050in}{2.266667in}}%
\pgfpathlineto{\pgfqpoint{1.712397in}{2.219384in}}%
\pgfpathlineto{\pgfqpoint{1.740584in}{2.160000in}}%
\pgfpathlineto{\pgfqpoint{1.761983in}{2.114721in}}%
\pgfpathlineto{\pgfqpoint{1.791122in}{2.053333in}}%
\pgfpathlineto{\pgfqpoint{1.811570in}{2.010067in}}%
\pgfpathlineto{\pgfqpoint{1.841664in}{1.946667in}}%
\pgfpathlineto{\pgfqpoint{1.861157in}{1.905422in}}%
\pgfpathlineto{\pgfqpoint{1.892210in}{1.840000in}}%
\pgfpathlineto{\pgfqpoint{1.910744in}{1.800785in}}%
\pgfpathlineto{\pgfqpoint{1.942761in}{1.733333in}}%
\pgfpathlineto{\pgfqpoint{1.960331in}{1.696158in}}%
\pgfpathlineto{\pgfqpoint{1.993316in}{1.626667in}}%
\pgfpathlineto{\pgfqpoint{2.009917in}{1.591540in}}%
\pgfpathlineto{\pgfqpoint{2.043876in}{1.520000in}}%
\pgfpathlineto{\pgfqpoint{2.059504in}{1.486931in}}%
\pgfpathlineto{\pgfqpoint{2.094440in}{1.413333in}}%
\pgfpathlineto{\pgfqpoint{2.109091in}{1.382332in}}%
\pgfpathlineto{\pgfqpoint{2.145008in}{1.306667in}}%
\pgfpathlineto{\pgfqpoint{2.158678in}{1.277742in}}%
\pgfpathlineto{\pgfqpoint{2.195581in}{1.200000in}}%
\pgfpathlineto{\pgfqpoint{2.208264in}{1.173162in}}%
\pgfpathlineto{\pgfqpoint{2.246159in}{1.093333in}}%
\pgfpathlineto{\pgfqpoint{2.257851in}{1.068592in}}%
\pgfpathlineto{\pgfqpoint{2.296741in}{0.986667in}}%
\pgfpathlineto{\pgfqpoint{2.307438in}{0.964032in}}%
\pgfpathlineto{\pgfqpoint{2.347328in}{0.880000in}}%
\pgfpathlineto{\pgfqpoint{2.357025in}{0.859482in}}%
\pgfpathlineto{\pgfqpoint{2.397920in}{0.773333in}}%
\pgfpathlineto{\pgfqpoint{2.406612in}{0.754942in}}%
\pgfpathlineto{\pgfqpoint{2.448517in}{0.666667in}}%
\pgfpathlineto{\pgfqpoint{2.456198in}{0.650412in}}%
\pgfpathlineto{\pgfqpoint{2.499119in}{0.560000in}}%
\pgfpathlineto{\pgfqpoint{2.505785in}{0.545893in}}%
\pgfpathlineto{\pgfqpoint{2.549726in}{0.453333in}}%
\pgfpathlineto{\pgfqpoint{2.555372in}{0.441385in}}%
\pgfpathlineto{\pgfqpoint{2.600337in}{0.346667in}}%
\pgfpathlineto{\pgfqpoint{2.604959in}{0.336887in}}%
\pgfpathlineto{\pgfqpoint{2.650954in}{0.240000in}}%
\pgfpathlineto{\pgfqpoint{2.654545in}{0.232401in}}%
\pgfpathlineto{\pgfqpoint{2.701577in}{0.133333in}}%
\pgfpathlineto{\pgfqpoint{2.704132in}{0.127925in}}%
\pgfpathlineto{\pgfqpoint{2.752204in}{0.026667in}}%
\pgfpathlineto{\pgfqpoint{2.753719in}{0.023461in}}%
\pgfpathlineto{\pgfqpoint{2.802837in}{-0.080000in}}%
\pgfpathlineto{\pgfqpoint{2.803306in}{-0.080992in}}%
\pgfpathlineto{\pgfqpoint{2.852893in}{-0.185684in}}%
\pgfpathlineto{\pgfqpoint{2.853359in}{-0.186667in}}%
\pgfpathlineto{\pgfqpoint{2.902479in}{-0.290571in}}%
\pgfpathlineto{\pgfqpoint{2.903790in}{-0.293333in}}%
\pgfpathlineto{\pgfqpoint{2.952066in}{-0.395452in}}%
\pgfpathlineto{\pgfqpoint{2.954224in}{-0.400000in}}%
\pgfpathlineto{\pgfqpoint{3.001653in}{-0.500327in}}%
\pgfpathlineto{\pgfqpoint{3.004661in}{-0.506667in}}%
\pgfpathlineto{\pgfqpoint{3.051240in}{-0.605196in}}%
\pgfpathlineto{\pgfqpoint{3.055101in}{-0.613333in}}%
\pgfpathlineto{\pgfqpoint{3.100826in}{-0.710058in}}%
\pgfpathlineto{\pgfqpoint{3.105544in}{-0.720000in}}%
\pgfpathlineto{\pgfqpoint{3.150413in}{-0.814915in}}%
\pgfpathlineto{\pgfqpoint{3.155990in}{-0.826667in}}%
\pgfpathlineto{\pgfqpoint{3.200000in}{-0.919764in}}%
\pgfpathlineto{\pgfqpoint{3.206439in}{-0.933333in}}%
\pgfpathlineto{\pgfqpoint{3.249587in}{-1.024608in}}%
\pgfpathlineto{\pgfqpoint{3.256891in}{-1.040000in}}%
\pgfpathlineto{\pgfqpoint{3.299174in}{-1.129445in}}%
\pgfpathlineto{\pgfqpoint{3.307346in}{-1.146667in}}%
\pgfpathlineto{\pgfqpoint{3.348760in}{-1.234275in}}%
\pgfpathlineto{\pgfqpoint{3.357805in}{-1.253333in}}%
\pgfpathlineto{\pgfqpoint{3.398347in}{-1.339099in}}%
\pgfpathlineto{\pgfqpoint{3.408266in}{-1.360000in}}%
\pgfpathlineto{\pgfqpoint{3.447934in}{-1.443916in}}%
\pgfpathlineto{\pgfqpoint{3.458731in}{-1.466667in}}%
\pgfpathlineto{\pgfqpoint{3.497521in}{-1.548726in}}%
\pgfpathlineto{\pgfqpoint{3.509199in}{-1.573333in}}%
\pgfpathlineto{\pgfqpoint{3.547107in}{-1.653529in}}%
\pgfpathlineto{\pgfqpoint{3.559670in}{-1.680000in}}%
\pgfpathlineto{\pgfqpoint{3.596694in}{-1.758326in}}%
\pgfpathlineto{\pgfqpoint{3.610144in}{-1.786667in}}%
\pgfpathlineto{\pgfqpoint{3.646281in}{-1.863115in}}%
\pgfpathlineto{\pgfqpoint{3.660622in}{-1.893333in}}%
\pgfpathlineto{\pgfqpoint{3.695868in}{-1.967897in}}%
\pgfpathlineto{\pgfqpoint{3.711104in}{-2.000000in}}%
\pgfpathclose%
\pgfusepath{fill}%
\end{pgfscope}%
\begin{pgfscope}%
\pgfpathrectangle{\pgfqpoint{0.800000in}{0.528000in}}{\pgfqpoint{1.963636in}{3.696000in}} %
\pgfusepath{clip}%
\pgfsetbuttcap%
\pgfsetroundjoin%
\definecolor{currentfill}{rgb}{0.944844,0.507658,0.302433}%
\pgfsetfillcolor{currentfill}%
\pgfsetlinewidth{0.000000pt}%
\definecolor{currentstroke}{rgb}{0.000000,0.000000,0.000000}%
\pgfsetstrokecolor{currentstroke}%
\pgfsetdash{}{0pt}%
\pgfpathmoveto{\pgfqpoint{1.514050in}{-2.553510in}}%
\pgfpathlineto{\pgfqpoint{1.555006in}{-2.640000in}}%
\pgfpathlineto{\pgfqpoint{1.563636in}{-2.640000in}}%
\pgfpathlineto{\pgfqpoint{1.613223in}{-2.640000in}}%
\pgfpathlineto{\pgfqpoint{1.662810in}{-2.640000in}}%
\pgfpathlineto{\pgfqpoint{1.712397in}{-2.640000in}}%
\pgfpathlineto{\pgfqpoint{1.761983in}{-2.640000in}}%
\pgfpathlineto{\pgfqpoint{1.811570in}{-2.640000in}}%
\pgfpathlineto{\pgfqpoint{1.861157in}{-2.640000in}}%
\pgfpathlineto{\pgfqpoint{1.910744in}{-2.640000in}}%
\pgfpathlineto{\pgfqpoint{1.960331in}{-2.640000in}}%
\pgfpathlineto{\pgfqpoint{2.009917in}{-2.640000in}}%
\pgfpathlineto{\pgfqpoint{2.059504in}{-2.640000in}}%
\pgfpathlineto{\pgfqpoint{2.109091in}{-2.640000in}}%
\pgfpathlineto{\pgfqpoint{2.158678in}{-2.640000in}}%
\pgfpathlineto{\pgfqpoint{2.208264in}{-2.640000in}}%
\pgfpathlineto{\pgfqpoint{2.257851in}{-2.640000in}}%
\pgfpathlineto{\pgfqpoint{2.307438in}{-2.640000in}}%
\pgfpathlineto{\pgfqpoint{2.357025in}{-2.640000in}}%
\pgfpathlineto{\pgfqpoint{2.406612in}{-2.640000in}}%
\pgfpathlineto{\pgfqpoint{2.456198in}{-2.640000in}}%
\pgfpathlineto{\pgfqpoint{2.505785in}{-2.640000in}}%
\pgfpathlineto{\pgfqpoint{2.555372in}{-2.640000in}}%
\pgfpathlineto{\pgfqpoint{2.604959in}{-2.640000in}}%
\pgfpathlineto{\pgfqpoint{2.654545in}{-2.640000in}}%
\pgfpathlineto{\pgfqpoint{2.704132in}{-2.640000in}}%
\pgfpathlineto{\pgfqpoint{2.753719in}{-2.640000in}}%
\pgfpathlineto{\pgfqpoint{2.803306in}{-2.640000in}}%
\pgfpathlineto{\pgfqpoint{2.852893in}{-2.640000in}}%
\pgfpathlineto{\pgfqpoint{2.902479in}{-2.640000in}}%
\pgfpathlineto{\pgfqpoint{2.952066in}{-2.640000in}}%
\pgfpathlineto{\pgfqpoint{3.001653in}{-2.640000in}}%
\pgfpathlineto{\pgfqpoint{3.051240in}{-2.640000in}}%
\pgfpathlineto{\pgfqpoint{3.084769in}{-2.640000in}}%
\pgfpathlineto{\pgfqpoint{3.051240in}{-2.569163in}}%
\pgfpathlineto{\pgfqpoint{3.034258in}{-2.533333in}}%
\pgfpathlineto{\pgfqpoint{3.001653in}{-2.464448in}}%
\pgfpathlineto{\pgfqpoint{2.983746in}{-2.426667in}}%
\pgfpathlineto{\pgfqpoint{2.952066in}{-2.359736in}}%
\pgfpathlineto{\pgfqpoint{2.933233in}{-2.320000in}}%
\pgfpathlineto{\pgfqpoint{2.902479in}{-2.255027in}}%
\pgfpathlineto{\pgfqpoint{2.882719in}{-2.213333in}}%
\pgfpathlineto{\pgfqpoint{2.852893in}{-2.150320in}}%
\pgfpathlineto{\pgfqpoint{2.832203in}{-2.106667in}}%
\pgfpathlineto{\pgfqpoint{2.803306in}{-2.045615in}}%
\pgfpathlineto{\pgfqpoint{2.781686in}{-2.000000in}}%
\pgfpathlineto{\pgfqpoint{2.753719in}{-1.940913in}}%
\pgfpathlineto{\pgfqpoint{2.731168in}{-1.893333in}}%
\pgfpathlineto{\pgfqpoint{2.704132in}{-1.836214in}}%
\pgfpathlineto{\pgfqpoint{2.680649in}{-1.786667in}}%
\pgfpathlineto{\pgfqpoint{2.654545in}{-1.731517in}}%
\pgfpathlineto{\pgfqpoint{2.630129in}{-1.680000in}}%
\pgfpathlineto{\pgfqpoint{2.604959in}{-1.626823in}}%
\pgfpathlineto{\pgfqpoint{2.579607in}{-1.573333in}}%
\pgfpathlineto{\pgfqpoint{2.555372in}{-1.522131in}}%
\pgfpathlineto{\pgfqpoint{2.529084in}{-1.466667in}}%
\pgfpathlineto{\pgfqpoint{2.505785in}{-1.417442in}}%
\pgfpathlineto{\pgfqpoint{2.478560in}{-1.360000in}}%
\pgfpathlineto{\pgfqpoint{2.456198in}{-1.312756in}}%
\pgfpathlineto{\pgfqpoint{2.428035in}{-1.253333in}}%
\pgfpathlineto{\pgfqpoint{2.406612in}{-1.208073in}}%
\pgfpathlineto{\pgfqpoint{2.377508in}{-1.146667in}}%
\pgfpathlineto{\pgfqpoint{2.357025in}{-1.103392in}}%
\pgfpathlineto{\pgfqpoint{2.326980in}{-1.040000in}}%
\pgfpathlineto{\pgfqpoint{2.307438in}{-0.998713in}}%
\pgfpathlineto{\pgfqpoint{2.276451in}{-0.933333in}}%
\pgfpathlineto{\pgfqpoint{2.257851in}{-0.894038in}}%
\pgfpathlineto{\pgfqpoint{2.225920in}{-0.826667in}}%
\pgfpathlineto{\pgfqpoint{2.208264in}{-0.789365in}}%
\pgfpathlineto{\pgfqpoint{2.175389in}{-0.720000in}}%
\pgfpathlineto{\pgfqpoint{2.158678in}{-0.684694in}}%
\pgfpathlineto{\pgfqpoint{2.124856in}{-0.613333in}}%
\pgfpathlineto{\pgfqpoint{2.109091in}{-0.580027in}}%
\pgfpathlineto{\pgfqpoint{2.074321in}{-0.506667in}}%
\pgfpathlineto{\pgfqpoint{2.059504in}{-0.475362in}}%
\pgfpathlineto{\pgfqpoint{2.023786in}{-0.400000in}}%
\pgfpathlineto{\pgfqpoint{2.009917in}{-0.370700in}}%
\pgfpathlineto{\pgfqpoint{1.973249in}{-0.293333in}}%
\pgfpathlineto{\pgfqpoint{1.960331in}{-0.266040in}}%
\pgfpathlineto{\pgfqpoint{1.922711in}{-0.186667in}}%
\pgfpathlineto{\pgfqpoint{1.910744in}{-0.161384in}}%
\pgfpathlineto{\pgfqpoint{1.872171in}{-0.080000in}}%
\pgfpathlineto{\pgfqpoint{1.861157in}{-0.056730in}}%
\pgfpathlineto{\pgfqpoint{1.821630in}{0.026667in}}%
\pgfpathlineto{\pgfqpoint{1.811570in}{0.047921in}}%
\pgfpathlineto{\pgfqpoint{1.771088in}{0.133333in}}%
\pgfpathlineto{\pgfqpoint{1.761983in}{0.152570in}}%
\pgfpathlineto{\pgfqpoint{1.720545in}{0.240000in}}%
\pgfpathlineto{\pgfqpoint{1.712397in}{0.257215in}}%
\pgfpathlineto{\pgfqpoint{1.670000in}{0.346667in}}%
\pgfpathlineto{\pgfqpoint{1.662810in}{0.361858in}}%
\pgfpathlineto{\pgfqpoint{1.619454in}{0.453333in}}%
\pgfpathlineto{\pgfqpoint{1.613223in}{0.466498in}}%
\pgfpathlineto{\pgfqpoint{1.568907in}{0.560000in}}%
\pgfpathlineto{\pgfqpoint{1.563636in}{0.571135in}}%
\pgfpathlineto{\pgfqpoint{1.518358in}{0.666667in}}%
\pgfpathlineto{\pgfqpoint{1.514050in}{0.675769in}}%
\pgfpathlineto{\pgfqpoint{1.467808in}{0.773333in}}%
\pgfpathlineto{\pgfqpoint{1.464463in}{0.780400in}}%
\pgfpathlineto{\pgfqpoint{1.417256in}{0.880000in}}%
\pgfpathlineto{\pgfqpoint{1.414876in}{0.885029in}}%
\pgfpathlineto{\pgfqpoint{1.366703in}{0.986667in}}%
\pgfpathlineto{\pgfqpoint{1.365289in}{0.989655in}}%
\pgfpathlineto{\pgfqpoint{1.316149in}{1.093333in}}%
\pgfpathlineto{\pgfqpoint{1.315702in}{1.094277in}}%
\pgfpathlineto{\pgfqpoint{1.266116in}{1.198973in}}%
\pgfpathlineto{\pgfqpoint{1.265629in}{1.200000in}}%
\pgfpathlineto{\pgfqpoint{1.216529in}{1.303732in}}%
\pgfpathlineto{\pgfqpoint{1.215138in}{1.306667in}}%
\pgfpathlineto{\pgfqpoint{1.166942in}{1.408488in}}%
\pgfpathlineto{\pgfqpoint{1.164646in}{1.413333in}}%
\pgfpathlineto{\pgfqpoint{1.117355in}{1.513242in}}%
\pgfpathlineto{\pgfqpoint{1.114153in}{1.520000in}}%
\pgfpathlineto{\pgfqpoint{1.067769in}{1.617994in}}%
\pgfpathlineto{\pgfqpoint{1.063658in}{1.626667in}}%
\pgfpathlineto{\pgfqpoint{1.018182in}{1.722743in}}%
\pgfpathlineto{\pgfqpoint{1.013163in}{1.733333in}}%
\pgfpathlineto{\pgfqpoint{0.968595in}{1.827490in}}%
\pgfpathlineto{\pgfqpoint{0.962666in}{1.840000in}}%
\pgfpathlineto{\pgfqpoint{0.919008in}{1.932234in}}%
\pgfpathlineto{\pgfqpoint{0.912168in}{1.946667in}}%
\pgfpathlineto{\pgfqpoint{0.869421in}{2.036976in}}%
\pgfpathlineto{\pgfqpoint{0.861669in}{2.053333in}}%
\pgfpathlineto{\pgfqpoint{0.819835in}{2.141716in}}%
\pgfpathlineto{\pgfqpoint{0.811169in}{2.160000in}}%
\pgfpathlineto{\pgfqpoint{0.770248in}{2.246453in}}%
\pgfpathlineto{\pgfqpoint{0.760668in}{2.266667in}}%
\pgfpathlineto{\pgfqpoint{0.720661in}{2.351187in}}%
\pgfpathlineto{\pgfqpoint{0.710165in}{2.373333in}}%
\pgfpathlineto{\pgfqpoint{0.671074in}{2.455920in}}%
\pgfpathlineto{\pgfqpoint{0.659662in}{2.480000in}}%
\pgfpathlineto{\pgfqpoint{0.621488in}{2.560649in}}%
\pgfpathlineto{\pgfqpoint{0.609157in}{2.586667in}}%
\pgfpathlineto{\pgfqpoint{0.571901in}{2.665377in}}%
\pgfpathlineto{\pgfqpoint{0.558651in}{2.693333in}}%
\pgfpathlineto{\pgfqpoint{0.522314in}{2.770102in}}%
\pgfpathlineto{\pgfqpoint{0.508144in}{2.800000in}}%
\pgfpathlineto{\pgfqpoint{0.472727in}{2.874824in}}%
\pgfpathlineto{\pgfqpoint{0.457635in}{2.906667in}}%
\pgfpathlineto{\pgfqpoint{0.423140in}{2.979544in}}%
\pgfpathlineto{\pgfqpoint{0.407126in}{3.013333in}}%
\pgfpathlineto{\pgfqpoint{0.373554in}{3.084261in}}%
\pgfpathlineto{\pgfqpoint{0.356615in}{3.120000in}}%
\pgfpathlineto{\pgfqpoint{0.323967in}{3.188976in}}%
\pgfpathlineto{\pgfqpoint{0.306103in}{3.226667in}}%
\pgfpathlineto{\pgfqpoint{0.274380in}{3.293688in}}%
\pgfpathlineto{\pgfqpoint{0.255590in}{3.333333in}}%
\pgfpathlineto{\pgfqpoint{0.224793in}{3.398398in}}%
\pgfpathlineto{\pgfqpoint{0.205076in}{3.440000in}}%
\pgfpathlineto{\pgfqpoint{0.175207in}{3.503105in}}%
\pgfpathlineto{\pgfqpoint{0.154560in}{3.546667in}}%
\pgfpathlineto{\pgfqpoint{0.125620in}{3.607810in}}%
\pgfpathlineto{\pgfqpoint{0.104044in}{3.653333in}}%
\pgfpathlineto{\pgfqpoint{0.076033in}{3.712512in}}%
\pgfpathlineto{\pgfqpoint{0.053526in}{3.760000in}}%
\pgfpathlineto{\pgfqpoint{0.026446in}{3.817211in}}%
\pgfpathlineto{\pgfqpoint{0.003007in}{3.866667in}}%
\pgfpathlineto{\pgfqpoint{-0.023140in}{3.921908in}}%
\pgfpathlineto{\pgfqpoint{-0.047514in}{3.973333in}}%
\pgfpathlineto{\pgfqpoint{-0.072727in}{4.026602in}}%
\pgfpathlineto{\pgfqpoint{-0.098035in}{4.080000in}}%
\pgfpathlineto{\pgfqpoint{-0.122314in}{4.131294in}}%
\pgfpathlineto{\pgfqpoint{-0.148558in}{4.186667in}}%
\pgfpathlineto{\pgfqpoint{-0.171901in}{4.235983in}}%
\pgfpathlineto{\pgfqpoint{-0.199082in}{4.293333in}}%
\pgfpathlineto{\pgfqpoint{-0.221488in}{4.340669in}}%
\pgfpathlineto{\pgfqpoint{-0.249608in}{4.400000in}}%
\pgfpathlineto{\pgfqpoint{-0.271074in}{4.445353in}}%
\pgfpathlineto{\pgfqpoint{-0.300134in}{4.506667in}}%
\pgfpathlineto{\pgfqpoint{-0.320661in}{4.550034in}}%
\pgfpathlineto{\pgfqpoint{-0.350662in}{4.613333in}}%
\pgfpathlineto{\pgfqpoint{-0.370248in}{4.654713in}}%
\pgfpathlineto{\pgfqpoint{-0.401191in}{4.720000in}}%
\pgfpathlineto{\pgfqpoint{-0.419835in}{4.759388in}}%
\pgfpathlineto{\pgfqpoint{-0.451722in}{4.826667in}}%
\pgfpathlineto{\pgfqpoint{-0.469421in}{4.864062in}}%
\pgfpathlineto{\pgfqpoint{-0.502253in}{4.933333in}}%
\pgfpathlineto{\pgfqpoint{-0.519008in}{4.968732in}}%
\pgfpathlineto{\pgfqpoint{-0.552786in}{5.040000in}}%
\pgfpathlineto{\pgfqpoint{-0.568595in}{5.073400in}}%
\pgfpathlineto{\pgfqpoint{-0.603320in}{5.146667in}}%
\pgfpathlineto{\pgfqpoint{-0.618182in}{5.178065in}}%
\pgfpathlineto{\pgfqpoint{-0.653856in}{5.253333in}}%
\pgfpathlineto{\pgfqpoint{-0.667769in}{5.282727in}}%
\pgfpathlineto{\pgfqpoint{-0.704393in}{5.360000in}}%
\pgfpathlineto{\pgfqpoint{-0.717355in}{5.387386in}}%
\pgfpathlineto{\pgfqpoint{-0.754931in}{5.466667in}}%
\pgfpathlineto{\pgfqpoint{-0.766942in}{5.492043in}}%
\pgfpathlineto{\pgfqpoint{-0.805470in}{5.573333in}}%
\pgfpathlineto{\pgfqpoint{-0.816529in}{5.596697in}}%
\pgfpathlineto{\pgfqpoint{-0.856011in}{5.680000in}}%
\pgfpathlineto{\pgfqpoint{-0.866116in}{5.701348in}}%
\pgfpathlineto{\pgfqpoint{-0.906553in}{5.786667in}}%
\pgfpathlineto{\pgfqpoint{-0.915702in}{5.805997in}}%
\pgfpathlineto{\pgfqpoint{-0.957096in}{5.893333in}}%
\pgfpathlineto{\pgfqpoint{-0.965289in}{5.910643in}}%
\pgfpathlineto{\pgfqpoint{-1.007641in}{6.000000in}}%
\pgfpathlineto{\pgfqpoint{-1.014876in}{6.015285in}}%
\pgfpathlineto{\pgfqpoint{-1.058187in}{6.106667in}}%
\pgfpathlineto{\pgfqpoint{-1.064463in}{6.119925in}}%
\pgfpathlineto{\pgfqpoint{-1.108735in}{6.213333in}}%
\pgfpathlineto{\pgfqpoint{-1.114050in}{6.224563in}}%
\pgfpathlineto{\pgfqpoint{-1.159283in}{6.320000in}}%
\pgfpathlineto{\pgfqpoint{-1.163636in}{6.329197in}}%
\pgfpathlineto{\pgfqpoint{-1.163636in}{6.320000in}}%
\pgfpathlineto{\pgfqpoint{-1.163636in}{6.213333in}}%
\pgfpathlineto{\pgfqpoint{-1.163636in}{6.106667in}}%
\pgfpathlineto{\pgfqpoint{-1.163636in}{6.000000in}}%
\pgfpathlineto{\pgfqpoint{-1.163636in}{5.893333in}}%
\pgfpathlineto{\pgfqpoint{-1.163636in}{5.786667in}}%
\pgfpathlineto{\pgfqpoint{-1.163636in}{5.680000in}}%
\pgfpathlineto{\pgfqpoint{-1.163636in}{5.573333in}}%
\pgfpathlineto{\pgfqpoint{-1.163636in}{5.466667in}}%
\pgfpathlineto{\pgfqpoint{-1.163636in}{5.360000in}}%
\pgfpathlineto{\pgfqpoint{-1.163636in}{5.253333in}}%
\pgfpathlineto{\pgfqpoint{-1.163636in}{5.146667in}}%
\pgfpathlineto{\pgfqpoint{-1.163636in}{5.040000in}}%
\pgfpathlineto{\pgfqpoint{-1.163636in}{4.933333in}}%
\pgfpathlineto{\pgfqpoint{-1.163636in}{4.826667in}}%
\pgfpathlineto{\pgfqpoint{-1.163636in}{4.720000in}}%
\pgfpathlineto{\pgfqpoint{-1.163636in}{4.613333in}}%
\pgfpathlineto{\pgfqpoint{-1.163636in}{4.506667in}}%
\pgfpathlineto{\pgfqpoint{-1.163636in}{4.400000in}}%
\pgfpathlineto{\pgfqpoint{-1.163636in}{4.293333in}}%
\pgfpathlineto{\pgfqpoint{-1.163636in}{4.186667in}}%
\pgfpathlineto{\pgfqpoint{-1.163636in}{4.080000in}}%
\pgfpathlineto{\pgfqpoint{-1.163636in}{3.973333in}}%
\pgfpathlineto{\pgfqpoint{-1.163636in}{3.866667in}}%
\pgfpathlineto{\pgfqpoint{-1.163636in}{3.760000in}}%
\pgfpathlineto{\pgfqpoint{-1.163636in}{3.653333in}}%
\pgfpathlineto{\pgfqpoint{-1.163636in}{3.546667in}}%
\pgfpathlineto{\pgfqpoint{-1.163636in}{3.440000in}}%
\pgfpathlineto{\pgfqpoint{-1.163636in}{3.333333in}}%
\pgfpathlineto{\pgfqpoint{-1.163636in}{3.226667in}}%
\pgfpathlineto{\pgfqpoint{-1.163636in}{3.120000in}}%
\pgfpathlineto{\pgfqpoint{-1.163636in}{3.099915in}}%
\pgfpathlineto{\pgfqpoint{-1.122636in}{3.013333in}}%
\pgfpathlineto{\pgfqpoint{-1.114050in}{2.995208in}}%
\pgfpathlineto{\pgfqpoint{-1.072122in}{2.906667in}}%
\pgfpathlineto{\pgfqpoint{-1.064463in}{2.890500in}}%
\pgfpathlineto{\pgfqpoint{-1.021608in}{2.800000in}}%
\pgfpathlineto{\pgfqpoint{-1.014876in}{2.785791in}}%
\pgfpathlineto{\pgfqpoint{-0.971094in}{2.693333in}}%
\pgfpathlineto{\pgfqpoint{-0.965289in}{2.681081in}}%
\pgfpathlineto{\pgfqpoint{-0.920580in}{2.586667in}}%
\pgfpathlineto{\pgfqpoint{-0.915702in}{2.576371in}}%
\pgfpathlineto{\pgfqpoint{-0.870067in}{2.480000in}}%
\pgfpathlineto{\pgfqpoint{-0.866116in}{2.471659in}}%
\pgfpathlineto{\pgfqpoint{-0.819554in}{2.373333in}}%
\pgfpathlineto{\pgfqpoint{-0.816529in}{2.366947in}}%
\pgfpathlineto{\pgfqpoint{-0.769042in}{2.266667in}}%
\pgfpathlineto{\pgfqpoint{-0.766942in}{2.262234in}}%
\pgfpathlineto{\pgfqpoint{-0.718530in}{2.160000in}}%
\pgfpathlineto{\pgfqpoint{-0.717355in}{2.157521in}}%
\pgfpathlineto{\pgfqpoint{-0.668018in}{2.053333in}}%
\pgfpathlineto{\pgfqpoint{-0.667769in}{2.052806in}}%
\pgfpathlineto{\pgfqpoint{-0.618182in}{1.948123in}}%
\pgfpathlineto{\pgfqpoint{-0.617492in}{1.946667in}}%
\pgfpathlineto{\pgfqpoint{-0.568595in}{1.843452in}}%
\pgfpathlineto{\pgfqpoint{-0.566960in}{1.840000in}}%
\pgfpathlineto{\pgfqpoint{-0.519008in}{1.738779in}}%
\pgfpathlineto{\pgfqpoint{-0.516429in}{1.733333in}}%
\pgfpathlineto{\pgfqpoint{-0.469421in}{1.634106in}}%
\pgfpathlineto{\pgfqpoint{-0.465899in}{1.626667in}}%
\pgfpathlineto{\pgfqpoint{-0.419835in}{1.529432in}}%
\pgfpathlineto{\pgfqpoint{-0.415368in}{1.520000in}}%
\pgfpathlineto{\pgfqpoint{-0.370248in}{1.424757in}}%
\pgfpathlineto{\pgfqpoint{-0.364838in}{1.413333in}}%
\pgfpathlineto{\pgfqpoint{-0.320661in}{1.320081in}}%
\pgfpathlineto{\pgfqpoint{-0.314309in}{1.306667in}}%
\pgfpathlineto{\pgfqpoint{-0.271074in}{1.215405in}}%
\pgfpathlineto{\pgfqpoint{-0.263780in}{1.200000in}}%
\pgfpathlineto{\pgfqpoint{-0.221488in}{1.110727in}}%
\pgfpathlineto{\pgfqpoint{-0.213251in}{1.093333in}}%
\pgfpathlineto{\pgfqpoint{-0.171901in}{1.006049in}}%
\pgfpathlineto{\pgfqpoint{-0.162723in}{0.986667in}}%
\pgfpathlineto{\pgfqpoint{-0.122314in}{0.901370in}}%
\pgfpathlineto{\pgfqpoint{-0.112195in}{0.880000in}}%
\pgfpathlineto{\pgfqpoint{-0.072727in}{0.796690in}}%
\pgfpathlineto{\pgfqpoint{-0.061667in}{0.773333in}}%
\pgfpathlineto{\pgfqpoint{-0.023140in}{0.692009in}}%
\pgfpathlineto{\pgfqpoint{-0.011140in}{0.666667in}}%
\pgfpathlineto{\pgfqpoint{0.026446in}{0.587327in}}%
\pgfpathlineto{\pgfqpoint{0.039387in}{0.560000in}}%
\pgfpathlineto{\pgfqpoint{0.076033in}{0.482644in}}%
\pgfpathlineto{\pgfqpoint{0.089913in}{0.453333in}}%
\pgfpathlineto{\pgfqpoint{0.125620in}{0.377961in}}%
\pgfpathlineto{\pgfqpoint{0.140439in}{0.346667in}}%
\pgfpathlineto{\pgfqpoint{0.175207in}{0.273277in}}%
\pgfpathlineto{\pgfqpoint{0.190965in}{0.240000in}}%
\pgfpathlineto{\pgfqpoint{0.224793in}{0.168592in}}%
\pgfpathlineto{\pgfqpoint{0.241490in}{0.133333in}}%
\pgfpathlineto{\pgfqpoint{0.274380in}{0.063906in}}%
\pgfpathlineto{\pgfqpoint{0.292014in}{0.026667in}}%
\pgfpathlineto{\pgfqpoint{0.323967in}{-0.040781in}}%
\pgfpathlineto{\pgfqpoint{0.342539in}{-0.080000in}}%
\pgfpathlineto{\pgfqpoint{0.373554in}{-0.145468in}}%
\pgfpathlineto{\pgfqpoint{0.393063in}{-0.186667in}}%
\pgfpathlineto{\pgfqpoint{0.423140in}{-0.250157in}}%
\pgfpathlineto{\pgfqpoint{0.443586in}{-0.293333in}}%
\pgfpathlineto{\pgfqpoint{0.472727in}{-0.354846in}}%
\pgfpathlineto{\pgfqpoint{0.494110in}{-0.400000in}}%
\pgfpathlineto{\pgfqpoint{0.522314in}{-0.459536in}}%
\pgfpathlineto{\pgfqpoint{0.544632in}{-0.506667in}}%
\pgfpathlineto{\pgfqpoint{0.571901in}{-0.564227in}}%
\pgfpathlineto{\pgfqpoint{0.595155in}{-0.613333in}}%
\pgfpathlineto{\pgfqpoint{0.621488in}{-0.668919in}}%
\pgfpathlineto{\pgfqpoint{0.645677in}{-0.720000in}}%
\pgfpathlineto{\pgfqpoint{0.671074in}{-0.773611in}}%
\pgfpathlineto{\pgfqpoint{0.696198in}{-0.826667in}}%
\pgfpathlineto{\pgfqpoint{0.720661in}{-0.878305in}}%
\pgfpathlineto{\pgfqpoint{0.746720in}{-0.933333in}}%
\pgfpathlineto{\pgfqpoint{0.770248in}{-0.982999in}}%
\pgfpathlineto{\pgfqpoint{0.797240in}{-1.040000in}}%
\pgfpathlineto{\pgfqpoint{0.819835in}{-1.087694in}}%
\pgfpathlineto{\pgfqpoint{0.847761in}{-1.146667in}}%
\pgfpathlineto{\pgfqpoint{0.869421in}{-1.192390in}}%
\pgfpathlineto{\pgfqpoint{0.898281in}{-1.253333in}}%
\pgfpathlineto{\pgfqpoint{0.919008in}{-1.297086in}}%
\pgfpathlineto{\pgfqpoint{0.948801in}{-1.360000in}}%
\pgfpathlineto{\pgfqpoint{0.968595in}{-1.401784in}}%
\pgfpathlineto{\pgfqpoint{0.999320in}{-1.466667in}}%
\pgfpathlineto{\pgfqpoint{1.018182in}{-1.506482in}}%
\pgfpathlineto{\pgfqpoint{1.049839in}{-1.573333in}}%
\pgfpathlineto{\pgfqpoint{1.067769in}{-1.611181in}}%
\pgfpathlineto{\pgfqpoint{1.100357in}{-1.680000in}}%
\pgfpathlineto{\pgfqpoint{1.117355in}{-1.715881in}}%
\pgfpathlineto{\pgfqpoint{1.150875in}{-1.786667in}}%
\pgfpathlineto{\pgfqpoint{1.166942in}{-1.820582in}}%
\pgfpathlineto{\pgfqpoint{1.201393in}{-1.893333in}}%
\pgfpathlineto{\pgfqpoint{1.216529in}{-1.925283in}}%
\pgfpathlineto{\pgfqpoint{1.251910in}{-2.000000in}}%
\pgfpathlineto{\pgfqpoint{1.266116in}{-2.029986in}}%
\pgfpathlineto{\pgfqpoint{1.302427in}{-2.106667in}}%
\pgfpathlineto{\pgfqpoint{1.315702in}{-2.134689in}}%
\pgfpathlineto{\pgfqpoint{1.352944in}{-2.213333in}}%
\pgfpathlineto{\pgfqpoint{1.365289in}{-2.239393in}}%
\pgfpathlineto{\pgfqpoint{1.403460in}{-2.320000in}}%
\pgfpathlineto{\pgfqpoint{1.414876in}{-2.344098in}}%
\pgfpathlineto{\pgfqpoint{1.453976in}{-2.426667in}}%
\pgfpathlineto{\pgfqpoint{1.464463in}{-2.448803in}}%
\pgfpathlineto{\pgfqpoint{1.504491in}{-2.533333in}}%
\pgfpathclose%
\pgfusepath{fill}%
\end{pgfscope}%
\begin{pgfscope}%
\pgfpathrectangle{\pgfqpoint{0.800000in}{0.528000in}}{\pgfqpoint{1.963636in}{3.696000in}} %
\pgfusepath{clip}%
\pgfsetbuttcap%
\pgfsetroundjoin%
\definecolor{currentfill}{rgb}{0.944844,0.507658,0.302433}%
\pgfsetfillcolor{currentfill}%
\pgfsetlinewidth{0.000000pt}%
\definecolor{currentstroke}{rgb}{0.000000,0.000000,0.000000}%
\pgfsetstrokecolor{currentstroke}%
\pgfsetdash{}{0pt}%
\pgfpathmoveto{\pgfqpoint{3.745455in}{-1.049197in}}%
\pgfpathlineto{\pgfqpoint{3.745455in}{-1.040000in}}%
\pgfpathlineto{\pgfqpoint{3.745455in}{-0.933333in}}%
\pgfpathlineto{\pgfqpoint{3.745455in}{-0.826667in}}%
\pgfpathlineto{\pgfqpoint{3.745455in}{-0.720000in}}%
\pgfpathlineto{\pgfqpoint{3.745455in}{-0.613333in}}%
\pgfpathlineto{\pgfqpoint{3.745455in}{-0.506667in}}%
\pgfpathlineto{\pgfqpoint{3.745455in}{-0.400000in}}%
\pgfpathlineto{\pgfqpoint{3.745455in}{-0.293333in}}%
\pgfpathlineto{\pgfqpoint{3.745455in}{-0.186667in}}%
\pgfpathlineto{\pgfqpoint{3.745455in}{-0.080000in}}%
\pgfpathlineto{\pgfqpoint{3.745455in}{0.026667in}}%
\pgfpathlineto{\pgfqpoint{3.745455in}{0.133333in}}%
\pgfpathlineto{\pgfqpoint{3.745455in}{0.240000in}}%
\pgfpathlineto{\pgfqpoint{3.745455in}{0.346667in}}%
\pgfpathlineto{\pgfqpoint{3.745455in}{0.453333in}}%
\pgfpathlineto{\pgfqpoint{3.745455in}{0.560000in}}%
\pgfpathlineto{\pgfqpoint{3.745455in}{0.666667in}}%
\pgfpathlineto{\pgfqpoint{3.745455in}{0.773333in}}%
\pgfpathlineto{\pgfqpoint{3.745455in}{0.880000in}}%
\pgfpathlineto{\pgfqpoint{3.745455in}{0.986667in}}%
\pgfpathlineto{\pgfqpoint{3.745455in}{1.093333in}}%
\pgfpathlineto{\pgfqpoint{3.745455in}{1.200000in}}%
\pgfpathlineto{\pgfqpoint{3.745455in}{1.306667in}}%
\pgfpathlineto{\pgfqpoint{3.745455in}{1.413333in}}%
\pgfpathlineto{\pgfqpoint{3.745455in}{1.520000in}}%
\pgfpathlineto{\pgfqpoint{3.745455in}{1.626667in}}%
\pgfpathlineto{\pgfqpoint{3.745455in}{1.733333in}}%
\pgfpathlineto{\pgfqpoint{3.745455in}{1.840000in}}%
\pgfpathlineto{\pgfqpoint{3.745455in}{1.946667in}}%
\pgfpathlineto{\pgfqpoint{3.745455in}{2.053333in}}%
\pgfpathlineto{\pgfqpoint{3.745455in}{2.160000in}}%
\pgfpathlineto{\pgfqpoint{3.745455in}{2.180085in}}%
\pgfpathlineto{\pgfqpoint{3.704455in}{2.266667in}}%
\pgfpathlineto{\pgfqpoint{3.695868in}{2.284792in}}%
\pgfpathlineto{\pgfqpoint{3.653940in}{2.373333in}}%
\pgfpathlineto{\pgfqpoint{3.646281in}{2.389500in}}%
\pgfpathlineto{\pgfqpoint{3.603426in}{2.480000in}}%
\pgfpathlineto{\pgfqpoint{3.596694in}{2.494209in}}%
\pgfpathlineto{\pgfqpoint{3.552912in}{2.586667in}}%
\pgfpathlineto{\pgfqpoint{3.547107in}{2.598919in}}%
\pgfpathlineto{\pgfqpoint{3.502398in}{2.693333in}}%
\pgfpathlineto{\pgfqpoint{3.497521in}{2.703629in}}%
\pgfpathlineto{\pgfqpoint{3.451885in}{2.800000in}}%
\pgfpathlineto{\pgfqpoint{3.447934in}{2.808341in}}%
\pgfpathlineto{\pgfqpoint{3.401372in}{2.906667in}}%
\pgfpathlineto{\pgfqpoint{3.398347in}{2.913053in}}%
\pgfpathlineto{\pgfqpoint{3.350860in}{3.013333in}}%
\pgfpathlineto{\pgfqpoint{3.348760in}{3.017766in}}%
\pgfpathlineto{\pgfqpoint{3.300348in}{3.120000in}}%
\pgfpathlineto{\pgfqpoint{3.299174in}{3.122479in}}%
\pgfpathlineto{\pgfqpoint{3.249837in}{3.226667in}}%
\pgfpathlineto{\pgfqpoint{3.249587in}{3.227194in}}%
\pgfpathlineto{\pgfqpoint{3.200000in}{3.331877in}}%
\pgfpathlineto{\pgfqpoint{3.199310in}{3.333333in}}%
\pgfpathlineto{\pgfqpoint{3.150413in}{3.436548in}}%
\pgfpathlineto{\pgfqpoint{3.148779in}{3.440000in}}%
\pgfpathlineto{\pgfqpoint{3.100826in}{3.541221in}}%
\pgfpathlineto{\pgfqpoint{3.098248in}{3.546667in}}%
\pgfpathlineto{\pgfqpoint{3.051240in}{3.645894in}}%
\pgfpathlineto{\pgfqpoint{3.047717in}{3.653333in}}%
\pgfpathlineto{\pgfqpoint{3.001653in}{3.750568in}}%
\pgfpathlineto{\pgfqpoint{2.997186in}{3.760000in}}%
\pgfpathlineto{\pgfqpoint{2.952066in}{3.855243in}}%
\pgfpathlineto{\pgfqpoint{2.946657in}{3.866667in}}%
\pgfpathlineto{\pgfqpoint{2.902479in}{3.959919in}}%
\pgfpathlineto{\pgfqpoint{2.896127in}{3.973333in}}%
\pgfpathlineto{\pgfqpoint{2.852893in}{4.064595in}}%
\pgfpathlineto{\pgfqpoint{2.845598in}{4.080000in}}%
\pgfpathlineto{\pgfqpoint{2.803306in}{4.169273in}}%
\pgfpathlineto{\pgfqpoint{2.795069in}{4.186667in}}%
\pgfpathlineto{\pgfqpoint{2.753719in}{4.273951in}}%
\pgfpathlineto{\pgfqpoint{2.744541in}{4.293333in}}%
\pgfpathlineto{\pgfqpoint{2.704132in}{4.378630in}}%
\pgfpathlineto{\pgfqpoint{2.694013in}{4.400000in}}%
\pgfpathlineto{\pgfqpoint{2.654545in}{4.483310in}}%
\pgfpathlineto{\pgfqpoint{2.643485in}{4.506667in}}%
\pgfpathlineto{\pgfqpoint{2.604959in}{4.587991in}}%
\pgfpathlineto{\pgfqpoint{2.592958in}{4.613333in}}%
\pgfpathlineto{\pgfqpoint{2.555372in}{4.692673in}}%
\pgfpathlineto{\pgfqpoint{2.542431in}{4.720000in}}%
\pgfpathlineto{\pgfqpoint{2.505785in}{4.797356in}}%
\pgfpathlineto{\pgfqpoint{2.491905in}{4.826667in}}%
\pgfpathlineto{\pgfqpoint{2.456198in}{4.902039in}}%
\pgfpathlineto{\pgfqpoint{2.441379in}{4.933333in}}%
\pgfpathlineto{\pgfqpoint{2.406612in}{5.006723in}}%
\pgfpathlineto{\pgfqpoint{2.390854in}{5.040000in}}%
\pgfpathlineto{\pgfqpoint{2.357025in}{5.111408in}}%
\pgfpathlineto{\pgfqpoint{2.340329in}{5.146667in}}%
\pgfpathlineto{\pgfqpoint{2.307438in}{5.216094in}}%
\pgfpathlineto{\pgfqpoint{2.289804in}{5.253333in}}%
\pgfpathlineto{\pgfqpoint{2.257851in}{5.320781in}}%
\pgfpathlineto{\pgfqpoint{2.239279in}{5.360000in}}%
\pgfpathlineto{\pgfqpoint{2.208264in}{5.425468in}}%
\pgfpathlineto{\pgfqpoint{2.188755in}{5.466667in}}%
\pgfpathlineto{\pgfqpoint{2.158678in}{5.530157in}}%
\pgfpathlineto{\pgfqpoint{2.138232in}{5.573333in}}%
\pgfpathlineto{\pgfqpoint{2.109091in}{5.634846in}}%
\pgfpathlineto{\pgfqpoint{2.087709in}{5.680000in}}%
\pgfpathlineto{\pgfqpoint{2.059504in}{5.739536in}}%
\pgfpathlineto{\pgfqpoint{2.037186in}{5.786667in}}%
\pgfpathlineto{\pgfqpoint{2.009917in}{5.844227in}}%
\pgfpathlineto{\pgfqpoint{1.986663in}{5.893333in}}%
\pgfpathlineto{\pgfqpoint{1.960331in}{5.948919in}}%
\pgfpathlineto{\pgfqpoint{1.936141in}{6.000000in}}%
\pgfpathlineto{\pgfqpoint{1.910744in}{6.053611in}}%
\pgfpathlineto{\pgfqpoint{1.885620in}{6.106667in}}%
\pgfpathlineto{\pgfqpoint{1.861157in}{6.158305in}}%
\pgfpathlineto{\pgfqpoint{1.835099in}{6.213333in}}%
\pgfpathlineto{\pgfqpoint{1.811570in}{6.262999in}}%
\pgfpathlineto{\pgfqpoint{1.784578in}{6.320000in}}%
\pgfpathlineto{\pgfqpoint{1.761983in}{6.367694in}}%
\pgfpathlineto{\pgfqpoint{1.734057in}{6.426667in}}%
\pgfpathlineto{\pgfqpoint{1.712397in}{6.472390in}}%
\pgfpathlineto{\pgfqpoint{1.683537in}{6.533333in}}%
\pgfpathlineto{\pgfqpoint{1.662810in}{6.577086in}}%
\pgfpathlineto{\pgfqpoint{1.633018in}{6.640000in}}%
\pgfpathlineto{\pgfqpoint{1.613223in}{6.681784in}}%
\pgfpathlineto{\pgfqpoint{1.582498in}{6.746667in}}%
\pgfpathlineto{\pgfqpoint{1.563636in}{6.786482in}}%
\pgfpathlineto{\pgfqpoint{1.531979in}{6.853333in}}%
\pgfpathlineto{\pgfqpoint{1.514050in}{6.891181in}}%
\pgfpathlineto{\pgfqpoint{1.481461in}{6.960000in}}%
\pgfpathlineto{\pgfqpoint{1.464463in}{6.995881in}}%
\pgfpathlineto{\pgfqpoint{1.430943in}{7.066667in}}%
\pgfpathlineto{\pgfqpoint{1.414876in}{7.100582in}}%
\pgfpathlineto{\pgfqpoint{1.380425in}{7.173333in}}%
\pgfpathlineto{\pgfqpoint{1.365289in}{7.205283in}}%
\pgfpathlineto{\pgfqpoint{1.329908in}{7.280000in}}%
\pgfpathlineto{\pgfqpoint{1.315702in}{7.309986in}}%
\pgfpathlineto{\pgfqpoint{1.279391in}{7.386667in}}%
\pgfpathlineto{\pgfqpoint{1.266116in}{7.414689in}}%
\pgfpathlineto{\pgfqpoint{1.228874in}{7.493333in}}%
\pgfpathlineto{\pgfqpoint{1.216529in}{7.519393in}}%
\pgfpathlineto{\pgfqpoint{1.178358in}{7.600000in}}%
\pgfpathlineto{\pgfqpoint{1.166942in}{7.624098in}}%
\pgfpathlineto{\pgfqpoint{1.127842in}{7.706667in}}%
\pgfpathlineto{\pgfqpoint{1.117355in}{7.728803in}}%
\pgfpathlineto{\pgfqpoint{1.077327in}{7.813333in}}%
\pgfpathlineto{\pgfqpoint{1.067769in}{7.833510in}}%
\pgfpathlineto{\pgfqpoint{1.026812in}{7.920000in}}%
\pgfpathlineto{\pgfqpoint{1.018182in}{7.920000in}}%
\pgfpathlineto{\pgfqpoint{0.968595in}{7.920000in}}%
\pgfpathlineto{\pgfqpoint{0.919008in}{7.920000in}}%
\pgfpathlineto{\pgfqpoint{0.869421in}{7.920000in}}%
\pgfpathlineto{\pgfqpoint{0.819835in}{7.920000in}}%
\pgfpathlineto{\pgfqpoint{0.770248in}{7.920000in}}%
\pgfpathlineto{\pgfqpoint{0.720661in}{7.920000in}}%
\pgfpathlineto{\pgfqpoint{0.671074in}{7.920000in}}%
\pgfpathlineto{\pgfqpoint{0.621488in}{7.920000in}}%
\pgfpathlineto{\pgfqpoint{0.571901in}{7.920000in}}%
\pgfpathlineto{\pgfqpoint{0.522314in}{7.920000in}}%
\pgfpathlineto{\pgfqpoint{0.472727in}{7.920000in}}%
\pgfpathlineto{\pgfqpoint{0.423140in}{7.920000in}}%
\pgfpathlineto{\pgfqpoint{0.373554in}{7.920000in}}%
\pgfpathlineto{\pgfqpoint{0.323967in}{7.920000in}}%
\pgfpathlineto{\pgfqpoint{0.274380in}{7.920000in}}%
\pgfpathlineto{\pgfqpoint{0.224793in}{7.920000in}}%
\pgfpathlineto{\pgfqpoint{0.175207in}{7.920000in}}%
\pgfpathlineto{\pgfqpoint{0.125620in}{7.920000in}}%
\pgfpathlineto{\pgfqpoint{0.076033in}{7.920000in}}%
\pgfpathlineto{\pgfqpoint{0.026446in}{7.920000in}}%
\pgfpathlineto{\pgfqpoint{-0.023140in}{7.920000in}}%
\pgfpathlineto{\pgfqpoint{-0.072727in}{7.920000in}}%
\pgfpathlineto{\pgfqpoint{-0.122314in}{7.920000in}}%
\pgfpathlineto{\pgfqpoint{-0.171901in}{7.920000in}}%
\pgfpathlineto{\pgfqpoint{-0.221488in}{7.920000in}}%
\pgfpathlineto{\pgfqpoint{-0.271074in}{7.920000in}}%
\pgfpathlineto{\pgfqpoint{-0.320661in}{7.920000in}}%
\pgfpathlineto{\pgfqpoint{-0.370248in}{7.920000in}}%
\pgfpathlineto{\pgfqpoint{-0.419835in}{7.920000in}}%
\pgfpathlineto{\pgfqpoint{-0.469421in}{7.920000in}}%
\pgfpathlineto{\pgfqpoint{-0.502951in}{7.920000in}}%
\pgfpathlineto{\pgfqpoint{-0.469421in}{7.849163in}}%
\pgfpathlineto{\pgfqpoint{-0.452440in}{7.813333in}}%
\pgfpathlineto{\pgfqpoint{-0.419835in}{7.744448in}}%
\pgfpathlineto{\pgfqpoint{-0.401928in}{7.706667in}}%
\pgfpathlineto{\pgfqpoint{-0.370248in}{7.639736in}}%
\pgfpathlineto{\pgfqpoint{-0.351415in}{7.600000in}}%
\pgfpathlineto{\pgfqpoint{-0.320661in}{7.535027in}}%
\pgfpathlineto{\pgfqpoint{-0.300901in}{7.493333in}}%
\pgfpathlineto{\pgfqpoint{-0.271074in}{7.430320in}}%
\pgfpathlineto{\pgfqpoint{-0.250385in}{7.386667in}}%
\pgfpathlineto{\pgfqpoint{-0.221488in}{7.325615in}}%
\pgfpathlineto{\pgfqpoint{-0.199868in}{7.280000in}}%
\pgfpathlineto{\pgfqpoint{-0.171901in}{7.220913in}}%
\pgfpathlineto{\pgfqpoint{-0.149350in}{7.173333in}}%
\pgfpathlineto{\pgfqpoint{-0.122314in}{7.116214in}}%
\pgfpathlineto{\pgfqpoint{-0.098831in}{7.066667in}}%
\pgfpathlineto{\pgfqpoint{-0.072727in}{7.011517in}}%
\pgfpathlineto{\pgfqpoint{-0.048311in}{6.960000in}}%
\pgfpathlineto{\pgfqpoint{-0.023140in}{6.906823in}}%
\pgfpathlineto{\pgfqpoint{0.002211in}{6.853333in}}%
\pgfpathlineto{\pgfqpoint{0.026446in}{6.802131in}}%
\pgfpathlineto{\pgfqpoint{0.052734in}{6.746667in}}%
\pgfpathlineto{\pgfqpoint{0.076033in}{6.697442in}}%
\pgfpathlineto{\pgfqpoint{0.103258in}{6.640000in}}%
\pgfpathlineto{\pgfqpoint{0.125620in}{6.592756in}}%
\pgfpathlineto{\pgfqpoint{0.153784in}{6.533333in}}%
\pgfpathlineto{\pgfqpoint{0.175207in}{6.488073in}}%
\pgfpathlineto{\pgfqpoint{0.204310in}{6.426667in}}%
\pgfpathlineto{\pgfqpoint{0.224793in}{6.383392in}}%
\pgfpathlineto{\pgfqpoint{0.254838in}{6.320000in}}%
\pgfpathlineto{\pgfqpoint{0.274380in}{6.278713in}}%
\pgfpathlineto{\pgfqpoint{0.305367in}{6.213333in}}%
\pgfpathlineto{\pgfqpoint{0.323967in}{6.174038in}}%
\pgfpathlineto{\pgfqpoint{0.355898in}{6.106667in}}%
\pgfpathlineto{\pgfqpoint{0.373554in}{6.069365in}}%
\pgfpathlineto{\pgfqpoint{0.406429in}{6.000000in}}%
\pgfpathlineto{\pgfqpoint{0.423140in}{5.964694in}}%
\pgfpathlineto{\pgfqpoint{0.456962in}{5.893333in}}%
\pgfpathlineto{\pgfqpoint{0.472727in}{5.860027in}}%
\pgfpathlineto{\pgfqpoint{0.507497in}{5.786667in}}%
\pgfpathlineto{\pgfqpoint{0.522314in}{5.755362in}}%
\pgfpathlineto{\pgfqpoint{0.558032in}{5.680000in}}%
\pgfpathlineto{\pgfqpoint{0.571901in}{5.650700in}}%
\pgfpathlineto{\pgfqpoint{0.608569in}{5.573333in}}%
\pgfpathlineto{\pgfqpoint{0.621488in}{5.546040in}}%
\pgfpathlineto{\pgfqpoint{0.659107in}{5.466667in}}%
\pgfpathlineto{\pgfqpoint{0.671074in}{5.441384in}}%
\pgfpathlineto{\pgfqpoint{0.709647in}{5.360000in}}%
\pgfpathlineto{\pgfqpoint{0.720661in}{5.336730in}}%
\pgfpathlineto{\pgfqpoint{0.760188in}{5.253333in}}%
\pgfpathlineto{\pgfqpoint{0.770248in}{5.232079in}}%
\pgfpathlineto{\pgfqpoint{0.810730in}{5.146667in}}%
\pgfpathlineto{\pgfqpoint{0.819835in}{5.127430in}}%
\pgfpathlineto{\pgfqpoint{0.861273in}{5.040000in}}%
\pgfpathlineto{\pgfqpoint{0.869421in}{5.022785in}}%
\pgfpathlineto{\pgfqpoint{0.911818in}{4.933333in}}%
\pgfpathlineto{\pgfqpoint{0.919008in}{4.918142in}}%
\pgfpathlineto{\pgfqpoint{0.962364in}{4.826667in}}%
\pgfpathlineto{\pgfqpoint{0.968595in}{4.813502in}}%
\pgfpathlineto{\pgfqpoint{1.012911in}{4.720000in}}%
\pgfpathlineto{\pgfqpoint{1.018182in}{4.708865in}}%
\pgfpathlineto{\pgfqpoint{1.063460in}{4.613333in}}%
\pgfpathlineto{\pgfqpoint{1.067769in}{4.604231in}}%
\pgfpathlineto{\pgfqpoint{1.114010in}{4.506667in}}%
\pgfpathlineto{\pgfqpoint{1.117355in}{4.499600in}}%
\pgfpathlineto{\pgfqpoint{1.164562in}{4.400000in}}%
\pgfpathlineto{\pgfqpoint{1.166942in}{4.394971in}}%
\pgfpathlineto{\pgfqpoint{1.215115in}{4.293333in}}%
\pgfpathlineto{\pgfqpoint{1.216529in}{4.290345in}}%
\pgfpathlineto{\pgfqpoint{1.265669in}{4.186667in}}%
\pgfpathlineto{\pgfqpoint{1.266116in}{4.185723in}}%
\pgfpathlineto{\pgfqpoint{1.315702in}{4.081027in}}%
\pgfpathlineto{\pgfqpoint{1.316189in}{4.080000in}}%
\pgfpathlineto{\pgfqpoint{1.365289in}{3.976268in}}%
\pgfpathlineto{\pgfqpoint{1.366680in}{3.973333in}}%
\pgfpathlineto{\pgfqpoint{1.414876in}{3.871512in}}%
\pgfpathlineto{\pgfqpoint{1.417172in}{3.866667in}}%
\pgfpathlineto{\pgfqpoint{1.464463in}{3.766758in}}%
\pgfpathlineto{\pgfqpoint{1.467666in}{3.760000in}}%
\pgfpathlineto{\pgfqpoint{1.514050in}{3.662006in}}%
\pgfpathlineto{\pgfqpoint{1.518160in}{3.653333in}}%
\pgfpathlineto{\pgfqpoint{1.563636in}{3.557257in}}%
\pgfpathlineto{\pgfqpoint{1.568656in}{3.546667in}}%
\pgfpathlineto{\pgfqpoint{1.613223in}{3.452510in}}%
\pgfpathlineto{\pgfqpoint{1.619152in}{3.440000in}}%
\pgfpathlineto{\pgfqpoint{1.662810in}{3.347766in}}%
\pgfpathlineto{\pgfqpoint{1.669650in}{3.333333in}}%
\pgfpathlineto{\pgfqpoint{1.712397in}{3.243024in}}%
\pgfpathlineto{\pgfqpoint{1.720149in}{3.226667in}}%
\pgfpathlineto{\pgfqpoint{1.761983in}{3.138284in}}%
\pgfpathlineto{\pgfqpoint{1.770649in}{3.120000in}}%
\pgfpathlineto{\pgfqpoint{1.811570in}{3.033547in}}%
\pgfpathlineto{\pgfqpoint{1.821151in}{3.013333in}}%
\pgfpathlineto{\pgfqpoint{1.861157in}{2.928813in}}%
\pgfpathlineto{\pgfqpoint{1.871653in}{2.906667in}}%
\pgfpathlineto{\pgfqpoint{1.910744in}{2.824080in}}%
\pgfpathlineto{\pgfqpoint{1.922157in}{2.800000in}}%
\pgfpathlineto{\pgfqpoint{1.960331in}{2.719351in}}%
\pgfpathlineto{\pgfqpoint{1.972661in}{2.693333in}}%
\pgfpathlineto{\pgfqpoint{2.009917in}{2.614623in}}%
\pgfpathlineto{\pgfqpoint{2.023167in}{2.586667in}}%
\pgfpathlineto{\pgfqpoint{2.059504in}{2.509898in}}%
\pgfpathlineto{\pgfqpoint{2.073674in}{2.480000in}}%
\pgfpathlineto{\pgfqpoint{2.109091in}{2.405176in}}%
\pgfpathlineto{\pgfqpoint{2.124183in}{2.373333in}}%
\pgfpathlineto{\pgfqpoint{2.158678in}{2.300456in}}%
\pgfpathlineto{\pgfqpoint{2.174692in}{2.266667in}}%
\pgfpathlineto{\pgfqpoint{2.208264in}{2.195739in}}%
\pgfpathlineto{\pgfqpoint{2.225203in}{2.160000in}}%
\pgfpathlineto{\pgfqpoint{2.257851in}{2.091024in}}%
\pgfpathlineto{\pgfqpoint{2.275715in}{2.053333in}}%
\pgfpathlineto{\pgfqpoint{2.307438in}{1.986312in}}%
\pgfpathlineto{\pgfqpoint{2.326228in}{1.946667in}}%
\pgfpathlineto{\pgfqpoint{2.357025in}{1.881602in}}%
\pgfpathlineto{\pgfqpoint{2.376742in}{1.840000in}}%
\pgfpathlineto{\pgfqpoint{2.406612in}{1.776895in}}%
\pgfpathlineto{\pgfqpoint{2.427258in}{1.733333in}}%
\pgfpathlineto{\pgfqpoint{2.456198in}{1.672190in}}%
\pgfpathlineto{\pgfqpoint{2.477774in}{1.626667in}}%
\pgfpathlineto{\pgfqpoint{2.505785in}{1.567488in}}%
\pgfpathlineto{\pgfqpoint{2.528292in}{1.520000in}}%
\pgfpathlineto{\pgfqpoint{2.555372in}{1.462789in}}%
\pgfpathlineto{\pgfqpoint{2.578811in}{1.413333in}}%
\pgfpathlineto{\pgfqpoint{2.604959in}{1.358092in}}%
\pgfpathlineto{\pgfqpoint{2.629332in}{1.306667in}}%
\pgfpathlineto{\pgfqpoint{2.654545in}{1.253398in}}%
\pgfpathlineto{\pgfqpoint{2.679854in}{1.200000in}}%
\pgfpathlineto{\pgfqpoint{2.704132in}{1.148706in}}%
\pgfpathlineto{\pgfqpoint{2.730376in}{1.093333in}}%
\pgfpathlineto{\pgfqpoint{2.753719in}{1.044017in}}%
\pgfpathlineto{\pgfqpoint{2.780900in}{0.986667in}}%
\pgfpathlineto{\pgfqpoint{2.803306in}{0.939331in}}%
\pgfpathlineto{\pgfqpoint{2.831426in}{0.880000in}}%
\pgfpathlineto{\pgfqpoint{2.852893in}{0.834647in}}%
\pgfpathlineto{\pgfqpoint{2.881952in}{0.773333in}}%
\pgfpathlineto{\pgfqpoint{2.902479in}{0.729966in}}%
\pgfpathlineto{\pgfqpoint{2.932480in}{0.666667in}}%
\pgfpathlineto{\pgfqpoint{2.952066in}{0.625287in}}%
\pgfpathlineto{\pgfqpoint{2.983009in}{0.560000in}}%
\pgfpathlineto{\pgfqpoint{3.001653in}{0.520612in}}%
\pgfpathlineto{\pgfqpoint{3.033540in}{0.453333in}}%
\pgfpathlineto{\pgfqpoint{3.051240in}{0.415938in}}%
\pgfpathlineto{\pgfqpoint{3.084071in}{0.346667in}}%
\pgfpathlineto{\pgfqpoint{3.100826in}{0.311268in}}%
\pgfpathlineto{\pgfqpoint{3.134604in}{0.240000in}}%
\pgfpathlineto{\pgfqpoint{3.150413in}{0.206600in}}%
\pgfpathlineto{\pgfqpoint{3.185139in}{0.133333in}}%
\pgfpathlineto{\pgfqpoint{3.200000in}{0.101935in}}%
\pgfpathlineto{\pgfqpoint{3.235674in}{0.026667in}}%
\pgfpathlineto{\pgfqpoint{3.249587in}{-0.002727in}}%
\pgfpathlineto{\pgfqpoint{3.286211in}{-0.080000in}}%
\pgfpathlineto{\pgfqpoint{3.299174in}{-0.107386in}}%
\pgfpathlineto{\pgfqpoint{3.336749in}{-0.186667in}}%
\pgfpathlineto{\pgfqpoint{3.348760in}{-0.212043in}}%
\pgfpathlineto{\pgfqpoint{3.387289in}{-0.293333in}}%
\pgfpathlineto{\pgfqpoint{3.398347in}{-0.316697in}}%
\pgfpathlineto{\pgfqpoint{3.437829in}{-0.400000in}}%
\pgfpathlineto{\pgfqpoint{3.447934in}{-0.421348in}}%
\pgfpathlineto{\pgfqpoint{3.488371in}{-0.506667in}}%
\pgfpathlineto{\pgfqpoint{3.497521in}{-0.525997in}}%
\pgfpathlineto{\pgfqpoint{3.538915in}{-0.613333in}}%
\pgfpathlineto{\pgfqpoint{3.547107in}{-0.630643in}}%
\pgfpathlineto{\pgfqpoint{3.589459in}{-0.720000in}}%
\pgfpathlineto{\pgfqpoint{3.596694in}{-0.735285in}}%
\pgfpathlineto{\pgfqpoint{3.640005in}{-0.826667in}}%
\pgfpathlineto{\pgfqpoint{3.646281in}{-0.839925in}}%
\pgfpathlineto{\pgfqpoint{3.690553in}{-0.933333in}}%
\pgfpathlineto{\pgfqpoint{3.695868in}{-0.944563in}}%
\pgfpathlineto{\pgfqpoint{3.741101in}{-1.040000in}}%
\pgfpathclose%
\pgfusepath{fill}%
\end{pgfscope}%
\begin{pgfscope}%
\pgfpathrectangle{\pgfqpoint{0.800000in}{0.528000in}}{\pgfqpoint{1.963636in}{3.696000in}} %
\pgfusepath{clip}%
\pgfsetbuttcap%
\pgfsetroundjoin%
\definecolor{currentfill}{rgb}{0.976265,0.868016,0.143351}%
\pgfsetfillcolor{currentfill}%
\pgfsetlinewidth{0.000000pt}%
\definecolor{currentstroke}{rgb}{0.000000,0.000000,0.000000}%
\pgfsetstrokecolor{currentstroke}%
\pgfsetdash{}{0pt}%
\pgfpathmoveto{\pgfqpoint{-1.114050in}{-2.640000in}}%
\pgfpathlineto{\pgfqpoint{-1.064463in}{-2.640000in}}%
\pgfpathlineto{\pgfqpoint{-1.014876in}{-2.640000in}}%
\pgfpathlineto{\pgfqpoint{-0.965289in}{-2.640000in}}%
\pgfpathlineto{\pgfqpoint{-0.915702in}{-2.640000in}}%
\pgfpathlineto{\pgfqpoint{-0.866116in}{-2.640000in}}%
\pgfpathlineto{\pgfqpoint{-0.816529in}{-2.640000in}}%
\pgfpathlineto{\pgfqpoint{-0.766942in}{-2.640000in}}%
\pgfpathlineto{\pgfqpoint{-0.717355in}{-2.640000in}}%
\pgfpathlineto{\pgfqpoint{-0.667769in}{-2.640000in}}%
\pgfpathlineto{\pgfqpoint{-0.618182in}{-2.640000in}}%
\pgfpathlineto{\pgfqpoint{-0.568595in}{-2.640000in}}%
\pgfpathlineto{\pgfqpoint{-0.519008in}{-2.640000in}}%
\pgfpathlineto{\pgfqpoint{-0.469421in}{-2.640000in}}%
\pgfpathlineto{\pgfqpoint{-0.419835in}{-2.640000in}}%
\pgfpathlineto{\pgfqpoint{-0.370248in}{-2.640000in}}%
\pgfpathlineto{\pgfqpoint{-0.320661in}{-2.640000in}}%
\pgfpathlineto{\pgfqpoint{-0.271074in}{-2.640000in}}%
\pgfpathlineto{\pgfqpoint{-0.221488in}{-2.640000in}}%
\pgfpathlineto{\pgfqpoint{-0.171901in}{-2.640000in}}%
\pgfpathlineto{\pgfqpoint{-0.122314in}{-2.640000in}}%
\pgfpathlineto{\pgfqpoint{-0.072727in}{-2.640000in}}%
\pgfpathlineto{\pgfqpoint{-0.023140in}{-2.640000in}}%
\pgfpathlineto{\pgfqpoint{0.026446in}{-2.640000in}}%
\pgfpathlineto{\pgfqpoint{0.076033in}{-2.640000in}}%
\pgfpathlineto{\pgfqpoint{0.125620in}{-2.640000in}}%
\pgfpathlineto{\pgfqpoint{0.175207in}{-2.640000in}}%
\pgfpathlineto{\pgfqpoint{0.224793in}{-2.640000in}}%
\pgfpathlineto{\pgfqpoint{0.274380in}{-2.640000in}}%
\pgfpathlineto{\pgfqpoint{0.323967in}{-2.640000in}}%
\pgfpathlineto{\pgfqpoint{0.373554in}{-2.640000in}}%
\pgfpathlineto{\pgfqpoint{0.423140in}{-2.640000in}}%
\pgfpathlineto{\pgfqpoint{0.472727in}{-2.640000in}}%
\pgfpathlineto{\pgfqpoint{0.522314in}{-2.640000in}}%
\pgfpathlineto{\pgfqpoint{0.571901in}{-2.640000in}}%
\pgfpathlineto{\pgfqpoint{0.621488in}{-2.640000in}}%
\pgfpathlineto{\pgfqpoint{0.671074in}{-2.640000in}}%
\pgfpathlineto{\pgfqpoint{0.720661in}{-2.640000in}}%
\pgfpathlineto{\pgfqpoint{0.770248in}{-2.640000in}}%
\pgfpathlineto{\pgfqpoint{0.819835in}{-2.640000in}}%
\pgfpathlineto{\pgfqpoint{0.869421in}{-2.640000in}}%
\pgfpathlineto{\pgfqpoint{0.919008in}{-2.640000in}}%
\pgfpathlineto{\pgfqpoint{0.968595in}{-2.640000in}}%
\pgfpathlineto{\pgfqpoint{1.018182in}{-2.640000in}}%
\pgfpathlineto{\pgfqpoint{1.067769in}{-2.640000in}}%
\pgfpathlineto{\pgfqpoint{1.117355in}{-2.640000in}}%
\pgfpathlineto{\pgfqpoint{1.166942in}{-2.640000in}}%
\pgfpathlineto{\pgfqpoint{1.216529in}{-2.640000in}}%
\pgfpathlineto{\pgfqpoint{1.266116in}{-2.640000in}}%
\pgfpathlineto{\pgfqpoint{1.315702in}{-2.640000in}}%
\pgfpathlineto{\pgfqpoint{1.365289in}{-2.640000in}}%
\pgfpathlineto{\pgfqpoint{1.414876in}{-2.640000in}}%
\pgfpathlineto{\pgfqpoint{1.464463in}{-2.640000in}}%
\pgfpathlineto{\pgfqpoint{1.514050in}{-2.640000in}}%
\pgfpathlineto{\pgfqpoint{1.555006in}{-2.640000in}}%
\pgfpathlineto{\pgfqpoint{1.514050in}{-2.553510in}}%
\pgfpathlineto{\pgfqpoint{1.504491in}{-2.533333in}}%
\pgfpathlineto{\pgfqpoint{1.464463in}{-2.448803in}}%
\pgfpathlineto{\pgfqpoint{1.453976in}{-2.426667in}}%
\pgfpathlineto{\pgfqpoint{1.414876in}{-2.344098in}}%
\pgfpathlineto{\pgfqpoint{1.403460in}{-2.320000in}}%
\pgfpathlineto{\pgfqpoint{1.365289in}{-2.239393in}}%
\pgfpathlineto{\pgfqpoint{1.352944in}{-2.213333in}}%
\pgfpathlineto{\pgfqpoint{1.315702in}{-2.134689in}}%
\pgfpathlineto{\pgfqpoint{1.302427in}{-2.106667in}}%
\pgfpathlineto{\pgfqpoint{1.266116in}{-2.029986in}}%
\pgfpathlineto{\pgfqpoint{1.251910in}{-2.000000in}}%
\pgfpathlineto{\pgfqpoint{1.216529in}{-1.925283in}}%
\pgfpathlineto{\pgfqpoint{1.201393in}{-1.893333in}}%
\pgfpathlineto{\pgfqpoint{1.166942in}{-1.820582in}}%
\pgfpathlineto{\pgfqpoint{1.150875in}{-1.786667in}}%
\pgfpathlineto{\pgfqpoint{1.117355in}{-1.715881in}}%
\pgfpathlineto{\pgfqpoint{1.100357in}{-1.680000in}}%
\pgfpathlineto{\pgfqpoint{1.067769in}{-1.611181in}}%
\pgfpathlineto{\pgfqpoint{1.049839in}{-1.573333in}}%
\pgfpathlineto{\pgfqpoint{1.018182in}{-1.506482in}}%
\pgfpathlineto{\pgfqpoint{0.999320in}{-1.466667in}}%
\pgfpathlineto{\pgfqpoint{0.968595in}{-1.401784in}}%
\pgfpathlineto{\pgfqpoint{0.948801in}{-1.360000in}}%
\pgfpathlineto{\pgfqpoint{0.919008in}{-1.297086in}}%
\pgfpathlineto{\pgfqpoint{0.898281in}{-1.253333in}}%
\pgfpathlineto{\pgfqpoint{0.869421in}{-1.192390in}}%
\pgfpathlineto{\pgfqpoint{0.847761in}{-1.146667in}}%
\pgfpathlineto{\pgfqpoint{0.819835in}{-1.087694in}}%
\pgfpathlineto{\pgfqpoint{0.797240in}{-1.040000in}}%
\pgfpathlineto{\pgfqpoint{0.770248in}{-0.982999in}}%
\pgfpathlineto{\pgfqpoint{0.746720in}{-0.933333in}}%
\pgfpathlineto{\pgfqpoint{0.720661in}{-0.878305in}}%
\pgfpathlineto{\pgfqpoint{0.696198in}{-0.826667in}}%
\pgfpathlineto{\pgfqpoint{0.671074in}{-0.773611in}}%
\pgfpathlineto{\pgfqpoint{0.645677in}{-0.720000in}}%
\pgfpathlineto{\pgfqpoint{0.621488in}{-0.668919in}}%
\pgfpathlineto{\pgfqpoint{0.595155in}{-0.613333in}}%
\pgfpathlineto{\pgfqpoint{0.571901in}{-0.564227in}}%
\pgfpathlineto{\pgfqpoint{0.544632in}{-0.506667in}}%
\pgfpathlineto{\pgfqpoint{0.522314in}{-0.459536in}}%
\pgfpathlineto{\pgfqpoint{0.494110in}{-0.400000in}}%
\pgfpathlineto{\pgfqpoint{0.472727in}{-0.354846in}}%
\pgfpathlineto{\pgfqpoint{0.443586in}{-0.293333in}}%
\pgfpathlineto{\pgfqpoint{0.423140in}{-0.250157in}}%
\pgfpathlineto{\pgfqpoint{0.393063in}{-0.186667in}}%
\pgfpathlineto{\pgfqpoint{0.373554in}{-0.145468in}}%
\pgfpathlineto{\pgfqpoint{0.342539in}{-0.080000in}}%
\pgfpathlineto{\pgfqpoint{0.323967in}{-0.040781in}}%
\pgfpathlineto{\pgfqpoint{0.292014in}{0.026667in}}%
\pgfpathlineto{\pgfqpoint{0.274380in}{0.063906in}}%
\pgfpathlineto{\pgfqpoint{0.241490in}{0.133333in}}%
\pgfpathlineto{\pgfqpoint{0.224793in}{0.168592in}}%
\pgfpathlineto{\pgfqpoint{0.190965in}{0.240000in}}%
\pgfpathlineto{\pgfqpoint{0.175207in}{0.273277in}}%
\pgfpathlineto{\pgfqpoint{0.140439in}{0.346667in}}%
\pgfpathlineto{\pgfqpoint{0.125620in}{0.377961in}}%
\pgfpathlineto{\pgfqpoint{0.089913in}{0.453333in}}%
\pgfpathlineto{\pgfqpoint{0.076033in}{0.482644in}}%
\pgfpathlineto{\pgfqpoint{0.039387in}{0.560000in}}%
\pgfpathlineto{\pgfqpoint{0.026446in}{0.587327in}}%
\pgfpathlineto{\pgfqpoint{-0.011140in}{0.666667in}}%
\pgfpathlineto{\pgfqpoint{-0.023140in}{0.692009in}}%
\pgfpathlineto{\pgfqpoint{-0.061667in}{0.773333in}}%
\pgfpathlineto{\pgfqpoint{-0.072727in}{0.796690in}}%
\pgfpathlineto{\pgfqpoint{-0.112195in}{0.880000in}}%
\pgfpathlineto{\pgfqpoint{-0.122314in}{0.901370in}}%
\pgfpathlineto{\pgfqpoint{-0.162723in}{0.986667in}}%
\pgfpathlineto{\pgfqpoint{-0.171901in}{1.006049in}}%
\pgfpathlineto{\pgfqpoint{-0.213251in}{1.093333in}}%
\pgfpathlineto{\pgfqpoint{-0.221488in}{1.110727in}}%
\pgfpathlineto{\pgfqpoint{-0.263780in}{1.200000in}}%
\pgfpathlineto{\pgfqpoint{-0.271074in}{1.215405in}}%
\pgfpathlineto{\pgfqpoint{-0.314309in}{1.306667in}}%
\pgfpathlineto{\pgfqpoint{-0.320661in}{1.320081in}}%
\pgfpathlineto{\pgfqpoint{-0.364838in}{1.413333in}}%
\pgfpathlineto{\pgfqpoint{-0.370248in}{1.424757in}}%
\pgfpathlineto{\pgfqpoint{-0.415368in}{1.520000in}}%
\pgfpathlineto{\pgfqpoint{-0.419835in}{1.529432in}}%
\pgfpathlineto{\pgfqpoint{-0.465899in}{1.626667in}}%
\pgfpathlineto{\pgfqpoint{-0.469421in}{1.634106in}}%
\pgfpathlineto{\pgfqpoint{-0.516429in}{1.733333in}}%
\pgfpathlineto{\pgfqpoint{-0.519008in}{1.738779in}}%
\pgfpathlineto{\pgfqpoint{-0.566960in}{1.840000in}}%
\pgfpathlineto{\pgfqpoint{-0.568595in}{1.843452in}}%
\pgfpathlineto{\pgfqpoint{-0.617492in}{1.946667in}}%
\pgfpathlineto{\pgfqpoint{-0.618182in}{1.948123in}}%
\pgfpathlineto{\pgfqpoint{-0.667769in}{2.052806in}}%
\pgfpathlineto{\pgfqpoint{-0.668018in}{2.053333in}}%
\pgfpathlineto{\pgfqpoint{-0.717355in}{2.157521in}}%
\pgfpathlineto{\pgfqpoint{-0.718530in}{2.160000in}}%
\pgfpathlineto{\pgfqpoint{-0.766942in}{2.262234in}}%
\pgfpathlineto{\pgfqpoint{-0.769042in}{2.266667in}}%
\pgfpathlineto{\pgfqpoint{-0.816529in}{2.366947in}}%
\pgfpathlineto{\pgfqpoint{-0.819554in}{2.373333in}}%
\pgfpathlineto{\pgfqpoint{-0.866116in}{2.471659in}}%
\pgfpathlineto{\pgfqpoint{-0.870067in}{2.480000in}}%
\pgfpathlineto{\pgfqpoint{-0.915702in}{2.576371in}}%
\pgfpathlineto{\pgfqpoint{-0.920580in}{2.586667in}}%
\pgfpathlineto{\pgfqpoint{-0.965289in}{2.681081in}}%
\pgfpathlineto{\pgfqpoint{-0.971094in}{2.693333in}}%
\pgfpathlineto{\pgfqpoint{-1.014876in}{2.785791in}}%
\pgfpathlineto{\pgfqpoint{-1.021608in}{2.800000in}}%
\pgfpathlineto{\pgfqpoint{-1.064463in}{2.890500in}}%
\pgfpathlineto{\pgfqpoint{-1.072122in}{2.906667in}}%
\pgfpathlineto{\pgfqpoint{-1.114050in}{2.995208in}}%
\pgfpathlineto{\pgfqpoint{-1.122636in}{3.013333in}}%
\pgfpathlineto{\pgfqpoint{-1.163636in}{3.099915in}}%
\pgfpathlineto{\pgfqpoint{-1.163636in}{3.013333in}}%
\pgfpathlineto{\pgfqpoint{-1.163636in}{2.906667in}}%
\pgfpathlineto{\pgfqpoint{-1.163636in}{2.800000in}}%
\pgfpathlineto{\pgfqpoint{-1.163636in}{2.693333in}}%
\pgfpathlineto{\pgfqpoint{-1.163636in}{2.586667in}}%
\pgfpathlineto{\pgfqpoint{-1.163636in}{2.480000in}}%
\pgfpathlineto{\pgfqpoint{-1.163636in}{2.373333in}}%
\pgfpathlineto{\pgfqpoint{-1.163636in}{2.266667in}}%
\pgfpathlineto{\pgfqpoint{-1.163636in}{2.160000in}}%
\pgfpathlineto{\pgfqpoint{-1.163636in}{2.053333in}}%
\pgfpathlineto{\pgfqpoint{-1.163636in}{1.946667in}}%
\pgfpathlineto{\pgfqpoint{-1.163636in}{1.840000in}}%
\pgfpathlineto{\pgfqpoint{-1.163636in}{1.733333in}}%
\pgfpathlineto{\pgfqpoint{-1.163636in}{1.626667in}}%
\pgfpathlineto{\pgfqpoint{-1.163636in}{1.520000in}}%
\pgfpathlineto{\pgfqpoint{-1.163636in}{1.413333in}}%
\pgfpathlineto{\pgfqpoint{-1.163636in}{1.306667in}}%
\pgfpathlineto{\pgfqpoint{-1.163636in}{1.200000in}}%
\pgfpathlineto{\pgfqpoint{-1.163636in}{1.093333in}}%
\pgfpathlineto{\pgfqpoint{-1.163636in}{0.986667in}}%
\pgfpathlineto{\pgfqpoint{-1.163636in}{0.880000in}}%
\pgfpathlineto{\pgfqpoint{-1.163636in}{0.773333in}}%
\pgfpathlineto{\pgfqpoint{-1.163636in}{0.666667in}}%
\pgfpathlineto{\pgfqpoint{-1.163636in}{0.560000in}}%
\pgfpathlineto{\pgfqpoint{-1.163636in}{0.453333in}}%
\pgfpathlineto{\pgfqpoint{-1.163636in}{0.346667in}}%
\pgfpathlineto{\pgfqpoint{-1.163636in}{0.240000in}}%
\pgfpathlineto{\pgfqpoint{-1.163636in}{0.133333in}}%
\pgfpathlineto{\pgfqpoint{-1.163636in}{0.026667in}}%
\pgfpathlineto{\pgfqpoint{-1.163636in}{-0.080000in}}%
\pgfpathlineto{\pgfqpoint{-1.163636in}{-0.186667in}}%
\pgfpathlineto{\pgfqpoint{-1.163636in}{-0.293333in}}%
\pgfpathlineto{\pgfqpoint{-1.163636in}{-0.400000in}}%
\pgfpathlineto{\pgfqpoint{-1.163636in}{-0.506667in}}%
\pgfpathlineto{\pgfqpoint{-1.163636in}{-0.613333in}}%
\pgfpathlineto{\pgfqpoint{-1.163636in}{-0.720000in}}%
\pgfpathlineto{\pgfqpoint{-1.163636in}{-0.826667in}}%
\pgfpathlineto{\pgfqpoint{-1.163636in}{-0.933333in}}%
\pgfpathlineto{\pgfqpoint{-1.163636in}{-1.040000in}}%
\pgfpathlineto{\pgfqpoint{-1.163636in}{-1.146667in}}%
\pgfpathlineto{\pgfqpoint{-1.163636in}{-1.253333in}}%
\pgfpathlineto{\pgfqpoint{-1.163636in}{-1.360000in}}%
\pgfpathlineto{\pgfqpoint{-1.163636in}{-1.466667in}}%
\pgfpathlineto{\pgfqpoint{-1.163636in}{-1.573333in}}%
\pgfpathlineto{\pgfqpoint{-1.163636in}{-1.680000in}}%
\pgfpathlineto{\pgfqpoint{-1.163636in}{-1.786667in}}%
\pgfpathlineto{\pgfqpoint{-1.163636in}{-1.893333in}}%
\pgfpathlineto{\pgfqpoint{-1.163636in}{-2.000000in}}%
\pgfpathlineto{\pgfqpoint{-1.163636in}{-2.106667in}}%
\pgfpathlineto{\pgfqpoint{-1.163636in}{-2.213333in}}%
\pgfpathlineto{\pgfqpoint{-1.163636in}{-2.320000in}}%
\pgfpathlineto{\pgfqpoint{-1.163636in}{-2.426667in}}%
\pgfpathlineto{\pgfqpoint{-1.163636in}{-2.533333in}}%
\pgfpathlineto{\pgfqpoint{-1.163636in}{-2.640000in}}%
\pgfpathclose%
\pgfusepath{fill}%
\end{pgfscope}%
\begin{pgfscope}%
\pgfpathrectangle{\pgfqpoint{0.800000in}{0.528000in}}{\pgfqpoint{1.963636in}{3.696000in}} %
\pgfusepath{clip}%
\pgfsetbuttcap%
\pgfsetroundjoin%
\definecolor{currentfill}{rgb}{0.976265,0.868016,0.143351}%
\pgfsetfillcolor{currentfill}%
\pgfsetlinewidth{0.000000pt}%
\definecolor{currentstroke}{rgb}{0.000000,0.000000,0.000000}%
\pgfsetstrokecolor{currentstroke}%
\pgfsetdash{}{0pt}%
\pgfpathmoveto{\pgfqpoint{3.745455in}{2.180085in}}%
\pgfpathlineto{\pgfqpoint{3.745455in}{2.266667in}}%
\pgfpathlineto{\pgfqpoint{3.745455in}{2.373333in}}%
\pgfpathlineto{\pgfqpoint{3.745455in}{2.480000in}}%
\pgfpathlineto{\pgfqpoint{3.745455in}{2.586667in}}%
\pgfpathlineto{\pgfqpoint{3.745455in}{2.693333in}}%
\pgfpathlineto{\pgfqpoint{3.745455in}{2.800000in}}%
\pgfpathlineto{\pgfqpoint{3.745455in}{2.906667in}}%
\pgfpathlineto{\pgfqpoint{3.745455in}{3.013333in}}%
\pgfpathlineto{\pgfqpoint{3.745455in}{3.120000in}}%
\pgfpathlineto{\pgfqpoint{3.745455in}{3.226667in}}%
\pgfpathlineto{\pgfqpoint{3.745455in}{3.333333in}}%
\pgfpathlineto{\pgfqpoint{3.745455in}{3.440000in}}%
\pgfpathlineto{\pgfqpoint{3.745455in}{3.546667in}}%
\pgfpathlineto{\pgfqpoint{3.745455in}{3.653333in}}%
\pgfpathlineto{\pgfqpoint{3.745455in}{3.760000in}}%
\pgfpathlineto{\pgfqpoint{3.745455in}{3.866667in}}%
\pgfpathlineto{\pgfqpoint{3.745455in}{3.973333in}}%
\pgfpathlineto{\pgfqpoint{3.745455in}{4.080000in}}%
\pgfpathlineto{\pgfqpoint{3.745455in}{4.186667in}}%
\pgfpathlineto{\pgfqpoint{3.745455in}{4.293333in}}%
\pgfpathlineto{\pgfqpoint{3.745455in}{4.400000in}}%
\pgfpathlineto{\pgfqpoint{3.745455in}{4.506667in}}%
\pgfpathlineto{\pgfqpoint{3.745455in}{4.613333in}}%
\pgfpathlineto{\pgfqpoint{3.745455in}{4.720000in}}%
\pgfpathlineto{\pgfqpoint{3.745455in}{4.826667in}}%
\pgfpathlineto{\pgfqpoint{3.745455in}{4.933333in}}%
\pgfpathlineto{\pgfqpoint{3.745455in}{5.040000in}}%
\pgfpathlineto{\pgfqpoint{3.745455in}{5.146667in}}%
\pgfpathlineto{\pgfqpoint{3.745455in}{5.253333in}}%
\pgfpathlineto{\pgfqpoint{3.745455in}{5.360000in}}%
\pgfpathlineto{\pgfqpoint{3.745455in}{5.466667in}}%
\pgfpathlineto{\pgfqpoint{3.745455in}{5.573333in}}%
\pgfpathlineto{\pgfqpoint{3.745455in}{5.680000in}}%
\pgfpathlineto{\pgfqpoint{3.745455in}{5.786667in}}%
\pgfpathlineto{\pgfqpoint{3.745455in}{5.893333in}}%
\pgfpathlineto{\pgfqpoint{3.745455in}{6.000000in}}%
\pgfpathlineto{\pgfqpoint{3.745455in}{6.106667in}}%
\pgfpathlineto{\pgfqpoint{3.745455in}{6.213333in}}%
\pgfpathlineto{\pgfqpoint{3.745455in}{6.320000in}}%
\pgfpathlineto{\pgfqpoint{3.745455in}{6.426667in}}%
\pgfpathlineto{\pgfqpoint{3.745455in}{6.533333in}}%
\pgfpathlineto{\pgfqpoint{3.745455in}{6.640000in}}%
\pgfpathlineto{\pgfqpoint{3.745455in}{6.746667in}}%
\pgfpathlineto{\pgfqpoint{3.745455in}{6.853333in}}%
\pgfpathlineto{\pgfqpoint{3.745455in}{6.960000in}}%
\pgfpathlineto{\pgfqpoint{3.745455in}{7.066667in}}%
\pgfpathlineto{\pgfqpoint{3.745455in}{7.173333in}}%
\pgfpathlineto{\pgfqpoint{3.745455in}{7.280000in}}%
\pgfpathlineto{\pgfqpoint{3.745455in}{7.386667in}}%
\pgfpathlineto{\pgfqpoint{3.745455in}{7.493333in}}%
\pgfpathlineto{\pgfqpoint{3.745455in}{7.600000in}}%
\pgfpathlineto{\pgfqpoint{3.745455in}{7.706667in}}%
\pgfpathlineto{\pgfqpoint{3.745455in}{7.813333in}}%
\pgfpathlineto{\pgfqpoint{3.745455in}{7.920000in}}%
\pgfpathlineto{\pgfqpoint{3.695868in}{7.920000in}}%
\pgfpathlineto{\pgfqpoint{3.646281in}{7.920000in}}%
\pgfpathlineto{\pgfqpoint{3.596694in}{7.920000in}}%
\pgfpathlineto{\pgfqpoint{3.547107in}{7.920000in}}%
\pgfpathlineto{\pgfqpoint{3.497521in}{7.920000in}}%
\pgfpathlineto{\pgfqpoint{3.447934in}{7.920000in}}%
\pgfpathlineto{\pgfqpoint{3.398347in}{7.920000in}}%
\pgfpathlineto{\pgfqpoint{3.348760in}{7.920000in}}%
\pgfpathlineto{\pgfqpoint{3.299174in}{7.920000in}}%
\pgfpathlineto{\pgfqpoint{3.249587in}{7.920000in}}%
\pgfpathlineto{\pgfqpoint{3.200000in}{7.920000in}}%
\pgfpathlineto{\pgfqpoint{3.150413in}{7.920000in}}%
\pgfpathlineto{\pgfqpoint{3.100826in}{7.920000in}}%
\pgfpathlineto{\pgfqpoint{3.051240in}{7.920000in}}%
\pgfpathlineto{\pgfqpoint{3.001653in}{7.920000in}}%
\pgfpathlineto{\pgfqpoint{2.952066in}{7.920000in}}%
\pgfpathlineto{\pgfqpoint{2.902479in}{7.920000in}}%
\pgfpathlineto{\pgfqpoint{2.852893in}{7.920000in}}%
\pgfpathlineto{\pgfqpoint{2.803306in}{7.920000in}}%
\pgfpathlineto{\pgfqpoint{2.753719in}{7.920000in}}%
\pgfpathlineto{\pgfqpoint{2.704132in}{7.920000in}}%
\pgfpathlineto{\pgfqpoint{2.654545in}{7.920000in}}%
\pgfpathlineto{\pgfqpoint{2.604959in}{7.920000in}}%
\pgfpathlineto{\pgfqpoint{2.555372in}{7.920000in}}%
\pgfpathlineto{\pgfqpoint{2.505785in}{7.920000in}}%
\pgfpathlineto{\pgfqpoint{2.456198in}{7.920000in}}%
\pgfpathlineto{\pgfqpoint{2.406612in}{7.920000in}}%
\pgfpathlineto{\pgfqpoint{2.357025in}{7.920000in}}%
\pgfpathlineto{\pgfqpoint{2.307438in}{7.920000in}}%
\pgfpathlineto{\pgfqpoint{2.257851in}{7.920000in}}%
\pgfpathlineto{\pgfqpoint{2.208264in}{7.920000in}}%
\pgfpathlineto{\pgfqpoint{2.158678in}{7.920000in}}%
\pgfpathlineto{\pgfqpoint{2.109091in}{7.920000in}}%
\pgfpathlineto{\pgfqpoint{2.059504in}{7.920000in}}%
\pgfpathlineto{\pgfqpoint{2.009917in}{7.920000in}}%
\pgfpathlineto{\pgfqpoint{1.960331in}{7.920000in}}%
\pgfpathlineto{\pgfqpoint{1.910744in}{7.920000in}}%
\pgfpathlineto{\pgfqpoint{1.861157in}{7.920000in}}%
\pgfpathlineto{\pgfqpoint{1.811570in}{7.920000in}}%
\pgfpathlineto{\pgfqpoint{1.761983in}{7.920000in}}%
\pgfpathlineto{\pgfqpoint{1.712397in}{7.920000in}}%
\pgfpathlineto{\pgfqpoint{1.662810in}{7.920000in}}%
\pgfpathlineto{\pgfqpoint{1.613223in}{7.920000in}}%
\pgfpathlineto{\pgfqpoint{1.563636in}{7.920000in}}%
\pgfpathlineto{\pgfqpoint{1.514050in}{7.920000in}}%
\pgfpathlineto{\pgfqpoint{1.464463in}{7.920000in}}%
\pgfpathlineto{\pgfqpoint{1.414876in}{7.920000in}}%
\pgfpathlineto{\pgfqpoint{1.365289in}{7.920000in}}%
\pgfpathlineto{\pgfqpoint{1.315702in}{7.920000in}}%
\pgfpathlineto{\pgfqpoint{1.266116in}{7.920000in}}%
\pgfpathlineto{\pgfqpoint{1.216529in}{7.920000in}}%
\pgfpathlineto{\pgfqpoint{1.166942in}{7.920000in}}%
\pgfpathlineto{\pgfqpoint{1.117355in}{7.920000in}}%
\pgfpathlineto{\pgfqpoint{1.067769in}{7.920000in}}%
\pgfpathlineto{\pgfqpoint{1.026812in}{7.920000in}}%
\pgfpathlineto{\pgfqpoint{1.067769in}{7.833510in}}%
\pgfpathlineto{\pgfqpoint{1.077327in}{7.813333in}}%
\pgfpathlineto{\pgfqpoint{1.117355in}{7.728803in}}%
\pgfpathlineto{\pgfqpoint{1.127842in}{7.706667in}}%
\pgfpathlineto{\pgfqpoint{1.166942in}{7.624098in}}%
\pgfpathlineto{\pgfqpoint{1.178358in}{7.600000in}}%
\pgfpathlineto{\pgfqpoint{1.216529in}{7.519393in}}%
\pgfpathlineto{\pgfqpoint{1.228874in}{7.493333in}}%
\pgfpathlineto{\pgfqpoint{1.266116in}{7.414689in}}%
\pgfpathlineto{\pgfqpoint{1.279391in}{7.386667in}}%
\pgfpathlineto{\pgfqpoint{1.315702in}{7.309986in}}%
\pgfpathlineto{\pgfqpoint{1.329908in}{7.280000in}}%
\pgfpathlineto{\pgfqpoint{1.365289in}{7.205283in}}%
\pgfpathlineto{\pgfqpoint{1.380425in}{7.173333in}}%
\pgfpathlineto{\pgfqpoint{1.414876in}{7.100582in}}%
\pgfpathlineto{\pgfqpoint{1.430943in}{7.066667in}}%
\pgfpathlineto{\pgfqpoint{1.464463in}{6.995881in}}%
\pgfpathlineto{\pgfqpoint{1.481461in}{6.960000in}}%
\pgfpathlineto{\pgfqpoint{1.514050in}{6.891181in}}%
\pgfpathlineto{\pgfqpoint{1.531979in}{6.853333in}}%
\pgfpathlineto{\pgfqpoint{1.563636in}{6.786482in}}%
\pgfpathlineto{\pgfqpoint{1.582498in}{6.746667in}}%
\pgfpathlineto{\pgfqpoint{1.613223in}{6.681784in}}%
\pgfpathlineto{\pgfqpoint{1.633018in}{6.640000in}}%
\pgfpathlineto{\pgfqpoint{1.662810in}{6.577086in}}%
\pgfpathlineto{\pgfqpoint{1.683537in}{6.533333in}}%
\pgfpathlineto{\pgfqpoint{1.712397in}{6.472390in}}%
\pgfpathlineto{\pgfqpoint{1.734057in}{6.426667in}}%
\pgfpathlineto{\pgfqpoint{1.761983in}{6.367694in}}%
\pgfpathlineto{\pgfqpoint{1.784578in}{6.320000in}}%
\pgfpathlineto{\pgfqpoint{1.811570in}{6.262999in}}%
\pgfpathlineto{\pgfqpoint{1.835099in}{6.213333in}}%
\pgfpathlineto{\pgfqpoint{1.861157in}{6.158305in}}%
\pgfpathlineto{\pgfqpoint{1.885620in}{6.106667in}}%
\pgfpathlineto{\pgfqpoint{1.910744in}{6.053611in}}%
\pgfpathlineto{\pgfqpoint{1.936141in}{6.000000in}}%
\pgfpathlineto{\pgfqpoint{1.960331in}{5.948919in}}%
\pgfpathlineto{\pgfqpoint{1.986663in}{5.893333in}}%
\pgfpathlineto{\pgfqpoint{2.009917in}{5.844227in}}%
\pgfpathlineto{\pgfqpoint{2.037186in}{5.786667in}}%
\pgfpathlineto{\pgfqpoint{2.059504in}{5.739536in}}%
\pgfpathlineto{\pgfqpoint{2.087709in}{5.680000in}}%
\pgfpathlineto{\pgfqpoint{2.109091in}{5.634846in}}%
\pgfpathlineto{\pgfqpoint{2.138232in}{5.573333in}}%
\pgfpathlineto{\pgfqpoint{2.158678in}{5.530157in}}%
\pgfpathlineto{\pgfqpoint{2.188755in}{5.466667in}}%
\pgfpathlineto{\pgfqpoint{2.208264in}{5.425468in}}%
\pgfpathlineto{\pgfqpoint{2.239279in}{5.360000in}}%
\pgfpathlineto{\pgfqpoint{2.257851in}{5.320781in}}%
\pgfpathlineto{\pgfqpoint{2.289804in}{5.253333in}}%
\pgfpathlineto{\pgfqpoint{2.307438in}{5.216094in}}%
\pgfpathlineto{\pgfqpoint{2.340329in}{5.146667in}}%
\pgfpathlineto{\pgfqpoint{2.357025in}{5.111408in}}%
\pgfpathlineto{\pgfqpoint{2.390854in}{5.040000in}}%
\pgfpathlineto{\pgfqpoint{2.406612in}{5.006723in}}%
\pgfpathlineto{\pgfqpoint{2.441379in}{4.933333in}}%
\pgfpathlineto{\pgfqpoint{2.456198in}{4.902039in}}%
\pgfpathlineto{\pgfqpoint{2.491905in}{4.826667in}}%
\pgfpathlineto{\pgfqpoint{2.505785in}{4.797356in}}%
\pgfpathlineto{\pgfqpoint{2.542431in}{4.720000in}}%
\pgfpathlineto{\pgfqpoint{2.555372in}{4.692673in}}%
\pgfpathlineto{\pgfqpoint{2.592958in}{4.613333in}}%
\pgfpathlineto{\pgfqpoint{2.604959in}{4.587991in}}%
\pgfpathlineto{\pgfqpoint{2.643485in}{4.506667in}}%
\pgfpathlineto{\pgfqpoint{2.654545in}{4.483310in}}%
\pgfpathlineto{\pgfqpoint{2.694013in}{4.400000in}}%
\pgfpathlineto{\pgfqpoint{2.704132in}{4.378630in}}%
\pgfpathlineto{\pgfqpoint{2.744541in}{4.293333in}}%
\pgfpathlineto{\pgfqpoint{2.753719in}{4.273951in}}%
\pgfpathlineto{\pgfqpoint{2.795069in}{4.186667in}}%
\pgfpathlineto{\pgfqpoint{2.803306in}{4.169273in}}%
\pgfpathlineto{\pgfqpoint{2.845598in}{4.080000in}}%
\pgfpathlineto{\pgfqpoint{2.852893in}{4.064595in}}%
\pgfpathlineto{\pgfqpoint{2.896127in}{3.973333in}}%
\pgfpathlineto{\pgfqpoint{2.902479in}{3.959919in}}%
\pgfpathlineto{\pgfqpoint{2.946657in}{3.866667in}}%
\pgfpathlineto{\pgfqpoint{2.952066in}{3.855243in}}%
\pgfpathlineto{\pgfqpoint{2.997186in}{3.760000in}}%
\pgfpathlineto{\pgfqpoint{3.001653in}{3.750568in}}%
\pgfpathlineto{\pgfqpoint{3.047717in}{3.653333in}}%
\pgfpathlineto{\pgfqpoint{3.051240in}{3.645894in}}%
\pgfpathlineto{\pgfqpoint{3.098248in}{3.546667in}}%
\pgfpathlineto{\pgfqpoint{3.100826in}{3.541221in}}%
\pgfpathlineto{\pgfqpoint{3.148779in}{3.440000in}}%
\pgfpathlineto{\pgfqpoint{3.150413in}{3.436548in}}%
\pgfpathlineto{\pgfqpoint{3.199310in}{3.333333in}}%
\pgfpathlineto{\pgfqpoint{3.200000in}{3.331877in}}%
\pgfpathlineto{\pgfqpoint{3.249587in}{3.227194in}}%
\pgfpathlineto{\pgfqpoint{3.249837in}{3.226667in}}%
\pgfpathlineto{\pgfqpoint{3.299174in}{3.122479in}}%
\pgfpathlineto{\pgfqpoint{3.300348in}{3.120000in}}%
\pgfpathlineto{\pgfqpoint{3.348760in}{3.017766in}}%
\pgfpathlineto{\pgfqpoint{3.350860in}{3.013333in}}%
\pgfpathlineto{\pgfqpoint{3.398347in}{2.913053in}}%
\pgfpathlineto{\pgfqpoint{3.401372in}{2.906667in}}%
\pgfpathlineto{\pgfqpoint{3.447934in}{2.808341in}}%
\pgfpathlineto{\pgfqpoint{3.451885in}{2.800000in}}%
\pgfpathlineto{\pgfqpoint{3.497521in}{2.703629in}}%
\pgfpathlineto{\pgfqpoint{3.502398in}{2.693333in}}%
\pgfpathlineto{\pgfqpoint{3.547107in}{2.598919in}}%
\pgfpathlineto{\pgfqpoint{3.552912in}{2.586667in}}%
\pgfpathlineto{\pgfqpoint{3.596694in}{2.494209in}}%
\pgfpathlineto{\pgfqpoint{3.603426in}{2.480000in}}%
\pgfpathlineto{\pgfqpoint{3.646281in}{2.389500in}}%
\pgfpathlineto{\pgfqpoint{3.653940in}{2.373333in}}%
\pgfpathlineto{\pgfqpoint{3.695868in}{2.284792in}}%
\pgfpathlineto{\pgfqpoint{3.704455in}{2.266667in}}%
\pgfpathclose%
\pgfusepath{fill}%
\end{pgfscope}%
\begin{pgfscope}%
\pgfpathrectangle{\pgfqpoint{0.800000in}{0.528000in}}{\pgfqpoint{1.963636in}{3.696000in}} %
\pgfusepath{clip}%
\pgfsetbuttcap%
\pgfsetmiterjoin%
\definecolor{currentfill}{rgb}{0.274510,0.509804,0.705882}%
\pgfsetfillcolor{currentfill}%
\pgfsetlinewidth{1.003750pt}%
\definecolor{currentstroke}{rgb}{0.274510,0.509804,0.705882}%
\pgfsetstrokecolor{currentstroke}%
\pgfsetdash{}{0pt}%
\pgfpathmoveto{\pgfqpoint{3.471987in}{5.671193in}}%
\pgfpathcurveto{\pgfqpoint{3.473289in}{5.671193in}}{\pgfqpoint{3.474537in}{5.672306in}}{\pgfqpoint{3.475458in}{5.674286in}}%
\pgfpathcurveto{\pgfqpoint{3.476378in}{5.676266in}}{\pgfqpoint{3.476896in}{5.678952in}}{\pgfqpoint{3.476896in}{5.681753in}}%
\pgfpathcurveto{\pgfqpoint{3.476896in}{5.684553in}}{\pgfqpoint{3.476378in}{5.687240in}}{\pgfqpoint{3.475458in}{5.689220in}}%
\pgfpathcurveto{\pgfqpoint{3.474537in}{5.691200in}}{\pgfqpoint{3.473289in}{5.692313in}}{\pgfqpoint{3.471987in}{5.692313in}}%
\pgfpathcurveto{\pgfqpoint{3.470685in}{5.692313in}}{\pgfqpoint{3.469436in}{5.691200in}}{\pgfqpoint{3.468515in}{5.689220in}}%
\pgfpathcurveto{\pgfqpoint{3.467595in}{5.687240in}}{\pgfqpoint{3.467078in}{5.684553in}}{\pgfqpoint{3.467078in}{5.681753in}}%
\pgfpathcurveto{\pgfqpoint{3.467078in}{5.678952in}}{\pgfqpoint{3.467595in}{5.676266in}}{\pgfqpoint{3.468515in}{5.674286in}}%
\pgfpathcurveto{\pgfqpoint{3.469436in}{5.672306in}}{\pgfqpoint{3.470685in}{5.671193in}}{\pgfqpoint{3.471987in}{5.671193in}}%
\pgfpathclose%
\pgfusepath{stroke,fill}%
\end{pgfscope}%
\begin{pgfscope}%
\pgfsetbuttcap%
\pgfsetroundjoin%
\definecolor{currentfill}{rgb}{0.000000,0.000000,0.000000}%
\pgfsetfillcolor{currentfill}%
\pgfsetlinewidth{0.803000pt}%
\definecolor{currentstroke}{rgb}{0.000000,0.000000,0.000000}%
\pgfsetstrokecolor{currentstroke}%
\pgfsetdash{}{0pt}%
\pgfsys@defobject{currentmarker}{\pgfqpoint{0.000000in}{-0.048611in}}{\pgfqpoint{0.000000in}{0.000000in}}{%
\pgfpathmoveto{\pgfqpoint{0.000000in}{0.000000in}}%
\pgfpathlineto{\pgfqpoint{0.000000in}{-0.048611in}}%
\pgfusepath{stroke,fill}%
}%
\begin{pgfscope}%
\pgfsys@transformshift{0.800000in}{0.528000in}%
\pgfsys@useobject{currentmarker}{}%
\end{pgfscope}%
\end{pgfscope}%
\begin{pgfscope}%
\pgftext[x=0.800000in,y=0.430778in,,top]{\sffamily\fontsize{10.000000}{12.000000}\selectfont −100}%
\end{pgfscope}%
\begin{pgfscope}%
\pgfsetbuttcap%
\pgfsetroundjoin%
\definecolor{currentfill}{rgb}{0.000000,0.000000,0.000000}%
\pgfsetfillcolor{currentfill}%
\pgfsetlinewidth{0.803000pt}%
\definecolor{currentstroke}{rgb}{0.000000,0.000000,0.000000}%
\pgfsetstrokecolor{currentstroke}%
\pgfsetdash{}{0pt}%
\pgfsys@defobject{currentmarker}{\pgfqpoint{0.000000in}{-0.048611in}}{\pgfqpoint{0.000000in}{0.000000in}}{%
\pgfpathmoveto{\pgfqpoint{0.000000in}{0.000000in}}%
\pgfpathlineto{\pgfqpoint{0.000000in}{-0.048611in}}%
\pgfusepath{stroke,fill}%
}%
\begin{pgfscope}%
\pgfsys@transformshift{1.290909in}{0.528000in}%
\pgfsys@useobject{currentmarker}{}%
\end{pgfscope}%
\end{pgfscope}%
\begin{pgfscope}%
\pgftext[x=1.290909in,y=0.430778in,,top]{\sffamily\fontsize{10.000000}{12.000000}\selectfont 0}%
\end{pgfscope}%
\begin{pgfscope}%
\pgfsetbuttcap%
\pgfsetroundjoin%
\definecolor{currentfill}{rgb}{0.000000,0.000000,0.000000}%
\pgfsetfillcolor{currentfill}%
\pgfsetlinewidth{0.803000pt}%
\definecolor{currentstroke}{rgb}{0.000000,0.000000,0.000000}%
\pgfsetstrokecolor{currentstroke}%
\pgfsetdash{}{0pt}%
\pgfsys@defobject{currentmarker}{\pgfqpoint{0.000000in}{-0.048611in}}{\pgfqpoint{0.000000in}{0.000000in}}{%
\pgfpathmoveto{\pgfqpoint{0.000000in}{0.000000in}}%
\pgfpathlineto{\pgfqpoint{0.000000in}{-0.048611in}}%
\pgfusepath{stroke,fill}%
}%
\begin{pgfscope}%
\pgfsys@transformshift{1.781818in}{0.528000in}%
\pgfsys@useobject{currentmarker}{}%
\end{pgfscope}%
\end{pgfscope}%
\begin{pgfscope}%
\pgftext[x=1.781818in,y=0.430778in,,top]{\sffamily\fontsize{10.000000}{12.000000}\selectfont 100}%
\end{pgfscope}%
\begin{pgfscope}%
\pgfsetbuttcap%
\pgfsetroundjoin%
\definecolor{currentfill}{rgb}{0.000000,0.000000,0.000000}%
\pgfsetfillcolor{currentfill}%
\pgfsetlinewidth{0.803000pt}%
\definecolor{currentstroke}{rgb}{0.000000,0.000000,0.000000}%
\pgfsetstrokecolor{currentstroke}%
\pgfsetdash{}{0pt}%
\pgfsys@defobject{currentmarker}{\pgfqpoint{0.000000in}{-0.048611in}}{\pgfqpoint{0.000000in}{0.000000in}}{%
\pgfpathmoveto{\pgfqpoint{0.000000in}{0.000000in}}%
\pgfpathlineto{\pgfqpoint{0.000000in}{-0.048611in}}%
\pgfusepath{stroke,fill}%
}%
\begin{pgfscope}%
\pgfsys@transformshift{2.272727in}{0.528000in}%
\pgfsys@useobject{currentmarker}{}%
\end{pgfscope}%
\end{pgfscope}%
\begin{pgfscope}%
\pgftext[x=2.272727in,y=0.430778in,,top]{\sffamily\fontsize{10.000000}{12.000000}\selectfont 200}%
\end{pgfscope}%
\begin{pgfscope}%
\pgfsetbuttcap%
\pgfsetroundjoin%
\definecolor{currentfill}{rgb}{0.000000,0.000000,0.000000}%
\pgfsetfillcolor{currentfill}%
\pgfsetlinewidth{0.803000pt}%
\definecolor{currentstroke}{rgb}{0.000000,0.000000,0.000000}%
\pgfsetstrokecolor{currentstroke}%
\pgfsetdash{}{0pt}%
\pgfsys@defobject{currentmarker}{\pgfqpoint{0.000000in}{-0.048611in}}{\pgfqpoint{0.000000in}{0.000000in}}{%
\pgfpathmoveto{\pgfqpoint{0.000000in}{0.000000in}}%
\pgfpathlineto{\pgfqpoint{0.000000in}{-0.048611in}}%
\pgfusepath{stroke,fill}%
}%
\begin{pgfscope}%
\pgfsys@transformshift{2.763636in}{0.528000in}%
\pgfsys@useobject{currentmarker}{}%
\end{pgfscope}%
\end{pgfscope}%
\begin{pgfscope}%
\pgftext[x=2.763636in,y=0.430778in,,top]{\sffamily\fontsize{10.000000}{12.000000}\selectfont 300}%
\end{pgfscope}%
\begin{pgfscope}%
\pgftext[x=1.781818in,y=0.240809in,,top]{\sffamily\fontsize{10.000000}{12.000000}\selectfont a}%
\end{pgfscope}%
\begin{pgfscope}%
\pgfsetbuttcap%
\pgfsetroundjoin%
\definecolor{currentfill}{rgb}{0.000000,0.000000,0.000000}%
\pgfsetfillcolor{currentfill}%
\pgfsetlinewidth{0.803000pt}%
\definecolor{currentstroke}{rgb}{0.000000,0.000000,0.000000}%
\pgfsetstrokecolor{currentstroke}%
\pgfsetdash{}{0pt}%
\pgfsys@defobject{currentmarker}{\pgfqpoint{-0.048611in}{0.000000in}}{\pgfqpoint{0.000000in}{0.000000in}}{%
\pgfpathmoveto{\pgfqpoint{0.000000in}{0.000000in}}%
\pgfpathlineto{\pgfqpoint{-0.048611in}{0.000000in}}%
\pgfusepath{stroke,fill}%
}%
\begin{pgfscope}%
\pgfsys@transformshift{0.800000in}{0.528000in}%
\pgfsys@useobject{currentmarker}{}%
\end{pgfscope}%
\end{pgfscope}%
\begin{pgfscope}%
\pgftext[x=0.321308in,y=0.475238in,left,base]{\sffamily\fontsize{10.000000}{12.000000}\selectfont −200}%
\end{pgfscope}%
\begin{pgfscope}%
\pgfsetbuttcap%
\pgfsetroundjoin%
\definecolor{currentfill}{rgb}{0.000000,0.000000,0.000000}%
\pgfsetfillcolor{currentfill}%
\pgfsetlinewidth{0.803000pt}%
\definecolor{currentstroke}{rgb}{0.000000,0.000000,0.000000}%
\pgfsetstrokecolor{currentstroke}%
\pgfsetdash{}{0pt}%
\pgfsys@defobject{currentmarker}{\pgfqpoint{-0.048611in}{0.000000in}}{\pgfqpoint{0.000000in}{0.000000in}}{%
\pgfpathmoveto{\pgfqpoint{0.000000in}{0.000000in}}%
\pgfpathlineto{\pgfqpoint{-0.048611in}{0.000000in}}%
\pgfusepath{stroke,fill}%
}%
\begin{pgfscope}%
\pgfsys@transformshift{0.800000in}{1.056000in}%
\pgfsys@useobject{currentmarker}{}%
\end{pgfscope}%
\end{pgfscope}%
\begin{pgfscope}%
\pgftext[x=0.321308in,y=1.003238in,left,base]{\sffamily\fontsize{10.000000}{12.000000}\selectfont −150}%
\end{pgfscope}%
\begin{pgfscope}%
\pgfsetbuttcap%
\pgfsetroundjoin%
\definecolor{currentfill}{rgb}{0.000000,0.000000,0.000000}%
\pgfsetfillcolor{currentfill}%
\pgfsetlinewidth{0.803000pt}%
\definecolor{currentstroke}{rgb}{0.000000,0.000000,0.000000}%
\pgfsetstrokecolor{currentstroke}%
\pgfsetdash{}{0pt}%
\pgfsys@defobject{currentmarker}{\pgfqpoint{-0.048611in}{0.000000in}}{\pgfqpoint{0.000000in}{0.000000in}}{%
\pgfpathmoveto{\pgfqpoint{0.000000in}{0.000000in}}%
\pgfpathlineto{\pgfqpoint{-0.048611in}{0.000000in}}%
\pgfusepath{stroke,fill}%
}%
\begin{pgfscope}%
\pgfsys@transformshift{0.800000in}{1.584000in}%
\pgfsys@useobject{currentmarker}{}%
\end{pgfscope}%
\end{pgfscope}%
\begin{pgfscope}%
\pgftext[x=0.321308in,y=1.531238in,left,base]{\sffamily\fontsize{10.000000}{12.000000}\selectfont −100}%
\end{pgfscope}%
\begin{pgfscope}%
\pgfsetbuttcap%
\pgfsetroundjoin%
\definecolor{currentfill}{rgb}{0.000000,0.000000,0.000000}%
\pgfsetfillcolor{currentfill}%
\pgfsetlinewidth{0.803000pt}%
\definecolor{currentstroke}{rgb}{0.000000,0.000000,0.000000}%
\pgfsetstrokecolor{currentstroke}%
\pgfsetdash{}{0pt}%
\pgfsys@defobject{currentmarker}{\pgfqpoint{-0.048611in}{0.000000in}}{\pgfqpoint{0.000000in}{0.000000in}}{%
\pgfpathmoveto{\pgfqpoint{0.000000in}{0.000000in}}%
\pgfpathlineto{\pgfqpoint{-0.048611in}{0.000000in}}%
\pgfusepath{stroke,fill}%
}%
\begin{pgfscope}%
\pgfsys@transformshift{0.800000in}{2.112000in}%
\pgfsys@useobject{currentmarker}{}%
\end{pgfscope}%
\end{pgfscope}%
\begin{pgfscope}%
\pgftext[x=0.409673in,y=2.059238in,left,base]{\sffamily\fontsize{10.000000}{12.000000}\selectfont −50}%
\end{pgfscope}%
\begin{pgfscope}%
\pgfsetbuttcap%
\pgfsetroundjoin%
\definecolor{currentfill}{rgb}{0.000000,0.000000,0.000000}%
\pgfsetfillcolor{currentfill}%
\pgfsetlinewidth{0.803000pt}%
\definecolor{currentstroke}{rgb}{0.000000,0.000000,0.000000}%
\pgfsetstrokecolor{currentstroke}%
\pgfsetdash{}{0pt}%
\pgfsys@defobject{currentmarker}{\pgfqpoint{-0.048611in}{0.000000in}}{\pgfqpoint{0.000000in}{0.000000in}}{%
\pgfpathmoveto{\pgfqpoint{0.000000in}{0.000000in}}%
\pgfpathlineto{\pgfqpoint{-0.048611in}{0.000000in}}%
\pgfusepath{stroke,fill}%
}%
\begin{pgfscope}%
\pgfsys@transformshift{0.800000in}{2.640000in}%
\pgfsys@useobject{currentmarker}{}%
\end{pgfscope}%
\end{pgfscope}%
\begin{pgfscope}%
\pgftext[x=0.614413in,y=2.587238in,left,base]{\sffamily\fontsize{10.000000}{12.000000}\selectfont 0}%
\end{pgfscope}%
\begin{pgfscope}%
\pgfsetbuttcap%
\pgfsetroundjoin%
\definecolor{currentfill}{rgb}{0.000000,0.000000,0.000000}%
\pgfsetfillcolor{currentfill}%
\pgfsetlinewidth{0.803000pt}%
\definecolor{currentstroke}{rgb}{0.000000,0.000000,0.000000}%
\pgfsetstrokecolor{currentstroke}%
\pgfsetdash{}{0pt}%
\pgfsys@defobject{currentmarker}{\pgfqpoint{-0.048611in}{0.000000in}}{\pgfqpoint{0.000000in}{0.000000in}}{%
\pgfpathmoveto{\pgfqpoint{0.000000in}{0.000000in}}%
\pgfpathlineto{\pgfqpoint{-0.048611in}{0.000000in}}%
\pgfusepath{stroke,fill}%
}%
\begin{pgfscope}%
\pgfsys@transformshift{0.800000in}{3.168000in}%
\pgfsys@useobject{currentmarker}{}%
\end{pgfscope}%
\end{pgfscope}%
\begin{pgfscope}%
\pgftext[x=0.526047in,y=3.115238in,left,base]{\sffamily\fontsize{10.000000}{12.000000}\selectfont 50}%
\end{pgfscope}%
\begin{pgfscope}%
\pgfsetbuttcap%
\pgfsetroundjoin%
\definecolor{currentfill}{rgb}{0.000000,0.000000,0.000000}%
\pgfsetfillcolor{currentfill}%
\pgfsetlinewidth{0.803000pt}%
\definecolor{currentstroke}{rgb}{0.000000,0.000000,0.000000}%
\pgfsetstrokecolor{currentstroke}%
\pgfsetdash{}{0pt}%
\pgfsys@defobject{currentmarker}{\pgfqpoint{-0.048611in}{0.000000in}}{\pgfqpoint{0.000000in}{0.000000in}}{%
\pgfpathmoveto{\pgfqpoint{0.000000in}{0.000000in}}%
\pgfpathlineto{\pgfqpoint{-0.048611in}{0.000000in}}%
\pgfusepath{stroke,fill}%
}%
\begin{pgfscope}%
\pgfsys@transformshift{0.800000in}{3.696000in}%
\pgfsys@useobject{currentmarker}{}%
\end{pgfscope}%
\end{pgfscope}%
\begin{pgfscope}%
\pgftext[x=0.437682in,y=3.643238in,left,base]{\sffamily\fontsize{10.000000}{12.000000}\selectfont 100}%
\end{pgfscope}%
\begin{pgfscope}%
\pgfsetbuttcap%
\pgfsetroundjoin%
\definecolor{currentfill}{rgb}{0.000000,0.000000,0.000000}%
\pgfsetfillcolor{currentfill}%
\pgfsetlinewidth{0.803000pt}%
\definecolor{currentstroke}{rgb}{0.000000,0.000000,0.000000}%
\pgfsetstrokecolor{currentstroke}%
\pgfsetdash{}{0pt}%
\pgfsys@defobject{currentmarker}{\pgfqpoint{-0.048611in}{0.000000in}}{\pgfqpoint{0.000000in}{0.000000in}}{%
\pgfpathmoveto{\pgfqpoint{0.000000in}{0.000000in}}%
\pgfpathlineto{\pgfqpoint{-0.048611in}{0.000000in}}%
\pgfusepath{stroke,fill}%
}%
\begin{pgfscope}%
\pgfsys@transformshift{0.800000in}{4.224000in}%
\pgfsys@useobject{currentmarker}{}%
\end{pgfscope}%
\end{pgfscope}%
\begin{pgfscope}%
\pgftext[x=0.437682in,y=4.171238in,left,base]{\sffamily\fontsize{10.000000}{12.000000}\selectfont 150}%
\end{pgfscope}%
\begin{pgfscope}%
\pgftext[x=0.265752in,y=2.376000in,,bottom,rotate=90.000000]{\sffamily\fontsize{10.000000}{12.000000}\selectfont b}%
\end{pgfscope}%
\begin{pgfscope}%
\pgfpathrectangle{\pgfqpoint{0.800000in}{0.528000in}}{\pgfqpoint{1.963636in}{3.696000in}} %
\pgfusepath{clip}%
\pgfsetrectcap%
\pgfsetroundjoin%
\pgfsetlinewidth{1.505625pt}%
\definecolor{currentstroke}{rgb}{1.000000,1.000000,1.000000}%
\pgfsetstrokecolor{currentstroke}%
\pgfsetdash{}{0pt}%
\pgfpathmoveto{\pgfqpoint{2.773636in}{3.056815in}}%
\pgfpathlineto{\pgfqpoint{2.715281in}{2.917550in}}%
\pgfpathlineto{\pgfqpoint{2.679210in}{2.749129in}}%
\pgfpathlineto{\pgfqpoint{2.615415in}{2.541521in}}%
\pgfpathlineto{\pgfqpoint{2.570518in}{2.380804in}}%
\pgfpathlineto{\pgfqpoint{2.559843in}{2.313359in}}%
\pgfpathlineto{\pgfqpoint{2.501674in}{2.144525in}}%
\pgfpathlineto{\pgfqpoint{2.416490in}{1.949885in}}%
\pgfpathlineto{\pgfqpoint{2.397313in}{1.878871in}}%
\pgfpathlineto{\pgfqpoint{2.372577in}{1.798373in}}%
\pgfpathlineto{\pgfqpoint{2.358973in}{1.747997in}}%
\pgfpathlineto{\pgfqpoint{2.361346in}{1.763922in}}%
\pgfpathlineto{\pgfqpoint{2.341253in}{1.688639in}}%
\pgfpathlineto{\pgfqpoint{2.345682in}{1.725230in}}%
\pgfpathlineto{\pgfqpoint{2.319540in}{1.649352in}}%
\pgfpathlineto{\pgfqpoint{2.322239in}{1.664610in}}%
\pgfpathlineto{\pgfqpoint{2.303912in}{1.598971in}}%
\pgfpathlineto{\pgfqpoint{2.290120in}{1.555408in}}%
\pgfpathlineto{\pgfqpoint{2.266692in}{1.492510in}}%
\pgfpathlineto{\pgfqpoint{2.272484in}{1.523608in}}%
\pgfpathlineto{\pgfqpoint{2.279261in}{1.572125in}}%
\pgfpathlineto{\pgfqpoint{2.270680in}{1.543330in}}%
\pgfpathlineto{\pgfqpoint{2.272566in}{1.644608in}}%
\pgfpathlineto{\pgfqpoint{2.242370in}{1.567398in}}%
\pgfpathlineto{\pgfqpoint{2.223559in}{1.515600in}}%
\pgfpathlineto{\pgfqpoint{2.225622in}{1.524120in}}%
\pgfpathlineto{\pgfqpoint{2.225622in}{1.635710in}}%
\pgfpathlineto{\pgfqpoint{2.201259in}{1.571894in}}%
\pgfpathlineto{\pgfqpoint{2.207573in}{1.639709in}}%
\pgfpathlineto{\pgfqpoint{2.203822in}{1.625817in}}%
\pgfpathlineto{\pgfqpoint{2.179226in}{1.562926in}}%
\pgfpathlineto{\pgfqpoint{2.182881in}{1.579282in}}%
\pgfpathlineto{\pgfqpoint{2.159550in}{1.521014in}}%
\pgfpathlineto{\pgfqpoint{2.164597in}{1.543596in}}%
\pgfpathlineto{\pgfqpoint{2.159020in}{1.525979in}}%
\pgfpathlineto{\pgfqpoint{2.165552in}{1.603926in}}%
\pgfpathlineto{\pgfqpoint{2.169283in}{1.621346in}}%
\pgfpathlineto{\pgfqpoint{2.134072in}{1.542567in}}%
\pgfpathlineto{\pgfqpoint{2.141789in}{1.601763in}}%
\pgfpathlineto{\pgfqpoint{2.118616in}{1.545206in}}%
\pgfpathlineto{\pgfqpoint{2.120419in}{1.551880in}}%
\pgfpathlineto{\pgfqpoint{2.126408in}{1.632278in}}%
\pgfpathlineto{\pgfqpoint{2.130924in}{1.654325in}}%
\pgfpathlineto{\pgfqpoint{2.076852in}{1.529351in}}%
\pgfpathlineto{\pgfqpoint{2.084437in}{1.603401in}}%
\pgfpathlineto{\pgfqpoint{2.065653in}{1.557554in}}%
\pgfpathlineto{\pgfqpoint{2.070983in}{1.579568in}}%
\pgfpathlineto{\pgfqpoint{2.047849in}{1.519682in}}%
\pgfpathlineto{\pgfqpoint{2.050939in}{1.629268in}}%
\pgfpathlineto{\pgfqpoint{2.055495in}{1.710822in}}%
\pgfpathlineto{\pgfqpoint{2.061351in}{1.737387in}}%
\pgfpathlineto{\pgfqpoint{2.057892in}{1.725406in}}%
\pgfpathlineto{\pgfqpoint{2.062698in}{1.751208in}}%
\pgfpathlineto{\pgfqpoint{2.048177in}{1.711223in}}%
\pgfpathlineto{\pgfqpoint{2.048347in}{1.711928in}}%
\pgfpathlineto{\pgfqpoint{2.024219in}{1.655598in}}%
\pgfpathlineto{\pgfqpoint{2.026973in}{1.666162in}}%
\pgfpathlineto{\pgfqpoint{2.033921in}{1.719458in}}%
\pgfpathlineto{\pgfqpoint{2.004828in}{1.655695in}}%
\pgfpathlineto{\pgfqpoint{2.011783in}{1.693041in}}%
\pgfpathlineto{\pgfqpoint{1.989777in}{1.642759in}}%
\pgfpathlineto{\pgfqpoint{1.990103in}{1.643820in}}%
\pgfpathlineto{\pgfqpoint{2.003436in}{1.764223in}}%
\pgfpathlineto{\pgfqpoint{2.010847in}{1.799722in}}%
\pgfpathlineto{\pgfqpoint{1.995967in}{1.757495in}}%
\pgfpathlineto{\pgfqpoint{2.001377in}{1.815595in}}%
\pgfpathlineto{\pgfqpoint{2.006574in}{1.858528in}}%
\pgfpathlineto{\pgfqpoint{2.004592in}{1.850927in}}%
\pgfpathlineto{\pgfqpoint{2.013409in}{1.937156in}}%
\pgfpathlineto{\pgfqpoint{2.009549in}{1.920944in}}%
\pgfpathlineto{\pgfqpoint{1.981693in}{1.854467in}}%
\pgfpathlineto{\pgfqpoint{1.986474in}{1.888693in}}%
\pgfpathlineto{\pgfqpoint{1.952574in}{1.812289in}}%
\pgfpathlineto{\pgfqpoint{1.951148in}{1.807348in}}%
\pgfpathlineto{\pgfqpoint{1.957355in}{1.853100in}}%
\pgfpathlineto{\pgfqpoint{1.959652in}{1.863383in}}%
\pgfpathlineto{\pgfqpoint{1.964005in}{1.915320in}}%
\pgfpathlineto{\pgfqpoint{1.923357in}{1.816033in}}%
\pgfpathlineto{\pgfqpoint{1.925432in}{1.890328in}}%
\pgfpathlineto{\pgfqpoint{1.904098in}{1.840521in}}%
\pgfpathlineto{\pgfqpoint{1.880817in}{1.788471in}}%
\pgfpathlineto{\pgfqpoint{1.882845in}{1.795496in}}%
\pgfpathlineto{\pgfqpoint{1.883608in}{1.877419in}}%
\pgfpathlineto{\pgfqpoint{1.885001in}{1.882764in}}%
\pgfpathlineto{\pgfqpoint{1.890160in}{1.915349in}}%
\pgfpathlineto{\pgfqpoint{1.872238in}{1.872581in}}%
\pgfpathlineto{\pgfqpoint{1.879292in}{1.904451in}}%
\pgfpathlineto{\pgfqpoint{1.873610in}{1.887500in}}%
\pgfpathlineto{\pgfqpoint{1.862150in}{1.858199in}}%
\pgfpathlineto{\pgfqpoint{1.866151in}{1.875386in}}%
\pgfpathlineto{\pgfqpoint{1.870985in}{1.927290in}}%
\pgfpathlineto{\pgfqpoint{1.875457in}{1.953970in}}%
\pgfpathlineto{\pgfqpoint{1.870298in}{1.937675in}}%
\pgfpathlineto{\pgfqpoint{1.874828in}{1.978220in}}%
\pgfpathlineto{\pgfqpoint{1.878004in}{2.026939in}}%
\pgfpathlineto{\pgfqpoint{1.879692in}{2.035574in}}%
\pgfpathlineto{\pgfqpoint{1.879692in}{2.096018in}}%
\pgfpathlineto{\pgfqpoint{1.866370in}{2.057352in}}%
\pgfpathlineto{\pgfqpoint{1.868802in}{2.100870in}}%
\pgfpathlineto{\pgfqpoint{1.863280in}{2.078858in}}%
\pgfpathlineto{\pgfqpoint{1.865996in}{2.098298in}}%
\pgfpathlineto{\pgfqpoint{1.858360in}{2.072273in}}%
\pgfpathlineto{\pgfqpoint{1.824006in}{1.981116in}}%
\pgfpathlineto{\pgfqpoint{1.825636in}{1.987846in}}%
\pgfpathlineto{\pgfqpoint{1.826765in}{2.048508in}}%
\pgfpathlineto{\pgfqpoint{1.810553in}{2.005447in}}%
\pgfpathlineto{\pgfqpoint{1.810907in}{2.006803in}}%
\pgfpathlineto{\pgfqpoint{1.814302in}{2.052378in}}%
\pgfpathlineto{\pgfqpoint{1.817734in}{2.075415in}}%
\pgfpathlineto{\pgfqpoint{1.820731in}{2.129487in}}%
\pgfpathlineto{\pgfqpoint{1.810550in}{2.099116in}}%
\pgfpathlineto{\pgfqpoint{1.812829in}{2.139901in}}%
\pgfpathlineto{\pgfqpoint{1.808133in}{2.123091in}}%
\pgfpathlineto{\pgfqpoint{1.809747in}{2.131198in}}%
\pgfpathlineto{\pgfqpoint{1.810613in}{2.177662in}}%
\pgfpathlineto{\pgfqpoint{1.807847in}{2.166237in}}%
\pgfpathlineto{\pgfqpoint{1.783206in}{2.109936in}}%
\pgfpathlineto{\pgfqpoint{1.783789in}{2.112543in}}%
\pgfpathlineto{\pgfqpoint{1.785441in}{2.156891in}}%
\pgfpathlineto{\pgfqpoint{1.781967in}{2.144029in}}%
\pgfpathlineto{\pgfqpoint{1.753242in}{2.071444in}}%
\pgfpathlineto{\pgfqpoint{1.754603in}{2.077069in}}%
\pgfpathlineto{\pgfqpoint{1.738584in}{2.040466in}}%
\pgfpathlineto{\pgfqpoint{1.739838in}{2.097413in}}%
\pgfpathlineto{\pgfqpoint{1.742035in}{2.107674in}}%
\pgfpathlineto{\pgfqpoint{1.731745in}{2.079037in}}%
\pgfpathlineto{\pgfqpoint{1.734249in}{2.123866in}}%
\pgfpathlineto{\pgfqpoint{1.725333in}{2.099928in}}%
\pgfpathlineto{\pgfqpoint{1.727784in}{2.111375in}}%
\pgfpathlineto{\pgfqpoint{1.731197in}{2.141918in}}%
\pgfpathlineto{\pgfqpoint{1.719867in}{2.112947in}}%
\pgfpathlineto{\pgfqpoint{1.721847in}{2.121804in}}%
\pgfpathlineto{\pgfqpoint{1.717800in}{2.109389in}}%
\pgfpathlineto{\pgfqpoint{1.720209in}{2.120636in}}%
\pgfpathlineto{\pgfqpoint{1.723562in}{2.150645in}}%
\pgfpathlineto{\pgfqpoint{1.719913in}{2.138770in}}%
\pgfpathlineto{\pgfqpoint{1.725331in}{2.171733in}}%
\pgfpathlineto{\pgfqpoint{1.725331in}{2.218561in}}%
\pgfpathlineto{\pgfqpoint{1.713460in}{2.186690in}}%
\pgfpathlineto{\pgfqpoint{1.714638in}{2.192440in}}%
\pgfpathlineto{\pgfqpoint{1.690670in}{2.132805in}}%
\pgfpathlineto{\pgfqpoint{1.693810in}{2.166528in}}%
\pgfpathlineto{\pgfqpoint{1.691571in}{2.159015in}}%
\pgfpathlineto{\pgfqpoint{1.672888in}{2.114141in}}%
\pgfpathlineto{\pgfqpoint{1.679232in}{2.149813in}}%
\pgfpathlineto{\pgfqpoint{1.674368in}{2.136067in}}%
\pgfpathlineto{\pgfqpoint{1.675566in}{2.187902in}}%
\pgfpathlineto{\pgfqpoint{1.677611in}{2.197224in}}%
\pgfpathlineto{\pgfqpoint{1.677611in}{2.197224in}}%
\pgfpathlineto{\pgfqpoint{1.677611in}{2.197224in}}%
\pgfpathlineto{\pgfqpoint{1.674538in}{2.187225in}}%
\pgfpathlineto{\pgfqpoint{1.676826in}{2.197277in}}%
\pgfpathlineto{\pgfqpoint{1.677916in}{2.202151in}}%
\pgfpathlineto{\pgfqpoint{1.679001in}{2.241007in}}%
\pgfpathlineto{\pgfqpoint{1.680877in}{2.252857in}}%
\pgfpathlineto{\pgfqpoint{1.682644in}{2.279963in}}%
\pgfpathlineto{\pgfqpoint{1.662981in}{2.235970in}}%
\pgfpathlineto{\pgfqpoint{1.665232in}{2.251081in}}%
\pgfpathlineto{\pgfqpoint{1.639848in}{2.180074in}}%
\pgfpathlineto{\pgfqpoint{1.642756in}{2.194945in}}%
\pgfpathlineto{\pgfqpoint{1.643690in}{2.243632in}}%
\pgfpathlineto{\pgfqpoint{1.628273in}{2.209841in}}%
\pgfpathlineto{\pgfqpoint{1.631256in}{2.234360in}}%
\pgfpathlineto{\pgfqpoint{1.627412in}{2.222893in}}%
\pgfpathlineto{\pgfqpoint{1.619519in}{2.203180in}}%
\pgfpathlineto{\pgfqpoint{1.622614in}{2.228744in}}%
\pgfpathlineto{\pgfqpoint{1.612493in}{2.205116in}}%
\pgfpathlineto{\pgfqpoint{1.613593in}{2.244521in}}%
\pgfpathlineto{\pgfqpoint{1.616252in}{2.264918in}}%
\pgfpathlineto{\pgfqpoint{1.610320in}{2.248583in}}%
\pgfpathlineto{\pgfqpoint{1.614679in}{2.288808in}}%
\pgfpathlineto{\pgfqpoint{1.615276in}{2.352926in}}%
\pgfpathlineto{\pgfqpoint{1.617653in}{2.377883in}}%
\pgfpathlineto{\pgfqpoint{1.617884in}{2.402710in}}%
\pgfpathlineto{\pgfqpoint{1.609695in}{2.372917in}}%
\pgfpathlineto{\pgfqpoint{1.611915in}{2.397005in}}%
\pgfpathlineto{\pgfqpoint{1.612878in}{2.411776in}}%
\pgfpathlineto{\pgfqpoint{1.592943in}{2.365319in}}%
\pgfpathlineto{\pgfqpoint{1.582661in}{2.339028in}}%
\pgfpathlineto{\pgfqpoint{1.586833in}{2.407959in}}%
\pgfpathlineto{\pgfqpoint{1.585293in}{2.401071in}}%
\pgfpathlineto{\pgfqpoint{1.573676in}{2.365190in}}%
\pgfpathlineto{\pgfqpoint{1.574922in}{2.384296in}}%
\pgfpathlineto{\pgfqpoint{1.573878in}{2.379814in}}%
\pgfpathlineto{\pgfqpoint{1.569225in}{2.365115in}}%
\pgfpathlineto{\pgfqpoint{1.558532in}{2.339598in}}%
\pgfpathlineto{\pgfqpoint{1.559483in}{2.365122in}}%
\pgfpathlineto{\pgfqpoint{1.558011in}{2.359672in}}%
\pgfpathlineto{\pgfqpoint{1.561076in}{2.396084in}}%
\pgfpathlineto{\pgfqpoint{1.559793in}{2.392154in}}%
\pgfpathlineto{\pgfqpoint{1.560212in}{2.414676in}}%
\pgfpathlineto{\pgfqpoint{1.546647in}{2.383680in}}%
\pgfpathlineto{\pgfqpoint{1.547981in}{2.399603in}}%
\pgfpathlineto{\pgfqpoint{1.544354in}{2.387798in}}%
\pgfpathlineto{\pgfqpoint{1.545731in}{2.401243in}}%
\pgfpathlineto{\pgfqpoint{1.538585in}{2.382058in}}%
\pgfpathlineto{\pgfqpoint{1.539993in}{2.416530in}}%
\pgfpathlineto{\pgfqpoint{1.540410in}{2.418890in}}%
\pgfpathlineto{\pgfqpoint{1.531124in}{2.395697in}}%
\pgfpathlineto{\pgfqpoint{1.531739in}{2.398844in}}%
\pgfpathlineto{\pgfqpoint{1.533409in}{2.420819in}}%
\pgfpathlineto{\pgfqpoint{1.526871in}{2.402816in}}%
\pgfpathlineto{\pgfqpoint{1.527862in}{2.420564in}}%
\pgfpathlineto{\pgfqpoint{1.523613in}{2.407528in}}%
\pgfpathlineto{\pgfqpoint{1.524681in}{2.423910in}}%
\pgfpathlineto{\pgfqpoint{1.524453in}{2.422844in}}%
\pgfpathlineto{\pgfqpoint{1.509253in}{2.380776in}}%
\pgfpathlineto{\pgfqpoint{1.499199in}{2.358058in}}%
\pgfpathlineto{\pgfqpoint{1.501426in}{2.421155in}}%
\pgfpathlineto{\pgfqpoint{1.492585in}{2.397309in}}%
\pgfpathlineto{\pgfqpoint{1.494116in}{2.412256in}}%
\pgfpathlineto{\pgfqpoint{1.493577in}{2.410260in}}%
\pgfpathlineto{\pgfqpoint{1.478002in}{2.373648in}}%
\pgfpathlineto{\pgfqpoint{1.482874in}{2.428047in}}%
\pgfpathlineto{\pgfqpoint{1.482736in}{2.427494in}}%
\pgfpathlineto{\pgfqpoint{1.480479in}{2.420570in}}%
\pgfpathlineto{\pgfqpoint{1.480786in}{2.421839in}}%
\pgfpathlineto{\pgfqpoint{1.482178in}{2.432515in}}%
\pgfpathlineto{\pgfqpoint{1.475004in}{2.412853in}}%
\pgfpathlineto{\pgfqpoint{1.476975in}{2.443036in}}%
\pgfpathlineto{\pgfqpoint{1.475679in}{2.438548in}}%
\pgfpathlineto{\pgfqpoint{1.476437in}{2.442426in}}%
\pgfpathlineto{\pgfqpoint{1.476437in}{2.442426in}}%
\pgfpathlineto{\pgfqpoint{1.476679in}{2.443508in}}%
\pgfpathlineto{\pgfqpoint{1.478806in}{2.470496in}}%
\pgfpathlineto{\pgfqpoint{1.471543in}{2.450841in}}%
\pgfpathlineto{\pgfqpoint{1.473430in}{2.475244in}}%
\pgfpathlineto{\pgfqpoint{1.468005in}{2.459605in}}%
\pgfpathlineto{\pgfqpoint{1.455501in}{2.430684in}}%
\pgfpathlineto{\pgfqpoint{1.456771in}{2.437175in}}%
\pgfpathlineto{\pgfqpoint{1.450529in}{2.423495in}}%
\pgfpathlineto{\pgfqpoint{1.453776in}{2.456373in}}%
\pgfpathlineto{\pgfqpoint{1.454232in}{2.472704in}}%
\pgfpathlineto{\pgfqpoint{1.452187in}{2.466244in}}%
\pgfpathlineto{\pgfqpoint{1.453203in}{2.473515in}}%
\pgfpathlineto{\pgfqpoint{1.452390in}{2.470506in}}%
\pgfpathlineto{\pgfqpoint{1.448037in}{2.459702in}}%
\pgfpathlineto{\pgfqpoint{1.444146in}{2.447778in}}%
\pgfpathlineto{\pgfqpoint{1.444646in}{2.451072in}}%
\pgfpathlineto{\pgfqpoint{1.445975in}{2.481081in}}%
\pgfpathlineto{\pgfqpoint{1.440026in}{2.466396in}}%
\pgfpathlineto{\pgfqpoint{1.441640in}{2.509120in}}%
\pgfpathlineto{\pgfqpoint{1.440668in}{2.505257in}}%
\pgfpathlineto{\pgfqpoint{1.436246in}{2.491251in}}%
\pgfpathlineto{\pgfqpoint{1.437509in}{2.523668in}}%
\pgfpathlineto{\pgfqpoint{1.435645in}{2.516997in}}%
\pgfpathlineto{\pgfqpoint{1.435645in}{2.529299in}}%
\pgfpathlineto{\pgfqpoint{1.435200in}{2.527123in}}%
\pgfpathlineto{\pgfqpoint{1.431981in}{2.516022in}}%
\pgfpathlineto{\pgfqpoint{1.423543in}{2.494046in}}%
\pgfpathlineto{\pgfqpoint{1.424311in}{2.498066in}}%
\pgfpathlineto{\pgfqpoint{1.425156in}{2.505632in}}%
\pgfpathlineto{\pgfqpoint{1.420281in}{2.493997in}}%
\pgfpathlineto{\pgfqpoint{1.421467in}{2.503554in}}%
\pgfpathlineto{\pgfqpoint{1.415954in}{2.491219in}}%
\pgfpathlineto{\pgfqpoint{1.416919in}{2.497316in}}%
\pgfpathlineto{\pgfqpoint{1.413360in}{2.488630in}}%
\pgfpathlineto{\pgfqpoint{1.414479in}{2.512633in}}%
\pgfpathlineto{\pgfqpoint{1.411067in}{2.503907in}}%
\pgfpathlineto{\pgfqpoint{1.412599in}{2.523651in}}%
\pgfpathlineto{\pgfqpoint{1.410655in}{2.517688in}}%
\pgfpathlineto{\pgfqpoint{1.412312in}{2.537543in}}%
\pgfpathlineto{\pgfqpoint{1.406169in}{2.521774in}}%
\pgfpathlineto{\pgfqpoint{1.406764in}{2.525532in}}%
\pgfpathlineto{\pgfqpoint{1.406764in}{2.525532in}}%
\pgfpathlineto{\pgfqpoint{1.406764in}{2.525532in}}%
\pgfpathlineto{\pgfqpoint{1.405046in}{2.520261in}}%
\pgfpathlineto{\pgfqpoint{1.406237in}{2.528792in}}%
\pgfpathlineto{\pgfqpoint{1.397424in}{2.506430in}}%
\pgfpathlineto{\pgfqpoint{1.397703in}{2.518082in}}%
\pgfpathlineto{\pgfqpoint{1.393620in}{2.507388in}}%
\pgfpathlineto{\pgfqpoint{1.394015in}{2.528200in}}%
\pgfpathlineto{\pgfqpoint{1.391442in}{2.521621in}}%
\pgfpathlineto{\pgfqpoint{1.392940in}{2.544673in}}%
\pgfpathlineto{\pgfqpoint{1.391951in}{2.541245in}}%
\pgfpathlineto{\pgfqpoint{1.392039in}{2.550708in}}%
\pgfpathlineto{\pgfqpoint{1.388481in}{2.541823in}}%
\pgfpathlineto{\pgfqpoint{1.389542in}{2.551819in}}%
\pgfpathlineto{\pgfqpoint{1.387026in}{2.544891in}}%
\pgfpathlineto{\pgfqpoint{1.387563in}{2.549702in}}%
\pgfpathlineto{\pgfqpoint{1.381845in}{2.538307in}}%
\pgfpathlineto{\pgfqpoint{1.382198in}{2.539956in}}%
\pgfpathlineto{\pgfqpoint{1.383245in}{2.555434in}}%
\pgfpathlineto{\pgfqpoint{1.384733in}{2.568876in}}%
\pgfpathlineto{\pgfqpoint{1.375598in}{2.544783in}}%
\pgfpathlineto{\pgfqpoint{1.375684in}{2.553963in}}%
\pgfpathlineto{\pgfqpoint{1.374709in}{2.550885in}}%
\pgfpathlineto{\pgfqpoint{1.372036in}{2.544506in}}%
\pgfpathlineto{\pgfqpoint{1.372509in}{2.551763in}}%
\pgfpathlineto{\pgfqpoint{1.370169in}{2.546051in}}%
\pgfpathlineto{\pgfqpoint{1.370495in}{2.547449in}}%
\pgfpathlineto{\pgfqpoint{1.371422in}{2.566734in}}%
\pgfpathlineto{\pgfqpoint{1.369302in}{2.561042in}}%
\pgfpathlineto{\pgfqpoint{1.369896in}{2.579419in}}%
\pgfpathlineto{\pgfqpoint{1.366554in}{2.569779in}}%
\pgfpathlineto{\pgfqpoint{1.366897in}{2.574373in}}%
\pgfpathlineto{\pgfqpoint{1.366273in}{2.572062in}}%
\pgfpathlineto{\pgfqpoint{1.360038in}{2.560733in}}%
\pgfpathlineto{\pgfqpoint{1.360723in}{2.570643in}}%
\pgfpathlineto{\pgfqpoint{1.357827in}{2.564163in}}%
\pgfpathlineto{\pgfqpoint{1.358178in}{2.565878in}}%
\pgfpathlineto{\pgfqpoint{1.358648in}{2.570081in}}%
\pgfpathlineto{\pgfqpoint{1.358054in}{2.568150in}}%
\pgfpathlineto{\pgfqpoint{1.353733in}{2.557790in}}%
\pgfpathlineto{\pgfqpoint{1.354197in}{2.564025in}}%
\pgfpathlineto{\pgfqpoint{1.349832in}{2.553726in}}%
\pgfpathlineto{\pgfqpoint{1.350535in}{2.557006in}}%
\pgfpathlineto{\pgfqpoint{1.351212in}{2.561553in}}%
\pgfpathlineto{\pgfqpoint{1.350682in}{2.560093in}}%
\pgfpathlineto{\pgfqpoint{1.351480in}{2.574863in}}%
\pgfpathlineto{\pgfqpoint{1.350461in}{2.572287in}}%
\pgfpathlineto{\pgfqpoint{1.346275in}{2.562453in}}%
\pgfpathlineto{\pgfqpoint{1.346457in}{2.563044in}}%
\pgfpathlineto{\pgfqpoint{1.348432in}{2.597326in}}%
\pgfpathlineto{\pgfqpoint{1.346329in}{2.590791in}}%
\pgfpathlineto{\pgfqpoint{1.346602in}{2.594960in}}%
\pgfpathlineto{\pgfqpoint{1.343273in}{2.586790in}}%
\pgfpathlineto{\pgfqpoint{1.343249in}{2.601498in}}%
\pgfpathlineto{\pgfqpoint{1.343208in}{2.601276in}}%
\pgfpathlineto{\pgfqpoint{1.338523in}{2.590052in}}%
\pgfpathlineto{\pgfqpoint{1.338594in}{2.590855in}}%
\pgfpathlineto{\pgfqpoint{1.338679in}{2.595413in}}%
\pgfpathlineto{\pgfqpoint{1.337684in}{2.592673in}}%
\pgfpathlineto{\pgfqpoint{1.336849in}{2.590375in}}%
\pgfpathlineto{\pgfqpoint{1.336893in}{2.595162in}}%
\pgfpathlineto{\pgfqpoint{1.333620in}{2.588622in}}%
\pgfpathlineto{\pgfqpoint{1.333883in}{2.604441in}}%
\pgfpathlineto{\pgfqpoint{1.332182in}{2.600470in}}%
\pgfpathlineto{\pgfqpoint{1.332538in}{2.609172in}}%
\pgfpathlineto{\pgfqpoint{1.331321in}{2.605211in}}%
\pgfpathlineto{\pgfqpoint{1.327956in}{2.598005in}}%
\pgfpathlineto{\pgfqpoint{1.327810in}{2.610111in}}%
\pgfpathlineto{\pgfqpoint{1.327540in}{2.609110in}}%
\pgfpathlineto{\pgfqpoint{1.324546in}{2.602008in}}%
\pgfpathlineto{\pgfqpoint{1.324842in}{2.605353in}}%
\pgfpathlineto{\pgfqpoint{1.325163in}{2.610630in}}%
\pgfpathlineto{\pgfqpoint{1.324570in}{2.608908in}}%
\pgfpathlineto{\pgfqpoint{1.320779in}{2.599837in}}%
\pgfpathlineto{\pgfqpoint{1.321231in}{2.603535in}}%
\pgfpathlineto{\pgfqpoint{1.321354in}{2.607980in}}%
\pgfpathlineto{\pgfqpoint{1.321566in}{2.609878in}}%
\pgfpathlineto{\pgfqpoint{1.321369in}{2.609239in}}%
\pgfpathlineto{\pgfqpoint{1.321369in}{2.609239in}}%
\pgfpathlineto{\pgfqpoint{1.320525in}{2.607269in}}%
\pgfpathlineto{\pgfqpoint{1.320525in}{2.607269in}}%
\pgfpathlineto{\pgfqpoint{1.320525in}{2.607269in}}%
\pgfpathlineto{\pgfqpoint{1.321111in}{2.614318in}}%
\pgfpathlineto{\pgfqpoint{1.319977in}{2.611084in}}%
\pgfpathlineto{\pgfqpoint{1.317747in}{2.605713in}}%
\pgfpathlineto{\pgfqpoint{1.317959in}{2.608882in}}%
\pgfpathlineto{\pgfqpoint{1.318039in}{2.611718in}}%
\pgfpathlineto{\pgfqpoint{1.317105in}{2.609397in}}%
\pgfpathlineto{\pgfqpoint{1.316509in}{2.608062in}}%
\pgfpathlineto{\pgfqpoint{1.316509in}{2.608062in}}%
\pgfpathlineto{\pgfqpoint{1.316509in}{2.608062in}}%
\pgfpathlineto{\pgfqpoint{1.316775in}{2.609849in}}%
\pgfpathlineto{\pgfqpoint{1.316107in}{2.608284in}}%
\pgfpathlineto{\pgfqpoint{1.316107in}{2.608284in}}%
\pgfpathlineto{\pgfqpoint{1.315832in}{2.607613in}}%
\pgfpathlineto{\pgfqpoint{1.316017in}{2.608276in}}%
\pgfpathlineto{\pgfqpoint{1.316017in}{2.608276in}}%
\pgfpathlineto{\pgfqpoint{1.316666in}{2.615191in}}%
\pgfpathlineto{\pgfqpoint{1.315991in}{2.614672in}}%
\pgfpathlineto{\pgfqpoint{1.316252in}{2.620542in}}%
\pgfpathlineto{\pgfqpoint{1.315081in}{2.618036in}}%
\pgfpathlineto{\pgfqpoint{1.314705in}{2.616999in}}%
\pgfpathlineto{\pgfqpoint{1.314705in}{2.616999in}}%
\pgfpathlineto{\pgfqpoint{1.314705in}{2.616999in}}%
\pgfpathlineto{\pgfqpoint{1.314820in}{2.619752in}}%
\pgfpathlineto{\pgfqpoint{1.313696in}{2.617140in}}%
\pgfpathlineto{\pgfqpoint{1.313473in}{2.616512in}}%
\pgfpathlineto{\pgfqpoint{1.313473in}{2.616512in}}%
\pgfpathlineto{\pgfqpoint{1.313473in}{2.616512in}}%
\pgfpathlineto{\pgfqpoint{1.313634in}{2.617418in}}%
\pgfpathlineto{\pgfqpoint{1.313634in}{2.617418in}}%
\pgfpathlineto{\pgfqpoint{1.313634in}{2.617418in}}%
\pgfpathlineto{\pgfqpoint{1.313049in}{2.616083in}}%
\pgfpathlineto{\pgfqpoint{1.313049in}{2.616083in}}%
\pgfpathlineto{\pgfqpoint{1.313049in}{2.616083in}}%
\pgfpathlineto{\pgfqpoint{1.312714in}{2.621454in}}%
\pgfpathlineto{\pgfqpoint{1.312208in}{2.620192in}}%
\pgfpathlineto{\pgfqpoint{1.312300in}{2.620709in}}%
\pgfpathlineto{\pgfqpoint{1.312300in}{2.620709in}}%
\pgfpathlineto{\pgfqpoint{1.312421in}{2.621575in}}%
\pgfpathlineto{\pgfqpoint{1.312421in}{2.621575in}}%
\pgfpathlineto{\pgfqpoint{1.312421in}{2.621575in}}%
\pgfpathlineto{\pgfqpoint{1.311387in}{2.619377in}}%
\pgfpathlineto{\pgfqpoint{1.311559in}{2.621651in}}%
\pgfpathlineto{\pgfqpoint{1.311559in}{2.621651in}}%
\pgfpathlineto{\pgfqpoint{1.311673in}{2.622420in}}%
\pgfpathlineto{\pgfqpoint{1.311673in}{2.622420in}}%
\pgfpathlineto{\pgfqpoint{1.311673in}{2.622420in}}%
\pgfpathlineto{\pgfqpoint{1.310301in}{2.619823in}}%
\pgfpathlineto{\pgfqpoint{1.310474in}{2.622388in}}%
\pgfpathlineto{\pgfqpoint{1.309979in}{2.625524in}}%
\pgfpathlineto{\pgfqpoint{1.309642in}{2.624596in}}%
\pgfpathlineto{\pgfqpoint{1.308800in}{2.622572in}}%
\pgfpathlineto{\pgfqpoint{1.308950in}{2.624252in}}%
\pgfpathlineto{\pgfqpoint{1.308950in}{2.624252in}}%
\pgfpathlineto{\pgfqpoint{1.309041in}{2.625336in}}%
\pgfpathlineto{\pgfqpoint{1.309041in}{2.625336in}}%
\pgfusepath{stroke}%
\end{pgfscope}%
\begin{pgfscope}%
\pgfsetrectcap%
\pgfsetmiterjoin%
\pgfsetlinewidth{0.803000pt}%
\definecolor{currentstroke}{rgb}{0.000000,0.000000,0.000000}%
\pgfsetstrokecolor{currentstroke}%
\pgfsetdash{}{0pt}%
\pgfpathmoveto{\pgfqpoint{0.800000in}{0.528000in}}%
\pgfpathlineto{\pgfqpoint{0.800000in}{4.224000in}}%
\pgfusepath{stroke}%
\end{pgfscope}%
\begin{pgfscope}%
\pgfsetrectcap%
\pgfsetmiterjoin%
\pgfsetlinewidth{0.803000pt}%
\definecolor{currentstroke}{rgb}{0.000000,0.000000,0.000000}%
\pgfsetstrokecolor{currentstroke}%
\pgfsetdash{}{0pt}%
\pgfpathmoveto{\pgfqpoint{2.763636in}{0.528000in}}%
\pgfpathlineto{\pgfqpoint{2.763636in}{4.224000in}}%
\pgfusepath{stroke}%
\end{pgfscope}%
\begin{pgfscope}%
\pgfsetrectcap%
\pgfsetmiterjoin%
\pgfsetlinewidth{0.803000pt}%
\definecolor{currentstroke}{rgb}{0.000000,0.000000,0.000000}%
\pgfsetstrokecolor{currentstroke}%
\pgfsetdash{}{0pt}%
\pgfpathmoveto{\pgfqpoint{0.800000in}{0.528000in}}%
\pgfpathlineto{\pgfqpoint{2.763636in}{0.528000in}}%
\pgfusepath{stroke}%
\end{pgfscope}%
\begin{pgfscope}%
\pgfsetrectcap%
\pgfsetmiterjoin%
\pgfsetlinewidth{0.803000pt}%
\definecolor{currentstroke}{rgb}{0.000000,0.000000,0.000000}%
\pgfsetstrokecolor{currentstroke}%
\pgfsetdash{}{0pt}%
\pgfpathmoveto{\pgfqpoint{0.800000in}{4.224000in}}%
\pgfpathlineto{\pgfqpoint{2.763636in}{4.224000in}}%
\pgfusepath{stroke}%
\end{pgfscope}%
\begin{pgfscope}%
\pgfsetbuttcap%
\pgfsetmiterjoin%
\definecolor{currentfill}{rgb}{1.000000,1.000000,1.000000}%
\pgfsetfillcolor{currentfill}%
\pgfsetlinewidth{0.000000pt}%
\definecolor{currentstroke}{rgb}{0.000000,0.000000,0.000000}%
\pgfsetstrokecolor{currentstroke}%
\pgfsetstrokeopacity{0.000000}%
\pgfsetdash{}{0pt}%
\pgfpathmoveto{\pgfqpoint{3.156364in}{0.528000in}}%
\pgfpathlineto{\pgfqpoint{5.120000in}{0.528000in}}%
\pgfpathlineto{\pgfqpoint{5.120000in}{4.224000in}}%
\pgfpathlineto{\pgfqpoint{3.156364in}{4.224000in}}%
\pgfpathclose%
\pgfusepath{fill}%
\end{pgfscope}%
\begin{pgfscope}%
\pgfpathrectangle{\pgfqpoint{3.156364in}{0.528000in}}{\pgfqpoint{1.963636in}{3.696000in}} %
\pgfusepath{clip}%
\pgfsetbuttcap%
\pgfsetroundjoin%
\definecolor{currentfill}{rgb}{0.050383,0.029803,0.527975}%
\pgfsetfillcolor{currentfill}%
\pgfsetlinewidth{0.000000pt}%
\definecolor{currentstroke}{rgb}{0.000000,0.000000,0.000000}%
\pgfsetstrokecolor{currentstroke}%
\pgfsetdash{}{0pt}%
\pgfpathmoveto{\pgfqpoint{3.672066in}{2.586655in}}%
\pgfpathlineto{\pgfqpoint{3.672072in}{2.586667in}}%
\pgfpathlineto{\pgfqpoint{3.672066in}{2.586679in}}%
\pgfpathlineto{\pgfqpoint{3.672061in}{2.586667in}}%
\pgfpathclose%
\pgfusepath{fill}%
\end{pgfscope}%
\begin{pgfscope}%
\pgfpathrectangle{\pgfqpoint{3.156364in}{0.528000in}}{\pgfqpoint{1.963636in}{3.696000in}} %
\pgfusepath{clip}%
\pgfsetbuttcap%
\pgfsetroundjoin%
\definecolor{currentfill}{rgb}{0.050383,0.029803,0.527975}%
\pgfsetfillcolor{currentfill}%
\pgfsetlinewidth{0.000000pt}%
\definecolor{currentstroke}{rgb}{0.000000,0.000000,0.000000}%
\pgfsetstrokecolor{currentstroke}%
\pgfsetdash{}{0pt}%
\pgfpathmoveto{\pgfqpoint{3.622479in}{2.693321in}}%
\pgfpathlineto{\pgfqpoint{3.622485in}{2.693333in}}%
\pgfpathlineto{\pgfqpoint{3.622479in}{2.693345in}}%
\pgfpathlineto{\pgfqpoint{3.622474in}{2.693333in}}%
\pgfpathclose%
\pgfusepath{fill}%
\end{pgfscope}%
\begin{pgfscope}%
\pgfpathrectangle{\pgfqpoint{3.156364in}{0.528000in}}{\pgfqpoint{1.963636in}{3.696000in}} %
\pgfusepath{clip}%
\pgfsetbuttcap%
\pgfsetroundjoin%
\definecolor{currentfill}{rgb}{0.050383,0.029803,0.527975}%
\pgfsetfillcolor{currentfill}%
\pgfsetlinewidth{0.000000pt}%
\definecolor{currentstroke}{rgb}{0.000000,0.000000,0.000000}%
\pgfsetstrokecolor{currentstroke}%
\pgfsetdash{}{0pt}%
\pgfpathmoveto{\pgfqpoint{3.721653in}{2.479880in}}%
\pgfpathlineto{\pgfqpoint{3.721718in}{2.480000in}}%
\pgfpathlineto{\pgfqpoint{3.721653in}{2.480134in}}%
\pgfpathlineto{\pgfqpoint{3.721595in}{2.480000in}}%
\pgfpathclose%
\pgfusepath{fill}%
\end{pgfscope}%
\begin{pgfscope}%
\pgfpathrectangle{\pgfqpoint{3.156364in}{0.528000in}}{\pgfqpoint{1.963636in}{3.696000in}} %
\pgfusepath{clip}%
\pgfsetbuttcap%
\pgfsetroundjoin%
\definecolor{currentfill}{rgb}{0.050383,0.029803,0.527975}%
\pgfsetfillcolor{currentfill}%
\pgfsetlinewidth{0.000000pt}%
\definecolor{currentstroke}{rgb}{0.000000,0.000000,0.000000}%
\pgfsetstrokecolor{currentstroke}%
\pgfsetdash{}{0pt}%
\pgfpathmoveto{\pgfqpoint{3.672066in}{2.586470in}}%
\pgfpathlineto{\pgfqpoint{3.672164in}{2.586667in}}%
\pgfpathlineto{\pgfqpoint{3.672066in}{2.586870in}}%
\pgfpathlineto{\pgfqpoint{3.671971in}{2.586667in}}%
\pgfpathclose%
\pgfpathmoveto{\pgfqpoint{3.672061in}{2.586667in}}%
\pgfpathlineto{\pgfqpoint{3.672066in}{2.586679in}}%
\pgfpathlineto{\pgfqpoint{3.672072in}{2.586667in}}%
\pgfpathlineto{\pgfqpoint{3.672066in}{2.586655in}}%
\pgfpathclose%
\pgfusepath{fill}%
\end{pgfscope}%
\begin{pgfscope}%
\pgfpathrectangle{\pgfqpoint{3.156364in}{0.528000in}}{\pgfqpoint{1.963636in}{3.696000in}} %
\pgfusepath{clip}%
\pgfsetbuttcap%
\pgfsetroundjoin%
\definecolor{currentfill}{rgb}{0.050383,0.029803,0.527975}%
\pgfsetfillcolor{currentfill}%
\pgfsetlinewidth{0.000000pt}%
\definecolor{currentstroke}{rgb}{0.000000,0.000000,0.000000}%
\pgfsetstrokecolor{currentstroke}%
\pgfsetdash{}{0pt}%
\pgfpathmoveto{\pgfqpoint{3.622479in}{2.693130in}}%
\pgfpathlineto{\pgfqpoint{3.622574in}{2.693333in}}%
\pgfpathlineto{\pgfqpoint{3.622479in}{2.693530in}}%
\pgfpathlineto{\pgfqpoint{3.622381in}{2.693333in}}%
\pgfpathclose%
\pgfpathmoveto{\pgfqpoint{3.622474in}{2.693333in}}%
\pgfpathlineto{\pgfqpoint{3.622479in}{2.693345in}}%
\pgfpathlineto{\pgfqpoint{3.622485in}{2.693333in}}%
\pgfpathlineto{\pgfqpoint{3.622479in}{2.693321in}}%
\pgfpathclose%
\pgfusepath{fill}%
\end{pgfscope}%
\begin{pgfscope}%
\pgfpathrectangle{\pgfqpoint{3.156364in}{0.528000in}}{\pgfqpoint{1.963636in}{3.696000in}} %
\pgfusepath{clip}%
\pgfsetbuttcap%
\pgfsetroundjoin%
\definecolor{currentfill}{rgb}{0.050383,0.029803,0.527975}%
\pgfsetfillcolor{currentfill}%
\pgfsetlinewidth{0.000000pt}%
\definecolor{currentstroke}{rgb}{0.000000,0.000000,0.000000}%
\pgfsetstrokecolor{currentstroke}%
\pgfsetdash{}{0pt}%
\pgfpathmoveto{\pgfqpoint{3.572893in}{2.799866in}}%
\pgfpathlineto{\pgfqpoint{3.572951in}{2.800000in}}%
\pgfpathlineto{\pgfqpoint{3.572893in}{2.800120in}}%
\pgfpathlineto{\pgfqpoint{3.572828in}{2.800000in}}%
\pgfpathclose%
\pgfusepath{fill}%
\end{pgfscope}%
\begin{pgfscope}%
\pgfpathrectangle{\pgfqpoint{3.156364in}{0.528000in}}{\pgfqpoint{1.963636in}{3.696000in}} %
\pgfusepath{clip}%
\pgfsetbuttcap%
\pgfsetroundjoin%
\definecolor{currentfill}{rgb}{0.050383,0.029803,0.527975}%
\pgfsetfillcolor{currentfill}%
\pgfsetlinewidth{0.000000pt}%
\definecolor{currentstroke}{rgb}{0.000000,0.000000,0.000000}%
\pgfsetstrokecolor{currentstroke}%
\pgfsetdash{}{0pt}%
\pgfpathmoveto{\pgfqpoint{6.101818in}{-2.534940in}}%
\pgfpathlineto{\pgfqpoint{6.101818in}{-2.533333in}}%
\pgfpathlineto{\pgfqpoint{6.101818in}{-2.532190in}}%
\pgfpathlineto{\pgfqpoint{6.101040in}{-2.533333in}}%
\pgfpathclose%
\pgfusepath{fill}%
\end{pgfscope}%
\begin{pgfscope}%
\pgfpathrectangle{\pgfqpoint{3.156364in}{0.528000in}}{\pgfqpoint{1.963636in}{3.696000in}} %
\pgfusepath{clip}%
\pgfsetbuttcap%
\pgfsetroundjoin%
\definecolor{currentfill}{rgb}{0.050383,0.029803,0.527975}%
\pgfsetfillcolor{currentfill}%
\pgfsetlinewidth{0.000000pt}%
\definecolor{currentstroke}{rgb}{0.000000,0.000000,0.000000}%
\pgfsetstrokecolor{currentstroke}%
\pgfsetdash{}{0pt}%
\pgfpathmoveto{\pgfqpoint{6.052231in}{-2.427884in}}%
\pgfpathlineto{\pgfqpoint{6.052618in}{-2.426667in}}%
\pgfpathlineto{\pgfqpoint{6.052231in}{-2.425864in}}%
\pgfpathlineto{\pgfqpoint{6.051641in}{-2.426667in}}%
\pgfpathclose%
\pgfusepath{fill}%
\end{pgfscope}%
\begin{pgfscope}%
\pgfpathrectangle{\pgfqpoint{3.156364in}{0.528000in}}{\pgfqpoint{1.963636in}{3.696000in}} %
\pgfusepath{clip}%
\pgfsetbuttcap%
\pgfsetroundjoin%
\definecolor{currentfill}{rgb}{0.050383,0.029803,0.527975}%
\pgfsetfillcolor{currentfill}%
\pgfsetlinewidth{0.000000pt}%
\definecolor{currentstroke}{rgb}{0.000000,0.000000,0.000000}%
\pgfsetstrokecolor{currentstroke}%
\pgfsetdash{}{0pt}%
\pgfpathmoveto{\pgfqpoint{6.002645in}{-2.320694in}}%
\pgfpathlineto{\pgfqpoint{6.002848in}{-2.320000in}}%
\pgfpathlineto{\pgfqpoint{6.002645in}{-2.319577in}}%
\pgfpathlineto{\pgfqpoint{6.002308in}{-2.320000in}}%
\pgfpathclose%
\pgfusepath{fill}%
\end{pgfscope}%
\begin{pgfscope}%
\pgfpathrectangle{\pgfqpoint{3.156364in}{0.528000in}}{\pgfqpoint{1.963636in}{3.696000in}} %
\pgfusepath{clip}%
\pgfsetbuttcap%
\pgfsetroundjoin%
\definecolor{currentfill}{rgb}{0.050383,0.029803,0.527975}%
\pgfsetfillcolor{currentfill}%
\pgfsetlinewidth{0.000000pt}%
\definecolor{currentstroke}{rgb}{0.000000,0.000000,0.000000}%
\pgfsetstrokecolor{currentstroke}%
\pgfsetdash{}{0pt}%
\pgfpathmoveto{\pgfqpoint{5.953058in}{-2.213349in}}%
\pgfpathlineto{\pgfqpoint{5.953062in}{-2.213333in}}%
\pgfpathlineto{\pgfqpoint{5.953058in}{-2.213325in}}%
\pgfpathlineto{\pgfqpoint{5.953050in}{-2.213333in}}%
\pgfpathclose%
\pgfusepath{fill}%
\end{pgfscope}%
\begin{pgfscope}%
\pgfpathrectangle{\pgfqpoint{3.156364in}{0.528000in}}{\pgfqpoint{1.963636in}{3.696000in}} %
\pgfusepath{clip}%
\pgfsetbuttcap%
\pgfsetroundjoin%
\definecolor{currentfill}{rgb}{0.050383,0.029803,0.527975}%
\pgfsetfillcolor{currentfill}%
\pgfsetlinewidth{0.000000pt}%
\definecolor{currentstroke}{rgb}{0.000000,0.000000,0.000000}%
\pgfsetstrokecolor{currentstroke}%
\pgfsetdash{}{0pt}%
\pgfpathmoveto{\pgfqpoint{4.019174in}{1.839971in}}%
\pgfpathlineto{\pgfqpoint{4.019198in}{1.840000in}}%
\pgfpathlineto{\pgfqpoint{4.019174in}{1.840051in}}%
\pgfpathlineto{\pgfqpoint{4.019160in}{1.840000in}}%
\pgfpathclose%
\pgfusepath{fill}%
\end{pgfscope}%
\begin{pgfscope}%
\pgfpathrectangle{\pgfqpoint{3.156364in}{0.528000in}}{\pgfqpoint{1.963636in}{3.696000in}} %
\pgfusepath{clip}%
\pgfsetbuttcap%
\pgfsetroundjoin%
\definecolor{currentfill}{rgb}{0.050383,0.029803,0.527975}%
\pgfsetfillcolor{currentfill}%
\pgfsetlinewidth{0.000000pt}%
\definecolor{currentstroke}{rgb}{0.000000,0.000000,0.000000}%
\pgfsetstrokecolor{currentstroke}%
\pgfsetdash{}{0pt}%
\pgfpathmoveto{\pgfqpoint{3.969587in}{1.946225in}}%
\pgfpathlineto{\pgfqpoint{3.969937in}{1.946667in}}%
\pgfpathlineto{\pgfqpoint{3.969587in}{1.947388in}}%
\pgfpathlineto{\pgfqpoint{3.969374in}{1.946667in}}%
\pgfpathclose%
\pgfusepath{fill}%
\end{pgfscope}%
\begin{pgfscope}%
\pgfpathrectangle{\pgfqpoint{3.156364in}{0.528000in}}{\pgfqpoint{1.963636in}{3.696000in}} %
\pgfusepath{clip}%
\pgfsetbuttcap%
\pgfsetroundjoin%
\definecolor{currentfill}{rgb}{0.050383,0.029803,0.527975}%
\pgfsetfillcolor{currentfill}%
\pgfsetlinewidth{0.000000pt}%
\definecolor{currentstroke}{rgb}{0.000000,0.000000,0.000000}%
\pgfsetstrokecolor{currentstroke}%
\pgfsetdash{}{0pt}%
\pgfpathmoveto{\pgfqpoint{3.920000in}{2.052514in}}%
\pgfpathlineto{\pgfqpoint{3.920600in}{2.053333in}}%
\pgfpathlineto{\pgfqpoint{3.920000in}{2.054572in}}%
\pgfpathlineto{\pgfqpoint{3.919606in}{2.053333in}}%
\pgfpathclose%
\pgfusepath{fill}%
\end{pgfscope}%
\begin{pgfscope}%
\pgfpathrectangle{\pgfqpoint{3.156364in}{0.528000in}}{\pgfqpoint{1.963636in}{3.696000in}} %
\pgfusepath{clip}%
\pgfsetbuttcap%
\pgfsetroundjoin%
\definecolor{currentfill}{rgb}{0.050383,0.029803,0.527975}%
\pgfsetfillcolor{currentfill}%
\pgfsetlinewidth{0.000000pt}%
\definecolor{currentstroke}{rgb}{0.000000,0.000000,0.000000}%
\pgfsetstrokecolor{currentstroke}%
\pgfsetdash{}{0pt}%
\pgfpathmoveto{\pgfqpoint{3.870413in}{2.158842in}}%
\pgfpathlineto{\pgfqpoint{3.871199in}{2.160000in}}%
\pgfpathlineto{\pgfqpoint{3.870413in}{2.161621in}}%
\pgfpathlineto{\pgfqpoint{3.869856in}{2.160000in}}%
\pgfpathclose%
\pgfusepath{fill}%
\end{pgfscope}%
\begin{pgfscope}%
\pgfpathrectangle{\pgfqpoint{3.156364in}{0.528000in}}{\pgfqpoint{1.963636in}{3.696000in}} %
\pgfusepath{clip}%
\pgfsetbuttcap%
\pgfsetroundjoin%
\definecolor{currentfill}{rgb}{0.050383,0.029803,0.527975}%
\pgfsetfillcolor{currentfill}%
\pgfsetlinewidth{0.000000pt}%
\definecolor{currentstroke}{rgb}{0.000000,0.000000,0.000000}%
\pgfsetstrokecolor{currentstroke}%
\pgfsetdash{}{0pt}%
\pgfpathmoveto{\pgfqpoint{3.820826in}{2.265212in}}%
\pgfpathlineto{\pgfqpoint{3.821740in}{2.266667in}}%
\pgfpathlineto{\pgfqpoint{3.820826in}{2.268555in}}%
\pgfpathlineto{\pgfqpoint{3.820126in}{2.266667in}}%
\pgfpathclose%
\pgfusepath{fill}%
\end{pgfscope}%
\begin{pgfscope}%
\pgfpathrectangle{\pgfqpoint{3.156364in}{0.528000in}}{\pgfqpoint{1.963636in}{3.696000in}} %
\pgfusepath{clip}%
\pgfsetbuttcap%
\pgfsetroundjoin%
\definecolor{currentfill}{rgb}{0.050383,0.029803,0.527975}%
\pgfsetfillcolor{currentfill}%
\pgfsetlinewidth{0.000000pt}%
\definecolor{currentstroke}{rgb}{0.000000,0.000000,0.000000}%
\pgfsetstrokecolor{currentstroke}%
\pgfsetdash{}{0pt}%
\pgfpathmoveto{\pgfqpoint{3.771240in}{2.371628in}}%
\pgfpathlineto{\pgfqpoint{3.772232in}{2.373333in}}%
\pgfpathlineto{\pgfqpoint{3.771240in}{2.375386in}}%
\pgfpathlineto{\pgfqpoint{3.770418in}{2.373333in}}%
\pgfpathclose%
\pgfusepath{fill}%
\end{pgfscope}%
\begin{pgfscope}%
\pgfpathrectangle{\pgfqpoint{3.156364in}{0.528000in}}{\pgfqpoint{1.963636in}{3.696000in}} %
\pgfusepath{clip}%
\pgfsetbuttcap%
\pgfsetroundjoin%
\definecolor{currentfill}{rgb}{0.050383,0.029803,0.527975}%
\pgfsetfillcolor{currentfill}%
\pgfsetlinewidth{0.000000pt}%
\definecolor{currentstroke}{rgb}{0.000000,0.000000,0.000000}%
\pgfsetstrokecolor{currentstroke}%
\pgfsetdash{}{0pt}%
\pgfpathmoveto{\pgfqpoint{3.721653in}{2.478097in}}%
\pgfpathlineto{\pgfqpoint{3.722680in}{2.480000in}}%
\pgfpathlineto{\pgfqpoint{3.721653in}{2.482127in}}%
\pgfpathlineto{\pgfqpoint{3.720735in}{2.480000in}}%
\pgfpathclose%
\pgfpathmoveto{\pgfqpoint{3.721595in}{2.480000in}}%
\pgfpathlineto{\pgfqpoint{3.721653in}{2.480134in}}%
\pgfpathlineto{\pgfqpoint{3.721718in}{2.480000in}}%
\pgfpathlineto{\pgfqpoint{3.721653in}{2.479880in}}%
\pgfpathclose%
\pgfusepath{fill}%
\end{pgfscope}%
\begin{pgfscope}%
\pgfpathrectangle{\pgfqpoint{3.156364in}{0.528000in}}{\pgfqpoint{1.963636in}{3.696000in}} %
\pgfusepath{clip}%
\pgfsetbuttcap%
\pgfsetroundjoin%
\definecolor{currentfill}{rgb}{0.050383,0.029803,0.527975}%
\pgfsetfillcolor{currentfill}%
\pgfsetlinewidth{0.000000pt}%
\definecolor{currentstroke}{rgb}{0.000000,0.000000,0.000000}%
\pgfsetstrokecolor{currentstroke}%
\pgfsetdash{}{0pt}%
\pgfpathmoveto{\pgfqpoint{3.672066in}{2.584623in}}%
\pgfpathlineto{\pgfqpoint{3.673090in}{2.586667in}}%
\pgfpathlineto{\pgfqpoint{3.672066in}{2.588788in}}%
\pgfpathlineto{\pgfqpoint{3.671080in}{2.586667in}}%
\pgfpathclose%
\pgfpathmoveto{\pgfqpoint{3.671971in}{2.586667in}}%
\pgfpathlineto{\pgfqpoint{3.672066in}{2.586870in}}%
\pgfpathlineto{\pgfqpoint{3.672164in}{2.586667in}}%
\pgfpathlineto{\pgfqpoint{3.672066in}{2.586470in}}%
\pgfpathclose%
\pgfusepath{fill}%
\end{pgfscope}%
\begin{pgfscope}%
\pgfpathrectangle{\pgfqpoint{3.156364in}{0.528000in}}{\pgfqpoint{1.963636in}{3.696000in}} %
\pgfusepath{clip}%
\pgfsetbuttcap%
\pgfsetroundjoin%
\definecolor{currentfill}{rgb}{0.050383,0.029803,0.527975}%
\pgfsetfillcolor{currentfill}%
\pgfsetlinewidth{0.000000pt}%
\definecolor{currentstroke}{rgb}{0.000000,0.000000,0.000000}%
\pgfsetstrokecolor{currentstroke}%
\pgfsetdash{}{0pt}%
\pgfpathmoveto{\pgfqpoint{3.622479in}{2.691212in}}%
\pgfpathlineto{\pgfqpoint{3.623465in}{2.693333in}}%
\pgfpathlineto{\pgfqpoint{3.622479in}{2.695377in}}%
\pgfpathlineto{\pgfqpoint{3.621455in}{2.693333in}}%
\pgfpathclose%
\pgfpathmoveto{\pgfqpoint{3.622381in}{2.693333in}}%
\pgfpathlineto{\pgfqpoint{3.622479in}{2.693530in}}%
\pgfpathlineto{\pgfqpoint{3.622574in}{2.693333in}}%
\pgfpathlineto{\pgfqpoint{3.622479in}{2.693130in}}%
\pgfpathclose%
\pgfusepath{fill}%
\end{pgfscope}%
\begin{pgfscope}%
\pgfpathrectangle{\pgfqpoint{3.156364in}{0.528000in}}{\pgfqpoint{1.963636in}{3.696000in}} %
\pgfusepath{clip}%
\pgfsetbuttcap%
\pgfsetroundjoin%
\definecolor{currentfill}{rgb}{0.050383,0.029803,0.527975}%
\pgfsetfillcolor{currentfill}%
\pgfsetlinewidth{0.000000pt}%
\definecolor{currentstroke}{rgb}{0.000000,0.000000,0.000000}%
\pgfsetstrokecolor{currentstroke}%
\pgfsetdash{}{0pt}%
\pgfpathmoveto{\pgfqpoint{3.572893in}{2.797873in}}%
\pgfpathlineto{\pgfqpoint{3.573810in}{2.800000in}}%
\pgfpathlineto{\pgfqpoint{3.572893in}{2.801903in}}%
\pgfpathlineto{\pgfqpoint{3.571865in}{2.800000in}}%
\pgfpathclose%
\pgfpathmoveto{\pgfqpoint{3.572828in}{2.800000in}}%
\pgfpathlineto{\pgfqpoint{3.572893in}{2.800120in}}%
\pgfpathlineto{\pgfqpoint{3.572951in}{2.800000in}}%
\pgfpathlineto{\pgfqpoint{3.572893in}{2.799866in}}%
\pgfpathclose%
\pgfusepath{fill}%
\end{pgfscope}%
\begin{pgfscope}%
\pgfpathrectangle{\pgfqpoint{3.156364in}{0.528000in}}{\pgfqpoint{1.963636in}{3.696000in}} %
\pgfusepath{clip}%
\pgfsetbuttcap%
\pgfsetroundjoin%
\definecolor{currentfill}{rgb}{0.050383,0.029803,0.527975}%
\pgfsetfillcolor{currentfill}%
\pgfsetlinewidth{0.000000pt}%
\definecolor{currentstroke}{rgb}{0.000000,0.000000,0.000000}%
\pgfsetstrokecolor{currentstroke}%
\pgfsetdash{}{0pt}%
\pgfpathmoveto{\pgfqpoint{3.523306in}{2.904614in}}%
\pgfpathlineto{\pgfqpoint{3.524127in}{2.906667in}}%
\pgfpathlineto{\pgfqpoint{3.523306in}{2.908372in}}%
\pgfpathlineto{\pgfqpoint{3.522313in}{2.906667in}}%
\pgfpathclose%
\pgfusepath{fill}%
\end{pgfscope}%
\begin{pgfscope}%
\pgfpathrectangle{\pgfqpoint{3.156364in}{0.528000in}}{\pgfqpoint{1.963636in}{3.696000in}} %
\pgfusepath{clip}%
\pgfsetbuttcap%
\pgfsetroundjoin%
\definecolor{currentfill}{rgb}{0.050383,0.029803,0.527975}%
\pgfsetfillcolor{currentfill}%
\pgfsetlinewidth{0.000000pt}%
\definecolor{currentstroke}{rgb}{0.000000,0.000000,0.000000}%
\pgfsetstrokecolor{currentstroke}%
\pgfsetdash{}{0pt}%
\pgfpathmoveto{\pgfqpoint{3.473719in}{3.011445in}}%
\pgfpathlineto{\pgfqpoint{3.474420in}{3.013333in}}%
\pgfpathlineto{\pgfqpoint{3.473719in}{3.014788in}}%
\pgfpathlineto{\pgfqpoint{3.472805in}{3.013333in}}%
\pgfpathclose%
\pgfusepath{fill}%
\end{pgfscope}%
\begin{pgfscope}%
\pgfpathrectangle{\pgfqpoint{3.156364in}{0.528000in}}{\pgfqpoint{1.963636in}{3.696000in}} %
\pgfusepath{clip}%
\pgfsetbuttcap%
\pgfsetroundjoin%
\definecolor{currentfill}{rgb}{0.050383,0.029803,0.527975}%
\pgfsetfillcolor{currentfill}%
\pgfsetlinewidth{0.000000pt}%
\definecolor{currentstroke}{rgb}{0.000000,0.000000,0.000000}%
\pgfsetstrokecolor{currentstroke}%
\pgfsetdash{}{0pt}%
\pgfpathmoveto{\pgfqpoint{3.424132in}{3.118379in}}%
\pgfpathlineto{\pgfqpoint{3.424690in}{3.120000in}}%
\pgfpathlineto{\pgfqpoint{3.424132in}{3.121158in}}%
\pgfpathlineto{\pgfqpoint{3.423347in}{3.120000in}}%
\pgfpathclose%
\pgfusepath{fill}%
\end{pgfscope}%
\begin{pgfscope}%
\pgfpathrectangle{\pgfqpoint{3.156364in}{0.528000in}}{\pgfqpoint{1.963636in}{3.696000in}} %
\pgfusepath{clip}%
\pgfsetbuttcap%
\pgfsetroundjoin%
\definecolor{currentfill}{rgb}{0.050383,0.029803,0.527975}%
\pgfsetfillcolor{currentfill}%
\pgfsetlinewidth{0.000000pt}%
\definecolor{currentstroke}{rgb}{0.000000,0.000000,0.000000}%
\pgfsetstrokecolor{currentstroke}%
\pgfsetdash{}{0pt}%
\pgfpathmoveto{\pgfqpoint{3.374545in}{3.225428in}}%
\pgfpathlineto{\pgfqpoint{3.374940in}{3.226667in}}%
\pgfpathlineto{\pgfqpoint{3.374545in}{3.227486in}}%
\pgfpathlineto{\pgfqpoint{3.373945in}{3.226667in}}%
\pgfpathclose%
\pgfusepath{fill}%
\end{pgfscope}%
\begin{pgfscope}%
\pgfpathrectangle{\pgfqpoint{3.156364in}{0.528000in}}{\pgfqpoint{1.963636in}{3.696000in}} %
\pgfusepath{clip}%
\pgfsetbuttcap%
\pgfsetroundjoin%
\definecolor{currentfill}{rgb}{0.050383,0.029803,0.527975}%
\pgfsetfillcolor{currentfill}%
\pgfsetlinewidth{0.000000pt}%
\definecolor{currentstroke}{rgb}{0.000000,0.000000,0.000000}%
\pgfsetstrokecolor{currentstroke}%
\pgfsetdash{}{0pt}%
\pgfpathmoveto{\pgfqpoint{3.324959in}{3.332612in}}%
\pgfpathlineto{\pgfqpoint{3.325171in}{3.333333in}}%
\pgfpathlineto{\pgfqpoint{3.324959in}{3.333775in}}%
\pgfpathlineto{\pgfqpoint{3.324608in}{3.333333in}}%
\pgfpathclose%
\pgfusepath{fill}%
\end{pgfscope}%
\begin{pgfscope}%
\pgfpathrectangle{\pgfqpoint{3.156364in}{0.528000in}}{\pgfqpoint{1.963636in}{3.696000in}} %
\pgfusepath{clip}%
\pgfsetbuttcap%
\pgfsetroundjoin%
\definecolor{currentfill}{rgb}{0.050383,0.029803,0.527975}%
\pgfsetfillcolor{currentfill}%
\pgfsetlinewidth{0.000000pt}%
\definecolor{currentstroke}{rgb}{0.000000,0.000000,0.000000}%
\pgfsetstrokecolor{currentstroke}%
\pgfsetdash{}{0pt}%
\pgfpathmoveto{\pgfqpoint{3.275372in}{3.439949in}}%
\pgfpathlineto{\pgfqpoint{3.275386in}{3.440000in}}%
\pgfpathlineto{\pgfqpoint{3.275372in}{3.440029in}}%
\pgfpathlineto{\pgfqpoint{3.275347in}{3.440000in}}%
\pgfpathclose%
\pgfusepath{fill}%
\end{pgfscope}%
\begin{pgfscope}%
\pgfpathrectangle{\pgfqpoint{3.156364in}{0.528000in}}{\pgfqpoint{1.963636in}{3.696000in}} %
\pgfusepath{clip}%
\pgfsetbuttcap%
\pgfsetroundjoin%
\definecolor{currentfill}{rgb}{0.050383,0.029803,0.527975}%
\pgfsetfillcolor{currentfill}%
\pgfsetlinewidth{0.000000pt}%
\definecolor{currentstroke}{rgb}{0.000000,0.000000,0.000000}%
\pgfsetstrokecolor{currentstroke}%
\pgfsetdash{}{0pt}%
\pgfpathmoveto{\pgfqpoint{1.341488in}{7.493325in}}%
\pgfpathlineto{\pgfqpoint{1.341495in}{7.493333in}}%
\pgfpathlineto{\pgfqpoint{1.341488in}{7.493349in}}%
\pgfpathlineto{\pgfqpoint{1.341483in}{7.493333in}}%
\pgfpathclose%
\pgfusepath{fill}%
\end{pgfscope}%
\begin{pgfscope}%
\pgfpathrectangle{\pgfqpoint{3.156364in}{0.528000in}}{\pgfqpoint{1.963636in}{3.696000in}} %
\pgfusepath{clip}%
\pgfsetbuttcap%
\pgfsetroundjoin%
\definecolor{currentfill}{rgb}{0.050383,0.029803,0.527975}%
\pgfsetfillcolor{currentfill}%
\pgfsetlinewidth{0.000000pt}%
\definecolor{currentstroke}{rgb}{0.000000,0.000000,0.000000}%
\pgfsetstrokecolor{currentstroke}%
\pgfsetdash{}{0pt}%
\pgfpathmoveto{\pgfqpoint{1.291901in}{7.599577in}}%
\pgfpathlineto{\pgfqpoint{1.292238in}{7.600000in}}%
\pgfpathlineto{\pgfqpoint{1.291901in}{7.600694in}}%
\pgfpathlineto{\pgfqpoint{1.291697in}{7.600000in}}%
\pgfpathclose%
\pgfusepath{fill}%
\end{pgfscope}%
\begin{pgfscope}%
\pgfpathrectangle{\pgfqpoint{3.156364in}{0.528000in}}{\pgfqpoint{1.963636in}{3.696000in}} %
\pgfusepath{clip}%
\pgfsetbuttcap%
\pgfsetroundjoin%
\definecolor{currentfill}{rgb}{0.050383,0.029803,0.527975}%
\pgfsetfillcolor{currentfill}%
\pgfsetlinewidth{0.000000pt}%
\definecolor{currentstroke}{rgb}{0.000000,0.000000,0.000000}%
\pgfsetstrokecolor{currentstroke}%
\pgfsetdash{}{0pt}%
\pgfpathmoveto{\pgfqpoint{1.242314in}{7.705864in}}%
\pgfpathlineto{\pgfqpoint{1.242904in}{7.706667in}}%
\pgfpathlineto{\pgfqpoint{1.242314in}{7.707884in}}%
\pgfpathlineto{\pgfqpoint{1.241928in}{7.706667in}}%
\pgfpathclose%
\pgfusepath{fill}%
\end{pgfscope}%
\begin{pgfscope}%
\pgfpathrectangle{\pgfqpoint{3.156364in}{0.528000in}}{\pgfqpoint{1.963636in}{3.696000in}} %
\pgfusepath{clip}%
\pgfsetbuttcap%
\pgfsetroundjoin%
\definecolor{currentfill}{rgb}{0.050383,0.029803,0.527975}%
\pgfsetfillcolor{currentfill}%
\pgfsetlinewidth{0.000000pt}%
\definecolor{currentstroke}{rgb}{0.000000,0.000000,0.000000}%
\pgfsetstrokecolor{currentstroke}%
\pgfsetdash{}{0pt}%
\pgfpathmoveto{\pgfqpoint{1.193505in}{7.813333in}}%
\pgfpathlineto{\pgfqpoint{1.192727in}{7.814940in}}%
\pgfpathlineto{\pgfqpoint{1.192727in}{7.813333in}}%
\pgfpathlineto{\pgfqpoint{1.192727in}{7.812190in}}%
\pgfpathclose%
\pgfusepath{fill}%
\end{pgfscope}%
\begin{pgfscope}%
\pgfpathrectangle{\pgfqpoint{3.156364in}{0.528000in}}{\pgfqpoint{1.963636in}{3.696000in}} %
\pgfusepath{clip}%
\pgfsetbuttcap%
\pgfsetroundjoin%
\definecolor{currentfill}{rgb}{0.050383,0.029803,0.527975}%
\pgfsetfillcolor{currentfill}%
\pgfsetlinewidth{0.000000pt}%
\definecolor{currentstroke}{rgb}{0.000000,0.000000,0.000000}%
\pgfsetstrokecolor{currentstroke}%
\pgfsetdash{}{0pt}%
\pgfpathmoveto{\pgfqpoint{6.101818in}{-2.557564in}}%
\pgfpathlineto{\pgfqpoint{6.101818in}{-2.534940in}}%
\pgfpathlineto{\pgfqpoint{6.101040in}{-2.533333in}}%
\pgfpathlineto{\pgfqpoint{6.101818in}{-2.532190in}}%
\pgfpathlineto{\pgfqpoint{6.101818in}{-2.516082in}}%
\pgfpathlineto{\pgfqpoint{6.090080in}{-2.533333in}}%
\pgfpathclose%
\pgfusepath{fill}%
\end{pgfscope}%
\begin{pgfscope}%
\pgfpathrectangle{\pgfqpoint{3.156364in}{0.528000in}}{\pgfqpoint{1.963636in}{3.696000in}} %
\pgfusepath{clip}%
\pgfsetbuttcap%
\pgfsetroundjoin%
\definecolor{currentfill}{rgb}{0.050383,0.029803,0.527975}%
\pgfsetfillcolor{currentfill}%
\pgfsetlinewidth{0.000000pt}%
\definecolor{currentstroke}{rgb}{0.000000,0.000000,0.000000}%
\pgfsetstrokecolor{currentstroke}%
\pgfsetdash{}{0pt}%
\pgfpathmoveto{\pgfqpoint{6.052231in}{-2.451562in}}%
\pgfpathlineto{\pgfqpoint{6.060128in}{-2.426667in}}%
\pgfpathlineto{\pgfqpoint{6.052231in}{-2.410251in}}%
\pgfpathlineto{\pgfqpoint{6.040159in}{-2.426667in}}%
\pgfpathclose%
\pgfpathmoveto{\pgfqpoint{6.051641in}{-2.426667in}}%
\pgfpathlineto{\pgfqpoint{6.052231in}{-2.425864in}}%
\pgfpathlineto{\pgfqpoint{6.052618in}{-2.426667in}}%
\pgfpathlineto{\pgfqpoint{6.052231in}{-2.427884in}}%
\pgfpathclose%
\pgfusepath{fill}%
\end{pgfscope}%
\begin{pgfscope}%
\pgfpathrectangle{\pgfqpoint{3.156364in}{0.528000in}}{\pgfqpoint{1.963636in}{3.696000in}} %
\pgfusepath{clip}%
\pgfsetbuttcap%
\pgfsetroundjoin%
\definecolor{currentfill}{rgb}{0.050383,0.029803,0.527975}%
\pgfsetfillcolor{currentfill}%
\pgfsetlinewidth{0.000000pt}%
\definecolor{currentstroke}{rgb}{0.000000,0.000000,0.000000}%
\pgfsetstrokecolor{currentstroke}%
\pgfsetdash{}{0pt}%
\pgfpathmoveto{\pgfqpoint{6.002645in}{-2.345528in}}%
\pgfpathlineto{\pgfqpoint{6.010131in}{-2.320000in}}%
\pgfpathlineto{\pgfqpoint{6.002645in}{-2.304428in}}%
\pgfpathlineto{\pgfqpoint{5.990251in}{-2.320000in}}%
\pgfpathclose%
\pgfpathmoveto{\pgfqpoint{6.002308in}{-2.320000in}}%
\pgfpathlineto{\pgfqpoint{6.002645in}{-2.319577in}}%
\pgfpathlineto{\pgfqpoint{6.002848in}{-2.320000in}}%
\pgfpathlineto{\pgfqpoint{6.002645in}{-2.320694in}}%
\pgfpathclose%
\pgfusepath{fill}%
\end{pgfscope}%
\begin{pgfscope}%
\pgfpathrectangle{\pgfqpoint{3.156364in}{0.528000in}}{\pgfqpoint{1.963636in}{3.696000in}} %
\pgfusepath{clip}%
\pgfsetbuttcap%
\pgfsetroundjoin%
\definecolor{currentfill}{rgb}{0.050383,0.029803,0.527975}%
\pgfsetfillcolor{currentfill}%
\pgfsetlinewidth{0.000000pt}%
\definecolor{currentstroke}{rgb}{0.000000,0.000000,0.000000}%
\pgfsetstrokecolor{currentstroke}%
\pgfsetdash{}{0pt}%
\pgfpathmoveto{\pgfqpoint{5.953058in}{-2.239458in}}%
\pgfpathlineto{\pgfqpoint{5.960132in}{-2.213333in}}%
\pgfpathlineto{\pgfqpoint{5.953058in}{-2.198615in}}%
\pgfpathlineto{\pgfqpoint{5.940358in}{-2.213333in}}%
\pgfpathclose%
\pgfpathmoveto{\pgfqpoint{5.953050in}{-2.213333in}}%
\pgfpathlineto{\pgfqpoint{5.953058in}{-2.213325in}}%
\pgfpathlineto{\pgfqpoint{5.953062in}{-2.213333in}}%
\pgfpathlineto{\pgfqpoint{5.953058in}{-2.213349in}}%
\pgfpathclose%
\pgfusepath{fill}%
\end{pgfscope}%
\begin{pgfscope}%
\pgfpathrectangle{\pgfqpoint{3.156364in}{0.528000in}}{\pgfqpoint{1.963636in}{3.696000in}} %
\pgfusepath{clip}%
\pgfsetbuttcap%
\pgfsetroundjoin%
\definecolor{currentfill}{rgb}{0.050383,0.029803,0.527975}%
\pgfsetfillcolor{currentfill}%
\pgfsetlinewidth{0.000000pt}%
\definecolor{currentstroke}{rgb}{0.000000,0.000000,0.000000}%
\pgfsetstrokecolor{currentstroke}%
\pgfsetdash{}{0pt}%
\pgfpathmoveto{\pgfqpoint{5.903471in}{-2.133346in}}%
\pgfpathlineto{\pgfqpoint{5.910129in}{-2.106667in}}%
\pgfpathlineto{\pgfqpoint{5.903471in}{-2.092808in}}%
\pgfpathlineto{\pgfqpoint{5.890483in}{-2.106667in}}%
\pgfpathclose%
\pgfusepath{fill}%
\end{pgfscope}%
\begin{pgfscope}%
\pgfpathrectangle{\pgfqpoint{3.156364in}{0.528000in}}{\pgfqpoint{1.963636in}{3.696000in}} %
\pgfusepath{clip}%
\pgfsetbuttcap%
\pgfsetroundjoin%
\definecolor{currentfill}{rgb}{0.050383,0.029803,0.527975}%
\pgfsetfillcolor{currentfill}%
\pgfsetlinewidth{0.000000pt}%
\definecolor{currentstroke}{rgb}{0.000000,0.000000,0.000000}%
\pgfsetstrokecolor{currentstroke}%
\pgfsetdash{}{0pt}%
\pgfpathmoveto{\pgfqpoint{5.853884in}{-2.027184in}}%
\pgfpathlineto{\pgfqpoint{5.860123in}{-2.000000in}}%
\pgfpathlineto{\pgfqpoint{5.853884in}{-1.987009in}}%
\pgfpathlineto{\pgfqpoint{5.840630in}{-2.000000in}}%
\pgfpathclose%
\pgfusepath{fill}%
\end{pgfscope}%
\begin{pgfscope}%
\pgfpathrectangle{\pgfqpoint{3.156364in}{0.528000in}}{\pgfqpoint{1.963636in}{3.696000in}} %
\pgfusepath{clip}%
\pgfsetbuttcap%
\pgfsetroundjoin%
\definecolor{currentfill}{rgb}{0.050383,0.029803,0.527975}%
\pgfsetfillcolor{currentfill}%
\pgfsetlinewidth{0.000000pt}%
\definecolor{currentstroke}{rgb}{0.000000,0.000000,0.000000}%
\pgfsetstrokecolor{currentstroke}%
\pgfsetdash{}{0pt}%
\pgfpathmoveto{\pgfqpoint{5.804298in}{-1.920964in}}%
\pgfpathlineto{\pgfqpoint{5.810114in}{-1.893333in}}%
\pgfpathlineto{\pgfqpoint{5.804298in}{-1.881216in}}%
\pgfpathlineto{\pgfqpoint{5.790801in}{-1.893333in}}%
\pgfpathclose%
\pgfusepath{fill}%
\end{pgfscope}%
\begin{pgfscope}%
\pgfpathrectangle{\pgfqpoint{3.156364in}{0.528000in}}{\pgfqpoint{1.963636in}{3.696000in}} %
\pgfusepath{clip}%
\pgfsetbuttcap%
\pgfsetroundjoin%
\definecolor{currentfill}{rgb}{0.050383,0.029803,0.527975}%
\pgfsetfillcolor{currentfill}%
\pgfsetlinewidth{0.000000pt}%
\definecolor{currentstroke}{rgb}{0.000000,0.000000,0.000000}%
\pgfsetstrokecolor{currentstroke}%
\pgfsetdash{}{0pt}%
\pgfpathmoveto{\pgfqpoint{5.754711in}{-1.814674in}}%
\pgfpathlineto{\pgfqpoint{5.760104in}{-1.786667in}}%
\pgfpathlineto{\pgfqpoint{5.754711in}{-1.775428in}}%
\pgfpathlineto{\pgfqpoint{5.741003in}{-1.786667in}}%
\pgfpathclose%
\pgfusepath{fill}%
\end{pgfscope}%
\begin{pgfscope}%
\pgfpathrectangle{\pgfqpoint{3.156364in}{0.528000in}}{\pgfqpoint{1.963636in}{3.696000in}} %
\pgfusepath{clip}%
\pgfsetbuttcap%
\pgfsetroundjoin%
\definecolor{currentfill}{rgb}{0.050383,0.029803,0.527975}%
\pgfsetfillcolor{currentfill}%
\pgfsetlinewidth{0.000000pt}%
\definecolor{currentstroke}{rgb}{0.000000,0.000000,0.000000}%
\pgfsetstrokecolor{currentstroke}%
\pgfsetdash{}{0pt}%
\pgfpathmoveto{\pgfqpoint{5.705124in}{-1.708300in}}%
\pgfpathlineto{\pgfqpoint{5.710091in}{-1.680000in}}%
\pgfpathlineto{\pgfqpoint{5.705124in}{-1.669646in}}%
\pgfpathlineto{\pgfqpoint{5.691241in}{-1.680000in}}%
\pgfpathclose%
\pgfusepath{fill}%
\end{pgfscope}%
\begin{pgfscope}%
\pgfpathrectangle{\pgfqpoint{3.156364in}{0.528000in}}{\pgfqpoint{1.963636in}{3.696000in}} %
\pgfusepath{clip}%
\pgfsetbuttcap%
\pgfsetroundjoin%
\definecolor{currentfill}{rgb}{0.050383,0.029803,0.527975}%
\pgfsetfillcolor{currentfill}%
\pgfsetlinewidth{0.000000pt}%
\definecolor{currentstroke}{rgb}{0.000000,0.000000,0.000000}%
\pgfsetstrokecolor{currentstroke}%
\pgfsetdash{}{0pt}%
\pgfpathmoveto{\pgfqpoint{5.655537in}{-1.601822in}}%
\pgfpathlineto{\pgfqpoint{5.660075in}{-1.573333in}}%
\pgfpathlineto{\pgfqpoint{5.655537in}{-1.563870in}}%
\pgfpathlineto{\pgfqpoint{5.641524in}{-1.573333in}}%
\pgfpathclose%
\pgfusepath{fill}%
\end{pgfscope}%
\begin{pgfscope}%
\pgfpathrectangle{\pgfqpoint{3.156364in}{0.528000in}}{\pgfqpoint{1.963636in}{3.696000in}} %
\pgfusepath{clip}%
\pgfsetbuttcap%
\pgfsetroundjoin%
\definecolor{currentfill}{rgb}{0.050383,0.029803,0.527975}%
\pgfsetfillcolor{currentfill}%
\pgfsetlinewidth{0.000000pt}%
\definecolor{currentstroke}{rgb}{0.000000,0.000000,0.000000}%
\pgfsetstrokecolor{currentstroke}%
\pgfsetdash{}{0pt}%
\pgfpathmoveto{\pgfqpoint{5.605950in}{-1.495217in}}%
\pgfpathlineto{\pgfqpoint{5.610058in}{-1.466667in}}%
\pgfpathlineto{\pgfqpoint{5.605950in}{-1.458098in}}%
\pgfpathlineto{\pgfqpoint{5.591863in}{-1.466667in}}%
\pgfpathclose%
\pgfusepath{fill}%
\end{pgfscope}%
\begin{pgfscope}%
\pgfpathrectangle{\pgfqpoint{3.156364in}{0.528000in}}{\pgfqpoint{1.963636in}{3.696000in}} %
\pgfusepath{clip}%
\pgfsetbuttcap%
\pgfsetroundjoin%
\definecolor{currentfill}{rgb}{0.050383,0.029803,0.527975}%
\pgfsetfillcolor{currentfill}%
\pgfsetlinewidth{0.000000pt}%
\definecolor{currentstroke}{rgb}{0.000000,0.000000,0.000000}%
\pgfsetstrokecolor{currentstroke}%
\pgfsetdash{}{0pt}%
\pgfpathmoveto{\pgfqpoint{5.556364in}{-1.388451in}}%
\pgfpathlineto{\pgfqpoint{5.560040in}{-1.360000in}}%
\pgfpathlineto{\pgfqpoint{5.556364in}{-1.352330in}}%
\pgfpathlineto{\pgfqpoint{5.542275in}{-1.360000in}}%
\pgfpathclose%
\pgfusepath{fill}%
\end{pgfscope}%
\begin{pgfscope}%
\pgfpathrectangle{\pgfqpoint{3.156364in}{0.528000in}}{\pgfqpoint{1.963636in}{3.696000in}} %
\pgfusepath{clip}%
\pgfsetbuttcap%
\pgfsetroundjoin%
\definecolor{currentfill}{rgb}{0.050383,0.029803,0.527975}%
\pgfsetfillcolor{currentfill}%
\pgfsetlinewidth{0.000000pt}%
\definecolor{currentstroke}{rgb}{0.000000,0.000000,0.000000}%
\pgfsetstrokecolor{currentstroke}%
\pgfsetdash{}{0pt}%
\pgfpathmoveto{\pgfqpoint{5.506777in}{-1.281477in}}%
\pgfpathlineto{\pgfqpoint{5.510019in}{-1.253333in}}%
\pgfpathlineto{\pgfqpoint{5.506777in}{-1.246566in}}%
\pgfpathlineto{\pgfqpoint{5.492779in}{-1.253333in}}%
\pgfpathclose%
\pgfusepath{fill}%
\end{pgfscope}%
\begin{pgfscope}%
\pgfpathrectangle{\pgfqpoint{3.156364in}{0.528000in}}{\pgfqpoint{1.963636in}{3.696000in}} %
\pgfusepath{clip}%
\pgfsetbuttcap%
\pgfsetroundjoin%
\definecolor{currentfill}{rgb}{0.050383,0.029803,0.527975}%
\pgfsetfillcolor{currentfill}%
\pgfsetlinewidth{0.000000pt}%
\definecolor{currentstroke}{rgb}{0.000000,0.000000,0.000000}%
\pgfsetstrokecolor{currentstroke}%
\pgfsetdash{}{0pt}%
\pgfpathmoveto{\pgfqpoint{5.457190in}{-1.174230in}}%
\pgfpathlineto{\pgfqpoint{5.459997in}{-1.146667in}}%
\pgfpathlineto{\pgfqpoint{5.457190in}{-1.140806in}}%
\pgfpathlineto{\pgfqpoint{5.443408in}{-1.146667in}}%
\pgfpathclose%
\pgfusepath{fill}%
\end{pgfscope}%
\begin{pgfscope}%
\pgfpathrectangle{\pgfqpoint{3.156364in}{0.528000in}}{\pgfqpoint{1.963636in}{3.696000in}} %
\pgfusepath{clip}%
\pgfsetbuttcap%
\pgfsetroundjoin%
\definecolor{currentfill}{rgb}{0.050383,0.029803,0.527975}%
\pgfsetfillcolor{currentfill}%
\pgfsetlinewidth{0.000000pt}%
\definecolor{currentstroke}{rgb}{0.000000,0.000000,0.000000}%
\pgfsetstrokecolor{currentstroke}%
\pgfsetdash{}{0pt}%
\pgfpathmoveto{\pgfqpoint{5.407603in}{-1.066613in}}%
\pgfpathlineto{\pgfqpoint{5.409974in}{-1.040000in}}%
\pgfpathlineto{\pgfqpoint{5.407603in}{-1.035050in}}%
\pgfpathlineto{\pgfqpoint{5.394208in}{-1.040000in}}%
\pgfpathclose%
\pgfusepath{fill}%
\end{pgfscope}%
\begin{pgfscope}%
\pgfpathrectangle{\pgfqpoint{3.156364in}{0.528000in}}{\pgfqpoint{1.963636in}{3.696000in}} %
\pgfusepath{clip}%
\pgfsetbuttcap%
\pgfsetroundjoin%
\definecolor{currentfill}{rgb}{0.050383,0.029803,0.527975}%
\pgfsetfillcolor{currentfill}%
\pgfsetlinewidth{0.000000pt}%
\definecolor{currentstroke}{rgb}{0.000000,0.000000,0.000000}%
\pgfsetstrokecolor{currentstroke}%
\pgfsetdash{}{0pt}%
\pgfpathmoveto{\pgfqpoint{5.358017in}{-0.958478in}}%
\pgfpathlineto{\pgfqpoint{5.359949in}{-0.933333in}}%
\pgfpathlineto{\pgfqpoint{5.358017in}{-0.929296in}}%
\pgfpathlineto{\pgfqpoint{5.345254in}{-0.933333in}}%
\pgfpathclose%
\pgfusepath{fill}%
\end{pgfscope}%
\begin{pgfscope}%
\pgfpathrectangle{\pgfqpoint{3.156364in}{0.528000in}}{\pgfqpoint{1.963636in}{3.696000in}} %
\pgfusepath{clip}%
\pgfsetbuttcap%
\pgfsetroundjoin%
\definecolor{currentfill}{rgb}{0.050383,0.029803,0.527975}%
\pgfsetfillcolor{currentfill}%
\pgfsetlinewidth{0.000000pt}%
\definecolor{currentstroke}{rgb}{0.000000,0.000000,0.000000}%
\pgfsetstrokecolor{currentstroke}%
\pgfsetdash{}{0pt}%
\pgfpathmoveto{\pgfqpoint{5.308430in}{-0.849585in}}%
\pgfpathlineto{\pgfqpoint{5.309923in}{-0.826667in}}%
\pgfpathlineto{\pgfqpoint{5.308430in}{-0.823546in}}%
\pgfpathlineto{\pgfqpoint{5.296667in}{-0.826667in}}%
\pgfpathclose%
\pgfusepath{fill}%
\end{pgfscope}%
\begin{pgfscope}%
\pgfpathrectangle{\pgfqpoint{3.156364in}{0.528000in}}{\pgfqpoint{1.963636in}{3.696000in}} %
\pgfusepath{clip}%
\pgfsetbuttcap%
\pgfsetroundjoin%
\definecolor{currentfill}{rgb}{0.050383,0.029803,0.527975}%
\pgfsetfillcolor{currentfill}%
\pgfsetlinewidth{0.000000pt}%
\definecolor{currentstroke}{rgb}{0.000000,0.000000,0.000000}%
\pgfsetstrokecolor{currentstroke}%
\pgfsetdash{}{0pt}%
\pgfpathmoveto{\pgfqpoint{5.258843in}{-0.739519in}}%
\pgfpathlineto{\pgfqpoint{5.259896in}{-0.720000in}}%
\pgfpathlineto{\pgfqpoint{5.258843in}{-0.717799in}}%
\pgfpathlineto{\pgfqpoint{5.248669in}{-0.720000in}}%
\pgfpathclose%
\pgfusepath{fill}%
\end{pgfscope}%
\begin{pgfscope}%
\pgfpathrectangle{\pgfqpoint{3.156364in}{0.528000in}}{\pgfqpoint{1.963636in}{3.696000in}} %
\pgfusepath{clip}%
\pgfsetbuttcap%
\pgfsetroundjoin%
\definecolor{currentfill}{rgb}{0.050383,0.029803,0.527975}%
\pgfsetfillcolor{currentfill}%
\pgfsetlinewidth{0.000000pt}%
\definecolor{currentstroke}{rgb}{0.000000,0.000000,0.000000}%
\pgfsetstrokecolor{currentstroke}%
\pgfsetdash{}{0pt}%
\pgfpathmoveto{\pgfqpoint{5.209256in}{-0.627492in}}%
\pgfpathlineto{\pgfqpoint{5.209868in}{-0.613333in}}%
\pgfpathlineto{\pgfqpoint{5.209256in}{-0.612055in}}%
\pgfpathlineto{\pgfqpoint{5.201707in}{-0.613333in}}%
\pgfpathclose%
\pgfusepath{fill}%
\end{pgfscope}%
\begin{pgfscope}%
\pgfpathrectangle{\pgfqpoint{3.156364in}{0.528000in}}{\pgfqpoint{1.963636in}{3.696000in}} %
\pgfusepath{clip}%
\pgfsetbuttcap%
\pgfsetroundjoin%
\definecolor{currentfill}{rgb}{0.050383,0.029803,0.527975}%
\pgfsetfillcolor{currentfill}%
\pgfsetlinewidth{0.000000pt}%
\definecolor{currentstroke}{rgb}{0.000000,0.000000,0.000000}%
\pgfsetstrokecolor{currentstroke}%
\pgfsetdash{}{0pt}%
\pgfpathmoveto{\pgfqpoint{5.159669in}{-0.511816in}}%
\pgfpathlineto{\pgfqpoint{5.159839in}{-0.506667in}}%
\pgfpathlineto{\pgfqpoint{5.159669in}{-0.506313in}}%
\pgfpathlineto{\pgfqpoint{5.156821in}{-0.506667in}}%
\pgfpathclose%
\pgfusepath{fill}%
\end{pgfscope}%
\begin{pgfscope}%
\pgfpathrectangle{\pgfqpoint{3.156364in}{0.528000in}}{\pgfqpoint{1.963636in}{3.696000in}} %
\pgfusepath{clip}%
\pgfsetbuttcap%
\pgfsetroundjoin%
\definecolor{currentfill}{rgb}{0.050383,0.029803,0.527975}%
\pgfsetfillcolor{currentfill}%
\pgfsetlinewidth{0.000000pt}%
\definecolor{currentstroke}{rgb}{0.000000,0.000000,0.000000}%
\pgfsetstrokecolor{currentstroke}%
\pgfsetdash{}{0pt}%
\pgfpathmoveto{\pgfqpoint{4.812562in}{0.132936in}}%
\pgfpathlineto{\pgfqpoint{4.815705in}{0.133333in}}%
\pgfpathlineto{\pgfqpoint{4.812562in}{0.139027in}}%
\pgfpathlineto{\pgfqpoint{4.812372in}{0.133333in}}%
\pgfpathclose%
\pgfusepath{fill}%
\end{pgfscope}%
\begin{pgfscope}%
\pgfpathrectangle{\pgfqpoint{3.156364in}{0.528000in}}{\pgfqpoint{1.963636in}{3.696000in}} %
\pgfusepath{clip}%
\pgfsetbuttcap%
\pgfsetroundjoin%
\definecolor{currentfill}{rgb}{0.050383,0.029803,0.527975}%
\pgfsetfillcolor{currentfill}%
\pgfsetlinewidth{0.000000pt}%
\definecolor{currentstroke}{rgb}{0.000000,0.000000,0.000000}%
\pgfsetstrokecolor{currentstroke}%
\pgfsetdash{}{0pt}%
\pgfpathmoveto{\pgfqpoint{4.762975in}{0.238678in}}%
\pgfpathlineto{\pgfqpoint{4.770681in}{0.240000in}}%
\pgfpathlineto{\pgfqpoint{4.762975in}{0.254471in}}%
\pgfpathlineto{\pgfqpoint{4.762343in}{0.240000in}}%
\pgfpathclose%
\pgfusepath{fill}%
\end{pgfscope}%
\begin{pgfscope}%
\pgfpathrectangle{\pgfqpoint{3.156364in}{0.528000in}}{\pgfqpoint{1.963636in}{3.696000in}} %
\pgfusepath{clip}%
\pgfsetbuttcap%
\pgfsetroundjoin%
\definecolor{currentfill}{rgb}{0.050383,0.029803,0.527975}%
\pgfsetfillcolor{currentfill}%
\pgfsetlinewidth{0.000000pt}%
\definecolor{currentstroke}{rgb}{0.000000,0.000000,0.000000}%
\pgfsetstrokecolor{currentstroke}%
\pgfsetdash{}{0pt}%
\pgfpathmoveto{\pgfqpoint{4.713388in}{0.344423in}}%
\pgfpathlineto{\pgfqpoint{4.723654in}{0.346667in}}%
\pgfpathlineto{\pgfqpoint{4.713388in}{0.366378in}}%
\pgfpathlineto{\pgfqpoint{4.712315in}{0.346667in}}%
\pgfpathclose%
\pgfusepath{fill}%
\end{pgfscope}%
\begin{pgfscope}%
\pgfpathrectangle{\pgfqpoint{3.156364in}{0.528000in}}{\pgfqpoint{1.963636in}{3.696000in}} %
\pgfusepath{clip}%
\pgfsetbuttcap%
\pgfsetroundjoin%
\definecolor{currentfill}{rgb}{0.050383,0.029803,0.527975}%
\pgfsetfillcolor{currentfill}%
\pgfsetlinewidth{0.000000pt}%
\definecolor{currentstroke}{rgb}{0.000000,0.000000,0.000000}%
\pgfsetstrokecolor{currentstroke}%
\pgfsetdash{}{0pt}%
\pgfpathmoveto{\pgfqpoint{4.663802in}{0.450170in}}%
\pgfpathlineto{\pgfqpoint{4.675622in}{0.453333in}}%
\pgfpathlineto{\pgfqpoint{4.663802in}{0.476377in}}%
\pgfpathlineto{\pgfqpoint{4.662288in}{0.453333in}}%
\pgfpathclose%
\pgfusepath{fill}%
\end{pgfscope}%
\begin{pgfscope}%
\pgfpathrectangle{\pgfqpoint{3.156364in}{0.528000in}}{\pgfqpoint{1.963636in}{3.696000in}} %
\pgfusepath{clip}%
\pgfsetbuttcap%
\pgfsetroundjoin%
\definecolor{currentfill}{rgb}{0.050383,0.029803,0.527975}%
\pgfsetfillcolor{currentfill}%
\pgfsetlinewidth{0.000000pt}%
\definecolor{currentstroke}{rgb}{0.000000,0.000000,0.000000}%
\pgfsetstrokecolor{currentstroke}%
\pgfsetdash{}{0pt}%
\pgfpathmoveto{\pgfqpoint{4.614215in}{0.555921in}}%
\pgfpathlineto{\pgfqpoint{4.627014in}{0.560000in}}%
\pgfpathlineto{\pgfqpoint{4.614215in}{0.585227in}}%
\pgfpathlineto{\pgfqpoint{4.612262in}{0.560000in}}%
\pgfpathclose%
\pgfusepath{fill}%
\end{pgfscope}%
\begin{pgfscope}%
\pgfpathrectangle{\pgfqpoint{3.156364in}{0.528000in}}{\pgfqpoint{1.963636in}{3.696000in}} %
\pgfusepath{clip}%
\pgfsetbuttcap%
\pgfsetroundjoin%
\definecolor{currentfill}{rgb}{0.050383,0.029803,0.527975}%
\pgfsetfillcolor{currentfill}%
\pgfsetlinewidth{0.000000pt}%
\definecolor{currentstroke}{rgb}{0.000000,0.000000,0.000000}%
\pgfsetstrokecolor{currentstroke}%
\pgfsetdash{}{0pt}%
\pgfpathmoveto{\pgfqpoint{4.564628in}{0.661674in}}%
\pgfpathlineto{\pgfqpoint{4.578046in}{0.666667in}}%
\pgfpathlineto{\pgfqpoint{4.564628in}{0.693334in}}%
\pgfpathlineto{\pgfqpoint{4.562237in}{0.666667in}}%
\pgfpathclose%
\pgfusepath{fill}%
\end{pgfscope}%
\begin{pgfscope}%
\pgfpathrectangle{\pgfqpoint{3.156364in}{0.528000in}}{\pgfqpoint{1.963636in}{3.696000in}} %
\pgfusepath{clip}%
\pgfsetbuttcap%
\pgfsetroundjoin%
\definecolor{currentfill}{rgb}{0.050383,0.029803,0.527975}%
\pgfsetfillcolor{currentfill}%
\pgfsetlinewidth{0.000000pt}%
\definecolor{currentstroke}{rgb}{0.000000,0.000000,0.000000}%
\pgfsetstrokecolor{currentstroke}%
\pgfsetdash{}{0pt}%
\pgfpathmoveto{\pgfqpoint{4.515041in}{0.767430in}}%
\pgfpathlineto{\pgfqpoint{4.528837in}{0.773333in}}%
\pgfpathlineto{\pgfqpoint{4.515041in}{0.800931in}}%
\pgfpathlineto{\pgfqpoint{4.512214in}{0.773333in}}%
\pgfpathclose%
\pgfusepath{fill}%
\end{pgfscope}%
\begin{pgfscope}%
\pgfpathrectangle{\pgfqpoint{3.156364in}{0.528000in}}{\pgfqpoint{1.963636in}{3.696000in}} %
\pgfusepath{clip}%
\pgfsetbuttcap%
\pgfsetroundjoin%
\definecolor{currentfill}{rgb}{0.050383,0.029803,0.527975}%
\pgfsetfillcolor{currentfill}%
\pgfsetlinewidth{0.000000pt}%
\definecolor{currentstroke}{rgb}{0.000000,0.000000,0.000000}%
\pgfsetstrokecolor{currentstroke}%
\pgfsetdash{}{0pt}%
\pgfpathmoveto{\pgfqpoint{4.465455in}{0.873191in}}%
\pgfpathlineto{\pgfqpoint{4.479459in}{0.880000in}}%
\pgfpathlineto{\pgfqpoint{4.465455in}{0.908163in}}%
\pgfpathlineto{\pgfqpoint{4.462192in}{0.880000in}}%
\pgfpathclose%
\pgfusepath{fill}%
\end{pgfscope}%
\begin{pgfscope}%
\pgfpathrectangle{\pgfqpoint{3.156364in}{0.528000in}}{\pgfqpoint{1.963636in}{3.696000in}} %
\pgfusepath{clip}%
\pgfsetbuttcap%
\pgfsetroundjoin%
\definecolor{currentfill}{rgb}{0.050383,0.029803,0.527975}%
\pgfsetfillcolor{currentfill}%
\pgfsetlinewidth{0.000000pt}%
\definecolor{currentstroke}{rgb}{0.000000,0.000000,0.000000}%
\pgfsetstrokecolor{currentstroke}%
\pgfsetdash{}{0pt}%
\pgfpathmoveto{\pgfqpoint{4.415868in}{0.978955in}}%
\pgfpathlineto{\pgfqpoint{4.429959in}{0.986667in}}%
\pgfpathlineto{\pgfqpoint{4.415868in}{1.015126in}}%
\pgfpathlineto{\pgfqpoint{4.412172in}{0.986667in}}%
\pgfpathclose%
\pgfusepath{fill}%
\end{pgfscope}%
\begin{pgfscope}%
\pgfpathrectangle{\pgfqpoint{3.156364in}{0.528000in}}{\pgfqpoint{1.963636in}{3.696000in}} %
\pgfusepath{clip}%
\pgfsetbuttcap%
\pgfsetroundjoin%
\definecolor{currentfill}{rgb}{0.050383,0.029803,0.527975}%
\pgfsetfillcolor{currentfill}%
\pgfsetlinewidth{0.000000pt}%
\definecolor{currentstroke}{rgb}{0.000000,0.000000,0.000000}%
\pgfsetstrokecolor{currentstroke}%
\pgfsetdash{}{0pt}%
\pgfpathmoveto{\pgfqpoint{4.366281in}{1.084723in}}%
\pgfpathlineto{\pgfqpoint{4.380366in}{1.093333in}}%
\pgfpathlineto{\pgfqpoint{4.366281in}{1.121884in}}%
\pgfpathlineto{\pgfqpoint{4.362153in}{1.093333in}}%
\pgfpathclose%
\pgfusepath{fill}%
\end{pgfscope}%
\begin{pgfscope}%
\pgfpathrectangle{\pgfqpoint{3.156364in}{0.528000in}}{\pgfqpoint{1.963636in}{3.696000in}} %
\pgfusepath{clip}%
\pgfsetbuttcap%
\pgfsetroundjoin%
\definecolor{currentfill}{rgb}{0.050383,0.029803,0.527975}%
\pgfsetfillcolor{currentfill}%
\pgfsetlinewidth{0.000000pt}%
\definecolor{currentstroke}{rgb}{0.000000,0.000000,0.000000}%
\pgfsetstrokecolor{currentstroke}%
\pgfsetdash{}{0pt}%
\pgfpathmoveto{\pgfqpoint{4.316694in}{1.190495in}}%
\pgfpathlineto{\pgfqpoint{4.330703in}{1.200000in}}%
\pgfpathlineto{\pgfqpoint{4.316694in}{1.228483in}}%
\pgfpathlineto{\pgfqpoint{4.312136in}{1.200000in}}%
\pgfpathclose%
\pgfusepath{fill}%
\end{pgfscope}%
\begin{pgfscope}%
\pgfpathrectangle{\pgfqpoint{3.156364in}{0.528000in}}{\pgfqpoint{1.963636in}{3.696000in}} %
\pgfusepath{clip}%
\pgfsetbuttcap%
\pgfsetroundjoin%
\definecolor{currentfill}{rgb}{0.050383,0.029803,0.527975}%
\pgfsetfillcolor{currentfill}%
\pgfsetlinewidth{0.000000pt}%
\definecolor{currentstroke}{rgb}{0.000000,0.000000,0.000000}%
\pgfsetstrokecolor{currentstroke}%
\pgfsetdash{}{0pt}%
\pgfpathmoveto{\pgfqpoint{4.267107in}{1.296272in}}%
\pgfpathlineto{\pgfqpoint{4.280984in}{1.306667in}}%
\pgfpathlineto{\pgfqpoint{4.267107in}{1.334955in}}%
\pgfpathlineto{\pgfqpoint{4.262121in}{1.306667in}}%
\pgfpathclose%
\pgfusepath{fill}%
\end{pgfscope}%
\begin{pgfscope}%
\pgfpathrectangle{\pgfqpoint{3.156364in}{0.528000in}}{\pgfqpoint{1.963636in}{3.696000in}} %
\pgfusepath{clip}%
\pgfsetbuttcap%
\pgfsetroundjoin%
\definecolor{currentfill}{rgb}{0.050383,0.029803,0.527975}%
\pgfsetfillcolor{currentfill}%
\pgfsetlinewidth{0.000000pt}%
\definecolor{currentstroke}{rgb}{0.000000,0.000000,0.000000}%
\pgfsetstrokecolor{currentstroke}%
\pgfsetdash{}{0pt}%
\pgfpathmoveto{\pgfqpoint{4.217521in}{1.402054in}}%
\pgfpathlineto{\pgfqpoint{4.231220in}{1.413333in}}%
\pgfpathlineto{\pgfqpoint{4.217521in}{1.441325in}}%
\pgfpathlineto{\pgfqpoint{4.212108in}{1.413333in}}%
\pgfpathclose%
\pgfusepath{fill}%
\end{pgfscope}%
\begin{pgfscope}%
\pgfpathrectangle{\pgfqpoint{3.156364in}{0.528000in}}{\pgfqpoint{1.963636in}{3.696000in}} %
\pgfusepath{clip}%
\pgfsetbuttcap%
\pgfsetroundjoin%
\definecolor{currentfill}{rgb}{0.050383,0.029803,0.527975}%
\pgfsetfillcolor{currentfill}%
\pgfsetlinewidth{0.000000pt}%
\definecolor{currentstroke}{rgb}{0.000000,0.000000,0.000000}%
\pgfsetstrokecolor{currentstroke}%
\pgfsetdash{}{0pt}%
\pgfpathmoveto{\pgfqpoint{4.167934in}{1.507841in}}%
\pgfpathlineto{\pgfqpoint{4.181420in}{1.520000in}}%
\pgfpathlineto{\pgfqpoint{4.167934in}{1.547611in}}%
\pgfpathlineto{\pgfqpoint{4.162097in}{1.520000in}}%
\pgfpathclose%
\pgfusepath{fill}%
\end{pgfscope}%
\begin{pgfscope}%
\pgfpathrectangle{\pgfqpoint{3.156364in}{0.528000in}}{\pgfqpoint{1.963636in}{3.696000in}} %
\pgfusepath{clip}%
\pgfsetbuttcap%
\pgfsetroundjoin%
\definecolor{currentfill}{rgb}{0.050383,0.029803,0.527975}%
\pgfsetfillcolor{currentfill}%
\pgfsetlinewidth{0.000000pt}%
\definecolor{currentstroke}{rgb}{0.000000,0.000000,0.000000}%
\pgfsetstrokecolor{currentstroke}%
\pgfsetdash{}{0pt}%
\pgfpathmoveto{\pgfqpoint{4.118347in}{1.613635in}}%
\pgfpathlineto{\pgfqpoint{4.131590in}{1.626667in}}%
\pgfpathlineto{\pgfqpoint{4.118347in}{1.653829in}}%
\pgfpathlineto{\pgfqpoint{4.112089in}{1.626667in}}%
\pgfpathclose%
\pgfusepath{fill}%
\end{pgfscope}%
\begin{pgfscope}%
\pgfpathrectangle{\pgfqpoint{3.156364in}{0.528000in}}{\pgfqpoint{1.963636in}{3.696000in}} %
\pgfusepath{clip}%
\pgfsetbuttcap%
\pgfsetroundjoin%
\definecolor{currentfill}{rgb}{0.050383,0.029803,0.527975}%
\pgfsetfillcolor{currentfill}%
\pgfsetlinewidth{0.000000pt}%
\definecolor{currentstroke}{rgb}{0.000000,0.000000,0.000000}%
\pgfsetstrokecolor{currentstroke}%
\pgfsetdash{}{0pt}%
\pgfpathmoveto{\pgfqpoint{4.068760in}{1.719435in}}%
\pgfpathlineto{\pgfqpoint{4.081735in}{1.733333in}}%
\pgfpathlineto{\pgfqpoint{4.068760in}{1.759988in}}%
\pgfpathlineto{\pgfqpoint{4.062083in}{1.733333in}}%
\pgfpathclose%
\pgfusepath{fill}%
\end{pgfscope}%
\begin{pgfscope}%
\pgfpathrectangle{\pgfqpoint{3.156364in}{0.528000in}}{\pgfqpoint{1.963636in}{3.696000in}} %
\pgfusepath{clip}%
\pgfsetbuttcap%
\pgfsetroundjoin%
\definecolor{currentfill}{rgb}{0.050383,0.029803,0.527975}%
\pgfsetfillcolor{currentfill}%
\pgfsetlinewidth{0.000000pt}%
\definecolor{currentstroke}{rgb}{0.000000,0.000000,0.000000}%
\pgfsetstrokecolor{currentstroke}%
\pgfsetdash{}{0pt}%
\pgfpathmoveto{\pgfqpoint{4.019174in}{1.825241in}}%
\pgfpathlineto{\pgfqpoint{4.031859in}{1.840000in}}%
\pgfpathlineto{\pgfqpoint{4.019174in}{1.866098in}}%
\pgfpathlineto{\pgfqpoint{4.012081in}{1.840000in}}%
\pgfpathclose%
\pgfpathmoveto{\pgfqpoint{4.019160in}{1.840000in}}%
\pgfpathlineto{\pgfqpoint{4.019174in}{1.840051in}}%
\pgfpathlineto{\pgfqpoint{4.019198in}{1.840000in}}%
\pgfpathlineto{\pgfqpoint{4.019174in}{1.839971in}}%
\pgfpathclose%
\pgfusepath{fill}%
\end{pgfscope}%
\begin{pgfscope}%
\pgfpathrectangle{\pgfqpoint{3.156364in}{0.528000in}}{\pgfqpoint{1.963636in}{3.696000in}} %
\pgfusepath{clip}%
\pgfsetbuttcap%
\pgfsetroundjoin%
\definecolor{currentfill}{rgb}{0.050383,0.029803,0.527975}%
\pgfsetfillcolor{currentfill}%
\pgfsetlinewidth{0.000000pt}%
\definecolor{currentstroke}{rgb}{0.000000,0.000000,0.000000}%
\pgfsetstrokecolor{currentstroke}%
\pgfsetdash{}{0pt}%
\pgfpathmoveto{\pgfqpoint{3.969587in}{1.931056in}}%
\pgfpathlineto{\pgfqpoint{3.981966in}{1.946667in}}%
\pgfpathlineto{\pgfqpoint{3.969587in}{1.972166in}}%
\pgfpathlineto{\pgfqpoint{3.962081in}{1.946667in}}%
\pgfpathclose%
\pgfpathmoveto{\pgfqpoint{3.969374in}{1.946667in}}%
\pgfpathlineto{\pgfqpoint{3.969587in}{1.947388in}}%
\pgfpathlineto{\pgfqpoint{3.969937in}{1.946667in}}%
\pgfpathlineto{\pgfqpoint{3.969587in}{1.946225in}}%
\pgfpathclose%
\pgfusepath{fill}%
\end{pgfscope}%
\begin{pgfscope}%
\pgfpathrectangle{\pgfqpoint{3.156364in}{0.528000in}}{\pgfqpoint{1.963636in}{3.696000in}} %
\pgfusepath{clip}%
\pgfsetbuttcap%
\pgfsetroundjoin%
\definecolor{currentfill}{rgb}{0.050383,0.029803,0.527975}%
\pgfsetfillcolor{currentfill}%
\pgfsetlinewidth{0.000000pt}%
\definecolor{currentstroke}{rgb}{0.000000,0.000000,0.000000}%
\pgfsetstrokecolor{currentstroke}%
\pgfsetdash{}{0pt}%
\pgfpathmoveto{\pgfqpoint{3.920000in}{2.036878in}}%
\pgfpathlineto{\pgfqpoint{3.932057in}{2.053333in}}%
\pgfpathlineto{\pgfqpoint{3.920000in}{2.078198in}}%
\pgfpathlineto{\pgfqpoint{3.912085in}{2.053333in}}%
\pgfpathclose%
\pgfpathmoveto{\pgfqpoint{3.919606in}{2.053333in}}%
\pgfpathlineto{\pgfqpoint{3.920000in}{2.054572in}}%
\pgfpathlineto{\pgfqpoint{3.920600in}{2.053333in}}%
\pgfpathlineto{\pgfqpoint{3.920000in}{2.052514in}}%
\pgfpathclose%
\pgfusepath{fill}%
\end{pgfscope}%
\begin{pgfscope}%
\pgfpathrectangle{\pgfqpoint{3.156364in}{0.528000in}}{\pgfqpoint{1.963636in}{3.696000in}} %
\pgfusepath{clip}%
\pgfsetbuttcap%
\pgfsetroundjoin%
\definecolor{currentfill}{rgb}{0.050383,0.029803,0.527975}%
\pgfsetfillcolor{currentfill}%
\pgfsetlinewidth{0.000000pt}%
\definecolor{currentstroke}{rgb}{0.000000,0.000000,0.000000}%
\pgfsetstrokecolor{currentstroke}%
\pgfsetdash{}{0pt}%
\pgfpathmoveto{\pgfqpoint{3.870413in}{2.142710in}}%
\pgfpathlineto{\pgfqpoint{3.882135in}{2.160000in}}%
\pgfpathlineto{\pgfqpoint{3.870413in}{2.184199in}}%
\pgfpathlineto{\pgfqpoint{3.862092in}{2.160000in}}%
\pgfpathclose%
\pgfpathmoveto{\pgfqpoint{3.869856in}{2.160000in}}%
\pgfpathlineto{\pgfqpoint{3.870413in}{2.161621in}}%
\pgfpathlineto{\pgfqpoint{3.871199in}{2.160000in}}%
\pgfpathlineto{\pgfqpoint{3.870413in}{2.158842in}}%
\pgfpathclose%
\pgfusepath{fill}%
\end{pgfscope}%
\begin{pgfscope}%
\pgfpathrectangle{\pgfqpoint{3.156364in}{0.528000in}}{\pgfqpoint{1.963636in}{3.696000in}} %
\pgfusepath{clip}%
\pgfsetbuttcap%
\pgfsetroundjoin%
\definecolor{currentfill}{rgb}{0.050383,0.029803,0.527975}%
\pgfsetfillcolor{currentfill}%
\pgfsetlinewidth{0.000000pt}%
\definecolor{currentstroke}{rgb}{0.000000,0.000000,0.000000}%
\pgfsetstrokecolor{currentstroke}%
\pgfsetdash{}{0pt}%
\pgfpathmoveto{\pgfqpoint{3.820826in}{2.248551in}}%
\pgfpathlineto{\pgfqpoint{3.832202in}{2.266667in}}%
\pgfpathlineto{\pgfqpoint{3.820826in}{2.290173in}}%
\pgfpathlineto{\pgfqpoint{3.812103in}{2.266667in}}%
\pgfpathclose%
\pgfpathmoveto{\pgfqpoint{3.820126in}{2.266667in}}%
\pgfpathlineto{\pgfqpoint{3.820826in}{2.268555in}}%
\pgfpathlineto{\pgfqpoint{3.821740in}{2.266667in}}%
\pgfpathlineto{\pgfqpoint{3.820826in}{2.265212in}}%
\pgfpathclose%
\pgfusepath{fill}%
\end{pgfscope}%
\begin{pgfscope}%
\pgfpathrectangle{\pgfqpoint{3.156364in}{0.528000in}}{\pgfqpoint{1.963636in}{3.696000in}} %
\pgfusepath{clip}%
\pgfsetbuttcap%
\pgfsetroundjoin%
\definecolor{currentfill}{rgb}{0.050383,0.029803,0.527975}%
\pgfsetfillcolor{currentfill}%
\pgfsetlinewidth{0.000000pt}%
\definecolor{currentstroke}{rgb}{0.000000,0.000000,0.000000}%
\pgfsetstrokecolor{currentstroke}%
\pgfsetdash{}{0pt}%
\pgfpathmoveto{\pgfqpoint{3.771240in}{2.354404in}}%
\pgfpathlineto{\pgfqpoint{3.782258in}{2.373333in}}%
\pgfpathlineto{\pgfqpoint{3.771240in}{2.396122in}}%
\pgfpathlineto{\pgfqpoint{3.762120in}{2.373333in}}%
\pgfpathclose%
\pgfpathmoveto{\pgfqpoint{3.770418in}{2.373333in}}%
\pgfpathlineto{\pgfqpoint{3.771240in}{2.375386in}}%
\pgfpathlineto{\pgfqpoint{3.772232in}{2.373333in}}%
\pgfpathlineto{\pgfqpoint{3.771240in}{2.371628in}}%
\pgfpathclose%
\pgfusepath{fill}%
\end{pgfscope}%
\begin{pgfscope}%
\pgfpathrectangle{\pgfqpoint{3.156364in}{0.528000in}}{\pgfqpoint{1.963636in}{3.696000in}} %
\pgfusepath{clip}%
\pgfsetbuttcap%
\pgfsetroundjoin%
\definecolor{currentfill}{rgb}{0.050383,0.029803,0.527975}%
\pgfsetfillcolor{currentfill}%
\pgfsetlinewidth{0.000000pt}%
\definecolor{currentstroke}{rgb}{0.000000,0.000000,0.000000}%
\pgfsetstrokecolor{currentstroke}%
\pgfsetdash{}{0pt}%
\pgfpathmoveto{\pgfqpoint{3.721653in}{2.460269in}}%
\pgfpathlineto{\pgfqpoint{3.732306in}{2.480000in}}%
\pgfpathlineto{\pgfqpoint{3.721653in}{2.502051in}}%
\pgfpathlineto{\pgfqpoint{3.712141in}{2.480000in}}%
\pgfpathclose%
\pgfpathmoveto{\pgfqpoint{3.720735in}{2.480000in}}%
\pgfpathlineto{\pgfqpoint{3.721653in}{2.482127in}}%
\pgfpathlineto{\pgfqpoint{3.722680in}{2.480000in}}%
\pgfpathlineto{\pgfqpoint{3.721653in}{2.478097in}}%
\pgfpathclose%
\pgfusepath{fill}%
\end{pgfscope}%
\begin{pgfscope}%
\pgfpathrectangle{\pgfqpoint{3.156364in}{0.528000in}}{\pgfqpoint{1.963636in}{3.696000in}} %
\pgfusepath{clip}%
\pgfsetbuttcap%
\pgfsetroundjoin%
\definecolor{currentfill}{rgb}{0.050383,0.029803,0.527975}%
\pgfsetfillcolor{currentfill}%
\pgfsetlinewidth{0.000000pt}%
\definecolor{currentstroke}{rgb}{0.000000,0.000000,0.000000}%
\pgfsetstrokecolor{currentstroke}%
\pgfsetdash{}{0pt}%
\pgfpathmoveto{\pgfqpoint{3.672066in}{2.566147in}}%
\pgfpathlineto{\pgfqpoint{3.682345in}{2.586667in}}%
\pgfpathlineto{\pgfqpoint{3.672066in}{2.607960in}}%
\pgfpathlineto{\pgfqpoint{3.662167in}{2.586667in}}%
\pgfpathclose%
\pgfpathmoveto{\pgfqpoint{3.671080in}{2.586667in}}%
\pgfpathlineto{\pgfqpoint{3.672066in}{2.588788in}}%
\pgfpathlineto{\pgfqpoint{3.673090in}{2.586667in}}%
\pgfpathlineto{\pgfqpoint{3.672066in}{2.584623in}}%
\pgfpathclose%
\pgfusepath{fill}%
\end{pgfscope}%
\begin{pgfscope}%
\pgfpathrectangle{\pgfqpoint{3.156364in}{0.528000in}}{\pgfqpoint{1.963636in}{3.696000in}} %
\pgfusepath{clip}%
\pgfsetbuttcap%
\pgfsetroundjoin%
\definecolor{currentfill}{rgb}{0.050383,0.029803,0.527975}%
\pgfsetfillcolor{currentfill}%
\pgfsetlinewidth{0.000000pt}%
\definecolor{currentstroke}{rgb}{0.000000,0.000000,0.000000}%
\pgfsetstrokecolor{currentstroke}%
\pgfsetdash{}{0pt}%
\pgfpathmoveto{\pgfqpoint{3.622479in}{2.672040in}}%
\pgfpathlineto{\pgfqpoint{3.632378in}{2.693333in}}%
\pgfpathlineto{\pgfqpoint{3.622479in}{2.713853in}}%
\pgfpathlineto{\pgfqpoint{3.612200in}{2.693333in}}%
\pgfpathclose%
\pgfpathmoveto{\pgfqpoint{3.621455in}{2.693333in}}%
\pgfpathlineto{\pgfqpoint{3.622479in}{2.695377in}}%
\pgfpathlineto{\pgfqpoint{3.623465in}{2.693333in}}%
\pgfpathlineto{\pgfqpoint{3.622479in}{2.691212in}}%
\pgfpathclose%
\pgfusepath{fill}%
\end{pgfscope}%
\begin{pgfscope}%
\pgfpathrectangle{\pgfqpoint{3.156364in}{0.528000in}}{\pgfqpoint{1.963636in}{3.696000in}} %
\pgfusepath{clip}%
\pgfsetbuttcap%
\pgfsetroundjoin%
\definecolor{currentfill}{rgb}{0.050383,0.029803,0.527975}%
\pgfsetfillcolor{currentfill}%
\pgfsetlinewidth{0.000000pt}%
\definecolor{currentstroke}{rgb}{0.000000,0.000000,0.000000}%
\pgfsetstrokecolor{currentstroke}%
\pgfsetdash{}{0pt}%
\pgfpathmoveto{\pgfqpoint{3.572893in}{2.777949in}}%
\pgfpathlineto{\pgfqpoint{3.582405in}{2.800000in}}%
\pgfpathlineto{\pgfqpoint{3.572893in}{2.819731in}}%
\pgfpathlineto{\pgfqpoint{3.562240in}{2.800000in}}%
\pgfpathclose%
\pgfpathmoveto{\pgfqpoint{3.571865in}{2.800000in}}%
\pgfpathlineto{\pgfqpoint{3.572893in}{2.801903in}}%
\pgfpathlineto{\pgfqpoint{3.573810in}{2.800000in}}%
\pgfpathlineto{\pgfqpoint{3.572893in}{2.797873in}}%
\pgfpathclose%
\pgfusepath{fill}%
\end{pgfscope}%
\begin{pgfscope}%
\pgfpathrectangle{\pgfqpoint{3.156364in}{0.528000in}}{\pgfqpoint{1.963636in}{3.696000in}} %
\pgfusepath{clip}%
\pgfsetbuttcap%
\pgfsetroundjoin%
\definecolor{currentfill}{rgb}{0.050383,0.029803,0.527975}%
\pgfsetfillcolor{currentfill}%
\pgfsetlinewidth{0.000000pt}%
\definecolor{currentstroke}{rgb}{0.000000,0.000000,0.000000}%
\pgfsetstrokecolor{currentstroke}%
\pgfsetdash{}{0pt}%
\pgfpathmoveto{\pgfqpoint{3.523306in}{2.883878in}}%
\pgfpathlineto{\pgfqpoint{3.532426in}{2.906667in}}%
\pgfpathlineto{\pgfqpoint{3.523306in}{2.925596in}}%
\pgfpathlineto{\pgfqpoint{3.512287in}{2.906667in}}%
\pgfpathclose%
\pgfpathmoveto{\pgfqpoint{3.522313in}{2.906667in}}%
\pgfpathlineto{\pgfqpoint{3.523306in}{2.908372in}}%
\pgfpathlineto{\pgfqpoint{3.524127in}{2.906667in}}%
\pgfpathlineto{\pgfqpoint{3.523306in}{2.904614in}}%
\pgfpathclose%
\pgfusepath{fill}%
\end{pgfscope}%
\begin{pgfscope}%
\pgfpathrectangle{\pgfqpoint{3.156364in}{0.528000in}}{\pgfqpoint{1.963636in}{3.696000in}} %
\pgfusepath{clip}%
\pgfsetbuttcap%
\pgfsetroundjoin%
\definecolor{currentfill}{rgb}{0.050383,0.029803,0.527975}%
\pgfsetfillcolor{currentfill}%
\pgfsetlinewidth{0.000000pt}%
\definecolor{currentstroke}{rgb}{0.000000,0.000000,0.000000}%
\pgfsetstrokecolor{currentstroke}%
\pgfsetdash{}{0pt}%
\pgfpathmoveto{\pgfqpoint{3.473719in}{2.989827in}}%
\pgfpathlineto{\pgfqpoint{3.482442in}{3.013333in}}%
\pgfpathlineto{\pgfqpoint{3.473719in}{3.031449in}}%
\pgfpathlineto{\pgfqpoint{3.462344in}{3.013333in}}%
\pgfpathclose%
\pgfpathmoveto{\pgfqpoint{3.472805in}{3.013333in}}%
\pgfpathlineto{\pgfqpoint{3.473719in}{3.014788in}}%
\pgfpathlineto{\pgfqpoint{3.474420in}{3.013333in}}%
\pgfpathlineto{\pgfqpoint{3.473719in}{3.011445in}}%
\pgfpathclose%
\pgfusepath{fill}%
\end{pgfscope}%
\begin{pgfscope}%
\pgfpathrectangle{\pgfqpoint{3.156364in}{0.528000in}}{\pgfqpoint{1.963636in}{3.696000in}} %
\pgfusepath{clip}%
\pgfsetbuttcap%
\pgfsetroundjoin%
\definecolor{currentfill}{rgb}{0.050383,0.029803,0.527975}%
\pgfsetfillcolor{currentfill}%
\pgfsetlinewidth{0.000000pt}%
\definecolor{currentstroke}{rgb}{0.000000,0.000000,0.000000}%
\pgfsetstrokecolor{currentstroke}%
\pgfsetdash{}{0pt}%
\pgfpathmoveto{\pgfqpoint{3.424132in}{3.095801in}}%
\pgfpathlineto{\pgfqpoint{3.432454in}{3.120000in}}%
\pgfpathlineto{\pgfqpoint{3.424132in}{3.137290in}}%
\pgfpathlineto{\pgfqpoint{3.412410in}{3.120000in}}%
\pgfpathclose%
\pgfpathmoveto{\pgfqpoint{3.423347in}{3.120000in}}%
\pgfpathlineto{\pgfqpoint{3.424132in}{3.121158in}}%
\pgfpathlineto{\pgfqpoint{3.424690in}{3.120000in}}%
\pgfpathlineto{\pgfqpoint{3.424132in}{3.118379in}}%
\pgfpathclose%
\pgfusepath{fill}%
\end{pgfscope}%
\begin{pgfscope}%
\pgfpathrectangle{\pgfqpoint{3.156364in}{0.528000in}}{\pgfqpoint{1.963636in}{3.696000in}} %
\pgfusepath{clip}%
\pgfsetbuttcap%
\pgfsetroundjoin%
\definecolor{currentfill}{rgb}{0.050383,0.029803,0.527975}%
\pgfsetfillcolor{currentfill}%
\pgfsetlinewidth{0.000000pt}%
\definecolor{currentstroke}{rgb}{0.000000,0.000000,0.000000}%
\pgfsetstrokecolor{currentstroke}%
\pgfsetdash{}{0pt}%
\pgfpathmoveto{\pgfqpoint{3.374545in}{3.201802in}}%
\pgfpathlineto{\pgfqpoint{3.382461in}{3.226667in}}%
\pgfpathlineto{\pgfqpoint{3.374545in}{3.243122in}}%
\pgfpathlineto{\pgfqpoint{3.362488in}{3.226667in}}%
\pgfpathclose%
\pgfpathmoveto{\pgfqpoint{3.373945in}{3.226667in}}%
\pgfpathlineto{\pgfqpoint{3.374545in}{3.227486in}}%
\pgfpathlineto{\pgfqpoint{3.374940in}{3.226667in}}%
\pgfpathlineto{\pgfqpoint{3.374545in}{3.225428in}}%
\pgfpathclose%
\pgfusepath{fill}%
\end{pgfscope}%
\begin{pgfscope}%
\pgfpathrectangle{\pgfqpoint{3.156364in}{0.528000in}}{\pgfqpoint{1.963636in}{3.696000in}} %
\pgfusepath{clip}%
\pgfsetbuttcap%
\pgfsetroundjoin%
\definecolor{currentfill}{rgb}{0.050383,0.029803,0.527975}%
\pgfsetfillcolor{currentfill}%
\pgfsetlinewidth{0.000000pt}%
\definecolor{currentstroke}{rgb}{0.000000,0.000000,0.000000}%
\pgfsetstrokecolor{currentstroke}%
\pgfsetdash{}{0pt}%
\pgfpathmoveto{\pgfqpoint{3.324959in}{3.307834in}}%
\pgfpathlineto{\pgfqpoint{3.332465in}{3.333333in}}%
\pgfpathlineto{\pgfqpoint{3.324959in}{3.348944in}}%
\pgfpathlineto{\pgfqpoint{3.312580in}{3.333333in}}%
\pgfpathclose%
\pgfpathmoveto{\pgfqpoint{3.324608in}{3.333333in}}%
\pgfpathlineto{\pgfqpoint{3.324959in}{3.333775in}}%
\pgfpathlineto{\pgfqpoint{3.325171in}{3.333333in}}%
\pgfpathlineto{\pgfqpoint{3.324959in}{3.332612in}}%
\pgfpathclose%
\pgfusepath{fill}%
\end{pgfscope}%
\begin{pgfscope}%
\pgfpathrectangle{\pgfqpoint{3.156364in}{0.528000in}}{\pgfqpoint{1.963636in}{3.696000in}} %
\pgfusepath{clip}%
\pgfsetbuttcap%
\pgfsetroundjoin%
\definecolor{currentfill}{rgb}{0.050383,0.029803,0.527975}%
\pgfsetfillcolor{currentfill}%
\pgfsetlinewidth{0.000000pt}%
\definecolor{currentstroke}{rgb}{0.000000,0.000000,0.000000}%
\pgfsetstrokecolor{currentstroke}%
\pgfsetdash{}{0pt}%
\pgfpathmoveto{\pgfqpoint{3.275372in}{3.413902in}}%
\pgfpathlineto{\pgfqpoint{3.282465in}{3.440000in}}%
\pgfpathlineto{\pgfqpoint{3.275372in}{3.454759in}}%
\pgfpathlineto{\pgfqpoint{3.262686in}{3.440000in}}%
\pgfpathclose%
\pgfpathmoveto{\pgfqpoint{3.275347in}{3.440000in}}%
\pgfpathlineto{\pgfqpoint{3.275372in}{3.440029in}}%
\pgfpathlineto{\pgfqpoint{3.275386in}{3.440000in}}%
\pgfpathlineto{\pgfqpoint{3.275372in}{3.439949in}}%
\pgfpathclose%
\pgfusepath{fill}%
\end{pgfscope}%
\begin{pgfscope}%
\pgfpathrectangle{\pgfqpoint{3.156364in}{0.528000in}}{\pgfqpoint{1.963636in}{3.696000in}} %
\pgfusepath{clip}%
\pgfsetbuttcap%
\pgfsetroundjoin%
\definecolor{currentfill}{rgb}{0.050383,0.029803,0.527975}%
\pgfsetfillcolor{currentfill}%
\pgfsetlinewidth{0.000000pt}%
\definecolor{currentstroke}{rgb}{0.000000,0.000000,0.000000}%
\pgfsetstrokecolor{currentstroke}%
\pgfsetdash{}{0pt}%
\pgfpathmoveto{\pgfqpoint{3.225785in}{3.520012in}}%
\pgfpathlineto{\pgfqpoint{3.232462in}{3.546667in}}%
\pgfpathlineto{\pgfqpoint{3.225785in}{3.560565in}}%
\pgfpathlineto{\pgfqpoint{3.212810in}{3.546667in}}%
\pgfpathclose%
\pgfusepath{fill}%
\end{pgfscope}%
\begin{pgfscope}%
\pgfpathrectangle{\pgfqpoint{3.156364in}{0.528000in}}{\pgfqpoint{1.963636in}{3.696000in}} %
\pgfusepath{clip}%
\pgfsetbuttcap%
\pgfsetroundjoin%
\definecolor{currentfill}{rgb}{0.050383,0.029803,0.527975}%
\pgfsetfillcolor{currentfill}%
\pgfsetlinewidth{0.000000pt}%
\definecolor{currentstroke}{rgb}{0.000000,0.000000,0.000000}%
\pgfsetstrokecolor{currentstroke}%
\pgfsetdash{}{0pt}%
\pgfpathmoveto{\pgfqpoint{3.176198in}{3.626171in}}%
\pgfpathlineto{\pgfqpoint{3.182456in}{3.653333in}}%
\pgfpathlineto{\pgfqpoint{3.176198in}{3.666365in}}%
\pgfpathlineto{\pgfqpoint{3.162956in}{3.653333in}}%
\pgfpathclose%
\pgfusepath{fill}%
\end{pgfscope}%
\begin{pgfscope}%
\pgfpathrectangle{\pgfqpoint{3.156364in}{0.528000in}}{\pgfqpoint{1.963636in}{3.696000in}} %
\pgfusepath{clip}%
\pgfsetbuttcap%
\pgfsetroundjoin%
\definecolor{currentfill}{rgb}{0.050383,0.029803,0.527975}%
\pgfsetfillcolor{currentfill}%
\pgfsetlinewidth{0.000000pt}%
\definecolor{currentstroke}{rgb}{0.000000,0.000000,0.000000}%
\pgfsetstrokecolor{currentstroke}%
\pgfsetdash{}{0pt}%
\pgfpathmoveto{\pgfqpoint{3.126612in}{3.732389in}}%
\pgfpathlineto{\pgfqpoint{3.132448in}{3.760000in}}%
\pgfpathlineto{\pgfqpoint{3.126612in}{3.772159in}}%
\pgfpathlineto{\pgfqpoint{3.113126in}{3.760000in}}%
\pgfpathclose%
\pgfusepath{fill}%
\end{pgfscope}%
\begin{pgfscope}%
\pgfpathrectangle{\pgfqpoint{3.156364in}{0.528000in}}{\pgfqpoint{1.963636in}{3.696000in}} %
\pgfusepath{clip}%
\pgfsetbuttcap%
\pgfsetroundjoin%
\definecolor{currentfill}{rgb}{0.050383,0.029803,0.527975}%
\pgfsetfillcolor{currentfill}%
\pgfsetlinewidth{0.000000pt}%
\definecolor{currentstroke}{rgb}{0.000000,0.000000,0.000000}%
\pgfsetstrokecolor{currentstroke}%
\pgfsetdash{}{0pt}%
\pgfpathmoveto{\pgfqpoint{3.077025in}{3.838675in}}%
\pgfpathlineto{\pgfqpoint{3.082437in}{3.866667in}}%
\pgfpathlineto{\pgfqpoint{3.077025in}{3.877946in}}%
\pgfpathlineto{\pgfqpoint{3.063326in}{3.866667in}}%
\pgfpathclose%
\pgfusepath{fill}%
\end{pgfscope}%
\begin{pgfscope}%
\pgfpathrectangle{\pgfqpoint{3.156364in}{0.528000in}}{\pgfqpoint{1.963636in}{3.696000in}} %
\pgfusepath{clip}%
\pgfsetbuttcap%
\pgfsetroundjoin%
\definecolor{currentfill}{rgb}{0.050383,0.029803,0.527975}%
\pgfsetfillcolor{currentfill}%
\pgfsetlinewidth{0.000000pt}%
\definecolor{currentstroke}{rgb}{0.000000,0.000000,0.000000}%
\pgfsetstrokecolor{currentstroke}%
\pgfsetdash{}{0pt}%
\pgfpathmoveto{\pgfqpoint{3.027438in}{3.945045in}}%
\pgfpathlineto{\pgfqpoint{3.032424in}{3.973333in}}%
\pgfpathlineto{\pgfqpoint{3.027438in}{3.983728in}}%
\pgfpathlineto{\pgfqpoint{3.013562in}{3.973333in}}%
\pgfpathclose%
\pgfusepath{fill}%
\end{pgfscope}%
\begin{pgfscope}%
\pgfpathrectangle{\pgfqpoint{3.156364in}{0.528000in}}{\pgfqpoint{1.963636in}{3.696000in}} %
\pgfusepath{clip}%
\pgfsetbuttcap%
\pgfsetroundjoin%
\definecolor{currentfill}{rgb}{0.050383,0.029803,0.527975}%
\pgfsetfillcolor{currentfill}%
\pgfsetlinewidth{0.000000pt}%
\definecolor{currentstroke}{rgb}{0.000000,0.000000,0.000000}%
\pgfsetstrokecolor{currentstroke}%
\pgfsetdash{}{0pt}%
\pgfpathmoveto{\pgfqpoint{2.977851in}{4.051517in}}%
\pgfpathlineto{\pgfqpoint{2.982409in}{4.080000in}}%
\pgfpathlineto{\pgfqpoint{2.977851in}{4.089505in}}%
\pgfpathlineto{\pgfqpoint{2.963843in}{4.080000in}}%
\pgfpathclose%
\pgfusepath{fill}%
\end{pgfscope}%
\begin{pgfscope}%
\pgfpathrectangle{\pgfqpoint{3.156364in}{0.528000in}}{\pgfqpoint{1.963636in}{3.696000in}} %
\pgfusepath{clip}%
\pgfsetbuttcap%
\pgfsetroundjoin%
\definecolor{currentfill}{rgb}{0.050383,0.029803,0.527975}%
\pgfsetfillcolor{currentfill}%
\pgfsetlinewidth{0.000000pt}%
\definecolor{currentstroke}{rgb}{0.000000,0.000000,0.000000}%
\pgfsetstrokecolor{currentstroke}%
\pgfsetdash{}{0pt}%
\pgfpathmoveto{\pgfqpoint{2.928264in}{4.158116in}}%
\pgfpathlineto{\pgfqpoint{2.932393in}{4.186667in}}%
\pgfpathlineto{\pgfqpoint{2.928264in}{4.195277in}}%
\pgfpathlineto{\pgfqpoint{2.914179in}{4.186667in}}%
\pgfpathclose%
\pgfusepath{fill}%
\end{pgfscope}%
\begin{pgfscope}%
\pgfpathrectangle{\pgfqpoint{3.156364in}{0.528000in}}{\pgfqpoint{1.963636in}{3.696000in}} %
\pgfusepath{clip}%
\pgfsetbuttcap%
\pgfsetroundjoin%
\definecolor{currentfill}{rgb}{0.050383,0.029803,0.527975}%
\pgfsetfillcolor{currentfill}%
\pgfsetlinewidth{0.000000pt}%
\definecolor{currentstroke}{rgb}{0.000000,0.000000,0.000000}%
\pgfsetstrokecolor{currentstroke}%
\pgfsetdash{}{0pt}%
\pgfpathmoveto{\pgfqpoint{2.878678in}{4.264874in}}%
\pgfpathlineto{\pgfqpoint{2.882374in}{4.293333in}}%
\pgfpathlineto{\pgfqpoint{2.878678in}{4.301045in}}%
\pgfpathlineto{\pgfqpoint{2.864587in}{4.293333in}}%
\pgfpathclose%
\pgfusepath{fill}%
\end{pgfscope}%
\begin{pgfscope}%
\pgfpathrectangle{\pgfqpoint{3.156364in}{0.528000in}}{\pgfqpoint{1.963636in}{3.696000in}} %
\pgfusepath{clip}%
\pgfsetbuttcap%
\pgfsetroundjoin%
\definecolor{currentfill}{rgb}{0.050383,0.029803,0.527975}%
\pgfsetfillcolor{currentfill}%
\pgfsetlinewidth{0.000000pt}%
\definecolor{currentstroke}{rgb}{0.000000,0.000000,0.000000}%
\pgfsetstrokecolor{currentstroke}%
\pgfsetdash{}{0pt}%
\pgfpathmoveto{\pgfqpoint{2.829091in}{4.371837in}}%
\pgfpathlineto{\pgfqpoint{2.832354in}{4.400000in}}%
\pgfpathlineto{\pgfqpoint{2.829091in}{4.406809in}}%
\pgfpathlineto{\pgfqpoint{2.815086in}{4.400000in}}%
\pgfpathclose%
\pgfusepath{fill}%
\end{pgfscope}%
\begin{pgfscope}%
\pgfpathrectangle{\pgfqpoint{3.156364in}{0.528000in}}{\pgfqpoint{1.963636in}{3.696000in}} %
\pgfusepath{clip}%
\pgfsetbuttcap%
\pgfsetroundjoin%
\definecolor{currentfill}{rgb}{0.050383,0.029803,0.527975}%
\pgfsetfillcolor{currentfill}%
\pgfsetlinewidth{0.000000pt}%
\definecolor{currentstroke}{rgb}{0.000000,0.000000,0.000000}%
\pgfsetstrokecolor{currentstroke}%
\pgfsetdash{}{0pt}%
\pgfpathmoveto{\pgfqpoint{2.779504in}{4.479069in}}%
\pgfpathlineto{\pgfqpoint{2.782332in}{4.506667in}}%
\pgfpathlineto{\pgfqpoint{2.779504in}{4.512570in}}%
\pgfpathlineto{\pgfqpoint{2.765709in}{4.506667in}}%
\pgfpathclose%
\pgfusepath{fill}%
\end{pgfscope}%
\begin{pgfscope}%
\pgfpathrectangle{\pgfqpoint{3.156364in}{0.528000in}}{\pgfqpoint{1.963636in}{3.696000in}} %
\pgfusepath{clip}%
\pgfsetbuttcap%
\pgfsetroundjoin%
\definecolor{currentfill}{rgb}{0.050383,0.029803,0.527975}%
\pgfsetfillcolor{currentfill}%
\pgfsetlinewidth{0.000000pt}%
\definecolor{currentstroke}{rgb}{0.000000,0.000000,0.000000}%
\pgfsetstrokecolor{currentstroke}%
\pgfsetdash{}{0pt}%
\pgfpathmoveto{\pgfqpoint{2.729917in}{4.586666in}}%
\pgfpathlineto{\pgfqpoint{2.732308in}{4.613333in}}%
\pgfpathlineto{\pgfqpoint{2.729917in}{4.618326in}}%
\pgfpathlineto{\pgfqpoint{2.716500in}{4.613333in}}%
\pgfpathclose%
\pgfusepath{fill}%
\end{pgfscope}%
\begin{pgfscope}%
\pgfpathrectangle{\pgfqpoint{3.156364in}{0.528000in}}{\pgfqpoint{1.963636in}{3.696000in}} %
\pgfusepath{clip}%
\pgfsetbuttcap%
\pgfsetroundjoin%
\definecolor{currentfill}{rgb}{0.050383,0.029803,0.527975}%
\pgfsetfillcolor{currentfill}%
\pgfsetlinewidth{0.000000pt}%
\definecolor{currentstroke}{rgb}{0.000000,0.000000,0.000000}%
\pgfsetstrokecolor{currentstroke}%
\pgfsetdash{}{0pt}%
\pgfpathmoveto{\pgfqpoint{2.680331in}{4.694773in}}%
\pgfpathlineto{\pgfqpoint{2.682284in}{4.720000in}}%
\pgfpathlineto{\pgfqpoint{2.680331in}{4.724079in}}%
\pgfpathlineto{\pgfqpoint{2.667531in}{4.720000in}}%
\pgfpathclose%
\pgfusepath{fill}%
\end{pgfscope}%
\begin{pgfscope}%
\pgfpathrectangle{\pgfqpoint{3.156364in}{0.528000in}}{\pgfqpoint{1.963636in}{3.696000in}} %
\pgfusepath{clip}%
\pgfsetbuttcap%
\pgfsetroundjoin%
\definecolor{currentfill}{rgb}{0.050383,0.029803,0.527975}%
\pgfsetfillcolor{currentfill}%
\pgfsetlinewidth{0.000000pt}%
\definecolor{currentstroke}{rgb}{0.000000,0.000000,0.000000}%
\pgfsetstrokecolor{currentstroke}%
\pgfsetdash{}{0pt}%
\pgfpathmoveto{\pgfqpoint{2.630744in}{4.803623in}}%
\pgfpathlineto{\pgfqpoint{2.632258in}{4.826667in}}%
\pgfpathlineto{\pgfqpoint{2.630744in}{4.829830in}}%
\pgfpathlineto{\pgfqpoint{2.618923in}{4.826667in}}%
\pgfpathclose%
\pgfusepath{fill}%
\end{pgfscope}%
\begin{pgfscope}%
\pgfpathrectangle{\pgfqpoint{3.156364in}{0.528000in}}{\pgfqpoint{1.963636in}{3.696000in}} %
\pgfusepath{clip}%
\pgfsetbuttcap%
\pgfsetroundjoin%
\definecolor{currentfill}{rgb}{0.050383,0.029803,0.527975}%
\pgfsetfillcolor{currentfill}%
\pgfsetlinewidth{0.000000pt}%
\definecolor{currentstroke}{rgb}{0.000000,0.000000,0.000000}%
\pgfsetstrokecolor{currentstroke}%
\pgfsetdash{}{0pt}%
\pgfpathmoveto{\pgfqpoint{2.581157in}{4.913622in}}%
\pgfpathlineto{\pgfqpoint{2.582231in}{4.933333in}}%
\pgfpathlineto{\pgfqpoint{2.581157in}{4.935577in}}%
\pgfpathlineto{\pgfqpoint{2.570891in}{4.933333in}}%
\pgfpathclose%
\pgfusepath{fill}%
\end{pgfscope}%
\begin{pgfscope}%
\pgfpathrectangle{\pgfqpoint{3.156364in}{0.528000in}}{\pgfqpoint{1.963636in}{3.696000in}} %
\pgfusepath{clip}%
\pgfsetbuttcap%
\pgfsetroundjoin%
\definecolor{currentfill}{rgb}{0.050383,0.029803,0.527975}%
\pgfsetfillcolor{currentfill}%
\pgfsetlinewidth{0.000000pt}%
\definecolor{currentstroke}{rgb}{0.000000,0.000000,0.000000}%
\pgfsetstrokecolor{currentstroke}%
\pgfsetdash{}{0pt}%
\pgfpathmoveto{\pgfqpoint{2.531570in}{5.025529in}}%
\pgfpathlineto{\pgfqpoint{2.532203in}{5.040000in}}%
\pgfpathlineto{\pgfqpoint{2.531570in}{5.041322in}}%
\pgfpathlineto{\pgfqpoint{2.523865in}{5.040000in}}%
\pgfpathclose%
\pgfusepath{fill}%
\end{pgfscope}%
\begin{pgfscope}%
\pgfpathrectangle{\pgfqpoint{3.156364in}{0.528000in}}{\pgfqpoint{1.963636in}{3.696000in}} %
\pgfusepath{clip}%
\pgfsetbuttcap%
\pgfsetroundjoin%
\definecolor{currentfill}{rgb}{0.050383,0.029803,0.527975}%
\pgfsetfillcolor{currentfill}%
\pgfsetlinewidth{0.000000pt}%
\definecolor{currentstroke}{rgb}{0.000000,0.000000,0.000000}%
\pgfsetstrokecolor{currentstroke}%
\pgfsetdash{}{0pt}%
\pgfpathmoveto{\pgfqpoint{2.481983in}{5.140973in}}%
\pgfpathlineto{\pgfqpoint{2.482173in}{5.146667in}}%
\pgfpathlineto{\pgfqpoint{2.481983in}{5.147064in}}%
\pgfpathlineto{\pgfqpoint{2.478841in}{5.146667in}}%
\pgfpathclose%
\pgfusepath{fill}%
\end{pgfscope}%
\begin{pgfscope}%
\pgfpathrectangle{\pgfqpoint{3.156364in}{0.528000in}}{\pgfqpoint{1.963636in}{3.696000in}} %
\pgfusepath{clip}%
\pgfsetbuttcap%
\pgfsetroundjoin%
\definecolor{currentfill}{rgb}{0.050383,0.029803,0.527975}%
\pgfsetfillcolor{currentfill}%
\pgfsetlinewidth{0.000000pt}%
\definecolor{currentstroke}{rgb}{0.000000,0.000000,0.000000}%
\pgfsetstrokecolor{currentstroke}%
\pgfsetdash{}{0pt}%
\pgfpathmoveto{\pgfqpoint{2.134876in}{5.786313in}}%
\pgfpathlineto{\pgfqpoint{2.137724in}{5.786667in}}%
\pgfpathlineto{\pgfqpoint{2.134876in}{5.791816in}}%
\pgfpathlineto{\pgfqpoint{2.134707in}{5.786667in}}%
\pgfpathclose%
\pgfusepath{fill}%
\end{pgfscope}%
\begin{pgfscope}%
\pgfpathrectangle{\pgfqpoint{3.156364in}{0.528000in}}{\pgfqpoint{1.963636in}{3.696000in}} %
\pgfusepath{clip}%
\pgfsetbuttcap%
\pgfsetroundjoin%
\definecolor{currentfill}{rgb}{0.050383,0.029803,0.527975}%
\pgfsetfillcolor{currentfill}%
\pgfsetlinewidth{0.000000pt}%
\definecolor{currentstroke}{rgb}{0.000000,0.000000,0.000000}%
\pgfsetstrokecolor{currentstroke}%
\pgfsetdash{}{0pt}%
\pgfpathmoveto{\pgfqpoint{2.085289in}{5.892055in}}%
\pgfpathlineto{\pgfqpoint{2.092839in}{5.893333in}}%
\pgfpathlineto{\pgfqpoint{2.085289in}{5.907492in}}%
\pgfpathlineto{\pgfqpoint{2.084677in}{5.893333in}}%
\pgfpathclose%
\pgfusepath{fill}%
\end{pgfscope}%
\begin{pgfscope}%
\pgfpathrectangle{\pgfqpoint{3.156364in}{0.528000in}}{\pgfqpoint{1.963636in}{3.696000in}} %
\pgfusepath{clip}%
\pgfsetbuttcap%
\pgfsetroundjoin%
\definecolor{currentfill}{rgb}{0.050383,0.029803,0.527975}%
\pgfsetfillcolor{currentfill}%
\pgfsetlinewidth{0.000000pt}%
\definecolor{currentstroke}{rgb}{0.000000,0.000000,0.000000}%
\pgfsetstrokecolor{currentstroke}%
\pgfsetdash{}{0pt}%
\pgfpathmoveto{\pgfqpoint{2.035702in}{5.997799in}}%
\pgfpathlineto{\pgfqpoint{2.045876in}{6.000000in}}%
\pgfpathlineto{\pgfqpoint{2.035702in}{6.019519in}}%
\pgfpathlineto{\pgfqpoint{2.034649in}{6.000000in}}%
\pgfpathclose%
\pgfusepath{fill}%
\end{pgfscope}%
\begin{pgfscope}%
\pgfpathrectangle{\pgfqpoint{3.156364in}{0.528000in}}{\pgfqpoint{1.963636in}{3.696000in}} %
\pgfusepath{clip}%
\pgfsetbuttcap%
\pgfsetroundjoin%
\definecolor{currentfill}{rgb}{0.050383,0.029803,0.527975}%
\pgfsetfillcolor{currentfill}%
\pgfsetlinewidth{0.000000pt}%
\definecolor{currentstroke}{rgb}{0.000000,0.000000,0.000000}%
\pgfsetstrokecolor{currentstroke}%
\pgfsetdash{}{0pt}%
\pgfpathmoveto{\pgfqpoint{1.986116in}{6.103546in}}%
\pgfpathlineto{\pgfqpoint{1.997879in}{6.106667in}}%
\pgfpathlineto{\pgfqpoint{1.986116in}{6.129585in}}%
\pgfpathlineto{\pgfqpoint{1.984622in}{6.106667in}}%
\pgfpathclose%
\pgfusepath{fill}%
\end{pgfscope}%
\begin{pgfscope}%
\pgfpathrectangle{\pgfqpoint{3.156364in}{0.528000in}}{\pgfqpoint{1.963636in}{3.696000in}} %
\pgfusepath{clip}%
\pgfsetbuttcap%
\pgfsetroundjoin%
\definecolor{currentfill}{rgb}{0.050383,0.029803,0.527975}%
\pgfsetfillcolor{currentfill}%
\pgfsetlinewidth{0.000000pt}%
\definecolor{currentstroke}{rgb}{0.000000,0.000000,0.000000}%
\pgfsetstrokecolor{currentstroke}%
\pgfsetdash{}{0pt}%
\pgfpathmoveto{\pgfqpoint{1.936529in}{6.209296in}}%
\pgfpathlineto{\pgfqpoint{1.949292in}{6.213333in}}%
\pgfpathlineto{\pgfqpoint{1.936529in}{6.238478in}}%
\pgfpathlineto{\pgfqpoint{1.934596in}{6.213333in}}%
\pgfpathclose%
\pgfusepath{fill}%
\end{pgfscope}%
\begin{pgfscope}%
\pgfpathrectangle{\pgfqpoint{3.156364in}{0.528000in}}{\pgfqpoint{1.963636in}{3.696000in}} %
\pgfusepath{clip}%
\pgfsetbuttcap%
\pgfsetroundjoin%
\definecolor{currentfill}{rgb}{0.050383,0.029803,0.527975}%
\pgfsetfillcolor{currentfill}%
\pgfsetlinewidth{0.000000pt}%
\definecolor{currentstroke}{rgb}{0.000000,0.000000,0.000000}%
\pgfsetstrokecolor{currentstroke}%
\pgfsetdash{}{0pt}%
\pgfpathmoveto{\pgfqpoint{1.886942in}{6.315050in}}%
\pgfpathlineto{\pgfqpoint{1.900337in}{6.320000in}}%
\pgfpathlineto{\pgfqpoint{1.886942in}{6.346613in}}%
\pgfpathlineto{\pgfqpoint{1.884571in}{6.320000in}}%
\pgfpathclose%
\pgfusepath{fill}%
\end{pgfscope}%
\begin{pgfscope}%
\pgfpathrectangle{\pgfqpoint{3.156364in}{0.528000in}}{\pgfqpoint{1.963636in}{3.696000in}} %
\pgfusepath{clip}%
\pgfsetbuttcap%
\pgfsetroundjoin%
\definecolor{currentfill}{rgb}{0.050383,0.029803,0.527975}%
\pgfsetfillcolor{currentfill}%
\pgfsetlinewidth{0.000000pt}%
\definecolor{currentstroke}{rgb}{0.000000,0.000000,0.000000}%
\pgfsetstrokecolor{currentstroke}%
\pgfsetdash{}{0pt}%
\pgfpathmoveto{\pgfqpoint{1.837355in}{6.420806in}}%
\pgfpathlineto{\pgfqpoint{1.851137in}{6.426667in}}%
\pgfpathlineto{\pgfqpoint{1.837355in}{6.454230in}}%
\pgfpathlineto{\pgfqpoint{1.834548in}{6.426667in}}%
\pgfpathclose%
\pgfusepath{fill}%
\end{pgfscope}%
\begin{pgfscope}%
\pgfpathrectangle{\pgfqpoint{3.156364in}{0.528000in}}{\pgfqpoint{1.963636in}{3.696000in}} %
\pgfusepath{clip}%
\pgfsetbuttcap%
\pgfsetroundjoin%
\definecolor{currentfill}{rgb}{0.050383,0.029803,0.527975}%
\pgfsetfillcolor{currentfill}%
\pgfsetlinewidth{0.000000pt}%
\definecolor{currentstroke}{rgb}{0.000000,0.000000,0.000000}%
\pgfsetstrokecolor{currentstroke}%
\pgfsetdash{}{0pt}%
\pgfpathmoveto{\pgfqpoint{1.787769in}{6.526566in}}%
\pgfpathlineto{\pgfqpoint{1.801766in}{6.533333in}}%
\pgfpathlineto{\pgfqpoint{1.787769in}{6.561477in}}%
\pgfpathlineto{\pgfqpoint{1.784526in}{6.533333in}}%
\pgfpathclose%
\pgfusepath{fill}%
\end{pgfscope}%
\begin{pgfscope}%
\pgfpathrectangle{\pgfqpoint{3.156364in}{0.528000in}}{\pgfqpoint{1.963636in}{3.696000in}} %
\pgfusepath{clip}%
\pgfsetbuttcap%
\pgfsetroundjoin%
\definecolor{currentfill}{rgb}{0.050383,0.029803,0.527975}%
\pgfsetfillcolor{currentfill}%
\pgfsetlinewidth{0.000000pt}%
\definecolor{currentstroke}{rgb}{0.000000,0.000000,0.000000}%
\pgfsetstrokecolor{currentstroke}%
\pgfsetdash{}{0pt}%
\pgfpathmoveto{\pgfqpoint{1.738182in}{6.632330in}}%
\pgfpathlineto{\pgfqpoint{1.752271in}{6.640000in}}%
\pgfpathlineto{\pgfqpoint{1.738182in}{6.668451in}}%
\pgfpathlineto{\pgfqpoint{1.734506in}{6.640000in}}%
\pgfpathclose%
\pgfusepath{fill}%
\end{pgfscope}%
\begin{pgfscope}%
\pgfpathrectangle{\pgfqpoint{3.156364in}{0.528000in}}{\pgfqpoint{1.963636in}{3.696000in}} %
\pgfusepath{clip}%
\pgfsetbuttcap%
\pgfsetroundjoin%
\definecolor{currentfill}{rgb}{0.050383,0.029803,0.527975}%
\pgfsetfillcolor{currentfill}%
\pgfsetlinewidth{0.000000pt}%
\definecolor{currentstroke}{rgb}{0.000000,0.000000,0.000000}%
\pgfsetstrokecolor{currentstroke}%
\pgfsetdash{}{0pt}%
\pgfpathmoveto{\pgfqpoint{1.688595in}{6.738098in}}%
\pgfpathlineto{\pgfqpoint{1.702682in}{6.746667in}}%
\pgfpathlineto{\pgfqpoint{1.688595in}{6.775217in}}%
\pgfpathlineto{\pgfqpoint{1.684487in}{6.746667in}}%
\pgfpathclose%
\pgfusepath{fill}%
\end{pgfscope}%
\begin{pgfscope}%
\pgfpathrectangle{\pgfqpoint{3.156364in}{0.528000in}}{\pgfqpoint{1.963636in}{3.696000in}} %
\pgfusepath{clip}%
\pgfsetbuttcap%
\pgfsetroundjoin%
\definecolor{currentfill}{rgb}{0.050383,0.029803,0.527975}%
\pgfsetfillcolor{currentfill}%
\pgfsetlinewidth{0.000000pt}%
\definecolor{currentstroke}{rgb}{0.000000,0.000000,0.000000}%
\pgfsetstrokecolor{currentstroke}%
\pgfsetdash{}{0pt}%
\pgfpathmoveto{\pgfqpoint{1.639008in}{6.843870in}}%
\pgfpathlineto{\pgfqpoint{1.653022in}{6.853333in}}%
\pgfpathlineto{\pgfqpoint{1.639008in}{6.881822in}}%
\pgfpathlineto{\pgfqpoint{1.634470in}{6.853333in}}%
\pgfpathclose%
\pgfusepath{fill}%
\end{pgfscope}%
\begin{pgfscope}%
\pgfpathrectangle{\pgfqpoint{3.156364in}{0.528000in}}{\pgfqpoint{1.963636in}{3.696000in}} %
\pgfusepath{clip}%
\pgfsetbuttcap%
\pgfsetroundjoin%
\definecolor{currentfill}{rgb}{0.050383,0.029803,0.527975}%
\pgfsetfillcolor{currentfill}%
\pgfsetlinewidth{0.000000pt}%
\definecolor{currentstroke}{rgb}{0.000000,0.000000,0.000000}%
\pgfsetstrokecolor{currentstroke}%
\pgfsetdash{}{0pt}%
\pgfpathmoveto{\pgfqpoint{1.589421in}{6.949646in}}%
\pgfpathlineto{\pgfqpoint{1.603305in}{6.960000in}}%
\pgfpathlineto{\pgfqpoint{1.589421in}{6.988300in}}%
\pgfpathlineto{\pgfqpoint{1.584455in}{6.960000in}}%
\pgfpathclose%
\pgfusepath{fill}%
\end{pgfscope}%
\begin{pgfscope}%
\pgfpathrectangle{\pgfqpoint{3.156364in}{0.528000in}}{\pgfqpoint{1.963636in}{3.696000in}} %
\pgfusepath{clip}%
\pgfsetbuttcap%
\pgfsetroundjoin%
\definecolor{currentfill}{rgb}{0.050383,0.029803,0.527975}%
\pgfsetfillcolor{currentfill}%
\pgfsetlinewidth{0.000000pt}%
\definecolor{currentstroke}{rgb}{0.000000,0.000000,0.000000}%
\pgfsetstrokecolor{currentstroke}%
\pgfsetdash{}{0pt}%
\pgfpathmoveto{\pgfqpoint{1.539835in}{7.055428in}}%
\pgfpathlineto{\pgfqpoint{1.553543in}{7.066667in}}%
\pgfpathlineto{\pgfqpoint{1.539835in}{7.094674in}}%
\pgfpathlineto{\pgfqpoint{1.534442in}{7.066667in}}%
\pgfpathclose%
\pgfusepath{fill}%
\end{pgfscope}%
\begin{pgfscope}%
\pgfpathrectangle{\pgfqpoint{3.156364in}{0.528000in}}{\pgfqpoint{1.963636in}{3.696000in}} %
\pgfusepath{clip}%
\pgfsetbuttcap%
\pgfsetroundjoin%
\definecolor{currentfill}{rgb}{0.050383,0.029803,0.527975}%
\pgfsetfillcolor{currentfill}%
\pgfsetlinewidth{0.000000pt}%
\definecolor{currentstroke}{rgb}{0.000000,0.000000,0.000000}%
\pgfsetstrokecolor{currentstroke}%
\pgfsetdash{}{0pt}%
\pgfpathmoveto{\pgfqpoint{1.490248in}{7.161216in}}%
\pgfpathlineto{\pgfqpoint{1.503744in}{7.173333in}}%
\pgfpathlineto{\pgfqpoint{1.490248in}{7.200964in}}%
\pgfpathlineto{\pgfqpoint{1.484431in}{7.173333in}}%
\pgfpathclose%
\pgfusepath{fill}%
\end{pgfscope}%
\begin{pgfscope}%
\pgfpathrectangle{\pgfqpoint{3.156364in}{0.528000in}}{\pgfqpoint{1.963636in}{3.696000in}} %
\pgfusepath{clip}%
\pgfsetbuttcap%
\pgfsetroundjoin%
\definecolor{currentfill}{rgb}{0.050383,0.029803,0.527975}%
\pgfsetfillcolor{currentfill}%
\pgfsetlinewidth{0.000000pt}%
\definecolor{currentstroke}{rgb}{0.000000,0.000000,0.000000}%
\pgfsetstrokecolor{currentstroke}%
\pgfsetdash{}{0pt}%
\pgfpathmoveto{\pgfqpoint{1.440661in}{7.267009in}}%
\pgfpathlineto{\pgfqpoint{1.453916in}{7.280000in}}%
\pgfpathlineto{\pgfqpoint{1.440661in}{7.307184in}}%
\pgfpathlineto{\pgfqpoint{1.434423in}{7.280000in}}%
\pgfpathclose%
\pgfusepath{fill}%
\end{pgfscope}%
\begin{pgfscope}%
\pgfpathrectangle{\pgfqpoint{3.156364in}{0.528000in}}{\pgfqpoint{1.963636in}{3.696000in}} %
\pgfusepath{clip}%
\pgfsetbuttcap%
\pgfsetroundjoin%
\definecolor{currentfill}{rgb}{0.050383,0.029803,0.527975}%
\pgfsetfillcolor{currentfill}%
\pgfsetlinewidth{0.000000pt}%
\definecolor{currentstroke}{rgb}{0.000000,0.000000,0.000000}%
\pgfsetstrokecolor{currentstroke}%
\pgfsetdash{}{0pt}%
\pgfpathmoveto{\pgfqpoint{1.391074in}{7.372808in}}%
\pgfpathlineto{\pgfqpoint{1.404062in}{7.386667in}}%
\pgfpathlineto{\pgfqpoint{1.391074in}{7.413346in}}%
\pgfpathlineto{\pgfqpoint{1.384417in}{7.386667in}}%
\pgfpathclose%
\pgfusepath{fill}%
\end{pgfscope}%
\begin{pgfscope}%
\pgfpathrectangle{\pgfqpoint{3.156364in}{0.528000in}}{\pgfqpoint{1.963636in}{3.696000in}} %
\pgfusepath{clip}%
\pgfsetbuttcap%
\pgfsetroundjoin%
\definecolor{currentfill}{rgb}{0.050383,0.029803,0.527975}%
\pgfsetfillcolor{currentfill}%
\pgfsetlinewidth{0.000000pt}%
\definecolor{currentstroke}{rgb}{0.000000,0.000000,0.000000}%
\pgfsetstrokecolor{currentstroke}%
\pgfsetdash{}{0pt}%
\pgfpathmoveto{\pgfqpoint{1.341488in}{7.478615in}}%
\pgfpathlineto{\pgfqpoint{1.354187in}{7.493333in}}%
\pgfpathlineto{\pgfqpoint{1.341488in}{7.519458in}}%
\pgfpathlineto{\pgfqpoint{1.334414in}{7.493333in}}%
\pgfpathclose%
\pgfpathmoveto{\pgfqpoint{1.341483in}{7.493333in}}%
\pgfpathlineto{\pgfqpoint{1.341488in}{7.493349in}}%
\pgfpathlineto{\pgfqpoint{1.341495in}{7.493333in}}%
\pgfpathlineto{\pgfqpoint{1.341488in}{7.493325in}}%
\pgfpathclose%
\pgfusepath{fill}%
\end{pgfscope}%
\begin{pgfscope}%
\pgfpathrectangle{\pgfqpoint{3.156364in}{0.528000in}}{\pgfqpoint{1.963636in}{3.696000in}} %
\pgfusepath{clip}%
\pgfsetbuttcap%
\pgfsetroundjoin%
\definecolor{currentfill}{rgb}{0.050383,0.029803,0.527975}%
\pgfsetfillcolor{currentfill}%
\pgfsetlinewidth{0.000000pt}%
\definecolor{currentstroke}{rgb}{0.000000,0.000000,0.000000}%
\pgfsetstrokecolor{currentstroke}%
\pgfsetdash{}{0pt}%
\pgfpathmoveto{\pgfqpoint{1.291901in}{7.584428in}}%
\pgfpathlineto{\pgfqpoint{1.304295in}{7.600000in}}%
\pgfpathlineto{\pgfqpoint{1.291901in}{7.625528in}}%
\pgfpathlineto{\pgfqpoint{1.284414in}{7.600000in}}%
\pgfpathclose%
\pgfpathmoveto{\pgfqpoint{1.291697in}{7.600000in}}%
\pgfpathlineto{\pgfqpoint{1.291901in}{7.600694in}}%
\pgfpathlineto{\pgfqpoint{1.292238in}{7.600000in}}%
\pgfpathlineto{\pgfqpoint{1.291901in}{7.599577in}}%
\pgfpathclose%
\pgfusepath{fill}%
\end{pgfscope}%
\begin{pgfscope}%
\pgfpathrectangle{\pgfqpoint{3.156364in}{0.528000in}}{\pgfqpoint{1.963636in}{3.696000in}} %
\pgfusepath{clip}%
\pgfsetbuttcap%
\pgfsetroundjoin%
\definecolor{currentfill}{rgb}{0.050383,0.029803,0.527975}%
\pgfsetfillcolor{currentfill}%
\pgfsetlinewidth{0.000000pt}%
\definecolor{currentstroke}{rgb}{0.000000,0.000000,0.000000}%
\pgfsetstrokecolor{currentstroke}%
\pgfsetdash{}{0pt}%
\pgfpathmoveto{\pgfqpoint{1.242314in}{7.690251in}}%
\pgfpathlineto{\pgfqpoint{1.254386in}{7.706667in}}%
\pgfpathlineto{\pgfqpoint{1.242314in}{7.731562in}}%
\pgfpathlineto{\pgfqpoint{1.234418in}{7.706667in}}%
\pgfpathclose%
\pgfpathmoveto{\pgfqpoint{1.241928in}{7.706667in}}%
\pgfpathlineto{\pgfqpoint{1.242314in}{7.707884in}}%
\pgfpathlineto{\pgfqpoint{1.242904in}{7.706667in}}%
\pgfpathlineto{\pgfqpoint{1.242314in}{7.705864in}}%
\pgfpathclose%
\pgfusepath{fill}%
\end{pgfscope}%
\begin{pgfscope}%
\pgfpathrectangle{\pgfqpoint{3.156364in}{0.528000in}}{\pgfqpoint{1.963636in}{3.696000in}} %
\pgfusepath{clip}%
\pgfsetbuttcap%
\pgfsetroundjoin%
\definecolor{currentfill}{rgb}{0.050383,0.029803,0.527975}%
\pgfsetfillcolor{currentfill}%
\pgfsetlinewidth{0.000000pt}%
\definecolor{currentstroke}{rgb}{0.000000,0.000000,0.000000}%
\pgfsetstrokecolor{currentstroke}%
\pgfsetdash{}{0pt}%
\pgfpathmoveto{\pgfqpoint{1.204465in}{7.813333in}}%
\pgfpathlineto{\pgfqpoint{1.192727in}{7.837564in}}%
\pgfpathlineto{\pgfqpoint{1.192727in}{7.814940in}}%
\pgfpathlineto{\pgfqpoint{1.193505in}{7.813333in}}%
\pgfpathlineto{\pgfqpoint{1.192727in}{7.812190in}}%
\pgfpathlineto{\pgfqpoint{1.192727in}{7.796082in}}%
\pgfpathclose%
\pgfusepath{fill}%
\end{pgfscope}%
\begin{pgfscope}%
\pgfpathrectangle{\pgfqpoint{3.156364in}{0.528000in}}{\pgfqpoint{1.963636in}{3.696000in}} %
\pgfusepath{clip}%
\pgfsetbuttcap%
\pgfsetroundjoin%
\definecolor{currentfill}{rgb}{0.050383,0.029803,0.527975}%
\pgfsetfillcolor{currentfill}%
\pgfsetlinewidth{0.000000pt}%
\definecolor{currentstroke}{rgb}{0.000000,0.000000,0.000000}%
\pgfsetstrokecolor{currentstroke}%
\pgfsetdash{}{0pt}%
\pgfpathmoveto{\pgfqpoint{6.052231in}{-2.577413in}}%
\pgfpathlineto{\pgfqpoint{6.081680in}{-2.640000in}}%
\pgfpathlineto{\pgfqpoint{6.101818in}{-2.640000in}}%
\pgfpathlineto{\pgfqpoint{6.101818in}{-2.557564in}}%
\pgfpathlineto{\pgfqpoint{6.090080in}{-2.533333in}}%
\pgfpathlineto{\pgfqpoint{6.101818in}{-2.516082in}}%
\pgfpathlineto{\pgfqpoint{6.101818in}{-2.426667in}}%
\pgfpathlineto{\pgfqpoint{6.101818in}{-2.400242in}}%
\pgfpathlineto{\pgfqpoint{6.064033in}{-2.320000in}}%
\pgfpathlineto{\pgfqpoint{6.052231in}{-2.295227in}}%
\pgfpathlineto{\pgfqpoint{6.013666in}{-2.213333in}}%
\pgfpathlineto{\pgfqpoint{6.002645in}{-2.190196in}}%
\pgfpathlineto{\pgfqpoint{5.963306in}{-2.106667in}}%
\pgfpathlineto{\pgfqpoint{5.953058in}{-2.085151in}}%
\pgfpathlineto{\pgfqpoint{5.912953in}{-2.000000in}}%
\pgfpathlineto{\pgfqpoint{5.903471in}{-1.980092in}}%
\pgfpathlineto{\pgfqpoint{5.862607in}{-1.893333in}}%
\pgfpathlineto{\pgfqpoint{5.853884in}{-1.875020in}}%
\pgfpathlineto{\pgfqpoint{5.812267in}{-1.786667in}}%
\pgfpathlineto{\pgfqpoint{5.804298in}{-1.769933in}}%
\pgfpathlineto{\pgfqpoint{5.761933in}{-1.680000in}}%
\pgfpathlineto{\pgfqpoint{5.754711in}{-1.664835in}}%
\pgfpathlineto{\pgfqpoint{5.711605in}{-1.573333in}}%
\pgfpathlineto{\pgfqpoint{5.705124in}{-1.559723in}}%
\pgfpathlineto{\pgfqpoint{5.661283in}{-1.466667in}}%
\pgfpathlineto{\pgfqpoint{5.655537in}{-1.454600in}}%
\pgfpathlineto{\pgfqpoint{5.610966in}{-1.360000in}}%
\pgfpathlineto{\pgfqpoint{5.605950in}{-1.349465in}}%
\pgfpathlineto{\pgfqpoint{5.560656in}{-1.253333in}}%
\pgfpathlineto{\pgfqpoint{5.556364in}{-1.244319in}}%
\pgfpathlineto{\pgfqpoint{5.510350in}{-1.146667in}}%
\pgfpathlineto{\pgfqpoint{5.506777in}{-1.139162in}}%
\pgfpathlineto{\pgfqpoint{5.460049in}{-1.040000in}}%
\pgfpathlineto{\pgfqpoint{5.457190in}{-1.033994in}}%
\pgfpathlineto{\pgfqpoint{5.409754in}{-0.933333in}}%
\pgfpathlineto{\pgfqpoint{5.407603in}{-0.928816in}}%
\pgfpathlineto{\pgfqpoint{5.359463in}{-0.826667in}}%
\pgfpathlineto{\pgfqpoint{5.358017in}{-0.823628in}}%
\pgfpathlineto{\pgfqpoint{5.309177in}{-0.720000in}}%
\pgfpathlineto{\pgfqpoint{5.308430in}{-0.718430in}}%
\pgfpathlineto{\pgfqpoint{5.258896in}{-0.613333in}}%
\pgfpathlineto{\pgfqpoint{5.258843in}{-0.613223in}}%
\pgfpathlineto{\pgfqpoint{5.209256in}{-0.509468in}}%
\pgfpathlineto{\pgfqpoint{5.207943in}{-0.506667in}}%
\pgfpathlineto{\pgfqpoint{5.159669in}{-0.405754in}}%
\pgfpathlineto{\pgfqpoint{5.156971in}{-0.400000in}}%
\pgfpathlineto{\pgfqpoint{5.110083in}{-0.301965in}}%
\pgfpathlineto{\pgfqpoint{5.106034in}{-0.293333in}}%
\pgfpathlineto{\pgfqpoint{5.060496in}{-0.198103in}}%
\pgfpathlineto{\pgfqpoint{5.055130in}{-0.186667in}}%
\pgfpathlineto{\pgfqpoint{5.010909in}{-0.094174in}}%
\pgfpathlineto{\pgfqpoint{5.004258in}{-0.080000in}}%
\pgfpathlineto{\pgfqpoint{4.961322in}{0.009820in}}%
\pgfpathlineto{\pgfqpoint{4.953415in}{0.026667in}}%
\pgfpathlineto{\pgfqpoint{4.911736in}{0.113874in}}%
\pgfpathlineto{\pgfqpoint{4.902601in}{0.133333in}}%
\pgfpathlineto{\pgfqpoint{4.862149in}{0.217986in}}%
\pgfpathlineto{\pgfqpoint{4.851813in}{0.240000in}}%
\pgfpathlineto{\pgfqpoint{4.812562in}{0.322152in}}%
\pgfpathlineto{\pgfqpoint{4.801051in}{0.346667in}}%
\pgfpathlineto{\pgfqpoint{4.762975in}{0.426370in}}%
\pgfpathlineto{\pgfqpoint{4.750312in}{0.453333in}}%
\pgfpathlineto{\pgfqpoint{4.713388in}{0.530637in}}%
\pgfpathlineto{\pgfqpoint{4.699597in}{0.560000in}}%
\pgfpathlineto{\pgfqpoint{4.663802in}{0.634950in}}%
\pgfpathlineto{\pgfqpoint{4.648903in}{0.666667in}}%
\pgfpathlineto{\pgfqpoint{4.614215in}{0.739308in}}%
\pgfpathlineto{\pgfqpoint{4.598229in}{0.773333in}}%
\pgfpathlineto{\pgfqpoint{4.564628in}{0.843709in}}%
\pgfpathlineto{\pgfqpoint{4.547575in}{0.880000in}}%
\pgfpathlineto{\pgfqpoint{4.515041in}{0.948149in}}%
\pgfpathlineto{\pgfqpoint{4.496940in}{0.986667in}}%
\pgfpathlineto{\pgfqpoint{4.465455in}{1.052628in}}%
\pgfpathlineto{\pgfqpoint{4.446323in}{1.093333in}}%
\pgfpathlineto{\pgfqpoint{4.415868in}{1.157143in}}%
\pgfpathlineto{\pgfqpoint{4.395723in}{1.200000in}}%
\pgfpathlineto{\pgfqpoint{4.366281in}{1.261694in}}%
\pgfpathlineto{\pgfqpoint{4.345139in}{1.306667in}}%
\pgfpathlineto{\pgfqpoint{4.316694in}{1.366278in}}%
\pgfpathlineto{\pgfqpoint{4.294571in}{1.413333in}}%
\pgfpathlineto{\pgfqpoint{4.267107in}{1.470894in}}%
\pgfpathlineto{\pgfqpoint{4.244017in}{1.520000in}}%
\pgfpathlineto{\pgfqpoint{4.217521in}{1.575540in}}%
\pgfpathlineto{\pgfqpoint{4.193478in}{1.626667in}}%
\pgfpathlineto{\pgfqpoint{4.167934in}{1.680216in}}%
\pgfpathlineto{\pgfqpoint{4.142952in}{1.733333in}}%
\pgfpathlineto{\pgfqpoint{4.118347in}{1.784920in}}%
\pgfpathlineto{\pgfqpoint{4.092440in}{1.840000in}}%
\pgfpathlineto{\pgfqpoint{4.068760in}{1.889651in}}%
\pgfpathlineto{\pgfqpoint{4.041940in}{1.946667in}}%
\pgfpathlineto{\pgfqpoint{4.019174in}{1.994407in}}%
\pgfpathlineto{\pgfqpoint{3.991452in}{2.053333in}}%
\pgfpathlineto{\pgfqpoint{3.969587in}{2.099188in}}%
\pgfpathlineto{\pgfqpoint{3.940976in}{2.160000in}}%
\pgfpathlineto{\pgfqpoint{3.920000in}{2.203993in}}%
\pgfpathlineto{\pgfqpoint{3.890511in}{2.266667in}}%
\pgfpathlineto{\pgfqpoint{3.870413in}{2.308821in}}%
\pgfpathlineto{\pgfqpoint{3.840056in}{2.373333in}}%
\pgfpathlineto{\pgfqpoint{3.820826in}{2.413671in}}%
\pgfpathlineto{\pgfqpoint{3.789612in}{2.480000in}}%
\pgfpathlineto{\pgfqpoint{3.771240in}{2.518542in}}%
\pgfpathlineto{\pgfqpoint{3.739177in}{2.586667in}}%
\pgfpathlineto{\pgfqpoint{3.721653in}{2.623434in}}%
\pgfpathlineto{\pgfqpoint{3.688752in}{2.693333in}}%
\pgfpathlineto{\pgfqpoint{3.672066in}{2.728345in}}%
\pgfpathlineto{\pgfqpoint{3.638337in}{2.800000in}}%
\pgfpathlineto{\pgfqpoint{3.622479in}{2.833274in}}%
\pgfpathlineto{\pgfqpoint{3.587930in}{2.906667in}}%
\pgfpathlineto{\pgfqpoint{3.572893in}{2.938222in}}%
\pgfpathlineto{\pgfqpoint{3.537531in}{3.013333in}}%
\pgfpathlineto{\pgfqpoint{3.523306in}{3.043188in}}%
\pgfpathlineto{\pgfqpoint{3.487141in}{3.120000in}}%
\pgfpathlineto{\pgfqpoint{3.473719in}{3.148170in}}%
\pgfpathlineto{\pgfqpoint{3.436758in}{3.226667in}}%
\pgfpathlineto{\pgfqpoint{3.424132in}{3.253169in}}%
\pgfpathlineto{\pgfqpoint{3.386384in}{3.333333in}}%
\pgfpathlineto{\pgfqpoint{3.374545in}{3.358183in}}%
\pgfpathlineto{\pgfqpoint{3.336016in}{3.440000in}}%
\pgfpathlineto{\pgfqpoint{3.324959in}{3.463213in}}%
\pgfpathlineto{\pgfqpoint{3.285656in}{3.546667in}}%
\pgfpathlineto{\pgfqpoint{3.275372in}{3.568257in}}%
\pgfpathlineto{\pgfqpoint{3.235303in}{3.653333in}}%
\pgfpathlineto{\pgfqpoint{3.225785in}{3.673315in}}%
\pgfpathlineto{\pgfqpoint{3.184956in}{3.760000in}}%
\pgfpathlineto{\pgfqpoint{3.176198in}{3.778388in}}%
\pgfpathlineto{\pgfqpoint{3.134615in}{3.866667in}}%
\pgfpathlineto{\pgfqpoint{3.126612in}{3.883473in}}%
\pgfpathlineto{\pgfqpoint{3.084281in}{3.973333in}}%
\pgfpathlineto{\pgfqpoint{3.077025in}{3.988571in}}%
\pgfpathlineto{\pgfqpoint{3.033953in}{4.080000in}}%
\pgfpathlineto{\pgfqpoint{3.027438in}{4.093682in}}%
\pgfpathlineto{\pgfqpoint{2.983631in}{4.186667in}}%
\pgfpathlineto{\pgfqpoint{2.977851in}{4.198805in}}%
\pgfpathlineto{\pgfqpoint{2.933314in}{4.293333in}}%
\pgfpathlineto{\pgfqpoint{2.928264in}{4.303939in}}%
\pgfpathlineto{\pgfqpoint{2.883003in}{4.400000in}}%
\pgfpathlineto{\pgfqpoint{2.878678in}{4.409085in}}%
\pgfpathlineto{\pgfqpoint{2.832697in}{4.506667in}}%
\pgfpathlineto{\pgfqpoint{2.829091in}{4.514242in}}%
\pgfpathlineto{\pgfqpoint{2.782397in}{4.613333in}}%
\pgfpathlineto{\pgfqpoint{2.779504in}{4.619409in}}%
\pgfpathlineto{\pgfqpoint{2.732101in}{4.720000in}}%
\pgfpathlineto{\pgfqpoint{2.729917in}{4.724587in}}%
\pgfpathlineto{\pgfqpoint{2.681810in}{4.826667in}}%
\pgfpathlineto{\pgfqpoint{2.680331in}{4.829774in}}%
\pgfpathlineto{\pgfqpoint{2.631524in}{4.933333in}}%
\pgfpathlineto{\pgfqpoint{2.630744in}{4.934971in}}%
\pgfpathlineto{\pgfqpoint{2.581242in}{5.040000in}}%
\pgfpathlineto{\pgfqpoint{2.581157in}{5.040178in}}%
\pgfpathlineto{\pgfqpoint{2.531570in}{5.144005in}}%
\pgfpathlineto{\pgfqpoint{2.530322in}{5.146667in}}%
\pgfpathlineto{\pgfqpoint{2.481983in}{5.247715in}}%
\pgfpathlineto{\pgfqpoint{2.479348in}{5.253333in}}%
\pgfpathlineto{\pgfqpoint{2.432397in}{5.351501in}}%
\pgfpathlineto{\pgfqpoint{2.428410in}{5.360000in}}%
\pgfpathlineto{\pgfqpoint{2.382810in}{5.455359in}}%
\pgfpathlineto{\pgfqpoint{2.377505in}{5.466667in}}%
\pgfpathlineto{\pgfqpoint{2.333223in}{5.559285in}}%
\pgfpathlineto{\pgfqpoint{2.326631in}{5.573333in}}%
\pgfpathlineto{\pgfqpoint{2.283636in}{5.663276in}}%
\pgfpathlineto{\pgfqpoint{2.275787in}{5.680000in}}%
\pgfpathlineto{\pgfqpoint{2.234050in}{5.767327in}}%
\pgfpathlineto{\pgfqpoint{2.224972in}{5.786667in}}%
\pgfpathlineto{\pgfqpoint{2.184463in}{5.871437in}}%
\pgfpathlineto{\pgfqpoint{2.174183in}{5.893333in}}%
\pgfpathlineto{\pgfqpoint{2.134876in}{5.975600in}}%
\pgfpathlineto{\pgfqpoint{2.123419in}{6.000000in}}%
\pgfpathlineto{\pgfqpoint{2.085289in}{6.079816in}}%
\pgfpathlineto{\pgfqpoint{2.072679in}{6.106667in}}%
\pgfpathlineto{\pgfqpoint{2.035702in}{6.184081in}}%
\pgfpathlineto{\pgfqpoint{2.021963in}{6.213333in}}%
\pgfpathlineto{\pgfqpoint{1.986116in}{6.288392in}}%
\pgfpathlineto{\pgfqpoint{1.971268in}{6.320000in}}%
\pgfpathlineto{\pgfqpoint{1.936529in}{6.392748in}}%
\pgfpathlineto{\pgfqpoint{1.920593in}{6.426667in}}%
\pgfpathlineto{\pgfqpoint{1.886942in}{6.497146in}}%
\pgfpathlineto{\pgfqpoint{1.869939in}{6.533333in}}%
\pgfpathlineto{\pgfqpoint{1.837355in}{6.601585in}}%
\pgfpathlineto{\pgfqpoint{1.819303in}{6.640000in}}%
\pgfpathlineto{\pgfqpoint{1.787769in}{6.706062in}}%
\pgfpathlineto{\pgfqpoint{1.768685in}{6.746667in}}%
\pgfpathlineto{\pgfqpoint{1.738182in}{6.810576in}}%
\pgfpathlineto{\pgfqpoint{1.718084in}{6.853333in}}%
\pgfpathlineto{\pgfqpoint{1.688595in}{6.915125in}}%
\pgfpathlineto{\pgfqpoint{1.667499in}{6.960000in}}%
\pgfpathlineto{\pgfqpoint{1.639008in}{7.019707in}}%
\pgfpathlineto{\pgfqpoint{1.616930in}{7.066667in}}%
\pgfpathlineto{\pgfqpoint{1.589421in}{7.124322in}}%
\pgfpathlineto{\pgfqpoint{1.566376in}{7.173333in}}%
\pgfpathlineto{\pgfqpoint{1.539835in}{7.228967in}}%
\pgfpathlineto{\pgfqpoint{1.515836in}{7.280000in}}%
\pgfpathlineto{\pgfqpoint{1.490248in}{7.333641in}}%
\pgfpathlineto{\pgfqpoint{1.465310in}{7.386667in}}%
\pgfpathlineto{\pgfqpoint{1.440661in}{7.438344in}}%
\pgfpathlineto{\pgfqpoint{1.414797in}{7.493333in}}%
\pgfpathlineto{\pgfqpoint{1.391074in}{7.543073in}}%
\pgfpathlineto{\pgfqpoint{1.364296in}{7.600000in}}%
\pgfpathlineto{\pgfqpoint{1.341488in}{7.647829in}}%
\pgfpathlineto{\pgfqpoint{1.313808in}{7.706667in}}%
\pgfpathlineto{\pgfqpoint{1.291901in}{7.752609in}}%
\pgfpathlineto{\pgfqpoint{1.263331in}{7.813333in}}%
\pgfpathlineto{\pgfqpoint{1.242314in}{7.857413in}}%
\pgfpathlineto{\pgfqpoint{1.212865in}{7.920000in}}%
\pgfpathlineto{\pgfqpoint{1.192727in}{7.920000in}}%
\pgfpathlineto{\pgfqpoint{1.192727in}{7.837564in}}%
\pgfpathlineto{\pgfqpoint{1.204465in}{7.813333in}}%
\pgfpathlineto{\pgfqpoint{1.192727in}{7.796082in}}%
\pgfpathlineto{\pgfqpoint{1.192727in}{7.706667in}}%
\pgfpathlineto{\pgfqpoint{1.192727in}{7.680242in}}%
\pgfpathlineto{\pgfqpoint{1.230512in}{7.600000in}}%
\pgfpathlineto{\pgfqpoint{1.242314in}{7.575227in}}%
\pgfpathlineto{\pgfqpoint{1.280879in}{7.493333in}}%
\pgfpathlineto{\pgfqpoint{1.291901in}{7.470196in}}%
\pgfpathlineto{\pgfqpoint{1.331239in}{7.386667in}}%
\pgfpathlineto{\pgfqpoint{1.341488in}{7.365151in}}%
\pgfpathlineto{\pgfqpoint{1.381592in}{7.280000in}}%
\pgfpathlineto{\pgfqpoint{1.391074in}{7.260092in}}%
\pgfpathlineto{\pgfqpoint{1.431939in}{7.173333in}}%
\pgfpathlineto{\pgfqpoint{1.440661in}{7.155020in}}%
\pgfpathlineto{\pgfqpoint{1.482279in}{7.066667in}}%
\pgfpathlineto{\pgfqpoint{1.490248in}{7.049933in}}%
\pgfpathlineto{\pgfqpoint{1.532613in}{6.960000in}}%
\pgfpathlineto{\pgfqpoint{1.539835in}{6.944835in}}%
\pgfpathlineto{\pgfqpoint{1.582941in}{6.853333in}}%
\pgfpathlineto{\pgfqpoint{1.589421in}{6.839723in}}%
\pgfpathlineto{\pgfqpoint{1.633263in}{6.746667in}}%
\pgfpathlineto{\pgfqpoint{1.639008in}{6.734600in}}%
\pgfpathlineto{\pgfqpoint{1.683579in}{6.640000in}}%
\pgfpathlineto{\pgfqpoint{1.688595in}{6.629465in}}%
\pgfpathlineto{\pgfqpoint{1.733890in}{6.533333in}}%
\pgfpathlineto{\pgfqpoint{1.738182in}{6.524319in}}%
\pgfpathlineto{\pgfqpoint{1.784196in}{6.426667in}}%
\pgfpathlineto{\pgfqpoint{1.787769in}{6.419162in}}%
\pgfpathlineto{\pgfqpoint{1.834496in}{6.320000in}}%
\pgfpathlineto{\pgfqpoint{1.837355in}{6.313994in}}%
\pgfpathlineto{\pgfqpoint{1.884792in}{6.213333in}}%
\pgfpathlineto{\pgfqpoint{1.886942in}{6.208816in}}%
\pgfpathlineto{\pgfqpoint{1.935082in}{6.106667in}}%
\pgfpathlineto{\pgfqpoint{1.936529in}{6.103628in}}%
\pgfpathlineto{\pgfqpoint{1.985368in}{6.000000in}}%
\pgfpathlineto{\pgfqpoint{1.986116in}{5.998430in}}%
\pgfpathlineto{\pgfqpoint{2.035650in}{5.893333in}}%
\pgfpathlineto{\pgfqpoint{2.035702in}{5.893223in}}%
\pgfpathlineto{\pgfqpoint{2.085289in}{5.789468in}}%
\pgfpathlineto{\pgfqpoint{2.086603in}{5.786667in}}%
\pgfpathlineto{\pgfqpoint{2.134876in}{5.685754in}}%
\pgfpathlineto{\pgfqpoint{2.137575in}{5.680000in}}%
\pgfpathlineto{\pgfqpoint{2.184463in}{5.581965in}}%
\pgfpathlineto{\pgfqpoint{2.188512in}{5.573333in}}%
\pgfpathlineto{\pgfqpoint{2.234050in}{5.478103in}}%
\pgfpathlineto{\pgfqpoint{2.239415in}{5.466667in}}%
\pgfpathlineto{\pgfqpoint{2.283636in}{5.374174in}}%
\pgfpathlineto{\pgfqpoint{2.290288in}{5.360000in}}%
\pgfpathlineto{\pgfqpoint{2.333223in}{5.270180in}}%
\pgfpathlineto{\pgfqpoint{2.341130in}{5.253333in}}%
\pgfpathlineto{\pgfqpoint{2.382810in}{5.166126in}}%
\pgfpathlineto{\pgfqpoint{2.391944in}{5.146667in}}%
\pgfpathlineto{\pgfqpoint{2.432397in}{5.062014in}}%
\pgfpathlineto{\pgfqpoint{2.442732in}{5.040000in}}%
\pgfpathlineto{\pgfqpoint{2.481983in}{4.957848in}}%
\pgfpathlineto{\pgfqpoint{2.493495in}{4.933333in}}%
\pgfpathlineto{\pgfqpoint{2.531570in}{4.853630in}}%
\pgfpathlineto{\pgfqpoint{2.544233in}{4.826667in}}%
\pgfpathlineto{\pgfqpoint{2.581157in}{4.749363in}}%
\pgfpathlineto{\pgfqpoint{2.594949in}{4.720000in}}%
\pgfpathlineto{\pgfqpoint{2.630744in}{4.645050in}}%
\pgfpathlineto{\pgfqpoint{2.645643in}{4.613333in}}%
\pgfpathlineto{\pgfqpoint{2.680331in}{4.540692in}}%
\pgfpathlineto{\pgfqpoint{2.696316in}{4.506667in}}%
\pgfpathlineto{\pgfqpoint{2.729917in}{4.436291in}}%
\pgfpathlineto{\pgfqpoint{2.746970in}{4.400000in}}%
\pgfpathlineto{\pgfqpoint{2.779504in}{4.331851in}}%
\pgfpathlineto{\pgfqpoint{2.797605in}{4.293333in}}%
\pgfpathlineto{\pgfqpoint{2.829091in}{4.227372in}}%
\pgfpathlineto{\pgfqpoint{2.848222in}{4.186667in}}%
\pgfpathlineto{\pgfqpoint{2.878678in}{4.122857in}}%
\pgfpathlineto{\pgfqpoint{2.898823in}{4.080000in}}%
\pgfpathlineto{\pgfqpoint{2.928264in}{4.018306in}}%
\pgfpathlineto{\pgfqpoint{2.949407in}{3.973333in}}%
\pgfpathlineto{\pgfqpoint{2.977851in}{3.913722in}}%
\pgfpathlineto{\pgfqpoint{2.999975in}{3.866667in}}%
\pgfpathlineto{\pgfqpoint{3.027438in}{3.809106in}}%
\pgfpathlineto{\pgfqpoint{3.050528in}{3.760000in}}%
\pgfpathlineto{\pgfqpoint{3.077025in}{3.704460in}}%
\pgfpathlineto{\pgfqpoint{3.101068in}{3.653333in}}%
\pgfpathlineto{\pgfqpoint{3.126612in}{3.599784in}}%
\pgfpathlineto{\pgfqpoint{3.151593in}{3.546667in}}%
\pgfpathlineto{\pgfqpoint{3.176198in}{3.495080in}}%
\pgfpathlineto{\pgfqpoint{3.202106in}{3.440000in}}%
\pgfpathlineto{\pgfqpoint{3.225785in}{3.390349in}}%
\pgfpathlineto{\pgfqpoint{3.252605in}{3.333333in}}%
\pgfpathlineto{\pgfqpoint{3.275372in}{3.285593in}}%
\pgfpathlineto{\pgfqpoint{3.303093in}{3.226667in}}%
\pgfpathlineto{\pgfqpoint{3.324959in}{3.180812in}}%
\pgfpathlineto{\pgfqpoint{3.353570in}{3.120000in}}%
\pgfpathlineto{\pgfqpoint{3.374545in}{3.076007in}}%
\pgfpathlineto{\pgfqpoint{3.404035in}{3.013333in}}%
\pgfpathlineto{\pgfqpoint{3.424132in}{2.971179in}}%
\pgfpathlineto{\pgfqpoint{3.454489in}{2.906667in}}%
\pgfpathlineto{\pgfqpoint{3.473719in}{2.866329in}}%
\pgfpathlineto{\pgfqpoint{3.504934in}{2.800000in}}%
\pgfpathlineto{\pgfqpoint{3.523306in}{2.761458in}}%
\pgfpathlineto{\pgfqpoint{3.555368in}{2.693333in}}%
\pgfpathlineto{\pgfqpoint{3.572893in}{2.656566in}}%
\pgfpathlineto{\pgfqpoint{3.605793in}{2.586667in}}%
\pgfpathlineto{\pgfqpoint{3.622479in}{2.551655in}}%
\pgfpathlineto{\pgfqpoint{3.656209in}{2.480000in}}%
\pgfpathlineto{\pgfqpoint{3.672066in}{2.446726in}}%
\pgfpathlineto{\pgfqpoint{3.706616in}{2.373333in}}%
\pgfpathlineto{\pgfqpoint{3.721653in}{2.341778in}}%
\pgfpathlineto{\pgfqpoint{3.757014in}{2.266667in}}%
\pgfpathlineto{\pgfqpoint{3.771240in}{2.236812in}}%
\pgfpathlineto{\pgfqpoint{3.807405in}{2.160000in}}%
\pgfpathlineto{\pgfqpoint{3.820826in}{2.131830in}}%
\pgfpathlineto{\pgfqpoint{3.857787in}{2.053333in}}%
\pgfpathlineto{\pgfqpoint{3.870413in}{2.026831in}}%
\pgfpathlineto{\pgfqpoint{3.908162in}{1.946667in}}%
\pgfpathlineto{\pgfqpoint{3.920000in}{1.921817in}}%
\pgfpathlineto{\pgfqpoint{3.958529in}{1.840000in}}%
\pgfpathlineto{\pgfqpoint{3.969587in}{1.816787in}}%
\pgfpathlineto{\pgfqpoint{4.008889in}{1.733333in}}%
\pgfpathlineto{\pgfqpoint{4.019174in}{1.711743in}}%
\pgfpathlineto{\pgfqpoint{4.059243in}{1.626667in}}%
\pgfpathlineto{\pgfqpoint{4.068760in}{1.606685in}}%
\pgfpathlineto{\pgfqpoint{4.109590in}{1.520000in}}%
\pgfpathlineto{\pgfqpoint{4.118347in}{1.501612in}}%
\pgfpathlineto{\pgfqpoint{4.159930in}{1.413333in}}%
\pgfpathlineto{\pgfqpoint{4.167934in}{1.396527in}}%
\pgfpathlineto{\pgfqpoint{4.210264in}{1.306667in}}%
\pgfpathlineto{\pgfqpoint{4.217521in}{1.291429in}}%
\pgfpathlineto{\pgfqpoint{4.260592in}{1.200000in}}%
\pgfpathlineto{\pgfqpoint{4.267107in}{1.186318in}}%
\pgfpathlineto{\pgfqpoint{4.310914in}{1.093333in}}%
\pgfpathlineto{\pgfqpoint{4.316694in}{1.081195in}}%
\pgfpathlineto{\pgfqpoint{4.361231in}{0.986667in}}%
\pgfpathlineto{\pgfqpoint{4.366281in}{0.976061in}}%
\pgfpathlineto{\pgfqpoint{4.411542in}{0.880000in}}%
\pgfpathlineto{\pgfqpoint{4.415868in}{0.870915in}}%
\pgfpathlineto{\pgfqpoint{4.461848in}{0.773333in}}%
\pgfpathlineto{\pgfqpoint{4.465455in}{0.765758in}}%
\pgfpathlineto{\pgfqpoint{4.512149in}{0.666667in}}%
\pgfpathlineto{\pgfqpoint{4.515041in}{0.660591in}}%
\pgfpathlineto{\pgfqpoint{4.562445in}{0.560000in}}%
\pgfpathlineto{\pgfqpoint{4.564628in}{0.555413in}}%
\pgfpathlineto{\pgfqpoint{4.612736in}{0.453333in}}%
\pgfpathlineto{\pgfqpoint{4.614215in}{0.450226in}}%
\pgfpathlineto{\pgfqpoint{4.663022in}{0.346667in}}%
\pgfpathlineto{\pgfqpoint{4.663802in}{0.345029in}}%
\pgfpathlineto{\pgfqpoint{4.713304in}{0.240000in}}%
\pgfpathlineto{\pgfqpoint{4.713388in}{0.239822in}}%
\pgfpathlineto{\pgfqpoint{4.762975in}{0.135995in}}%
\pgfpathlineto{\pgfqpoint{4.764223in}{0.133333in}}%
\pgfpathlineto{\pgfqpoint{4.812562in}{0.032285in}}%
\pgfpathlineto{\pgfqpoint{4.815197in}{0.026667in}}%
\pgfpathlineto{\pgfqpoint{4.862149in}{-0.071501in}}%
\pgfpathlineto{\pgfqpoint{4.866136in}{-0.080000in}}%
\pgfpathlineto{\pgfqpoint{4.911736in}{-0.175359in}}%
\pgfpathlineto{\pgfqpoint{4.917041in}{-0.186667in}}%
\pgfpathlineto{\pgfqpoint{4.961322in}{-0.279285in}}%
\pgfpathlineto{\pgfqpoint{4.967914in}{-0.293333in}}%
\pgfpathlineto{\pgfqpoint{5.010909in}{-0.383276in}}%
\pgfpathlineto{\pgfqpoint{5.018758in}{-0.400000in}}%
\pgfpathlineto{\pgfqpoint{5.060496in}{-0.487327in}}%
\pgfpathlineto{\pgfqpoint{5.069574in}{-0.506667in}}%
\pgfpathlineto{\pgfqpoint{5.110083in}{-0.591437in}}%
\pgfpathlineto{\pgfqpoint{5.120363in}{-0.613333in}}%
\pgfpathlineto{\pgfqpoint{5.159669in}{-0.695600in}}%
\pgfpathlineto{\pgfqpoint{5.171127in}{-0.720000in}}%
\pgfpathlineto{\pgfqpoint{5.209256in}{-0.799816in}}%
\pgfpathlineto{\pgfqpoint{5.221866in}{-0.826667in}}%
\pgfpathlineto{\pgfqpoint{5.258843in}{-0.904081in}}%
\pgfpathlineto{\pgfqpoint{5.272583in}{-0.933333in}}%
\pgfpathlineto{\pgfqpoint{5.308430in}{-1.008392in}}%
\pgfpathlineto{\pgfqpoint{5.323278in}{-1.040000in}}%
\pgfpathlineto{\pgfqpoint{5.358017in}{-1.112748in}}%
\pgfpathlineto{\pgfqpoint{5.373952in}{-1.146667in}}%
\pgfpathlineto{\pgfqpoint{5.407603in}{-1.217146in}}%
\pgfpathlineto{\pgfqpoint{5.424607in}{-1.253333in}}%
\pgfpathlineto{\pgfqpoint{5.457190in}{-1.321585in}}%
\pgfpathlineto{\pgfqpoint{5.475243in}{-1.360000in}}%
\pgfpathlineto{\pgfqpoint{5.506777in}{-1.426062in}}%
\pgfpathlineto{\pgfqpoint{5.525861in}{-1.466667in}}%
\pgfpathlineto{\pgfqpoint{5.556364in}{-1.530576in}}%
\pgfpathlineto{\pgfqpoint{5.576462in}{-1.573333in}}%
\pgfpathlineto{\pgfqpoint{5.605950in}{-1.635125in}}%
\pgfpathlineto{\pgfqpoint{5.627046in}{-1.680000in}}%
\pgfpathlineto{\pgfqpoint{5.655537in}{-1.739707in}}%
\pgfpathlineto{\pgfqpoint{5.677616in}{-1.786667in}}%
\pgfpathlineto{\pgfqpoint{5.705124in}{-1.844322in}}%
\pgfpathlineto{\pgfqpoint{5.728170in}{-1.893333in}}%
\pgfpathlineto{\pgfqpoint{5.754711in}{-1.948967in}}%
\pgfpathlineto{\pgfqpoint{5.778709in}{-2.000000in}}%
\pgfpathlineto{\pgfqpoint{5.804298in}{-2.053641in}}%
\pgfpathlineto{\pgfqpoint{5.829236in}{-2.106667in}}%
\pgfpathlineto{\pgfqpoint{5.853884in}{-2.158344in}}%
\pgfpathlineto{\pgfqpoint{5.879749in}{-2.213333in}}%
\pgfpathlineto{\pgfqpoint{5.903471in}{-2.263073in}}%
\pgfpathlineto{\pgfqpoint{5.930249in}{-2.320000in}}%
\pgfpathlineto{\pgfqpoint{5.953058in}{-2.367829in}}%
\pgfpathlineto{\pgfqpoint{5.980738in}{-2.426667in}}%
\pgfpathlineto{\pgfqpoint{6.002645in}{-2.472609in}}%
\pgfpathlineto{\pgfqpoint{6.031214in}{-2.533333in}}%
\pgfpathclose%
\pgfpathmoveto{\pgfqpoint{6.040159in}{-2.426667in}}%
\pgfpathlineto{\pgfqpoint{6.052231in}{-2.410251in}}%
\pgfpathlineto{\pgfqpoint{6.060128in}{-2.426667in}}%
\pgfpathlineto{\pgfqpoint{6.052231in}{-2.451562in}}%
\pgfpathclose%
\pgfpathmoveto{\pgfqpoint{5.990251in}{-2.320000in}}%
\pgfpathlineto{\pgfqpoint{6.002645in}{-2.304428in}}%
\pgfpathlineto{\pgfqpoint{6.010131in}{-2.320000in}}%
\pgfpathlineto{\pgfqpoint{6.002645in}{-2.345528in}}%
\pgfpathclose%
\pgfpathmoveto{\pgfqpoint{5.940358in}{-2.213333in}}%
\pgfpathlineto{\pgfqpoint{5.953058in}{-2.198615in}}%
\pgfpathlineto{\pgfqpoint{5.960132in}{-2.213333in}}%
\pgfpathlineto{\pgfqpoint{5.953058in}{-2.239458in}}%
\pgfpathclose%
\pgfpathmoveto{\pgfqpoint{5.890483in}{-2.106667in}}%
\pgfpathlineto{\pgfqpoint{5.903471in}{-2.092808in}}%
\pgfpathlineto{\pgfqpoint{5.910129in}{-2.106667in}}%
\pgfpathlineto{\pgfqpoint{5.903471in}{-2.133346in}}%
\pgfpathclose%
\pgfpathmoveto{\pgfqpoint{5.840630in}{-2.000000in}}%
\pgfpathlineto{\pgfqpoint{5.853884in}{-1.987009in}}%
\pgfpathlineto{\pgfqpoint{5.860123in}{-2.000000in}}%
\pgfpathlineto{\pgfqpoint{5.853884in}{-2.027184in}}%
\pgfpathclose%
\pgfpathmoveto{\pgfqpoint{5.790801in}{-1.893333in}}%
\pgfpathlineto{\pgfqpoint{5.804298in}{-1.881216in}}%
\pgfpathlineto{\pgfqpoint{5.810114in}{-1.893333in}}%
\pgfpathlineto{\pgfqpoint{5.804298in}{-1.920964in}}%
\pgfpathclose%
\pgfpathmoveto{\pgfqpoint{5.741003in}{-1.786667in}}%
\pgfpathlineto{\pgfqpoint{5.754711in}{-1.775428in}}%
\pgfpathlineto{\pgfqpoint{5.760104in}{-1.786667in}}%
\pgfpathlineto{\pgfqpoint{5.754711in}{-1.814674in}}%
\pgfpathclose%
\pgfpathmoveto{\pgfqpoint{5.691241in}{-1.680000in}}%
\pgfpathlineto{\pgfqpoint{5.705124in}{-1.669646in}}%
\pgfpathlineto{\pgfqpoint{5.710091in}{-1.680000in}}%
\pgfpathlineto{\pgfqpoint{5.705124in}{-1.708300in}}%
\pgfpathclose%
\pgfpathmoveto{\pgfqpoint{5.641524in}{-1.573333in}}%
\pgfpathlineto{\pgfqpoint{5.655537in}{-1.563870in}}%
\pgfpathlineto{\pgfqpoint{5.660075in}{-1.573333in}}%
\pgfpathlineto{\pgfqpoint{5.655537in}{-1.601822in}}%
\pgfpathclose%
\pgfpathmoveto{\pgfqpoint{5.591863in}{-1.466667in}}%
\pgfpathlineto{\pgfqpoint{5.605950in}{-1.458098in}}%
\pgfpathlineto{\pgfqpoint{5.610058in}{-1.466667in}}%
\pgfpathlineto{\pgfqpoint{5.605950in}{-1.495217in}}%
\pgfpathclose%
\pgfpathmoveto{\pgfqpoint{5.542275in}{-1.360000in}}%
\pgfpathlineto{\pgfqpoint{5.556364in}{-1.352330in}}%
\pgfpathlineto{\pgfqpoint{5.560040in}{-1.360000in}}%
\pgfpathlineto{\pgfqpoint{5.556364in}{-1.388451in}}%
\pgfpathclose%
\pgfpathmoveto{\pgfqpoint{5.492779in}{-1.253333in}}%
\pgfpathlineto{\pgfqpoint{5.506777in}{-1.246566in}}%
\pgfpathlineto{\pgfqpoint{5.510019in}{-1.253333in}}%
\pgfpathlineto{\pgfqpoint{5.506777in}{-1.281477in}}%
\pgfpathclose%
\pgfpathmoveto{\pgfqpoint{5.443408in}{-1.146667in}}%
\pgfpathlineto{\pgfqpoint{5.457190in}{-1.140806in}}%
\pgfpathlineto{\pgfqpoint{5.459997in}{-1.146667in}}%
\pgfpathlineto{\pgfqpoint{5.457190in}{-1.174230in}}%
\pgfpathclose%
\pgfpathmoveto{\pgfqpoint{5.394208in}{-1.040000in}}%
\pgfpathlineto{\pgfqpoint{5.407603in}{-1.035050in}}%
\pgfpathlineto{\pgfqpoint{5.409974in}{-1.040000in}}%
\pgfpathlineto{\pgfqpoint{5.407603in}{-1.066613in}}%
\pgfpathclose%
\pgfpathmoveto{\pgfqpoint{5.345254in}{-0.933333in}}%
\pgfpathlineto{\pgfqpoint{5.358017in}{-0.929296in}}%
\pgfpathlineto{\pgfqpoint{5.359949in}{-0.933333in}}%
\pgfpathlineto{\pgfqpoint{5.358017in}{-0.958478in}}%
\pgfpathclose%
\pgfpathmoveto{\pgfqpoint{5.296667in}{-0.826667in}}%
\pgfpathlineto{\pgfqpoint{5.308430in}{-0.823546in}}%
\pgfpathlineto{\pgfqpoint{5.309923in}{-0.826667in}}%
\pgfpathlineto{\pgfqpoint{5.308430in}{-0.849585in}}%
\pgfpathclose%
\pgfpathmoveto{\pgfqpoint{5.248669in}{-0.720000in}}%
\pgfpathlineto{\pgfqpoint{5.258843in}{-0.717799in}}%
\pgfpathlineto{\pgfqpoint{5.259896in}{-0.720000in}}%
\pgfpathlineto{\pgfqpoint{5.258843in}{-0.739519in}}%
\pgfpathclose%
\pgfpathmoveto{\pgfqpoint{5.201707in}{-0.613333in}}%
\pgfpathlineto{\pgfqpoint{5.209256in}{-0.612055in}}%
\pgfpathlineto{\pgfqpoint{5.209868in}{-0.613333in}}%
\pgfpathlineto{\pgfqpoint{5.209256in}{-0.627492in}}%
\pgfpathclose%
\pgfpathmoveto{\pgfqpoint{5.156821in}{-0.506667in}}%
\pgfpathlineto{\pgfqpoint{5.159669in}{-0.506313in}}%
\pgfpathlineto{\pgfqpoint{5.159839in}{-0.506667in}}%
\pgfpathlineto{\pgfqpoint{5.159669in}{-0.511816in}}%
\pgfpathclose%
\pgfpathmoveto{\pgfqpoint{4.812372in}{0.133333in}}%
\pgfpathlineto{\pgfqpoint{4.812562in}{0.139027in}}%
\pgfpathlineto{\pgfqpoint{4.815705in}{0.133333in}}%
\pgfpathlineto{\pgfqpoint{4.812562in}{0.132936in}}%
\pgfpathclose%
\pgfpathmoveto{\pgfqpoint{4.762343in}{0.240000in}}%
\pgfpathlineto{\pgfqpoint{4.762975in}{0.254471in}}%
\pgfpathlineto{\pgfqpoint{4.770681in}{0.240000in}}%
\pgfpathlineto{\pgfqpoint{4.762975in}{0.238678in}}%
\pgfpathclose%
\pgfpathmoveto{\pgfqpoint{4.712315in}{0.346667in}}%
\pgfpathlineto{\pgfqpoint{4.713388in}{0.366378in}}%
\pgfpathlineto{\pgfqpoint{4.723654in}{0.346667in}}%
\pgfpathlineto{\pgfqpoint{4.713388in}{0.344423in}}%
\pgfpathclose%
\pgfpathmoveto{\pgfqpoint{4.662288in}{0.453333in}}%
\pgfpathlineto{\pgfqpoint{4.663802in}{0.476377in}}%
\pgfpathlineto{\pgfqpoint{4.675622in}{0.453333in}}%
\pgfpathlineto{\pgfqpoint{4.663802in}{0.450170in}}%
\pgfpathclose%
\pgfpathmoveto{\pgfqpoint{4.612262in}{0.560000in}}%
\pgfpathlineto{\pgfqpoint{4.614215in}{0.585227in}}%
\pgfpathlineto{\pgfqpoint{4.627014in}{0.560000in}}%
\pgfpathlineto{\pgfqpoint{4.614215in}{0.555921in}}%
\pgfpathclose%
\pgfpathmoveto{\pgfqpoint{4.562237in}{0.666667in}}%
\pgfpathlineto{\pgfqpoint{4.564628in}{0.693334in}}%
\pgfpathlineto{\pgfqpoint{4.578046in}{0.666667in}}%
\pgfpathlineto{\pgfqpoint{4.564628in}{0.661674in}}%
\pgfpathclose%
\pgfpathmoveto{\pgfqpoint{4.512214in}{0.773333in}}%
\pgfpathlineto{\pgfqpoint{4.515041in}{0.800931in}}%
\pgfpathlineto{\pgfqpoint{4.528837in}{0.773333in}}%
\pgfpathlineto{\pgfqpoint{4.515041in}{0.767430in}}%
\pgfpathclose%
\pgfpathmoveto{\pgfqpoint{4.462192in}{0.880000in}}%
\pgfpathlineto{\pgfqpoint{4.465455in}{0.908163in}}%
\pgfpathlineto{\pgfqpoint{4.479459in}{0.880000in}}%
\pgfpathlineto{\pgfqpoint{4.465455in}{0.873191in}}%
\pgfpathclose%
\pgfpathmoveto{\pgfqpoint{4.412172in}{0.986667in}}%
\pgfpathlineto{\pgfqpoint{4.415868in}{1.015126in}}%
\pgfpathlineto{\pgfqpoint{4.429959in}{0.986667in}}%
\pgfpathlineto{\pgfqpoint{4.415868in}{0.978955in}}%
\pgfpathclose%
\pgfpathmoveto{\pgfqpoint{4.362153in}{1.093333in}}%
\pgfpathlineto{\pgfqpoint{4.366281in}{1.121884in}}%
\pgfpathlineto{\pgfqpoint{4.380366in}{1.093333in}}%
\pgfpathlineto{\pgfqpoint{4.366281in}{1.084723in}}%
\pgfpathclose%
\pgfpathmoveto{\pgfqpoint{4.312136in}{1.200000in}}%
\pgfpathlineto{\pgfqpoint{4.316694in}{1.228483in}}%
\pgfpathlineto{\pgfqpoint{4.330703in}{1.200000in}}%
\pgfpathlineto{\pgfqpoint{4.316694in}{1.190495in}}%
\pgfpathclose%
\pgfpathmoveto{\pgfqpoint{4.262121in}{1.306667in}}%
\pgfpathlineto{\pgfqpoint{4.267107in}{1.334955in}}%
\pgfpathlineto{\pgfqpoint{4.280984in}{1.306667in}}%
\pgfpathlineto{\pgfqpoint{4.267107in}{1.296272in}}%
\pgfpathclose%
\pgfpathmoveto{\pgfqpoint{4.212108in}{1.413333in}}%
\pgfpathlineto{\pgfqpoint{4.217521in}{1.441325in}}%
\pgfpathlineto{\pgfqpoint{4.231220in}{1.413333in}}%
\pgfpathlineto{\pgfqpoint{4.217521in}{1.402054in}}%
\pgfpathclose%
\pgfpathmoveto{\pgfqpoint{4.162097in}{1.520000in}}%
\pgfpathlineto{\pgfqpoint{4.167934in}{1.547611in}}%
\pgfpathlineto{\pgfqpoint{4.181420in}{1.520000in}}%
\pgfpathlineto{\pgfqpoint{4.167934in}{1.507841in}}%
\pgfpathclose%
\pgfpathmoveto{\pgfqpoint{4.112089in}{1.626667in}}%
\pgfpathlineto{\pgfqpoint{4.118347in}{1.653829in}}%
\pgfpathlineto{\pgfqpoint{4.131590in}{1.626667in}}%
\pgfpathlineto{\pgfqpoint{4.118347in}{1.613635in}}%
\pgfpathclose%
\pgfpathmoveto{\pgfqpoint{4.062083in}{1.733333in}}%
\pgfpathlineto{\pgfqpoint{4.068760in}{1.759988in}}%
\pgfpathlineto{\pgfqpoint{4.081735in}{1.733333in}}%
\pgfpathlineto{\pgfqpoint{4.068760in}{1.719435in}}%
\pgfpathclose%
\pgfpathmoveto{\pgfqpoint{4.012081in}{1.840000in}}%
\pgfpathlineto{\pgfqpoint{4.019174in}{1.866098in}}%
\pgfpathlineto{\pgfqpoint{4.031859in}{1.840000in}}%
\pgfpathlineto{\pgfqpoint{4.019174in}{1.825241in}}%
\pgfpathclose%
\pgfpathmoveto{\pgfqpoint{3.962081in}{1.946667in}}%
\pgfpathlineto{\pgfqpoint{3.969587in}{1.972166in}}%
\pgfpathlineto{\pgfqpoint{3.981966in}{1.946667in}}%
\pgfpathlineto{\pgfqpoint{3.969587in}{1.931056in}}%
\pgfpathclose%
\pgfpathmoveto{\pgfqpoint{3.912085in}{2.053333in}}%
\pgfpathlineto{\pgfqpoint{3.920000in}{2.078198in}}%
\pgfpathlineto{\pgfqpoint{3.932057in}{2.053333in}}%
\pgfpathlineto{\pgfqpoint{3.920000in}{2.036878in}}%
\pgfpathclose%
\pgfpathmoveto{\pgfqpoint{3.862092in}{2.160000in}}%
\pgfpathlineto{\pgfqpoint{3.870413in}{2.184199in}}%
\pgfpathlineto{\pgfqpoint{3.882135in}{2.160000in}}%
\pgfpathlineto{\pgfqpoint{3.870413in}{2.142710in}}%
\pgfpathclose%
\pgfpathmoveto{\pgfqpoint{3.812103in}{2.266667in}}%
\pgfpathlineto{\pgfqpoint{3.820826in}{2.290173in}}%
\pgfpathlineto{\pgfqpoint{3.832202in}{2.266667in}}%
\pgfpathlineto{\pgfqpoint{3.820826in}{2.248551in}}%
\pgfpathclose%
\pgfpathmoveto{\pgfqpoint{3.762120in}{2.373333in}}%
\pgfpathlineto{\pgfqpoint{3.771240in}{2.396122in}}%
\pgfpathlineto{\pgfqpoint{3.782258in}{2.373333in}}%
\pgfpathlineto{\pgfqpoint{3.771240in}{2.354404in}}%
\pgfpathclose%
\pgfpathmoveto{\pgfqpoint{3.712141in}{2.480000in}}%
\pgfpathlineto{\pgfqpoint{3.721653in}{2.502051in}}%
\pgfpathlineto{\pgfqpoint{3.732306in}{2.480000in}}%
\pgfpathlineto{\pgfqpoint{3.721653in}{2.460269in}}%
\pgfpathclose%
\pgfpathmoveto{\pgfqpoint{3.662167in}{2.586667in}}%
\pgfpathlineto{\pgfqpoint{3.672066in}{2.607960in}}%
\pgfpathlineto{\pgfqpoint{3.682345in}{2.586667in}}%
\pgfpathlineto{\pgfqpoint{3.672066in}{2.566147in}}%
\pgfpathclose%
\pgfpathmoveto{\pgfqpoint{3.612200in}{2.693333in}}%
\pgfpathlineto{\pgfqpoint{3.622479in}{2.713853in}}%
\pgfpathlineto{\pgfqpoint{3.632378in}{2.693333in}}%
\pgfpathlineto{\pgfqpoint{3.622479in}{2.672040in}}%
\pgfpathclose%
\pgfpathmoveto{\pgfqpoint{3.562240in}{2.800000in}}%
\pgfpathlineto{\pgfqpoint{3.572893in}{2.819731in}}%
\pgfpathlineto{\pgfqpoint{3.582405in}{2.800000in}}%
\pgfpathlineto{\pgfqpoint{3.572893in}{2.777949in}}%
\pgfpathclose%
\pgfpathmoveto{\pgfqpoint{3.512287in}{2.906667in}}%
\pgfpathlineto{\pgfqpoint{3.523306in}{2.925596in}}%
\pgfpathlineto{\pgfqpoint{3.532426in}{2.906667in}}%
\pgfpathlineto{\pgfqpoint{3.523306in}{2.883878in}}%
\pgfpathclose%
\pgfpathmoveto{\pgfqpoint{3.462344in}{3.013333in}}%
\pgfpathlineto{\pgfqpoint{3.473719in}{3.031449in}}%
\pgfpathlineto{\pgfqpoint{3.482442in}{3.013333in}}%
\pgfpathlineto{\pgfqpoint{3.473719in}{2.989827in}}%
\pgfpathclose%
\pgfpathmoveto{\pgfqpoint{3.412410in}{3.120000in}}%
\pgfpathlineto{\pgfqpoint{3.424132in}{3.137290in}}%
\pgfpathlineto{\pgfqpoint{3.432454in}{3.120000in}}%
\pgfpathlineto{\pgfqpoint{3.424132in}{3.095801in}}%
\pgfpathclose%
\pgfpathmoveto{\pgfqpoint{3.362488in}{3.226667in}}%
\pgfpathlineto{\pgfqpoint{3.374545in}{3.243122in}}%
\pgfpathlineto{\pgfqpoint{3.382461in}{3.226667in}}%
\pgfpathlineto{\pgfqpoint{3.374545in}{3.201802in}}%
\pgfpathclose%
\pgfpathmoveto{\pgfqpoint{3.312580in}{3.333333in}}%
\pgfpathlineto{\pgfqpoint{3.324959in}{3.348944in}}%
\pgfpathlineto{\pgfqpoint{3.332465in}{3.333333in}}%
\pgfpathlineto{\pgfqpoint{3.324959in}{3.307834in}}%
\pgfpathclose%
\pgfpathmoveto{\pgfqpoint{3.262686in}{3.440000in}}%
\pgfpathlineto{\pgfqpoint{3.275372in}{3.454759in}}%
\pgfpathlineto{\pgfqpoint{3.282465in}{3.440000in}}%
\pgfpathlineto{\pgfqpoint{3.275372in}{3.413902in}}%
\pgfpathclose%
\pgfpathmoveto{\pgfqpoint{3.212810in}{3.546667in}}%
\pgfpathlineto{\pgfqpoint{3.225785in}{3.560565in}}%
\pgfpathlineto{\pgfqpoint{3.232462in}{3.546667in}}%
\pgfpathlineto{\pgfqpoint{3.225785in}{3.520012in}}%
\pgfpathclose%
\pgfpathmoveto{\pgfqpoint{3.162956in}{3.653333in}}%
\pgfpathlineto{\pgfqpoint{3.176198in}{3.666365in}}%
\pgfpathlineto{\pgfqpoint{3.182456in}{3.653333in}}%
\pgfpathlineto{\pgfqpoint{3.176198in}{3.626171in}}%
\pgfpathclose%
\pgfpathmoveto{\pgfqpoint{3.113126in}{3.760000in}}%
\pgfpathlineto{\pgfqpoint{3.126612in}{3.772159in}}%
\pgfpathlineto{\pgfqpoint{3.132448in}{3.760000in}}%
\pgfpathlineto{\pgfqpoint{3.126612in}{3.732389in}}%
\pgfpathclose%
\pgfpathmoveto{\pgfqpoint{3.063326in}{3.866667in}}%
\pgfpathlineto{\pgfqpoint{3.077025in}{3.877946in}}%
\pgfpathlineto{\pgfqpoint{3.082437in}{3.866667in}}%
\pgfpathlineto{\pgfqpoint{3.077025in}{3.838675in}}%
\pgfpathclose%
\pgfpathmoveto{\pgfqpoint{3.013562in}{3.973333in}}%
\pgfpathlineto{\pgfqpoint{3.027438in}{3.983728in}}%
\pgfpathlineto{\pgfqpoint{3.032424in}{3.973333in}}%
\pgfpathlineto{\pgfqpoint{3.027438in}{3.945045in}}%
\pgfpathclose%
\pgfpathmoveto{\pgfqpoint{2.963843in}{4.080000in}}%
\pgfpathlineto{\pgfqpoint{2.977851in}{4.089505in}}%
\pgfpathlineto{\pgfqpoint{2.982409in}{4.080000in}}%
\pgfpathlineto{\pgfqpoint{2.977851in}{4.051517in}}%
\pgfpathclose%
\pgfpathmoveto{\pgfqpoint{2.914179in}{4.186667in}}%
\pgfpathlineto{\pgfqpoint{2.928264in}{4.195277in}}%
\pgfpathlineto{\pgfqpoint{2.932393in}{4.186667in}}%
\pgfpathlineto{\pgfqpoint{2.928264in}{4.158116in}}%
\pgfpathclose%
\pgfpathmoveto{\pgfqpoint{2.864587in}{4.293333in}}%
\pgfpathlineto{\pgfqpoint{2.878678in}{4.301045in}}%
\pgfpathlineto{\pgfqpoint{2.882374in}{4.293333in}}%
\pgfpathlineto{\pgfqpoint{2.878678in}{4.264874in}}%
\pgfpathclose%
\pgfpathmoveto{\pgfqpoint{2.815086in}{4.400000in}}%
\pgfpathlineto{\pgfqpoint{2.829091in}{4.406809in}}%
\pgfpathlineto{\pgfqpoint{2.832354in}{4.400000in}}%
\pgfpathlineto{\pgfqpoint{2.829091in}{4.371837in}}%
\pgfpathclose%
\pgfpathmoveto{\pgfqpoint{2.765709in}{4.506667in}}%
\pgfpathlineto{\pgfqpoint{2.779504in}{4.512570in}}%
\pgfpathlineto{\pgfqpoint{2.782332in}{4.506667in}}%
\pgfpathlineto{\pgfqpoint{2.779504in}{4.479069in}}%
\pgfpathclose%
\pgfpathmoveto{\pgfqpoint{2.716500in}{4.613333in}}%
\pgfpathlineto{\pgfqpoint{2.729917in}{4.618326in}}%
\pgfpathlineto{\pgfqpoint{2.732308in}{4.613333in}}%
\pgfpathlineto{\pgfqpoint{2.729917in}{4.586666in}}%
\pgfpathclose%
\pgfpathmoveto{\pgfqpoint{2.667531in}{4.720000in}}%
\pgfpathlineto{\pgfqpoint{2.680331in}{4.724079in}}%
\pgfpathlineto{\pgfqpoint{2.682284in}{4.720000in}}%
\pgfpathlineto{\pgfqpoint{2.680331in}{4.694773in}}%
\pgfpathclose%
\pgfpathmoveto{\pgfqpoint{2.618923in}{4.826667in}}%
\pgfpathlineto{\pgfqpoint{2.630744in}{4.829830in}}%
\pgfpathlineto{\pgfqpoint{2.632258in}{4.826667in}}%
\pgfpathlineto{\pgfqpoint{2.630744in}{4.803623in}}%
\pgfpathclose%
\pgfpathmoveto{\pgfqpoint{2.570891in}{4.933333in}}%
\pgfpathlineto{\pgfqpoint{2.581157in}{4.935577in}}%
\pgfpathlineto{\pgfqpoint{2.582231in}{4.933333in}}%
\pgfpathlineto{\pgfqpoint{2.581157in}{4.913622in}}%
\pgfpathclose%
\pgfpathmoveto{\pgfqpoint{2.523865in}{5.040000in}}%
\pgfpathlineto{\pgfqpoint{2.531570in}{5.041322in}}%
\pgfpathlineto{\pgfqpoint{2.532203in}{5.040000in}}%
\pgfpathlineto{\pgfqpoint{2.531570in}{5.025529in}}%
\pgfpathclose%
\pgfpathmoveto{\pgfqpoint{2.478841in}{5.146667in}}%
\pgfpathlineto{\pgfqpoint{2.481983in}{5.147064in}}%
\pgfpathlineto{\pgfqpoint{2.482173in}{5.146667in}}%
\pgfpathlineto{\pgfqpoint{2.481983in}{5.140973in}}%
\pgfpathclose%
\pgfpathmoveto{\pgfqpoint{2.134707in}{5.786667in}}%
\pgfpathlineto{\pgfqpoint{2.134876in}{5.791816in}}%
\pgfpathlineto{\pgfqpoint{2.137724in}{5.786667in}}%
\pgfpathlineto{\pgfqpoint{2.134876in}{5.786313in}}%
\pgfpathclose%
\pgfpathmoveto{\pgfqpoint{2.084677in}{5.893333in}}%
\pgfpathlineto{\pgfqpoint{2.085289in}{5.907492in}}%
\pgfpathlineto{\pgfqpoint{2.092839in}{5.893333in}}%
\pgfpathlineto{\pgfqpoint{2.085289in}{5.892055in}}%
\pgfpathclose%
\pgfpathmoveto{\pgfqpoint{2.034649in}{6.000000in}}%
\pgfpathlineto{\pgfqpoint{2.035702in}{6.019519in}}%
\pgfpathlineto{\pgfqpoint{2.045876in}{6.000000in}}%
\pgfpathlineto{\pgfqpoint{2.035702in}{5.997799in}}%
\pgfpathclose%
\pgfpathmoveto{\pgfqpoint{1.984622in}{6.106667in}}%
\pgfpathlineto{\pgfqpoint{1.986116in}{6.129585in}}%
\pgfpathlineto{\pgfqpoint{1.997879in}{6.106667in}}%
\pgfpathlineto{\pgfqpoint{1.986116in}{6.103546in}}%
\pgfpathclose%
\pgfpathmoveto{\pgfqpoint{1.934596in}{6.213333in}}%
\pgfpathlineto{\pgfqpoint{1.936529in}{6.238478in}}%
\pgfpathlineto{\pgfqpoint{1.949292in}{6.213333in}}%
\pgfpathlineto{\pgfqpoint{1.936529in}{6.209296in}}%
\pgfpathclose%
\pgfpathmoveto{\pgfqpoint{1.884571in}{6.320000in}}%
\pgfpathlineto{\pgfqpoint{1.886942in}{6.346613in}}%
\pgfpathlineto{\pgfqpoint{1.900337in}{6.320000in}}%
\pgfpathlineto{\pgfqpoint{1.886942in}{6.315050in}}%
\pgfpathclose%
\pgfpathmoveto{\pgfqpoint{1.834548in}{6.426667in}}%
\pgfpathlineto{\pgfqpoint{1.837355in}{6.454230in}}%
\pgfpathlineto{\pgfqpoint{1.851137in}{6.426667in}}%
\pgfpathlineto{\pgfqpoint{1.837355in}{6.420806in}}%
\pgfpathclose%
\pgfpathmoveto{\pgfqpoint{1.784526in}{6.533333in}}%
\pgfpathlineto{\pgfqpoint{1.787769in}{6.561477in}}%
\pgfpathlineto{\pgfqpoint{1.801766in}{6.533333in}}%
\pgfpathlineto{\pgfqpoint{1.787769in}{6.526566in}}%
\pgfpathclose%
\pgfpathmoveto{\pgfqpoint{1.734506in}{6.640000in}}%
\pgfpathlineto{\pgfqpoint{1.738182in}{6.668451in}}%
\pgfpathlineto{\pgfqpoint{1.752271in}{6.640000in}}%
\pgfpathlineto{\pgfqpoint{1.738182in}{6.632330in}}%
\pgfpathclose%
\pgfpathmoveto{\pgfqpoint{1.684487in}{6.746667in}}%
\pgfpathlineto{\pgfqpoint{1.688595in}{6.775217in}}%
\pgfpathlineto{\pgfqpoint{1.702682in}{6.746667in}}%
\pgfpathlineto{\pgfqpoint{1.688595in}{6.738098in}}%
\pgfpathclose%
\pgfpathmoveto{\pgfqpoint{1.634470in}{6.853333in}}%
\pgfpathlineto{\pgfqpoint{1.639008in}{6.881822in}}%
\pgfpathlineto{\pgfqpoint{1.653022in}{6.853333in}}%
\pgfpathlineto{\pgfqpoint{1.639008in}{6.843870in}}%
\pgfpathclose%
\pgfpathmoveto{\pgfqpoint{1.584455in}{6.960000in}}%
\pgfpathlineto{\pgfqpoint{1.589421in}{6.988300in}}%
\pgfpathlineto{\pgfqpoint{1.603305in}{6.960000in}}%
\pgfpathlineto{\pgfqpoint{1.589421in}{6.949646in}}%
\pgfpathclose%
\pgfpathmoveto{\pgfqpoint{1.534442in}{7.066667in}}%
\pgfpathlineto{\pgfqpoint{1.539835in}{7.094674in}}%
\pgfpathlineto{\pgfqpoint{1.553543in}{7.066667in}}%
\pgfpathlineto{\pgfqpoint{1.539835in}{7.055428in}}%
\pgfpathclose%
\pgfpathmoveto{\pgfqpoint{1.484431in}{7.173333in}}%
\pgfpathlineto{\pgfqpoint{1.490248in}{7.200964in}}%
\pgfpathlineto{\pgfqpoint{1.503744in}{7.173333in}}%
\pgfpathlineto{\pgfqpoint{1.490248in}{7.161216in}}%
\pgfpathclose%
\pgfpathmoveto{\pgfqpoint{1.434423in}{7.280000in}}%
\pgfpathlineto{\pgfqpoint{1.440661in}{7.307184in}}%
\pgfpathlineto{\pgfqpoint{1.453916in}{7.280000in}}%
\pgfpathlineto{\pgfqpoint{1.440661in}{7.267009in}}%
\pgfpathclose%
\pgfpathmoveto{\pgfqpoint{1.384417in}{7.386667in}}%
\pgfpathlineto{\pgfqpoint{1.391074in}{7.413346in}}%
\pgfpathlineto{\pgfqpoint{1.404062in}{7.386667in}}%
\pgfpathlineto{\pgfqpoint{1.391074in}{7.372808in}}%
\pgfpathclose%
\pgfpathmoveto{\pgfqpoint{1.334414in}{7.493333in}}%
\pgfpathlineto{\pgfqpoint{1.341488in}{7.519458in}}%
\pgfpathlineto{\pgfqpoint{1.354187in}{7.493333in}}%
\pgfpathlineto{\pgfqpoint{1.341488in}{7.478615in}}%
\pgfpathclose%
\pgfpathmoveto{\pgfqpoint{1.284414in}{7.600000in}}%
\pgfpathlineto{\pgfqpoint{1.291901in}{7.625528in}}%
\pgfpathlineto{\pgfqpoint{1.304295in}{7.600000in}}%
\pgfpathlineto{\pgfqpoint{1.291901in}{7.584428in}}%
\pgfpathclose%
\pgfpathmoveto{\pgfqpoint{1.234418in}{7.706667in}}%
\pgfpathlineto{\pgfqpoint{1.242314in}{7.731562in}}%
\pgfpathlineto{\pgfqpoint{1.254386in}{7.706667in}}%
\pgfpathlineto{\pgfqpoint{1.242314in}{7.690251in}}%
\pgfpathclose%
\pgfusepath{fill}%
\end{pgfscope}%
\begin{pgfscope}%
\pgfpathrectangle{\pgfqpoint{3.156364in}{0.528000in}}{\pgfqpoint{1.963636in}{3.696000in}} %
\pgfusepath{clip}%
\pgfsetbuttcap%
\pgfsetroundjoin%
\definecolor{currentfill}{rgb}{0.423689,0.000646,0.658956}%
\pgfsetfillcolor{currentfill}%
\pgfsetlinewidth{0.000000pt}%
\definecolor{currentstroke}{rgb}{0.000000,0.000000,0.000000}%
\pgfsetstrokecolor{currentstroke}%
\pgfsetdash{}{0pt}%
\pgfpathmoveto{\pgfqpoint{5.903471in}{-2.592811in}}%
\pgfpathlineto{\pgfqpoint{5.925774in}{-2.640000in}}%
\pgfpathlineto{\pgfqpoint{5.953058in}{-2.640000in}}%
\pgfpathlineto{\pgfqpoint{6.002645in}{-2.640000in}}%
\pgfpathlineto{\pgfqpoint{6.052231in}{-2.640000in}}%
\pgfpathlineto{\pgfqpoint{6.081680in}{-2.640000in}}%
\pgfpathlineto{\pgfqpoint{6.052231in}{-2.577413in}}%
\pgfpathlineto{\pgfqpoint{6.031214in}{-2.533333in}}%
\pgfpathlineto{\pgfqpoint{6.002645in}{-2.472609in}}%
\pgfpathlineto{\pgfqpoint{5.980738in}{-2.426667in}}%
\pgfpathlineto{\pgfqpoint{5.953058in}{-2.367829in}}%
\pgfpathlineto{\pgfqpoint{5.930249in}{-2.320000in}}%
\pgfpathlineto{\pgfqpoint{5.903471in}{-2.263073in}}%
\pgfpathlineto{\pgfqpoint{5.879749in}{-2.213333in}}%
\pgfpathlineto{\pgfqpoint{5.853884in}{-2.158344in}}%
\pgfpathlineto{\pgfqpoint{5.829236in}{-2.106667in}}%
\pgfpathlineto{\pgfqpoint{5.804298in}{-2.053641in}}%
\pgfpathlineto{\pgfqpoint{5.778709in}{-2.000000in}}%
\pgfpathlineto{\pgfqpoint{5.754711in}{-1.948967in}}%
\pgfpathlineto{\pgfqpoint{5.728170in}{-1.893333in}}%
\pgfpathlineto{\pgfqpoint{5.705124in}{-1.844322in}}%
\pgfpathlineto{\pgfqpoint{5.677616in}{-1.786667in}}%
\pgfpathlineto{\pgfqpoint{5.655537in}{-1.739707in}}%
\pgfpathlineto{\pgfqpoint{5.627046in}{-1.680000in}}%
\pgfpathlineto{\pgfqpoint{5.605950in}{-1.635125in}}%
\pgfpathlineto{\pgfqpoint{5.576462in}{-1.573333in}}%
\pgfpathlineto{\pgfqpoint{5.556364in}{-1.530576in}}%
\pgfpathlineto{\pgfqpoint{5.525861in}{-1.466667in}}%
\pgfpathlineto{\pgfqpoint{5.506777in}{-1.426062in}}%
\pgfpathlineto{\pgfqpoint{5.475243in}{-1.360000in}}%
\pgfpathlineto{\pgfqpoint{5.457190in}{-1.321585in}}%
\pgfpathlineto{\pgfqpoint{5.424607in}{-1.253333in}}%
\pgfpathlineto{\pgfqpoint{5.407603in}{-1.217146in}}%
\pgfpathlineto{\pgfqpoint{5.373952in}{-1.146667in}}%
\pgfpathlineto{\pgfqpoint{5.358017in}{-1.112748in}}%
\pgfpathlineto{\pgfqpoint{5.323278in}{-1.040000in}}%
\pgfpathlineto{\pgfqpoint{5.308430in}{-1.008392in}}%
\pgfpathlineto{\pgfqpoint{5.272583in}{-0.933333in}}%
\pgfpathlineto{\pgfqpoint{5.258843in}{-0.904081in}}%
\pgfpathlineto{\pgfqpoint{5.221866in}{-0.826667in}}%
\pgfpathlineto{\pgfqpoint{5.209256in}{-0.799816in}}%
\pgfpathlineto{\pgfqpoint{5.171127in}{-0.720000in}}%
\pgfpathlineto{\pgfqpoint{5.159669in}{-0.695600in}}%
\pgfpathlineto{\pgfqpoint{5.120363in}{-0.613333in}}%
\pgfpathlineto{\pgfqpoint{5.110083in}{-0.591437in}}%
\pgfpathlineto{\pgfqpoint{5.069574in}{-0.506667in}}%
\pgfpathlineto{\pgfqpoint{5.060496in}{-0.487327in}}%
\pgfpathlineto{\pgfqpoint{5.018758in}{-0.400000in}}%
\pgfpathlineto{\pgfqpoint{5.010909in}{-0.383276in}}%
\pgfpathlineto{\pgfqpoint{4.967914in}{-0.293333in}}%
\pgfpathlineto{\pgfqpoint{4.961322in}{-0.279285in}}%
\pgfpathlineto{\pgfqpoint{4.917041in}{-0.186667in}}%
\pgfpathlineto{\pgfqpoint{4.911736in}{-0.175359in}}%
\pgfpathlineto{\pgfqpoint{4.866136in}{-0.080000in}}%
\pgfpathlineto{\pgfqpoint{4.862149in}{-0.071501in}}%
\pgfpathlineto{\pgfqpoint{4.815197in}{0.026667in}}%
\pgfpathlineto{\pgfqpoint{4.812562in}{0.032285in}}%
\pgfpathlineto{\pgfqpoint{4.764223in}{0.133333in}}%
\pgfpathlineto{\pgfqpoint{4.762975in}{0.135995in}}%
\pgfpathlineto{\pgfqpoint{4.713388in}{0.239822in}}%
\pgfpathlineto{\pgfqpoint{4.713304in}{0.240000in}}%
\pgfpathlineto{\pgfqpoint{4.663802in}{0.345029in}}%
\pgfpathlineto{\pgfqpoint{4.663022in}{0.346667in}}%
\pgfpathlineto{\pgfqpoint{4.614215in}{0.450226in}}%
\pgfpathlineto{\pgfqpoint{4.612736in}{0.453333in}}%
\pgfpathlineto{\pgfqpoint{4.564628in}{0.555413in}}%
\pgfpathlineto{\pgfqpoint{4.562445in}{0.560000in}}%
\pgfpathlineto{\pgfqpoint{4.515041in}{0.660591in}}%
\pgfpathlineto{\pgfqpoint{4.512149in}{0.666667in}}%
\pgfpathlineto{\pgfqpoint{4.465455in}{0.765758in}}%
\pgfpathlineto{\pgfqpoint{4.461848in}{0.773333in}}%
\pgfpathlineto{\pgfqpoint{4.415868in}{0.870915in}}%
\pgfpathlineto{\pgfqpoint{4.411542in}{0.880000in}}%
\pgfpathlineto{\pgfqpoint{4.366281in}{0.976061in}}%
\pgfpathlineto{\pgfqpoint{4.361231in}{0.986667in}}%
\pgfpathlineto{\pgfqpoint{4.316694in}{1.081195in}}%
\pgfpathlineto{\pgfqpoint{4.310914in}{1.093333in}}%
\pgfpathlineto{\pgfqpoint{4.267107in}{1.186318in}}%
\pgfpathlineto{\pgfqpoint{4.260592in}{1.200000in}}%
\pgfpathlineto{\pgfqpoint{4.217521in}{1.291429in}}%
\pgfpathlineto{\pgfqpoint{4.210264in}{1.306667in}}%
\pgfpathlineto{\pgfqpoint{4.167934in}{1.396527in}}%
\pgfpathlineto{\pgfqpoint{4.159930in}{1.413333in}}%
\pgfpathlineto{\pgfqpoint{4.118347in}{1.501612in}}%
\pgfpathlineto{\pgfqpoint{4.109590in}{1.520000in}}%
\pgfpathlineto{\pgfqpoint{4.068760in}{1.606685in}}%
\pgfpathlineto{\pgfqpoint{4.059243in}{1.626667in}}%
\pgfpathlineto{\pgfqpoint{4.019174in}{1.711743in}}%
\pgfpathlineto{\pgfqpoint{4.008889in}{1.733333in}}%
\pgfpathlineto{\pgfqpoint{3.969587in}{1.816787in}}%
\pgfpathlineto{\pgfqpoint{3.958529in}{1.840000in}}%
\pgfpathlineto{\pgfqpoint{3.920000in}{1.921817in}}%
\pgfpathlineto{\pgfqpoint{3.908162in}{1.946667in}}%
\pgfpathlineto{\pgfqpoint{3.870413in}{2.026831in}}%
\pgfpathlineto{\pgfqpoint{3.857787in}{2.053333in}}%
\pgfpathlineto{\pgfqpoint{3.820826in}{2.131830in}}%
\pgfpathlineto{\pgfqpoint{3.807405in}{2.160000in}}%
\pgfpathlineto{\pgfqpoint{3.771240in}{2.236812in}}%
\pgfpathlineto{\pgfqpoint{3.757014in}{2.266667in}}%
\pgfpathlineto{\pgfqpoint{3.721653in}{2.341778in}}%
\pgfpathlineto{\pgfqpoint{3.706616in}{2.373333in}}%
\pgfpathlineto{\pgfqpoint{3.672066in}{2.446726in}}%
\pgfpathlineto{\pgfqpoint{3.656209in}{2.480000in}}%
\pgfpathlineto{\pgfqpoint{3.622479in}{2.551655in}}%
\pgfpathlineto{\pgfqpoint{3.605793in}{2.586667in}}%
\pgfpathlineto{\pgfqpoint{3.572893in}{2.656566in}}%
\pgfpathlineto{\pgfqpoint{3.555368in}{2.693333in}}%
\pgfpathlineto{\pgfqpoint{3.523306in}{2.761458in}}%
\pgfpathlineto{\pgfqpoint{3.504934in}{2.800000in}}%
\pgfpathlineto{\pgfqpoint{3.473719in}{2.866329in}}%
\pgfpathlineto{\pgfqpoint{3.454489in}{2.906667in}}%
\pgfpathlineto{\pgfqpoint{3.424132in}{2.971179in}}%
\pgfpathlineto{\pgfqpoint{3.404035in}{3.013333in}}%
\pgfpathlineto{\pgfqpoint{3.374545in}{3.076007in}}%
\pgfpathlineto{\pgfqpoint{3.353570in}{3.120000in}}%
\pgfpathlineto{\pgfqpoint{3.324959in}{3.180812in}}%
\pgfpathlineto{\pgfqpoint{3.303093in}{3.226667in}}%
\pgfpathlineto{\pgfqpoint{3.275372in}{3.285593in}}%
\pgfpathlineto{\pgfqpoint{3.252605in}{3.333333in}}%
\pgfpathlineto{\pgfqpoint{3.225785in}{3.390349in}}%
\pgfpathlineto{\pgfqpoint{3.202106in}{3.440000in}}%
\pgfpathlineto{\pgfqpoint{3.176198in}{3.495080in}}%
\pgfpathlineto{\pgfqpoint{3.151593in}{3.546667in}}%
\pgfpathlineto{\pgfqpoint{3.126612in}{3.599784in}}%
\pgfpathlineto{\pgfqpoint{3.101068in}{3.653333in}}%
\pgfpathlineto{\pgfqpoint{3.077025in}{3.704460in}}%
\pgfpathlineto{\pgfqpoint{3.050528in}{3.760000in}}%
\pgfpathlineto{\pgfqpoint{3.027438in}{3.809106in}}%
\pgfpathlineto{\pgfqpoint{2.999975in}{3.866667in}}%
\pgfpathlineto{\pgfqpoint{2.977851in}{3.913722in}}%
\pgfpathlineto{\pgfqpoint{2.949407in}{3.973333in}}%
\pgfpathlineto{\pgfqpoint{2.928264in}{4.018306in}}%
\pgfpathlineto{\pgfqpoint{2.898823in}{4.080000in}}%
\pgfpathlineto{\pgfqpoint{2.878678in}{4.122857in}}%
\pgfpathlineto{\pgfqpoint{2.848222in}{4.186667in}}%
\pgfpathlineto{\pgfqpoint{2.829091in}{4.227372in}}%
\pgfpathlineto{\pgfqpoint{2.797605in}{4.293333in}}%
\pgfpathlineto{\pgfqpoint{2.779504in}{4.331851in}}%
\pgfpathlineto{\pgfqpoint{2.746970in}{4.400000in}}%
\pgfpathlineto{\pgfqpoint{2.729917in}{4.436291in}}%
\pgfpathlineto{\pgfqpoint{2.696316in}{4.506667in}}%
\pgfpathlineto{\pgfqpoint{2.680331in}{4.540692in}}%
\pgfpathlineto{\pgfqpoint{2.645643in}{4.613333in}}%
\pgfpathlineto{\pgfqpoint{2.630744in}{4.645050in}}%
\pgfpathlineto{\pgfqpoint{2.594949in}{4.720000in}}%
\pgfpathlineto{\pgfqpoint{2.581157in}{4.749363in}}%
\pgfpathlineto{\pgfqpoint{2.544233in}{4.826667in}}%
\pgfpathlineto{\pgfqpoint{2.531570in}{4.853630in}}%
\pgfpathlineto{\pgfqpoint{2.493495in}{4.933333in}}%
\pgfpathlineto{\pgfqpoint{2.481983in}{4.957848in}}%
\pgfpathlineto{\pgfqpoint{2.442732in}{5.040000in}}%
\pgfpathlineto{\pgfqpoint{2.432397in}{5.062014in}}%
\pgfpathlineto{\pgfqpoint{2.391944in}{5.146667in}}%
\pgfpathlineto{\pgfqpoint{2.382810in}{5.166126in}}%
\pgfpathlineto{\pgfqpoint{2.341130in}{5.253333in}}%
\pgfpathlineto{\pgfqpoint{2.333223in}{5.270180in}}%
\pgfpathlineto{\pgfqpoint{2.290288in}{5.360000in}}%
\pgfpathlineto{\pgfqpoint{2.283636in}{5.374174in}}%
\pgfpathlineto{\pgfqpoint{2.239415in}{5.466667in}}%
\pgfpathlineto{\pgfqpoint{2.234050in}{5.478103in}}%
\pgfpathlineto{\pgfqpoint{2.188512in}{5.573333in}}%
\pgfpathlineto{\pgfqpoint{2.184463in}{5.581965in}}%
\pgfpathlineto{\pgfqpoint{2.137575in}{5.680000in}}%
\pgfpathlineto{\pgfqpoint{2.134876in}{5.685754in}}%
\pgfpathlineto{\pgfqpoint{2.086603in}{5.786667in}}%
\pgfpathlineto{\pgfqpoint{2.085289in}{5.789468in}}%
\pgfpathlineto{\pgfqpoint{2.035702in}{5.893223in}}%
\pgfpathlineto{\pgfqpoint{2.035650in}{5.893333in}}%
\pgfpathlineto{\pgfqpoint{1.986116in}{5.998430in}}%
\pgfpathlineto{\pgfqpoint{1.985368in}{6.000000in}}%
\pgfpathlineto{\pgfqpoint{1.936529in}{6.103628in}}%
\pgfpathlineto{\pgfqpoint{1.935082in}{6.106667in}}%
\pgfpathlineto{\pgfqpoint{1.886942in}{6.208816in}}%
\pgfpathlineto{\pgfqpoint{1.884792in}{6.213333in}}%
\pgfpathlineto{\pgfqpoint{1.837355in}{6.313994in}}%
\pgfpathlineto{\pgfqpoint{1.834496in}{6.320000in}}%
\pgfpathlineto{\pgfqpoint{1.787769in}{6.419162in}}%
\pgfpathlineto{\pgfqpoint{1.784196in}{6.426667in}}%
\pgfpathlineto{\pgfqpoint{1.738182in}{6.524319in}}%
\pgfpathlineto{\pgfqpoint{1.733890in}{6.533333in}}%
\pgfpathlineto{\pgfqpoint{1.688595in}{6.629465in}}%
\pgfpathlineto{\pgfqpoint{1.683579in}{6.640000in}}%
\pgfpathlineto{\pgfqpoint{1.639008in}{6.734600in}}%
\pgfpathlineto{\pgfqpoint{1.633263in}{6.746667in}}%
\pgfpathlineto{\pgfqpoint{1.589421in}{6.839723in}}%
\pgfpathlineto{\pgfqpoint{1.582941in}{6.853333in}}%
\pgfpathlineto{\pgfqpoint{1.539835in}{6.944835in}}%
\pgfpathlineto{\pgfqpoint{1.532613in}{6.960000in}}%
\pgfpathlineto{\pgfqpoint{1.490248in}{7.049933in}}%
\pgfpathlineto{\pgfqpoint{1.482279in}{7.066667in}}%
\pgfpathlineto{\pgfqpoint{1.440661in}{7.155020in}}%
\pgfpathlineto{\pgfqpoint{1.431939in}{7.173333in}}%
\pgfpathlineto{\pgfqpoint{1.391074in}{7.260092in}}%
\pgfpathlineto{\pgfqpoint{1.381592in}{7.280000in}}%
\pgfpathlineto{\pgfqpoint{1.341488in}{7.365151in}}%
\pgfpathlineto{\pgfqpoint{1.331239in}{7.386667in}}%
\pgfpathlineto{\pgfqpoint{1.291901in}{7.470196in}}%
\pgfpathlineto{\pgfqpoint{1.280879in}{7.493333in}}%
\pgfpathlineto{\pgfqpoint{1.242314in}{7.575227in}}%
\pgfpathlineto{\pgfqpoint{1.230512in}{7.600000in}}%
\pgfpathlineto{\pgfqpoint{1.192727in}{7.680242in}}%
\pgfpathlineto{\pgfqpoint{1.192727in}{7.600000in}}%
\pgfpathlineto{\pgfqpoint{1.192727in}{7.493333in}}%
\pgfpathlineto{\pgfqpoint{1.192727in}{7.386667in}}%
\pgfpathlineto{\pgfqpoint{1.192727in}{7.352672in}}%
\pgfpathlineto{\pgfqpoint{1.227078in}{7.280000in}}%
\pgfpathlineto{\pgfqpoint{1.242314in}{7.247897in}}%
\pgfpathlineto{\pgfqpoint{1.277559in}{7.173333in}}%
\pgfpathlineto{\pgfqpoint{1.291901in}{7.143115in}}%
\pgfpathlineto{\pgfqpoint{1.328037in}{7.066667in}}%
\pgfpathlineto{\pgfqpoint{1.341488in}{7.038326in}}%
\pgfpathlineto{\pgfqpoint{1.378512in}{6.960000in}}%
\pgfpathlineto{\pgfqpoint{1.391074in}{6.933529in}}%
\pgfpathlineto{\pgfqpoint{1.428983in}{6.853333in}}%
\pgfpathlineto{\pgfqpoint{1.440661in}{6.828726in}}%
\pgfpathlineto{\pgfqpoint{1.479451in}{6.746667in}}%
\pgfpathlineto{\pgfqpoint{1.490248in}{6.723916in}}%
\pgfpathlineto{\pgfqpoint{1.529916in}{6.640000in}}%
\pgfpathlineto{\pgfqpoint{1.539835in}{6.619099in}}%
\pgfpathlineto{\pgfqpoint{1.580377in}{6.533333in}}%
\pgfpathlineto{\pgfqpoint{1.589421in}{6.514275in}}%
\pgfpathlineto{\pgfqpoint{1.630835in}{6.426667in}}%
\pgfpathlineto{\pgfqpoint{1.639008in}{6.409445in}}%
\pgfpathlineto{\pgfqpoint{1.681291in}{6.320000in}}%
\pgfpathlineto{\pgfqpoint{1.688595in}{6.304608in}}%
\pgfpathlineto{\pgfqpoint{1.731743in}{6.213333in}}%
\pgfpathlineto{\pgfqpoint{1.738182in}{6.199764in}}%
\pgfpathlineto{\pgfqpoint{1.782192in}{6.106667in}}%
\pgfpathlineto{\pgfqpoint{1.787769in}{6.094915in}}%
\pgfpathlineto{\pgfqpoint{1.832638in}{6.000000in}}%
\pgfpathlineto{\pgfqpoint{1.837355in}{5.990058in}}%
\pgfpathlineto{\pgfqpoint{1.883080in}{5.893333in}}%
\pgfpathlineto{\pgfqpoint{1.886942in}{5.885196in}}%
\pgfpathlineto{\pgfqpoint{1.933520in}{5.786667in}}%
\pgfpathlineto{\pgfqpoint{1.936529in}{5.780327in}}%
\pgfpathlineto{\pgfqpoint{1.983957in}{5.680000in}}%
\pgfpathlineto{\pgfqpoint{1.986116in}{5.675452in}}%
\pgfpathlineto{\pgfqpoint{2.034392in}{5.573333in}}%
\pgfpathlineto{\pgfqpoint{2.035702in}{5.570571in}}%
\pgfpathlineto{\pgfqpoint{2.084823in}{5.466667in}}%
\pgfpathlineto{\pgfqpoint{2.085289in}{5.465684in}}%
\pgfpathlineto{\pgfqpoint{2.134876in}{5.360992in}}%
\pgfpathlineto{\pgfqpoint{2.135345in}{5.360000in}}%
\pgfpathlineto{\pgfqpoint{2.184463in}{5.256539in}}%
\pgfpathlineto{\pgfqpoint{2.185978in}{5.253333in}}%
\pgfpathlineto{\pgfqpoint{2.234050in}{5.152075in}}%
\pgfpathlineto{\pgfqpoint{2.236605in}{5.146667in}}%
\pgfpathlineto{\pgfqpoint{2.283636in}{5.047599in}}%
\pgfpathlineto{\pgfqpoint{2.287227in}{5.040000in}}%
\pgfpathlineto{\pgfqpoint{2.333223in}{4.943113in}}%
\pgfpathlineto{\pgfqpoint{2.337844in}{4.933333in}}%
\pgfpathlineto{\pgfqpoint{2.382810in}{4.838615in}}%
\pgfpathlineto{\pgfqpoint{2.388456in}{4.826667in}}%
\pgfpathlineto{\pgfqpoint{2.432397in}{4.734107in}}%
\pgfpathlineto{\pgfqpoint{2.439063in}{4.720000in}}%
\pgfpathlineto{\pgfqpoint{2.481983in}{4.629588in}}%
\pgfpathlineto{\pgfqpoint{2.489665in}{4.613333in}}%
\pgfpathlineto{\pgfqpoint{2.531570in}{4.525058in}}%
\pgfpathlineto{\pgfqpoint{2.540262in}{4.506667in}}%
\pgfpathlineto{\pgfqpoint{2.581157in}{4.420518in}}%
\pgfpathlineto{\pgfqpoint{2.590853in}{4.400000in}}%
\pgfpathlineto{\pgfqpoint{2.630744in}{4.315968in}}%
\pgfpathlineto{\pgfqpoint{2.641441in}{4.293333in}}%
\pgfpathlineto{\pgfqpoint{2.680331in}{4.211408in}}%
\pgfpathlineto{\pgfqpoint{2.692023in}{4.186667in}}%
\pgfpathlineto{\pgfqpoint{2.729917in}{4.106838in}}%
\pgfpathlineto{\pgfqpoint{2.742601in}{4.080000in}}%
\pgfpathlineto{\pgfqpoint{2.779504in}{4.002258in}}%
\pgfpathlineto{\pgfqpoint{2.793174in}{3.973333in}}%
\pgfpathlineto{\pgfqpoint{2.829091in}{3.897668in}}%
\pgfpathlineto{\pgfqpoint{2.843742in}{3.866667in}}%
\pgfpathlineto{\pgfqpoint{2.878678in}{3.793069in}}%
\pgfpathlineto{\pgfqpoint{2.894306in}{3.760000in}}%
\pgfpathlineto{\pgfqpoint{2.928264in}{3.688460in}}%
\pgfpathlineto{\pgfqpoint{2.944866in}{3.653333in}}%
\pgfpathlineto{\pgfqpoint{2.977851in}{3.583842in}}%
\pgfpathlineto{\pgfqpoint{2.995421in}{3.546667in}}%
\pgfpathlineto{\pgfqpoint{3.027438in}{3.479215in}}%
\pgfpathlineto{\pgfqpoint{3.045971in}{3.440000in}}%
\pgfpathlineto{\pgfqpoint{3.077025in}{3.374578in}}%
\pgfpathlineto{\pgfqpoint{3.096518in}{3.333333in}}%
\pgfpathlineto{\pgfqpoint{3.126612in}{3.269933in}}%
\pgfpathlineto{\pgfqpoint{3.147060in}{3.226667in}}%
\pgfpathlineto{\pgfqpoint{3.176198in}{3.165279in}}%
\pgfpathlineto{\pgfqpoint{3.197598in}{3.120000in}}%
\pgfpathlineto{\pgfqpoint{3.225785in}{3.060616in}}%
\pgfpathlineto{\pgfqpoint{3.248132in}{3.013333in}}%
\pgfpathlineto{\pgfqpoint{3.275372in}{2.955944in}}%
\pgfpathlineto{\pgfqpoint{3.298662in}{2.906667in}}%
\pgfpathlineto{\pgfqpoint{3.324959in}{2.851264in}}%
\pgfpathlineto{\pgfqpoint{3.349188in}{2.800000in}}%
\pgfpathlineto{\pgfqpoint{3.374545in}{2.746575in}}%
\pgfpathlineto{\pgfqpoint{3.399710in}{2.693333in}}%
\pgfpathlineto{\pgfqpoint{3.424132in}{2.641879in}}%
\pgfpathlineto{\pgfqpoint{3.450228in}{2.586667in}}%
\pgfpathlineto{\pgfqpoint{3.473719in}{2.537174in}}%
\pgfpathlineto{\pgfqpoint{3.500742in}{2.480000in}}%
\pgfpathlineto{\pgfqpoint{3.523306in}{2.432461in}}%
\pgfpathlineto{\pgfqpoint{3.551253in}{2.373333in}}%
\pgfpathlineto{\pgfqpoint{3.572893in}{2.327739in}}%
\pgfpathlineto{\pgfqpoint{3.601759in}{2.266667in}}%
\pgfpathlineto{\pgfqpoint{3.622479in}{2.223010in}}%
\pgfpathlineto{\pgfqpoint{3.652262in}{2.160000in}}%
\pgfpathlineto{\pgfqpoint{3.672066in}{2.118274in}}%
\pgfpathlineto{\pgfqpoint{3.702761in}{2.053333in}}%
\pgfpathlineto{\pgfqpoint{3.721653in}{2.013529in}}%
\pgfpathlineto{\pgfqpoint{3.753257in}{1.946667in}}%
\pgfpathlineto{\pgfqpoint{3.771240in}{1.908777in}}%
\pgfpathlineto{\pgfqpoint{3.803749in}{1.840000in}}%
\pgfpathlineto{\pgfqpoint{3.820826in}{1.804017in}}%
\pgfpathlineto{\pgfqpoint{3.854237in}{1.733333in}}%
\pgfpathlineto{\pgfqpoint{3.870413in}{1.699250in}}%
\pgfpathlineto{\pgfqpoint{3.904722in}{1.626667in}}%
\pgfpathlineto{\pgfqpoint{3.920000in}{1.594476in}}%
\pgfpathlineto{\pgfqpoint{3.955204in}{1.520000in}}%
\pgfpathlineto{\pgfqpoint{3.969587in}{1.489694in}}%
\pgfpathlineto{\pgfqpoint{4.005682in}{1.413333in}}%
\pgfpathlineto{\pgfqpoint{4.019174in}{1.384905in}}%
\pgfpathlineto{\pgfqpoint{4.056157in}{1.306667in}}%
\pgfpathlineto{\pgfqpoint{4.068760in}{1.280109in}}%
\pgfpathlineto{\pgfqpoint{4.106628in}{1.200000in}}%
\pgfpathlineto{\pgfqpoint{4.118347in}{1.175306in}}%
\pgfpathlineto{\pgfqpoint{4.157096in}{1.093333in}}%
\pgfpathlineto{\pgfqpoint{4.167934in}{1.070497in}}%
\pgfpathlineto{\pgfqpoint{4.207561in}{0.986667in}}%
\pgfpathlineto{\pgfqpoint{4.217521in}{0.965680in}}%
\pgfpathlineto{\pgfqpoint{4.258023in}{0.880000in}}%
\pgfpathlineto{\pgfqpoint{4.267107in}{0.860856in}}%
\pgfpathlineto{\pgfqpoint{4.308481in}{0.773333in}}%
\pgfpathlineto{\pgfqpoint{4.316694in}{0.756026in}}%
\pgfpathlineto{\pgfqpoint{4.358936in}{0.666667in}}%
\pgfpathlineto{\pgfqpoint{4.366281in}{0.651190in}}%
\pgfpathlineto{\pgfqpoint{4.409388in}{0.560000in}}%
\pgfpathlineto{\pgfqpoint{4.415868in}{0.546347in}}%
\pgfpathlineto{\pgfqpoint{4.459838in}{0.453333in}}%
\pgfpathlineto{\pgfqpoint{4.465455in}{0.441497in}}%
\pgfpathlineto{\pgfqpoint{4.510284in}{0.346667in}}%
\pgfpathlineto{\pgfqpoint{4.515041in}{0.336641in}}%
\pgfpathlineto{\pgfqpoint{4.560727in}{0.240000in}}%
\pgfpathlineto{\pgfqpoint{4.564628in}{0.231779in}}%
\pgfpathlineto{\pgfqpoint{4.611167in}{0.133333in}}%
\pgfpathlineto{\pgfqpoint{4.614215in}{0.126910in}}%
\pgfpathlineto{\pgfqpoint{4.661604in}{0.026667in}}%
\pgfpathlineto{\pgfqpoint{4.663802in}{0.022035in}}%
\pgfpathlineto{\pgfqpoint{4.712038in}{-0.080000in}}%
\pgfpathlineto{\pgfqpoint{4.713388in}{-0.082845in}}%
\pgfpathlineto{\pgfqpoint{4.762470in}{-0.186667in}}%
\pgfpathlineto{\pgfqpoint{4.762975in}{-0.187732in}}%
\pgfpathlineto{\pgfqpoint{4.812562in}{-0.292445in}}%
\pgfpathlineto{\pgfqpoint{4.812982in}{-0.293333in}}%
\pgfpathlineto{\pgfqpoint{4.862149in}{-0.396897in}}%
\pgfpathlineto{\pgfqpoint{4.863615in}{-0.400000in}}%
\pgfpathlineto{\pgfqpoint{4.911736in}{-0.501361in}}%
\pgfpathlineto{\pgfqpoint{4.914243in}{-0.506667in}}%
\pgfpathlineto{\pgfqpoint{4.961322in}{-0.605836in}}%
\pgfpathlineto{\pgfqpoint{4.964865in}{-0.613333in}}%
\pgfpathlineto{\pgfqpoint{5.010909in}{-0.710322in}}%
\pgfpathlineto{\pgfqpoint{5.015483in}{-0.720000in}}%
\pgfpathlineto{\pgfqpoint{5.060496in}{-0.814819in}}%
\pgfpathlineto{\pgfqpoint{5.066095in}{-0.826667in}}%
\pgfpathlineto{\pgfqpoint{5.110083in}{-0.919327in}}%
\pgfpathlineto{\pgfqpoint{5.116702in}{-0.933333in}}%
\pgfpathlineto{\pgfqpoint{5.159669in}{-1.023845in}}%
\pgfpathlineto{\pgfqpoint{5.167304in}{-1.040000in}}%
\pgfpathlineto{\pgfqpoint{5.209256in}{-1.128374in}}%
\pgfpathlineto{\pgfqpoint{5.217901in}{-1.146667in}}%
\pgfpathlineto{\pgfqpoint{5.258843in}{-1.232914in}}%
\pgfpathlineto{\pgfqpoint{5.268493in}{-1.253333in}}%
\pgfpathlineto{\pgfqpoint{5.308430in}{-1.337463in}}%
\pgfpathlineto{\pgfqpoint{5.319080in}{-1.360000in}}%
\pgfpathlineto{\pgfqpoint{5.358017in}{-1.442023in}}%
\pgfpathlineto{\pgfqpoint{5.369663in}{-1.466667in}}%
\pgfpathlineto{\pgfqpoint{5.407603in}{-1.546593in}}%
\pgfpathlineto{\pgfqpoint{5.420241in}{-1.573333in}}%
\pgfpathlineto{\pgfqpoint{5.457190in}{-1.651172in}}%
\pgfpathlineto{\pgfqpoint{5.470814in}{-1.680000in}}%
\pgfpathlineto{\pgfqpoint{5.506777in}{-1.755762in}}%
\pgfpathlineto{\pgfqpoint{5.521382in}{-1.786667in}}%
\pgfpathlineto{\pgfqpoint{5.556364in}{-1.860361in}}%
\pgfpathlineto{\pgfqpoint{5.571947in}{-1.893333in}}%
\pgfpathlineto{\pgfqpoint{5.605950in}{-1.964969in}}%
\pgfpathlineto{\pgfqpoint{5.622506in}{-2.000000in}}%
\pgfpathlineto{\pgfqpoint{5.655537in}{-2.069587in}}%
\pgfpathlineto{\pgfqpoint{5.673062in}{-2.106667in}}%
\pgfpathlineto{\pgfqpoint{5.705124in}{-2.174213in}}%
\pgfpathlineto{\pgfqpoint{5.723613in}{-2.213333in}}%
\pgfpathlineto{\pgfqpoint{5.754711in}{-2.278849in}}%
\pgfpathlineto{\pgfqpoint{5.774159in}{-2.320000in}}%
\pgfpathlineto{\pgfqpoint{5.804298in}{-2.383494in}}%
\pgfpathlineto{\pgfqpoint{5.824702in}{-2.426667in}}%
\pgfpathlineto{\pgfqpoint{5.853884in}{-2.488148in}}%
\pgfpathlineto{\pgfqpoint{5.875240in}{-2.533333in}}%
\pgfpathclose%
\pgfusepath{fill}%
\end{pgfscope}%
\begin{pgfscope}%
\pgfpathrectangle{\pgfqpoint{3.156364in}{0.528000in}}{\pgfqpoint{1.963636in}{3.696000in}} %
\pgfusepath{clip}%
\pgfsetbuttcap%
\pgfsetroundjoin%
\definecolor{currentfill}{rgb}{0.423689,0.000646,0.658956}%
\pgfsetfillcolor{currentfill}%
\pgfsetlinewidth{0.000000pt}%
\definecolor{currentstroke}{rgb}{0.000000,0.000000,0.000000}%
\pgfsetstrokecolor{currentstroke}%
\pgfsetdash{}{0pt}%
\pgfpathmoveto{\pgfqpoint{6.101818in}{-2.400242in}}%
\pgfpathlineto{\pgfqpoint{6.101818in}{-2.320000in}}%
\pgfpathlineto{\pgfqpoint{6.101818in}{-2.213333in}}%
\pgfpathlineto{\pgfqpoint{6.101818in}{-2.106667in}}%
\pgfpathlineto{\pgfqpoint{6.101818in}{-2.072672in}}%
\pgfpathlineto{\pgfqpoint{6.067467in}{-2.000000in}}%
\pgfpathlineto{\pgfqpoint{6.052231in}{-1.967897in}}%
\pgfpathlineto{\pgfqpoint{6.016986in}{-1.893333in}}%
\pgfpathlineto{\pgfqpoint{6.002645in}{-1.863115in}}%
\pgfpathlineto{\pgfqpoint{5.966508in}{-1.786667in}}%
\pgfpathlineto{\pgfqpoint{5.953058in}{-1.758326in}}%
\pgfpathlineto{\pgfqpoint{5.916034in}{-1.680000in}}%
\pgfpathlineto{\pgfqpoint{5.903471in}{-1.653529in}}%
\pgfpathlineto{\pgfqpoint{5.865562in}{-1.573333in}}%
\pgfpathlineto{\pgfqpoint{5.853884in}{-1.548726in}}%
\pgfpathlineto{\pgfqpoint{5.815094in}{-1.466667in}}%
\pgfpathlineto{\pgfqpoint{5.804298in}{-1.443916in}}%
\pgfpathlineto{\pgfqpoint{5.764630in}{-1.360000in}}%
\pgfpathlineto{\pgfqpoint{5.754711in}{-1.339099in}}%
\pgfpathlineto{\pgfqpoint{5.714168in}{-1.253333in}}%
\pgfpathlineto{\pgfqpoint{5.705124in}{-1.234275in}}%
\pgfpathlineto{\pgfqpoint{5.663710in}{-1.146667in}}%
\pgfpathlineto{\pgfqpoint{5.655537in}{-1.129445in}}%
\pgfpathlineto{\pgfqpoint{5.613255in}{-1.040000in}}%
\pgfpathlineto{\pgfqpoint{5.605950in}{-1.024608in}}%
\pgfpathlineto{\pgfqpoint{5.562803in}{-0.933333in}}%
\pgfpathlineto{\pgfqpoint{5.556364in}{-0.919764in}}%
\pgfpathlineto{\pgfqpoint{5.512354in}{-0.826667in}}%
\pgfpathlineto{\pgfqpoint{5.506777in}{-0.814915in}}%
\pgfpathlineto{\pgfqpoint{5.461908in}{-0.720000in}}%
\pgfpathlineto{\pgfqpoint{5.457190in}{-0.710058in}}%
\pgfpathlineto{\pgfqpoint{5.411465in}{-0.613333in}}%
\pgfpathlineto{\pgfqpoint{5.407603in}{-0.605196in}}%
\pgfpathlineto{\pgfqpoint{5.361025in}{-0.506667in}}%
\pgfpathlineto{\pgfqpoint{5.358017in}{-0.500327in}}%
\pgfpathlineto{\pgfqpoint{5.310588in}{-0.400000in}}%
\pgfpathlineto{\pgfqpoint{5.308430in}{-0.395452in}}%
\pgfpathlineto{\pgfqpoint{5.260154in}{-0.293333in}}%
\pgfpathlineto{\pgfqpoint{5.258843in}{-0.290571in}}%
\pgfpathlineto{\pgfqpoint{5.209723in}{-0.186667in}}%
\pgfpathlineto{\pgfqpoint{5.209256in}{-0.185684in}}%
\pgfpathlineto{\pgfqpoint{5.159669in}{-0.080992in}}%
\pgfpathlineto{\pgfqpoint{5.159201in}{-0.080000in}}%
\pgfpathlineto{\pgfqpoint{5.110083in}{0.023461in}}%
\pgfpathlineto{\pgfqpoint{5.108568in}{0.026667in}}%
\pgfpathlineto{\pgfqpoint{5.060496in}{0.127925in}}%
\pgfpathlineto{\pgfqpoint{5.057940in}{0.133333in}}%
\pgfpathlineto{\pgfqpoint{5.010909in}{0.232401in}}%
\pgfpathlineto{\pgfqpoint{5.007318in}{0.240000in}}%
\pgfpathlineto{\pgfqpoint{4.961322in}{0.336887in}}%
\pgfpathlineto{\pgfqpoint{4.956701in}{0.346667in}}%
\pgfpathlineto{\pgfqpoint{4.911736in}{0.441385in}}%
\pgfpathlineto{\pgfqpoint{4.906089in}{0.453333in}}%
\pgfpathlineto{\pgfqpoint{4.862149in}{0.545893in}}%
\pgfpathlineto{\pgfqpoint{4.855482in}{0.560000in}}%
\pgfpathlineto{\pgfqpoint{4.812562in}{0.650412in}}%
\pgfpathlineto{\pgfqpoint{4.804881in}{0.666667in}}%
\pgfpathlineto{\pgfqpoint{4.762975in}{0.754942in}}%
\pgfpathlineto{\pgfqpoint{4.754284in}{0.773333in}}%
\pgfpathlineto{\pgfqpoint{4.713388in}{0.859482in}}%
\pgfpathlineto{\pgfqpoint{4.703692in}{0.880000in}}%
\pgfpathlineto{\pgfqpoint{4.663802in}{0.964032in}}%
\pgfpathlineto{\pgfqpoint{4.653105in}{0.986667in}}%
\pgfpathlineto{\pgfqpoint{4.614215in}{1.068592in}}%
\pgfpathlineto{\pgfqpoint{4.602523in}{1.093333in}}%
\pgfpathlineto{\pgfqpoint{4.564628in}{1.173162in}}%
\pgfpathlineto{\pgfqpoint{4.551945in}{1.200000in}}%
\pgfpathlineto{\pgfqpoint{4.515041in}{1.277742in}}%
\pgfpathlineto{\pgfqpoint{4.501372in}{1.306667in}}%
\pgfpathlineto{\pgfqpoint{4.465455in}{1.382332in}}%
\pgfpathlineto{\pgfqpoint{4.450803in}{1.413333in}}%
\pgfpathlineto{\pgfqpoint{4.415868in}{1.486931in}}%
\pgfpathlineto{\pgfqpoint{4.400239in}{1.520000in}}%
\pgfpathlineto{\pgfqpoint{4.366281in}{1.591540in}}%
\pgfpathlineto{\pgfqpoint{4.349680in}{1.626667in}}%
\pgfpathlineto{\pgfqpoint{4.316694in}{1.696158in}}%
\pgfpathlineto{\pgfqpoint{4.299125in}{1.733333in}}%
\pgfpathlineto{\pgfqpoint{4.267107in}{1.800785in}}%
\pgfpathlineto{\pgfqpoint{4.248574in}{1.840000in}}%
\pgfpathlineto{\pgfqpoint{4.217521in}{1.905422in}}%
\pgfpathlineto{\pgfqpoint{4.198028in}{1.946667in}}%
\pgfpathlineto{\pgfqpoint{4.167934in}{2.010067in}}%
\pgfpathlineto{\pgfqpoint{4.147485in}{2.053333in}}%
\pgfpathlineto{\pgfqpoint{4.118347in}{2.114721in}}%
\pgfpathlineto{\pgfqpoint{4.096947in}{2.160000in}}%
\pgfpathlineto{\pgfqpoint{4.068760in}{2.219384in}}%
\pgfpathlineto{\pgfqpoint{4.046413in}{2.266667in}}%
\pgfpathlineto{\pgfqpoint{4.019174in}{2.324056in}}%
\pgfpathlineto{\pgfqpoint{3.995883in}{2.373333in}}%
\pgfpathlineto{\pgfqpoint{3.969587in}{2.428736in}}%
\pgfpathlineto{\pgfqpoint{3.945357in}{2.480000in}}%
\pgfpathlineto{\pgfqpoint{3.920000in}{2.533425in}}%
\pgfpathlineto{\pgfqpoint{3.894835in}{2.586667in}}%
\pgfpathlineto{\pgfqpoint{3.870413in}{2.638121in}}%
\pgfpathlineto{\pgfqpoint{3.844317in}{2.693333in}}%
\pgfpathlineto{\pgfqpoint{3.820826in}{2.742826in}}%
\pgfpathlineto{\pgfqpoint{3.793803in}{2.800000in}}%
\pgfpathlineto{\pgfqpoint{3.771240in}{2.847539in}}%
\pgfpathlineto{\pgfqpoint{3.743293in}{2.906667in}}%
\pgfpathlineto{\pgfqpoint{3.721653in}{2.952261in}}%
\pgfpathlineto{\pgfqpoint{3.692786in}{3.013333in}}%
\pgfpathlineto{\pgfqpoint{3.672066in}{3.056990in}}%
\pgfpathlineto{\pgfqpoint{3.642283in}{3.120000in}}%
\pgfpathlineto{\pgfqpoint{3.622479in}{3.161726in}}%
\pgfpathlineto{\pgfqpoint{3.591784in}{3.226667in}}%
\pgfpathlineto{\pgfqpoint{3.572893in}{3.266471in}}%
\pgfpathlineto{\pgfqpoint{3.541288in}{3.333333in}}%
\pgfpathlineto{\pgfqpoint{3.523306in}{3.371223in}}%
\pgfpathlineto{\pgfqpoint{3.490796in}{3.440000in}}%
\pgfpathlineto{\pgfqpoint{3.473719in}{3.475983in}}%
\pgfpathlineto{\pgfqpoint{3.440308in}{3.546667in}}%
\pgfpathlineto{\pgfqpoint{3.424132in}{3.580750in}}%
\pgfpathlineto{\pgfqpoint{3.389823in}{3.653333in}}%
\pgfpathlineto{\pgfqpoint{3.374545in}{3.685524in}}%
\pgfpathlineto{\pgfqpoint{3.339342in}{3.760000in}}%
\pgfpathlineto{\pgfqpoint{3.324959in}{3.790306in}}%
\pgfpathlineto{\pgfqpoint{3.288863in}{3.866667in}}%
\pgfpathlineto{\pgfqpoint{3.275372in}{3.895095in}}%
\pgfpathlineto{\pgfqpoint{3.238389in}{3.973333in}}%
\pgfpathlineto{\pgfqpoint{3.225785in}{3.999891in}}%
\pgfpathlineto{\pgfqpoint{3.187917in}{4.080000in}}%
\pgfpathlineto{\pgfqpoint{3.176198in}{4.104694in}}%
\pgfpathlineto{\pgfqpoint{3.137449in}{4.186667in}}%
\pgfpathlineto{\pgfqpoint{3.126612in}{4.209503in}}%
\pgfpathlineto{\pgfqpoint{3.086985in}{4.293333in}}%
\pgfpathlineto{\pgfqpoint{3.077025in}{4.314320in}}%
\pgfpathlineto{\pgfqpoint{3.036523in}{4.400000in}}%
\pgfpathlineto{\pgfqpoint{3.027438in}{4.419144in}}%
\pgfpathlineto{\pgfqpoint{2.986064in}{4.506667in}}%
\pgfpathlineto{\pgfqpoint{2.977851in}{4.523974in}}%
\pgfpathlineto{\pgfqpoint{2.935609in}{4.613333in}}%
\pgfpathlineto{\pgfqpoint{2.928264in}{4.628810in}}%
\pgfpathlineto{\pgfqpoint{2.885157in}{4.720000in}}%
\pgfpathlineto{\pgfqpoint{2.878678in}{4.733653in}}%
\pgfpathlineto{\pgfqpoint{2.834708in}{4.826667in}}%
\pgfpathlineto{\pgfqpoint{2.829091in}{4.838503in}}%
\pgfpathlineto{\pgfqpoint{2.784262in}{4.933333in}}%
\pgfpathlineto{\pgfqpoint{2.779504in}{4.943359in}}%
\pgfpathlineto{\pgfqpoint{2.733819in}{5.040000in}}%
\pgfpathlineto{\pgfqpoint{2.729917in}{5.048221in}}%
\pgfpathlineto{\pgfqpoint{2.683379in}{5.146667in}}%
\pgfpathlineto{\pgfqpoint{2.680331in}{5.153090in}}%
\pgfpathlineto{\pgfqpoint{2.632942in}{5.253333in}}%
\pgfpathlineto{\pgfqpoint{2.630744in}{5.257965in}}%
\pgfpathlineto{\pgfqpoint{2.582507in}{5.360000in}}%
\pgfpathlineto{\pgfqpoint{2.581157in}{5.362845in}}%
\pgfpathlineto{\pgfqpoint{2.532076in}{5.466667in}}%
\pgfpathlineto{\pgfqpoint{2.531570in}{5.467732in}}%
\pgfpathlineto{\pgfqpoint{2.481983in}{5.572445in}}%
\pgfpathlineto{\pgfqpoint{2.481564in}{5.573333in}}%
\pgfpathlineto{\pgfqpoint{2.432397in}{5.676897in}}%
\pgfpathlineto{\pgfqpoint{2.430930in}{5.680000in}}%
\pgfpathlineto{\pgfqpoint{2.382810in}{5.781361in}}%
\pgfpathlineto{\pgfqpoint{2.380303in}{5.786667in}}%
\pgfpathlineto{\pgfqpoint{2.333223in}{5.885836in}}%
\pgfpathlineto{\pgfqpoint{2.329680in}{5.893333in}}%
\pgfpathlineto{\pgfqpoint{2.283636in}{5.990322in}}%
\pgfpathlineto{\pgfqpoint{2.279063in}{6.000000in}}%
\pgfpathlineto{\pgfqpoint{2.234050in}{6.094819in}}%
\pgfpathlineto{\pgfqpoint{2.228451in}{6.106667in}}%
\pgfpathlineto{\pgfqpoint{2.184463in}{6.199327in}}%
\pgfpathlineto{\pgfqpoint{2.177844in}{6.213333in}}%
\pgfpathlineto{\pgfqpoint{2.134876in}{6.303845in}}%
\pgfpathlineto{\pgfqpoint{2.127242in}{6.320000in}}%
\pgfpathlineto{\pgfqpoint{2.085289in}{6.408374in}}%
\pgfpathlineto{\pgfqpoint{2.076645in}{6.426667in}}%
\pgfpathlineto{\pgfqpoint{2.035702in}{6.512914in}}%
\pgfpathlineto{\pgfqpoint{2.026053in}{6.533333in}}%
\pgfpathlineto{\pgfqpoint{1.986116in}{6.617463in}}%
\pgfpathlineto{\pgfqpoint{1.975465in}{6.640000in}}%
\pgfpathlineto{\pgfqpoint{1.936529in}{6.722023in}}%
\pgfpathlineto{\pgfqpoint{1.924883in}{6.746667in}}%
\pgfpathlineto{\pgfqpoint{1.886942in}{6.826593in}}%
\pgfpathlineto{\pgfqpoint{1.874305in}{6.853333in}}%
\pgfpathlineto{\pgfqpoint{1.837355in}{6.931172in}}%
\pgfpathlineto{\pgfqpoint{1.823732in}{6.960000in}}%
\pgfpathlineto{\pgfqpoint{1.787769in}{7.035762in}}%
\pgfpathlineto{\pgfqpoint{1.773163in}{7.066667in}}%
\pgfpathlineto{\pgfqpoint{1.738182in}{7.140361in}}%
\pgfpathlineto{\pgfqpoint{1.722599in}{7.173333in}}%
\pgfpathlineto{\pgfqpoint{1.688595in}{7.244969in}}%
\pgfpathlineto{\pgfqpoint{1.672039in}{7.280000in}}%
\pgfpathlineto{\pgfqpoint{1.639008in}{7.349587in}}%
\pgfpathlineto{\pgfqpoint{1.621484in}{7.386667in}}%
\pgfpathlineto{\pgfqpoint{1.589421in}{7.454213in}}%
\pgfpathlineto{\pgfqpoint{1.570933in}{7.493333in}}%
\pgfpathlineto{\pgfqpoint{1.539835in}{7.558849in}}%
\pgfpathlineto{\pgfqpoint{1.520386in}{7.600000in}}%
\pgfpathlineto{\pgfqpoint{1.490248in}{7.663494in}}%
\pgfpathlineto{\pgfqpoint{1.469844in}{7.706667in}}%
\pgfpathlineto{\pgfqpoint{1.440661in}{7.768148in}}%
\pgfpathlineto{\pgfqpoint{1.419305in}{7.813333in}}%
\pgfpathlineto{\pgfqpoint{1.391074in}{7.872811in}}%
\pgfpathlineto{\pgfqpoint{1.368771in}{7.920000in}}%
\pgfpathlineto{\pgfqpoint{1.341488in}{7.920000in}}%
\pgfpathlineto{\pgfqpoint{1.291901in}{7.920000in}}%
\pgfpathlineto{\pgfqpoint{1.242314in}{7.920000in}}%
\pgfpathlineto{\pgfqpoint{1.212865in}{7.920000in}}%
\pgfpathlineto{\pgfqpoint{1.242314in}{7.857413in}}%
\pgfpathlineto{\pgfqpoint{1.263331in}{7.813333in}}%
\pgfpathlineto{\pgfqpoint{1.291901in}{7.752609in}}%
\pgfpathlineto{\pgfqpoint{1.313808in}{7.706667in}}%
\pgfpathlineto{\pgfqpoint{1.341488in}{7.647829in}}%
\pgfpathlineto{\pgfqpoint{1.364296in}{7.600000in}}%
\pgfpathlineto{\pgfqpoint{1.391074in}{7.543073in}}%
\pgfpathlineto{\pgfqpoint{1.414797in}{7.493333in}}%
\pgfpathlineto{\pgfqpoint{1.440661in}{7.438344in}}%
\pgfpathlineto{\pgfqpoint{1.465310in}{7.386667in}}%
\pgfpathlineto{\pgfqpoint{1.490248in}{7.333641in}}%
\pgfpathlineto{\pgfqpoint{1.515836in}{7.280000in}}%
\pgfpathlineto{\pgfqpoint{1.539835in}{7.228967in}}%
\pgfpathlineto{\pgfqpoint{1.566376in}{7.173333in}}%
\pgfpathlineto{\pgfqpoint{1.589421in}{7.124322in}}%
\pgfpathlineto{\pgfqpoint{1.616930in}{7.066667in}}%
\pgfpathlineto{\pgfqpoint{1.639008in}{7.019707in}}%
\pgfpathlineto{\pgfqpoint{1.667499in}{6.960000in}}%
\pgfpathlineto{\pgfqpoint{1.688595in}{6.915125in}}%
\pgfpathlineto{\pgfqpoint{1.718084in}{6.853333in}}%
\pgfpathlineto{\pgfqpoint{1.738182in}{6.810576in}}%
\pgfpathlineto{\pgfqpoint{1.768685in}{6.746667in}}%
\pgfpathlineto{\pgfqpoint{1.787769in}{6.706062in}}%
\pgfpathlineto{\pgfqpoint{1.819303in}{6.640000in}}%
\pgfpathlineto{\pgfqpoint{1.837355in}{6.601585in}}%
\pgfpathlineto{\pgfqpoint{1.869939in}{6.533333in}}%
\pgfpathlineto{\pgfqpoint{1.886942in}{6.497146in}}%
\pgfpathlineto{\pgfqpoint{1.920593in}{6.426667in}}%
\pgfpathlineto{\pgfqpoint{1.936529in}{6.392748in}}%
\pgfpathlineto{\pgfqpoint{1.971268in}{6.320000in}}%
\pgfpathlineto{\pgfqpoint{1.986116in}{6.288392in}}%
\pgfpathlineto{\pgfqpoint{2.021963in}{6.213333in}}%
\pgfpathlineto{\pgfqpoint{2.035702in}{6.184081in}}%
\pgfpathlineto{\pgfqpoint{2.072679in}{6.106667in}}%
\pgfpathlineto{\pgfqpoint{2.085289in}{6.079816in}}%
\pgfpathlineto{\pgfqpoint{2.123419in}{6.000000in}}%
\pgfpathlineto{\pgfqpoint{2.134876in}{5.975600in}}%
\pgfpathlineto{\pgfqpoint{2.174183in}{5.893333in}}%
\pgfpathlineto{\pgfqpoint{2.184463in}{5.871437in}}%
\pgfpathlineto{\pgfqpoint{2.224972in}{5.786667in}}%
\pgfpathlineto{\pgfqpoint{2.234050in}{5.767327in}}%
\pgfpathlineto{\pgfqpoint{2.275787in}{5.680000in}}%
\pgfpathlineto{\pgfqpoint{2.283636in}{5.663276in}}%
\pgfpathlineto{\pgfqpoint{2.326631in}{5.573333in}}%
\pgfpathlineto{\pgfqpoint{2.333223in}{5.559285in}}%
\pgfpathlineto{\pgfqpoint{2.377505in}{5.466667in}}%
\pgfpathlineto{\pgfqpoint{2.382810in}{5.455359in}}%
\pgfpathlineto{\pgfqpoint{2.428410in}{5.360000in}}%
\pgfpathlineto{\pgfqpoint{2.432397in}{5.351501in}}%
\pgfpathlineto{\pgfqpoint{2.479348in}{5.253333in}}%
\pgfpathlineto{\pgfqpoint{2.481983in}{5.247715in}}%
\pgfpathlineto{\pgfqpoint{2.530322in}{5.146667in}}%
\pgfpathlineto{\pgfqpoint{2.531570in}{5.144005in}}%
\pgfpathlineto{\pgfqpoint{2.581157in}{5.040178in}}%
\pgfpathlineto{\pgfqpoint{2.581242in}{5.040000in}}%
\pgfpathlineto{\pgfqpoint{2.630744in}{4.934971in}}%
\pgfpathlineto{\pgfqpoint{2.631524in}{4.933333in}}%
\pgfpathlineto{\pgfqpoint{2.680331in}{4.829774in}}%
\pgfpathlineto{\pgfqpoint{2.681810in}{4.826667in}}%
\pgfpathlineto{\pgfqpoint{2.729917in}{4.724587in}}%
\pgfpathlineto{\pgfqpoint{2.732101in}{4.720000in}}%
\pgfpathlineto{\pgfqpoint{2.779504in}{4.619409in}}%
\pgfpathlineto{\pgfqpoint{2.782397in}{4.613333in}}%
\pgfpathlineto{\pgfqpoint{2.829091in}{4.514242in}}%
\pgfpathlineto{\pgfqpoint{2.832697in}{4.506667in}}%
\pgfpathlineto{\pgfqpoint{2.878678in}{4.409085in}}%
\pgfpathlineto{\pgfqpoint{2.883003in}{4.400000in}}%
\pgfpathlineto{\pgfqpoint{2.928264in}{4.303939in}}%
\pgfpathlineto{\pgfqpoint{2.933314in}{4.293333in}}%
\pgfpathlineto{\pgfqpoint{2.977851in}{4.198805in}}%
\pgfpathlineto{\pgfqpoint{2.983631in}{4.186667in}}%
\pgfpathlineto{\pgfqpoint{3.027438in}{4.093682in}}%
\pgfpathlineto{\pgfqpoint{3.033953in}{4.080000in}}%
\pgfpathlineto{\pgfqpoint{3.077025in}{3.988571in}}%
\pgfpathlineto{\pgfqpoint{3.084281in}{3.973333in}}%
\pgfpathlineto{\pgfqpoint{3.126612in}{3.883473in}}%
\pgfpathlineto{\pgfqpoint{3.134615in}{3.866667in}}%
\pgfpathlineto{\pgfqpoint{3.176198in}{3.778388in}}%
\pgfpathlineto{\pgfqpoint{3.184956in}{3.760000in}}%
\pgfpathlineto{\pgfqpoint{3.225785in}{3.673315in}}%
\pgfpathlineto{\pgfqpoint{3.235303in}{3.653333in}}%
\pgfpathlineto{\pgfqpoint{3.275372in}{3.568257in}}%
\pgfpathlineto{\pgfqpoint{3.285656in}{3.546667in}}%
\pgfpathlineto{\pgfqpoint{3.324959in}{3.463213in}}%
\pgfpathlineto{\pgfqpoint{3.336016in}{3.440000in}}%
\pgfpathlineto{\pgfqpoint{3.374545in}{3.358183in}}%
\pgfpathlineto{\pgfqpoint{3.386384in}{3.333333in}}%
\pgfpathlineto{\pgfqpoint{3.424132in}{3.253169in}}%
\pgfpathlineto{\pgfqpoint{3.436758in}{3.226667in}}%
\pgfpathlineto{\pgfqpoint{3.473719in}{3.148170in}}%
\pgfpathlineto{\pgfqpoint{3.487141in}{3.120000in}}%
\pgfpathlineto{\pgfqpoint{3.523306in}{3.043188in}}%
\pgfpathlineto{\pgfqpoint{3.537531in}{3.013333in}}%
\pgfpathlineto{\pgfqpoint{3.572893in}{2.938222in}}%
\pgfpathlineto{\pgfqpoint{3.587930in}{2.906667in}}%
\pgfpathlineto{\pgfqpoint{3.622479in}{2.833274in}}%
\pgfpathlineto{\pgfqpoint{3.638337in}{2.800000in}}%
\pgfpathlineto{\pgfqpoint{3.672066in}{2.728345in}}%
\pgfpathlineto{\pgfqpoint{3.688752in}{2.693333in}}%
\pgfpathlineto{\pgfqpoint{3.721653in}{2.623434in}}%
\pgfpathlineto{\pgfqpoint{3.739177in}{2.586667in}}%
\pgfpathlineto{\pgfqpoint{3.771240in}{2.518542in}}%
\pgfpathlineto{\pgfqpoint{3.789612in}{2.480000in}}%
\pgfpathlineto{\pgfqpoint{3.820826in}{2.413671in}}%
\pgfpathlineto{\pgfqpoint{3.840056in}{2.373333in}}%
\pgfpathlineto{\pgfqpoint{3.870413in}{2.308821in}}%
\pgfpathlineto{\pgfqpoint{3.890511in}{2.266667in}}%
\pgfpathlineto{\pgfqpoint{3.920000in}{2.203993in}}%
\pgfpathlineto{\pgfqpoint{3.940976in}{2.160000in}}%
\pgfpathlineto{\pgfqpoint{3.969587in}{2.099188in}}%
\pgfpathlineto{\pgfqpoint{3.991452in}{2.053333in}}%
\pgfpathlineto{\pgfqpoint{4.019174in}{1.994407in}}%
\pgfpathlineto{\pgfqpoint{4.041940in}{1.946667in}}%
\pgfpathlineto{\pgfqpoint{4.068760in}{1.889651in}}%
\pgfpathlineto{\pgfqpoint{4.092440in}{1.840000in}}%
\pgfpathlineto{\pgfqpoint{4.118347in}{1.784920in}}%
\pgfpathlineto{\pgfqpoint{4.142952in}{1.733333in}}%
\pgfpathlineto{\pgfqpoint{4.167934in}{1.680216in}}%
\pgfpathlineto{\pgfqpoint{4.193478in}{1.626667in}}%
\pgfpathlineto{\pgfqpoint{4.217521in}{1.575540in}}%
\pgfpathlineto{\pgfqpoint{4.244017in}{1.520000in}}%
\pgfpathlineto{\pgfqpoint{4.267107in}{1.470894in}}%
\pgfpathlineto{\pgfqpoint{4.294571in}{1.413333in}}%
\pgfpathlineto{\pgfqpoint{4.316694in}{1.366278in}}%
\pgfpathlineto{\pgfqpoint{4.345139in}{1.306667in}}%
\pgfpathlineto{\pgfqpoint{4.366281in}{1.261694in}}%
\pgfpathlineto{\pgfqpoint{4.395723in}{1.200000in}}%
\pgfpathlineto{\pgfqpoint{4.415868in}{1.157143in}}%
\pgfpathlineto{\pgfqpoint{4.446323in}{1.093333in}}%
\pgfpathlineto{\pgfqpoint{4.465455in}{1.052628in}}%
\pgfpathlineto{\pgfqpoint{4.496940in}{0.986667in}}%
\pgfpathlineto{\pgfqpoint{4.515041in}{0.948149in}}%
\pgfpathlineto{\pgfqpoint{4.547575in}{0.880000in}}%
\pgfpathlineto{\pgfqpoint{4.564628in}{0.843709in}}%
\pgfpathlineto{\pgfqpoint{4.598229in}{0.773333in}}%
\pgfpathlineto{\pgfqpoint{4.614215in}{0.739308in}}%
\pgfpathlineto{\pgfqpoint{4.648903in}{0.666667in}}%
\pgfpathlineto{\pgfqpoint{4.663802in}{0.634950in}}%
\pgfpathlineto{\pgfqpoint{4.699597in}{0.560000in}}%
\pgfpathlineto{\pgfqpoint{4.713388in}{0.530637in}}%
\pgfpathlineto{\pgfqpoint{4.750312in}{0.453333in}}%
\pgfpathlineto{\pgfqpoint{4.762975in}{0.426370in}}%
\pgfpathlineto{\pgfqpoint{4.801051in}{0.346667in}}%
\pgfpathlineto{\pgfqpoint{4.812562in}{0.322152in}}%
\pgfpathlineto{\pgfqpoint{4.851813in}{0.240000in}}%
\pgfpathlineto{\pgfqpoint{4.862149in}{0.217986in}}%
\pgfpathlineto{\pgfqpoint{4.902601in}{0.133333in}}%
\pgfpathlineto{\pgfqpoint{4.911736in}{0.113874in}}%
\pgfpathlineto{\pgfqpoint{4.953415in}{0.026667in}}%
\pgfpathlineto{\pgfqpoint{4.961322in}{0.009820in}}%
\pgfpathlineto{\pgfqpoint{5.004258in}{-0.080000in}}%
\pgfpathlineto{\pgfqpoint{5.010909in}{-0.094174in}}%
\pgfpathlineto{\pgfqpoint{5.055130in}{-0.186667in}}%
\pgfpathlineto{\pgfqpoint{5.060496in}{-0.198103in}}%
\pgfpathlineto{\pgfqpoint{5.106034in}{-0.293333in}}%
\pgfpathlineto{\pgfqpoint{5.110083in}{-0.301965in}}%
\pgfpathlineto{\pgfqpoint{5.156971in}{-0.400000in}}%
\pgfpathlineto{\pgfqpoint{5.159669in}{-0.405754in}}%
\pgfpathlineto{\pgfqpoint{5.207943in}{-0.506667in}}%
\pgfpathlineto{\pgfqpoint{5.209256in}{-0.509468in}}%
\pgfpathlineto{\pgfqpoint{5.258843in}{-0.613223in}}%
\pgfpathlineto{\pgfqpoint{5.258896in}{-0.613333in}}%
\pgfpathlineto{\pgfqpoint{5.308430in}{-0.718430in}}%
\pgfpathlineto{\pgfqpoint{5.309177in}{-0.720000in}}%
\pgfpathlineto{\pgfqpoint{5.358017in}{-0.823628in}}%
\pgfpathlineto{\pgfqpoint{5.359463in}{-0.826667in}}%
\pgfpathlineto{\pgfqpoint{5.407603in}{-0.928816in}}%
\pgfpathlineto{\pgfqpoint{5.409754in}{-0.933333in}}%
\pgfpathlineto{\pgfqpoint{5.457190in}{-1.033994in}}%
\pgfpathlineto{\pgfqpoint{5.460049in}{-1.040000in}}%
\pgfpathlineto{\pgfqpoint{5.506777in}{-1.139162in}}%
\pgfpathlineto{\pgfqpoint{5.510350in}{-1.146667in}}%
\pgfpathlineto{\pgfqpoint{5.556364in}{-1.244319in}}%
\pgfpathlineto{\pgfqpoint{5.560656in}{-1.253333in}}%
\pgfpathlineto{\pgfqpoint{5.605950in}{-1.349465in}}%
\pgfpathlineto{\pgfqpoint{5.610966in}{-1.360000in}}%
\pgfpathlineto{\pgfqpoint{5.655537in}{-1.454600in}}%
\pgfpathlineto{\pgfqpoint{5.661283in}{-1.466667in}}%
\pgfpathlineto{\pgfqpoint{5.705124in}{-1.559723in}}%
\pgfpathlineto{\pgfqpoint{5.711605in}{-1.573333in}}%
\pgfpathlineto{\pgfqpoint{5.754711in}{-1.664835in}}%
\pgfpathlineto{\pgfqpoint{5.761933in}{-1.680000in}}%
\pgfpathlineto{\pgfqpoint{5.804298in}{-1.769933in}}%
\pgfpathlineto{\pgfqpoint{5.812267in}{-1.786667in}}%
\pgfpathlineto{\pgfqpoint{5.853884in}{-1.875020in}}%
\pgfpathlineto{\pgfqpoint{5.862607in}{-1.893333in}}%
\pgfpathlineto{\pgfqpoint{5.903471in}{-1.980092in}}%
\pgfpathlineto{\pgfqpoint{5.912953in}{-2.000000in}}%
\pgfpathlineto{\pgfqpoint{5.953058in}{-2.085151in}}%
\pgfpathlineto{\pgfqpoint{5.963306in}{-2.106667in}}%
\pgfpathlineto{\pgfqpoint{6.002645in}{-2.190196in}}%
\pgfpathlineto{\pgfqpoint{6.013666in}{-2.213333in}}%
\pgfpathlineto{\pgfqpoint{6.052231in}{-2.295227in}}%
\pgfpathlineto{\pgfqpoint{6.064033in}{-2.320000in}}%
\pgfpathclose%
\pgfusepath{fill}%
\end{pgfscope}%
\begin{pgfscope}%
\pgfpathrectangle{\pgfqpoint{3.156364in}{0.528000in}}{\pgfqpoint{1.963636in}{3.696000in}} %
\pgfusepath{clip}%
\pgfsetbuttcap%
\pgfsetroundjoin%
\definecolor{currentfill}{rgb}{0.744232,0.218288,0.520524}%
\pgfsetfillcolor{currentfill}%
\pgfsetlinewidth{0.000000pt}%
\definecolor{currentstroke}{rgb}{0.000000,0.000000,0.000000}%
\pgfsetstrokecolor{currentstroke}%
\pgfsetdash{}{0pt}%
\pgfpathmoveto{\pgfqpoint{5.407603in}{-2.569163in}}%
\pgfpathlineto{\pgfqpoint{5.441133in}{-2.640000in}}%
\pgfpathlineto{\pgfqpoint{5.457190in}{-2.640000in}}%
\pgfpathlineto{\pgfqpoint{5.506777in}{-2.640000in}}%
\pgfpathlineto{\pgfqpoint{5.556364in}{-2.640000in}}%
\pgfpathlineto{\pgfqpoint{5.605950in}{-2.640000in}}%
\pgfpathlineto{\pgfqpoint{5.655537in}{-2.640000in}}%
\pgfpathlineto{\pgfqpoint{5.705124in}{-2.640000in}}%
\pgfpathlineto{\pgfqpoint{5.754711in}{-2.640000in}}%
\pgfpathlineto{\pgfqpoint{5.804298in}{-2.640000in}}%
\pgfpathlineto{\pgfqpoint{5.853884in}{-2.640000in}}%
\pgfpathlineto{\pgfqpoint{5.903471in}{-2.640000in}}%
\pgfpathlineto{\pgfqpoint{5.925774in}{-2.640000in}}%
\pgfpathlineto{\pgfqpoint{5.903471in}{-2.592811in}}%
\pgfpathlineto{\pgfqpoint{5.875240in}{-2.533333in}}%
\pgfpathlineto{\pgfqpoint{5.853884in}{-2.488148in}}%
\pgfpathlineto{\pgfqpoint{5.824702in}{-2.426667in}}%
\pgfpathlineto{\pgfqpoint{5.804298in}{-2.383494in}}%
\pgfpathlineto{\pgfqpoint{5.774159in}{-2.320000in}}%
\pgfpathlineto{\pgfqpoint{5.754711in}{-2.278849in}}%
\pgfpathlineto{\pgfqpoint{5.723613in}{-2.213333in}}%
\pgfpathlineto{\pgfqpoint{5.705124in}{-2.174213in}}%
\pgfpathlineto{\pgfqpoint{5.673062in}{-2.106667in}}%
\pgfpathlineto{\pgfqpoint{5.655537in}{-2.069587in}}%
\pgfpathlineto{\pgfqpoint{5.622506in}{-2.000000in}}%
\pgfpathlineto{\pgfqpoint{5.605950in}{-1.964969in}}%
\pgfpathlineto{\pgfqpoint{5.571947in}{-1.893333in}}%
\pgfpathlineto{\pgfqpoint{5.556364in}{-1.860361in}}%
\pgfpathlineto{\pgfqpoint{5.521382in}{-1.786667in}}%
\pgfpathlineto{\pgfqpoint{5.506777in}{-1.755762in}}%
\pgfpathlineto{\pgfqpoint{5.470814in}{-1.680000in}}%
\pgfpathlineto{\pgfqpoint{5.457190in}{-1.651172in}}%
\pgfpathlineto{\pgfqpoint{5.420241in}{-1.573333in}}%
\pgfpathlineto{\pgfqpoint{5.407603in}{-1.546593in}}%
\pgfpathlineto{\pgfqpoint{5.369663in}{-1.466667in}}%
\pgfpathlineto{\pgfqpoint{5.358017in}{-1.442023in}}%
\pgfpathlineto{\pgfqpoint{5.319080in}{-1.360000in}}%
\pgfpathlineto{\pgfqpoint{5.308430in}{-1.337463in}}%
\pgfpathlineto{\pgfqpoint{5.268493in}{-1.253333in}}%
\pgfpathlineto{\pgfqpoint{5.258843in}{-1.232914in}}%
\pgfpathlineto{\pgfqpoint{5.217901in}{-1.146667in}}%
\pgfpathlineto{\pgfqpoint{5.209256in}{-1.128374in}}%
\pgfpathlineto{\pgfqpoint{5.167304in}{-1.040000in}}%
\pgfpathlineto{\pgfqpoint{5.159669in}{-1.023845in}}%
\pgfpathlineto{\pgfqpoint{5.116702in}{-0.933333in}}%
\pgfpathlineto{\pgfqpoint{5.110083in}{-0.919327in}}%
\pgfpathlineto{\pgfqpoint{5.066095in}{-0.826667in}}%
\pgfpathlineto{\pgfqpoint{5.060496in}{-0.814819in}}%
\pgfpathlineto{\pgfqpoint{5.015483in}{-0.720000in}}%
\pgfpathlineto{\pgfqpoint{5.010909in}{-0.710322in}}%
\pgfpathlineto{\pgfqpoint{4.964865in}{-0.613333in}}%
\pgfpathlineto{\pgfqpoint{4.961322in}{-0.605836in}}%
\pgfpathlineto{\pgfqpoint{4.914243in}{-0.506667in}}%
\pgfpathlineto{\pgfqpoint{4.911736in}{-0.501361in}}%
\pgfpathlineto{\pgfqpoint{4.863615in}{-0.400000in}}%
\pgfpathlineto{\pgfqpoint{4.862149in}{-0.396897in}}%
\pgfpathlineto{\pgfqpoint{4.812982in}{-0.293333in}}%
\pgfpathlineto{\pgfqpoint{4.812562in}{-0.292445in}}%
\pgfpathlineto{\pgfqpoint{4.762975in}{-0.187732in}}%
\pgfpathlineto{\pgfqpoint{4.762470in}{-0.186667in}}%
\pgfpathlineto{\pgfqpoint{4.713388in}{-0.082845in}}%
\pgfpathlineto{\pgfqpoint{4.712038in}{-0.080000in}}%
\pgfpathlineto{\pgfqpoint{4.663802in}{0.022035in}}%
\pgfpathlineto{\pgfqpoint{4.661604in}{0.026667in}}%
\pgfpathlineto{\pgfqpoint{4.614215in}{0.126910in}}%
\pgfpathlineto{\pgfqpoint{4.611167in}{0.133333in}}%
\pgfpathlineto{\pgfqpoint{4.564628in}{0.231779in}}%
\pgfpathlineto{\pgfqpoint{4.560727in}{0.240000in}}%
\pgfpathlineto{\pgfqpoint{4.515041in}{0.336641in}}%
\pgfpathlineto{\pgfqpoint{4.510284in}{0.346667in}}%
\pgfpathlineto{\pgfqpoint{4.465455in}{0.441497in}}%
\pgfpathlineto{\pgfqpoint{4.459838in}{0.453333in}}%
\pgfpathlineto{\pgfqpoint{4.415868in}{0.546347in}}%
\pgfpathlineto{\pgfqpoint{4.409388in}{0.560000in}}%
\pgfpathlineto{\pgfqpoint{4.366281in}{0.651190in}}%
\pgfpathlineto{\pgfqpoint{4.358936in}{0.666667in}}%
\pgfpathlineto{\pgfqpoint{4.316694in}{0.756026in}}%
\pgfpathlineto{\pgfqpoint{4.308481in}{0.773333in}}%
\pgfpathlineto{\pgfqpoint{4.267107in}{0.860856in}}%
\pgfpathlineto{\pgfqpoint{4.258023in}{0.880000in}}%
\pgfpathlineto{\pgfqpoint{4.217521in}{0.965680in}}%
\pgfpathlineto{\pgfqpoint{4.207561in}{0.986667in}}%
\pgfpathlineto{\pgfqpoint{4.167934in}{1.070497in}}%
\pgfpathlineto{\pgfqpoint{4.157096in}{1.093333in}}%
\pgfpathlineto{\pgfqpoint{4.118347in}{1.175306in}}%
\pgfpathlineto{\pgfqpoint{4.106628in}{1.200000in}}%
\pgfpathlineto{\pgfqpoint{4.068760in}{1.280109in}}%
\pgfpathlineto{\pgfqpoint{4.056157in}{1.306667in}}%
\pgfpathlineto{\pgfqpoint{4.019174in}{1.384905in}}%
\pgfpathlineto{\pgfqpoint{4.005682in}{1.413333in}}%
\pgfpathlineto{\pgfqpoint{3.969587in}{1.489694in}}%
\pgfpathlineto{\pgfqpoint{3.955204in}{1.520000in}}%
\pgfpathlineto{\pgfqpoint{3.920000in}{1.594476in}}%
\pgfpathlineto{\pgfqpoint{3.904722in}{1.626667in}}%
\pgfpathlineto{\pgfqpoint{3.870413in}{1.699250in}}%
\pgfpathlineto{\pgfqpoint{3.854237in}{1.733333in}}%
\pgfpathlineto{\pgfqpoint{3.820826in}{1.804017in}}%
\pgfpathlineto{\pgfqpoint{3.803749in}{1.840000in}}%
\pgfpathlineto{\pgfqpoint{3.771240in}{1.908777in}}%
\pgfpathlineto{\pgfqpoint{3.753257in}{1.946667in}}%
\pgfpathlineto{\pgfqpoint{3.721653in}{2.013529in}}%
\pgfpathlineto{\pgfqpoint{3.702761in}{2.053333in}}%
\pgfpathlineto{\pgfqpoint{3.672066in}{2.118274in}}%
\pgfpathlineto{\pgfqpoint{3.652262in}{2.160000in}}%
\pgfpathlineto{\pgfqpoint{3.622479in}{2.223010in}}%
\pgfpathlineto{\pgfqpoint{3.601759in}{2.266667in}}%
\pgfpathlineto{\pgfqpoint{3.572893in}{2.327739in}}%
\pgfpathlineto{\pgfqpoint{3.551253in}{2.373333in}}%
\pgfpathlineto{\pgfqpoint{3.523306in}{2.432461in}}%
\pgfpathlineto{\pgfqpoint{3.500742in}{2.480000in}}%
\pgfpathlineto{\pgfqpoint{3.473719in}{2.537174in}}%
\pgfpathlineto{\pgfqpoint{3.450228in}{2.586667in}}%
\pgfpathlineto{\pgfqpoint{3.424132in}{2.641879in}}%
\pgfpathlineto{\pgfqpoint{3.399710in}{2.693333in}}%
\pgfpathlineto{\pgfqpoint{3.374545in}{2.746575in}}%
\pgfpathlineto{\pgfqpoint{3.349188in}{2.800000in}}%
\pgfpathlineto{\pgfqpoint{3.324959in}{2.851264in}}%
\pgfpathlineto{\pgfqpoint{3.298662in}{2.906667in}}%
\pgfpathlineto{\pgfqpoint{3.275372in}{2.955944in}}%
\pgfpathlineto{\pgfqpoint{3.248132in}{3.013333in}}%
\pgfpathlineto{\pgfqpoint{3.225785in}{3.060616in}}%
\pgfpathlineto{\pgfqpoint{3.197598in}{3.120000in}}%
\pgfpathlineto{\pgfqpoint{3.176198in}{3.165279in}}%
\pgfpathlineto{\pgfqpoint{3.147060in}{3.226667in}}%
\pgfpathlineto{\pgfqpoint{3.126612in}{3.269933in}}%
\pgfpathlineto{\pgfqpoint{3.096518in}{3.333333in}}%
\pgfpathlineto{\pgfqpoint{3.077025in}{3.374578in}}%
\pgfpathlineto{\pgfqpoint{3.045971in}{3.440000in}}%
\pgfpathlineto{\pgfqpoint{3.027438in}{3.479215in}}%
\pgfpathlineto{\pgfqpoint{2.995421in}{3.546667in}}%
\pgfpathlineto{\pgfqpoint{2.977851in}{3.583842in}}%
\pgfpathlineto{\pgfqpoint{2.944866in}{3.653333in}}%
\pgfpathlineto{\pgfqpoint{2.928264in}{3.688460in}}%
\pgfpathlineto{\pgfqpoint{2.894306in}{3.760000in}}%
\pgfpathlineto{\pgfqpoint{2.878678in}{3.793069in}}%
\pgfpathlineto{\pgfqpoint{2.843742in}{3.866667in}}%
\pgfpathlineto{\pgfqpoint{2.829091in}{3.897668in}}%
\pgfpathlineto{\pgfqpoint{2.793174in}{3.973333in}}%
\pgfpathlineto{\pgfqpoint{2.779504in}{4.002258in}}%
\pgfpathlineto{\pgfqpoint{2.742601in}{4.080000in}}%
\pgfpathlineto{\pgfqpoint{2.729917in}{4.106838in}}%
\pgfpathlineto{\pgfqpoint{2.692023in}{4.186667in}}%
\pgfpathlineto{\pgfqpoint{2.680331in}{4.211408in}}%
\pgfpathlineto{\pgfqpoint{2.641441in}{4.293333in}}%
\pgfpathlineto{\pgfqpoint{2.630744in}{4.315968in}}%
\pgfpathlineto{\pgfqpoint{2.590853in}{4.400000in}}%
\pgfpathlineto{\pgfqpoint{2.581157in}{4.420518in}}%
\pgfpathlineto{\pgfqpoint{2.540262in}{4.506667in}}%
\pgfpathlineto{\pgfqpoint{2.531570in}{4.525058in}}%
\pgfpathlineto{\pgfqpoint{2.489665in}{4.613333in}}%
\pgfpathlineto{\pgfqpoint{2.481983in}{4.629588in}}%
\pgfpathlineto{\pgfqpoint{2.439063in}{4.720000in}}%
\pgfpathlineto{\pgfqpoint{2.432397in}{4.734107in}}%
\pgfpathlineto{\pgfqpoint{2.388456in}{4.826667in}}%
\pgfpathlineto{\pgfqpoint{2.382810in}{4.838615in}}%
\pgfpathlineto{\pgfqpoint{2.337844in}{4.933333in}}%
\pgfpathlineto{\pgfqpoint{2.333223in}{4.943113in}}%
\pgfpathlineto{\pgfqpoint{2.287227in}{5.040000in}}%
\pgfpathlineto{\pgfqpoint{2.283636in}{5.047599in}}%
\pgfpathlineto{\pgfqpoint{2.236605in}{5.146667in}}%
\pgfpathlineto{\pgfqpoint{2.234050in}{5.152075in}}%
\pgfpathlineto{\pgfqpoint{2.185978in}{5.253333in}}%
\pgfpathlineto{\pgfqpoint{2.184463in}{5.256539in}}%
\pgfpathlineto{\pgfqpoint{2.135345in}{5.360000in}}%
\pgfpathlineto{\pgfqpoint{2.134876in}{5.360992in}}%
\pgfpathlineto{\pgfqpoint{2.085289in}{5.465684in}}%
\pgfpathlineto{\pgfqpoint{2.084823in}{5.466667in}}%
\pgfpathlineto{\pgfqpoint{2.035702in}{5.570571in}}%
\pgfpathlineto{\pgfqpoint{2.034392in}{5.573333in}}%
\pgfpathlineto{\pgfqpoint{1.986116in}{5.675452in}}%
\pgfpathlineto{\pgfqpoint{1.983957in}{5.680000in}}%
\pgfpathlineto{\pgfqpoint{1.936529in}{5.780327in}}%
\pgfpathlineto{\pgfqpoint{1.933520in}{5.786667in}}%
\pgfpathlineto{\pgfqpoint{1.886942in}{5.885196in}}%
\pgfpathlineto{\pgfqpoint{1.883080in}{5.893333in}}%
\pgfpathlineto{\pgfqpoint{1.837355in}{5.990058in}}%
\pgfpathlineto{\pgfqpoint{1.832638in}{6.000000in}}%
\pgfpathlineto{\pgfqpoint{1.787769in}{6.094915in}}%
\pgfpathlineto{\pgfqpoint{1.782192in}{6.106667in}}%
\pgfpathlineto{\pgfqpoint{1.738182in}{6.199764in}}%
\pgfpathlineto{\pgfqpoint{1.731743in}{6.213333in}}%
\pgfpathlineto{\pgfqpoint{1.688595in}{6.304608in}}%
\pgfpathlineto{\pgfqpoint{1.681291in}{6.320000in}}%
\pgfpathlineto{\pgfqpoint{1.639008in}{6.409445in}}%
\pgfpathlineto{\pgfqpoint{1.630835in}{6.426667in}}%
\pgfpathlineto{\pgfqpoint{1.589421in}{6.514275in}}%
\pgfpathlineto{\pgfqpoint{1.580377in}{6.533333in}}%
\pgfpathlineto{\pgfqpoint{1.539835in}{6.619099in}}%
\pgfpathlineto{\pgfqpoint{1.529916in}{6.640000in}}%
\pgfpathlineto{\pgfqpoint{1.490248in}{6.723916in}}%
\pgfpathlineto{\pgfqpoint{1.479451in}{6.746667in}}%
\pgfpathlineto{\pgfqpoint{1.440661in}{6.828726in}}%
\pgfpathlineto{\pgfqpoint{1.428983in}{6.853333in}}%
\pgfpathlineto{\pgfqpoint{1.391074in}{6.933529in}}%
\pgfpathlineto{\pgfqpoint{1.378512in}{6.960000in}}%
\pgfpathlineto{\pgfqpoint{1.341488in}{7.038326in}}%
\pgfpathlineto{\pgfqpoint{1.328037in}{7.066667in}}%
\pgfpathlineto{\pgfqpoint{1.291901in}{7.143115in}}%
\pgfpathlineto{\pgfqpoint{1.277559in}{7.173333in}}%
\pgfpathlineto{\pgfqpoint{1.242314in}{7.247897in}}%
\pgfpathlineto{\pgfqpoint{1.227078in}{7.280000in}}%
\pgfpathlineto{\pgfqpoint{1.192727in}{7.352672in}}%
\pgfpathlineto{\pgfqpoint{1.192727in}{7.280000in}}%
\pgfpathlineto{\pgfqpoint{1.192727in}{7.173333in}}%
\pgfpathlineto{\pgfqpoint{1.192727in}{7.066667in}}%
\pgfpathlineto{\pgfqpoint{1.192727in}{6.960000in}}%
\pgfpathlineto{\pgfqpoint{1.192727in}{6.853333in}}%
\pgfpathlineto{\pgfqpoint{1.192727in}{6.746667in}}%
\pgfpathlineto{\pgfqpoint{1.192727in}{6.640000in}}%
\pgfpathlineto{\pgfqpoint{1.192727in}{6.533333in}}%
\pgfpathlineto{\pgfqpoint{1.192727in}{6.426667in}}%
\pgfpathlineto{\pgfqpoint{1.192727in}{6.329197in}}%
\pgfpathlineto{\pgfqpoint{1.197080in}{6.320000in}}%
\pgfpathlineto{\pgfqpoint{1.242314in}{6.224563in}}%
\pgfpathlineto{\pgfqpoint{1.247629in}{6.213333in}}%
\pgfpathlineto{\pgfqpoint{1.291901in}{6.119925in}}%
\pgfpathlineto{\pgfqpoint{1.298176in}{6.106667in}}%
\pgfpathlineto{\pgfqpoint{1.341488in}{6.015285in}}%
\pgfpathlineto{\pgfqpoint{1.348722in}{6.000000in}}%
\pgfpathlineto{\pgfqpoint{1.391074in}{5.910643in}}%
\pgfpathlineto{\pgfqpoint{1.399267in}{5.893333in}}%
\pgfpathlineto{\pgfqpoint{1.440661in}{5.805997in}}%
\pgfpathlineto{\pgfqpoint{1.449811in}{5.786667in}}%
\pgfpathlineto{\pgfqpoint{1.490248in}{5.701348in}}%
\pgfpathlineto{\pgfqpoint{1.500353in}{5.680000in}}%
\pgfpathlineto{\pgfqpoint{1.539835in}{5.596697in}}%
\pgfpathlineto{\pgfqpoint{1.550893in}{5.573333in}}%
\pgfpathlineto{\pgfqpoint{1.589421in}{5.492043in}}%
\pgfpathlineto{\pgfqpoint{1.601433in}{5.466667in}}%
\pgfpathlineto{\pgfqpoint{1.639008in}{5.387386in}}%
\pgfpathlineto{\pgfqpoint{1.651971in}{5.360000in}}%
\pgfpathlineto{\pgfqpoint{1.688595in}{5.282727in}}%
\pgfpathlineto{\pgfqpoint{1.702508in}{5.253333in}}%
\pgfpathlineto{\pgfqpoint{1.738182in}{5.178065in}}%
\pgfpathlineto{\pgfqpoint{1.753043in}{5.146667in}}%
\pgfpathlineto{\pgfqpoint{1.787769in}{5.073400in}}%
\pgfpathlineto{\pgfqpoint{1.803577in}{5.040000in}}%
\pgfpathlineto{\pgfqpoint{1.837355in}{4.968732in}}%
\pgfpathlineto{\pgfqpoint{1.854110in}{4.933333in}}%
\pgfpathlineto{\pgfqpoint{1.886942in}{4.864062in}}%
\pgfpathlineto{\pgfqpoint{1.904642in}{4.826667in}}%
\pgfpathlineto{\pgfqpoint{1.936529in}{4.759388in}}%
\pgfpathlineto{\pgfqpoint{1.955172in}{4.720000in}}%
\pgfpathlineto{\pgfqpoint{1.986116in}{4.654713in}}%
\pgfpathlineto{\pgfqpoint{2.005702in}{4.613333in}}%
\pgfpathlineto{\pgfqpoint{2.035702in}{4.550034in}}%
\pgfpathlineto{\pgfqpoint{2.056229in}{4.506667in}}%
\pgfpathlineto{\pgfqpoint{2.085289in}{4.445353in}}%
\pgfpathlineto{\pgfqpoint{2.106756in}{4.400000in}}%
\pgfpathlineto{\pgfqpoint{2.134876in}{4.340669in}}%
\pgfpathlineto{\pgfqpoint{2.157281in}{4.293333in}}%
\pgfpathlineto{\pgfqpoint{2.184463in}{4.235983in}}%
\pgfpathlineto{\pgfqpoint{2.207805in}{4.186667in}}%
\pgfpathlineto{\pgfqpoint{2.234050in}{4.131294in}}%
\pgfpathlineto{\pgfqpoint{2.258328in}{4.080000in}}%
\pgfpathlineto{\pgfqpoint{2.283636in}{4.026602in}}%
\pgfpathlineto{\pgfqpoint{2.308850in}{3.973333in}}%
\pgfpathlineto{\pgfqpoint{2.333223in}{3.921908in}}%
\pgfpathlineto{\pgfqpoint{2.359370in}{3.866667in}}%
\pgfpathlineto{\pgfqpoint{2.382810in}{3.817211in}}%
\pgfpathlineto{\pgfqpoint{2.409889in}{3.760000in}}%
\pgfpathlineto{\pgfqpoint{2.432397in}{3.712512in}}%
\pgfpathlineto{\pgfqpoint{2.460407in}{3.653333in}}%
\pgfpathlineto{\pgfqpoint{2.481983in}{3.607810in}}%
\pgfpathlineto{\pgfqpoint{2.510924in}{3.546667in}}%
\pgfpathlineto{\pgfqpoint{2.531570in}{3.503105in}}%
\pgfpathlineto{\pgfqpoint{2.561440in}{3.440000in}}%
\pgfpathlineto{\pgfqpoint{2.581157in}{3.398398in}}%
\pgfpathlineto{\pgfqpoint{2.611954in}{3.333333in}}%
\pgfpathlineto{\pgfqpoint{2.630744in}{3.293688in}}%
\pgfpathlineto{\pgfqpoint{2.662467in}{3.226667in}}%
\pgfpathlineto{\pgfqpoint{2.680331in}{3.188976in}}%
\pgfpathlineto{\pgfqpoint{2.712979in}{3.120000in}}%
\pgfpathlineto{\pgfqpoint{2.729917in}{3.084261in}}%
\pgfpathlineto{\pgfqpoint{2.763490in}{3.013333in}}%
\pgfpathlineto{\pgfqpoint{2.779504in}{2.979544in}}%
\pgfpathlineto{\pgfqpoint{2.813999in}{2.906667in}}%
\pgfpathlineto{\pgfqpoint{2.829091in}{2.874824in}}%
\pgfpathlineto{\pgfqpoint{2.864507in}{2.800000in}}%
\pgfpathlineto{\pgfqpoint{2.878678in}{2.770102in}}%
\pgfpathlineto{\pgfqpoint{2.915014in}{2.693333in}}%
\pgfpathlineto{\pgfqpoint{2.928264in}{2.665377in}}%
\pgfpathlineto{\pgfqpoint{2.965520in}{2.586667in}}%
\pgfpathlineto{\pgfqpoint{2.977851in}{2.560649in}}%
\pgfpathlineto{\pgfqpoint{3.016025in}{2.480000in}}%
\pgfpathlineto{\pgfqpoint{3.027438in}{2.455920in}}%
\pgfpathlineto{\pgfqpoint{3.066529in}{2.373333in}}%
\pgfpathlineto{\pgfqpoint{3.077025in}{2.351187in}}%
\pgfpathlineto{\pgfqpoint{3.117031in}{2.266667in}}%
\pgfpathlineto{\pgfqpoint{3.126612in}{2.246453in}}%
\pgfpathlineto{\pgfqpoint{3.167533in}{2.160000in}}%
\pgfpathlineto{\pgfqpoint{3.176198in}{2.141716in}}%
\pgfpathlineto{\pgfqpoint{3.218033in}{2.053333in}}%
\pgfpathlineto{\pgfqpoint{3.225785in}{2.036976in}}%
\pgfpathlineto{\pgfqpoint{3.268532in}{1.946667in}}%
\pgfpathlineto{\pgfqpoint{3.275372in}{1.932234in}}%
\pgfpathlineto{\pgfqpoint{3.319030in}{1.840000in}}%
\pgfpathlineto{\pgfqpoint{3.324959in}{1.827490in}}%
\pgfpathlineto{\pgfqpoint{3.369526in}{1.733333in}}%
\pgfpathlineto{\pgfqpoint{3.374545in}{1.722743in}}%
\pgfpathlineto{\pgfqpoint{3.420022in}{1.626667in}}%
\pgfpathlineto{\pgfqpoint{3.424132in}{1.617994in}}%
\pgfpathlineto{\pgfqpoint{3.470516in}{1.520000in}}%
\pgfpathlineto{\pgfqpoint{3.473719in}{1.513242in}}%
\pgfpathlineto{\pgfqpoint{3.521010in}{1.413333in}}%
\pgfpathlineto{\pgfqpoint{3.523306in}{1.408488in}}%
\pgfpathlineto{\pgfqpoint{3.571502in}{1.306667in}}%
\pgfpathlineto{\pgfqpoint{3.572893in}{1.303732in}}%
\pgfpathlineto{\pgfqpoint{3.621993in}{1.200000in}}%
\pgfpathlineto{\pgfqpoint{3.622479in}{1.198973in}}%
\pgfpathlineto{\pgfqpoint{3.672066in}{1.094277in}}%
\pgfpathlineto{\pgfqpoint{3.672513in}{1.093333in}}%
\pgfpathlineto{\pgfqpoint{3.721653in}{0.989655in}}%
\pgfpathlineto{\pgfqpoint{3.723067in}{0.986667in}}%
\pgfpathlineto{\pgfqpoint{3.771240in}{0.885029in}}%
\pgfpathlineto{\pgfqpoint{3.773620in}{0.880000in}}%
\pgfpathlineto{\pgfqpoint{3.820826in}{0.780400in}}%
\pgfpathlineto{\pgfqpoint{3.824171in}{0.773333in}}%
\pgfpathlineto{\pgfqpoint{3.870413in}{0.675769in}}%
\pgfpathlineto{\pgfqpoint{3.874722in}{0.666667in}}%
\pgfpathlineto{\pgfqpoint{3.920000in}{0.571135in}}%
\pgfpathlineto{\pgfqpoint{3.925270in}{0.560000in}}%
\pgfpathlineto{\pgfqpoint{3.969587in}{0.466498in}}%
\pgfpathlineto{\pgfqpoint{3.975818in}{0.453333in}}%
\pgfpathlineto{\pgfqpoint{4.019174in}{0.361858in}}%
\pgfpathlineto{\pgfqpoint{4.026364in}{0.346667in}}%
\pgfpathlineto{\pgfqpoint{4.068760in}{0.257215in}}%
\pgfpathlineto{\pgfqpoint{4.076909in}{0.240000in}}%
\pgfpathlineto{\pgfqpoint{4.118347in}{0.152570in}}%
\pgfpathlineto{\pgfqpoint{4.127452in}{0.133333in}}%
\pgfpathlineto{\pgfqpoint{4.167934in}{0.047921in}}%
\pgfpathlineto{\pgfqpoint{4.177994in}{0.026667in}}%
\pgfpathlineto{\pgfqpoint{4.217521in}{-0.056730in}}%
\pgfpathlineto{\pgfqpoint{4.228535in}{-0.080000in}}%
\pgfpathlineto{\pgfqpoint{4.267107in}{-0.161384in}}%
\pgfpathlineto{\pgfqpoint{4.279074in}{-0.186667in}}%
\pgfpathlineto{\pgfqpoint{4.316694in}{-0.266040in}}%
\pgfpathlineto{\pgfqpoint{4.329613in}{-0.293333in}}%
\pgfpathlineto{\pgfqpoint{4.366281in}{-0.370700in}}%
\pgfpathlineto{\pgfqpoint{4.380149in}{-0.400000in}}%
\pgfpathlineto{\pgfqpoint{4.415868in}{-0.475362in}}%
\pgfpathlineto{\pgfqpoint{4.430685in}{-0.506667in}}%
\pgfpathlineto{\pgfqpoint{4.465455in}{-0.580027in}}%
\pgfpathlineto{\pgfqpoint{4.481219in}{-0.613333in}}%
\pgfpathlineto{\pgfqpoint{4.515041in}{-0.684694in}}%
\pgfpathlineto{\pgfqpoint{4.531752in}{-0.720000in}}%
\pgfpathlineto{\pgfqpoint{4.564628in}{-0.789365in}}%
\pgfpathlineto{\pgfqpoint{4.582284in}{-0.826667in}}%
\pgfpathlineto{\pgfqpoint{4.614215in}{-0.894038in}}%
\pgfpathlineto{\pgfqpoint{4.632815in}{-0.933333in}}%
\pgfpathlineto{\pgfqpoint{4.663802in}{-0.998713in}}%
\pgfpathlineto{\pgfqpoint{4.683344in}{-1.040000in}}%
\pgfpathlineto{\pgfqpoint{4.713388in}{-1.103392in}}%
\pgfpathlineto{\pgfqpoint{4.733872in}{-1.146667in}}%
\pgfpathlineto{\pgfqpoint{4.762975in}{-1.208073in}}%
\pgfpathlineto{\pgfqpoint{4.784398in}{-1.253333in}}%
\pgfpathlineto{\pgfqpoint{4.812562in}{-1.312756in}}%
\pgfpathlineto{\pgfqpoint{4.834924in}{-1.360000in}}%
\pgfpathlineto{\pgfqpoint{4.862149in}{-1.417442in}}%
\pgfpathlineto{\pgfqpoint{4.885448in}{-1.466667in}}%
\pgfpathlineto{\pgfqpoint{4.911736in}{-1.522131in}}%
\pgfpathlineto{\pgfqpoint{4.935971in}{-1.573333in}}%
\pgfpathlineto{\pgfqpoint{4.961322in}{-1.626823in}}%
\pgfpathlineto{\pgfqpoint{4.986492in}{-1.680000in}}%
\pgfpathlineto{\pgfqpoint{5.010909in}{-1.731517in}}%
\pgfpathlineto{\pgfqpoint{5.037013in}{-1.786667in}}%
\pgfpathlineto{\pgfqpoint{5.060496in}{-1.836214in}}%
\pgfpathlineto{\pgfqpoint{5.087532in}{-1.893333in}}%
\pgfpathlineto{\pgfqpoint{5.110083in}{-1.940913in}}%
\pgfpathlineto{\pgfqpoint{5.138050in}{-2.000000in}}%
\pgfpathlineto{\pgfqpoint{5.159669in}{-2.045615in}}%
\pgfpathlineto{\pgfqpoint{5.188567in}{-2.106667in}}%
\pgfpathlineto{\pgfqpoint{5.209256in}{-2.150320in}}%
\pgfpathlineto{\pgfqpoint{5.239082in}{-2.213333in}}%
\pgfpathlineto{\pgfqpoint{5.258843in}{-2.255027in}}%
\pgfpathlineto{\pgfqpoint{5.289597in}{-2.320000in}}%
\pgfpathlineto{\pgfqpoint{5.308430in}{-2.359736in}}%
\pgfpathlineto{\pgfqpoint{5.340110in}{-2.426667in}}%
\pgfpathlineto{\pgfqpoint{5.358017in}{-2.464448in}}%
\pgfpathlineto{\pgfqpoint{5.390622in}{-2.533333in}}%
\pgfpathclose%
\pgfusepath{fill}%
\end{pgfscope}%
\begin{pgfscope}%
\pgfpathrectangle{\pgfqpoint{3.156364in}{0.528000in}}{\pgfqpoint{1.963636in}{3.696000in}} %
\pgfusepath{clip}%
\pgfsetbuttcap%
\pgfsetroundjoin%
\definecolor{currentfill}{rgb}{0.744232,0.218288,0.520524}%
\pgfsetfillcolor{currentfill}%
\pgfsetlinewidth{0.000000pt}%
\definecolor{currentstroke}{rgb}{0.000000,0.000000,0.000000}%
\pgfsetstrokecolor{currentstroke}%
\pgfsetdash{}{0pt}%
\pgfpathmoveto{\pgfqpoint{6.101818in}{-2.072672in}}%
\pgfpathlineto{\pgfqpoint{6.101818in}{-2.000000in}}%
\pgfpathlineto{\pgfqpoint{6.101818in}{-1.893333in}}%
\pgfpathlineto{\pgfqpoint{6.101818in}{-1.786667in}}%
\pgfpathlineto{\pgfqpoint{6.101818in}{-1.680000in}}%
\pgfpathlineto{\pgfqpoint{6.101818in}{-1.573333in}}%
\pgfpathlineto{\pgfqpoint{6.101818in}{-1.466667in}}%
\pgfpathlineto{\pgfqpoint{6.101818in}{-1.360000in}}%
\pgfpathlineto{\pgfqpoint{6.101818in}{-1.253333in}}%
\pgfpathlineto{\pgfqpoint{6.101818in}{-1.146667in}}%
\pgfpathlineto{\pgfqpoint{6.101818in}{-1.049197in}}%
\pgfpathlineto{\pgfqpoint{6.097465in}{-1.040000in}}%
\pgfpathlineto{\pgfqpoint{6.052231in}{-0.944563in}}%
\pgfpathlineto{\pgfqpoint{6.046916in}{-0.933333in}}%
\pgfpathlineto{\pgfqpoint{6.002645in}{-0.839925in}}%
\pgfpathlineto{\pgfqpoint{5.996369in}{-0.826667in}}%
\pgfpathlineto{\pgfqpoint{5.953058in}{-0.735285in}}%
\pgfpathlineto{\pgfqpoint{5.945823in}{-0.720000in}}%
\pgfpathlineto{\pgfqpoint{5.903471in}{-0.630643in}}%
\pgfpathlineto{\pgfqpoint{5.895278in}{-0.613333in}}%
\pgfpathlineto{\pgfqpoint{5.853884in}{-0.525997in}}%
\pgfpathlineto{\pgfqpoint{5.844735in}{-0.506667in}}%
\pgfpathlineto{\pgfqpoint{5.804298in}{-0.421348in}}%
\pgfpathlineto{\pgfqpoint{5.794193in}{-0.400000in}}%
\pgfpathlineto{\pgfqpoint{5.754711in}{-0.316697in}}%
\pgfpathlineto{\pgfqpoint{5.743652in}{-0.293333in}}%
\pgfpathlineto{\pgfqpoint{5.705124in}{-0.212043in}}%
\pgfpathlineto{\pgfqpoint{5.693113in}{-0.186667in}}%
\pgfpathlineto{\pgfqpoint{5.655537in}{-0.107386in}}%
\pgfpathlineto{\pgfqpoint{5.642575in}{-0.080000in}}%
\pgfpathlineto{\pgfqpoint{5.605950in}{-0.002727in}}%
\pgfpathlineto{\pgfqpoint{5.592038in}{0.026667in}}%
\pgfpathlineto{\pgfqpoint{5.556364in}{0.101935in}}%
\pgfpathlineto{\pgfqpoint{5.541502in}{0.133333in}}%
\pgfpathlineto{\pgfqpoint{5.506777in}{0.206600in}}%
\pgfpathlineto{\pgfqpoint{5.490968in}{0.240000in}}%
\pgfpathlineto{\pgfqpoint{5.457190in}{0.311268in}}%
\pgfpathlineto{\pgfqpoint{5.440435in}{0.346667in}}%
\pgfpathlineto{\pgfqpoint{5.407603in}{0.415938in}}%
\pgfpathlineto{\pgfqpoint{5.389903in}{0.453333in}}%
\pgfpathlineto{\pgfqpoint{5.358017in}{0.520612in}}%
\pgfpathlineto{\pgfqpoint{5.339373in}{0.560000in}}%
\pgfpathlineto{\pgfqpoint{5.308430in}{0.625287in}}%
\pgfpathlineto{\pgfqpoint{5.288844in}{0.666667in}}%
\pgfpathlineto{\pgfqpoint{5.258843in}{0.729966in}}%
\pgfpathlineto{\pgfqpoint{5.238316in}{0.773333in}}%
\pgfpathlineto{\pgfqpoint{5.209256in}{0.834647in}}%
\pgfpathlineto{\pgfqpoint{5.187789in}{0.880000in}}%
\pgfpathlineto{\pgfqpoint{5.159669in}{0.939331in}}%
\pgfpathlineto{\pgfqpoint{5.137264in}{0.986667in}}%
\pgfpathlineto{\pgfqpoint{5.110083in}{1.044017in}}%
\pgfpathlineto{\pgfqpoint{5.086740in}{1.093333in}}%
\pgfpathlineto{\pgfqpoint{5.060496in}{1.148706in}}%
\pgfpathlineto{\pgfqpoint{5.036217in}{1.200000in}}%
\pgfpathlineto{\pgfqpoint{5.010909in}{1.253398in}}%
\pgfpathlineto{\pgfqpoint{4.985696in}{1.306667in}}%
\pgfpathlineto{\pgfqpoint{4.961322in}{1.358092in}}%
\pgfpathlineto{\pgfqpoint{4.935175in}{1.413333in}}%
\pgfpathlineto{\pgfqpoint{4.911736in}{1.462789in}}%
\pgfpathlineto{\pgfqpoint{4.884656in}{1.520000in}}%
\pgfpathlineto{\pgfqpoint{4.862149in}{1.567488in}}%
\pgfpathlineto{\pgfqpoint{4.834138in}{1.626667in}}%
\pgfpathlineto{\pgfqpoint{4.812562in}{1.672190in}}%
\pgfpathlineto{\pgfqpoint{4.783621in}{1.733333in}}%
\pgfpathlineto{\pgfqpoint{4.762975in}{1.776895in}}%
\pgfpathlineto{\pgfqpoint{4.733106in}{1.840000in}}%
\pgfpathlineto{\pgfqpoint{4.713388in}{1.881602in}}%
\pgfpathlineto{\pgfqpoint{4.682592in}{1.946667in}}%
\pgfpathlineto{\pgfqpoint{4.663802in}{1.986312in}}%
\pgfpathlineto{\pgfqpoint{4.632078in}{2.053333in}}%
\pgfpathlineto{\pgfqpoint{4.614215in}{2.091024in}}%
\pgfpathlineto{\pgfqpoint{4.581567in}{2.160000in}}%
\pgfpathlineto{\pgfqpoint{4.564628in}{2.195739in}}%
\pgfpathlineto{\pgfqpoint{4.531056in}{2.266667in}}%
\pgfpathlineto{\pgfqpoint{4.515041in}{2.300456in}}%
\pgfpathlineto{\pgfqpoint{4.480546in}{2.373333in}}%
\pgfpathlineto{\pgfqpoint{4.465455in}{2.405176in}}%
\pgfpathlineto{\pgfqpoint{4.430038in}{2.480000in}}%
\pgfpathlineto{\pgfqpoint{4.415868in}{2.509898in}}%
\pgfpathlineto{\pgfqpoint{4.379531in}{2.586667in}}%
\pgfpathlineto{\pgfqpoint{4.366281in}{2.614623in}}%
\pgfpathlineto{\pgfqpoint{4.329025in}{2.693333in}}%
\pgfpathlineto{\pgfqpoint{4.316694in}{2.719351in}}%
\pgfpathlineto{\pgfqpoint{4.278520in}{2.800000in}}%
\pgfpathlineto{\pgfqpoint{4.267107in}{2.824080in}}%
\pgfpathlineto{\pgfqpoint{4.228017in}{2.906667in}}%
\pgfpathlineto{\pgfqpoint{4.217521in}{2.928813in}}%
\pgfpathlineto{\pgfqpoint{4.177514in}{3.013333in}}%
\pgfpathlineto{\pgfqpoint{4.167934in}{3.033547in}}%
\pgfpathlineto{\pgfqpoint{4.127013in}{3.120000in}}%
\pgfpathlineto{\pgfqpoint{4.118347in}{3.138284in}}%
\pgfpathlineto{\pgfqpoint{4.076513in}{3.226667in}}%
\pgfpathlineto{\pgfqpoint{4.068760in}{3.243024in}}%
\pgfpathlineto{\pgfqpoint{4.026014in}{3.333333in}}%
\pgfpathlineto{\pgfqpoint{4.019174in}{3.347766in}}%
\pgfpathlineto{\pgfqpoint{3.975516in}{3.440000in}}%
\pgfpathlineto{\pgfqpoint{3.969587in}{3.452510in}}%
\pgfpathlineto{\pgfqpoint{3.925019in}{3.546667in}}%
\pgfpathlineto{\pgfqpoint{3.920000in}{3.557257in}}%
\pgfpathlineto{\pgfqpoint{3.874524in}{3.653333in}}%
\pgfpathlineto{\pgfqpoint{3.870413in}{3.662006in}}%
\pgfpathlineto{\pgfqpoint{3.824029in}{3.760000in}}%
\pgfpathlineto{\pgfqpoint{3.820826in}{3.766758in}}%
\pgfpathlineto{\pgfqpoint{3.773536in}{3.866667in}}%
\pgfpathlineto{\pgfqpoint{3.771240in}{3.871512in}}%
\pgfpathlineto{\pgfqpoint{3.723044in}{3.973333in}}%
\pgfpathlineto{\pgfqpoint{3.721653in}{3.976268in}}%
\pgfpathlineto{\pgfqpoint{3.672553in}{4.080000in}}%
\pgfpathlineto{\pgfqpoint{3.672066in}{4.081027in}}%
\pgfpathlineto{\pgfqpoint{3.622479in}{4.185723in}}%
\pgfpathlineto{\pgfqpoint{3.622033in}{4.186667in}}%
\pgfpathlineto{\pgfqpoint{3.572893in}{4.290345in}}%
\pgfpathlineto{\pgfqpoint{3.571478in}{4.293333in}}%
\pgfpathlineto{\pgfqpoint{3.523306in}{4.394971in}}%
\pgfpathlineto{\pgfqpoint{3.520925in}{4.400000in}}%
\pgfpathlineto{\pgfqpoint{3.473719in}{4.499600in}}%
\pgfpathlineto{\pgfqpoint{3.470374in}{4.506667in}}%
\pgfpathlineto{\pgfqpoint{3.424132in}{4.604231in}}%
\pgfpathlineto{\pgfqpoint{3.419824in}{4.613333in}}%
\pgfpathlineto{\pgfqpoint{3.374545in}{4.708865in}}%
\pgfpathlineto{\pgfqpoint{3.369275in}{4.720000in}}%
\pgfpathlineto{\pgfqpoint{3.324959in}{4.813502in}}%
\pgfpathlineto{\pgfqpoint{3.318728in}{4.826667in}}%
\pgfpathlineto{\pgfqpoint{3.275372in}{4.918142in}}%
\pgfpathlineto{\pgfqpoint{3.268182in}{4.933333in}}%
\pgfpathlineto{\pgfqpoint{3.225785in}{5.022785in}}%
\pgfpathlineto{\pgfqpoint{3.217637in}{5.040000in}}%
\pgfpathlineto{\pgfqpoint{3.176198in}{5.127430in}}%
\pgfpathlineto{\pgfqpoint{3.167093in}{5.146667in}}%
\pgfpathlineto{\pgfqpoint{3.126612in}{5.232079in}}%
\pgfpathlineto{\pgfqpoint{3.116551in}{5.253333in}}%
\pgfpathlineto{\pgfqpoint{3.077025in}{5.336730in}}%
\pgfpathlineto{\pgfqpoint{3.066011in}{5.360000in}}%
\pgfpathlineto{\pgfqpoint{3.027438in}{5.441384in}}%
\pgfpathlineto{\pgfqpoint{3.015471in}{5.466667in}}%
\pgfpathlineto{\pgfqpoint{2.977851in}{5.546040in}}%
\pgfpathlineto{\pgfqpoint{2.964933in}{5.573333in}}%
\pgfpathlineto{\pgfqpoint{2.928264in}{5.650700in}}%
\pgfpathlineto{\pgfqpoint{2.914396in}{5.680000in}}%
\pgfpathlineto{\pgfqpoint{2.878678in}{5.755362in}}%
\pgfpathlineto{\pgfqpoint{2.863860in}{5.786667in}}%
\pgfpathlineto{\pgfqpoint{2.829091in}{5.860027in}}%
\pgfpathlineto{\pgfqpoint{2.813326in}{5.893333in}}%
\pgfpathlineto{\pgfqpoint{2.779504in}{5.964694in}}%
\pgfpathlineto{\pgfqpoint{2.762793in}{6.000000in}}%
\pgfpathlineto{\pgfqpoint{2.729917in}{6.069365in}}%
\pgfpathlineto{\pgfqpoint{2.712261in}{6.106667in}}%
\pgfpathlineto{\pgfqpoint{2.680331in}{6.174038in}}%
\pgfpathlineto{\pgfqpoint{2.661731in}{6.213333in}}%
\pgfpathlineto{\pgfqpoint{2.630744in}{6.278713in}}%
\pgfpathlineto{\pgfqpoint{2.611202in}{6.320000in}}%
\pgfpathlineto{\pgfqpoint{2.581157in}{6.383392in}}%
\pgfpathlineto{\pgfqpoint{2.560674in}{6.426667in}}%
\pgfpathlineto{\pgfqpoint{2.531570in}{6.488073in}}%
\pgfpathlineto{\pgfqpoint{2.510147in}{6.533333in}}%
\pgfpathlineto{\pgfqpoint{2.481983in}{6.592756in}}%
\pgfpathlineto{\pgfqpoint{2.459622in}{6.640000in}}%
\pgfpathlineto{\pgfqpoint{2.432397in}{6.697442in}}%
\pgfpathlineto{\pgfqpoint{2.409098in}{6.746667in}}%
\pgfpathlineto{\pgfqpoint{2.382810in}{6.802131in}}%
\pgfpathlineto{\pgfqpoint{2.358575in}{6.853333in}}%
\pgfpathlineto{\pgfqpoint{2.333223in}{6.906823in}}%
\pgfpathlineto{\pgfqpoint{2.308053in}{6.960000in}}%
\pgfpathlineto{\pgfqpoint{2.283636in}{7.011517in}}%
\pgfpathlineto{\pgfqpoint{2.257533in}{7.066667in}}%
\pgfpathlineto{\pgfqpoint{2.234050in}{7.116214in}}%
\pgfpathlineto{\pgfqpoint{2.207013in}{7.173333in}}%
\pgfpathlineto{\pgfqpoint{2.184463in}{7.220913in}}%
\pgfpathlineto{\pgfqpoint{2.156495in}{7.280000in}}%
\pgfpathlineto{\pgfqpoint{2.134876in}{7.325615in}}%
\pgfpathlineto{\pgfqpoint{2.105979in}{7.386667in}}%
\pgfpathlineto{\pgfqpoint{2.085289in}{7.430320in}}%
\pgfpathlineto{\pgfqpoint{2.055463in}{7.493333in}}%
\pgfpathlineto{\pgfqpoint{2.035702in}{7.535027in}}%
\pgfpathlineto{\pgfqpoint{2.004949in}{7.600000in}}%
\pgfpathlineto{\pgfqpoint{1.986116in}{7.639736in}}%
\pgfpathlineto{\pgfqpoint{1.954436in}{7.706667in}}%
\pgfpathlineto{\pgfqpoint{1.936529in}{7.744448in}}%
\pgfpathlineto{\pgfqpoint{1.903924in}{7.813333in}}%
\pgfpathlineto{\pgfqpoint{1.886942in}{7.849163in}}%
\pgfpathlineto{\pgfqpoint{1.853413in}{7.920000in}}%
\pgfpathlineto{\pgfqpoint{1.837355in}{7.920000in}}%
\pgfpathlineto{\pgfqpoint{1.787769in}{7.920000in}}%
\pgfpathlineto{\pgfqpoint{1.738182in}{7.920000in}}%
\pgfpathlineto{\pgfqpoint{1.688595in}{7.920000in}}%
\pgfpathlineto{\pgfqpoint{1.639008in}{7.920000in}}%
\pgfpathlineto{\pgfqpoint{1.589421in}{7.920000in}}%
\pgfpathlineto{\pgfqpoint{1.539835in}{7.920000in}}%
\pgfpathlineto{\pgfqpoint{1.490248in}{7.920000in}}%
\pgfpathlineto{\pgfqpoint{1.440661in}{7.920000in}}%
\pgfpathlineto{\pgfqpoint{1.391074in}{7.920000in}}%
\pgfpathlineto{\pgfqpoint{1.368771in}{7.920000in}}%
\pgfpathlineto{\pgfqpoint{1.391074in}{7.872811in}}%
\pgfpathlineto{\pgfqpoint{1.419305in}{7.813333in}}%
\pgfpathlineto{\pgfqpoint{1.440661in}{7.768148in}}%
\pgfpathlineto{\pgfqpoint{1.469844in}{7.706667in}}%
\pgfpathlineto{\pgfqpoint{1.490248in}{7.663494in}}%
\pgfpathlineto{\pgfqpoint{1.520386in}{7.600000in}}%
\pgfpathlineto{\pgfqpoint{1.539835in}{7.558849in}}%
\pgfpathlineto{\pgfqpoint{1.570933in}{7.493333in}}%
\pgfpathlineto{\pgfqpoint{1.589421in}{7.454213in}}%
\pgfpathlineto{\pgfqpoint{1.621484in}{7.386667in}}%
\pgfpathlineto{\pgfqpoint{1.639008in}{7.349587in}}%
\pgfpathlineto{\pgfqpoint{1.672039in}{7.280000in}}%
\pgfpathlineto{\pgfqpoint{1.688595in}{7.244969in}}%
\pgfpathlineto{\pgfqpoint{1.722599in}{7.173333in}}%
\pgfpathlineto{\pgfqpoint{1.738182in}{7.140361in}}%
\pgfpathlineto{\pgfqpoint{1.773163in}{7.066667in}}%
\pgfpathlineto{\pgfqpoint{1.787769in}{7.035762in}}%
\pgfpathlineto{\pgfqpoint{1.823732in}{6.960000in}}%
\pgfpathlineto{\pgfqpoint{1.837355in}{6.931172in}}%
\pgfpathlineto{\pgfqpoint{1.874305in}{6.853333in}}%
\pgfpathlineto{\pgfqpoint{1.886942in}{6.826593in}}%
\pgfpathlineto{\pgfqpoint{1.924883in}{6.746667in}}%
\pgfpathlineto{\pgfqpoint{1.936529in}{6.722023in}}%
\pgfpathlineto{\pgfqpoint{1.975465in}{6.640000in}}%
\pgfpathlineto{\pgfqpoint{1.986116in}{6.617463in}}%
\pgfpathlineto{\pgfqpoint{2.026053in}{6.533333in}}%
\pgfpathlineto{\pgfqpoint{2.035702in}{6.512914in}}%
\pgfpathlineto{\pgfqpoint{2.076645in}{6.426667in}}%
\pgfpathlineto{\pgfqpoint{2.085289in}{6.408374in}}%
\pgfpathlineto{\pgfqpoint{2.127242in}{6.320000in}}%
\pgfpathlineto{\pgfqpoint{2.134876in}{6.303845in}}%
\pgfpathlineto{\pgfqpoint{2.177844in}{6.213333in}}%
\pgfpathlineto{\pgfqpoint{2.184463in}{6.199327in}}%
\pgfpathlineto{\pgfqpoint{2.228451in}{6.106667in}}%
\pgfpathlineto{\pgfqpoint{2.234050in}{6.094819in}}%
\pgfpathlineto{\pgfqpoint{2.279063in}{6.000000in}}%
\pgfpathlineto{\pgfqpoint{2.283636in}{5.990322in}}%
\pgfpathlineto{\pgfqpoint{2.329680in}{5.893333in}}%
\pgfpathlineto{\pgfqpoint{2.333223in}{5.885836in}}%
\pgfpathlineto{\pgfqpoint{2.380303in}{5.786667in}}%
\pgfpathlineto{\pgfqpoint{2.382810in}{5.781361in}}%
\pgfpathlineto{\pgfqpoint{2.430930in}{5.680000in}}%
\pgfpathlineto{\pgfqpoint{2.432397in}{5.676897in}}%
\pgfpathlineto{\pgfqpoint{2.481564in}{5.573333in}}%
\pgfpathlineto{\pgfqpoint{2.481983in}{5.572445in}}%
\pgfpathlineto{\pgfqpoint{2.531570in}{5.467732in}}%
\pgfpathlineto{\pgfqpoint{2.532076in}{5.466667in}}%
\pgfpathlineto{\pgfqpoint{2.581157in}{5.362845in}}%
\pgfpathlineto{\pgfqpoint{2.582507in}{5.360000in}}%
\pgfpathlineto{\pgfqpoint{2.630744in}{5.257965in}}%
\pgfpathlineto{\pgfqpoint{2.632942in}{5.253333in}}%
\pgfpathlineto{\pgfqpoint{2.680331in}{5.153090in}}%
\pgfpathlineto{\pgfqpoint{2.683379in}{5.146667in}}%
\pgfpathlineto{\pgfqpoint{2.729917in}{5.048221in}}%
\pgfpathlineto{\pgfqpoint{2.733819in}{5.040000in}}%
\pgfpathlineto{\pgfqpoint{2.779504in}{4.943359in}}%
\pgfpathlineto{\pgfqpoint{2.784262in}{4.933333in}}%
\pgfpathlineto{\pgfqpoint{2.829091in}{4.838503in}}%
\pgfpathlineto{\pgfqpoint{2.834708in}{4.826667in}}%
\pgfpathlineto{\pgfqpoint{2.878678in}{4.733653in}}%
\pgfpathlineto{\pgfqpoint{2.885157in}{4.720000in}}%
\pgfpathlineto{\pgfqpoint{2.928264in}{4.628810in}}%
\pgfpathlineto{\pgfqpoint{2.935609in}{4.613333in}}%
\pgfpathlineto{\pgfqpoint{2.977851in}{4.523974in}}%
\pgfpathlineto{\pgfqpoint{2.986064in}{4.506667in}}%
\pgfpathlineto{\pgfqpoint{3.027438in}{4.419144in}}%
\pgfpathlineto{\pgfqpoint{3.036523in}{4.400000in}}%
\pgfpathlineto{\pgfqpoint{3.077025in}{4.314320in}}%
\pgfpathlineto{\pgfqpoint{3.086985in}{4.293333in}}%
\pgfpathlineto{\pgfqpoint{3.126612in}{4.209503in}}%
\pgfpathlineto{\pgfqpoint{3.137449in}{4.186667in}}%
\pgfpathlineto{\pgfqpoint{3.176198in}{4.104694in}}%
\pgfpathlineto{\pgfqpoint{3.187917in}{4.080000in}}%
\pgfpathlineto{\pgfqpoint{3.225785in}{3.999891in}}%
\pgfpathlineto{\pgfqpoint{3.238389in}{3.973333in}}%
\pgfpathlineto{\pgfqpoint{3.275372in}{3.895095in}}%
\pgfpathlineto{\pgfqpoint{3.288863in}{3.866667in}}%
\pgfpathlineto{\pgfqpoint{3.324959in}{3.790306in}}%
\pgfpathlineto{\pgfqpoint{3.339342in}{3.760000in}}%
\pgfpathlineto{\pgfqpoint{3.374545in}{3.685524in}}%
\pgfpathlineto{\pgfqpoint{3.389823in}{3.653333in}}%
\pgfpathlineto{\pgfqpoint{3.424132in}{3.580750in}}%
\pgfpathlineto{\pgfqpoint{3.440308in}{3.546667in}}%
\pgfpathlineto{\pgfqpoint{3.473719in}{3.475983in}}%
\pgfpathlineto{\pgfqpoint{3.490796in}{3.440000in}}%
\pgfpathlineto{\pgfqpoint{3.523306in}{3.371223in}}%
\pgfpathlineto{\pgfqpoint{3.541288in}{3.333333in}}%
\pgfpathlineto{\pgfqpoint{3.572893in}{3.266471in}}%
\pgfpathlineto{\pgfqpoint{3.591784in}{3.226667in}}%
\pgfpathlineto{\pgfqpoint{3.622479in}{3.161726in}}%
\pgfpathlineto{\pgfqpoint{3.642283in}{3.120000in}}%
\pgfpathlineto{\pgfqpoint{3.672066in}{3.056990in}}%
\pgfpathlineto{\pgfqpoint{3.692786in}{3.013333in}}%
\pgfpathlineto{\pgfqpoint{3.721653in}{2.952261in}}%
\pgfpathlineto{\pgfqpoint{3.743293in}{2.906667in}}%
\pgfpathlineto{\pgfqpoint{3.771240in}{2.847539in}}%
\pgfpathlineto{\pgfqpoint{3.793803in}{2.800000in}}%
\pgfpathlineto{\pgfqpoint{3.820826in}{2.742826in}}%
\pgfpathlineto{\pgfqpoint{3.844317in}{2.693333in}}%
\pgfpathlineto{\pgfqpoint{3.870413in}{2.638121in}}%
\pgfpathlineto{\pgfqpoint{3.894835in}{2.586667in}}%
\pgfpathlineto{\pgfqpoint{3.920000in}{2.533425in}}%
\pgfpathlineto{\pgfqpoint{3.945357in}{2.480000in}}%
\pgfpathlineto{\pgfqpoint{3.969587in}{2.428736in}}%
\pgfpathlineto{\pgfqpoint{3.995883in}{2.373333in}}%
\pgfpathlineto{\pgfqpoint{4.019174in}{2.324056in}}%
\pgfpathlineto{\pgfqpoint{4.046413in}{2.266667in}}%
\pgfpathlineto{\pgfqpoint{4.068760in}{2.219384in}}%
\pgfpathlineto{\pgfqpoint{4.096947in}{2.160000in}}%
\pgfpathlineto{\pgfqpoint{4.118347in}{2.114721in}}%
\pgfpathlineto{\pgfqpoint{4.147485in}{2.053333in}}%
\pgfpathlineto{\pgfqpoint{4.167934in}{2.010067in}}%
\pgfpathlineto{\pgfqpoint{4.198028in}{1.946667in}}%
\pgfpathlineto{\pgfqpoint{4.217521in}{1.905422in}}%
\pgfpathlineto{\pgfqpoint{4.248574in}{1.840000in}}%
\pgfpathlineto{\pgfqpoint{4.267107in}{1.800785in}}%
\pgfpathlineto{\pgfqpoint{4.299125in}{1.733333in}}%
\pgfpathlineto{\pgfqpoint{4.316694in}{1.696158in}}%
\pgfpathlineto{\pgfqpoint{4.349680in}{1.626667in}}%
\pgfpathlineto{\pgfqpoint{4.366281in}{1.591540in}}%
\pgfpathlineto{\pgfqpoint{4.400239in}{1.520000in}}%
\pgfpathlineto{\pgfqpoint{4.415868in}{1.486931in}}%
\pgfpathlineto{\pgfqpoint{4.450803in}{1.413333in}}%
\pgfpathlineto{\pgfqpoint{4.465455in}{1.382332in}}%
\pgfpathlineto{\pgfqpoint{4.501372in}{1.306667in}}%
\pgfpathlineto{\pgfqpoint{4.515041in}{1.277742in}}%
\pgfpathlineto{\pgfqpoint{4.551945in}{1.200000in}}%
\pgfpathlineto{\pgfqpoint{4.564628in}{1.173162in}}%
\pgfpathlineto{\pgfqpoint{4.602523in}{1.093333in}}%
\pgfpathlineto{\pgfqpoint{4.614215in}{1.068592in}}%
\pgfpathlineto{\pgfqpoint{4.653105in}{0.986667in}}%
\pgfpathlineto{\pgfqpoint{4.663802in}{0.964032in}}%
\pgfpathlineto{\pgfqpoint{4.703692in}{0.880000in}}%
\pgfpathlineto{\pgfqpoint{4.713388in}{0.859482in}}%
\pgfpathlineto{\pgfqpoint{4.754284in}{0.773333in}}%
\pgfpathlineto{\pgfqpoint{4.762975in}{0.754942in}}%
\pgfpathlineto{\pgfqpoint{4.804881in}{0.666667in}}%
\pgfpathlineto{\pgfqpoint{4.812562in}{0.650412in}}%
\pgfpathlineto{\pgfqpoint{4.855482in}{0.560000in}}%
\pgfpathlineto{\pgfqpoint{4.862149in}{0.545893in}}%
\pgfpathlineto{\pgfqpoint{4.906089in}{0.453333in}}%
\pgfpathlineto{\pgfqpoint{4.911736in}{0.441385in}}%
\pgfpathlineto{\pgfqpoint{4.956701in}{0.346667in}}%
\pgfpathlineto{\pgfqpoint{4.961322in}{0.336887in}}%
\pgfpathlineto{\pgfqpoint{5.007318in}{0.240000in}}%
\pgfpathlineto{\pgfqpoint{5.010909in}{0.232401in}}%
\pgfpathlineto{\pgfqpoint{5.057940in}{0.133333in}}%
\pgfpathlineto{\pgfqpoint{5.060496in}{0.127925in}}%
\pgfpathlineto{\pgfqpoint{5.108568in}{0.026667in}}%
\pgfpathlineto{\pgfqpoint{5.110083in}{0.023461in}}%
\pgfpathlineto{\pgfqpoint{5.159201in}{-0.080000in}}%
\pgfpathlineto{\pgfqpoint{5.159669in}{-0.080992in}}%
\pgfpathlineto{\pgfqpoint{5.209256in}{-0.185684in}}%
\pgfpathlineto{\pgfqpoint{5.209723in}{-0.186667in}}%
\pgfpathlineto{\pgfqpoint{5.258843in}{-0.290571in}}%
\pgfpathlineto{\pgfqpoint{5.260154in}{-0.293333in}}%
\pgfpathlineto{\pgfqpoint{5.308430in}{-0.395452in}}%
\pgfpathlineto{\pgfqpoint{5.310588in}{-0.400000in}}%
\pgfpathlineto{\pgfqpoint{5.358017in}{-0.500327in}}%
\pgfpathlineto{\pgfqpoint{5.361025in}{-0.506667in}}%
\pgfpathlineto{\pgfqpoint{5.407603in}{-0.605196in}}%
\pgfpathlineto{\pgfqpoint{5.411465in}{-0.613333in}}%
\pgfpathlineto{\pgfqpoint{5.457190in}{-0.710058in}}%
\pgfpathlineto{\pgfqpoint{5.461908in}{-0.720000in}}%
\pgfpathlineto{\pgfqpoint{5.506777in}{-0.814915in}}%
\pgfpathlineto{\pgfqpoint{5.512354in}{-0.826667in}}%
\pgfpathlineto{\pgfqpoint{5.556364in}{-0.919764in}}%
\pgfpathlineto{\pgfqpoint{5.562803in}{-0.933333in}}%
\pgfpathlineto{\pgfqpoint{5.605950in}{-1.024608in}}%
\pgfpathlineto{\pgfqpoint{5.613255in}{-1.040000in}}%
\pgfpathlineto{\pgfqpoint{5.655537in}{-1.129445in}}%
\pgfpathlineto{\pgfqpoint{5.663710in}{-1.146667in}}%
\pgfpathlineto{\pgfqpoint{5.705124in}{-1.234275in}}%
\pgfpathlineto{\pgfqpoint{5.714168in}{-1.253333in}}%
\pgfpathlineto{\pgfqpoint{5.754711in}{-1.339099in}}%
\pgfpathlineto{\pgfqpoint{5.764630in}{-1.360000in}}%
\pgfpathlineto{\pgfqpoint{5.804298in}{-1.443916in}}%
\pgfpathlineto{\pgfqpoint{5.815094in}{-1.466667in}}%
\pgfpathlineto{\pgfqpoint{5.853884in}{-1.548726in}}%
\pgfpathlineto{\pgfqpoint{5.865562in}{-1.573333in}}%
\pgfpathlineto{\pgfqpoint{5.903471in}{-1.653529in}}%
\pgfpathlineto{\pgfqpoint{5.916034in}{-1.680000in}}%
\pgfpathlineto{\pgfqpoint{5.953058in}{-1.758326in}}%
\pgfpathlineto{\pgfqpoint{5.966508in}{-1.786667in}}%
\pgfpathlineto{\pgfqpoint{6.002645in}{-1.863115in}}%
\pgfpathlineto{\pgfqpoint{6.016986in}{-1.893333in}}%
\pgfpathlineto{\pgfqpoint{6.052231in}{-1.967897in}}%
\pgfpathlineto{\pgfqpoint{6.067467in}{-2.000000in}}%
\pgfpathclose%
\pgfusepath{fill}%
\end{pgfscope}%
\begin{pgfscope}%
\pgfpathrectangle{\pgfqpoint{3.156364in}{0.528000in}}{\pgfqpoint{1.963636in}{3.696000in}} %
\pgfusepath{clip}%
\pgfsetbuttcap%
\pgfsetroundjoin%
\definecolor{currentfill}{rgb}{0.944844,0.507658,0.302433}%
\pgfsetfillcolor{currentfill}%
\pgfsetlinewidth{0.000000pt}%
\definecolor{currentstroke}{rgb}{0.000000,0.000000,0.000000}%
\pgfsetstrokecolor{currentstroke}%
\pgfsetdash{}{0pt}%
\pgfpathmoveto{\pgfqpoint{3.870413in}{-2.553510in}}%
\pgfpathlineto{\pgfqpoint{3.911370in}{-2.640000in}}%
\pgfpathlineto{\pgfqpoint{3.920000in}{-2.640000in}}%
\pgfpathlineto{\pgfqpoint{3.969587in}{-2.640000in}}%
\pgfpathlineto{\pgfqpoint{4.019174in}{-2.640000in}}%
\pgfpathlineto{\pgfqpoint{4.068760in}{-2.640000in}}%
\pgfpathlineto{\pgfqpoint{4.118347in}{-2.640000in}}%
\pgfpathlineto{\pgfqpoint{4.167934in}{-2.640000in}}%
\pgfpathlineto{\pgfqpoint{4.217521in}{-2.640000in}}%
\pgfpathlineto{\pgfqpoint{4.267107in}{-2.640000in}}%
\pgfpathlineto{\pgfqpoint{4.316694in}{-2.640000in}}%
\pgfpathlineto{\pgfqpoint{4.366281in}{-2.640000in}}%
\pgfpathlineto{\pgfqpoint{4.415868in}{-2.640000in}}%
\pgfpathlineto{\pgfqpoint{4.465455in}{-2.640000in}}%
\pgfpathlineto{\pgfqpoint{4.515041in}{-2.640000in}}%
\pgfpathlineto{\pgfqpoint{4.564628in}{-2.640000in}}%
\pgfpathlineto{\pgfqpoint{4.614215in}{-2.640000in}}%
\pgfpathlineto{\pgfqpoint{4.663802in}{-2.640000in}}%
\pgfpathlineto{\pgfqpoint{4.713388in}{-2.640000in}}%
\pgfpathlineto{\pgfqpoint{4.762975in}{-2.640000in}}%
\pgfpathlineto{\pgfqpoint{4.812562in}{-2.640000in}}%
\pgfpathlineto{\pgfqpoint{4.862149in}{-2.640000in}}%
\pgfpathlineto{\pgfqpoint{4.911736in}{-2.640000in}}%
\pgfpathlineto{\pgfqpoint{4.961322in}{-2.640000in}}%
\pgfpathlineto{\pgfqpoint{5.010909in}{-2.640000in}}%
\pgfpathlineto{\pgfqpoint{5.060496in}{-2.640000in}}%
\pgfpathlineto{\pgfqpoint{5.110083in}{-2.640000in}}%
\pgfpathlineto{\pgfqpoint{5.159669in}{-2.640000in}}%
\pgfpathlineto{\pgfqpoint{5.209256in}{-2.640000in}}%
\pgfpathlineto{\pgfqpoint{5.258843in}{-2.640000in}}%
\pgfpathlineto{\pgfqpoint{5.308430in}{-2.640000in}}%
\pgfpathlineto{\pgfqpoint{5.358017in}{-2.640000in}}%
\pgfpathlineto{\pgfqpoint{5.407603in}{-2.640000in}}%
\pgfpathlineto{\pgfqpoint{5.441133in}{-2.640000in}}%
\pgfpathlineto{\pgfqpoint{5.407603in}{-2.569163in}}%
\pgfpathlineto{\pgfqpoint{5.390622in}{-2.533333in}}%
\pgfpathlineto{\pgfqpoint{5.358017in}{-2.464448in}}%
\pgfpathlineto{\pgfqpoint{5.340110in}{-2.426667in}}%
\pgfpathlineto{\pgfqpoint{5.308430in}{-2.359736in}}%
\pgfpathlineto{\pgfqpoint{5.289597in}{-2.320000in}}%
\pgfpathlineto{\pgfqpoint{5.258843in}{-2.255027in}}%
\pgfpathlineto{\pgfqpoint{5.239082in}{-2.213333in}}%
\pgfpathlineto{\pgfqpoint{5.209256in}{-2.150320in}}%
\pgfpathlineto{\pgfqpoint{5.188567in}{-2.106667in}}%
\pgfpathlineto{\pgfqpoint{5.159669in}{-2.045615in}}%
\pgfpathlineto{\pgfqpoint{5.138050in}{-2.000000in}}%
\pgfpathlineto{\pgfqpoint{5.110083in}{-1.940913in}}%
\pgfpathlineto{\pgfqpoint{5.087532in}{-1.893333in}}%
\pgfpathlineto{\pgfqpoint{5.060496in}{-1.836214in}}%
\pgfpathlineto{\pgfqpoint{5.037013in}{-1.786667in}}%
\pgfpathlineto{\pgfqpoint{5.010909in}{-1.731517in}}%
\pgfpathlineto{\pgfqpoint{4.986492in}{-1.680000in}}%
\pgfpathlineto{\pgfqpoint{4.961322in}{-1.626823in}}%
\pgfpathlineto{\pgfqpoint{4.935971in}{-1.573333in}}%
\pgfpathlineto{\pgfqpoint{4.911736in}{-1.522131in}}%
\pgfpathlineto{\pgfqpoint{4.885448in}{-1.466667in}}%
\pgfpathlineto{\pgfqpoint{4.862149in}{-1.417442in}}%
\pgfpathlineto{\pgfqpoint{4.834924in}{-1.360000in}}%
\pgfpathlineto{\pgfqpoint{4.812562in}{-1.312756in}}%
\pgfpathlineto{\pgfqpoint{4.784398in}{-1.253333in}}%
\pgfpathlineto{\pgfqpoint{4.762975in}{-1.208073in}}%
\pgfpathlineto{\pgfqpoint{4.733872in}{-1.146667in}}%
\pgfpathlineto{\pgfqpoint{4.713388in}{-1.103392in}}%
\pgfpathlineto{\pgfqpoint{4.683344in}{-1.040000in}}%
\pgfpathlineto{\pgfqpoint{4.663802in}{-0.998713in}}%
\pgfpathlineto{\pgfqpoint{4.632815in}{-0.933333in}}%
\pgfpathlineto{\pgfqpoint{4.614215in}{-0.894038in}}%
\pgfpathlineto{\pgfqpoint{4.582284in}{-0.826667in}}%
\pgfpathlineto{\pgfqpoint{4.564628in}{-0.789365in}}%
\pgfpathlineto{\pgfqpoint{4.531752in}{-0.720000in}}%
\pgfpathlineto{\pgfqpoint{4.515041in}{-0.684694in}}%
\pgfpathlineto{\pgfqpoint{4.481219in}{-0.613333in}}%
\pgfpathlineto{\pgfqpoint{4.465455in}{-0.580027in}}%
\pgfpathlineto{\pgfqpoint{4.430685in}{-0.506667in}}%
\pgfpathlineto{\pgfqpoint{4.415868in}{-0.475362in}}%
\pgfpathlineto{\pgfqpoint{4.380149in}{-0.400000in}}%
\pgfpathlineto{\pgfqpoint{4.366281in}{-0.370700in}}%
\pgfpathlineto{\pgfqpoint{4.329613in}{-0.293333in}}%
\pgfpathlineto{\pgfqpoint{4.316694in}{-0.266040in}}%
\pgfpathlineto{\pgfqpoint{4.279074in}{-0.186667in}}%
\pgfpathlineto{\pgfqpoint{4.267107in}{-0.161384in}}%
\pgfpathlineto{\pgfqpoint{4.228535in}{-0.080000in}}%
\pgfpathlineto{\pgfqpoint{4.217521in}{-0.056730in}}%
\pgfpathlineto{\pgfqpoint{4.177994in}{0.026667in}}%
\pgfpathlineto{\pgfqpoint{4.167934in}{0.047921in}}%
\pgfpathlineto{\pgfqpoint{4.127452in}{0.133333in}}%
\pgfpathlineto{\pgfqpoint{4.118347in}{0.152570in}}%
\pgfpathlineto{\pgfqpoint{4.076909in}{0.240000in}}%
\pgfpathlineto{\pgfqpoint{4.068760in}{0.257215in}}%
\pgfpathlineto{\pgfqpoint{4.026364in}{0.346667in}}%
\pgfpathlineto{\pgfqpoint{4.019174in}{0.361858in}}%
\pgfpathlineto{\pgfqpoint{3.975818in}{0.453333in}}%
\pgfpathlineto{\pgfqpoint{3.969587in}{0.466498in}}%
\pgfpathlineto{\pgfqpoint{3.925270in}{0.560000in}}%
\pgfpathlineto{\pgfqpoint{3.920000in}{0.571135in}}%
\pgfpathlineto{\pgfqpoint{3.874722in}{0.666667in}}%
\pgfpathlineto{\pgfqpoint{3.870413in}{0.675769in}}%
\pgfpathlineto{\pgfqpoint{3.824171in}{0.773333in}}%
\pgfpathlineto{\pgfqpoint{3.820826in}{0.780400in}}%
\pgfpathlineto{\pgfqpoint{3.773620in}{0.880000in}}%
\pgfpathlineto{\pgfqpoint{3.771240in}{0.885029in}}%
\pgfpathlineto{\pgfqpoint{3.723067in}{0.986667in}}%
\pgfpathlineto{\pgfqpoint{3.721653in}{0.989655in}}%
\pgfpathlineto{\pgfqpoint{3.672513in}{1.093333in}}%
\pgfpathlineto{\pgfqpoint{3.672066in}{1.094277in}}%
\pgfpathlineto{\pgfqpoint{3.622479in}{1.198973in}}%
\pgfpathlineto{\pgfqpoint{3.621993in}{1.200000in}}%
\pgfpathlineto{\pgfqpoint{3.572893in}{1.303732in}}%
\pgfpathlineto{\pgfqpoint{3.571502in}{1.306667in}}%
\pgfpathlineto{\pgfqpoint{3.523306in}{1.408488in}}%
\pgfpathlineto{\pgfqpoint{3.521010in}{1.413333in}}%
\pgfpathlineto{\pgfqpoint{3.473719in}{1.513242in}}%
\pgfpathlineto{\pgfqpoint{3.470516in}{1.520000in}}%
\pgfpathlineto{\pgfqpoint{3.424132in}{1.617994in}}%
\pgfpathlineto{\pgfqpoint{3.420022in}{1.626667in}}%
\pgfpathlineto{\pgfqpoint{3.374545in}{1.722743in}}%
\pgfpathlineto{\pgfqpoint{3.369526in}{1.733333in}}%
\pgfpathlineto{\pgfqpoint{3.324959in}{1.827490in}}%
\pgfpathlineto{\pgfqpoint{3.319030in}{1.840000in}}%
\pgfpathlineto{\pgfqpoint{3.275372in}{1.932234in}}%
\pgfpathlineto{\pgfqpoint{3.268532in}{1.946667in}}%
\pgfpathlineto{\pgfqpoint{3.225785in}{2.036976in}}%
\pgfpathlineto{\pgfqpoint{3.218033in}{2.053333in}}%
\pgfpathlineto{\pgfqpoint{3.176198in}{2.141716in}}%
\pgfpathlineto{\pgfqpoint{3.167533in}{2.160000in}}%
\pgfpathlineto{\pgfqpoint{3.126612in}{2.246453in}}%
\pgfpathlineto{\pgfqpoint{3.117031in}{2.266667in}}%
\pgfpathlineto{\pgfqpoint{3.077025in}{2.351187in}}%
\pgfpathlineto{\pgfqpoint{3.066529in}{2.373333in}}%
\pgfpathlineto{\pgfqpoint{3.027438in}{2.455920in}}%
\pgfpathlineto{\pgfqpoint{3.016025in}{2.480000in}}%
\pgfpathlineto{\pgfqpoint{2.977851in}{2.560649in}}%
\pgfpathlineto{\pgfqpoint{2.965520in}{2.586667in}}%
\pgfpathlineto{\pgfqpoint{2.928264in}{2.665377in}}%
\pgfpathlineto{\pgfqpoint{2.915014in}{2.693333in}}%
\pgfpathlineto{\pgfqpoint{2.878678in}{2.770102in}}%
\pgfpathlineto{\pgfqpoint{2.864507in}{2.800000in}}%
\pgfpathlineto{\pgfqpoint{2.829091in}{2.874824in}}%
\pgfpathlineto{\pgfqpoint{2.813999in}{2.906667in}}%
\pgfpathlineto{\pgfqpoint{2.779504in}{2.979544in}}%
\pgfpathlineto{\pgfqpoint{2.763490in}{3.013333in}}%
\pgfpathlineto{\pgfqpoint{2.729917in}{3.084261in}}%
\pgfpathlineto{\pgfqpoint{2.712979in}{3.120000in}}%
\pgfpathlineto{\pgfqpoint{2.680331in}{3.188976in}}%
\pgfpathlineto{\pgfqpoint{2.662467in}{3.226667in}}%
\pgfpathlineto{\pgfqpoint{2.630744in}{3.293688in}}%
\pgfpathlineto{\pgfqpoint{2.611954in}{3.333333in}}%
\pgfpathlineto{\pgfqpoint{2.581157in}{3.398398in}}%
\pgfpathlineto{\pgfqpoint{2.561440in}{3.440000in}}%
\pgfpathlineto{\pgfqpoint{2.531570in}{3.503105in}}%
\pgfpathlineto{\pgfqpoint{2.510924in}{3.546667in}}%
\pgfpathlineto{\pgfqpoint{2.481983in}{3.607810in}}%
\pgfpathlineto{\pgfqpoint{2.460407in}{3.653333in}}%
\pgfpathlineto{\pgfqpoint{2.432397in}{3.712512in}}%
\pgfpathlineto{\pgfqpoint{2.409889in}{3.760000in}}%
\pgfpathlineto{\pgfqpoint{2.382810in}{3.817211in}}%
\pgfpathlineto{\pgfqpoint{2.359370in}{3.866667in}}%
\pgfpathlineto{\pgfqpoint{2.333223in}{3.921908in}}%
\pgfpathlineto{\pgfqpoint{2.308850in}{3.973333in}}%
\pgfpathlineto{\pgfqpoint{2.283636in}{4.026602in}}%
\pgfpathlineto{\pgfqpoint{2.258328in}{4.080000in}}%
\pgfpathlineto{\pgfqpoint{2.234050in}{4.131294in}}%
\pgfpathlineto{\pgfqpoint{2.207805in}{4.186667in}}%
\pgfpathlineto{\pgfqpoint{2.184463in}{4.235983in}}%
\pgfpathlineto{\pgfqpoint{2.157281in}{4.293333in}}%
\pgfpathlineto{\pgfqpoint{2.134876in}{4.340669in}}%
\pgfpathlineto{\pgfqpoint{2.106756in}{4.400000in}}%
\pgfpathlineto{\pgfqpoint{2.085289in}{4.445353in}}%
\pgfpathlineto{\pgfqpoint{2.056229in}{4.506667in}}%
\pgfpathlineto{\pgfqpoint{2.035702in}{4.550034in}}%
\pgfpathlineto{\pgfqpoint{2.005702in}{4.613333in}}%
\pgfpathlineto{\pgfqpoint{1.986116in}{4.654713in}}%
\pgfpathlineto{\pgfqpoint{1.955172in}{4.720000in}}%
\pgfpathlineto{\pgfqpoint{1.936529in}{4.759388in}}%
\pgfpathlineto{\pgfqpoint{1.904642in}{4.826667in}}%
\pgfpathlineto{\pgfqpoint{1.886942in}{4.864062in}}%
\pgfpathlineto{\pgfqpoint{1.854110in}{4.933333in}}%
\pgfpathlineto{\pgfqpoint{1.837355in}{4.968732in}}%
\pgfpathlineto{\pgfqpoint{1.803577in}{5.040000in}}%
\pgfpathlineto{\pgfqpoint{1.787769in}{5.073400in}}%
\pgfpathlineto{\pgfqpoint{1.753043in}{5.146667in}}%
\pgfpathlineto{\pgfqpoint{1.738182in}{5.178065in}}%
\pgfpathlineto{\pgfqpoint{1.702508in}{5.253333in}}%
\pgfpathlineto{\pgfqpoint{1.688595in}{5.282727in}}%
\pgfpathlineto{\pgfqpoint{1.651971in}{5.360000in}}%
\pgfpathlineto{\pgfqpoint{1.639008in}{5.387386in}}%
\pgfpathlineto{\pgfqpoint{1.601433in}{5.466667in}}%
\pgfpathlineto{\pgfqpoint{1.589421in}{5.492043in}}%
\pgfpathlineto{\pgfqpoint{1.550893in}{5.573333in}}%
\pgfpathlineto{\pgfqpoint{1.539835in}{5.596697in}}%
\pgfpathlineto{\pgfqpoint{1.500353in}{5.680000in}}%
\pgfpathlineto{\pgfqpoint{1.490248in}{5.701348in}}%
\pgfpathlineto{\pgfqpoint{1.449811in}{5.786667in}}%
\pgfpathlineto{\pgfqpoint{1.440661in}{5.805997in}}%
\pgfpathlineto{\pgfqpoint{1.399267in}{5.893333in}}%
\pgfpathlineto{\pgfqpoint{1.391074in}{5.910643in}}%
\pgfpathlineto{\pgfqpoint{1.348722in}{6.000000in}}%
\pgfpathlineto{\pgfqpoint{1.341488in}{6.015285in}}%
\pgfpathlineto{\pgfqpoint{1.298176in}{6.106667in}}%
\pgfpathlineto{\pgfqpoint{1.291901in}{6.119925in}}%
\pgfpathlineto{\pgfqpoint{1.247629in}{6.213333in}}%
\pgfpathlineto{\pgfqpoint{1.242314in}{6.224563in}}%
\pgfpathlineto{\pgfqpoint{1.197080in}{6.320000in}}%
\pgfpathlineto{\pgfqpoint{1.192727in}{6.329197in}}%
\pgfpathlineto{\pgfqpoint{1.192727in}{6.320000in}}%
\pgfpathlineto{\pgfqpoint{1.192727in}{6.213333in}}%
\pgfpathlineto{\pgfqpoint{1.192727in}{6.106667in}}%
\pgfpathlineto{\pgfqpoint{1.192727in}{6.000000in}}%
\pgfpathlineto{\pgfqpoint{1.192727in}{5.893333in}}%
\pgfpathlineto{\pgfqpoint{1.192727in}{5.786667in}}%
\pgfpathlineto{\pgfqpoint{1.192727in}{5.680000in}}%
\pgfpathlineto{\pgfqpoint{1.192727in}{5.573333in}}%
\pgfpathlineto{\pgfqpoint{1.192727in}{5.466667in}}%
\pgfpathlineto{\pgfqpoint{1.192727in}{5.360000in}}%
\pgfpathlineto{\pgfqpoint{1.192727in}{5.253333in}}%
\pgfpathlineto{\pgfqpoint{1.192727in}{5.146667in}}%
\pgfpathlineto{\pgfqpoint{1.192727in}{5.040000in}}%
\pgfpathlineto{\pgfqpoint{1.192727in}{4.933333in}}%
\pgfpathlineto{\pgfqpoint{1.192727in}{4.826667in}}%
\pgfpathlineto{\pgfqpoint{1.192727in}{4.720000in}}%
\pgfpathlineto{\pgfqpoint{1.192727in}{4.613333in}}%
\pgfpathlineto{\pgfqpoint{1.192727in}{4.506667in}}%
\pgfpathlineto{\pgfqpoint{1.192727in}{4.400000in}}%
\pgfpathlineto{\pgfqpoint{1.192727in}{4.293333in}}%
\pgfpathlineto{\pgfqpoint{1.192727in}{4.186667in}}%
\pgfpathlineto{\pgfqpoint{1.192727in}{4.080000in}}%
\pgfpathlineto{\pgfqpoint{1.192727in}{3.973333in}}%
\pgfpathlineto{\pgfqpoint{1.192727in}{3.866667in}}%
\pgfpathlineto{\pgfqpoint{1.192727in}{3.760000in}}%
\pgfpathlineto{\pgfqpoint{1.192727in}{3.653333in}}%
\pgfpathlineto{\pgfqpoint{1.192727in}{3.546667in}}%
\pgfpathlineto{\pgfqpoint{1.192727in}{3.440000in}}%
\pgfpathlineto{\pgfqpoint{1.192727in}{3.333333in}}%
\pgfpathlineto{\pgfqpoint{1.192727in}{3.226667in}}%
\pgfpathlineto{\pgfqpoint{1.192727in}{3.120000in}}%
\pgfpathlineto{\pgfqpoint{1.192727in}{3.099915in}}%
\pgfpathlineto{\pgfqpoint{1.233727in}{3.013333in}}%
\pgfpathlineto{\pgfqpoint{1.242314in}{2.995208in}}%
\pgfpathlineto{\pgfqpoint{1.284242in}{2.906667in}}%
\pgfpathlineto{\pgfqpoint{1.291901in}{2.890500in}}%
\pgfpathlineto{\pgfqpoint{1.334756in}{2.800000in}}%
\pgfpathlineto{\pgfqpoint{1.341488in}{2.785791in}}%
\pgfpathlineto{\pgfqpoint{1.385270in}{2.693333in}}%
\pgfpathlineto{\pgfqpoint{1.391074in}{2.681081in}}%
\pgfpathlineto{\pgfqpoint{1.435783in}{2.586667in}}%
\pgfpathlineto{\pgfqpoint{1.440661in}{2.576371in}}%
\pgfpathlineto{\pgfqpoint{1.486297in}{2.480000in}}%
\pgfpathlineto{\pgfqpoint{1.490248in}{2.471659in}}%
\pgfpathlineto{\pgfqpoint{1.536809in}{2.373333in}}%
\pgfpathlineto{\pgfqpoint{1.539835in}{2.366947in}}%
\pgfpathlineto{\pgfqpoint{1.587322in}{2.266667in}}%
\pgfpathlineto{\pgfqpoint{1.589421in}{2.262234in}}%
\pgfpathlineto{\pgfqpoint{1.637834in}{2.160000in}}%
\pgfpathlineto{\pgfqpoint{1.639008in}{2.157521in}}%
\pgfpathlineto{\pgfqpoint{1.688345in}{2.053333in}}%
\pgfpathlineto{\pgfqpoint{1.688595in}{2.052806in}}%
\pgfpathlineto{\pgfqpoint{1.738182in}{1.948123in}}%
\pgfpathlineto{\pgfqpoint{1.738872in}{1.946667in}}%
\pgfpathlineto{\pgfqpoint{1.787769in}{1.843452in}}%
\pgfpathlineto{\pgfqpoint{1.789403in}{1.840000in}}%
\pgfpathlineto{\pgfqpoint{1.837355in}{1.738779in}}%
\pgfpathlineto{\pgfqpoint{1.839934in}{1.733333in}}%
\pgfpathlineto{\pgfqpoint{1.886942in}{1.634106in}}%
\pgfpathlineto{\pgfqpoint{1.890465in}{1.626667in}}%
\pgfpathlineto{\pgfqpoint{1.936529in}{1.529432in}}%
\pgfpathlineto{\pgfqpoint{1.940995in}{1.520000in}}%
\pgfpathlineto{\pgfqpoint{1.986116in}{1.424757in}}%
\pgfpathlineto{\pgfqpoint{1.991525in}{1.413333in}}%
\pgfpathlineto{\pgfqpoint{2.035702in}{1.320081in}}%
\pgfpathlineto{\pgfqpoint{2.042055in}{1.306667in}}%
\pgfpathlineto{\pgfqpoint{2.085289in}{1.215405in}}%
\pgfpathlineto{\pgfqpoint{2.092584in}{1.200000in}}%
\pgfpathlineto{\pgfqpoint{2.134876in}{1.110727in}}%
\pgfpathlineto{\pgfqpoint{2.143113in}{1.093333in}}%
\pgfpathlineto{\pgfqpoint{2.184463in}{1.006049in}}%
\pgfpathlineto{\pgfqpoint{2.193641in}{0.986667in}}%
\pgfpathlineto{\pgfqpoint{2.234050in}{0.901370in}}%
\pgfpathlineto{\pgfqpoint{2.244169in}{0.880000in}}%
\pgfpathlineto{\pgfqpoint{2.283636in}{0.796690in}}%
\pgfpathlineto{\pgfqpoint{2.294696in}{0.773333in}}%
\pgfpathlineto{\pgfqpoint{2.333223in}{0.692009in}}%
\pgfpathlineto{\pgfqpoint{2.345224in}{0.666667in}}%
\pgfpathlineto{\pgfqpoint{2.382810in}{0.587327in}}%
\pgfpathlineto{\pgfqpoint{2.395750in}{0.560000in}}%
\pgfpathlineto{\pgfqpoint{2.432397in}{0.482644in}}%
\pgfpathlineto{\pgfqpoint{2.446277in}{0.453333in}}%
\pgfpathlineto{\pgfqpoint{2.481983in}{0.377961in}}%
\pgfpathlineto{\pgfqpoint{2.496803in}{0.346667in}}%
\pgfpathlineto{\pgfqpoint{2.531570in}{0.273277in}}%
\pgfpathlineto{\pgfqpoint{2.547328in}{0.240000in}}%
\pgfpathlineto{\pgfqpoint{2.581157in}{0.168592in}}%
\pgfpathlineto{\pgfqpoint{2.597853in}{0.133333in}}%
\pgfpathlineto{\pgfqpoint{2.630744in}{0.063906in}}%
\pgfpathlineto{\pgfqpoint{2.648378in}{0.026667in}}%
\pgfpathlineto{\pgfqpoint{2.680331in}{-0.040781in}}%
\pgfpathlineto{\pgfqpoint{2.698902in}{-0.080000in}}%
\pgfpathlineto{\pgfqpoint{2.729917in}{-0.145468in}}%
\pgfpathlineto{\pgfqpoint{2.749426in}{-0.186667in}}%
\pgfpathlineto{\pgfqpoint{2.779504in}{-0.250157in}}%
\pgfpathlineto{\pgfqpoint{2.799950in}{-0.293333in}}%
\pgfpathlineto{\pgfqpoint{2.829091in}{-0.354846in}}%
\pgfpathlineto{\pgfqpoint{2.850473in}{-0.400000in}}%
\pgfpathlineto{\pgfqpoint{2.878678in}{-0.459536in}}%
\pgfpathlineto{\pgfqpoint{2.900996in}{-0.506667in}}%
\pgfpathlineto{\pgfqpoint{2.928264in}{-0.564227in}}%
\pgfpathlineto{\pgfqpoint{2.951518in}{-0.613333in}}%
\pgfpathlineto{\pgfqpoint{2.977851in}{-0.668919in}}%
\pgfpathlineto{\pgfqpoint{3.002040in}{-0.720000in}}%
\pgfpathlineto{\pgfqpoint{3.027438in}{-0.773611in}}%
\pgfpathlineto{\pgfqpoint{3.052562in}{-0.826667in}}%
\pgfpathlineto{\pgfqpoint{3.077025in}{-0.878305in}}%
\pgfpathlineto{\pgfqpoint{3.103083in}{-0.933333in}}%
\pgfpathlineto{\pgfqpoint{3.126612in}{-0.982999in}}%
\pgfpathlineto{\pgfqpoint{3.153604in}{-1.040000in}}%
\pgfpathlineto{\pgfqpoint{3.176198in}{-1.087694in}}%
\pgfpathlineto{\pgfqpoint{3.204124in}{-1.146667in}}%
\pgfpathlineto{\pgfqpoint{3.225785in}{-1.192390in}}%
\pgfpathlineto{\pgfqpoint{3.254644in}{-1.253333in}}%
\pgfpathlineto{\pgfqpoint{3.275372in}{-1.297086in}}%
\pgfpathlineto{\pgfqpoint{3.305164in}{-1.360000in}}%
\pgfpathlineto{\pgfqpoint{3.324959in}{-1.401784in}}%
\pgfpathlineto{\pgfqpoint{3.355683in}{-1.466667in}}%
\pgfpathlineto{\pgfqpoint{3.374545in}{-1.506482in}}%
\pgfpathlineto{\pgfqpoint{3.406202in}{-1.573333in}}%
\pgfpathlineto{\pgfqpoint{3.424132in}{-1.611181in}}%
\pgfpathlineto{\pgfqpoint{3.456721in}{-1.680000in}}%
\pgfpathlineto{\pgfqpoint{3.473719in}{-1.715881in}}%
\pgfpathlineto{\pgfqpoint{3.507239in}{-1.786667in}}%
\pgfpathlineto{\pgfqpoint{3.523306in}{-1.820582in}}%
\pgfpathlineto{\pgfqpoint{3.557757in}{-1.893333in}}%
\pgfpathlineto{\pgfqpoint{3.572893in}{-1.925283in}}%
\pgfpathlineto{\pgfqpoint{3.608274in}{-2.000000in}}%
\pgfpathlineto{\pgfqpoint{3.622479in}{-2.029986in}}%
\pgfpathlineto{\pgfqpoint{3.658791in}{-2.106667in}}%
\pgfpathlineto{\pgfqpoint{3.672066in}{-2.134689in}}%
\pgfpathlineto{\pgfqpoint{3.709308in}{-2.213333in}}%
\pgfpathlineto{\pgfqpoint{3.721653in}{-2.239393in}}%
\pgfpathlineto{\pgfqpoint{3.759824in}{-2.320000in}}%
\pgfpathlineto{\pgfqpoint{3.771240in}{-2.344098in}}%
\pgfpathlineto{\pgfqpoint{3.810340in}{-2.426667in}}%
\pgfpathlineto{\pgfqpoint{3.820826in}{-2.448803in}}%
\pgfpathlineto{\pgfqpoint{3.860855in}{-2.533333in}}%
\pgfpathclose%
\pgfusepath{fill}%
\end{pgfscope}%
\begin{pgfscope}%
\pgfpathrectangle{\pgfqpoint{3.156364in}{0.528000in}}{\pgfqpoint{1.963636in}{3.696000in}} %
\pgfusepath{clip}%
\pgfsetbuttcap%
\pgfsetroundjoin%
\definecolor{currentfill}{rgb}{0.944844,0.507658,0.302433}%
\pgfsetfillcolor{currentfill}%
\pgfsetlinewidth{0.000000pt}%
\definecolor{currentstroke}{rgb}{0.000000,0.000000,0.000000}%
\pgfsetstrokecolor{currentstroke}%
\pgfsetdash{}{0pt}%
\pgfpathmoveto{\pgfqpoint{6.101818in}{-1.049197in}}%
\pgfpathlineto{\pgfqpoint{6.101818in}{-1.040000in}}%
\pgfpathlineto{\pgfqpoint{6.101818in}{-0.933333in}}%
\pgfpathlineto{\pgfqpoint{6.101818in}{-0.826667in}}%
\pgfpathlineto{\pgfqpoint{6.101818in}{-0.720000in}}%
\pgfpathlineto{\pgfqpoint{6.101818in}{-0.613333in}}%
\pgfpathlineto{\pgfqpoint{6.101818in}{-0.506667in}}%
\pgfpathlineto{\pgfqpoint{6.101818in}{-0.400000in}}%
\pgfpathlineto{\pgfqpoint{6.101818in}{-0.293333in}}%
\pgfpathlineto{\pgfqpoint{6.101818in}{-0.186667in}}%
\pgfpathlineto{\pgfqpoint{6.101818in}{-0.080000in}}%
\pgfpathlineto{\pgfqpoint{6.101818in}{0.026667in}}%
\pgfpathlineto{\pgfqpoint{6.101818in}{0.133333in}}%
\pgfpathlineto{\pgfqpoint{6.101818in}{0.240000in}}%
\pgfpathlineto{\pgfqpoint{6.101818in}{0.346667in}}%
\pgfpathlineto{\pgfqpoint{6.101818in}{0.453333in}}%
\pgfpathlineto{\pgfqpoint{6.101818in}{0.560000in}}%
\pgfpathlineto{\pgfqpoint{6.101818in}{0.666667in}}%
\pgfpathlineto{\pgfqpoint{6.101818in}{0.773333in}}%
\pgfpathlineto{\pgfqpoint{6.101818in}{0.880000in}}%
\pgfpathlineto{\pgfqpoint{6.101818in}{0.986667in}}%
\pgfpathlineto{\pgfqpoint{6.101818in}{1.093333in}}%
\pgfpathlineto{\pgfqpoint{6.101818in}{1.200000in}}%
\pgfpathlineto{\pgfqpoint{6.101818in}{1.306667in}}%
\pgfpathlineto{\pgfqpoint{6.101818in}{1.413333in}}%
\pgfpathlineto{\pgfqpoint{6.101818in}{1.520000in}}%
\pgfpathlineto{\pgfqpoint{6.101818in}{1.626667in}}%
\pgfpathlineto{\pgfqpoint{6.101818in}{1.733333in}}%
\pgfpathlineto{\pgfqpoint{6.101818in}{1.840000in}}%
\pgfpathlineto{\pgfqpoint{6.101818in}{1.946667in}}%
\pgfpathlineto{\pgfqpoint{6.101818in}{2.053333in}}%
\pgfpathlineto{\pgfqpoint{6.101818in}{2.160000in}}%
\pgfpathlineto{\pgfqpoint{6.101818in}{2.180085in}}%
\pgfpathlineto{\pgfqpoint{6.060818in}{2.266667in}}%
\pgfpathlineto{\pgfqpoint{6.052231in}{2.284792in}}%
\pgfpathlineto{\pgfqpoint{6.010304in}{2.373333in}}%
\pgfpathlineto{\pgfqpoint{6.002645in}{2.389500in}}%
\pgfpathlineto{\pgfqpoint{5.959789in}{2.480000in}}%
\pgfpathlineto{\pgfqpoint{5.953058in}{2.494209in}}%
\pgfpathlineto{\pgfqpoint{5.909275in}{2.586667in}}%
\pgfpathlineto{\pgfqpoint{5.903471in}{2.598919in}}%
\pgfpathlineto{\pgfqpoint{5.858762in}{2.693333in}}%
\pgfpathlineto{\pgfqpoint{5.853884in}{2.703629in}}%
\pgfpathlineto{\pgfqpoint{5.808249in}{2.800000in}}%
\pgfpathlineto{\pgfqpoint{5.804298in}{2.808341in}}%
\pgfpathlineto{\pgfqpoint{5.757736in}{2.906667in}}%
\pgfpathlineto{\pgfqpoint{5.754711in}{2.913053in}}%
\pgfpathlineto{\pgfqpoint{5.707224in}{3.013333in}}%
\pgfpathlineto{\pgfqpoint{5.705124in}{3.017766in}}%
\pgfpathlineto{\pgfqpoint{5.656712in}{3.120000in}}%
\pgfpathlineto{\pgfqpoint{5.655537in}{3.122479in}}%
\pgfpathlineto{\pgfqpoint{5.606200in}{3.226667in}}%
\pgfpathlineto{\pgfqpoint{5.605950in}{3.227194in}}%
\pgfpathlineto{\pgfqpoint{5.556364in}{3.331877in}}%
\pgfpathlineto{\pgfqpoint{5.555674in}{3.333333in}}%
\pgfpathlineto{\pgfqpoint{5.506777in}{3.436548in}}%
\pgfpathlineto{\pgfqpoint{5.505142in}{3.440000in}}%
\pgfpathlineto{\pgfqpoint{5.457190in}{3.541221in}}%
\pgfpathlineto{\pgfqpoint{5.454611in}{3.546667in}}%
\pgfpathlineto{\pgfqpoint{5.407603in}{3.645894in}}%
\pgfpathlineto{\pgfqpoint{5.404080in}{3.653333in}}%
\pgfpathlineto{\pgfqpoint{5.358017in}{3.750568in}}%
\pgfpathlineto{\pgfqpoint{5.353550in}{3.760000in}}%
\pgfpathlineto{\pgfqpoint{5.308430in}{3.855243in}}%
\pgfpathlineto{\pgfqpoint{5.303020in}{3.866667in}}%
\pgfpathlineto{\pgfqpoint{5.258843in}{3.959919in}}%
\pgfpathlineto{\pgfqpoint{5.252491in}{3.973333in}}%
\pgfpathlineto{\pgfqpoint{5.209256in}{4.064595in}}%
\pgfpathlineto{\pgfqpoint{5.201962in}{4.080000in}}%
\pgfpathlineto{\pgfqpoint{5.159669in}{4.169273in}}%
\pgfpathlineto{\pgfqpoint{5.151433in}{4.186667in}}%
\pgfpathlineto{\pgfqpoint{5.110083in}{4.273951in}}%
\pgfpathlineto{\pgfqpoint{5.100904in}{4.293333in}}%
\pgfpathlineto{\pgfqpoint{5.060496in}{4.378630in}}%
\pgfpathlineto{\pgfqpoint{5.050377in}{4.400000in}}%
\pgfpathlineto{\pgfqpoint{5.010909in}{4.483310in}}%
\pgfpathlineto{\pgfqpoint{4.999849in}{4.506667in}}%
\pgfpathlineto{\pgfqpoint{4.961322in}{4.587991in}}%
\pgfpathlineto{\pgfqpoint{4.949322in}{4.613333in}}%
\pgfpathlineto{\pgfqpoint{4.911736in}{4.692673in}}%
\pgfpathlineto{\pgfqpoint{4.898795in}{4.720000in}}%
\pgfpathlineto{\pgfqpoint{4.862149in}{4.797356in}}%
\pgfpathlineto{\pgfqpoint{4.848269in}{4.826667in}}%
\pgfpathlineto{\pgfqpoint{4.812562in}{4.902039in}}%
\pgfpathlineto{\pgfqpoint{4.797743in}{4.933333in}}%
\pgfpathlineto{\pgfqpoint{4.762975in}{5.006723in}}%
\pgfpathlineto{\pgfqpoint{4.747217in}{5.040000in}}%
\pgfpathlineto{\pgfqpoint{4.713388in}{5.111408in}}%
\pgfpathlineto{\pgfqpoint{4.696692in}{5.146667in}}%
\pgfpathlineto{\pgfqpoint{4.663802in}{5.216094in}}%
\pgfpathlineto{\pgfqpoint{4.646167in}{5.253333in}}%
\pgfpathlineto{\pgfqpoint{4.614215in}{5.320781in}}%
\pgfpathlineto{\pgfqpoint{4.595643in}{5.360000in}}%
\pgfpathlineto{\pgfqpoint{4.564628in}{5.425468in}}%
\pgfpathlineto{\pgfqpoint{4.545119in}{5.466667in}}%
\pgfpathlineto{\pgfqpoint{4.515041in}{5.530157in}}%
\pgfpathlineto{\pgfqpoint{4.494595in}{5.573333in}}%
\pgfpathlineto{\pgfqpoint{4.465455in}{5.634846in}}%
\pgfpathlineto{\pgfqpoint{4.444072in}{5.680000in}}%
\pgfpathlineto{\pgfqpoint{4.415868in}{5.739536in}}%
\pgfpathlineto{\pgfqpoint{4.393550in}{5.786667in}}%
\pgfpathlineto{\pgfqpoint{4.366281in}{5.844227in}}%
\pgfpathlineto{\pgfqpoint{4.343027in}{5.893333in}}%
\pgfpathlineto{\pgfqpoint{4.316694in}{5.948919in}}%
\pgfpathlineto{\pgfqpoint{4.292505in}{6.000000in}}%
\pgfpathlineto{\pgfqpoint{4.267107in}{6.053611in}}%
\pgfpathlineto{\pgfqpoint{4.241984in}{6.106667in}}%
\pgfpathlineto{\pgfqpoint{4.217521in}{6.158305in}}%
\pgfpathlineto{\pgfqpoint{4.191462in}{6.213333in}}%
\pgfpathlineto{\pgfqpoint{4.167934in}{6.262999in}}%
\pgfpathlineto{\pgfqpoint{4.140941in}{6.320000in}}%
\pgfpathlineto{\pgfqpoint{4.118347in}{6.367694in}}%
\pgfpathlineto{\pgfqpoint{4.090421in}{6.426667in}}%
\pgfpathlineto{\pgfqpoint{4.068760in}{6.472390in}}%
\pgfpathlineto{\pgfqpoint{4.039901in}{6.533333in}}%
\pgfpathlineto{\pgfqpoint{4.019174in}{6.577086in}}%
\pgfpathlineto{\pgfqpoint{3.989381in}{6.640000in}}%
\pgfpathlineto{\pgfqpoint{3.969587in}{6.681784in}}%
\pgfpathlineto{\pgfqpoint{3.938862in}{6.746667in}}%
\pgfpathlineto{\pgfqpoint{3.920000in}{6.786482in}}%
\pgfpathlineto{\pgfqpoint{3.888343in}{6.853333in}}%
\pgfpathlineto{\pgfqpoint{3.870413in}{6.891181in}}%
\pgfpathlineto{\pgfqpoint{3.837825in}{6.960000in}}%
\pgfpathlineto{\pgfqpoint{3.820826in}{6.995881in}}%
\pgfpathlineto{\pgfqpoint{3.787307in}{7.066667in}}%
\pgfpathlineto{\pgfqpoint{3.771240in}{7.100582in}}%
\pgfpathlineto{\pgfqpoint{3.736789in}{7.173333in}}%
\pgfpathlineto{\pgfqpoint{3.721653in}{7.205283in}}%
\pgfpathlineto{\pgfqpoint{3.686271in}{7.280000in}}%
\pgfpathlineto{\pgfqpoint{3.672066in}{7.309986in}}%
\pgfpathlineto{\pgfqpoint{3.635754in}{7.386667in}}%
\pgfpathlineto{\pgfqpoint{3.622479in}{7.414689in}}%
\pgfpathlineto{\pgfqpoint{3.585238in}{7.493333in}}%
\pgfpathlineto{\pgfqpoint{3.572893in}{7.519393in}}%
\pgfpathlineto{\pgfqpoint{3.534722in}{7.600000in}}%
\pgfpathlineto{\pgfqpoint{3.523306in}{7.624098in}}%
\pgfpathlineto{\pgfqpoint{3.484206in}{7.706667in}}%
\pgfpathlineto{\pgfqpoint{3.473719in}{7.728803in}}%
\pgfpathlineto{\pgfqpoint{3.433690in}{7.813333in}}%
\pgfpathlineto{\pgfqpoint{3.424132in}{7.833510in}}%
\pgfpathlineto{\pgfqpoint{3.383175in}{7.920000in}}%
\pgfpathlineto{\pgfqpoint{3.374545in}{7.920000in}}%
\pgfpathlineto{\pgfqpoint{3.324959in}{7.920000in}}%
\pgfpathlineto{\pgfqpoint{3.275372in}{7.920000in}}%
\pgfpathlineto{\pgfqpoint{3.225785in}{7.920000in}}%
\pgfpathlineto{\pgfqpoint{3.176198in}{7.920000in}}%
\pgfpathlineto{\pgfqpoint{3.126612in}{7.920000in}}%
\pgfpathlineto{\pgfqpoint{3.077025in}{7.920000in}}%
\pgfpathlineto{\pgfqpoint{3.027438in}{7.920000in}}%
\pgfpathlineto{\pgfqpoint{2.977851in}{7.920000in}}%
\pgfpathlineto{\pgfqpoint{2.928264in}{7.920000in}}%
\pgfpathlineto{\pgfqpoint{2.878678in}{7.920000in}}%
\pgfpathlineto{\pgfqpoint{2.829091in}{7.920000in}}%
\pgfpathlineto{\pgfqpoint{2.779504in}{7.920000in}}%
\pgfpathlineto{\pgfqpoint{2.729917in}{7.920000in}}%
\pgfpathlineto{\pgfqpoint{2.680331in}{7.920000in}}%
\pgfpathlineto{\pgfqpoint{2.630744in}{7.920000in}}%
\pgfpathlineto{\pgfqpoint{2.581157in}{7.920000in}}%
\pgfpathlineto{\pgfqpoint{2.531570in}{7.920000in}}%
\pgfpathlineto{\pgfqpoint{2.481983in}{7.920000in}}%
\pgfpathlineto{\pgfqpoint{2.432397in}{7.920000in}}%
\pgfpathlineto{\pgfqpoint{2.382810in}{7.920000in}}%
\pgfpathlineto{\pgfqpoint{2.333223in}{7.920000in}}%
\pgfpathlineto{\pgfqpoint{2.283636in}{7.920000in}}%
\pgfpathlineto{\pgfqpoint{2.234050in}{7.920000in}}%
\pgfpathlineto{\pgfqpoint{2.184463in}{7.920000in}}%
\pgfpathlineto{\pgfqpoint{2.134876in}{7.920000in}}%
\pgfpathlineto{\pgfqpoint{2.085289in}{7.920000in}}%
\pgfpathlineto{\pgfqpoint{2.035702in}{7.920000in}}%
\pgfpathlineto{\pgfqpoint{1.986116in}{7.920000in}}%
\pgfpathlineto{\pgfqpoint{1.936529in}{7.920000in}}%
\pgfpathlineto{\pgfqpoint{1.886942in}{7.920000in}}%
\pgfpathlineto{\pgfqpoint{1.853413in}{7.920000in}}%
\pgfpathlineto{\pgfqpoint{1.886942in}{7.849163in}}%
\pgfpathlineto{\pgfqpoint{1.903924in}{7.813333in}}%
\pgfpathlineto{\pgfqpoint{1.936529in}{7.744448in}}%
\pgfpathlineto{\pgfqpoint{1.954436in}{7.706667in}}%
\pgfpathlineto{\pgfqpoint{1.986116in}{7.639736in}}%
\pgfpathlineto{\pgfqpoint{2.004949in}{7.600000in}}%
\pgfpathlineto{\pgfqpoint{2.035702in}{7.535027in}}%
\pgfpathlineto{\pgfqpoint{2.055463in}{7.493333in}}%
\pgfpathlineto{\pgfqpoint{2.085289in}{7.430320in}}%
\pgfpathlineto{\pgfqpoint{2.105979in}{7.386667in}}%
\pgfpathlineto{\pgfqpoint{2.134876in}{7.325615in}}%
\pgfpathlineto{\pgfqpoint{2.156495in}{7.280000in}}%
\pgfpathlineto{\pgfqpoint{2.184463in}{7.220913in}}%
\pgfpathlineto{\pgfqpoint{2.207013in}{7.173333in}}%
\pgfpathlineto{\pgfqpoint{2.234050in}{7.116214in}}%
\pgfpathlineto{\pgfqpoint{2.257533in}{7.066667in}}%
\pgfpathlineto{\pgfqpoint{2.283636in}{7.011517in}}%
\pgfpathlineto{\pgfqpoint{2.308053in}{6.960000in}}%
\pgfpathlineto{\pgfqpoint{2.333223in}{6.906823in}}%
\pgfpathlineto{\pgfqpoint{2.358575in}{6.853333in}}%
\pgfpathlineto{\pgfqpoint{2.382810in}{6.802131in}}%
\pgfpathlineto{\pgfqpoint{2.409098in}{6.746667in}}%
\pgfpathlineto{\pgfqpoint{2.432397in}{6.697442in}}%
\pgfpathlineto{\pgfqpoint{2.459622in}{6.640000in}}%
\pgfpathlineto{\pgfqpoint{2.481983in}{6.592756in}}%
\pgfpathlineto{\pgfqpoint{2.510147in}{6.533333in}}%
\pgfpathlineto{\pgfqpoint{2.531570in}{6.488073in}}%
\pgfpathlineto{\pgfqpoint{2.560674in}{6.426667in}}%
\pgfpathlineto{\pgfqpoint{2.581157in}{6.383392in}}%
\pgfpathlineto{\pgfqpoint{2.611202in}{6.320000in}}%
\pgfpathlineto{\pgfqpoint{2.630744in}{6.278713in}}%
\pgfpathlineto{\pgfqpoint{2.661731in}{6.213333in}}%
\pgfpathlineto{\pgfqpoint{2.680331in}{6.174038in}}%
\pgfpathlineto{\pgfqpoint{2.712261in}{6.106667in}}%
\pgfpathlineto{\pgfqpoint{2.729917in}{6.069365in}}%
\pgfpathlineto{\pgfqpoint{2.762793in}{6.000000in}}%
\pgfpathlineto{\pgfqpoint{2.779504in}{5.964694in}}%
\pgfpathlineto{\pgfqpoint{2.813326in}{5.893333in}}%
\pgfpathlineto{\pgfqpoint{2.829091in}{5.860027in}}%
\pgfpathlineto{\pgfqpoint{2.863860in}{5.786667in}}%
\pgfpathlineto{\pgfqpoint{2.878678in}{5.755362in}}%
\pgfpathlineto{\pgfqpoint{2.914396in}{5.680000in}}%
\pgfpathlineto{\pgfqpoint{2.928264in}{5.650700in}}%
\pgfpathlineto{\pgfqpoint{2.964933in}{5.573333in}}%
\pgfpathlineto{\pgfqpoint{2.977851in}{5.546040in}}%
\pgfpathlineto{\pgfqpoint{3.015471in}{5.466667in}}%
\pgfpathlineto{\pgfqpoint{3.027438in}{5.441384in}}%
\pgfpathlineto{\pgfqpoint{3.066011in}{5.360000in}}%
\pgfpathlineto{\pgfqpoint{3.077025in}{5.336730in}}%
\pgfpathlineto{\pgfqpoint{3.116551in}{5.253333in}}%
\pgfpathlineto{\pgfqpoint{3.126612in}{5.232079in}}%
\pgfpathlineto{\pgfqpoint{3.167093in}{5.146667in}}%
\pgfpathlineto{\pgfqpoint{3.176198in}{5.127430in}}%
\pgfpathlineto{\pgfqpoint{3.217637in}{5.040000in}}%
\pgfpathlineto{\pgfqpoint{3.225785in}{5.022785in}}%
\pgfpathlineto{\pgfqpoint{3.268182in}{4.933333in}}%
\pgfpathlineto{\pgfqpoint{3.275372in}{4.918142in}}%
\pgfpathlineto{\pgfqpoint{3.318728in}{4.826667in}}%
\pgfpathlineto{\pgfqpoint{3.324959in}{4.813502in}}%
\pgfpathlineto{\pgfqpoint{3.369275in}{4.720000in}}%
\pgfpathlineto{\pgfqpoint{3.374545in}{4.708865in}}%
\pgfpathlineto{\pgfqpoint{3.419824in}{4.613333in}}%
\pgfpathlineto{\pgfqpoint{3.424132in}{4.604231in}}%
\pgfpathlineto{\pgfqpoint{3.470374in}{4.506667in}}%
\pgfpathlineto{\pgfqpoint{3.473719in}{4.499600in}}%
\pgfpathlineto{\pgfqpoint{3.520925in}{4.400000in}}%
\pgfpathlineto{\pgfqpoint{3.523306in}{4.394971in}}%
\pgfpathlineto{\pgfqpoint{3.571478in}{4.293333in}}%
\pgfpathlineto{\pgfqpoint{3.572893in}{4.290345in}}%
\pgfpathlineto{\pgfqpoint{3.622033in}{4.186667in}}%
\pgfpathlineto{\pgfqpoint{3.622479in}{4.185723in}}%
\pgfpathlineto{\pgfqpoint{3.672066in}{4.081027in}}%
\pgfpathlineto{\pgfqpoint{3.672553in}{4.080000in}}%
\pgfpathlineto{\pgfqpoint{3.721653in}{3.976268in}}%
\pgfpathlineto{\pgfqpoint{3.723044in}{3.973333in}}%
\pgfpathlineto{\pgfqpoint{3.771240in}{3.871512in}}%
\pgfpathlineto{\pgfqpoint{3.773536in}{3.866667in}}%
\pgfpathlineto{\pgfqpoint{3.820826in}{3.766758in}}%
\pgfpathlineto{\pgfqpoint{3.824029in}{3.760000in}}%
\pgfpathlineto{\pgfqpoint{3.870413in}{3.662006in}}%
\pgfpathlineto{\pgfqpoint{3.874524in}{3.653333in}}%
\pgfpathlineto{\pgfqpoint{3.920000in}{3.557257in}}%
\pgfpathlineto{\pgfqpoint{3.925019in}{3.546667in}}%
\pgfpathlineto{\pgfqpoint{3.969587in}{3.452510in}}%
\pgfpathlineto{\pgfqpoint{3.975516in}{3.440000in}}%
\pgfpathlineto{\pgfqpoint{4.019174in}{3.347766in}}%
\pgfpathlineto{\pgfqpoint{4.026014in}{3.333333in}}%
\pgfpathlineto{\pgfqpoint{4.068760in}{3.243024in}}%
\pgfpathlineto{\pgfqpoint{4.076513in}{3.226667in}}%
\pgfpathlineto{\pgfqpoint{4.118347in}{3.138284in}}%
\pgfpathlineto{\pgfqpoint{4.127013in}{3.120000in}}%
\pgfpathlineto{\pgfqpoint{4.167934in}{3.033547in}}%
\pgfpathlineto{\pgfqpoint{4.177514in}{3.013333in}}%
\pgfpathlineto{\pgfqpoint{4.217521in}{2.928813in}}%
\pgfpathlineto{\pgfqpoint{4.228017in}{2.906667in}}%
\pgfpathlineto{\pgfqpoint{4.267107in}{2.824080in}}%
\pgfpathlineto{\pgfqpoint{4.278520in}{2.800000in}}%
\pgfpathlineto{\pgfqpoint{4.316694in}{2.719351in}}%
\pgfpathlineto{\pgfqpoint{4.329025in}{2.693333in}}%
\pgfpathlineto{\pgfqpoint{4.366281in}{2.614623in}}%
\pgfpathlineto{\pgfqpoint{4.379531in}{2.586667in}}%
\pgfpathlineto{\pgfqpoint{4.415868in}{2.509898in}}%
\pgfpathlineto{\pgfqpoint{4.430038in}{2.480000in}}%
\pgfpathlineto{\pgfqpoint{4.465455in}{2.405176in}}%
\pgfpathlineto{\pgfqpoint{4.480546in}{2.373333in}}%
\pgfpathlineto{\pgfqpoint{4.515041in}{2.300456in}}%
\pgfpathlineto{\pgfqpoint{4.531056in}{2.266667in}}%
\pgfpathlineto{\pgfqpoint{4.564628in}{2.195739in}}%
\pgfpathlineto{\pgfqpoint{4.581567in}{2.160000in}}%
\pgfpathlineto{\pgfqpoint{4.614215in}{2.091024in}}%
\pgfpathlineto{\pgfqpoint{4.632078in}{2.053333in}}%
\pgfpathlineto{\pgfqpoint{4.663802in}{1.986312in}}%
\pgfpathlineto{\pgfqpoint{4.682592in}{1.946667in}}%
\pgfpathlineto{\pgfqpoint{4.713388in}{1.881602in}}%
\pgfpathlineto{\pgfqpoint{4.733106in}{1.840000in}}%
\pgfpathlineto{\pgfqpoint{4.762975in}{1.776895in}}%
\pgfpathlineto{\pgfqpoint{4.783621in}{1.733333in}}%
\pgfpathlineto{\pgfqpoint{4.812562in}{1.672190in}}%
\pgfpathlineto{\pgfqpoint{4.834138in}{1.626667in}}%
\pgfpathlineto{\pgfqpoint{4.862149in}{1.567488in}}%
\pgfpathlineto{\pgfqpoint{4.884656in}{1.520000in}}%
\pgfpathlineto{\pgfqpoint{4.911736in}{1.462789in}}%
\pgfpathlineto{\pgfqpoint{4.935175in}{1.413333in}}%
\pgfpathlineto{\pgfqpoint{4.961322in}{1.358092in}}%
\pgfpathlineto{\pgfqpoint{4.985696in}{1.306667in}}%
\pgfpathlineto{\pgfqpoint{5.010909in}{1.253398in}}%
\pgfpathlineto{\pgfqpoint{5.036217in}{1.200000in}}%
\pgfpathlineto{\pgfqpoint{5.060496in}{1.148706in}}%
\pgfpathlineto{\pgfqpoint{5.086740in}{1.093333in}}%
\pgfpathlineto{\pgfqpoint{5.110083in}{1.044017in}}%
\pgfpathlineto{\pgfqpoint{5.137264in}{0.986667in}}%
\pgfpathlineto{\pgfqpoint{5.159669in}{0.939331in}}%
\pgfpathlineto{\pgfqpoint{5.187789in}{0.880000in}}%
\pgfpathlineto{\pgfqpoint{5.209256in}{0.834647in}}%
\pgfpathlineto{\pgfqpoint{5.238316in}{0.773333in}}%
\pgfpathlineto{\pgfqpoint{5.258843in}{0.729966in}}%
\pgfpathlineto{\pgfqpoint{5.288844in}{0.666667in}}%
\pgfpathlineto{\pgfqpoint{5.308430in}{0.625287in}}%
\pgfpathlineto{\pgfqpoint{5.339373in}{0.560000in}}%
\pgfpathlineto{\pgfqpoint{5.358017in}{0.520612in}}%
\pgfpathlineto{\pgfqpoint{5.389903in}{0.453333in}}%
\pgfpathlineto{\pgfqpoint{5.407603in}{0.415938in}}%
\pgfpathlineto{\pgfqpoint{5.440435in}{0.346667in}}%
\pgfpathlineto{\pgfqpoint{5.457190in}{0.311268in}}%
\pgfpathlineto{\pgfqpoint{5.490968in}{0.240000in}}%
\pgfpathlineto{\pgfqpoint{5.506777in}{0.206600in}}%
\pgfpathlineto{\pgfqpoint{5.541502in}{0.133333in}}%
\pgfpathlineto{\pgfqpoint{5.556364in}{0.101935in}}%
\pgfpathlineto{\pgfqpoint{5.592038in}{0.026667in}}%
\pgfpathlineto{\pgfqpoint{5.605950in}{-0.002727in}}%
\pgfpathlineto{\pgfqpoint{5.642575in}{-0.080000in}}%
\pgfpathlineto{\pgfqpoint{5.655537in}{-0.107386in}}%
\pgfpathlineto{\pgfqpoint{5.693113in}{-0.186667in}}%
\pgfpathlineto{\pgfqpoint{5.705124in}{-0.212043in}}%
\pgfpathlineto{\pgfqpoint{5.743652in}{-0.293333in}}%
\pgfpathlineto{\pgfqpoint{5.754711in}{-0.316697in}}%
\pgfpathlineto{\pgfqpoint{5.794193in}{-0.400000in}}%
\pgfpathlineto{\pgfqpoint{5.804298in}{-0.421348in}}%
\pgfpathlineto{\pgfqpoint{5.844735in}{-0.506667in}}%
\pgfpathlineto{\pgfqpoint{5.853884in}{-0.525997in}}%
\pgfpathlineto{\pgfqpoint{5.895278in}{-0.613333in}}%
\pgfpathlineto{\pgfqpoint{5.903471in}{-0.630643in}}%
\pgfpathlineto{\pgfqpoint{5.945823in}{-0.720000in}}%
\pgfpathlineto{\pgfqpoint{5.953058in}{-0.735285in}}%
\pgfpathlineto{\pgfqpoint{5.996369in}{-0.826667in}}%
\pgfpathlineto{\pgfqpoint{6.002645in}{-0.839925in}}%
\pgfpathlineto{\pgfqpoint{6.046916in}{-0.933333in}}%
\pgfpathlineto{\pgfqpoint{6.052231in}{-0.944563in}}%
\pgfpathlineto{\pgfqpoint{6.097465in}{-1.040000in}}%
\pgfpathclose%
\pgfusepath{fill}%
\end{pgfscope}%
\begin{pgfscope}%
\pgfpathrectangle{\pgfqpoint{3.156364in}{0.528000in}}{\pgfqpoint{1.963636in}{3.696000in}} %
\pgfusepath{clip}%
\pgfsetbuttcap%
\pgfsetroundjoin%
\definecolor{currentfill}{rgb}{0.976265,0.868016,0.143351}%
\pgfsetfillcolor{currentfill}%
\pgfsetlinewidth{0.000000pt}%
\definecolor{currentstroke}{rgb}{0.000000,0.000000,0.000000}%
\pgfsetstrokecolor{currentstroke}%
\pgfsetdash{}{0pt}%
\pgfpathmoveto{\pgfqpoint{1.242314in}{-2.640000in}}%
\pgfpathlineto{\pgfqpoint{1.291901in}{-2.640000in}}%
\pgfpathlineto{\pgfqpoint{1.341488in}{-2.640000in}}%
\pgfpathlineto{\pgfqpoint{1.391074in}{-2.640000in}}%
\pgfpathlineto{\pgfqpoint{1.440661in}{-2.640000in}}%
\pgfpathlineto{\pgfqpoint{1.490248in}{-2.640000in}}%
\pgfpathlineto{\pgfqpoint{1.539835in}{-2.640000in}}%
\pgfpathlineto{\pgfqpoint{1.589421in}{-2.640000in}}%
\pgfpathlineto{\pgfqpoint{1.639008in}{-2.640000in}}%
\pgfpathlineto{\pgfqpoint{1.688595in}{-2.640000in}}%
\pgfpathlineto{\pgfqpoint{1.738182in}{-2.640000in}}%
\pgfpathlineto{\pgfqpoint{1.787769in}{-2.640000in}}%
\pgfpathlineto{\pgfqpoint{1.837355in}{-2.640000in}}%
\pgfpathlineto{\pgfqpoint{1.886942in}{-2.640000in}}%
\pgfpathlineto{\pgfqpoint{1.936529in}{-2.640000in}}%
\pgfpathlineto{\pgfqpoint{1.986116in}{-2.640000in}}%
\pgfpathlineto{\pgfqpoint{2.035702in}{-2.640000in}}%
\pgfpathlineto{\pgfqpoint{2.085289in}{-2.640000in}}%
\pgfpathlineto{\pgfqpoint{2.134876in}{-2.640000in}}%
\pgfpathlineto{\pgfqpoint{2.184463in}{-2.640000in}}%
\pgfpathlineto{\pgfqpoint{2.234050in}{-2.640000in}}%
\pgfpathlineto{\pgfqpoint{2.283636in}{-2.640000in}}%
\pgfpathlineto{\pgfqpoint{2.333223in}{-2.640000in}}%
\pgfpathlineto{\pgfqpoint{2.382810in}{-2.640000in}}%
\pgfpathlineto{\pgfqpoint{2.432397in}{-2.640000in}}%
\pgfpathlineto{\pgfqpoint{2.481983in}{-2.640000in}}%
\pgfpathlineto{\pgfqpoint{2.531570in}{-2.640000in}}%
\pgfpathlineto{\pgfqpoint{2.581157in}{-2.640000in}}%
\pgfpathlineto{\pgfqpoint{2.630744in}{-2.640000in}}%
\pgfpathlineto{\pgfqpoint{2.680331in}{-2.640000in}}%
\pgfpathlineto{\pgfqpoint{2.729917in}{-2.640000in}}%
\pgfpathlineto{\pgfqpoint{2.779504in}{-2.640000in}}%
\pgfpathlineto{\pgfqpoint{2.829091in}{-2.640000in}}%
\pgfpathlineto{\pgfqpoint{2.878678in}{-2.640000in}}%
\pgfpathlineto{\pgfqpoint{2.928264in}{-2.640000in}}%
\pgfpathlineto{\pgfqpoint{2.977851in}{-2.640000in}}%
\pgfpathlineto{\pgfqpoint{3.027438in}{-2.640000in}}%
\pgfpathlineto{\pgfqpoint{3.077025in}{-2.640000in}}%
\pgfpathlineto{\pgfqpoint{3.126612in}{-2.640000in}}%
\pgfpathlineto{\pgfqpoint{3.176198in}{-2.640000in}}%
\pgfpathlineto{\pgfqpoint{3.225785in}{-2.640000in}}%
\pgfpathlineto{\pgfqpoint{3.275372in}{-2.640000in}}%
\pgfpathlineto{\pgfqpoint{3.324959in}{-2.640000in}}%
\pgfpathlineto{\pgfqpoint{3.374545in}{-2.640000in}}%
\pgfpathlineto{\pgfqpoint{3.424132in}{-2.640000in}}%
\pgfpathlineto{\pgfqpoint{3.473719in}{-2.640000in}}%
\pgfpathlineto{\pgfqpoint{3.523306in}{-2.640000in}}%
\pgfpathlineto{\pgfqpoint{3.572893in}{-2.640000in}}%
\pgfpathlineto{\pgfqpoint{3.622479in}{-2.640000in}}%
\pgfpathlineto{\pgfqpoint{3.672066in}{-2.640000in}}%
\pgfpathlineto{\pgfqpoint{3.721653in}{-2.640000in}}%
\pgfpathlineto{\pgfqpoint{3.771240in}{-2.640000in}}%
\pgfpathlineto{\pgfqpoint{3.820826in}{-2.640000in}}%
\pgfpathlineto{\pgfqpoint{3.870413in}{-2.640000in}}%
\pgfpathlineto{\pgfqpoint{3.911370in}{-2.640000in}}%
\pgfpathlineto{\pgfqpoint{3.870413in}{-2.553510in}}%
\pgfpathlineto{\pgfqpoint{3.860855in}{-2.533333in}}%
\pgfpathlineto{\pgfqpoint{3.820826in}{-2.448803in}}%
\pgfpathlineto{\pgfqpoint{3.810340in}{-2.426667in}}%
\pgfpathlineto{\pgfqpoint{3.771240in}{-2.344098in}}%
\pgfpathlineto{\pgfqpoint{3.759824in}{-2.320000in}}%
\pgfpathlineto{\pgfqpoint{3.721653in}{-2.239393in}}%
\pgfpathlineto{\pgfqpoint{3.709308in}{-2.213333in}}%
\pgfpathlineto{\pgfqpoint{3.672066in}{-2.134689in}}%
\pgfpathlineto{\pgfqpoint{3.658791in}{-2.106667in}}%
\pgfpathlineto{\pgfqpoint{3.622479in}{-2.029986in}}%
\pgfpathlineto{\pgfqpoint{3.608274in}{-2.000000in}}%
\pgfpathlineto{\pgfqpoint{3.572893in}{-1.925283in}}%
\pgfpathlineto{\pgfqpoint{3.557757in}{-1.893333in}}%
\pgfpathlineto{\pgfqpoint{3.523306in}{-1.820582in}}%
\pgfpathlineto{\pgfqpoint{3.507239in}{-1.786667in}}%
\pgfpathlineto{\pgfqpoint{3.473719in}{-1.715881in}}%
\pgfpathlineto{\pgfqpoint{3.456721in}{-1.680000in}}%
\pgfpathlineto{\pgfqpoint{3.424132in}{-1.611181in}}%
\pgfpathlineto{\pgfqpoint{3.406202in}{-1.573333in}}%
\pgfpathlineto{\pgfqpoint{3.374545in}{-1.506482in}}%
\pgfpathlineto{\pgfqpoint{3.355683in}{-1.466667in}}%
\pgfpathlineto{\pgfqpoint{3.324959in}{-1.401784in}}%
\pgfpathlineto{\pgfqpoint{3.305164in}{-1.360000in}}%
\pgfpathlineto{\pgfqpoint{3.275372in}{-1.297086in}}%
\pgfpathlineto{\pgfqpoint{3.254644in}{-1.253333in}}%
\pgfpathlineto{\pgfqpoint{3.225785in}{-1.192390in}}%
\pgfpathlineto{\pgfqpoint{3.204124in}{-1.146667in}}%
\pgfpathlineto{\pgfqpoint{3.176198in}{-1.087694in}}%
\pgfpathlineto{\pgfqpoint{3.153604in}{-1.040000in}}%
\pgfpathlineto{\pgfqpoint{3.126612in}{-0.982999in}}%
\pgfpathlineto{\pgfqpoint{3.103083in}{-0.933333in}}%
\pgfpathlineto{\pgfqpoint{3.077025in}{-0.878305in}}%
\pgfpathlineto{\pgfqpoint{3.052562in}{-0.826667in}}%
\pgfpathlineto{\pgfqpoint{3.027438in}{-0.773611in}}%
\pgfpathlineto{\pgfqpoint{3.002040in}{-0.720000in}}%
\pgfpathlineto{\pgfqpoint{2.977851in}{-0.668919in}}%
\pgfpathlineto{\pgfqpoint{2.951518in}{-0.613333in}}%
\pgfpathlineto{\pgfqpoint{2.928264in}{-0.564227in}}%
\pgfpathlineto{\pgfqpoint{2.900996in}{-0.506667in}}%
\pgfpathlineto{\pgfqpoint{2.878678in}{-0.459536in}}%
\pgfpathlineto{\pgfqpoint{2.850473in}{-0.400000in}}%
\pgfpathlineto{\pgfqpoint{2.829091in}{-0.354846in}}%
\pgfpathlineto{\pgfqpoint{2.799950in}{-0.293333in}}%
\pgfpathlineto{\pgfqpoint{2.779504in}{-0.250157in}}%
\pgfpathlineto{\pgfqpoint{2.749426in}{-0.186667in}}%
\pgfpathlineto{\pgfqpoint{2.729917in}{-0.145468in}}%
\pgfpathlineto{\pgfqpoint{2.698902in}{-0.080000in}}%
\pgfpathlineto{\pgfqpoint{2.680331in}{-0.040781in}}%
\pgfpathlineto{\pgfqpoint{2.648378in}{0.026667in}}%
\pgfpathlineto{\pgfqpoint{2.630744in}{0.063906in}}%
\pgfpathlineto{\pgfqpoint{2.597853in}{0.133333in}}%
\pgfpathlineto{\pgfqpoint{2.581157in}{0.168592in}}%
\pgfpathlineto{\pgfqpoint{2.547328in}{0.240000in}}%
\pgfpathlineto{\pgfqpoint{2.531570in}{0.273277in}}%
\pgfpathlineto{\pgfqpoint{2.496803in}{0.346667in}}%
\pgfpathlineto{\pgfqpoint{2.481983in}{0.377961in}}%
\pgfpathlineto{\pgfqpoint{2.446277in}{0.453333in}}%
\pgfpathlineto{\pgfqpoint{2.432397in}{0.482644in}}%
\pgfpathlineto{\pgfqpoint{2.395750in}{0.560000in}}%
\pgfpathlineto{\pgfqpoint{2.382810in}{0.587327in}}%
\pgfpathlineto{\pgfqpoint{2.345224in}{0.666667in}}%
\pgfpathlineto{\pgfqpoint{2.333223in}{0.692009in}}%
\pgfpathlineto{\pgfqpoint{2.294696in}{0.773333in}}%
\pgfpathlineto{\pgfqpoint{2.283636in}{0.796690in}}%
\pgfpathlineto{\pgfqpoint{2.244169in}{0.880000in}}%
\pgfpathlineto{\pgfqpoint{2.234050in}{0.901370in}}%
\pgfpathlineto{\pgfqpoint{2.193641in}{0.986667in}}%
\pgfpathlineto{\pgfqpoint{2.184463in}{1.006049in}}%
\pgfpathlineto{\pgfqpoint{2.143113in}{1.093333in}}%
\pgfpathlineto{\pgfqpoint{2.134876in}{1.110727in}}%
\pgfpathlineto{\pgfqpoint{2.092584in}{1.200000in}}%
\pgfpathlineto{\pgfqpoint{2.085289in}{1.215405in}}%
\pgfpathlineto{\pgfqpoint{2.042055in}{1.306667in}}%
\pgfpathlineto{\pgfqpoint{2.035702in}{1.320081in}}%
\pgfpathlineto{\pgfqpoint{1.991525in}{1.413333in}}%
\pgfpathlineto{\pgfqpoint{1.986116in}{1.424757in}}%
\pgfpathlineto{\pgfqpoint{1.940995in}{1.520000in}}%
\pgfpathlineto{\pgfqpoint{1.936529in}{1.529432in}}%
\pgfpathlineto{\pgfqpoint{1.890465in}{1.626667in}}%
\pgfpathlineto{\pgfqpoint{1.886942in}{1.634106in}}%
\pgfpathlineto{\pgfqpoint{1.839934in}{1.733333in}}%
\pgfpathlineto{\pgfqpoint{1.837355in}{1.738779in}}%
\pgfpathlineto{\pgfqpoint{1.789403in}{1.840000in}}%
\pgfpathlineto{\pgfqpoint{1.787769in}{1.843452in}}%
\pgfpathlineto{\pgfqpoint{1.738872in}{1.946667in}}%
\pgfpathlineto{\pgfqpoint{1.738182in}{1.948123in}}%
\pgfpathlineto{\pgfqpoint{1.688595in}{2.052806in}}%
\pgfpathlineto{\pgfqpoint{1.688345in}{2.053333in}}%
\pgfpathlineto{\pgfqpoint{1.639008in}{2.157521in}}%
\pgfpathlineto{\pgfqpoint{1.637834in}{2.160000in}}%
\pgfpathlineto{\pgfqpoint{1.589421in}{2.262234in}}%
\pgfpathlineto{\pgfqpoint{1.587322in}{2.266667in}}%
\pgfpathlineto{\pgfqpoint{1.539835in}{2.366947in}}%
\pgfpathlineto{\pgfqpoint{1.536809in}{2.373333in}}%
\pgfpathlineto{\pgfqpoint{1.490248in}{2.471659in}}%
\pgfpathlineto{\pgfqpoint{1.486297in}{2.480000in}}%
\pgfpathlineto{\pgfqpoint{1.440661in}{2.576371in}}%
\pgfpathlineto{\pgfqpoint{1.435783in}{2.586667in}}%
\pgfpathlineto{\pgfqpoint{1.391074in}{2.681081in}}%
\pgfpathlineto{\pgfqpoint{1.385270in}{2.693333in}}%
\pgfpathlineto{\pgfqpoint{1.341488in}{2.785791in}}%
\pgfpathlineto{\pgfqpoint{1.334756in}{2.800000in}}%
\pgfpathlineto{\pgfqpoint{1.291901in}{2.890500in}}%
\pgfpathlineto{\pgfqpoint{1.284242in}{2.906667in}}%
\pgfpathlineto{\pgfqpoint{1.242314in}{2.995208in}}%
\pgfpathlineto{\pgfqpoint{1.233727in}{3.013333in}}%
\pgfpathlineto{\pgfqpoint{1.192727in}{3.099915in}}%
\pgfpathlineto{\pgfqpoint{1.192727in}{3.013333in}}%
\pgfpathlineto{\pgfqpoint{1.192727in}{2.906667in}}%
\pgfpathlineto{\pgfqpoint{1.192727in}{2.800000in}}%
\pgfpathlineto{\pgfqpoint{1.192727in}{2.693333in}}%
\pgfpathlineto{\pgfqpoint{1.192727in}{2.586667in}}%
\pgfpathlineto{\pgfqpoint{1.192727in}{2.480000in}}%
\pgfpathlineto{\pgfqpoint{1.192727in}{2.373333in}}%
\pgfpathlineto{\pgfqpoint{1.192727in}{2.266667in}}%
\pgfpathlineto{\pgfqpoint{1.192727in}{2.160000in}}%
\pgfpathlineto{\pgfqpoint{1.192727in}{2.053333in}}%
\pgfpathlineto{\pgfqpoint{1.192727in}{1.946667in}}%
\pgfpathlineto{\pgfqpoint{1.192727in}{1.840000in}}%
\pgfpathlineto{\pgfqpoint{1.192727in}{1.733333in}}%
\pgfpathlineto{\pgfqpoint{1.192727in}{1.626667in}}%
\pgfpathlineto{\pgfqpoint{1.192727in}{1.520000in}}%
\pgfpathlineto{\pgfqpoint{1.192727in}{1.413333in}}%
\pgfpathlineto{\pgfqpoint{1.192727in}{1.306667in}}%
\pgfpathlineto{\pgfqpoint{1.192727in}{1.200000in}}%
\pgfpathlineto{\pgfqpoint{1.192727in}{1.093333in}}%
\pgfpathlineto{\pgfqpoint{1.192727in}{0.986667in}}%
\pgfpathlineto{\pgfqpoint{1.192727in}{0.880000in}}%
\pgfpathlineto{\pgfqpoint{1.192727in}{0.773333in}}%
\pgfpathlineto{\pgfqpoint{1.192727in}{0.666667in}}%
\pgfpathlineto{\pgfqpoint{1.192727in}{0.560000in}}%
\pgfpathlineto{\pgfqpoint{1.192727in}{0.453333in}}%
\pgfpathlineto{\pgfqpoint{1.192727in}{0.346667in}}%
\pgfpathlineto{\pgfqpoint{1.192727in}{0.240000in}}%
\pgfpathlineto{\pgfqpoint{1.192727in}{0.133333in}}%
\pgfpathlineto{\pgfqpoint{1.192727in}{0.026667in}}%
\pgfpathlineto{\pgfqpoint{1.192727in}{-0.080000in}}%
\pgfpathlineto{\pgfqpoint{1.192727in}{-0.186667in}}%
\pgfpathlineto{\pgfqpoint{1.192727in}{-0.293333in}}%
\pgfpathlineto{\pgfqpoint{1.192727in}{-0.400000in}}%
\pgfpathlineto{\pgfqpoint{1.192727in}{-0.506667in}}%
\pgfpathlineto{\pgfqpoint{1.192727in}{-0.613333in}}%
\pgfpathlineto{\pgfqpoint{1.192727in}{-0.720000in}}%
\pgfpathlineto{\pgfqpoint{1.192727in}{-0.826667in}}%
\pgfpathlineto{\pgfqpoint{1.192727in}{-0.933333in}}%
\pgfpathlineto{\pgfqpoint{1.192727in}{-1.040000in}}%
\pgfpathlineto{\pgfqpoint{1.192727in}{-1.146667in}}%
\pgfpathlineto{\pgfqpoint{1.192727in}{-1.253333in}}%
\pgfpathlineto{\pgfqpoint{1.192727in}{-1.360000in}}%
\pgfpathlineto{\pgfqpoint{1.192727in}{-1.466667in}}%
\pgfpathlineto{\pgfqpoint{1.192727in}{-1.573333in}}%
\pgfpathlineto{\pgfqpoint{1.192727in}{-1.680000in}}%
\pgfpathlineto{\pgfqpoint{1.192727in}{-1.786667in}}%
\pgfpathlineto{\pgfqpoint{1.192727in}{-1.893333in}}%
\pgfpathlineto{\pgfqpoint{1.192727in}{-2.000000in}}%
\pgfpathlineto{\pgfqpoint{1.192727in}{-2.106667in}}%
\pgfpathlineto{\pgfqpoint{1.192727in}{-2.213333in}}%
\pgfpathlineto{\pgfqpoint{1.192727in}{-2.320000in}}%
\pgfpathlineto{\pgfqpoint{1.192727in}{-2.426667in}}%
\pgfpathlineto{\pgfqpoint{1.192727in}{-2.533333in}}%
\pgfpathlineto{\pgfqpoint{1.192727in}{-2.640000in}}%
\pgfpathclose%
\pgfusepath{fill}%
\end{pgfscope}%
\begin{pgfscope}%
\pgfpathrectangle{\pgfqpoint{3.156364in}{0.528000in}}{\pgfqpoint{1.963636in}{3.696000in}} %
\pgfusepath{clip}%
\pgfsetbuttcap%
\pgfsetroundjoin%
\definecolor{currentfill}{rgb}{0.976265,0.868016,0.143351}%
\pgfsetfillcolor{currentfill}%
\pgfsetlinewidth{0.000000pt}%
\definecolor{currentstroke}{rgb}{0.000000,0.000000,0.000000}%
\pgfsetstrokecolor{currentstroke}%
\pgfsetdash{}{0pt}%
\pgfpathmoveto{\pgfqpoint{6.101818in}{2.180085in}}%
\pgfpathlineto{\pgfqpoint{6.101818in}{2.266667in}}%
\pgfpathlineto{\pgfqpoint{6.101818in}{2.373333in}}%
\pgfpathlineto{\pgfqpoint{6.101818in}{2.480000in}}%
\pgfpathlineto{\pgfqpoint{6.101818in}{2.586667in}}%
\pgfpathlineto{\pgfqpoint{6.101818in}{2.693333in}}%
\pgfpathlineto{\pgfqpoint{6.101818in}{2.800000in}}%
\pgfpathlineto{\pgfqpoint{6.101818in}{2.906667in}}%
\pgfpathlineto{\pgfqpoint{6.101818in}{3.013333in}}%
\pgfpathlineto{\pgfqpoint{6.101818in}{3.120000in}}%
\pgfpathlineto{\pgfqpoint{6.101818in}{3.226667in}}%
\pgfpathlineto{\pgfqpoint{6.101818in}{3.333333in}}%
\pgfpathlineto{\pgfqpoint{6.101818in}{3.440000in}}%
\pgfpathlineto{\pgfqpoint{6.101818in}{3.546667in}}%
\pgfpathlineto{\pgfqpoint{6.101818in}{3.653333in}}%
\pgfpathlineto{\pgfqpoint{6.101818in}{3.760000in}}%
\pgfpathlineto{\pgfqpoint{6.101818in}{3.866667in}}%
\pgfpathlineto{\pgfqpoint{6.101818in}{3.973333in}}%
\pgfpathlineto{\pgfqpoint{6.101818in}{4.080000in}}%
\pgfpathlineto{\pgfqpoint{6.101818in}{4.186667in}}%
\pgfpathlineto{\pgfqpoint{6.101818in}{4.293333in}}%
\pgfpathlineto{\pgfqpoint{6.101818in}{4.400000in}}%
\pgfpathlineto{\pgfqpoint{6.101818in}{4.506667in}}%
\pgfpathlineto{\pgfqpoint{6.101818in}{4.613333in}}%
\pgfpathlineto{\pgfqpoint{6.101818in}{4.720000in}}%
\pgfpathlineto{\pgfqpoint{6.101818in}{4.826667in}}%
\pgfpathlineto{\pgfqpoint{6.101818in}{4.933333in}}%
\pgfpathlineto{\pgfqpoint{6.101818in}{5.040000in}}%
\pgfpathlineto{\pgfqpoint{6.101818in}{5.146667in}}%
\pgfpathlineto{\pgfqpoint{6.101818in}{5.253333in}}%
\pgfpathlineto{\pgfqpoint{6.101818in}{5.360000in}}%
\pgfpathlineto{\pgfqpoint{6.101818in}{5.466667in}}%
\pgfpathlineto{\pgfqpoint{6.101818in}{5.573333in}}%
\pgfpathlineto{\pgfqpoint{6.101818in}{5.680000in}}%
\pgfpathlineto{\pgfqpoint{6.101818in}{5.786667in}}%
\pgfpathlineto{\pgfqpoint{6.101818in}{5.893333in}}%
\pgfpathlineto{\pgfqpoint{6.101818in}{6.000000in}}%
\pgfpathlineto{\pgfqpoint{6.101818in}{6.106667in}}%
\pgfpathlineto{\pgfqpoint{6.101818in}{6.213333in}}%
\pgfpathlineto{\pgfqpoint{6.101818in}{6.320000in}}%
\pgfpathlineto{\pgfqpoint{6.101818in}{6.426667in}}%
\pgfpathlineto{\pgfqpoint{6.101818in}{6.533333in}}%
\pgfpathlineto{\pgfqpoint{6.101818in}{6.640000in}}%
\pgfpathlineto{\pgfqpoint{6.101818in}{6.746667in}}%
\pgfpathlineto{\pgfqpoint{6.101818in}{6.853333in}}%
\pgfpathlineto{\pgfqpoint{6.101818in}{6.960000in}}%
\pgfpathlineto{\pgfqpoint{6.101818in}{7.066667in}}%
\pgfpathlineto{\pgfqpoint{6.101818in}{7.173333in}}%
\pgfpathlineto{\pgfqpoint{6.101818in}{7.280000in}}%
\pgfpathlineto{\pgfqpoint{6.101818in}{7.386667in}}%
\pgfpathlineto{\pgfqpoint{6.101818in}{7.493333in}}%
\pgfpathlineto{\pgfqpoint{6.101818in}{7.600000in}}%
\pgfpathlineto{\pgfqpoint{6.101818in}{7.706667in}}%
\pgfpathlineto{\pgfqpoint{6.101818in}{7.813333in}}%
\pgfpathlineto{\pgfqpoint{6.101818in}{7.920000in}}%
\pgfpathlineto{\pgfqpoint{6.052231in}{7.920000in}}%
\pgfpathlineto{\pgfqpoint{6.002645in}{7.920000in}}%
\pgfpathlineto{\pgfqpoint{5.953058in}{7.920000in}}%
\pgfpathlineto{\pgfqpoint{5.903471in}{7.920000in}}%
\pgfpathlineto{\pgfqpoint{5.853884in}{7.920000in}}%
\pgfpathlineto{\pgfqpoint{5.804298in}{7.920000in}}%
\pgfpathlineto{\pgfqpoint{5.754711in}{7.920000in}}%
\pgfpathlineto{\pgfqpoint{5.705124in}{7.920000in}}%
\pgfpathlineto{\pgfqpoint{5.655537in}{7.920000in}}%
\pgfpathlineto{\pgfqpoint{5.605950in}{7.920000in}}%
\pgfpathlineto{\pgfqpoint{5.556364in}{7.920000in}}%
\pgfpathlineto{\pgfqpoint{5.506777in}{7.920000in}}%
\pgfpathlineto{\pgfqpoint{5.457190in}{7.920000in}}%
\pgfpathlineto{\pgfqpoint{5.407603in}{7.920000in}}%
\pgfpathlineto{\pgfqpoint{5.358017in}{7.920000in}}%
\pgfpathlineto{\pgfqpoint{5.308430in}{7.920000in}}%
\pgfpathlineto{\pgfqpoint{5.258843in}{7.920000in}}%
\pgfpathlineto{\pgfqpoint{5.209256in}{7.920000in}}%
\pgfpathlineto{\pgfqpoint{5.159669in}{7.920000in}}%
\pgfpathlineto{\pgfqpoint{5.110083in}{7.920000in}}%
\pgfpathlineto{\pgfqpoint{5.060496in}{7.920000in}}%
\pgfpathlineto{\pgfqpoint{5.010909in}{7.920000in}}%
\pgfpathlineto{\pgfqpoint{4.961322in}{7.920000in}}%
\pgfpathlineto{\pgfqpoint{4.911736in}{7.920000in}}%
\pgfpathlineto{\pgfqpoint{4.862149in}{7.920000in}}%
\pgfpathlineto{\pgfqpoint{4.812562in}{7.920000in}}%
\pgfpathlineto{\pgfqpoint{4.762975in}{7.920000in}}%
\pgfpathlineto{\pgfqpoint{4.713388in}{7.920000in}}%
\pgfpathlineto{\pgfqpoint{4.663802in}{7.920000in}}%
\pgfpathlineto{\pgfqpoint{4.614215in}{7.920000in}}%
\pgfpathlineto{\pgfqpoint{4.564628in}{7.920000in}}%
\pgfpathlineto{\pgfqpoint{4.515041in}{7.920000in}}%
\pgfpathlineto{\pgfqpoint{4.465455in}{7.920000in}}%
\pgfpathlineto{\pgfqpoint{4.415868in}{7.920000in}}%
\pgfpathlineto{\pgfqpoint{4.366281in}{7.920000in}}%
\pgfpathlineto{\pgfqpoint{4.316694in}{7.920000in}}%
\pgfpathlineto{\pgfqpoint{4.267107in}{7.920000in}}%
\pgfpathlineto{\pgfqpoint{4.217521in}{7.920000in}}%
\pgfpathlineto{\pgfqpoint{4.167934in}{7.920000in}}%
\pgfpathlineto{\pgfqpoint{4.118347in}{7.920000in}}%
\pgfpathlineto{\pgfqpoint{4.068760in}{7.920000in}}%
\pgfpathlineto{\pgfqpoint{4.019174in}{7.920000in}}%
\pgfpathlineto{\pgfqpoint{3.969587in}{7.920000in}}%
\pgfpathlineto{\pgfqpoint{3.920000in}{7.920000in}}%
\pgfpathlineto{\pgfqpoint{3.870413in}{7.920000in}}%
\pgfpathlineto{\pgfqpoint{3.820826in}{7.920000in}}%
\pgfpathlineto{\pgfqpoint{3.771240in}{7.920000in}}%
\pgfpathlineto{\pgfqpoint{3.721653in}{7.920000in}}%
\pgfpathlineto{\pgfqpoint{3.672066in}{7.920000in}}%
\pgfpathlineto{\pgfqpoint{3.622479in}{7.920000in}}%
\pgfpathlineto{\pgfqpoint{3.572893in}{7.920000in}}%
\pgfpathlineto{\pgfqpoint{3.523306in}{7.920000in}}%
\pgfpathlineto{\pgfqpoint{3.473719in}{7.920000in}}%
\pgfpathlineto{\pgfqpoint{3.424132in}{7.920000in}}%
\pgfpathlineto{\pgfqpoint{3.383175in}{7.920000in}}%
\pgfpathlineto{\pgfqpoint{3.424132in}{7.833510in}}%
\pgfpathlineto{\pgfqpoint{3.433690in}{7.813333in}}%
\pgfpathlineto{\pgfqpoint{3.473719in}{7.728803in}}%
\pgfpathlineto{\pgfqpoint{3.484206in}{7.706667in}}%
\pgfpathlineto{\pgfqpoint{3.523306in}{7.624098in}}%
\pgfpathlineto{\pgfqpoint{3.534722in}{7.600000in}}%
\pgfpathlineto{\pgfqpoint{3.572893in}{7.519393in}}%
\pgfpathlineto{\pgfqpoint{3.585238in}{7.493333in}}%
\pgfpathlineto{\pgfqpoint{3.622479in}{7.414689in}}%
\pgfpathlineto{\pgfqpoint{3.635754in}{7.386667in}}%
\pgfpathlineto{\pgfqpoint{3.672066in}{7.309986in}}%
\pgfpathlineto{\pgfqpoint{3.686271in}{7.280000in}}%
\pgfpathlineto{\pgfqpoint{3.721653in}{7.205283in}}%
\pgfpathlineto{\pgfqpoint{3.736789in}{7.173333in}}%
\pgfpathlineto{\pgfqpoint{3.771240in}{7.100582in}}%
\pgfpathlineto{\pgfqpoint{3.787307in}{7.066667in}}%
\pgfpathlineto{\pgfqpoint{3.820826in}{6.995881in}}%
\pgfpathlineto{\pgfqpoint{3.837825in}{6.960000in}}%
\pgfpathlineto{\pgfqpoint{3.870413in}{6.891181in}}%
\pgfpathlineto{\pgfqpoint{3.888343in}{6.853333in}}%
\pgfpathlineto{\pgfqpoint{3.920000in}{6.786482in}}%
\pgfpathlineto{\pgfqpoint{3.938862in}{6.746667in}}%
\pgfpathlineto{\pgfqpoint{3.969587in}{6.681784in}}%
\pgfpathlineto{\pgfqpoint{3.989381in}{6.640000in}}%
\pgfpathlineto{\pgfqpoint{4.019174in}{6.577086in}}%
\pgfpathlineto{\pgfqpoint{4.039901in}{6.533333in}}%
\pgfpathlineto{\pgfqpoint{4.068760in}{6.472390in}}%
\pgfpathlineto{\pgfqpoint{4.090421in}{6.426667in}}%
\pgfpathlineto{\pgfqpoint{4.118347in}{6.367694in}}%
\pgfpathlineto{\pgfqpoint{4.140941in}{6.320000in}}%
\pgfpathlineto{\pgfqpoint{4.167934in}{6.262999in}}%
\pgfpathlineto{\pgfqpoint{4.191462in}{6.213333in}}%
\pgfpathlineto{\pgfqpoint{4.217521in}{6.158305in}}%
\pgfpathlineto{\pgfqpoint{4.241984in}{6.106667in}}%
\pgfpathlineto{\pgfqpoint{4.267107in}{6.053611in}}%
\pgfpathlineto{\pgfqpoint{4.292505in}{6.000000in}}%
\pgfpathlineto{\pgfqpoint{4.316694in}{5.948919in}}%
\pgfpathlineto{\pgfqpoint{4.343027in}{5.893333in}}%
\pgfpathlineto{\pgfqpoint{4.366281in}{5.844227in}}%
\pgfpathlineto{\pgfqpoint{4.393550in}{5.786667in}}%
\pgfpathlineto{\pgfqpoint{4.415868in}{5.739536in}}%
\pgfpathlineto{\pgfqpoint{4.444072in}{5.680000in}}%
\pgfpathlineto{\pgfqpoint{4.465455in}{5.634846in}}%
\pgfpathlineto{\pgfqpoint{4.494595in}{5.573333in}}%
\pgfpathlineto{\pgfqpoint{4.515041in}{5.530157in}}%
\pgfpathlineto{\pgfqpoint{4.545119in}{5.466667in}}%
\pgfpathlineto{\pgfqpoint{4.564628in}{5.425468in}}%
\pgfpathlineto{\pgfqpoint{4.595643in}{5.360000in}}%
\pgfpathlineto{\pgfqpoint{4.614215in}{5.320781in}}%
\pgfpathlineto{\pgfqpoint{4.646167in}{5.253333in}}%
\pgfpathlineto{\pgfqpoint{4.663802in}{5.216094in}}%
\pgfpathlineto{\pgfqpoint{4.696692in}{5.146667in}}%
\pgfpathlineto{\pgfqpoint{4.713388in}{5.111408in}}%
\pgfpathlineto{\pgfqpoint{4.747217in}{5.040000in}}%
\pgfpathlineto{\pgfqpoint{4.762975in}{5.006723in}}%
\pgfpathlineto{\pgfqpoint{4.797743in}{4.933333in}}%
\pgfpathlineto{\pgfqpoint{4.812562in}{4.902039in}}%
\pgfpathlineto{\pgfqpoint{4.848269in}{4.826667in}}%
\pgfpathlineto{\pgfqpoint{4.862149in}{4.797356in}}%
\pgfpathlineto{\pgfqpoint{4.898795in}{4.720000in}}%
\pgfpathlineto{\pgfqpoint{4.911736in}{4.692673in}}%
\pgfpathlineto{\pgfqpoint{4.949322in}{4.613333in}}%
\pgfpathlineto{\pgfqpoint{4.961322in}{4.587991in}}%
\pgfpathlineto{\pgfqpoint{4.999849in}{4.506667in}}%
\pgfpathlineto{\pgfqpoint{5.010909in}{4.483310in}}%
\pgfpathlineto{\pgfqpoint{5.050377in}{4.400000in}}%
\pgfpathlineto{\pgfqpoint{5.060496in}{4.378630in}}%
\pgfpathlineto{\pgfqpoint{5.100904in}{4.293333in}}%
\pgfpathlineto{\pgfqpoint{5.110083in}{4.273951in}}%
\pgfpathlineto{\pgfqpoint{5.151433in}{4.186667in}}%
\pgfpathlineto{\pgfqpoint{5.159669in}{4.169273in}}%
\pgfpathlineto{\pgfqpoint{5.201962in}{4.080000in}}%
\pgfpathlineto{\pgfqpoint{5.209256in}{4.064595in}}%
\pgfpathlineto{\pgfqpoint{5.252491in}{3.973333in}}%
\pgfpathlineto{\pgfqpoint{5.258843in}{3.959919in}}%
\pgfpathlineto{\pgfqpoint{5.303020in}{3.866667in}}%
\pgfpathlineto{\pgfqpoint{5.308430in}{3.855243in}}%
\pgfpathlineto{\pgfqpoint{5.353550in}{3.760000in}}%
\pgfpathlineto{\pgfqpoint{5.358017in}{3.750568in}}%
\pgfpathlineto{\pgfqpoint{5.404080in}{3.653333in}}%
\pgfpathlineto{\pgfqpoint{5.407603in}{3.645894in}}%
\pgfpathlineto{\pgfqpoint{5.454611in}{3.546667in}}%
\pgfpathlineto{\pgfqpoint{5.457190in}{3.541221in}}%
\pgfpathlineto{\pgfqpoint{5.505142in}{3.440000in}}%
\pgfpathlineto{\pgfqpoint{5.506777in}{3.436548in}}%
\pgfpathlineto{\pgfqpoint{5.555674in}{3.333333in}}%
\pgfpathlineto{\pgfqpoint{5.556364in}{3.331877in}}%
\pgfpathlineto{\pgfqpoint{5.605950in}{3.227194in}}%
\pgfpathlineto{\pgfqpoint{5.606200in}{3.226667in}}%
\pgfpathlineto{\pgfqpoint{5.655537in}{3.122479in}}%
\pgfpathlineto{\pgfqpoint{5.656712in}{3.120000in}}%
\pgfpathlineto{\pgfqpoint{5.705124in}{3.017766in}}%
\pgfpathlineto{\pgfqpoint{5.707224in}{3.013333in}}%
\pgfpathlineto{\pgfqpoint{5.754711in}{2.913053in}}%
\pgfpathlineto{\pgfqpoint{5.757736in}{2.906667in}}%
\pgfpathlineto{\pgfqpoint{5.804298in}{2.808341in}}%
\pgfpathlineto{\pgfqpoint{5.808249in}{2.800000in}}%
\pgfpathlineto{\pgfqpoint{5.853884in}{2.703629in}}%
\pgfpathlineto{\pgfqpoint{5.858762in}{2.693333in}}%
\pgfpathlineto{\pgfqpoint{5.903471in}{2.598919in}}%
\pgfpathlineto{\pgfqpoint{5.909275in}{2.586667in}}%
\pgfpathlineto{\pgfqpoint{5.953058in}{2.494209in}}%
\pgfpathlineto{\pgfqpoint{5.959789in}{2.480000in}}%
\pgfpathlineto{\pgfqpoint{6.002645in}{2.389500in}}%
\pgfpathlineto{\pgfqpoint{6.010304in}{2.373333in}}%
\pgfpathlineto{\pgfqpoint{6.052231in}{2.284792in}}%
\pgfpathlineto{\pgfqpoint{6.060818in}{2.266667in}}%
\pgfpathclose%
\pgfusepath{fill}%
\end{pgfscope}%
\begin{pgfscope}%
\pgfpathrectangle{\pgfqpoint{3.156364in}{0.528000in}}{\pgfqpoint{1.963636in}{3.696000in}} %
\pgfusepath{clip}%
\pgfsetbuttcap%
\pgfsetmiterjoin%
\definecolor{currentfill}{rgb}{0.274510,0.509804,0.705882}%
\pgfsetfillcolor{currentfill}%
\pgfsetlinewidth{1.003750pt}%
\definecolor{currentstroke}{rgb}{0.274510,0.509804,0.705882}%
\pgfsetstrokecolor{currentstroke}%
\pgfsetdash{}{0pt}%
\pgfpathmoveto{\pgfqpoint{5.901398in}{5.798488in}}%
\pgfpathcurveto{\pgfqpoint{5.902700in}{5.798488in}}{\pgfqpoint{5.903949in}{5.799601in}}{\pgfqpoint{5.904869in}{5.801581in}}%
\pgfpathcurveto{\pgfqpoint{5.905790in}{5.803561in}}{\pgfqpoint{5.906307in}{5.806247in}}{\pgfqpoint{5.906307in}{5.809048in}}%
\pgfpathcurveto{\pgfqpoint{5.906307in}{5.811848in}}{\pgfqpoint{5.905790in}{5.814535in}}{\pgfqpoint{5.904869in}{5.816515in}}%
\pgfpathcurveto{\pgfqpoint{5.903949in}{5.818495in}}{\pgfqpoint{5.902700in}{5.819608in}}{\pgfqpoint{5.901398in}{5.819608in}}%
\pgfpathcurveto{\pgfqpoint{5.900096in}{5.819608in}}{\pgfqpoint{5.898847in}{5.818495in}}{\pgfqpoint{5.897927in}{5.816515in}}%
\pgfpathcurveto{\pgfqpoint{5.897006in}{5.814535in}}{\pgfqpoint{5.896489in}{5.811848in}}{\pgfqpoint{5.896489in}{5.809048in}}%
\pgfpathcurveto{\pgfqpoint{5.896489in}{5.806247in}}{\pgfqpoint{5.897006in}{5.803561in}}{\pgfqpoint{5.897927in}{5.801581in}}%
\pgfpathcurveto{\pgfqpoint{5.898847in}{5.799601in}}{\pgfqpoint{5.900096in}{5.798488in}}{\pgfqpoint{5.901398in}{5.798488in}}%
\pgfpathclose%
\pgfusepath{stroke,fill}%
\end{pgfscope}%
\begin{pgfscope}%
\pgfsetbuttcap%
\pgfsetroundjoin%
\definecolor{currentfill}{rgb}{0.000000,0.000000,0.000000}%
\pgfsetfillcolor{currentfill}%
\pgfsetlinewidth{0.803000pt}%
\definecolor{currentstroke}{rgb}{0.000000,0.000000,0.000000}%
\pgfsetstrokecolor{currentstroke}%
\pgfsetdash{}{0pt}%
\pgfsys@defobject{currentmarker}{\pgfqpoint{0.000000in}{-0.048611in}}{\pgfqpoint{0.000000in}{0.000000in}}{%
\pgfpathmoveto{\pgfqpoint{0.000000in}{0.000000in}}%
\pgfpathlineto{\pgfqpoint{0.000000in}{-0.048611in}}%
\pgfusepath{stroke,fill}%
}%
\begin{pgfscope}%
\pgfsys@transformshift{3.156364in}{0.528000in}%
\pgfsys@useobject{currentmarker}{}%
\end{pgfscope}%
\end{pgfscope}%
\begin{pgfscope}%
\pgftext[x=3.156364in,y=0.430778in,,top]{\sffamily\fontsize{10.000000}{12.000000}\selectfont −100}%
\end{pgfscope}%
\begin{pgfscope}%
\pgfsetbuttcap%
\pgfsetroundjoin%
\definecolor{currentfill}{rgb}{0.000000,0.000000,0.000000}%
\pgfsetfillcolor{currentfill}%
\pgfsetlinewidth{0.803000pt}%
\definecolor{currentstroke}{rgb}{0.000000,0.000000,0.000000}%
\pgfsetstrokecolor{currentstroke}%
\pgfsetdash{}{0pt}%
\pgfsys@defobject{currentmarker}{\pgfqpoint{0.000000in}{-0.048611in}}{\pgfqpoint{0.000000in}{0.000000in}}{%
\pgfpathmoveto{\pgfqpoint{0.000000in}{0.000000in}}%
\pgfpathlineto{\pgfqpoint{0.000000in}{-0.048611in}}%
\pgfusepath{stroke,fill}%
}%
\begin{pgfscope}%
\pgfsys@transformshift{3.647273in}{0.528000in}%
\pgfsys@useobject{currentmarker}{}%
\end{pgfscope}%
\end{pgfscope}%
\begin{pgfscope}%
\pgftext[x=3.647273in,y=0.430778in,,top]{\sffamily\fontsize{10.000000}{12.000000}\selectfont 0}%
\end{pgfscope}%
\begin{pgfscope}%
\pgfsetbuttcap%
\pgfsetroundjoin%
\definecolor{currentfill}{rgb}{0.000000,0.000000,0.000000}%
\pgfsetfillcolor{currentfill}%
\pgfsetlinewidth{0.803000pt}%
\definecolor{currentstroke}{rgb}{0.000000,0.000000,0.000000}%
\pgfsetstrokecolor{currentstroke}%
\pgfsetdash{}{0pt}%
\pgfsys@defobject{currentmarker}{\pgfqpoint{0.000000in}{-0.048611in}}{\pgfqpoint{0.000000in}{0.000000in}}{%
\pgfpathmoveto{\pgfqpoint{0.000000in}{0.000000in}}%
\pgfpathlineto{\pgfqpoint{0.000000in}{-0.048611in}}%
\pgfusepath{stroke,fill}%
}%
\begin{pgfscope}%
\pgfsys@transformshift{4.138182in}{0.528000in}%
\pgfsys@useobject{currentmarker}{}%
\end{pgfscope}%
\end{pgfscope}%
\begin{pgfscope}%
\pgftext[x=4.138182in,y=0.430778in,,top]{\sffamily\fontsize{10.000000}{12.000000}\selectfont 100}%
\end{pgfscope}%
\begin{pgfscope}%
\pgfsetbuttcap%
\pgfsetroundjoin%
\definecolor{currentfill}{rgb}{0.000000,0.000000,0.000000}%
\pgfsetfillcolor{currentfill}%
\pgfsetlinewidth{0.803000pt}%
\definecolor{currentstroke}{rgb}{0.000000,0.000000,0.000000}%
\pgfsetstrokecolor{currentstroke}%
\pgfsetdash{}{0pt}%
\pgfsys@defobject{currentmarker}{\pgfqpoint{0.000000in}{-0.048611in}}{\pgfqpoint{0.000000in}{0.000000in}}{%
\pgfpathmoveto{\pgfqpoint{0.000000in}{0.000000in}}%
\pgfpathlineto{\pgfqpoint{0.000000in}{-0.048611in}}%
\pgfusepath{stroke,fill}%
}%
\begin{pgfscope}%
\pgfsys@transformshift{4.629091in}{0.528000in}%
\pgfsys@useobject{currentmarker}{}%
\end{pgfscope}%
\end{pgfscope}%
\begin{pgfscope}%
\pgftext[x=4.629091in,y=0.430778in,,top]{\sffamily\fontsize{10.000000}{12.000000}\selectfont 200}%
\end{pgfscope}%
\begin{pgfscope}%
\pgfsetbuttcap%
\pgfsetroundjoin%
\definecolor{currentfill}{rgb}{0.000000,0.000000,0.000000}%
\pgfsetfillcolor{currentfill}%
\pgfsetlinewidth{0.803000pt}%
\definecolor{currentstroke}{rgb}{0.000000,0.000000,0.000000}%
\pgfsetstrokecolor{currentstroke}%
\pgfsetdash{}{0pt}%
\pgfsys@defobject{currentmarker}{\pgfqpoint{0.000000in}{-0.048611in}}{\pgfqpoint{0.000000in}{0.000000in}}{%
\pgfpathmoveto{\pgfqpoint{0.000000in}{0.000000in}}%
\pgfpathlineto{\pgfqpoint{0.000000in}{-0.048611in}}%
\pgfusepath{stroke,fill}%
}%
\begin{pgfscope}%
\pgfsys@transformshift{5.120000in}{0.528000in}%
\pgfsys@useobject{currentmarker}{}%
\end{pgfscope}%
\end{pgfscope}%
\begin{pgfscope}%
\pgftext[x=5.120000in,y=0.430778in,,top]{\sffamily\fontsize{10.000000}{12.000000}\selectfont 300}%
\end{pgfscope}%
\begin{pgfscope}%
\pgftext[x=4.138182in,y=0.240809in,,top]{\sffamily\fontsize{10.000000}{12.000000}\selectfont a}%
\end{pgfscope}%
\begin{pgfscope}%
\pgfsetbuttcap%
\pgfsetroundjoin%
\definecolor{currentfill}{rgb}{0.000000,0.000000,0.000000}%
\pgfsetfillcolor{currentfill}%
\pgfsetlinewidth{0.803000pt}%
\definecolor{currentstroke}{rgb}{0.000000,0.000000,0.000000}%
\pgfsetstrokecolor{currentstroke}%
\pgfsetdash{}{0pt}%
\pgfsys@defobject{currentmarker}{\pgfqpoint{-0.048611in}{0.000000in}}{\pgfqpoint{0.000000in}{0.000000in}}{%
\pgfpathmoveto{\pgfqpoint{0.000000in}{0.000000in}}%
\pgfpathlineto{\pgfqpoint{-0.048611in}{0.000000in}}%
\pgfusepath{stroke,fill}%
}%
\begin{pgfscope}%
\pgfsys@transformshift{3.156364in}{0.528000in}%
\pgfsys@useobject{currentmarker}{}%
\end{pgfscope}%
\end{pgfscope}%
\begin{pgfscope}%
\pgfsetbuttcap%
\pgfsetroundjoin%
\definecolor{currentfill}{rgb}{0.000000,0.000000,0.000000}%
\pgfsetfillcolor{currentfill}%
\pgfsetlinewidth{0.803000pt}%
\definecolor{currentstroke}{rgb}{0.000000,0.000000,0.000000}%
\pgfsetstrokecolor{currentstroke}%
\pgfsetdash{}{0pt}%
\pgfsys@defobject{currentmarker}{\pgfqpoint{-0.048611in}{0.000000in}}{\pgfqpoint{0.000000in}{0.000000in}}{%
\pgfpathmoveto{\pgfqpoint{0.000000in}{0.000000in}}%
\pgfpathlineto{\pgfqpoint{-0.048611in}{0.000000in}}%
\pgfusepath{stroke,fill}%
}%
\begin{pgfscope}%
\pgfsys@transformshift{3.156364in}{1.056000in}%
\pgfsys@useobject{currentmarker}{}%
\end{pgfscope}%
\end{pgfscope}%
\begin{pgfscope}%
\pgfsetbuttcap%
\pgfsetroundjoin%
\definecolor{currentfill}{rgb}{0.000000,0.000000,0.000000}%
\pgfsetfillcolor{currentfill}%
\pgfsetlinewidth{0.803000pt}%
\definecolor{currentstroke}{rgb}{0.000000,0.000000,0.000000}%
\pgfsetstrokecolor{currentstroke}%
\pgfsetdash{}{0pt}%
\pgfsys@defobject{currentmarker}{\pgfqpoint{-0.048611in}{0.000000in}}{\pgfqpoint{0.000000in}{0.000000in}}{%
\pgfpathmoveto{\pgfqpoint{0.000000in}{0.000000in}}%
\pgfpathlineto{\pgfqpoint{-0.048611in}{0.000000in}}%
\pgfusepath{stroke,fill}%
}%
\begin{pgfscope}%
\pgfsys@transformshift{3.156364in}{1.584000in}%
\pgfsys@useobject{currentmarker}{}%
\end{pgfscope}%
\end{pgfscope}%
\begin{pgfscope}%
\pgfsetbuttcap%
\pgfsetroundjoin%
\definecolor{currentfill}{rgb}{0.000000,0.000000,0.000000}%
\pgfsetfillcolor{currentfill}%
\pgfsetlinewidth{0.803000pt}%
\definecolor{currentstroke}{rgb}{0.000000,0.000000,0.000000}%
\pgfsetstrokecolor{currentstroke}%
\pgfsetdash{}{0pt}%
\pgfsys@defobject{currentmarker}{\pgfqpoint{-0.048611in}{0.000000in}}{\pgfqpoint{0.000000in}{0.000000in}}{%
\pgfpathmoveto{\pgfqpoint{0.000000in}{0.000000in}}%
\pgfpathlineto{\pgfqpoint{-0.048611in}{0.000000in}}%
\pgfusepath{stroke,fill}%
}%
\begin{pgfscope}%
\pgfsys@transformshift{3.156364in}{2.112000in}%
\pgfsys@useobject{currentmarker}{}%
\end{pgfscope}%
\end{pgfscope}%
\begin{pgfscope}%
\pgfsetbuttcap%
\pgfsetroundjoin%
\definecolor{currentfill}{rgb}{0.000000,0.000000,0.000000}%
\pgfsetfillcolor{currentfill}%
\pgfsetlinewidth{0.803000pt}%
\definecolor{currentstroke}{rgb}{0.000000,0.000000,0.000000}%
\pgfsetstrokecolor{currentstroke}%
\pgfsetdash{}{0pt}%
\pgfsys@defobject{currentmarker}{\pgfqpoint{-0.048611in}{0.000000in}}{\pgfqpoint{0.000000in}{0.000000in}}{%
\pgfpathmoveto{\pgfqpoint{0.000000in}{0.000000in}}%
\pgfpathlineto{\pgfqpoint{-0.048611in}{0.000000in}}%
\pgfusepath{stroke,fill}%
}%
\begin{pgfscope}%
\pgfsys@transformshift{3.156364in}{2.640000in}%
\pgfsys@useobject{currentmarker}{}%
\end{pgfscope}%
\end{pgfscope}%
\begin{pgfscope}%
\pgfsetbuttcap%
\pgfsetroundjoin%
\definecolor{currentfill}{rgb}{0.000000,0.000000,0.000000}%
\pgfsetfillcolor{currentfill}%
\pgfsetlinewidth{0.803000pt}%
\definecolor{currentstroke}{rgb}{0.000000,0.000000,0.000000}%
\pgfsetstrokecolor{currentstroke}%
\pgfsetdash{}{0pt}%
\pgfsys@defobject{currentmarker}{\pgfqpoint{-0.048611in}{0.000000in}}{\pgfqpoint{0.000000in}{0.000000in}}{%
\pgfpathmoveto{\pgfqpoint{0.000000in}{0.000000in}}%
\pgfpathlineto{\pgfqpoint{-0.048611in}{0.000000in}}%
\pgfusepath{stroke,fill}%
}%
\begin{pgfscope}%
\pgfsys@transformshift{3.156364in}{3.168000in}%
\pgfsys@useobject{currentmarker}{}%
\end{pgfscope}%
\end{pgfscope}%
\begin{pgfscope}%
\pgfsetbuttcap%
\pgfsetroundjoin%
\definecolor{currentfill}{rgb}{0.000000,0.000000,0.000000}%
\pgfsetfillcolor{currentfill}%
\pgfsetlinewidth{0.803000pt}%
\definecolor{currentstroke}{rgb}{0.000000,0.000000,0.000000}%
\pgfsetstrokecolor{currentstroke}%
\pgfsetdash{}{0pt}%
\pgfsys@defobject{currentmarker}{\pgfqpoint{-0.048611in}{0.000000in}}{\pgfqpoint{0.000000in}{0.000000in}}{%
\pgfpathmoveto{\pgfqpoint{0.000000in}{0.000000in}}%
\pgfpathlineto{\pgfqpoint{-0.048611in}{0.000000in}}%
\pgfusepath{stroke,fill}%
}%
\begin{pgfscope}%
\pgfsys@transformshift{3.156364in}{3.696000in}%
\pgfsys@useobject{currentmarker}{}%
\end{pgfscope}%
\end{pgfscope}%
\begin{pgfscope}%
\pgfsetbuttcap%
\pgfsetroundjoin%
\definecolor{currentfill}{rgb}{0.000000,0.000000,0.000000}%
\pgfsetfillcolor{currentfill}%
\pgfsetlinewidth{0.803000pt}%
\definecolor{currentstroke}{rgb}{0.000000,0.000000,0.000000}%
\pgfsetstrokecolor{currentstroke}%
\pgfsetdash{}{0pt}%
\pgfsys@defobject{currentmarker}{\pgfqpoint{-0.048611in}{0.000000in}}{\pgfqpoint{0.000000in}{0.000000in}}{%
\pgfpathmoveto{\pgfqpoint{0.000000in}{0.000000in}}%
\pgfpathlineto{\pgfqpoint{-0.048611in}{0.000000in}}%
\pgfusepath{stroke,fill}%
}%
\begin{pgfscope}%
\pgfsys@transformshift{3.156364in}{4.224000in}%
\pgfsys@useobject{currentmarker}{}%
\end{pgfscope}%
\end{pgfscope}%
\begin{pgfscope}%
\pgfpathrectangle{\pgfqpoint{3.156364in}{0.528000in}}{\pgfqpoint{1.963636in}{3.696000in}} %
\pgfusepath{clip}%
\pgfsetrectcap%
\pgfsetroundjoin%
\pgfsetlinewidth{1.505625pt}%
\definecolor{currentstroke}{rgb}{1.000000,1.000000,1.000000}%
\pgfsetstrokecolor{currentstroke}%
\pgfsetdash{}{0pt}%
\pgfpathmoveto{\pgfqpoint{5.130000in}{2.000935in}}%
\pgfpathlineto{\pgfqpoint{4.961446in}{1.640554in}}%
\pgfpathlineto{\pgfqpoint{4.776223in}{1.344480in}}%
\pgfpathlineto{\pgfqpoint{4.623497in}{1.115827in}}%
\pgfpathlineto{\pgfqpoint{4.506603in}{0.998573in}}%
\pgfpathlineto{\pgfqpoint{4.401628in}{0.886345in}}%
\pgfpathlineto{\pgfqpoint{4.330128in}{0.917880in}}%
\pgfpathlineto{\pgfqpoint{4.291455in}{1.008836in}}%
\pgfpathlineto{\pgfqpoint{4.279587in}{1.188674in}}%
\pgfpathlineto{\pgfqpoint{4.273327in}{1.478570in}}%
\pgfpathlineto{\pgfqpoint{4.275972in}{1.750145in}}%
\pgfpathlineto{\pgfqpoint{4.282796in}{2.051161in}}%
\pgfpathlineto{\pgfqpoint{4.288414in}{2.306139in}}%
\pgfpathlineto{\pgfqpoint{4.253483in}{2.430171in}}%
\pgfpathlineto{\pgfqpoint{4.193010in}{2.452958in}}%
\pgfpathlineto{\pgfqpoint{4.103712in}{2.383854in}}%
\pgfpathlineto{\pgfqpoint{3.996008in}{2.255418in}}%
\pgfpathlineto{\pgfqpoint{3.906441in}{2.172864in}}%
\pgfpathlineto{\pgfqpoint{3.834621in}{2.119840in}}%
\pgfpathlineto{\pgfqpoint{3.780550in}{2.091418in}}%
\pgfpathlineto{\pgfqpoint{3.741447in}{2.113288in}}%
\pgfpathlineto{\pgfqpoint{3.713172in}{2.180321in}}%
\pgfpathlineto{\pgfqpoint{3.693176in}{2.280657in}}%
\pgfpathlineto{\pgfqpoint{3.684501in}{2.396762in}}%
\pgfpathlineto{\pgfqpoint{3.678858in}{2.518672in}}%
\pgfpathlineto{\pgfqpoint{3.676385in}{2.631211in}}%
\pgfpathlineto{\pgfqpoint{3.674145in}{2.725994in}}%
\pgfpathlineto{\pgfqpoint{3.669309in}{2.795312in}}%
\pgfpathlineto{\pgfqpoint{3.665052in}{2.838630in}}%
\pgfpathlineto{\pgfqpoint{3.655424in}{2.853934in}}%
\pgfpathlineto{\pgfqpoint{3.643615in}{2.845316in}}%
\pgfpathlineto{\pgfqpoint{3.626814in}{2.818875in}}%
\pgfpathlineto{\pgfqpoint{3.608893in}{2.781367in}}%
\pgfpathlineto{\pgfqpoint{3.591231in}{2.741334in}}%
\pgfpathlineto{\pgfqpoint{3.575569in}{2.699554in}}%
\pgfpathlineto{\pgfqpoint{3.562158in}{2.663387in}}%
\pgfpathlineto{\pgfqpoint{3.552318in}{2.639618in}}%
\pgfpathlineto{\pgfqpoint{3.549160in}{2.634944in}}%
\pgfpathlineto{\pgfqpoint{3.551419in}{2.646588in}}%
\pgfpathlineto{\pgfqpoint{3.553358in}{2.657139in}}%
\pgfpathlineto{\pgfqpoint{3.558099in}{2.677179in}}%
\pgfpathlineto{\pgfqpoint{3.562205in}{2.694959in}}%
\pgfpathlineto{\pgfqpoint{3.565514in}{2.707295in}}%
\pgfpathlineto{\pgfqpoint{3.570398in}{2.724136in}}%
\pgfpathlineto{\pgfqpoint{3.575832in}{2.742384in}}%
\pgfpathlineto{\pgfqpoint{3.581393in}{2.760491in}}%
\pgfpathlineto{\pgfqpoint{3.585043in}{2.770854in}}%
\pgfpathlineto{\pgfqpoint{3.587185in}{2.776716in}}%
\pgfpathlineto{\pgfqpoint{3.588648in}{2.768877in}}%
\pgfpathlineto{\pgfqpoint{3.589450in}{2.761111in}}%
\pgfpathlineto{\pgfqpoint{3.590156in}{2.754570in}}%
\pgfpathlineto{\pgfqpoint{3.589721in}{2.745038in}}%
\pgfpathlineto{\pgfqpoint{3.588428in}{2.732627in}}%
\pgfpathlineto{\pgfqpoint{3.587026in}{2.713926in}}%
\pgfpathlineto{\pgfqpoint{3.586038in}{2.698787in}}%
\pgfpathlineto{\pgfqpoint{3.585281in}{2.686320in}}%
\pgfpathlineto{\pgfqpoint{3.584465in}{2.673407in}}%
\pgfpathlineto{\pgfqpoint{3.584694in}{2.666397in}}%
\pgfpathlineto{\pgfqpoint{3.587513in}{2.668267in}}%
\pgfpathlineto{\pgfqpoint{3.590773in}{2.673720in}}%
\pgfpathlineto{\pgfqpoint{3.596235in}{2.685580in}}%
\pgfpathlineto{\pgfqpoint{3.602511in}{2.700275in}}%
\pgfpathlineto{\pgfqpoint{3.607454in}{2.709536in}}%
\pgfpathlineto{\pgfqpoint{3.611650in}{2.717392in}}%
\pgfpathlineto{\pgfqpoint{3.615600in}{2.724908in}}%
\pgfpathlineto{\pgfqpoint{3.618366in}{2.724242in}}%
\pgfpathlineto{\pgfqpoint{3.620359in}{2.715989in}}%
\pgfpathlineto{\pgfqpoint{3.621160in}{2.706312in}}%
\pgfpathlineto{\pgfqpoint{3.621368in}{2.696814in}}%
\pgfpathlineto{\pgfqpoint{3.620981in}{2.685989in}}%
\pgfpathlineto{\pgfqpoint{3.620434in}{2.675956in}}%
\pgfpathlineto{\pgfqpoint{3.619794in}{2.666434in}}%
\pgfpathlineto{\pgfqpoint{3.620431in}{2.661434in}}%
\pgfpathlineto{\pgfqpoint{3.620946in}{2.656974in}}%
\pgfpathlineto{\pgfqpoint{3.623626in}{2.658100in}}%
\pgfpathlineto{\pgfqpoint{3.626923in}{2.661935in}}%
\pgfpathlineto{\pgfqpoint{3.629725in}{2.663003in}}%
\pgfpathlineto{\pgfqpoint{3.632858in}{2.665785in}}%
\pgfpathlineto{\pgfqpoint{3.635338in}{2.666923in}}%
\pgfpathlineto{\pgfqpoint{3.637305in}{2.667387in}}%
\pgfpathlineto{\pgfqpoint{3.638952in}{2.665108in}}%
\pgfpathlineto{\pgfqpoint{3.640097in}{2.662030in}}%
\pgfpathlineto{\pgfqpoint{3.640981in}{2.657480in}}%
\pgfpathlineto{\pgfqpoint{3.641589in}{2.652590in}}%
\pgfusepath{stroke}%
\end{pgfscope}%
\begin{pgfscope}%
\pgfsetrectcap%
\pgfsetmiterjoin%
\pgfsetlinewidth{0.803000pt}%
\definecolor{currentstroke}{rgb}{0.000000,0.000000,0.000000}%
\pgfsetstrokecolor{currentstroke}%
\pgfsetdash{}{0pt}%
\pgfpathmoveto{\pgfqpoint{3.156364in}{0.528000in}}%
\pgfpathlineto{\pgfqpoint{3.156364in}{4.224000in}}%
\pgfusepath{stroke}%
\end{pgfscope}%
\begin{pgfscope}%
\pgfsetrectcap%
\pgfsetmiterjoin%
\pgfsetlinewidth{0.803000pt}%
\definecolor{currentstroke}{rgb}{0.000000,0.000000,0.000000}%
\pgfsetstrokecolor{currentstroke}%
\pgfsetdash{}{0pt}%
\pgfpathmoveto{\pgfqpoint{5.120000in}{0.528000in}}%
\pgfpathlineto{\pgfqpoint{5.120000in}{4.224000in}}%
\pgfusepath{stroke}%
\end{pgfscope}%
\begin{pgfscope}%
\pgfsetrectcap%
\pgfsetmiterjoin%
\pgfsetlinewidth{0.803000pt}%
\definecolor{currentstroke}{rgb}{0.000000,0.000000,0.000000}%
\pgfsetstrokecolor{currentstroke}%
\pgfsetdash{}{0pt}%
\pgfpathmoveto{\pgfqpoint{3.156364in}{0.528000in}}%
\pgfpathlineto{\pgfqpoint{5.120000in}{0.528000in}}%
\pgfusepath{stroke}%
\end{pgfscope}%
\begin{pgfscope}%
\pgfsetrectcap%
\pgfsetmiterjoin%
\pgfsetlinewidth{0.803000pt}%
\definecolor{currentstroke}{rgb}{0.000000,0.000000,0.000000}%
\pgfsetstrokecolor{currentstroke}%
\pgfsetdash{}{0pt}%
\pgfpathmoveto{\pgfqpoint{3.156364in}{4.224000in}}%
\pgfpathlineto{\pgfqpoint{5.120000in}{4.224000in}}%
\pgfusepath{stroke}%
\end{pgfscope}%
\begin{pgfscope}%
\pgfpathrectangle{\pgfqpoint{5.440000in}{0.720000in}}{\pgfqpoint{0.320000in}{3.360000in}} %
\pgfusepath{clip}%
\pgfsetbuttcap%
\pgfsetmiterjoin%
\definecolor{currentfill}{rgb}{1.000000,1.000000,1.000000}%
\pgfsetfillcolor{currentfill}%
\pgfsetlinewidth{0.010037pt}%
\definecolor{currentstroke}{rgb}{1.000000,1.000000,1.000000}%
\pgfsetstrokecolor{currentstroke}%
\pgfsetdash{}{0pt}%
\pgfpathmoveto{\pgfqpoint{5.440000in}{0.720000in}}%
\pgfpathlineto{\pgfqpoint{5.440000in}{1.025455in}}%
\pgfpathlineto{\pgfqpoint{5.440000in}{3.774545in}}%
\pgfpathlineto{\pgfqpoint{5.440000in}{4.080000in}}%
\pgfpathlineto{\pgfqpoint{5.760000in}{4.080000in}}%
\pgfpathlineto{\pgfqpoint{5.760000in}{3.774545in}}%
\pgfpathlineto{\pgfqpoint{5.760000in}{1.025455in}}%
\pgfpathlineto{\pgfqpoint{5.760000in}{0.720000in}}%
\pgfpathclose%
\pgfusepath{stroke,fill}%
\end{pgfscope}%
\begin{pgfscope}%
\pgfpathrectangle{\pgfqpoint{5.440000in}{0.720000in}}{\pgfqpoint{0.320000in}{3.360000in}} %
\pgfusepath{clip}%
\pgfsetbuttcap%
\pgfsetroundjoin%
\definecolor{currentfill}{rgb}{0.050383,0.029803,0.527975}%
\pgfsetfillcolor{currentfill}%
\pgfsetlinewidth{0.000000pt}%
\definecolor{currentstroke}{rgb}{0.000000,0.000000,0.000000}%
\pgfsetstrokecolor{currentstroke}%
\pgfsetdash{}{0pt}%
\pgfpathmoveto{\pgfqpoint{5.440000in}{0.720000in}}%
\pgfpathlineto{\pgfqpoint{5.760000in}{0.720000in}}%
\pgfpathlineto{\pgfqpoint{5.760000in}{1.025455in}}%
\pgfpathlineto{\pgfqpoint{5.440000in}{1.025455in}}%
\pgfpathlineto{\pgfqpoint{5.440000in}{0.720000in}}%
\pgfusepath{fill}%
\end{pgfscope}%
\begin{pgfscope}%
\pgfpathrectangle{\pgfqpoint{5.440000in}{0.720000in}}{\pgfqpoint{0.320000in}{3.360000in}} %
\pgfusepath{clip}%
\pgfsetbuttcap%
\pgfsetroundjoin%
\definecolor{currentfill}{rgb}{0.050383,0.029803,0.527975}%
\pgfsetfillcolor{currentfill}%
\pgfsetlinewidth{0.000000pt}%
\definecolor{currentstroke}{rgb}{0.000000,0.000000,0.000000}%
\pgfsetstrokecolor{currentstroke}%
\pgfsetdash{}{0pt}%
\pgfpathmoveto{\pgfqpoint{5.440000in}{1.025455in}}%
\pgfpathlineto{\pgfqpoint{5.760000in}{1.025455in}}%
\pgfpathlineto{\pgfqpoint{5.760000in}{1.330909in}}%
\pgfpathlineto{\pgfqpoint{5.440000in}{1.330909in}}%
\pgfpathlineto{\pgfqpoint{5.440000in}{1.025455in}}%
\pgfusepath{fill}%
\end{pgfscope}%
\begin{pgfscope}%
\pgfpathrectangle{\pgfqpoint{5.440000in}{0.720000in}}{\pgfqpoint{0.320000in}{3.360000in}} %
\pgfusepath{clip}%
\pgfsetbuttcap%
\pgfsetroundjoin%
\definecolor{currentfill}{rgb}{0.050383,0.029803,0.527975}%
\pgfsetfillcolor{currentfill}%
\pgfsetlinewidth{0.000000pt}%
\definecolor{currentstroke}{rgb}{0.000000,0.000000,0.000000}%
\pgfsetstrokecolor{currentstroke}%
\pgfsetdash{}{0pt}%
\pgfpathmoveto{\pgfqpoint{5.440000in}{1.330909in}}%
\pgfpathlineto{\pgfqpoint{5.760000in}{1.330909in}}%
\pgfpathlineto{\pgfqpoint{5.760000in}{1.636364in}}%
\pgfpathlineto{\pgfqpoint{5.440000in}{1.636364in}}%
\pgfpathlineto{\pgfqpoint{5.440000in}{1.330909in}}%
\pgfusepath{fill}%
\end{pgfscope}%
\begin{pgfscope}%
\pgfpathrectangle{\pgfqpoint{5.440000in}{0.720000in}}{\pgfqpoint{0.320000in}{3.360000in}} %
\pgfusepath{clip}%
\pgfsetbuttcap%
\pgfsetroundjoin%
\definecolor{currentfill}{rgb}{0.050383,0.029803,0.527975}%
\pgfsetfillcolor{currentfill}%
\pgfsetlinewidth{0.000000pt}%
\definecolor{currentstroke}{rgb}{0.000000,0.000000,0.000000}%
\pgfsetstrokecolor{currentstroke}%
\pgfsetdash{}{0pt}%
\pgfpathmoveto{\pgfqpoint{5.440000in}{1.636364in}}%
\pgfpathlineto{\pgfqpoint{5.760000in}{1.636364in}}%
\pgfpathlineto{\pgfqpoint{5.760000in}{1.941818in}}%
\pgfpathlineto{\pgfqpoint{5.440000in}{1.941818in}}%
\pgfpathlineto{\pgfqpoint{5.440000in}{1.636364in}}%
\pgfusepath{fill}%
\end{pgfscope}%
\begin{pgfscope}%
\pgfpathrectangle{\pgfqpoint{5.440000in}{0.720000in}}{\pgfqpoint{0.320000in}{3.360000in}} %
\pgfusepath{clip}%
\pgfsetbuttcap%
\pgfsetroundjoin%
\definecolor{currentfill}{rgb}{0.050383,0.029803,0.527975}%
\pgfsetfillcolor{currentfill}%
\pgfsetlinewidth{0.000000pt}%
\definecolor{currentstroke}{rgb}{0.000000,0.000000,0.000000}%
\pgfsetstrokecolor{currentstroke}%
\pgfsetdash{}{0pt}%
\pgfpathmoveto{\pgfqpoint{5.440000in}{1.941818in}}%
\pgfpathlineto{\pgfqpoint{5.760000in}{1.941818in}}%
\pgfpathlineto{\pgfqpoint{5.760000in}{2.247273in}}%
\pgfpathlineto{\pgfqpoint{5.440000in}{2.247273in}}%
\pgfpathlineto{\pgfqpoint{5.440000in}{1.941818in}}%
\pgfusepath{fill}%
\end{pgfscope}%
\begin{pgfscope}%
\pgfpathrectangle{\pgfqpoint{5.440000in}{0.720000in}}{\pgfqpoint{0.320000in}{3.360000in}} %
\pgfusepath{clip}%
\pgfsetbuttcap%
\pgfsetroundjoin%
\definecolor{currentfill}{rgb}{0.050383,0.029803,0.527975}%
\pgfsetfillcolor{currentfill}%
\pgfsetlinewidth{0.000000pt}%
\definecolor{currentstroke}{rgb}{0.000000,0.000000,0.000000}%
\pgfsetstrokecolor{currentstroke}%
\pgfsetdash{}{0pt}%
\pgfpathmoveto{\pgfqpoint{5.440000in}{2.247273in}}%
\pgfpathlineto{\pgfqpoint{5.760000in}{2.247273in}}%
\pgfpathlineto{\pgfqpoint{5.760000in}{2.552727in}}%
\pgfpathlineto{\pgfqpoint{5.440000in}{2.552727in}}%
\pgfpathlineto{\pgfqpoint{5.440000in}{2.247273in}}%
\pgfusepath{fill}%
\end{pgfscope}%
\begin{pgfscope}%
\pgfpathrectangle{\pgfqpoint{5.440000in}{0.720000in}}{\pgfqpoint{0.320000in}{3.360000in}} %
\pgfusepath{clip}%
\pgfsetbuttcap%
\pgfsetroundjoin%
\definecolor{currentfill}{rgb}{0.423689,0.000646,0.658956}%
\pgfsetfillcolor{currentfill}%
\pgfsetlinewidth{0.000000pt}%
\definecolor{currentstroke}{rgb}{0.000000,0.000000,0.000000}%
\pgfsetstrokecolor{currentstroke}%
\pgfsetdash{}{0pt}%
\pgfpathmoveto{\pgfqpoint{5.440000in}{2.552727in}}%
\pgfpathlineto{\pgfqpoint{5.760000in}{2.552727in}}%
\pgfpathlineto{\pgfqpoint{5.760000in}{2.858182in}}%
\pgfpathlineto{\pgfqpoint{5.440000in}{2.858182in}}%
\pgfpathlineto{\pgfqpoint{5.440000in}{2.552727in}}%
\pgfusepath{fill}%
\end{pgfscope}%
\begin{pgfscope}%
\pgfpathrectangle{\pgfqpoint{5.440000in}{0.720000in}}{\pgfqpoint{0.320000in}{3.360000in}} %
\pgfusepath{clip}%
\pgfsetbuttcap%
\pgfsetroundjoin%
\definecolor{currentfill}{rgb}{0.744232,0.218288,0.520524}%
\pgfsetfillcolor{currentfill}%
\pgfsetlinewidth{0.000000pt}%
\definecolor{currentstroke}{rgb}{0.000000,0.000000,0.000000}%
\pgfsetstrokecolor{currentstroke}%
\pgfsetdash{}{0pt}%
\pgfpathmoveto{\pgfqpoint{5.440000in}{2.858182in}}%
\pgfpathlineto{\pgfqpoint{5.760000in}{2.858182in}}%
\pgfpathlineto{\pgfqpoint{5.760000in}{3.163636in}}%
\pgfpathlineto{\pgfqpoint{5.440000in}{3.163636in}}%
\pgfpathlineto{\pgfqpoint{5.440000in}{2.858182in}}%
\pgfusepath{fill}%
\end{pgfscope}%
\begin{pgfscope}%
\pgfpathrectangle{\pgfqpoint{5.440000in}{0.720000in}}{\pgfqpoint{0.320000in}{3.360000in}} %
\pgfusepath{clip}%
\pgfsetbuttcap%
\pgfsetroundjoin%
\definecolor{currentfill}{rgb}{0.944844,0.507658,0.302433}%
\pgfsetfillcolor{currentfill}%
\pgfsetlinewidth{0.000000pt}%
\definecolor{currentstroke}{rgb}{0.000000,0.000000,0.000000}%
\pgfsetstrokecolor{currentstroke}%
\pgfsetdash{}{0pt}%
\pgfpathmoveto{\pgfqpoint{5.440000in}{3.163636in}}%
\pgfpathlineto{\pgfqpoint{5.760000in}{3.163636in}}%
\pgfpathlineto{\pgfqpoint{5.760000in}{3.469091in}}%
\pgfpathlineto{\pgfqpoint{5.440000in}{3.469091in}}%
\pgfpathlineto{\pgfqpoint{5.440000in}{3.163636in}}%
\pgfusepath{fill}%
\end{pgfscope}%
\begin{pgfscope}%
\pgfpathrectangle{\pgfqpoint{5.440000in}{0.720000in}}{\pgfqpoint{0.320000in}{3.360000in}} %
\pgfusepath{clip}%
\pgfsetbuttcap%
\pgfsetroundjoin%
\definecolor{currentfill}{rgb}{0.976265,0.868016,0.143351}%
\pgfsetfillcolor{currentfill}%
\pgfsetlinewidth{0.000000pt}%
\definecolor{currentstroke}{rgb}{0.000000,0.000000,0.000000}%
\pgfsetstrokecolor{currentstroke}%
\pgfsetdash{}{0pt}%
\pgfpathmoveto{\pgfqpoint{5.440000in}{3.469091in}}%
\pgfpathlineto{\pgfqpoint{5.760000in}{3.469091in}}%
\pgfpathlineto{\pgfqpoint{5.760000in}{3.774545in}}%
\pgfpathlineto{\pgfqpoint{5.440000in}{3.774545in}}%
\pgfpathlineto{\pgfqpoint{5.440000in}{3.469091in}}%
\pgfusepath{fill}%
\end{pgfscope}%
\begin{pgfscope}%
\pgfpathrectangle{\pgfqpoint{5.440000in}{0.720000in}}{\pgfqpoint{0.320000in}{3.360000in}} %
\pgfusepath{clip}%
\pgfsetbuttcap%
\pgfsetroundjoin%
\definecolor{currentfill}{rgb}{0.940015,0.975158,0.131326}%
\pgfsetfillcolor{currentfill}%
\pgfsetlinewidth{0.000000pt}%
\definecolor{currentstroke}{rgb}{0.000000,0.000000,0.000000}%
\pgfsetstrokecolor{currentstroke}%
\pgfsetdash{}{0pt}%
\pgfpathmoveto{\pgfqpoint{5.440000in}{3.774545in}}%
\pgfpathlineto{\pgfqpoint{5.760000in}{3.774545in}}%
\pgfpathlineto{\pgfqpoint{5.760000in}{4.080000in}}%
\pgfpathlineto{\pgfqpoint{5.440000in}{4.080000in}}%
\pgfpathlineto{\pgfqpoint{5.440000in}{3.774545in}}%
\pgfusepath{fill}%
\end{pgfscope}%
\begin{pgfscope}%
\pgfsetbuttcap%
\pgfsetroundjoin%
\definecolor{currentfill}{rgb}{0.000000,0.000000,0.000000}%
\pgfsetfillcolor{currentfill}%
\pgfsetlinewidth{0.803000pt}%
\definecolor{currentstroke}{rgb}{0.000000,0.000000,0.000000}%
\pgfsetstrokecolor{currentstroke}%
\pgfsetdash{}{0pt}%
\pgfsys@defobject{currentmarker}{\pgfqpoint{0.000000in}{0.000000in}}{\pgfqpoint{0.048611in}{0.000000in}}{%
\pgfpathmoveto{\pgfqpoint{0.000000in}{0.000000in}}%
\pgfpathlineto{\pgfqpoint{0.048611in}{0.000000in}}%
\pgfusepath{stroke,fill}%
}%
\begin{pgfscope}%
\pgfsys@transformshift{5.760000in}{0.720000in}%
\pgfsys@useobject{currentmarker}{}%
\end{pgfscope}%
\end{pgfscope}%
\begin{pgfscope}%
\pgftext[x=5.857222in,y=0.667238in,left,base]{\sffamily\fontsize{10.000000}{12.000000}\selectfont \(\displaystyle {10^{-4}}\)}%
\end{pgfscope}%
\begin{pgfscope}%
\pgfsetbuttcap%
\pgfsetroundjoin%
\definecolor{currentfill}{rgb}{0.000000,0.000000,0.000000}%
\pgfsetfillcolor{currentfill}%
\pgfsetlinewidth{0.803000pt}%
\definecolor{currentstroke}{rgb}{0.000000,0.000000,0.000000}%
\pgfsetstrokecolor{currentstroke}%
\pgfsetdash{}{0pt}%
\pgfsys@defobject{currentmarker}{\pgfqpoint{0.000000in}{0.000000in}}{\pgfqpoint{0.048611in}{0.000000in}}{%
\pgfpathmoveto{\pgfqpoint{0.000000in}{0.000000in}}%
\pgfpathlineto{\pgfqpoint{0.048611in}{0.000000in}}%
\pgfusepath{stroke,fill}%
}%
\begin{pgfscope}%
\pgfsys@transformshift{5.760000in}{1.330909in}%
\pgfsys@useobject{currentmarker}{}%
\end{pgfscope}%
\end{pgfscope}%
\begin{pgfscope}%
\pgftext[x=5.857222in,y=1.278148in,left,base]{\sffamily\fontsize{10.000000}{12.000000}\selectfont \(\displaystyle {10^{-2}}\)}%
\end{pgfscope}%
\begin{pgfscope}%
\pgfsetbuttcap%
\pgfsetroundjoin%
\definecolor{currentfill}{rgb}{0.000000,0.000000,0.000000}%
\pgfsetfillcolor{currentfill}%
\pgfsetlinewidth{0.803000pt}%
\definecolor{currentstroke}{rgb}{0.000000,0.000000,0.000000}%
\pgfsetstrokecolor{currentstroke}%
\pgfsetdash{}{0pt}%
\pgfsys@defobject{currentmarker}{\pgfqpoint{0.000000in}{0.000000in}}{\pgfqpoint{0.048611in}{0.000000in}}{%
\pgfpathmoveto{\pgfqpoint{0.000000in}{0.000000in}}%
\pgfpathlineto{\pgfqpoint{0.048611in}{0.000000in}}%
\pgfusepath{stroke,fill}%
}%
\begin{pgfscope}%
\pgfsys@transformshift{5.760000in}{1.941818in}%
\pgfsys@useobject{currentmarker}{}%
\end{pgfscope}%
\end{pgfscope}%
\begin{pgfscope}%
\pgftext[x=5.857222in,y=1.889057in,left,base]{\sffamily\fontsize{10.000000}{12.000000}\selectfont \(\displaystyle {10^{0}}\)}%
\end{pgfscope}%
\begin{pgfscope}%
\pgfsetbuttcap%
\pgfsetroundjoin%
\definecolor{currentfill}{rgb}{0.000000,0.000000,0.000000}%
\pgfsetfillcolor{currentfill}%
\pgfsetlinewidth{0.803000pt}%
\definecolor{currentstroke}{rgb}{0.000000,0.000000,0.000000}%
\pgfsetstrokecolor{currentstroke}%
\pgfsetdash{}{0pt}%
\pgfsys@defobject{currentmarker}{\pgfqpoint{0.000000in}{0.000000in}}{\pgfqpoint{0.048611in}{0.000000in}}{%
\pgfpathmoveto{\pgfqpoint{0.000000in}{0.000000in}}%
\pgfpathlineto{\pgfqpoint{0.048611in}{0.000000in}}%
\pgfusepath{stroke,fill}%
}%
\begin{pgfscope}%
\pgfsys@transformshift{5.760000in}{2.552727in}%
\pgfsys@useobject{currentmarker}{}%
\end{pgfscope}%
\end{pgfscope}%
\begin{pgfscope}%
\pgftext[x=5.857222in,y=2.499966in,left,base]{\sffamily\fontsize{10.000000}{12.000000}\selectfont \(\displaystyle {10^{2}}\)}%
\end{pgfscope}%
\begin{pgfscope}%
\pgfsetbuttcap%
\pgfsetroundjoin%
\definecolor{currentfill}{rgb}{0.000000,0.000000,0.000000}%
\pgfsetfillcolor{currentfill}%
\pgfsetlinewidth{0.803000pt}%
\definecolor{currentstroke}{rgb}{0.000000,0.000000,0.000000}%
\pgfsetstrokecolor{currentstroke}%
\pgfsetdash{}{0pt}%
\pgfsys@defobject{currentmarker}{\pgfqpoint{0.000000in}{0.000000in}}{\pgfqpoint{0.048611in}{0.000000in}}{%
\pgfpathmoveto{\pgfqpoint{0.000000in}{0.000000in}}%
\pgfpathlineto{\pgfqpoint{0.048611in}{0.000000in}}%
\pgfusepath{stroke,fill}%
}%
\begin{pgfscope}%
\pgfsys@transformshift{5.760000in}{3.163636in}%
\pgfsys@useobject{currentmarker}{}%
\end{pgfscope}%
\end{pgfscope}%
\begin{pgfscope}%
\pgftext[x=5.857222in,y=3.110875in,left,base]{\sffamily\fontsize{10.000000}{12.000000}\selectfont \(\displaystyle {10^{4}}\)}%
\end{pgfscope}%
\begin{pgfscope}%
\pgfsetbuttcap%
\pgfsetroundjoin%
\definecolor{currentfill}{rgb}{0.000000,0.000000,0.000000}%
\pgfsetfillcolor{currentfill}%
\pgfsetlinewidth{0.803000pt}%
\definecolor{currentstroke}{rgb}{0.000000,0.000000,0.000000}%
\pgfsetstrokecolor{currentstroke}%
\pgfsetdash{}{0pt}%
\pgfsys@defobject{currentmarker}{\pgfqpoint{0.000000in}{0.000000in}}{\pgfqpoint{0.048611in}{0.000000in}}{%
\pgfpathmoveto{\pgfqpoint{0.000000in}{0.000000in}}%
\pgfpathlineto{\pgfqpoint{0.048611in}{0.000000in}}%
\pgfusepath{stroke,fill}%
}%
\begin{pgfscope}%
\pgfsys@transformshift{5.760000in}{3.774545in}%
\pgfsys@useobject{currentmarker}{}%
\end{pgfscope}%
\end{pgfscope}%
\begin{pgfscope}%
\pgftext[x=5.857222in,y=3.721784in,left,base]{\sffamily\fontsize{10.000000}{12.000000}\selectfont \(\displaystyle {10^{6}}\)}%
\end{pgfscope}%
\begin{pgfscope}%
\pgftext[x=6.200780in,y=2.400000in,,top,rotate=90.000000]{\sffamily\fontsize{10.000000}{12.000000}\selectfont Error}%
\end{pgfscope}%
\begin{pgfscope}%
\pgfsetbuttcap%
\pgfsetmiterjoin%
\pgfsetlinewidth{0.803000pt}%
\definecolor{currentstroke}{rgb}{0.000000,0.000000,0.000000}%
\pgfsetstrokecolor{currentstroke}%
\pgfsetdash{}{0pt}%
\pgfpathmoveto{\pgfqpoint{5.440000in}{0.720000in}}%
\pgfpathlineto{\pgfqpoint{5.440000in}{1.025455in}}%
\pgfpathlineto{\pgfqpoint{5.440000in}{3.774545in}}%
\pgfpathlineto{\pgfqpoint{5.440000in}{4.080000in}}%
\pgfpathlineto{\pgfqpoint{5.760000in}{4.080000in}}%
\pgfpathlineto{\pgfqpoint{5.760000in}{3.774545in}}%
\pgfpathlineto{\pgfqpoint{5.760000in}{1.025455in}}%
\pgfpathlineto{\pgfqpoint{5.760000in}{0.720000in}}%
\pgfpathclose%
\pgfusepath{stroke}%
\end{pgfscope}%
\end{pgfpicture}%
\makeatother%
\endgroup%
}
		\end{center}
		\caption[]{The effect of momentum on SGD, applied to fitting a line to a noisy two-dimensional dataset with $\eta = 0.1$. \textbf{Left:} Standard SGD. Near the optimum, the algorithm begins zig-zagging and takes many steps to converge. \textbf{Right:} SGD with momentum and $\gamma = 0.85$. Although the algorithm overshoots the minimum a few times, the number of steps until convergence is much smaller.}
		\label{fig:momentum}
	\end {figure}


	\subsubsection {Learning Rate Decay}
Another optimization that is often used is \textit{Learning Rate decay}, also called \textit{Learning Rate Annealing}. It exchanges the fixed learning rate $\eta$ with a learning rate that is a function of the iteration number so that based on how long the optimization has been running, the step size becomes smaller. This has the benefit of being able to take large steps towards the minimum initially, but to allow the search to be narrowed down when near the minimum and avoid endless overshooting. The most popular of these is the step decay, which reduces the learning rate by some factor $\zeta$ every $x$ iterations. Let $i$ be a variable counting the number of iterations. The learning rate in an iteration is then given by

\[ \eta_i = \eta_{i-1} * {\zeta}^{\left \lfloor \frac{i}{x} \right \rfloor} \]

\noindent The SGD update then works as normal, only that after each iteration, $\eta$ is calculated anew given the above formula.




	\subsection{The Perceptron Algorithm}
\label{subsec:perceptron_algo}

A single-layer ANN is a network consisting of just one neuron, and is also called ``Perceptron''. The idea for the Perceptron was first proposed by Rosenblatt \cite{rosenblatt_report, rosenblatt_book}, who used it to perform binary classifications, using what is now called the ``Perceptron algorithm''. Given a linearly separable set of $m$ data points of dimension $n$, one can construct a $(n+1) \times m$ input matrix $x$. The last entry in every column is set to $1$ to simplify the notation later on.

The \textit{Perceptron classifier}, the function that decides which class a given sample belongs to, then takes the form of a linear function

\[ d(x) = h(w^{T}x), \label{eq:perc_class} \]

\noindent where $w$ is a vector of dimension $n+1$ and contains the $n$ \textit{weights} and the \textit{bias} value. The function $h$ is the \textit{activation function}, which is defined as 

\[ h(a) =  \begin{cases}
		+1 \text{ if } a \geq 0 \\
	   	 -1 \text{ otherwise}
	    \end{cases}\]

\noindent and maps its input to one of two possible classes.\\

\noindent The task of \textit{training} is defined as finding values for $w_0, \dots, w_{n+1}$ so that each data point is classified correctly by the resulting line, plane or (hyper-)plane. To this account, there needs to be a $m$-dimensional vector $t$ containing the \textit{labels} of each data point to verify the correctness of $w$. These labels are integers that indicate which class each data point belongs to. In the case of the Perceptron algorithm, the labels are either $1$ for the first class $C_1$ or $-1$ for the second class, $C_2$. If all data points are classified correctly by some $w$, then it holds that

\[ w^T x_n \, t_n > 0 \,\,\forall\, x_n \in x \label{eq:perceptron_label} \]

\noindent Using this definition of correctness, a correctness measure can be formulated, which is known as the ``Perceptron criterion''. Functions of this kind are also called \textit{error functions} or \textit{loss functions}. This particular loss function is given by

\[ E_p(w) = - \sum \limits_{n \in \mathcal{M}} w^T x_n\, t_n, \label{eq:perc_error}\]

\noindent where $\mathcal{M}$ is the set of all data points that were misclassfied, meaning that evaluating equation \ref{eq:perceptron_label} for that data yields a value $< 0$. \cite[pp. 192--194]{bishop_pattern}\\

\noindent The correct Perceptron weights can then be determined by some arbitrary learning algorithm, although here, the Gradient Descent scheme (see algorithm \ref{alg:grad_desc}) is applied to the error function \ref{eq:perc_error}. Either (Batch) Gradient Descent or Stochastic Gradient Descent can be used.

Either way, the gradient of $E_p$ has to be calculated. The Perceptron criterion only depends on the weights $w$, so the gradient vector has the shape

\[ \nabla_E = \left[ \frac{\partial E_p}{\partial w_0}, \frac{\partial E_p}{\partial w_1}, \dots, \frac{\partial E_p}{\partial w_{n+1}} \right ] \,. \]

\noindent Taking the partial derivatives in respect to $w$ leads to

\begin{align}
 	\nabla_E &= \frac{\partial E_p}{\partial w} = \frac{\partial}{\partial w} \left (- \sum \limits_{n \in \mathcal{M}} w^T x_n t_n \right ) \\
 	&= -\sum \limits_{n \in \mathcal{M}} x_n t_n
\end{align}

\noindent If SGD is used, this is reduced to simply

\[ \nabla_E = -x_n t_n\]

\noindent for some previously misclassified point $x_n \in \mathcal{M}$ in the dataset. Hence, the update rule for the Perceptron becomes

\[w_i = w_{i - 1} + x_n t_n * \eta\]

\noindent Using this update rule leads to the training algorithm shown in \ref{alg:perceptron_algorithm}, which is applied to an example dataset in figure \ref{fig:perceptron}.

\begin {figure}[!ht]
	\begin{center}
		\includegraphics[scale=0.6]{img/fig_perceptron}
	\end{center}
	\caption{\textbf{Left}: Separating two datasets with the Perceptron algorithm in $\mathbb{R}^2$. The correct classification boundary takes the form of a line, i.e. $0 = w^{T}x = w_0 x + w_1 y + w_2$. The line pictured is given by $w = (-2.285, -2.908, 5.938)$, the weights after 19 iterations. \textbf{Right}: A not-linearly separable data set, on which the algorithm would never converge, because it is based on finding linear classifiers and thus cannot separate the data correctly.}
	\label{fig:perceptron}
\end {figure}

\begin {algorithm}
	\begin {algorithmic}[1]
		\State $w$ = random
		\While {true}
			\For{$x_i$ in $x$}
				\If{$w^T x_i t_i \leq 0$}
					\State $w$ += $x_i t_i * \eta$
				\EndIf

				\If{$w^T x_i t_i > 0$ for all $x_i$ in $x$}
					\State stop
				\EndIf
			\EndFor
		\EndWhile
	\end{algorithmic}
	\caption{Stochastic Gradient Descent applied to the task of finding the Perceptron weights $w$. $x$ is assumed to be linearly separable.}
	\label{alg:perceptron_algorithm}
\end{algorithm}



		\subsection{Multiple Neurons and Backpropagation}
\label{subsec:mlp_backprop}
While the Perceptron algorithm works fine for linear problems, those are only a small subset of real-world problems (see Figure \ref{fig:perceptron}). The solution to this problem is to add more neurons to the model, which form a network that consists of layers that pass the outputs of neurons in the previous layers through non-linear activation functions and then use the results as inputs. These layers are referred to as the ``input layer'', the initial layer, the ``output layer'', the last layer, and ``hidden layers'', which are all the layers inbetween.

	\subsubsection{Multi-Layer Perceptrons}
The resulting networks are called \textit{Multi-Layer Perceptrons} (MLPs), \textit{Feedforward Artificial Neural Networks}(ANNs), or, in the case of a multitude of layers, \textit{Deep Neural Networks}, and are much more powerful than simple Perceptrons, as they can approximate haphazard continuous functions that operate on closed and bounded subsets of $\mathbb{R}^n$ given that the network possesses at least one hidden layer of neurons. This capability makes using MLPs for learning real-life problems feasible. \cite{universal_approx}\cite{universal_approx2}\\

\noindent Again, each neuron in the network computes a weighted sum of its $i$ inputs, just like in Equation \textbf{\ref{eq:perc_class}}, although indexing has been added to identify neurons and weights in the entire network:

\[ a_j^{(l)} = \sum \limits_{i} w^{(l)}_{ji}\,\, z_i^{(l-1)} + b^{(l)}_{j} \]

\noindent Here, $z_i$ are the input values to the current neuron. Note that these either are the direct inputs $x_i$ to the network, or outputs of previous neurons, depending on where in the network the term is evaluated. $w$ is the matrix of weights, and $w^{(l)}_{ji}$ denotes the weight of the connection from the $i$th neuron in the previous layer $l - 1$ to the $j$th neuron in the  layer $l$. The term $b$ is the bias weight associated with the neuron $j$. The bias can also be made part of the weight matrix by introducing an additional input dimension with a value of $1$, like it was done for the Perceptron algorithm in Section \textbf{\ref{subsec:perceptron_algo}}.

Afterwards, the sum $a^{(l)}_j$ is passed into an activation function $h(\cdot)$ to produce the output, also called activation, of the $j$th neuron in the $l$th layer, $z_j^{(l)}$

\[ z_j^{(l)} = h(a^{(l)}_j) \]

\noindent The activations of the input layer are simply the input values $x_1 \dots x_n$. Expanding the term for the 2-layer network shown in Figure \textbf{\ref{fig:mlp}} gives the following formula for the $k$th network output $out_k$:

\[ out_k = \sigma \left ( \sum \limits_{j=0}^{4} w^{(2)}_{kj}\,\, h \left ( \sum \limits_{i=0}^{3} w^{(1)}_{ji} x_i \right ) \right ) \label{eq:mlp_out} \]

\noindent Here, \textbf{$\sigma$} is an output function that is applied in the last layer of the network, and is dependent on the task the network is supposed to fulfill: For regression problems, i.e. fitting a function to a set of data and then predicting new values using that function, the identity function can be used. For binary classification problems, instead of using a step function like in the original Perceptron, one instead uses the \textit{Sigmoid} function and interprets the output as

\[ \sigma_{sig}(x_i) = \begin{cases}
				x_i \in C_1 \text{ if } \frac{1}{1 + e^{-x_i}} \geq 0.5 \\
				x_i \in C_2 \text{ else }\\
			 \end{cases}
\]

\noindent Also, the choice of the activation function $h(w, x)$ is different from the Perceptron: It is not chosen to be the Heaviside step function, but instead, to be a non-linear, continuous and differentiable function, such as the general Sigmoid function and its derivative:

\begin {align}
	 h(x) &= \frac{1}{1 + e^{-x}}\\
	\frac{\partial h}{\partial x} &= h(x)(1 - h(x))
\end {align}

\afterpage{
	\begin {figure}[!ht]
		\begin{center}
			\def\svgwidth{0.8\columnwidth}
			\input{img/mlp.pdf_tex}
		\end{center}
		\caption[]{An MLP with one hidden layer, forming an acyclic directed graph, where connections are color-coded per layer by the receiving neuron. The layer count $L$ starts at the layer after the input layer\protect\footnotemark. Therefore $L = 2$, with three inputs $x_i$ and neuron counts $D^{(1)} = 4$ and $D^{(2)} = 2$. The variables $b^{(1)}$ and $b^{(2)}$ refer to the biases for the neurons in the layers $(1)$ or $(2)$, respectively. Each neuron has its own, changeable bias value, but for the sake of cleanness, the connections from the biases to each neuron of their corresponding layer are not shown.}
		\label{fig:mlp}
	\end {figure}

	\footnotetext{There is no consensus on which method of counting is the most appropriate. Sometimes, only the number of hidden layers or even all layers are counted. \cite[pp. 229]{bishop_pattern}}
}

\noindent The goal of training an MLP is - as previously for the Perceptron - finding specific weights so that all inputs are classified correctly, or at least, so that the network makes only few mistakes. Therefore, a loss function is added to the end of the network, measuring the correctness of the output. Instead of the Perceptron criterion \ref{eq:perc_error}, a more general loss function is employed which compares the difference between the output $\sigma(x)$ of the network for a sample $x$ and the preferred output for $x$, the \textit{label} or \textit{ground truth} of $x$. Typically, in binary classification problems, Sigmoid outputs are used together with the \textit{Cross-Entropy} loss function, which is defined as 

\[ CE(w) =  -t \, \ln (out_x) + (1 - t) \ln (1 - out_x).\footnote{The error is only given as a function of $w$ because both the input $x$ and its correct label $t$ are fixed in the particular case presented here.}  \]

\noindent For multi-class classifications, an adaption of the Cross-Entropy loss can be used, which is further described in Section \ref{subsec:cross_ent}. \cite[pp. 232-236]{bishop_pattern}\\

	\subsubsection{The Backpropagation algorithm}
Training a network with multiple layers is harder than training a single Perceptron because the influence of each of the weights in any layer of the network on the error function has to be computed when using gradient-based optimization methods, while the output of the network no longer is the result of a single function, but instead, a composition of an usually large number of functions (see Equation \textbf{\ref{eq:mlp_out}}). Although it is possible to manually calculate the partial derivatives for the entire network, this becomes more and more cumbersome as the network grows in size.

Luckily, there exists an iterative algorithm called \textit{Backpropagation} \cite{backprop} that finds the gradient of the loss with respect to each weight so that an optimization method can be applied to change every weight. The algorithm makes use of the fact that each neuron can be viewed (almost) independently from the rest of the network. Remembering that the analytic way of obtaining the derivative of a composition of functions 

\[ f \circ g = f(g(x)) \]

\noindent with respect to some $x$ is to apply the chain rule

\[ \frac{\partial f}{\partial x} =\frac{\partial f}{\partial g} \frac{\partial g}{\partial x} \, , \]

\noindent one can see that this concept can also be applied to neurons in a network, because using the output value of a neuron in a previous layer as the input for a node is nothing else than function composition.\\

\noindent Backpropagation thus starts with a \textit{forward pass}, which means that an input vector $x$ is passed through the network and the corresponding sums $a_j$ and activations $z_j$ are calculated and stored. Comparing the final outputs to the ground truth in the loss layer, the network is then traversed in the opposite direction in what is called a \textit{backward pass}, aiming to obtain the influence of each weight in each layer on the final loss, that is, the gradient of the loss function with respect to the weights. After this first phase, the partial derivatives for each weight are known, and the weights are updated in a second phase, in which an optimization algorithm like Gradient Descent is used as described in Section \textbf{\ref{subsec:grad_desc}}.\\

\noindent The central idea of the backward pass is that when traversing the network from the output layer to the input layer, the derivatives of the loss function with respect to the weights of the layer closest to the output of the network depend on outputs of the neurons in the layers that precede that layer. This relationship recursively continues backwards throughout the network and is described by the formula

\begin {align}
	\frac{\partial E}{\partial w_{ji}} &= \frac{\partial E}{\partial a_j} \frac{\partial a_j}{\partial w_{ji}} \label{eq:backprop_recur}
\end {align}

\noindent where $E$ is the loss function of the network.\footnote{Because the formula concerns a particular weight $w_{ji}^{(l)}$ in a particular layer $l$, the $l$-superscript is omitted for convenience.} According to the chain rule, the influence of the weight $w_{ji}$ on the loss is equal to the influence of the weighted sum $a_j$ on the loss function multiplied by the influence of the weight $w_{ji}$ on $a_j$. Taking the partial derivative of the weighted sum $a_j$ with respect to the weight $w_{ji}$ means that

\begin {align}
	\frac{\partial a_j}{\partial w_{ji}} &= \frac{\partial}{\partial w_{ji}} \bigg ( \sum \limits_{k} w_{jk} z_k \bigg )
\end {align}

\noindent where $k$ is the number of neurons in the neurons in the previous layer $l-1$. Because when $k = i$, it holds that

\[ \frac{\partial}{\partial w_{ji}} \bigg ( w_{jk} z_k \bigg ) = z_i \]

\noindent and for $k \neq i$ it holds that 

\[ \frac{\partial}{\partial w_{ji}} \bigg ( w_{jk} z_k \bigg ) = 0 \]

\noindent The derivative of the entire sum is therefore

\[ \frac{\partial}{\partial w_{ji}} \bigg ( \sum \limits_{k} w_{jk} z_k \bigg ) = z_i \]

\noindent In literature, the term $\frac{\partial E}{\partial a_j}$ is often denoted as $\delta_j^{(l)}$, the ``error'' of the neuron $j$ in layer $l$. Using this notation, Equation \textbf{\ref{eq:backprop_recur}} can then be written as

\[ \frac{\partial E}{\partial w_{ji}} = \delta_j^{(l)} z_i \]

\noindent Since $z_i$, the activation from the input end of the connection that is multiplied with $w_{ji}$, is already known for all neurons due to being stored during the forward pass, the only thing that needs to be computed to figure out the loss gradient are the $\delta$-values. The computation of $\delta_j^{(l)}$ depends on the position of the neuron in the network:

\[ \delta_j^{(l)} =  \begin{cases}
				\frac{\partial E}{\partial a_j} \text{ if neuron } j^{(l)} \text{ is an output neuron} \\
			           h'(a_j) \sum \limits_{k} w_{kj} \, \delta_k^{(l+1)} \text{ otherwise}
			     \end{cases}
\]

\noindent In the case that $j$ is an output neuron, the partial derivative is calculated directly. Otherwise, $\delta_j$ depends on the $\delta$-values in deeper layers (see Figure \textbf{\ref{fig:backprop}}). The definition of $\delta$ for hidden neurons is also the reason why a differentiable function must be chosen as the activation $h(\cdot)$. The Backpropagation algorithm continues backwards through the network until the entire weight gradient is known and a single (Stochastic) Gradient Descent step can be taken towards the loss function minimum.

\noindent The main advantage of the algorithm is that it scales linearly with the number of weights, i.e. the algorithm is in $\mathcal{O}(w)$. Furthermore, the fact that all $\delta^{(l)}$-values of a layer have to be calculated to calculate all $\delta^{(l-1)}$-values in the layer that precedes it naturally leads to vectorizing the algorithm layerwise in efficient implementations. \cite{nielsen_book, ng_lecture}\cite[pp. 241-245]{bishop_pattern}


\afterpage{
	\begin {figure}[!ht]
		\begin{center}
			\def\svgwidth{0.8\columnwidth}
			\input{img/backprop.pdf_tex}
		\end{center}
		\caption[]{Backpropagation calculations, applied to a hidden layer $l$.\protect\footnotemark$\,$ Operations during the forward pass are shown in blue, while operations during the backward pass are orange. The central neuron is the only neuron in the layer $l$, while there are two neurons each in the hidden layers $l-1$ and $l+1$. The forward pass calculates the weighted sum $a_1^{(l)}$ and the activation $h(a_1^{(l)})$. Once the backward pass reaches the neuron again, it calculates $\delta_1^{(l)}$ and the partial derivatives of the weights of layer $l$, that is, the weights of the connections that flow into the central neuron in the forward pass, using the backpropagated value $\delta_k^{(l+1)}$.}
		\label{fig:backprop}
	\end {figure}

	\footnotetext{This example is based on a visualization in \cite{karpathy_lecture}.}
}

\noindent Depending on how deep the networks are, i.e. how many hidden layers there are and how many neurons there are in these hidden layers, neural networks can learn very complex classification boundaries. The role of the activation function is akin to the kernel principle of State Vector Machines (SVMs), since it maps inputs to a representation in which they are linearly separable by a hyperplane (see Figure \textbf{\ref{fig:mlp_trick}}).\\

\begin {figure}[!ht]
		\begin{center}
			\textbf{TODO: uncomment in code}
			%%% Creator: Matplotlib, PGF backend
%%
%% To include the figure in your LaTeX document, write
%%   \input{<filename>.pgf}
%%
%% Make sure the required packages are loaded in your preamble
%%   \usepackage{pgf}
%%
%% Figures using additional raster images can only be included by \input if
%% they are in the same directory as the main LaTeX file. For loading figures
%% from other directories you can use the `import` package
%%   \usepackage{import}
%% and then include the figures with
%%   \import{<path to file>}{<filename>.pgf}
%%
%% Matplotlib used the following preamble
%%   \usepackage{fontspec}
%%   \setmainfont{DejaVu Serif}
%%   \setsansfont{DejaVu Sans}
%%   \setmonofont{DejaVu Sans Mono}
%%
\begingroup%
\makeatletter%
\begin{pgfpicture}%
\pgfpathrectangle{\pgfpointorigin}{\pgfqpoint{7.110000in}{3.810000in}}%
\pgfusepath{use as bounding box, clip}%
\begin{pgfscope}%
\pgfsetbuttcap%
\pgfsetmiterjoin%
\definecolor{currentfill}{rgb}{1.000000,1.000000,1.000000}%
\pgfsetfillcolor{currentfill}%
\pgfsetlinewidth{0.000000pt}%
\definecolor{currentstroke}{rgb}{1.000000,1.000000,1.000000}%
\pgfsetstrokecolor{currentstroke}%
\pgfsetdash{}{0pt}%
\pgfpathmoveto{\pgfqpoint{0.000000in}{0.000000in}}%
\pgfpathlineto{\pgfqpoint{7.110000in}{0.000000in}}%
\pgfpathlineto{\pgfqpoint{7.110000in}{3.810000in}}%
\pgfpathlineto{\pgfqpoint{0.000000in}{3.810000in}}%
\pgfpathclose%
\pgfusepath{fill}%
\end{pgfscope}%
\begin{pgfscope}%
\pgfsetbuttcap%
\pgfsetmiterjoin%
\definecolor{currentfill}{rgb}{1.000000,1.000000,1.000000}%
\pgfsetfillcolor{currentfill}%
\pgfsetlinewidth{0.000000pt}%
\definecolor{currentstroke}{rgb}{0.000000,0.000000,0.000000}%
\pgfsetstrokecolor{currentstroke}%
\pgfsetstrokeopacity{0.000000}%
\pgfsetdash{}{0pt}%
\pgfpathmoveto{\pgfqpoint{0.888750in}{0.419100in}}%
\pgfpathlineto{\pgfqpoint{3.393409in}{0.419100in}}%
\pgfpathlineto{\pgfqpoint{3.393409in}{3.352800in}}%
\pgfpathlineto{\pgfqpoint{0.888750in}{3.352800in}}%
\pgfpathclose%
\pgfusepath{fill}%
\end{pgfscope}%
\begin{pgfscope}%
\pgfpathrectangle{\pgfqpoint{0.888750in}{0.419100in}}{\pgfqpoint{2.504659in}{2.933700in}} %
\pgfusepath{clip}%
\pgfsetbuttcap%
\pgfsetroundjoin%
\definecolor{currentfill}{rgb}{0.827465,0.000000,0.000000}%
\pgfsetfillcolor{currentfill}%
\pgfsetfillopacity{0.300000}%
\pgfsetlinewidth{0.000000pt}%
\definecolor{currentstroke}{rgb}{0.000000,0.000000,0.000000}%
\pgfsetstrokecolor{currentstroke}%
\pgfsetdash{}{0pt}%
\pgfpathmoveto{\pgfqpoint{2.173994in}{0.898734in}}%
\pgfpathlineto{\pgfqpoint{2.177129in}{0.898734in}}%
\pgfpathlineto{\pgfqpoint{2.180264in}{0.898734in}}%
\pgfpathlineto{\pgfqpoint{2.180734in}{0.899296in}}%
\pgfpathlineto{\pgfqpoint{2.183399in}{0.902485in}}%
\pgfpathlineto{\pgfqpoint{2.186533in}{0.902485in}}%
\pgfpathlineto{\pgfqpoint{2.189668in}{0.902485in}}%
\pgfpathlineto{\pgfqpoint{2.190138in}{0.903048in}}%
\pgfpathlineto{\pgfqpoint{2.192803in}{0.906237in}}%
\pgfpathlineto{\pgfqpoint{2.195938in}{0.906237in}}%
\pgfpathlineto{\pgfqpoint{2.196408in}{0.906799in}}%
\pgfpathlineto{\pgfqpoint{2.199072in}{0.909988in}}%
\pgfpathlineto{\pgfqpoint{2.202207in}{0.909988in}}%
\pgfpathlineto{\pgfqpoint{2.205342in}{0.909988in}}%
\pgfpathlineto{\pgfqpoint{2.205812in}{0.910551in}}%
\pgfpathlineto{\pgfqpoint{2.208477in}{0.913740in}}%
\pgfpathlineto{\pgfqpoint{2.211611in}{0.913740in}}%
\pgfpathlineto{\pgfqpoint{2.214746in}{0.913740in}}%
\pgfpathlineto{\pgfqpoint{2.215216in}{0.914303in}}%
\pgfpathlineto{\pgfqpoint{2.217881in}{0.917491in}}%
\pgfpathlineto{\pgfqpoint{2.221015in}{0.917491in}}%
\pgfpathlineto{\pgfqpoint{2.224150in}{0.917491in}}%
\pgfpathlineto{\pgfqpoint{2.224620in}{0.918054in}}%
\pgfpathlineto{\pgfqpoint{2.227285in}{0.921243in}}%
\pgfpathlineto{\pgfqpoint{2.230420in}{0.921243in}}%
\pgfpathlineto{\pgfqpoint{2.230890in}{0.921806in}}%
\pgfpathlineto{\pgfqpoint{2.233554in}{0.924994in}}%
\pgfpathlineto{\pgfqpoint{2.236689in}{0.924994in}}%
\pgfpathlineto{\pgfqpoint{2.239824in}{0.924994in}}%
\pgfpathlineto{\pgfqpoint{2.240294in}{0.925557in}}%
\pgfpathlineto{\pgfqpoint{2.242959in}{0.928746in}}%
\pgfpathlineto{\pgfqpoint{2.246093in}{0.928746in}}%
\pgfpathlineto{\pgfqpoint{2.249228in}{0.928746in}}%
\pgfpathlineto{\pgfqpoint{2.249698in}{0.929309in}}%
\pgfpathlineto{\pgfqpoint{2.252363in}{0.932498in}}%
\pgfpathlineto{\pgfqpoint{2.255498in}{0.932498in}}%
\pgfpathlineto{\pgfqpoint{2.258632in}{0.932498in}}%
\pgfpathlineto{\pgfqpoint{2.259103in}{0.933060in}}%
\pgfpathlineto{\pgfqpoint{2.261767in}{0.936249in}}%
\pgfpathlineto{\pgfqpoint{2.264902in}{0.936249in}}%
\pgfpathlineto{\pgfqpoint{2.268037in}{0.936249in}}%
\pgfpathlineto{\pgfqpoint{2.268507in}{0.936812in}}%
\pgfpathlineto{\pgfqpoint{2.271171in}{0.940001in}}%
\pgfpathlineto{\pgfqpoint{2.274306in}{0.940001in}}%
\pgfpathlineto{\pgfqpoint{2.274776in}{0.940563in}}%
\pgfpathlineto{\pgfqpoint{2.277441in}{0.943752in}}%
\pgfpathlineto{\pgfqpoint{2.280576in}{0.943752in}}%
\pgfpathlineto{\pgfqpoint{2.283710in}{0.943752in}}%
\pgfpathlineto{\pgfqpoint{2.284181in}{0.944315in}}%
\pgfpathlineto{\pgfqpoint{2.286845in}{0.947504in}}%
\pgfpathlineto{\pgfqpoint{2.289980in}{0.947504in}}%
\pgfpathlineto{\pgfqpoint{2.293115in}{0.947504in}}%
\pgfpathlineto{\pgfqpoint{2.293585in}{0.948066in}}%
\pgfpathlineto{\pgfqpoint{2.296249in}{0.951255in}}%
\pgfpathlineto{\pgfqpoint{2.299384in}{0.951255in}}%
\pgfpathlineto{\pgfqpoint{2.302519in}{0.951255in}}%
\pgfpathlineto{\pgfqpoint{2.302989in}{0.951818in}}%
\pgfpathlineto{\pgfqpoint{2.305654in}{0.955007in}}%
\pgfpathlineto{\pgfqpoint{2.308788in}{0.955007in}}%
\pgfpathlineto{\pgfqpoint{2.309258in}{0.955569in}}%
\pgfpathlineto{\pgfqpoint{2.311923in}{0.958758in}}%
\pgfpathlineto{\pgfqpoint{2.315058in}{0.958758in}}%
\pgfpathlineto{\pgfqpoint{2.318192in}{0.958758in}}%
\pgfpathlineto{\pgfqpoint{2.318663in}{0.959321in}}%
\pgfpathlineto{\pgfqpoint{2.321327in}{0.962510in}}%
\pgfpathlineto{\pgfqpoint{2.324462in}{0.962510in}}%
\pgfpathlineto{\pgfqpoint{2.327597in}{0.962510in}}%
\pgfpathlineto{\pgfqpoint{2.328067in}{0.963073in}}%
\pgfpathlineto{\pgfqpoint{2.330731in}{0.966261in}}%
\pgfpathlineto{\pgfqpoint{2.333866in}{0.966261in}}%
\pgfpathlineto{\pgfqpoint{2.337001in}{0.966261in}}%
\pgfpathlineto{\pgfqpoint{2.337471in}{0.966824in}}%
\pgfpathlineto{\pgfqpoint{2.340136in}{0.970013in}}%
\pgfpathlineto{\pgfqpoint{2.343270in}{0.970013in}}%
\pgfpathlineto{\pgfqpoint{2.343741in}{0.970576in}}%
\pgfpathlineto{\pgfqpoint{2.346405in}{0.973764in}}%
\pgfpathlineto{\pgfqpoint{2.349540in}{0.973764in}}%
\pgfpathlineto{\pgfqpoint{2.352675in}{0.973764in}}%
\pgfpathlineto{\pgfqpoint{2.353145in}{0.974327in}}%
\pgfpathlineto{\pgfqpoint{2.355809in}{0.977516in}}%
\pgfpathlineto{\pgfqpoint{2.358944in}{0.977516in}}%
\pgfpathlineto{\pgfqpoint{2.362079in}{0.977516in}}%
\pgfpathlineto{\pgfqpoint{2.362549in}{0.978079in}}%
\pgfpathlineto{\pgfqpoint{2.365214in}{0.981267in}}%
\pgfpathlineto{\pgfqpoint{2.368348in}{0.981267in}}%
\pgfpathlineto{\pgfqpoint{2.371483in}{0.981267in}}%
\pgfpathlineto{\pgfqpoint{2.371953in}{0.981830in}}%
\pgfpathlineto{\pgfqpoint{2.374618in}{0.985019in}}%
\pgfpathlineto{\pgfqpoint{2.377753in}{0.985019in}}%
\pgfpathlineto{\pgfqpoint{2.378223in}{0.985582in}}%
\pgfpathlineto{\pgfqpoint{2.380887in}{0.988771in}}%
\pgfpathlineto{\pgfqpoint{2.384022in}{0.988771in}}%
\pgfpathlineto{\pgfqpoint{2.387157in}{0.988771in}}%
\pgfpathlineto{\pgfqpoint{2.387627in}{0.989333in}}%
\pgfpathlineto{\pgfqpoint{2.390292in}{0.992522in}}%
\pgfpathlineto{\pgfqpoint{2.393426in}{0.992522in}}%
\pgfpathlineto{\pgfqpoint{2.396561in}{0.992522in}}%
\pgfpathlineto{\pgfqpoint{2.397031in}{0.993085in}}%
\pgfpathlineto{\pgfqpoint{2.399696in}{0.996274in}}%
\pgfpathlineto{\pgfqpoint{2.402831in}{0.996274in}}%
\pgfpathlineto{\pgfqpoint{2.405965in}{0.996274in}}%
\pgfpathlineto{\pgfqpoint{2.406435in}{0.996836in}}%
\pgfpathlineto{\pgfqpoint{2.409100in}{1.000025in}}%
\pgfpathlineto{\pgfqpoint{2.412235in}{1.000025in}}%
\pgfpathlineto{\pgfqpoint{2.415369in}{1.000025in}}%
\pgfpathlineto{\pgfqpoint{2.415840in}{1.000588in}}%
\pgfpathlineto{\pgfqpoint{2.418504in}{1.003777in}}%
\pgfpathlineto{\pgfqpoint{2.421639in}{1.003777in}}%
\pgfpathlineto{\pgfqpoint{2.422109in}{1.004339in}}%
\pgfpathlineto{\pgfqpoint{2.424774in}{1.007528in}}%
\pgfpathlineto{\pgfqpoint{2.427908in}{1.007528in}}%
\pgfpathlineto{\pgfqpoint{2.431043in}{1.007528in}}%
\pgfpathlineto{\pgfqpoint{2.431513in}{1.008091in}}%
\pgfpathlineto{\pgfqpoint{2.434178in}{1.011280in}}%
\pgfpathlineto{\pgfqpoint{2.437313in}{1.011280in}}%
\pgfpathlineto{\pgfqpoint{2.440447in}{1.011280in}}%
\pgfpathlineto{\pgfqpoint{2.440918in}{1.011842in}}%
\pgfpathlineto{\pgfqpoint{2.443582in}{1.015031in}}%
\pgfpathlineto{\pgfqpoint{2.446717in}{1.015031in}}%
\pgfpathlineto{\pgfqpoint{2.449852in}{1.015031in}}%
\pgfpathlineto{\pgfqpoint{2.450322in}{1.015594in}}%
\pgfpathlineto{\pgfqpoint{2.452986in}{1.018783in}}%
\pgfpathlineto{\pgfqpoint{2.456121in}{1.018783in}}%
\pgfpathlineto{\pgfqpoint{2.456591in}{1.019346in}}%
\pgfpathlineto{\pgfqpoint{2.459256in}{1.022534in}}%
\pgfpathlineto{\pgfqpoint{2.462391in}{1.022534in}}%
\pgfpathlineto{\pgfqpoint{2.465525in}{1.022534in}}%
\pgfpathlineto{\pgfqpoint{2.465996in}{1.023097in}}%
\pgfpathlineto{\pgfqpoint{2.468660in}{1.026286in}}%
\pgfpathlineto{\pgfqpoint{2.471795in}{1.026286in}}%
\pgfpathlineto{\pgfqpoint{2.474930in}{1.026286in}}%
\pgfpathlineto{\pgfqpoint{2.475400in}{1.026849in}}%
\pgfpathlineto{\pgfqpoint{2.478064in}{1.030037in}}%
\pgfpathlineto{\pgfqpoint{2.481199in}{1.030037in}}%
\pgfpathlineto{\pgfqpoint{2.484334in}{1.030037in}}%
\pgfpathlineto{\pgfqpoint{2.484804in}{1.030600in}}%
\pgfpathlineto{\pgfqpoint{2.487469in}{1.033789in}}%
\pgfpathlineto{\pgfqpoint{2.490603in}{1.033789in}}%
\pgfpathlineto{\pgfqpoint{2.491074in}{1.034352in}}%
\pgfpathlineto{\pgfqpoint{2.493738in}{1.037540in}}%
\pgfpathlineto{\pgfqpoint{2.496873in}{1.037540in}}%
\pgfpathlineto{\pgfqpoint{2.500008in}{1.037540in}}%
\pgfpathlineto{\pgfqpoint{2.500478in}{1.038103in}}%
\pgfpathlineto{\pgfqpoint{2.503142in}{1.041292in}}%
\pgfpathlineto{\pgfqpoint{2.506277in}{1.041292in}}%
\pgfpathlineto{\pgfqpoint{2.509412in}{1.041292in}}%
\pgfpathlineto{\pgfqpoint{2.509882in}{1.041855in}}%
\pgfpathlineto{\pgfqpoint{2.512547in}{1.045044in}}%
\pgfpathlineto{\pgfqpoint{2.515681in}{1.045044in}}%
\pgfpathlineto{\pgfqpoint{2.518816in}{1.045044in}}%
\pgfpathlineto{\pgfqpoint{2.519286in}{1.045606in}}%
\pgfpathlineto{\pgfqpoint{2.521951in}{1.048795in}}%
\pgfpathlineto{\pgfqpoint{2.525085in}{1.048795in}}%
\pgfpathlineto{\pgfqpoint{2.528220in}{1.048795in}}%
\pgfpathlineto{\pgfqpoint{2.528690in}{1.049358in}}%
\pgfpathlineto{\pgfqpoint{2.531355in}{1.052547in}}%
\pgfpathlineto{\pgfqpoint{2.534490in}{1.052547in}}%
\pgfpathlineto{\pgfqpoint{2.534960in}{1.053109in}}%
\pgfpathlineto{\pgfqpoint{2.537624in}{1.056298in}}%
\pgfpathlineto{\pgfqpoint{2.540759in}{1.056298in}}%
\pgfpathlineto{\pgfqpoint{2.543894in}{1.056298in}}%
\pgfpathlineto{\pgfqpoint{2.544364in}{1.056861in}}%
\pgfpathlineto{\pgfqpoint{2.547029in}{1.060050in}}%
\pgfpathlineto{\pgfqpoint{2.550163in}{1.060050in}}%
\pgfpathlineto{\pgfqpoint{2.553298in}{1.060050in}}%
\pgfpathlineto{\pgfqpoint{2.553768in}{1.060612in}}%
\pgfpathlineto{\pgfqpoint{2.556433in}{1.063801in}}%
\pgfpathlineto{\pgfqpoint{2.559568in}{1.063801in}}%
\pgfpathlineto{\pgfqpoint{2.562702in}{1.063801in}}%
\pgfpathlineto{\pgfqpoint{2.563173in}{1.064364in}}%
\pgfpathlineto{\pgfqpoint{2.565837in}{1.067553in}}%
\pgfpathlineto{\pgfqpoint{2.568972in}{1.067553in}}%
\pgfpathlineto{\pgfqpoint{2.569442in}{1.068115in}}%
\pgfpathlineto{\pgfqpoint{2.572107in}{1.071304in}}%
\pgfpathlineto{\pgfqpoint{2.575241in}{1.071304in}}%
\pgfpathlineto{\pgfqpoint{2.578376in}{1.071304in}}%
\pgfpathlineto{\pgfqpoint{2.578846in}{1.071867in}}%
\pgfpathlineto{\pgfqpoint{2.581511in}{1.075056in}}%
\pgfpathlineto{\pgfqpoint{2.584646in}{1.075056in}}%
\pgfpathlineto{\pgfqpoint{2.587780in}{1.075056in}}%
\pgfpathlineto{\pgfqpoint{2.588251in}{1.075619in}}%
\pgfpathlineto{\pgfqpoint{2.590915in}{1.078807in}}%
\pgfpathlineto{\pgfqpoint{2.594050in}{1.078807in}}%
\pgfpathlineto{\pgfqpoint{2.597185in}{1.078807in}}%
\pgfpathlineto{\pgfqpoint{2.597655in}{1.079370in}}%
\pgfpathlineto{\pgfqpoint{2.600319in}{1.082559in}}%
\pgfpathlineto{\pgfqpoint{2.603454in}{1.082559in}}%
\pgfpathlineto{\pgfqpoint{2.603924in}{1.083122in}}%
\pgfpathlineto{\pgfqpoint{2.606589in}{1.086310in}}%
\pgfpathlineto{\pgfqpoint{2.609724in}{1.086310in}}%
\pgfpathlineto{\pgfqpoint{2.612858in}{1.086310in}}%
\pgfpathlineto{\pgfqpoint{2.613328in}{1.086873in}}%
\pgfpathlineto{\pgfqpoint{2.615993in}{1.090062in}}%
\pgfpathlineto{\pgfqpoint{2.619128in}{1.090062in}}%
\pgfpathlineto{\pgfqpoint{2.622262in}{1.090062in}}%
\pgfpathlineto{\pgfqpoint{2.622733in}{1.090625in}}%
\pgfpathlineto{\pgfqpoint{2.625397in}{1.093813in}}%
\pgfpathlineto{\pgfqpoint{2.628532in}{1.093813in}}%
\pgfpathlineto{\pgfqpoint{2.631667in}{1.093813in}}%
\pgfpathlineto{\pgfqpoint{2.632137in}{1.094376in}}%
\pgfpathlineto{\pgfqpoint{2.634801in}{1.097565in}}%
\pgfpathlineto{\pgfqpoint{2.637936in}{1.097565in}}%
\pgfpathlineto{\pgfqpoint{2.641071in}{1.097565in}}%
\pgfpathlineto{\pgfqpoint{2.641541in}{1.098128in}}%
\pgfpathlineto{\pgfqpoint{2.644206in}{1.101317in}}%
\pgfpathlineto{\pgfqpoint{2.647340in}{1.101317in}}%
\pgfpathlineto{\pgfqpoint{2.647811in}{1.101879in}}%
\pgfpathlineto{\pgfqpoint{2.650475in}{1.105068in}}%
\pgfpathlineto{\pgfqpoint{2.653610in}{1.105068in}}%
\pgfpathlineto{\pgfqpoint{2.656745in}{1.105068in}}%
\pgfpathlineto{\pgfqpoint{2.657215in}{1.105631in}}%
\pgfpathlineto{\pgfqpoint{2.659879in}{1.108820in}}%
\pgfpathlineto{\pgfqpoint{2.663014in}{1.108820in}}%
\pgfpathlineto{\pgfqpoint{2.666149in}{1.108820in}}%
\pgfpathlineto{\pgfqpoint{2.666619in}{1.109382in}}%
\pgfpathlineto{\pgfqpoint{2.669284in}{1.112571in}}%
\pgfpathlineto{\pgfqpoint{2.672418in}{1.112571in}}%
\pgfpathlineto{\pgfqpoint{2.675553in}{1.112571in}}%
\pgfpathlineto{\pgfqpoint{2.676023in}{1.113134in}}%
\pgfpathlineto{\pgfqpoint{2.678688in}{1.116323in}}%
\pgfpathlineto{\pgfqpoint{2.681823in}{1.116323in}}%
\pgfpathlineto{\pgfqpoint{2.682293in}{1.116885in}}%
\pgfpathlineto{\pgfqpoint{2.684957in}{1.120074in}}%
\pgfpathlineto{\pgfqpoint{2.688092in}{1.120074in}}%
\pgfpathlineto{\pgfqpoint{2.691227in}{1.120074in}}%
\pgfpathlineto{\pgfqpoint{2.691697in}{1.120637in}}%
\pgfpathlineto{\pgfqpoint{2.694362in}{1.123826in}}%
\pgfpathlineto{\pgfqpoint{2.697496in}{1.123826in}}%
\pgfpathlineto{\pgfqpoint{2.700631in}{1.123826in}}%
\pgfpathlineto{\pgfqpoint{2.701101in}{1.124388in}}%
\pgfpathlineto{\pgfqpoint{2.703766in}{1.127577in}}%
\pgfpathlineto{\pgfqpoint{2.706901in}{1.127577in}}%
\pgfpathlineto{\pgfqpoint{2.710035in}{1.127577in}}%
\pgfpathlineto{\pgfqpoint{2.710505in}{1.128140in}}%
\pgfpathlineto{\pgfqpoint{2.713170in}{1.131329in}}%
\pgfpathlineto{\pgfqpoint{2.716305in}{1.131329in}}%
\pgfpathlineto{\pgfqpoint{2.716775in}{1.131892in}}%
\pgfpathlineto{\pgfqpoint{2.719439in}{1.135080in}}%
\pgfpathlineto{\pgfqpoint{2.722574in}{1.135080in}}%
\pgfpathlineto{\pgfqpoint{2.725709in}{1.135080in}}%
\pgfpathlineto{\pgfqpoint{2.726179in}{1.135643in}}%
\pgfpathlineto{\pgfqpoint{2.728844in}{1.138832in}}%
\pgfpathlineto{\pgfqpoint{2.731978in}{1.138832in}}%
\pgfpathlineto{\pgfqpoint{2.735113in}{1.138832in}}%
\pgfpathlineto{\pgfqpoint{2.735583in}{1.139395in}}%
\pgfpathlineto{\pgfqpoint{2.738248in}{1.142583in}}%
\pgfpathlineto{\pgfqpoint{2.741383in}{1.142583in}}%
\pgfpathlineto{\pgfqpoint{2.744517in}{1.142583in}}%
\pgfpathlineto{\pgfqpoint{2.744988in}{1.143146in}}%
\pgfpathlineto{\pgfqpoint{2.747652in}{1.146335in}}%
\pgfpathlineto{\pgfqpoint{2.750787in}{1.146335in}}%
\pgfpathlineto{\pgfqpoint{2.753922in}{1.146335in}}%
\pgfpathlineto{\pgfqpoint{2.754392in}{1.146898in}}%
\pgfpathlineto{\pgfqpoint{2.757056in}{1.150087in}}%
\pgfpathlineto{\pgfqpoint{2.760191in}{1.150087in}}%
\pgfpathlineto{\pgfqpoint{2.760661in}{1.150649in}}%
\pgfpathlineto{\pgfqpoint{2.763326in}{1.153838in}}%
\pgfpathlineto{\pgfqpoint{2.766461in}{1.153838in}}%
\pgfpathlineto{\pgfqpoint{2.769595in}{1.153838in}}%
\pgfpathlineto{\pgfqpoint{2.770066in}{1.154401in}}%
\pgfpathlineto{\pgfqpoint{2.772730in}{1.157590in}}%
\pgfpathlineto{\pgfqpoint{2.775865in}{1.157590in}}%
\pgfpathlineto{\pgfqpoint{2.779000in}{1.157590in}}%
\pgfpathlineto{\pgfqpoint{2.779470in}{1.158152in}}%
\pgfpathlineto{\pgfqpoint{2.782134in}{1.161341in}}%
\pgfpathlineto{\pgfqpoint{2.785269in}{1.161341in}}%
\pgfpathlineto{\pgfqpoint{2.788404in}{1.161341in}}%
\pgfpathlineto{\pgfqpoint{2.788874in}{1.161904in}}%
\pgfpathlineto{\pgfqpoint{2.791539in}{1.165093in}}%
\pgfpathlineto{\pgfqpoint{2.794673in}{1.165093in}}%
\pgfpathlineto{\pgfqpoint{2.795144in}{1.165655in}}%
\pgfpathlineto{\pgfqpoint{2.797808in}{1.168844in}}%
\pgfpathlineto{\pgfqpoint{2.800943in}{1.168844in}}%
\pgfpathlineto{\pgfqpoint{2.804078in}{1.168844in}}%
\pgfpathlineto{\pgfqpoint{2.804548in}{1.169407in}}%
\pgfpathlineto{\pgfqpoint{2.807212in}{1.172596in}}%
\pgfpathlineto{\pgfqpoint{2.810347in}{1.172596in}}%
\pgfpathlineto{\pgfqpoint{2.813482in}{1.172596in}}%
\pgfpathlineto{\pgfqpoint{2.813952in}{1.173158in}}%
\pgfpathlineto{\pgfqpoint{2.813952in}{1.176910in}}%
\pgfpathlineto{\pgfqpoint{2.816617in}{1.180099in}}%
\pgfpathlineto{\pgfqpoint{2.817087in}{1.180662in}}%
\pgfpathlineto{\pgfqpoint{2.817087in}{1.184413in}}%
\pgfpathlineto{\pgfqpoint{2.817087in}{1.188165in}}%
\pgfpathlineto{\pgfqpoint{2.819751in}{1.191353in}}%
\pgfpathlineto{\pgfqpoint{2.820221in}{1.191916in}}%
\pgfpathlineto{\pgfqpoint{2.820221in}{1.195668in}}%
\pgfpathlineto{\pgfqpoint{2.820221in}{1.199419in}}%
\pgfpathlineto{\pgfqpoint{2.822886in}{1.202608in}}%
\pgfpathlineto{\pgfqpoint{2.823356in}{1.203171in}}%
\pgfpathlineto{\pgfqpoint{2.823356in}{1.206922in}}%
\pgfpathlineto{\pgfqpoint{2.823356in}{1.210674in}}%
\pgfpathlineto{\pgfqpoint{2.826021in}{1.213863in}}%
\pgfpathlineto{\pgfqpoint{2.826491in}{1.214425in}}%
\pgfpathlineto{\pgfqpoint{2.826491in}{1.218177in}}%
\pgfpathlineto{\pgfqpoint{2.826491in}{1.221928in}}%
\pgfpathlineto{\pgfqpoint{2.829155in}{1.225117in}}%
\pgfpathlineto{\pgfqpoint{2.829626in}{1.225680in}}%
\pgfpathlineto{\pgfqpoint{2.829626in}{1.229431in}}%
\pgfpathlineto{\pgfqpoint{2.829626in}{1.233183in}}%
\pgfpathlineto{\pgfqpoint{2.832290in}{1.236372in}}%
\pgfpathlineto{\pgfqpoint{2.832760in}{1.236935in}}%
\pgfpathlineto{\pgfqpoint{2.832760in}{1.240686in}}%
\pgfpathlineto{\pgfqpoint{2.832760in}{1.244438in}}%
\pgfpathlineto{\pgfqpoint{2.835425in}{1.247626in}}%
\pgfpathlineto{\pgfqpoint{2.835895in}{1.248189in}}%
\pgfpathlineto{\pgfqpoint{2.835895in}{1.251941in}}%
\pgfpathlineto{\pgfqpoint{2.835895in}{1.255692in}}%
\pgfpathlineto{\pgfqpoint{2.835895in}{1.259444in}}%
\pgfpathlineto{\pgfqpoint{2.838560in}{1.262633in}}%
\pgfpathlineto{\pgfqpoint{2.839030in}{1.263195in}}%
\pgfpathlineto{\pgfqpoint{2.839030in}{1.266947in}}%
\pgfpathlineto{\pgfqpoint{2.839030in}{1.270698in}}%
\pgfpathlineto{\pgfqpoint{2.841694in}{1.273887in}}%
\pgfpathlineto{\pgfqpoint{2.842165in}{1.274450in}}%
\pgfpathlineto{\pgfqpoint{2.842165in}{1.278201in}}%
\pgfpathlineto{\pgfqpoint{2.842165in}{1.281953in}}%
\pgfpathlineto{\pgfqpoint{2.844829in}{1.285142in}}%
\pgfpathlineto{\pgfqpoint{2.845299in}{1.285704in}}%
\pgfpathlineto{\pgfqpoint{2.845299in}{1.289456in}}%
\pgfpathlineto{\pgfqpoint{2.845299in}{1.293208in}}%
\pgfpathlineto{\pgfqpoint{2.847964in}{1.296396in}}%
\pgfpathlineto{\pgfqpoint{2.848434in}{1.296959in}}%
\pgfpathlineto{\pgfqpoint{2.848434in}{1.300711in}}%
\pgfpathlineto{\pgfqpoint{2.848434in}{1.304462in}}%
\pgfpathlineto{\pgfqpoint{2.851099in}{1.307651in}}%
\pgfpathlineto{\pgfqpoint{2.851569in}{1.308214in}}%
\pgfpathlineto{\pgfqpoint{2.851569in}{1.311965in}}%
\pgfpathlineto{\pgfqpoint{2.851569in}{1.315717in}}%
\pgfpathlineto{\pgfqpoint{2.854233in}{1.318906in}}%
\pgfpathlineto{\pgfqpoint{2.854704in}{1.319468in}}%
\pgfpathlineto{\pgfqpoint{2.854704in}{1.323220in}}%
\pgfpathlineto{\pgfqpoint{2.854704in}{1.326971in}}%
\pgfpathlineto{\pgfqpoint{2.857368in}{1.330160in}}%
\pgfpathlineto{\pgfqpoint{2.857838in}{1.330723in}}%
\pgfpathlineto{\pgfqpoint{2.857838in}{1.334474in}}%
\pgfpathlineto{\pgfqpoint{2.857838in}{1.338226in}}%
\pgfpathlineto{\pgfqpoint{2.860503in}{1.341415in}}%
\pgfpathlineto{\pgfqpoint{2.860973in}{1.341977in}}%
\pgfpathlineto{\pgfqpoint{2.860973in}{1.345729in}}%
\pgfpathlineto{\pgfqpoint{2.860973in}{1.349481in}}%
\pgfpathlineto{\pgfqpoint{2.863638in}{1.352669in}}%
\pgfpathlineto{\pgfqpoint{2.864108in}{1.353232in}}%
\pgfpathlineto{\pgfqpoint{2.864108in}{1.356984in}}%
\pgfpathlineto{\pgfqpoint{2.864108in}{1.360735in}}%
\pgfpathlineto{\pgfqpoint{2.864108in}{1.364487in}}%
\pgfpathlineto{\pgfqpoint{2.866772in}{1.367676in}}%
\pgfpathlineto{\pgfqpoint{2.867243in}{1.368238in}}%
\pgfpathlineto{\pgfqpoint{2.867243in}{1.371990in}}%
\pgfpathlineto{\pgfqpoint{2.867243in}{1.375741in}}%
\pgfpathlineto{\pgfqpoint{2.869907in}{1.378930in}}%
\pgfpathlineto{\pgfqpoint{2.870377in}{1.379493in}}%
\pgfpathlineto{\pgfqpoint{2.870377in}{1.383244in}}%
\pgfpathlineto{\pgfqpoint{2.870377in}{1.386996in}}%
\pgfpathlineto{\pgfqpoint{2.873042in}{1.390185in}}%
\pgfpathlineto{\pgfqpoint{2.873512in}{1.390747in}}%
\pgfpathlineto{\pgfqpoint{2.873512in}{1.394499in}}%
\pgfpathlineto{\pgfqpoint{2.873512in}{1.398251in}}%
\pgfpathlineto{\pgfqpoint{2.876177in}{1.401439in}}%
\pgfpathlineto{\pgfqpoint{2.876647in}{1.402002in}}%
\pgfpathlineto{\pgfqpoint{2.876647in}{1.405754in}}%
\pgfpathlineto{\pgfqpoint{2.876647in}{1.409505in}}%
\pgfpathlineto{\pgfqpoint{2.879311in}{1.412694in}}%
\pgfpathlineto{\pgfqpoint{2.879782in}{1.413257in}}%
\pgfpathlineto{\pgfqpoint{2.879782in}{1.417008in}}%
\pgfpathlineto{\pgfqpoint{2.879782in}{1.420760in}}%
\pgfpathlineto{\pgfqpoint{2.882446in}{1.423949in}}%
\pgfpathlineto{\pgfqpoint{2.882916in}{1.424511in}}%
\pgfpathlineto{\pgfqpoint{2.882916in}{1.428263in}}%
\pgfpathlineto{\pgfqpoint{2.882916in}{1.432014in}}%
\pgfpathlineto{\pgfqpoint{2.885581in}{1.435203in}}%
\pgfpathlineto{\pgfqpoint{2.886051in}{1.435766in}}%
\pgfpathlineto{\pgfqpoint{2.886051in}{1.439517in}}%
\pgfpathlineto{\pgfqpoint{2.886051in}{1.443269in}}%
\pgfpathlineto{\pgfqpoint{2.888716in}{1.446458in}}%
\pgfpathlineto{\pgfqpoint{2.889186in}{1.447020in}}%
\pgfpathlineto{\pgfqpoint{2.889186in}{1.450772in}}%
\pgfpathlineto{\pgfqpoint{2.889186in}{1.454524in}}%
\pgfpathlineto{\pgfqpoint{2.889186in}{1.458275in}}%
\pgfpathlineto{\pgfqpoint{2.891850in}{1.461464in}}%
\pgfpathlineto{\pgfqpoint{2.892321in}{1.462027in}}%
\pgfpathlineto{\pgfqpoint{2.892321in}{1.465778in}}%
\pgfpathlineto{\pgfqpoint{2.892321in}{1.469530in}}%
\pgfpathlineto{\pgfqpoint{2.894985in}{1.472718in}}%
\pgfpathlineto{\pgfqpoint{2.895455in}{1.473281in}}%
\pgfpathlineto{\pgfqpoint{2.895455in}{1.477033in}}%
\pgfpathlineto{\pgfqpoint{2.895455in}{1.480784in}}%
\pgfpathlineto{\pgfqpoint{2.898120in}{1.483973in}}%
\pgfpathlineto{\pgfqpoint{2.898590in}{1.484536in}}%
\pgfpathlineto{\pgfqpoint{2.898590in}{1.488287in}}%
\pgfpathlineto{\pgfqpoint{2.898590in}{1.492039in}}%
\pgfpathlineto{\pgfqpoint{2.901255in}{1.495228in}}%
\pgfpathlineto{\pgfqpoint{2.901725in}{1.495790in}}%
\pgfpathlineto{\pgfqpoint{2.901725in}{1.499542in}}%
\pgfpathlineto{\pgfqpoint{2.901725in}{1.503293in}}%
\pgfpathlineto{\pgfqpoint{2.904389in}{1.506482in}}%
\pgfpathlineto{\pgfqpoint{2.904860in}{1.507045in}}%
\pgfpathlineto{\pgfqpoint{2.904860in}{1.510797in}}%
\pgfpathlineto{\pgfqpoint{2.904860in}{1.514548in}}%
\pgfpathlineto{\pgfqpoint{2.907524in}{1.517737in}}%
\pgfpathlineto{\pgfqpoint{2.907994in}{1.518300in}}%
\pgfpathlineto{\pgfqpoint{2.907994in}{1.522051in}}%
\pgfpathlineto{\pgfqpoint{2.907994in}{1.525803in}}%
\pgfpathlineto{\pgfqpoint{2.910659in}{1.528991in}}%
\pgfpathlineto{\pgfqpoint{2.911129in}{1.529554in}}%
\pgfpathlineto{\pgfqpoint{2.911129in}{1.533306in}}%
\pgfpathlineto{\pgfqpoint{2.911129in}{1.537057in}}%
\pgfpathlineto{\pgfqpoint{2.913794in}{1.540246in}}%
\pgfpathlineto{\pgfqpoint{2.914264in}{1.540809in}}%
\pgfpathlineto{\pgfqpoint{2.914264in}{1.544560in}}%
\pgfpathlineto{\pgfqpoint{2.914264in}{1.548312in}}%
\pgfpathlineto{\pgfqpoint{2.916928in}{1.551501in}}%
\pgfpathlineto{\pgfqpoint{2.917398in}{1.552063in}}%
\pgfpathlineto{\pgfqpoint{2.917398in}{1.555815in}}%
\pgfpathlineto{\pgfqpoint{2.917398in}{1.559566in}}%
\pgfpathlineto{\pgfqpoint{2.917398in}{1.563318in}}%
\pgfpathlineto{\pgfqpoint{2.920063in}{1.566507in}}%
\pgfpathlineto{\pgfqpoint{2.920533in}{1.567070in}}%
\pgfpathlineto{\pgfqpoint{2.920533in}{1.570821in}}%
\pgfpathlineto{\pgfqpoint{2.920533in}{1.574573in}}%
\pgfpathlineto{\pgfqpoint{2.923198in}{1.577761in}}%
\pgfpathlineto{\pgfqpoint{2.923668in}{1.578324in}}%
\pgfpathlineto{\pgfqpoint{2.923668in}{1.582076in}}%
\pgfpathlineto{\pgfqpoint{2.923668in}{1.585827in}}%
\pgfpathlineto{\pgfqpoint{2.926332in}{1.589016in}}%
\pgfpathlineto{\pgfqpoint{2.926803in}{1.589579in}}%
\pgfpathlineto{\pgfqpoint{2.926803in}{1.593330in}}%
\pgfpathlineto{\pgfqpoint{2.926803in}{1.597082in}}%
\pgfpathlineto{\pgfqpoint{2.929467in}{1.600271in}}%
\pgfpathlineto{\pgfqpoint{2.929937in}{1.600833in}}%
\pgfpathlineto{\pgfqpoint{2.929937in}{1.604585in}}%
\pgfpathlineto{\pgfqpoint{2.929937in}{1.608336in}}%
\pgfpathlineto{\pgfqpoint{2.932602in}{1.611525in}}%
\pgfpathlineto{\pgfqpoint{2.933072in}{1.612088in}}%
\pgfpathlineto{\pgfqpoint{2.933072in}{1.615840in}}%
\pgfpathlineto{\pgfqpoint{2.933072in}{1.619591in}}%
\pgfpathlineto{\pgfqpoint{2.935737in}{1.622780in}}%
\pgfpathlineto{\pgfqpoint{2.936207in}{1.623343in}}%
\pgfpathlineto{\pgfqpoint{2.936207in}{1.627094in}}%
\pgfpathlineto{\pgfqpoint{2.936207in}{1.630846in}}%
\pgfpathlineto{\pgfqpoint{2.938871in}{1.634034in}}%
\pgfpathlineto{\pgfqpoint{2.939342in}{1.634597in}}%
\pgfpathlineto{\pgfqpoint{2.939342in}{1.638349in}}%
\pgfpathlineto{\pgfqpoint{2.939342in}{1.642100in}}%
\pgfpathlineto{\pgfqpoint{2.942006in}{1.645289in}}%
\pgfpathlineto{\pgfqpoint{2.942476in}{1.645852in}}%
\pgfpathlineto{\pgfqpoint{2.942476in}{1.649603in}}%
\pgfpathlineto{\pgfqpoint{2.942476in}{1.653355in}}%
\pgfpathlineto{\pgfqpoint{2.942476in}{1.657106in}}%
\pgfpathlineto{\pgfqpoint{2.945141in}{1.660295in}}%
\pgfpathlineto{\pgfqpoint{2.945611in}{1.660858in}}%
\pgfpathlineto{\pgfqpoint{2.945611in}{1.664609in}}%
\pgfpathlineto{\pgfqpoint{2.945611in}{1.668361in}}%
\pgfpathlineto{\pgfqpoint{2.948276in}{1.671550in}}%
\pgfpathlineto{\pgfqpoint{2.948746in}{1.672113in}}%
\pgfpathlineto{\pgfqpoint{2.948746in}{1.675864in}}%
\pgfpathlineto{\pgfqpoint{2.948746in}{1.679616in}}%
\pgfpathlineto{\pgfqpoint{2.951410in}{1.682804in}}%
\pgfpathlineto{\pgfqpoint{2.951881in}{1.683367in}}%
\pgfpathlineto{\pgfqpoint{2.951881in}{1.687119in}}%
\pgfpathlineto{\pgfqpoint{2.951881in}{1.690870in}}%
\pgfpathlineto{\pgfqpoint{2.954545in}{1.694059in}}%
\pgfpathlineto{\pgfqpoint{2.955015in}{1.694622in}}%
\pgfpathlineto{\pgfqpoint{2.955015in}{1.698373in}}%
\pgfpathlineto{\pgfqpoint{2.955015in}{1.702125in}}%
\pgfpathlineto{\pgfqpoint{2.957680in}{1.705314in}}%
\pgfpathlineto{\pgfqpoint{2.958150in}{1.705876in}}%
\pgfpathlineto{\pgfqpoint{2.958150in}{1.709628in}}%
\pgfpathlineto{\pgfqpoint{2.958150in}{1.713379in}}%
\pgfpathlineto{\pgfqpoint{2.960815in}{1.716568in}}%
\pgfpathlineto{\pgfqpoint{2.961285in}{1.717131in}}%
\pgfpathlineto{\pgfqpoint{2.961285in}{1.720882in}}%
\pgfpathlineto{\pgfqpoint{2.961285in}{1.724634in}}%
\pgfpathlineto{\pgfqpoint{2.963949in}{1.727823in}}%
\pgfpathlineto{\pgfqpoint{2.964420in}{1.728386in}}%
\pgfpathlineto{\pgfqpoint{2.964420in}{1.732137in}}%
\pgfpathlineto{\pgfqpoint{2.964420in}{1.735889in}}%
\pgfpathlineto{\pgfqpoint{2.967084in}{1.739077in}}%
\pgfpathlineto{\pgfqpoint{2.967554in}{1.739640in}}%
\pgfpathlineto{\pgfqpoint{2.967554in}{1.743392in}}%
\pgfpathlineto{\pgfqpoint{2.967554in}{1.747143in}}%
\pgfpathlineto{\pgfqpoint{2.970219in}{1.750332in}}%
\pgfpathlineto{\pgfqpoint{2.970689in}{1.750895in}}%
\pgfpathlineto{\pgfqpoint{2.970689in}{1.754646in}}%
\pgfpathlineto{\pgfqpoint{2.970689in}{1.758398in}}%
\pgfpathlineto{\pgfqpoint{2.970689in}{1.762149in}}%
\pgfpathlineto{\pgfqpoint{2.973354in}{1.765338in}}%
\pgfpathlineto{\pgfqpoint{2.973824in}{1.765901in}}%
\pgfpathlineto{\pgfqpoint{2.973824in}{1.769652in}}%
\pgfpathlineto{\pgfqpoint{2.973824in}{1.773404in}}%
\pgfpathlineto{\pgfqpoint{2.976488in}{1.776593in}}%
\pgfpathlineto{\pgfqpoint{2.976959in}{1.777155in}}%
\pgfpathlineto{\pgfqpoint{2.976959in}{1.780907in}}%
\pgfpathlineto{\pgfqpoint{2.976959in}{1.784659in}}%
\pgfpathlineto{\pgfqpoint{2.979623in}{1.787847in}}%
\pgfpathlineto{\pgfqpoint{2.980093in}{1.788410in}}%
\pgfpathlineto{\pgfqpoint{2.980093in}{1.792162in}}%
\pgfpathlineto{\pgfqpoint{2.980093in}{1.795913in}}%
\pgfpathlineto{\pgfqpoint{2.982758in}{1.799102in}}%
\pgfpathlineto{\pgfqpoint{2.983228in}{1.799665in}}%
\pgfpathlineto{\pgfqpoint{2.983228in}{1.803416in}}%
\pgfpathlineto{\pgfqpoint{2.983228in}{1.807168in}}%
\pgfpathlineto{\pgfqpoint{2.985893in}{1.810357in}}%
\pgfpathlineto{\pgfqpoint{2.986363in}{1.810919in}}%
\pgfpathlineto{\pgfqpoint{2.986363in}{1.814671in}}%
\pgfpathlineto{\pgfqpoint{2.986363in}{1.818422in}}%
\pgfpathlineto{\pgfqpoint{2.989027in}{1.821611in}}%
\pgfpathlineto{\pgfqpoint{2.989498in}{1.822174in}}%
\pgfpathlineto{\pgfqpoint{2.989498in}{1.825925in}}%
\pgfpathlineto{\pgfqpoint{2.989498in}{1.829677in}}%
\pgfpathlineto{\pgfqpoint{2.992162in}{1.832866in}}%
\pgfpathlineto{\pgfqpoint{2.992632in}{1.833429in}}%
\pgfpathlineto{\pgfqpoint{2.992632in}{1.837180in}}%
\pgfpathlineto{\pgfqpoint{2.992632in}{1.840932in}}%
\pgfpathlineto{\pgfqpoint{2.995297in}{1.844120in}}%
\pgfpathlineto{\pgfqpoint{2.995767in}{1.844683in}}%
\pgfpathlineto{\pgfqpoint{2.995767in}{1.848435in}}%
\pgfpathlineto{\pgfqpoint{2.995767in}{1.852186in}}%
\pgfpathlineto{\pgfqpoint{2.995767in}{1.855938in}}%
\pgfpathlineto{\pgfqpoint{2.998432in}{1.859127in}}%
\pgfpathlineto{\pgfqpoint{2.998902in}{1.859689in}}%
\pgfpathlineto{\pgfqpoint{2.998902in}{1.863441in}}%
\pgfpathlineto{\pgfqpoint{2.998902in}{1.867192in}}%
\pgfpathlineto{\pgfqpoint{3.001566in}{1.870381in}}%
\pgfpathlineto{\pgfqpoint{3.002037in}{1.870944in}}%
\pgfpathlineto{\pgfqpoint{3.002037in}{1.874695in}}%
\pgfpathlineto{\pgfqpoint{3.002037in}{1.878447in}}%
\pgfpathlineto{\pgfqpoint{3.004701in}{1.881636in}}%
\pgfpathlineto{\pgfqpoint{3.005171in}{1.882198in}}%
\pgfpathlineto{\pgfqpoint{3.005171in}{1.885950in}}%
\pgfpathlineto{\pgfqpoint{3.005171in}{1.889702in}}%
\pgfpathlineto{\pgfqpoint{3.007836in}{1.892890in}}%
\pgfpathlineto{\pgfqpoint{3.008306in}{1.893453in}}%
\pgfpathlineto{\pgfqpoint{3.008306in}{1.897205in}}%
\pgfpathlineto{\pgfqpoint{3.008306in}{1.900956in}}%
\pgfpathlineto{\pgfqpoint{3.010971in}{1.904145in}}%
\pgfpathlineto{\pgfqpoint{3.011441in}{1.904708in}}%
\pgfpathlineto{\pgfqpoint{3.011441in}{1.908459in}}%
\pgfpathlineto{\pgfqpoint{3.011441in}{1.912211in}}%
\pgfpathlineto{\pgfqpoint{3.014105in}{1.915400in}}%
\pgfpathlineto{\pgfqpoint{3.014575in}{1.915962in}}%
\pgfpathlineto{\pgfqpoint{3.014575in}{1.919714in}}%
\pgfpathlineto{\pgfqpoint{3.014575in}{1.923465in}}%
\pgfpathlineto{\pgfqpoint{3.017240in}{1.926654in}}%
\pgfpathlineto{\pgfqpoint{3.017710in}{1.927217in}}%
\pgfpathlineto{\pgfqpoint{3.017710in}{1.930968in}}%
\pgfpathlineto{\pgfqpoint{3.017710in}{1.934720in}}%
\pgfpathlineto{\pgfqpoint{3.020375in}{1.937909in}}%
\pgfpathlineto{\pgfqpoint{3.020845in}{1.938471in}}%
\pgfpathlineto{\pgfqpoint{3.020845in}{1.942223in}}%
\pgfpathlineto{\pgfqpoint{3.020845in}{1.945975in}}%
\pgfpathlineto{\pgfqpoint{3.023510in}{1.949163in}}%
\pgfpathlineto{\pgfqpoint{3.023980in}{1.949726in}}%
\pgfpathlineto{\pgfqpoint{3.023980in}{1.953478in}}%
\pgfpathlineto{\pgfqpoint{3.023980in}{1.957229in}}%
\pgfpathlineto{\pgfqpoint{3.023980in}{1.960981in}}%
\pgfpathlineto{\pgfqpoint{3.026644in}{1.964169in}}%
\pgfpathlineto{\pgfqpoint{3.027114in}{1.964732in}}%
\pgfpathlineto{\pgfqpoint{3.027114in}{1.968484in}}%
\pgfpathlineto{\pgfqpoint{3.027114in}{1.972235in}}%
\pgfpathlineto{\pgfqpoint{3.029779in}{1.975424in}}%
\pgfpathlineto{\pgfqpoint{3.030249in}{1.975987in}}%
\pgfpathlineto{\pgfqpoint{3.030249in}{1.979738in}}%
\pgfpathlineto{\pgfqpoint{3.030249in}{1.983490in}}%
\pgfpathlineto{\pgfqpoint{3.032914in}{1.986679in}}%
\pgfpathlineto{\pgfqpoint{3.033384in}{1.987241in}}%
\pgfpathlineto{\pgfqpoint{3.033384in}{1.990993in}}%
\pgfpathlineto{\pgfqpoint{3.033384in}{1.994745in}}%
\pgfpathlineto{\pgfqpoint{3.036048in}{1.997933in}}%
\pgfpathlineto{\pgfqpoint{3.036519in}{1.998496in}}%
\pgfpathlineto{\pgfqpoint{3.036519in}{2.002248in}}%
\pgfpathlineto{\pgfqpoint{3.036519in}{2.005999in}}%
\pgfpathlineto{\pgfqpoint{3.039183in}{2.009188in}}%
\pgfpathlineto{\pgfqpoint{3.039653in}{2.009751in}}%
\pgfpathlineto{\pgfqpoint{3.039653in}{2.013502in}}%
\pgfpathlineto{\pgfqpoint{3.039653in}{2.017254in}}%
\pgfpathlineto{\pgfqpoint{3.042318in}{2.020443in}}%
\pgfpathlineto{\pgfqpoint{3.042788in}{2.021005in}}%
\pgfpathlineto{\pgfqpoint{3.042788in}{2.024757in}}%
\pgfpathlineto{\pgfqpoint{3.042788in}{2.028508in}}%
\pgfpathlineto{\pgfqpoint{3.045453in}{2.031697in}}%
\pgfpathlineto{\pgfqpoint{3.045923in}{2.032260in}}%
\pgfpathlineto{\pgfqpoint{3.045923in}{2.036011in}}%
\pgfpathlineto{\pgfqpoint{3.045923in}{2.039763in}}%
\pgfpathlineto{\pgfqpoint{3.048587in}{2.042952in}}%
\pgfpathlineto{\pgfqpoint{3.049058in}{2.043514in}}%
\pgfpathlineto{\pgfqpoint{3.049058in}{2.047266in}}%
\pgfpathlineto{\pgfqpoint{3.049058in}{2.051018in}}%
\pgfpathlineto{\pgfqpoint{3.049058in}{2.054769in}}%
\pgfpathlineto{\pgfqpoint{3.051722in}{2.057958in}}%
\pgfpathlineto{\pgfqpoint{3.052192in}{2.058521in}}%
\pgfpathlineto{\pgfqpoint{3.052192in}{2.062272in}}%
\pgfpathlineto{\pgfqpoint{3.052192in}{2.066024in}}%
\pgfpathlineto{\pgfqpoint{3.054857in}{2.069212in}}%
\pgfpathlineto{\pgfqpoint{3.055327in}{2.069775in}}%
\pgfpathlineto{\pgfqpoint{3.055327in}{2.073527in}}%
\pgfpathlineto{\pgfqpoint{3.055327in}{2.077278in}}%
\pgfpathlineto{\pgfqpoint{3.057992in}{2.080467in}}%
\pgfpathlineto{\pgfqpoint{3.058462in}{2.081030in}}%
\pgfpathlineto{\pgfqpoint{3.058462in}{2.084781in}}%
\pgfpathlineto{\pgfqpoint{3.058462in}{2.088533in}}%
\pgfpathlineto{\pgfqpoint{3.061126in}{2.091722in}}%
\pgfpathlineto{\pgfqpoint{3.061597in}{2.092284in}}%
\pgfpathlineto{\pgfqpoint{3.061597in}{2.096036in}}%
\pgfpathlineto{\pgfqpoint{3.061597in}{2.099787in}}%
\pgfpathlineto{\pgfqpoint{3.064261in}{2.102976in}}%
\pgfpathlineto{\pgfqpoint{3.064731in}{2.103539in}}%
\pgfpathlineto{\pgfqpoint{3.064731in}{2.107291in}}%
\pgfpathlineto{\pgfqpoint{3.064731in}{2.111042in}}%
\pgfpathlineto{\pgfqpoint{3.067396in}{2.114231in}}%
\pgfpathlineto{\pgfqpoint{3.067866in}{2.114794in}}%
\pgfpathlineto{\pgfqpoint{3.067866in}{2.118545in}}%
\pgfpathlineto{\pgfqpoint{3.067866in}{2.122297in}}%
\pgfpathlineto{\pgfqpoint{3.070531in}{2.125485in}}%
\pgfpathlineto{\pgfqpoint{3.071001in}{2.126048in}}%
\pgfpathlineto{\pgfqpoint{3.071001in}{2.129800in}}%
\pgfpathlineto{\pgfqpoint{3.071001in}{2.133551in}}%
\pgfpathlineto{\pgfqpoint{3.073665in}{2.136740in}}%
\pgfpathlineto{\pgfqpoint{3.074136in}{2.137303in}}%
\pgfpathlineto{\pgfqpoint{3.074136in}{2.141054in}}%
\pgfpathlineto{\pgfqpoint{3.074136in}{2.144806in}}%
\pgfpathlineto{\pgfqpoint{3.076800in}{2.147995in}}%
\pgfpathlineto{\pgfqpoint{3.077270in}{2.148557in}}%
\pgfpathlineto{\pgfqpoint{3.077270in}{2.152309in}}%
\pgfpathlineto{\pgfqpoint{3.077270in}{2.156060in}}%
\pgfpathlineto{\pgfqpoint{3.077270in}{2.159812in}}%
\pgfpathlineto{\pgfqpoint{3.079935in}{2.163001in}}%
\pgfpathlineto{\pgfqpoint{3.080405in}{2.163564in}}%
\pgfpathlineto{\pgfqpoint{3.080405in}{2.167315in}}%
\pgfpathlineto{\pgfqpoint{3.080405in}{2.171067in}}%
\pgfpathlineto{\pgfqpoint{3.083070in}{2.174255in}}%
\pgfpathlineto{\pgfqpoint{3.083540in}{2.174818in}}%
\pgfpathlineto{\pgfqpoint{3.083540in}{2.178570in}}%
\pgfpathlineto{\pgfqpoint{3.083540in}{2.182321in}}%
\pgfpathlineto{\pgfqpoint{3.086204in}{2.185510in}}%
\pgfpathlineto{\pgfqpoint{3.086675in}{2.186073in}}%
\pgfpathlineto{\pgfqpoint{3.086675in}{2.189824in}}%
\pgfpathlineto{\pgfqpoint{3.086675in}{2.193576in}}%
\pgfpathlineto{\pgfqpoint{3.089339in}{2.196765in}}%
\pgfpathlineto{\pgfqpoint{3.089809in}{2.197327in}}%
\pgfpathlineto{\pgfqpoint{3.089809in}{2.201079in}}%
\pgfpathlineto{\pgfqpoint{3.089809in}{2.204830in}}%
\pgfpathlineto{\pgfqpoint{3.092474in}{2.208019in}}%
\pgfpathlineto{\pgfqpoint{3.092944in}{2.208582in}}%
\pgfpathlineto{\pgfqpoint{3.092944in}{2.212334in}}%
\pgfpathlineto{\pgfqpoint{3.092474in}{2.212896in}}%
\pgfpathlineto{\pgfqpoint{3.089809in}{2.216085in}}%
\pgfpathlineto{\pgfqpoint{3.089339in}{2.216648in}}%
\pgfpathlineto{\pgfqpoint{3.086675in}{2.219837in}}%
\pgfpathlineto{\pgfqpoint{3.086204in}{2.220399in}}%
\pgfpathlineto{\pgfqpoint{3.083540in}{2.223588in}}%
\pgfpathlineto{\pgfqpoint{3.083070in}{2.224151in}}%
\pgfpathlineto{\pgfqpoint{3.080405in}{2.227340in}}%
\pgfpathlineto{\pgfqpoint{3.079935in}{2.227902in}}%
\pgfpathlineto{\pgfqpoint{3.077270in}{2.231091in}}%
\pgfpathlineto{\pgfqpoint{3.076800in}{2.231654in}}%
\pgfpathlineto{\pgfqpoint{3.074136in}{2.234843in}}%
\pgfpathlineto{\pgfqpoint{3.073665in}{2.235405in}}%
\pgfpathlineto{\pgfqpoint{3.071001in}{2.238594in}}%
\pgfpathlineto{\pgfqpoint{3.070531in}{2.239157in}}%
\pgfpathlineto{\pgfqpoint{3.067866in}{2.242346in}}%
\pgfpathlineto{\pgfqpoint{3.067396in}{2.242909in}}%
\pgfpathlineto{\pgfqpoint{3.064731in}{2.246097in}}%
\pgfpathlineto{\pgfqpoint{3.064261in}{2.246660in}}%
\pgfpathlineto{\pgfqpoint{3.061597in}{2.249849in}}%
\pgfpathlineto{\pgfqpoint{3.061126in}{2.250412in}}%
\pgfpathlineto{\pgfqpoint{3.058462in}{2.253600in}}%
\pgfpathlineto{\pgfqpoint{3.057992in}{2.254163in}}%
\pgfpathlineto{\pgfqpoint{3.055327in}{2.257352in}}%
\pgfpathlineto{\pgfqpoint{3.054857in}{2.257915in}}%
\pgfpathlineto{\pgfqpoint{3.052192in}{2.261103in}}%
\pgfpathlineto{\pgfqpoint{3.052192in}{2.264855in}}%
\pgfpathlineto{\pgfqpoint{3.051722in}{2.265418in}}%
\pgfpathlineto{\pgfqpoint{3.049058in}{2.268607in}}%
\pgfpathlineto{\pgfqpoint{3.048587in}{2.269169in}}%
\pgfpathlineto{\pgfqpoint{3.045923in}{2.272358in}}%
\pgfpathlineto{\pgfqpoint{3.045453in}{2.272921in}}%
\pgfpathlineto{\pgfqpoint{3.042788in}{2.276110in}}%
\pgfpathlineto{\pgfqpoint{3.042318in}{2.276672in}}%
\pgfpathlineto{\pgfqpoint{3.039653in}{2.279861in}}%
\pgfpathlineto{\pgfqpoint{3.039183in}{2.280424in}}%
\pgfpathlineto{\pgfqpoint{3.036519in}{2.283613in}}%
\pgfpathlineto{\pgfqpoint{3.036048in}{2.284175in}}%
\pgfpathlineto{\pgfqpoint{3.033384in}{2.287364in}}%
\pgfpathlineto{\pgfqpoint{3.032914in}{2.287927in}}%
\pgfpathlineto{\pgfqpoint{3.030249in}{2.291116in}}%
\pgfpathlineto{\pgfqpoint{3.029779in}{2.291678in}}%
\pgfpathlineto{\pgfqpoint{3.027114in}{2.294867in}}%
\pgfpathlineto{\pgfqpoint{3.026644in}{2.295430in}}%
\pgfpathlineto{\pgfqpoint{3.023980in}{2.298619in}}%
\pgfpathlineto{\pgfqpoint{3.023510in}{2.299182in}}%
\pgfpathlineto{\pgfqpoint{3.020845in}{2.302370in}}%
\pgfpathlineto{\pgfqpoint{3.020375in}{2.302933in}}%
\pgfpathlineto{\pgfqpoint{3.017710in}{2.306122in}}%
\pgfpathlineto{\pgfqpoint{3.017240in}{2.306685in}}%
\pgfpathlineto{\pgfqpoint{3.014575in}{2.309873in}}%
\pgfpathlineto{\pgfqpoint{3.014105in}{2.310436in}}%
\pgfpathlineto{\pgfqpoint{3.011441in}{2.313625in}}%
\pgfpathlineto{\pgfqpoint{3.010971in}{2.314188in}}%
\pgfpathlineto{\pgfqpoint{3.008306in}{2.317376in}}%
\pgfpathlineto{\pgfqpoint{3.007836in}{2.317939in}}%
\pgfpathlineto{\pgfqpoint{3.005171in}{2.321128in}}%
\pgfpathlineto{\pgfqpoint{3.004701in}{2.321691in}}%
\pgfpathlineto{\pgfqpoint{3.002037in}{2.324880in}}%
\pgfpathlineto{\pgfqpoint{3.002037in}{2.328631in}}%
\pgfpathlineto{\pgfqpoint{3.001566in}{2.329194in}}%
\pgfpathlineto{\pgfqpoint{2.998902in}{2.332383in}}%
\pgfpathlineto{\pgfqpoint{2.998432in}{2.332945in}}%
\pgfpathlineto{\pgfqpoint{2.995767in}{2.336134in}}%
\pgfpathlineto{\pgfqpoint{2.995297in}{2.336697in}}%
\pgfpathlineto{\pgfqpoint{2.992632in}{2.339886in}}%
\pgfpathlineto{\pgfqpoint{2.992162in}{2.340448in}}%
\pgfpathlineto{\pgfqpoint{2.989498in}{2.343637in}}%
\pgfpathlineto{\pgfqpoint{2.989027in}{2.344200in}}%
\pgfpathlineto{\pgfqpoint{2.986363in}{2.347389in}}%
\pgfpathlineto{\pgfqpoint{2.985893in}{2.347951in}}%
\pgfpathlineto{\pgfqpoint{2.983228in}{2.351140in}}%
\pgfpathlineto{\pgfqpoint{2.982758in}{2.351703in}}%
\pgfpathlineto{\pgfqpoint{2.980093in}{2.354892in}}%
\pgfpathlineto{\pgfqpoint{2.979623in}{2.355455in}}%
\pgfpathlineto{\pgfqpoint{2.976959in}{2.358643in}}%
\pgfpathlineto{\pgfqpoint{2.976488in}{2.359206in}}%
\pgfpathlineto{\pgfqpoint{2.973824in}{2.362395in}}%
\pgfpathlineto{\pgfqpoint{2.973354in}{2.362958in}}%
\pgfpathlineto{\pgfqpoint{2.970689in}{2.366146in}}%
\pgfpathlineto{\pgfqpoint{2.970219in}{2.366709in}}%
\pgfpathlineto{\pgfqpoint{2.967554in}{2.369898in}}%
\pgfpathlineto{\pgfqpoint{2.967084in}{2.370461in}}%
\pgfpathlineto{\pgfqpoint{2.964420in}{2.373649in}}%
\pgfpathlineto{\pgfqpoint{2.963949in}{2.374212in}}%
\pgfpathlineto{\pgfqpoint{2.961285in}{2.377401in}}%
\pgfpathlineto{\pgfqpoint{2.960815in}{2.377964in}}%
\pgfpathlineto{\pgfqpoint{2.958150in}{2.381153in}}%
\pgfpathlineto{\pgfqpoint{2.957680in}{2.381715in}}%
\pgfpathlineto{\pgfqpoint{2.955015in}{2.384904in}}%
\pgfpathlineto{\pgfqpoint{2.955015in}{2.388656in}}%
\pgfpathlineto{\pgfqpoint{2.954545in}{2.389218in}}%
\pgfpathlineto{\pgfqpoint{2.951881in}{2.392407in}}%
\pgfpathlineto{\pgfqpoint{2.951410in}{2.392970in}}%
\pgfpathlineto{\pgfqpoint{2.948746in}{2.396159in}}%
\pgfpathlineto{\pgfqpoint{2.948276in}{2.396721in}}%
\pgfpathlineto{\pgfqpoint{2.945611in}{2.399910in}}%
\pgfpathlineto{\pgfqpoint{2.945141in}{2.400473in}}%
\pgfpathlineto{\pgfqpoint{2.942476in}{2.403662in}}%
\pgfpathlineto{\pgfqpoint{2.942006in}{2.404224in}}%
\pgfpathlineto{\pgfqpoint{2.939342in}{2.407413in}}%
\pgfpathlineto{\pgfqpoint{2.938871in}{2.407976in}}%
\pgfpathlineto{\pgfqpoint{2.936207in}{2.411165in}}%
\pgfpathlineto{\pgfqpoint{2.935737in}{2.411728in}}%
\pgfpathlineto{\pgfqpoint{2.933072in}{2.414916in}}%
\pgfpathlineto{\pgfqpoint{2.932602in}{2.415479in}}%
\pgfpathlineto{\pgfqpoint{2.929937in}{2.418668in}}%
\pgfpathlineto{\pgfqpoint{2.929467in}{2.419231in}}%
\pgfpathlineto{\pgfqpoint{2.926803in}{2.422419in}}%
\pgfpathlineto{\pgfqpoint{2.926332in}{2.422982in}}%
\pgfpathlineto{\pgfqpoint{2.923668in}{2.426171in}}%
\pgfpathlineto{\pgfqpoint{2.923198in}{2.426734in}}%
\pgfpathlineto{\pgfqpoint{2.920533in}{2.429923in}}%
\pgfpathlineto{\pgfqpoint{2.920063in}{2.430485in}}%
\pgfpathlineto{\pgfqpoint{2.917398in}{2.433674in}}%
\pgfpathlineto{\pgfqpoint{2.916928in}{2.434237in}}%
\pgfpathlineto{\pgfqpoint{2.914264in}{2.437426in}}%
\pgfpathlineto{\pgfqpoint{2.913794in}{2.437988in}}%
\pgfpathlineto{\pgfqpoint{2.911129in}{2.441177in}}%
\pgfpathlineto{\pgfqpoint{2.910659in}{2.441740in}}%
\pgfpathlineto{\pgfqpoint{2.907994in}{2.444929in}}%
\pgfpathlineto{\pgfqpoint{2.907524in}{2.445491in}}%
\pgfpathlineto{\pgfqpoint{2.904860in}{2.448680in}}%
\pgfpathlineto{\pgfqpoint{2.904860in}{2.452432in}}%
\pgfpathlineto{\pgfqpoint{2.904389in}{2.452994in}}%
\pgfpathlineto{\pgfqpoint{2.901725in}{2.456183in}}%
\pgfpathlineto{\pgfqpoint{2.901255in}{2.456746in}}%
\pgfpathlineto{\pgfqpoint{2.898590in}{2.459935in}}%
\pgfpathlineto{\pgfqpoint{2.898120in}{2.460498in}}%
\pgfpathlineto{\pgfqpoint{2.895455in}{2.463686in}}%
\pgfpathlineto{\pgfqpoint{2.894985in}{2.464249in}}%
\pgfpathlineto{\pgfqpoint{2.892321in}{2.467438in}}%
\pgfpathlineto{\pgfqpoint{2.891850in}{2.468001in}}%
\pgfpathlineto{\pgfqpoint{2.889186in}{2.471189in}}%
\pgfpathlineto{\pgfqpoint{2.888716in}{2.471752in}}%
\pgfpathlineto{\pgfqpoint{2.886051in}{2.474941in}}%
\pgfpathlineto{\pgfqpoint{2.885581in}{2.475504in}}%
\pgfpathlineto{\pgfqpoint{2.882916in}{2.478692in}}%
\pgfpathlineto{\pgfqpoint{2.882446in}{2.479255in}}%
\pgfpathlineto{\pgfqpoint{2.879782in}{2.482444in}}%
\pgfpathlineto{\pgfqpoint{2.879311in}{2.483007in}}%
\pgfpathlineto{\pgfqpoint{2.876647in}{2.486196in}}%
\pgfpathlineto{\pgfqpoint{2.876177in}{2.486758in}}%
\pgfpathlineto{\pgfqpoint{2.873512in}{2.489947in}}%
\pgfpathlineto{\pgfqpoint{2.873042in}{2.490510in}}%
\pgfpathlineto{\pgfqpoint{2.870377in}{2.493699in}}%
\pgfpathlineto{\pgfqpoint{2.869907in}{2.494261in}}%
\pgfpathlineto{\pgfqpoint{2.867243in}{2.497450in}}%
\pgfpathlineto{\pgfqpoint{2.866772in}{2.498013in}}%
\pgfpathlineto{\pgfqpoint{2.864108in}{2.501202in}}%
\pgfpathlineto{\pgfqpoint{2.863638in}{2.501764in}}%
\pgfpathlineto{\pgfqpoint{2.860973in}{2.504953in}}%
\pgfpathlineto{\pgfqpoint{2.860503in}{2.505516in}}%
\pgfpathlineto{\pgfqpoint{2.857838in}{2.508705in}}%
\pgfpathlineto{\pgfqpoint{2.857838in}{2.512456in}}%
\pgfpathlineto{\pgfqpoint{2.857368in}{2.513019in}}%
\pgfpathlineto{\pgfqpoint{2.854704in}{2.516208in}}%
\pgfpathlineto{\pgfqpoint{2.854233in}{2.516771in}}%
\pgfpathlineto{\pgfqpoint{2.851569in}{2.519959in}}%
\pgfpathlineto{\pgfqpoint{2.851099in}{2.520522in}}%
\pgfpathlineto{\pgfqpoint{2.848434in}{2.523711in}}%
\pgfpathlineto{\pgfqpoint{2.847964in}{2.524274in}}%
\pgfpathlineto{\pgfqpoint{2.845299in}{2.527462in}}%
\pgfpathlineto{\pgfqpoint{2.844829in}{2.528025in}}%
\pgfpathlineto{\pgfqpoint{2.842165in}{2.531214in}}%
\pgfpathlineto{\pgfqpoint{2.841694in}{2.531777in}}%
\pgfpathlineto{\pgfqpoint{2.839030in}{2.534965in}}%
\pgfpathlineto{\pgfqpoint{2.838560in}{2.535528in}}%
\pgfpathlineto{\pgfqpoint{2.835895in}{2.538717in}}%
\pgfpathlineto{\pgfqpoint{2.835425in}{2.539280in}}%
\pgfpathlineto{\pgfqpoint{2.832760in}{2.542469in}}%
\pgfpathlineto{\pgfqpoint{2.832290in}{2.543031in}}%
\pgfpathlineto{\pgfqpoint{2.829626in}{2.546220in}}%
\pgfpathlineto{\pgfqpoint{2.829155in}{2.546783in}}%
\pgfpathlineto{\pgfqpoint{2.826491in}{2.549972in}}%
\pgfpathlineto{\pgfqpoint{2.826021in}{2.550534in}}%
\pgfpathlineto{\pgfqpoint{2.823356in}{2.553723in}}%
\pgfpathlineto{\pgfqpoint{2.822886in}{2.554286in}}%
\pgfpathlineto{\pgfqpoint{2.820221in}{2.557475in}}%
\pgfpathlineto{\pgfqpoint{2.819751in}{2.558037in}}%
\pgfpathlineto{\pgfqpoint{2.817087in}{2.561226in}}%
\pgfpathlineto{\pgfqpoint{2.816617in}{2.561789in}}%
\pgfpathlineto{\pgfqpoint{2.813952in}{2.564978in}}%
\pgfpathlineto{\pgfqpoint{2.813482in}{2.565540in}}%
\pgfpathlineto{\pgfqpoint{2.810817in}{2.568729in}}%
\pgfpathlineto{\pgfqpoint{2.810347in}{2.569292in}}%
\pgfpathlineto{\pgfqpoint{2.807682in}{2.572481in}}%
\pgfpathlineto{\pgfqpoint{2.807682in}{2.576232in}}%
\pgfpathlineto{\pgfqpoint{2.807212in}{2.576795in}}%
\pgfpathlineto{\pgfqpoint{2.804548in}{2.579984in}}%
\pgfpathlineto{\pgfqpoint{2.804078in}{2.580547in}}%
\pgfpathlineto{\pgfqpoint{2.801413in}{2.583735in}}%
\pgfpathlineto{\pgfqpoint{2.800943in}{2.584298in}}%
\pgfpathlineto{\pgfqpoint{2.798278in}{2.587487in}}%
\pgfpathlineto{\pgfqpoint{2.797808in}{2.588050in}}%
\pgfpathlineto{\pgfqpoint{2.795144in}{2.591238in}}%
\pgfpathlineto{\pgfqpoint{2.794673in}{2.591801in}}%
\pgfpathlineto{\pgfqpoint{2.792009in}{2.594990in}}%
\pgfpathlineto{\pgfqpoint{2.791539in}{2.595553in}}%
\pgfpathlineto{\pgfqpoint{2.788874in}{2.598742in}}%
\pgfpathlineto{\pgfqpoint{2.788404in}{2.599304in}}%
\pgfpathlineto{\pgfqpoint{2.785739in}{2.602493in}}%
\pgfpathlineto{\pgfqpoint{2.785269in}{2.603056in}}%
\pgfpathlineto{\pgfqpoint{2.782605in}{2.606245in}}%
\pgfpathlineto{\pgfqpoint{2.782134in}{2.606807in}}%
\pgfpathlineto{\pgfqpoint{2.779470in}{2.609996in}}%
\pgfpathlineto{\pgfqpoint{2.779000in}{2.610559in}}%
\pgfpathlineto{\pgfqpoint{2.776335in}{2.613748in}}%
\pgfpathlineto{\pgfqpoint{2.775865in}{2.614310in}}%
\pgfpathlineto{\pgfqpoint{2.773200in}{2.617499in}}%
\pgfpathlineto{\pgfqpoint{2.772730in}{2.618062in}}%
\pgfpathlineto{\pgfqpoint{2.770066in}{2.621251in}}%
\pgfpathlineto{\pgfqpoint{2.769595in}{2.621813in}}%
\pgfpathlineto{\pgfqpoint{2.766931in}{2.625002in}}%
\pgfpathlineto{\pgfqpoint{2.766461in}{2.625565in}}%
\pgfpathlineto{\pgfqpoint{2.763796in}{2.628754in}}%
\pgfpathlineto{\pgfqpoint{2.763326in}{2.629317in}}%
\pgfpathlineto{\pgfqpoint{2.760661in}{2.632505in}}%
\pgfpathlineto{\pgfqpoint{2.760661in}{2.636257in}}%
\pgfpathlineto{\pgfqpoint{2.760191in}{2.636820in}}%
\pgfpathlineto{\pgfqpoint{2.757527in}{2.640008in}}%
\pgfpathlineto{\pgfqpoint{2.757056in}{2.640571in}}%
\pgfpathlineto{\pgfqpoint{2.754392in}{2.643760in}}%
\pgfpathlineto{\pgfqpoint{2.753922in}{2.644323in}}%
\pgfpathlineto{\pgfqpoint{2.751257in}{2.647512in}}%
\pgfpathlineto{\pgfqpoint{2.750787in}{2.648074in}}%
\pgfpathlineto{\pgfqpoint{2.748122in}{2.651263in}}%
\pgfpathlineto{\pgfqpoint{2.747652in}{2.651826in}}%
\pgfpathlineto{\pgfqpoint{2.744988in}{2.655015in}}%
\pgfpathlineto{\pgfqpoint{2.744517in}{2.655577in}}%
\pgfpathlineto{\pgfqpoint{2.741853in}{2.658766in}}%
\pgfpathlineto{\pgfqpoint{2.741383in}{2.659329in}}%
\pgfpathlineto{\pgfqpoint{2.738718in}{2.662518in}}%
\pgfpathlineto{\pgfqpoint{2.738248in}{2.663080in}}%
\pgfpathlineto{\pgfqpoint{2.735583in}{2.666269in}}%
\pgfpathlineto{\pgfqpoint{2.735113in}{2.666832in}}%
\pgfpathlineto{\pgfqpoint{2.732449in}{2.670021in}}%
\pgfpathlineto{\pgfqpoint{2.731978in}{2.670583in}}%
\pgfpathlineto{\pgfqpoint{2.729314in}{2.673772in}}%
\pgfpathlineto{\pgfqpoint{2.728844in}{2.674335in}}%
\pgfpathlineto{\pgfqpoint{2.726179in}{2.677524in}}%
\pgfpathlineto{\pgfqpoint{2.725709in}{2.678087in}}%
\pgfpathlineto{\pgfqpoint{2.723044in}{2.681275in}}%
\pgfpathlineto{\pgfqpoint{2.722574in}{2.681838in}}%
\pgfpathlineto{\pgfqpoint{2.719910in}{2.685027in}}%
\pgfpathlineto{\pgfqpoint{2.719439in}{2.685590in}}%
\pgfpathlineto{\pgfqpoint{2.716775in}{2.688778in}}%
\pgfpathlineto{\pgfqpoint{2.716305in}{2.689341in}}%
\pgfpathlineto{\pgfqpoint{2.713640in}{2.692530in}}%
\pgfpathlineto{\pgfqpoint{2.713170in}{2.693093in}}%
\pgfpathlineto{\pgfqpoint{2.710505in}{2.696281in}}%
\pgfpathlineto{\pgfqpoint{2.710505in}{2.700033in}}%
\pgfpathlineto{\pgfqpoint{2.710035in}{2.700596in}}%
\pgfpathlineto{\pgfqpoint{2.707371in}{2.703785in}}%
\pgfpathlineto{\pgfqpoint{2.706901in}{2.704347in}}%
\pgfpathlineto{\pgfqpoint{2.704236in}{2.707536in}}%
\pgfpathlineto{\pgfqpoint{2.703766in}{2.708099in}}%
\pgfpathlineto{\pgfqpoint{2.701101in}{2.711288in}}%
\pgfpathlineto{\pgfqpoint{2.700631in}{2.711850in}}%
\pgfpathlineto{\pgfqpoint{2.697967in}{2.715039in}}%
\pgfpathlineto{\pgfqpoint{2.697496in}{2.715602in}}%
\pgfpathlineto{\pgfqpoint{2.694832in}{2.718791in}}%
\pgfpathlineto{\pgfqpoint{2.694362in}{2.719353in}}%
\pgfpathlineto{\pgfqpoint{2.691697in}{2.722542in}}%
\pgfpathlineto{\pgfqpoint{2.691227in}{2.723105in}}%
\pgfpathlineto{\pgfqpoint{2.688562in}{2.726294in}}%
\pgfpathlineto{\pgfqpoint{2.688092in}{2.726856in}}%
\pgfpathlineto{\pgfqpoint{2.685428in}{2.730045in}}%
\pgfpathlineto{\pgfqpoint{2.684957in}{2.730608in}}%
\pgfpathlineto{\pgfqpoint{2.682293in}{2.733797in}}%
\pgfpathlineto{\pgfqpoint{2.681823in}{2.734360in}}%
\pgfpathlineto{\pgfqpoint{2.679158in}{2.737548in}}%
\pgfpathlineto{\pgfqpoint{2.678688in}{2.738111in}}%
\pgfpathlineto{\pgfqpoint{2.676023in}{2.741300in}}%
\pgfpathlineto{\pgfqpoint{2.675553in}{2.741863in}}%
\pgfpathlineto{\pgfqpoint{2.672889in}{2.745051in}}%
\pgfpathlineto{\pgfqpoint{2.672418in}{2.745614in}}%
\pgfpathlineto{\pgfqpoint{2.669284in}{2.745614in}}%
\pgfpathlineto{\pgfqpoint{2.666149in}{2.745614in}}%
\pgfpathlineto{\pgfqpoint{2.663014in}{2.745614in}}%
\pgfpathlineto{\pgfqpoint{2.659879in}{2.745614in}}%
\pgfpathlineto{\pgfqpoint{2.659409in}{2.745051in}}%
\pgfpathlineto{\pgfqpoint{2.656745in}{2.741863in}}%
\pgfpathlineto{\pgfqpoint{2.653610in}{2.741863in}}%
\pgfpathlineto{\pgfqpoint{2.650475in}{2.741863in}}%
\pgfpathlineto{\pgfqpoint{2.647340in}{2.741863in}}%
\pgfpathlineto{\pgfqpoint{2.646870in}{2.741300in}}%
\pgfpathlineto{\pgfqpoint{2.644206in}{2.738111in}}%
\pgfpathlineto{\pgfqpoint{2.641071in}{2.738111in}}%
\pgfpathlineto{\pgfqpoint{2.637936in}{2.738111in}}%
\pgfpathlineto{\pgfqpoint{2.634801in}{2.738111in}}%
\pgfpathlineto{\pgfqpoint{2.634331in}{2.737548in}}%
\pgfpathlineto{\pgfqpoint{2.631667in}{2.734360in}}%
\pgfpathlineto{\pgfqpoint{2.628532in}{2.734360in}}%
\pgfpathlineto{\pgfqpoint{2.625397in}{2.734360in}}%
\pgfpathlineto{\pgfqpoint{2.622262in}{2.734360in}}%
\pgfpathlineto{\pgfqpoint{2.619128in}{2.734360in}}%
\pgfpathlineto{\pgfqpoint{2.618658in}{2.733797in}}%
\pgfpathlineto{\pgfqpoint{2.615993in}{2.730608in}}%
\pgfpathlineto{\pgfqpoint{2.612858in}{2.730608in}}%
\pgfpathlineto{\pgfqpoint{2.609724in}{2.730608in}}%
\pgfpathlineto{\pgfqpoint{2.606589in}{2.730608in}}%
\pgfpathlineto{\pgfqpoint{2.606119in}{2.730045in}}%
\pgfpathlineto{\pgfqpoint{2.603454in}{2.726856in}}%
\pgfpathlineto{\pgfqpoint{2.600319in}{2.726856in}}%
\pgfpathlineto{\pgfqpoint{2.597185in}{2.726856in}}%
\pgfpathlineto{\pgfqpoint{2.594050in}{2.726856in}}%
\pgfpathlineto{\pgfqpoint{2.593580in}{2.726294in}}%
\pgfpathlineto{\pgfqpoint{2.590915in}{2.723105in}}%
\pgfpathlineto{\pgfqpoint{2.587780in}{2.723105in}}%
\pgfpathlineto{\pgfqpoint{2.584646in}{2.723105in}}%
\pgfpathlineto{\pgfqpoint{2.581511in}{2.723105in}}%
\pgfpathlineto{\pgfqpoint{2.581041in}{2.722542in}}%
\pgfpathlineto{\pgfqpoint{2.578376in}{2.719353in}}%
\pgfpathlineto{\pgfqpoint{2.575241in}{2.719353in}}%
\pgfpathlineto{\pgfqpoint{2.572107in}{2.719353in}}%
\pgfpathlineto{\pgfqpoint{2.568972in}{2.719353in}}%
\pgfpathlineto{\pgfqpoint{2.568502in}{2.718791in}}%
\pgfpathlineto{\pgfqpoint{2.565837in}{2.715602in}}%
\pgfpathlineto{\pgfqpoint{2.562702in}{2.715602in}}%
\pgfpathlineto{\pgfqpoint{2.559568in}{2.715602in}}%
\pgfpathlineto{\pgfqpoint{2.556433in}{2.715602in}}%
\pgfpathlineto{\pgfqpoint{2.553298in}{2.715602in}}%
\pgfpathlineto{\pgfqpoint{2.552828in}{2.715039in}}%
\pgfpathlineto{\pgfqpoint{2.550163in}{2.711850in}}%
\pgfpathlineto{\pgfqpoint{2.547029in}{2.711850in}}%
\pgfpathlineto{\pgfqpoint{2.543894in}{2.711850in}}%
\pgfpathlineto{\pgfqpoint{2.540759in}{2.711850in}}%
\pgfpathlineto{\pgfqpoint{2.540289in}{2.711288in}}%
\pgfpathlineto{\pgfqpoint{2.537624in}{2.708099in}}%
\pgfpathlineto{\pgfqpoint{2.534490in}{2.708099in}}%
\pgfpathlineto{\pgfqpoint{2.531355in}{2.708099in}}%
\pgfpathlineto{\pgfqpoint{2.528220in}{2.708099in}}%
\pgfpathlineto{\pgfqpoint{2.527750in}{2.707536in}}%
\pgfpathlineto{\pgfqpoint{2.525085in}{2.704347in}}%
\pgfpathlineto{\pgfqpoint{2.521951in}{2.704347in}}%
\pgfpathlineto{\pgfqpoint{2.518816in}{2.704347in}}%
\pgfpathlineto{\pgfqpoint{2.515681in}{2.704347in}}%
\pgfpathlineto{\pgfqpoint{2.515211in}{2.703785in}}%
\pgfpathlineto{\pgfqpoint{2.512547in}{2.700596in}}%
\pgfpathlineto{\pgfqpoint{2.509412in}{2.700596in}}%
\pgfpathlineto{\pgfqpoint{2.506277in}{2.700596in}}%
\pgfpathlineto{\pgfqpoint{2.503142in}{2.700596in}}%
\pgfpathlineto{\pgfqpoint{2.500008in}{2.700596in}}%
\pgfpathlineto{\pgfqpoint{2.499537in}{2.700033in}}%
\pgfpathlineto{\pgfqpoint{2.496873in}{2.696844in}}%
\pgfpathlineto{\pgfqpoint{2.493738in}{2.696844in}}%
\pgfpathlineto{\pgfqpoint{2.490603in}{2.696844in}}%
\pgfpathlineto{\pgfqpoint{2.487469in}{2.696844in}}%
\pgfpathlineto{\pgfqpoint{2.486998in}{2.696281in}}%
\pgfpathlineto{\pgfqpoint{2.484334in}{2.693093in}}%
\pgfpathlineto{\pgfqpoint{2.481199in}{2.693093in}}%
\pgfpathlineto{\pgfqpoint{2.478064in}{2.693093in}}%
\pgfpathlineto{\pgfqpoint{2.474930in}{2.693093in}}%
\pgfpathlineto{\pgfqpoint{2.474459in}{2.692530in}}%
\pgfpathlineto{\pgfqpoint{2.471795in}{2.689341in}}%
\pgfpathlineto{\pgfqpoint{2.468660in}{2.689341in}}%
\pgfpathlineto{\pgfqpoint{2.465525in}{2.689341in}}%
\pgfpathlineto{\pgfqpoint{2.462391in}{2.689341in}}%
\pgfpathlineto{\pgfqpoint{2.461920in}{2.688778in}}%
\pgfpathlineto{\pgfqpoint{2.459256in}{2.685590in}}%
\pgfpathlineto{\pgfqpoint{2.456121in}{2.685590in}}%
\pgfpathlineto{\pgfqpoint{2.452986in}{2.685590in}}%
\pgfpathlineto{\pgfqpoint{2.449852in}{2.685590in}}%
\pgfpathlineto{\pgfqpoint{2.449381in}{2.685027in}}%
\pgfpathlineto{\pgfqpoint{2.446717in}{2.681838in}}%
\pgfpathlineto{\pgfqpoint{2.443582in}{2.681838in}}%
\pgfpathlineto{\pgfqpoint{2.440447in}{2.681838in}}%
\pgfpathlineto{\pgfqpoint{2.437313in}{2.681838in}}%
\pgfpathlineto{\pgfqpoint{2.434178in}{2.681838in}}%
\pgfpathlineto{\pgfqpoint{2.433708in}{2.681275in}}%
\pgfpathlineto{\pgfqpoint{2.431043in}{2.678087in}}%
\pgfpathlineto{\pgfqpoint{2.427908in}{2.678087in}}%
\pgfpathlineto{\pgfqpoint{2.424774in}{2.678087in}}%
\pgfpathlineto{\pgfqpoint{2.421639in}{2.678087in}}%
\pgfpathlineto{\pgfqpoint{2.421169in}{2.677524in}}%
\pgfpathlineto{\pgfqpoint{2.418504in}{2.674335in}}%
\pgfpathlineto{\pgfqpoint{2.415369in}{2.674335in}}%
\pgfpathlineto{\pgfqpoint{2.412235in}{2.674335in}}%
\pgfpathlineto{\pgfqpoint{2.409100in}{2.674335in}}%
\pgfpathlineto{\pgfqpoint{2.408630in}{2.673772in}}%
\pgfpathlineto{\pgfqpoint{2.405965in}{2.670583in}}%
\pgfpathlineto{\pgfqpoint{2.402831in}{2.670583in}}%
\pgfpathlineto{\pgfqpoint{2.399696in}{2.670583in}}%
\pgfpathlineto{\pgfqpoint{2.396561in}{2.670583in}}%
\pgfpathlineto{\pgfqpoint{2.396091in}{2.670021in}}%
\pgfpathlineto{\pgfqpoint{2.393426in}{2.666832in}}%
\pgfpathlineto{\pgfqpoint{2.390292in}{2.666832in}}%
\pgfpathlineto{\pgfqpoint{2.387157in}{2.666832in}}%
\pgfpathlineto{\pgfqpoint{2.384022in}{2.666832in}}%
\pgfpathlineto{\pgfqpoint{2.380887in}{2.666832in}}%
\pgfpathlineto{\pgfqpoint{2.380417in}{2.666269in}}%
\pgfpathlineto{\pgfqpoint{2.377753in}{2.663080in}}%
\pgfpathlineto{\pgfqpoint{2.374618in}{2.663080in}}%
\pgfpathlineto{\pgfqpoint{2.371483in}{2.663080in}}%
\pgfpathlineto{\pgfqpoint{2.368348in}{2.663080in}}%
\pgfpathlineto{\pgfqpoint{2.367878in}{2.662518in}}%
\pgfpathlineto{\pgfqpoint{2.365214in}{2.659329in}}%
\pgfpathlineto{\pgfqpoint{2.362079in}{2.659329in}}%
\pgfpathlineto{\pgfqpoint{2.358944in}{2.659329in}}%
\pgfpathlineto{\pgfqpoint{2.355809in}{2.659329in}}%
\pgfpathlineto{\pgfqpoint{2.355339in}{2.658766in}}%
\pgfpathlineto{\pgfqpoint{2.352675in}{2.655577in}}%
\pgfpathlineto{\pgfqpoint{2.349540in}{2.655577in}}%
\pgfpathlineto{\pgfqpoint{2.346405in}{2.655577in}}%
\pgfpathlineto{\pgfqpoint{2.343270in}{2.655577in}}%
\pgfpathlineto{\pgfqpoint{2.342800in}{2.655015in}}%
\pgfpathlineto{\pgfqpoint{2.340136in}{2.651826in}}%
\pgfpathlineto{\pgfqpoint{2.337001in}{2.651826in}}%
\pgfpathlineto{\pgfqpoint{2.333866in}{2.651826in}}%
\pgfpathlineto{\pgfqpoint{2.330731in}{2.651826in}}%
\pgfpathlineto{\pgfqpoint{2.327597in}{2.651826in}}%
\pgfpathlineto{\pgfqpoint{2.327127in}{2.651263in}}%
\pgfpathlineto{\pgfqpoint{2.324462in}{2.648074in}}%
\pgfpathlineto{\pgfqpoint{2.321327in}{2.648074in}}%
\pgfpathlineto{\pgfqpoint{2.318192in}{2.648074in}}%
\pgfpathlineto{\pgfqpoint{2.315058in}{2.648074in}}%
\pgfpathlineto{\pgfqpoint{2.314588in}{2.647512in}}%
\pgfpathlineto{\pgfqpoint{2.311923in}{2.644323in}}%
\pgfpathlineto{\pgfqpoint{2.308788in}{2.644323in}}%
\pgfpathlineto{\pgfqpoint{2.305654in}{2.644323in}}%
\pgfpathlineto{\pgfqpoint{2.302519in}{2.644323in}}%
\pgfpathlineto{\pgfqpoint{2.302049in}{2.643760in}}%
\pgfpathlineto{\pgfqpoint{2.299384in}{2.640571in}}%
\pgfpathlineto{\pgfqpoint{2.296249in}{2.640571in}}%
\pgfpathlineto{\pgfqpoint{2.293115in}{2.640571in}}%
\pgfpathlineto{\pgfqpoint{2.289980in}{2.640571in}}%
\pgfpathlineto{\pgfqpoint{2.289510in}{2.640008in}}%
\pgfpathlineto{\pgfqpoint{2.286845in}{2.636820in}}%
\pgfpathlineto{\pgfqpoint{2.283710in}{2.636820in}}%
\pgfpathlineto{\pgfqpoint{2.280576in}{2.636820in}}%
\pgfpathlineto{\pgfqpoint{2.277441in}{2.636820in}}%
\pgfpathlineto{\pgfqpoint{2.276971in}{2.636257in}}%
\pgfpathlineto{\pgfqpoint{2.274306in}{2.633068in}}%
\pgfpathlineto{\pgfqpoint{2.271171in}{2.633068in}}%
\pgfpathlineto{\pgfqpoint{2.268037in}{2.633068in}}%
\pgfpathlineto{\pgfqpoint{2.264902in}{2.633068in}}%
\pgfpathlineto{\pgfqpoint{2.261767in}{2.633068in}}%
\pgfpathlineto{\pgfqpoint{2.261297in}{2.632505in}}%
\pgfpathlineto{\pgfqpoint{2.258632in}{2.629317in}}%
\pgfpathlineto{\pgfqpoint{2.255498in}{2.629317in}}%
\pgfpathlineto{\pgfqpoint{2.252363in}{2.629317in}}%
\pgfpathlineto{\pgfqpoint{2.249228in}{2.629317in}}%
\pgfpathlineto{\pgfqpoint{2.248758in}{2.628754in}}%
\pgfpathlineto{\pgfqpoint{2.246093in}{2.625565in}}%
\pgfpathlineto{\pgfqpoint{2.242959in}{2.625565in}}%
\pgfpathlineto{\pgfqpoint{2.239824in}{2.625565in}}%
\pgfpathlineto{\pgfqpoint{2.236689in}{2.625565in}}%
\pgfpathlineto{\pgfqpoint{2.236219in}{2.625002in}}%
\pgfpathlineto{\pgfqpoint{2.233554in}{2.621813in}}%
\pgfpathlineto{\pgfqpoint{2.230420in}{2.621813in}}%
\pgfpathlineto{\pgfqpoint{2.227285in}{2.621813in}}%
\pgfpathlineto{\pgfqpoint{2.224150in}{2.621813in}}%
\pgfpathlineto{\pgfqpoint{2.223680in}{2.621251in}}%
\pgfpathlineto{\pgfqpoint{2.221015in}{2.618062in}}%
\pgfpathlineto{\pgfqpoint{2.217881in}{2.618062in}}%
\pgfpathlineto{\pgfqpoint{2.214746in}{2.618062in}}%
\pgfpathlineto{\pgfqpoint{2.211611in}{2.618062in}}%
\pgfpathlineto{\pgfqpoint{2.208477in}{2.618062in}}%
\pgfpathlineto{\pgfqpoint{2.208006in}{2.617499in}}%
\pgfpathlineto{\pgfqpoint{2.205342in}{2.614310in}}%
\pgfpathlineto{\pgfqpoint{2.202207in}{2.614310in}}%
\pgfpathlineto{\pgfqpoint{2.199072in}{2.614310in}}%
\pgfpathlineto{\pgfqpoint{2.195938in}{2.614310in}}%
\pgfpathlineto{\pgfqpoint{2.195467in}{2.613748in}}%
\pgfpathlineto{\pgfqpoint{2.192803in}{2.610559in}}%
\pgfpathlineto{\pgfqpoint{2.189668in}{2.610559in}}%
\pgfpathlineto{\pgfqpoint{2.186533in}{2.610559in}}%
\pgfpathlineto{\pgfqpoint{2.183399in}{2.610559in}}%
\pgfpathlineto{\pgfqpoint{2.182928in}{2.609996in}}%
\pgfpathlineto{\pgfqpoint{2.180264in}{2.606807in}}%
\pgfpathlineto{\pgfqpoint{2.177129in}{2.606807in}}%
\pgfpathlineto{\pgfqpoint{2.173994in}{2.606807in}}%
\pgfpathlineto{\pgfqpoint{2.170860in}{2.606807in}}%
\pgfpathlineto{\pgfqpoint{2.170389in}{2.606245in}}%
\pgfpathlineto{\pgfqpoint{2.167725in}{2.603056in}}%
\pgfpathlineto{\pgfqpoint{2.164590in}{2.603056in}}%
\pgfpathlineto{\pgfqpoint{2.161455in}{2.603056in}}%
\pgfpathlineto{\pgfqpoint{2.158321in}{2.603056in}}%
\pgfpathlineto{\pgfqpoint{2.155186in}{2.603056in}}%
\pgfpathlineto{\pgfqpoint{2.154716in}{2.602493in}}%
\pgfpathlineto{\pgfqpoint{2.152051in}{2.599304in}}%
\pgfpathlineto{\pgfqpoint{2.148916in}{2.599304in}}%
\pgfpathlineto{\pgfqpoint{2.145782in}{2.599304in}}%
\pgfpathlineto{\pgfqpoint{2.142647in}{2.599304in}}%
\pgfpathlineto{\pgfqpoint{2.142177in}{2.598742in}}%
\pgfpathlineto{\pgfqpoint{2.139512in}{2.595553in}}%
\pgfpathlineto{\pgfqpoint{2.136377in}{2.595553in}}%
\pgfpathlineto{\pgfqpoint{2.133243in}{2.595553in}}%
\pgfpathlineto{\pgfqpoint{2.130108in}{2.595553in}}%
\pgfpathlineto{\pgfqpoint{2.129638in}{2.594990in}}%
\pgfpathlineto{\pgfqpoint{2.126973in}{2.591801in}}%
\pgfpathlineto{\pgfqpoint{2.123838in}{2.591801in}}%
\pgfpathlineto{\pgfqpoint{2.120704in}{2.591801in}}%
\pgfpathlineto{\pgfqpoint{2.117569in}{2.591801in}}%
\pgfpathlineto{\pgfqpoint{2.117099in}{2.591238in}}%
\pgfpathlineto{\pgfqpoint{2.114434in}{2.588050in}}%
\pgfpathlineto{\pgfqpoint{2.111299in}{2.588050in}}%
\pgfpathlineto{\pgfqpoint{2.108165in}{2.588050in}}%
\pgfpathlineto{\pgfqpoint{2.105030in}{2.588050in}}%
\pgfpathlineto{\pgfqpoint{2.104560in}{2.587487in}}%
\pgfpathlineto{\pgfqpoint{2.101895in}{2.584298in}}%
\pgfpathlineto{\pgfqpoint{2.098761in}{2.584298in}}%
\pgfpathlineto{\pgfqpoint{2.095626in}{2.584298in}}%
\pgfpathlineto{\pgfqpoint{2.092491in}{2.584298in}}%
\pgfpathlineto{\pgfqpoint{2.089356in}{2.584298in}}%
\pgfpathlineto{\pgfqpoint{2.088886in}{2.583735in}}%
\pgfpathlineto{\pgfqpoint{2.086222in}{2.580547in}}%
\pgfpathlineto{\pgfqpoint{2.083087in}{2.580547in}}%
\pgfpathlineto{\pgfqpoint{2.079952in}{2.580547in}}%
\pgfpathlineto{\pgfqpoint{2.076817in}{2.580547in}}%
\pgfpathlineto{\pgfqpoint{2.076347in}{2.579984in}}%
\pgfpathlineto{\pgfqpoint{2.073683in}{2.576795in}}%
\pgfpathlineto{\pgfqpoint{2.070548in}{2.576795in}}%
\pgfpathlineto{\pgfqpoint{2.067413in}{2.576795in}}%
\pgfpathlineto{\pgfqpoint{2.064278in}{2.576795in}}%
\pgfpathlineto{\pgfqpoint{2.063808in}{2.576232in}}%
\pgfpathlineto{\pgfqpoint{2.061144in}{2.573044in}}%
\pgfpathlineto{\pgfqpoint{2.058009in}{2.573044in}}%
\pgfpathlineto{\pgfqpoint{2.054874in}{2.573044in}}%
\pgfpathlineto{\pgfqpoint{2.051739in}{2.573044in}}%
\pgfpathlineto{\pgfqpoint{2.051269in}{2.572481in}}%
\pgfpathlineto{\pgfqpoint{2.048605in}{2.569292in}}%
\pgfpathlineto{\pgfqpoint{2.045470in}{2.569292in}}%
\pgfpathlineto{\pgfqpoint{2.042335in}{2.569292in}}%
\pgfpathlineto{\pgfqpoint{2.039200in}{2.569292in}}%
\pgfpathlineto{\pgfqpoint{2.036066in}{2.569292in}}%
\pgfpathlineto{\pgfqpoint{2.035595in}{2.568729in}}%
\pgfpathlineto{\pgfqpoint{2.032931in}{2.565540in}}%
\pgfpathlineto{\pgfqpoint{2.029796in}{2.565540in}}%
\pgfpathlineto{\pgfqpoint{2.026661in}{2.565540in}}%
\pgfpathlineto{\pgfqpoint{2.023527in}{2.565540in}}%
\pgfpathlineto{\pgfqpoint{2.023056in}{2.564978in}}%
\pgfpathlineto{\pgfqpoint{2.020392in}{2.561789in}}%
\pgfpathlineto{\pgfqpoint{2.017257in}{2.561789in}}%
\pgfpathlineto{\pgfqpoint{2.014122in}{2.561789in}}%
\pgfpathlineto{\pgfqpoint{2.010988in}{2.561789in}}%
\pgfpathlineto{\pgfqpoint{2.010518in}{2.561226in}}%
\pgfpathlineto{\pgfqpoint{2.007853in}{2.558037in}}%
\pgfpathlineto{\pgfqpoint{2.004718in}{2.558037in}}%
\pgfpathlineto{\pgfqpoint{2.001584in}{2.558037in}}%
\pgfpathlineto{\pgfqpoint{1.998449in}{2.558037in}}%
\pgfpathlineto{\pgfqpoint{1.997979in}{2.557475in}}%
\pgfpathlineto{\pgfqpoint{1.995314in}{2.554286in}}%
\pgfpathlineto{\pgfqpoint{1.992179in}{2.554286in}}%
\pgfpathlineto{\pgfqpoint{1.989045in}{2.554286in}}%
\pgfpathlineto{\pgfqpoint{1.985910in}{2.554286in}}%
\pgfpathlineto{\pgfqpoint{1.982775in}{2.554286in}}%
\pgfpathlineto{\pgfqpoint{1.982305in}{2.553723in}}%
\pgfpathlineto{\pgfqpoint{1.979640in}{2.550534in}}%
\pgfpathlineto{\pgfqpoint{1.976506in}{2.550534in}}%
\pgfpathlineto{\pgfqpoint{1.973371in}{2.550534in}}%
\pgfpathlineto{\pgfqpoint{1.970236in}{2.550534in}}%
\pgfpathlineto{\pgfqpoint{1.969766in}{2.549972in}}%
\pgfpathlineto{\pgfqpoint{1.967101in}{2.546783in}}%
\pgfpathlineto{\pgfqpoint{1.963967in}{2.546783in}}%
\pgfpathlineto{\pgfqpoint{1.960832in}{2.546783in}}%
\pgfpathlineto{\pgfqpoint{1.957697in}{2.546783in}}%
\pgfpathlineto{\pgfqpoint{1.957227in}{2.546220in}}%
\pgfpathlineto{\pgfqpoint{1.954562in}{2.543031in}}%
\pgfpathlineto{\pgfqpoint{1.951428in}{2.543031in}}%
\pgfpathlineto{\pgfqpoint{1.948293in}{2.543031in}}%
\pgfpathlineto{\pgfqpoint{1.945158in}{2.543031in}}%
\pgfpathlineto{\pgfqpoint{1.944688in}{2.542469in}}%
\pgfpathlineto{\pgfqpoint{1.942023in}{2.539280in}}%
\pgfpathlineto{\pgfqpoint{1.938889in}{2.539280in}}%
\pgfpathlineto{\pgfqpoint{1.935754in}{2.539280in}}%
\pgfpathlineto{\pgfqpoint{1.932619in}{2.539280in}}%
\pgfpathlineto{\pgfqpoint{1.932149in}{2.538717in}}%
\pgfpathlineto{\pgfqpoint{1.929484in}{2.535528in}}%
\pgfpathlineto{\pgfqpoint{1.926350in}{2.535528in}}%
\pgfpathlineto{\pgfqpoint{1.923215in}{2.535528in}}%
\pgfpathlineto{\pgfqpoint{1.920080in}{2.535528in}}%
\pgfpathlineto{\pgfqpoint{1.916945in}{2.535528in}}%
\pgfpathlineto{\pgfqpoint{1.916475in}{2.534965in}}%
\pgfpathlineto{\pgfqpoint{1.913811in}{2.531777in}}%
\pgfpathlineto{\pgfqpoint{1.910676in}{2.531777in}}%
\pgfpathlineto{\pgfqpoint{1.907541in}{2.531777in}}%
\pgfpathlineto{\pgfqpoint{1.904407in}{2.531777in}}%
\pgfpathlineto{\pgfqpoint{1.903936in}{2.531214in}}%
\pgfpathlineto{\pgfqpoint{1.901272in}{2.528025in}}%
\pgfpathlineto{\pgfqpoint{1.898137in}{2.528025in}}%
\pgfpathlineto{\pgfqpoint{1.895002in}{2.528025in}}%
\pgfpathlineto{\pgfqpoint{1.891868in}{2.528025in}}%
\pgfpathlineto{\pgfqpoint{1.891397in}{2.527462in}}%
\pgfpathlineto{\pgfqpoint{1.888733in}{2.524274in}}%
\pgfpathlineto{\pgfqpoint{1.885598in}{2.524274in}}%
\pgfpathlineto{\pgfqpoint{1.882463in}{2.524274in}}%
\pgfpathlineto{\pgfqpoint{1.879329in}{2.524274in}}%
\pgfpathlineto{\pgfqpoint{1.878858in}{2.523711in}}%
\pgfpathlineto{\pgfqpoint{1.876194in}{2.520522in}}%
\pgfpathlineto{\pgfqpoint{1.873059in}{2.520522in}}%
\pgfpathlineto{\pgfqpoint{1.869924in}{2.520522in}}%
\pgfpathlineto{\pgfqpoint{1.866790in}{2.520522in}}%
\pgfpathlineto{\pgfqpoint{1.863655in}{2.520522in}}%
\pgfpathlineto{\pgfqpoint{1.863185in}{2.519959in}}%
\pgfpathlineto{\pgfqpoint{1.860520in}{2.516771in}}%
\pgfpathlineto{\pgfqpoint{1.857385in}{2.516771in}}%
\pgfpathlineto{\pgfqpoint{1.854251in}{2.516771in}}%
\pgfpathlineto{\pgfqpoint{1.851116in}{2.516771in}}%
\pgfpathlineto{\pgfqpoint{1.850646in}{2.516208in}}%
\pgfpathlineto{\pgfqpoint{1.847981in}{2.513019in}}%
\pgfpathlineto{\pgfqpoint{1.844846in}{2.513019in}}%
\pgfpathlineto{\pgfqpoint{1.841712in}{2.513019in}}%
\pgfpathlineto{\pgfqpoint{1.838577in}{2.513019in}}%
\pgfpathlineto{\pgfqpoint{1.838107in}{2.512456in}}%
\pgfpathlineto{\pgfqpoint{1.835442in}{2.509267in}}%
\pgfpathlineto{\pgfqpoint{1.832307in}{2.509267in}}%
\pgfpathlineto{\pgfqpoint{1.829173in}{2.509267in}}%
\pgfpathlineto{\pgfqpoint{1.826038in}{2.509267in}}%
\pgfpathlineto{\pgfqpoint{1.825568in}{2.508705in}}%
\pgfpathlineto{\pgfqpoint{1.822903in}{2.505516in}}%
\pgfpathlineto{\pgfqpoint{1.819768in}{2.505516in}}%
\pgfpathlineto{\pgfqpoint{1.816634in}{2.505516in}}%
\pgfpathlineto{\pgfqpoint{1.813499in}{2.505516in}}%
\pgfpathlineto{\pgfqpoint{1.813029in}{2.504953in}}%
\pgfpathlineto{\pgfqpoint{1.810364in}{2.501764in}}%
\pgfpathlineto{\pgfqpoint{1.807229in}{2.501764in}}%
\pgfpathlineto{\pgfqpoint{1.804095in}{2.501764in}}%
\pgfpathlineto{\pgfqpoint{1.800960in}{2.501764in}}%
\pgfpathlineto{\pgfqpoint{1.797825in}{2.501764in}}%
\pgfpathlineto{\pgfqpoint{1.797355in}{2.501202in}}%
\pgfpathlineto{\pgfqpoint{1.794691in}{2.498013in}}%
\pgfpathlineto{\pgfqpoint{1.791556in}{2.498013in}}%
\pgfpathlineto{\pgfqpoint{1.788421in}{2.498013in}}%
\pgfpathlineto{\pgfqpoint{1.785286in}{2.498013in}}%
\pgfpathlineto{\pgfqpoint{1.784816in}{2.497450in}}%
\pgfpathlineto{\pgfqpoint{1.782152in}{2.494261in}}%
\pgfpathlineto{\pgfqpoint{1.779017in}{2.494261in}}%
\pgfpathlineto{\pgfqpoint{1.775882in}{2.494261in}}%
\pgfpathlineto{\pgfqpoint{1.772747in}{2.494261in}}%
\pgfpathlineto{\pgfqpoint{1.772277in}{2.493699in}}%
\pgfpathlineto{\pgfqpoint{1.769613in}{2.490510in}}%
\pgfpathlineto{\pgfqpoint{1.766478in}{2.490510in}}%
\pgfpathlineto{\pgfqpoint{1.763343in}{2.490510in}}%
\pgfpathlineto{\pgfqpoint{1.760208in}{2.490510in}}%
\pgfpathlineto{\pgfqpoint{1.759738in}{2.489947in}}%
\pgfpathlineto{\pgfqpoint{1.757074in}{2.486758in}}%
\pgfpathlineto{\pgfqpoint{1.753939in}{2.486758in}}%
\pgfpathlineto{\pgfqpoint{1.750804in}{2.486758in}}%
\pgfpathlineto{\pgfqpoint{1.747669in}{2.486758in}}%
\pgfpathlineto{\pgfqpoint{1.744535in}{2.486758in}}%
\pgfpathlineto{\pgfqpoint{1.744064in}{2.486196in}}%
\pgfpathlineto{\pgfqpoint{1.741400in}{2.483007in}}%
\pgfpathlineto{\pgfqpoint{1.738265in}{2.483007in}}%
\pgfpathlineto{\pgfqpoint{1.735130in}{2.483007in}}%
\pgfpathlineto{\pgfqpoint{1.731996in}{2.483007in}}%
\pgfpathlineto{\pgfqpoint{1.731525in}{2.482444in}}%
\pgfpathlineto{\pgfqpoint{1.728861in}{2.479255in}}%
\pgfpathlineto{\pgfqpoint{1.725726in}{2.479255in}}%
\pgfpathlineto{\pgfqpoint{1.722591in}{2.479255in}}%
\pgfpathlineto{\pgfqpoint{1.719457in}{2.479255in}}%
\pgfpathlineto{\pgfqpoint{1.718986in}{2.478692in}}%
\pgfpathlineto{\pgfqpoint{1.716322in}{2.475504in}}%
\pgfpathlineto{\pgfqpoint{1.713187in}{2.475504in}}%
\pgfpathlineto{\pgfqpoint{1.710052in}{2.475504in}}%
\pgfpathlineto{\pgfqpoint{1.706918in}{2.475504in}}%
\pgfpathlineto{\pgfqpoint{1.706448in}{2.474941in}}%
\pgfpathlineto{\pgfqpoint{1.703783in}{2.471752in}}%
\pgfpathlineto{\pgfqpoint{1.700648in}{2.471752in}}%
\pgfpathlineto{\pgfqpoint{1.697514in}{2.471752in}}%
\pgfpathlineto{\pgfqpoint{1.694379in}{2.471752in}}%
\pgfpathlineto{\pgfqpoint{1.691244in}{2.471752in}}%
\pgfpathlineto{\pgfqpoint{1.690774in}{2.471189in}}%
\pgfpathlineto{\pgfqpoint{1.688109in}{2.468001in}}%
\pgfpathlineto{\pgfqpoint{1.684975in}{2.468001in}}%
\pgfpathlineto{\pgfqpoint{1.681840in}{2.468001in}}%
\pgfpathlineto{\pgfqpoint{1.678705in}{2.468001in}}%
\pgfpathlineto{\pgfqpoint{1.678235in}{2.467438in}}%
\pgfpathlineto{\pgfqpoint{1.675570in}{2.464249in}}%
\pgfpathlineto{\pgfqpoint{1.672436in}{2.464249in}}%
\pgfpathlineto{\pgfqpoint{1.669301in}{2.464249in}}%
\pgfpathlineto{\pgfqpoint{1.666166in}{2.464249in}}%
\pgfpathlineto{\pgfqpoint{1.665696in}{2.463686in}}%
\pgfpathlineto{\pgfqpoint{1.663031in}{2.460498in}}%
\pgfpathlineto{\pgfqpoint{1.659897in}{2.460498in}}%
\pgfpathlineto{\pgfqpoint{1.656762in}{2.460498in}}%
\pgfpathlineto{\pgfqpoint{1.653627in}{2.460498in}}%
\pgfpathlineto{\pgfqpoint{1.653157in}{2.459935in}}%
\pgfpathlineto{\pgfqpoint{1.650492in}{2.456746in}}%
\pgfpathlineto{\pgfqpoint{1.647358in}{2.456746in}}%
\pgfpathlineto{\pgfqpoint{1.644223in}{2.456746in}}%
\pgfpathlineto{\pgfqpoint{1.641088in}{2.456746in}}%
\pgfpathlineto{\pgfqpoint{1.640618in}{2.456183in}}%
\pgfpathlineto{\pgfqpoint{1.637953in}{2.452994in}}%
\pgfpathlineto{\pgfqpoint{1.634819in}{2.452994in}}%
\pgfpathlineto{\pgfqpoint{1.631684in}{2.452994in}}%
\pgfpathlineto{\pgfqpoint{1.628549in}{2.452994in}}%
\pgfpathlineto{\pgfqpoint{1.625414in}{2.452994in}}%
\pgfpathlineto{\pgfqpoint{1.624944in}{2.452432in}}%
\pgfpathlineto{\pgfqpoint{1.622280in}{2.449243in}}%
\pgfpathlineto{\pgfqpoint{1.619145in}{2.449243in}}%
\pgfpathlineto{\pgfqpoint{1.616010in}{2.449243in}}%
\pgfpathlineto{\pgfqpoint{1.612875in}{2.449243in}}%
\pgfpathlineto{\pgfqpoint{1.612405in}{2.448680in}}%
\pgfpathlineto{\pgfqpoint{1.609741in}{2.445491in}}%
\pgfpathlineto{\pgfqpoint{1.606606in}{2.445491in}}%
\pgfpathlineto{\pgfqpoint{1.603471in}{2.445491in}}%
\pgfpathlineto{\pgfqpoint{1.600337in}{2.445491in}}%
\pgfpathlineto{\pgfqpoint{1.599866in}{2.444929in}}%
\pgfpathlineto{\pgfqpoint{1.597202in}{2.441740in}}%
\pgfpathlineto{\pgfqpoint{1.594067in}{2.441740in}}%
\pgfpathlineto{\pgfqpoint{1.590932in}{2.441740in}}%
\pgfpathlineto{\pgfqpoint{1.587798in}{2.441740in}}%
\pgfpathlineto{\pgfqpoint{1.587327in}{2.441177in}}%
\pgfpathlineto{\pgfqpoint{1.584663in}{2.437988in}}%
\pgfpathlineto{\pgfqpoint{1.581528in}{2.437988in}}%
\pgfpathlineto{\pgfqpoint{1.578393in}{2.437988in}}%
\pgfpathlineto{\pgfqpoint{1.575259in}{2.437988in}}%
\pgfpathlineto{\pgfqpoint{1.572124in}{2.437988in}}%
\pgfpathlineto{\pgfqpoint{1.571654in}{2.437426in}}%
\pgfpathlineto{\pgfqpoint{1.568989in}{2.434237in}}%
\pgfpathlineto{\pgfqpoint{1.565854in}{2.434237in}}%
\pgfpathlineto{\pgfqpoint{1.562720in}{2.434237in}}%
\pgfpathlineto{\pgfqpoint{1.559585in}{2.434237in}}%
\pgfpathlineto{\pgfqpoint{1.559115in}{2.433674in}}%
\pgfpathlineto{\pgfqpoint{1.556450in}{2.430485in}}%
\pgfpathlineto{\pgfqpoint{1.553315in}{2.430485in}}%
\pgfpathlineto{\pgfqpoint{1.550181in}{2.430485in}}%
\pgfpathlineto{\pgfqpoint{1.547046in}{2.430485in}}%
\pgfpathlineto{\pgfqpoint{1.546576in}{2.429923in}}%
\pgfpathlineto{\pgfqpoint{1.543911in}{2.426734in}}%
\pgfpathlineto{\pgfqpoint{1.540776in}{2.426734in}}%
\pgfpathlineto{\pgfqpoint{1.537642in}{2.426734in}}%
\pgfpathlineto{\pgfqpoint{1.534507in}{2.426734in}}%
\pgfpathlineto{\pgfqpoint{1.534037in}{2.426171in}}%
\pgfpathlineto{\pgfqpoint{1.531372in}{2.422982in}}%
\pgfpathlineto{\pgfqpoint{1.528237in}{2.422982in}}%
\pgfpathlineto{\pgfqpoint{1.525103in}{2.422982in}}%
\pgfpathlineto{\pgfqpoint{1.521968in}{2.422982in}}%
\pgfpathlineto{\pgfqpoint{1.518833in}{2.422982in}}%
\pgfpathlineto{\pgfqpoint{1.518363in}{2.422419in}}%
\pgfpathlineto{\pgfqpoint{1.515698in}{2.419231in}}%
\pgfpathlineto{\pgfqpoint{1.512564in}{2.419231in}}%
\pgfpathlineto{\pgfqpoint{1.509429in}{2.419231in}}%
\pgfpathlineto{\pgfqpoint{1.506294in}{2.419231in}}%
\pgfpathlineto{\pgfqpoint{1.505824in}{2.418668in}}%
\pgfpathlineto{\pgfqpoint{1.503159in}{2.415479in}}%
\pgfpathlineto{\pgfqpoint{1.500025in}{2.415479in}}%
\pgfpathlineto{\pgfqpoint{1.496890in}{2.415479in}}%
\pgfpathlineto{\pgfqpoint{1.493755in}{2.415479in}}%
\pgfpathlineto{\pgfqpoint{1.493285in}{2.414916in}}%
\pgfpathlineto{\pgfqpoint{1.490621in}{2.411728in}}%
\pgfpathlineto{\pgfqpoint{1.487486in}{2.411728in}}%
\pgfpathlineto{\pgfqpoint{1.484351in}{2.411728in}}%
\pgfpathlineto{\pgfqpoint{1.481216in}{2.411728in}}%
\pgfpathlineto{\pgfqpoint{1.480746in}{2.411165in}}%
\pgfpathlineto{\pgfqpoint{1.478082in}{2.407976in}}%
\pgfpathlineto{\pgfqpoint{1.474947in}{2.407976in}}%
\pgfpathlineto{\pgfqpoint{1.471812in}{2.407976in}}%
\pgfpathlineto{\pgfqpoint{1.468677in}{2.407976in}}%
\pgfpathlineto{\pgfqpoint{1.468207in}{2.407413in}}%
\pgfpathlineto{\pgfqpoint{1.465543in}{2.404224in}}%
\pgfpathlineto{\pgfqpoint{1.462408in}{2.404224in}}%
\pgfpathlineto{\pgfqpoint{1.461938in}{2.403662in}}%
\pgfpathlineto{\pgfqpoint{1.461938in}{2.399910in}}%
\pgfpathlineto{\pgfqpoint{1.459273in}{2.396721in}}%
\pgfpathlineto{\pgfqpoint{1.458803in}{2.396159in}}%
\pgfpathlineto{\pgfqpoint{1.458803in}{2.392407in}}%
\pgfpathlineto{\pgfqpoint{1.458803in}{2.388656in}}%
\pgfpathlineto{\pgfqpoint{1.456138in}{2.385467in}}%
\pgfpathlineto{\pgfqpoint{1.455668in}{2.384904in}}%
\pgfpathlineto{\pgfqpoint{1.455668in}{2.381153in}}%
\pgfpathlineto{\pgfqpoint{1.455668in}{2.377401in}}%
\pgfpathlineto{\pgfqpoint{1.455668in}{2.373649in}}%
\pgfpathlineto{\pgfqpoint{1.453004in}{2.370461in}}%
\pgfpathlineto{\pgfqpoint{1.452533in}{2.369898in}}%
\pgfpathlineto{\pgfqpoint{1.452533in}{2.366146in}}%
\pgfpathlineto{\pgfqpoint{1.452533in}{2.362395in}}%
\pgfpathlineto{\pgfqpoint{1.449869in}{2.359206in}}%
\pgfpathlineto{\pgfqpoint{1.449399in}{2.358643in}}%
\pgfpathlineto{\pgfqpoint{1.449399in}{2.354892in}}%
\pgfpathlineto{\pgfqpoint{1.449399in}{2.351140in}}%
\pgfpathlineto{\pgfqpoint{1.449399in}{2.347389in}}%
\pgfpathlineto{\pgfqpoint{1.446734in}{2.344200in}}%
\pgfpathlineto{\pgfqpoint{1.446264in}{2.343637in}}%
\pgfpathlineto{\pgfqpoint{1.446264in}{2.339886in}}%
\pgfpathlineto{\pgfqpoint{1.446264in}{2.336134in}}%
\pgfpathlineto{\pgfqpoint{1.443599in}{2.332945in}}%
\pgfpathlineto{\pgfqpoint{1.443129in}{2.332383in}}%
\pgfpathlineto{\pgfqpoint{1.443129in}{2.328631in}}%
\pgfpathlineto{\pgfqpoint{1.443129in}{2.324880in}}%
\pgfpathlineto{\pgfqpoint{1.443129in}{2.321128in}}%
\pgfpathlineto{\pgfqpoint{1.440465in}{2.317939in}}%
\pgfpathlineto{\pgfqpoint{1.439994in}{2.317376in}}%
\pgfpathlineto{\pgfqpoint{1.439994in}{2.313625in}}%
\pgfpathlineto{\pgfqpoint{1.439994in}{2.309873in}}%
\pgfpathlineto{\pgfqpoint{1.437330in}{2.306685in}}%
\pgfpathlineto{\pgfqpoint{1.436860in}{2.306122in}}%
\pgfpathlineto{\pgfqpoint{1.436860in}{2.302370in}}%
\pgfpathlineto{\pgfqpoint{1.436860in}{2.298619in}}%
\pgfpathlineto{\pgfqpoint{1.436860in}{2.294867in}}%
\pgfpathlineto{\pgfqpoint{1.434195in}{2.291678in}}%
\pgfpathlineto{\pgfqpoint{1.433725in}{2.291116in}}%
\pgfpathlineto{\pgfqpoint{1.433725in}{2.287364in}}%
\pgfpathlineto{\pgfqpoint{1.433725in}{2.283613in}}%
\pgfpathlineto{\pgfqpoint{1.431060in}{2.280424in}}%
\pgfpathlineto{\pgfqpoint{1.430590in}{2.279861in}}%
\pgfpathlineto{\pgfqpoint{1.430590in}{2.276110in}}%
\pgfpathlineto{\pgfqpoint{1.430590in}{2.272358in}}%
\pgfpathlineto{\pgfqpoint{1.430590in}{2.268607in}}%
\pgfpathlineto{\pgfqpoint{1.427926in}{2.265418in}}%
\pgfpathlineto{\pgfqpoint{1.427455in}{2.264855in}}%
\pgfpathlineto{\pgfqpoint{1.427455in}{2.261103in}}%
\pgfpathlineto{\pgfqpoint{1.427455in}{2.257352in}}%
\pgfpathlineto{\pgfqpoint{1.424791in}{2.254163in}}%
\pgfpathlineto{\pgfqpoint{1.424321in}{2.253600in}}%
\pgfpathlineto{\pgfqpoint{1.424321in}{2.249849in}}%
\pgfpathlineto{\pgfqpoint{1.424321in}{2.246097in}}%
\pgfpathlineto{\pgfqpoint{1.421656in}{2.242909in}}%
\pgfpathlineto{\pgfqpoint{1.421186in}{2.242346in}}%
\pgfpathlineto{\pgfqpoint{1.421186in}{2.238594in}}%
\pgfpathlineto{\pgfqpoint{1.421186in}{2.234843in}}%
\pgfpathlineto{\pgfqpoint{1.421186in}{2.231091in}}%
\pgfpathlineto{\pgfqpoint{1.418521in}{2.227902in}}%
\pgfpathlineto{\pgfqpoint{1.418051in}{2.227340in}}%
\pgfpathlineto{\pgfqpoint{1.418051in}{2.223588in}}%
\pgfpathlineto{\pgfqpoint{1.418051in}{2.219837in}}%
\pgfpathlineto{\pgfqpoint{1.415387in}{2.216648in}}%
\pgfpathlineto{\pgfqpoint{1.414916in}{2.216085in}}%
\pgfpathlineto{\pgfqpoint{1.414916in}{2.212334in}}%
\pgfpathlineto{\pgfqpoint{1.414916in}{2.208582in}}%
\pgfpathlineto{\pgfqpoint{1.414916in}{2.204830in}}%
\pgfpathlineto{\pgfqpoint{1.412252in}{2.201642in}}%
\pgfpathlineto{\pgfqpoint{1.411782in}{2.201079in}}%
\pgfpathlineto{\pgfqpoint{1.411782in}{2.197327in}}%
\pgfpathlineto{\pgfqpoint{1.411782in}{2.193576in}}%
\pgfpathlineto{\pgfqpoint{1.409117in}{2.190387in}}%
\pgfpathlineto{\pgfqpoint{1.408647in}{2.189824in}}%
\pgfpathlineto{\pgfqpoint{1.408647in}{2.186073in}}%
\pgfpathlineto{\pgfqpoint{1.408647in}{2.182321in}}%
\pgfpathlineto{\pgfqpoint{1.408647in}{2.178570in}}%
\pgfpathlineto{\pgfqpoint{1.405982in}{2.175381in}}%
\pgfpathlineto{\pgfqpoint{1.405512in}{2.174818in}}%
\pgfpathlineto{\pgfqpoint{1.405512in}{2.171067in}}%
\pgfpathlineto{\pgfqpoint{1.405512in}{2.167315in}}%
\pgfpathlineto{\pgfqpoint{1.402848in}{2.164126in}}%
\pgfpathlineto{\pgfqpoint{1.402378in}{2.163564in}}%
\pgfpathlineto{\pgfqpoint{1.402378in}{2.159812in}}%
\pgfpathlineto{\pgfqpoint{1.402378in}{2.156060in}}%
\pgfpathlineto{\pgfqpoint{1.402378in}{2.152309in}}%
\pgfpathlineto{\pgfqpoint{1.399713in}{2.149120in}}%
\pgfpathlineto{\pgfqpoint{1.399243in}{2.148557in}}%
\pgfpathlineto{\pgfqpoint{1.399243in}{2.144806in}}%
\pgfpathlineto{\pgfqpoint{1.399243in}{2.141054in}}%
\pgfpathlineto{\pgfqpoint{1.396578in}{2.137866in}}%
\pgfpathlineto{\pgfqpoint{1.396108in}{2.137303in}}%
\pgfpathlineto{\pgfqpoint{1.396108in}{2.133551in}}%
\pgfpathlineto{\pgfqpoint{1.396108in}{2.129800in}}%
\pgfpathlineto{\pgfqpoint{1.396108in}{2.126048in}}%
\pgfpathlineto{\pgfqpoint{1.393444in}{2.122859in}}%
\pgfpathlineto{\pgfqpoint{1.392973in}{2.122297in}}%
\pgfpathlineto{\pgfqpoint{1.392973in}{2.118545in}}%
\pgfpathlineto{\pgfqpoint{1.392973in}{2.114794in}}%
\pgfpathlineto{\pgfqpoint{1.390309in}{2.111605in}}%
\pgfpathlineto{\pgfqpoint{1.389839in}{2.111042in}}%
\pgfpathlineto{\pgfqpoint{1.389839in}{2.107291in}}%
\pgfpathlineto{\pgfqpoint{1.389839in}{2.103539in}}%
\pgfpathlineto{\pgfqpoint{1.389839in}{2.099787in}}%
\pgfpathlineto{\pgfqpoint{1.387174in}{2.096599in}}%
\pgfpathlineto{\pgfqpoint{1.386704in}{2.096036in}}%
\pgfpathlineto{\pgfqpoint{1.386704in}{2.092284in}}%
\pgfpathlineto{\pgfqpoint{1.386704in}{2.088533in}}%
\pgfpathlineto{\pgfqpoint{1.384039in}{2.085344in}}%
\pgfpathlineto{\pgfqpoint{1.383569in}{2.084781in}}%
\pgfpathlineto{\pgfqpoint{1.383569in}{2.081030in}}%
\pgfpathlineto{\pgfqpoint{1.383569in}{2.077278in}}%
\pgfpathlineto{\pgfqpoint{1.383569in}{2.073527in}}%
\pgfpathlineto{\pgfqpoint{1.380905in}{2.070338in}}%
\pgfpathlineto{\pgfqpoint{1.380434in}{2.069775in}}%
\pgfpathlineto{\pgfqpoint{1.380434in}{2.066024in}}%
\pgfpathlineto{\pgfqpoint{1.380434in}{2.062272in}}%
\pgfpathlineto{\pgfqpoint{1.377770in}{2.059083in}}%
\pgfpathlineto{\pgfqpoint{1.377300in}{2.058521in}}%
\pgfpathlineto{\pgfqpoint{1.377300in}{2.054769in}}%
\pgfpathlineto{\pgfqpoint{1.377300in}{2.051018in}}%
\pgfpathlineto{\pgfqpoint{1.377300in}{2.047266in}}%
\pgfpathlineto{\pgfqpoint{1.374635in}{2.044077in}}%
\pgfpathlineto{\pgfqpoint{1.374165in}{2.043514in}}%
\pgfpathlineto{\pgfqpoint{1.374165in}{2.039763in}}%
\pgfpathlineto{\pgfqpoint{1.374165in}{2.036011in}}%
\pgfpathlineto{\pgfqpoint{1.371500in}{2.032823in}}%
\pgfpathlineto{\pgfqpoint{1.371030in}{2.032260in}}%
\pgfpathlineto{\pgfqpoint{1.371030in}{2.028508in}}%
\pgfpathlineto{\pgfqpoint{1.371030in}{2.024757in}}%
\pgfpathlineto{\pgfqpoint{1.368366in}{2.021568in}}%
\pgfpathlineto{\pgfqpoint{1.367895in}{2.021005in}}%
\pgfpathlineto{\pgfqpoint{1.367895in}{2.017254in}}%
\pgfpathlineto{\pgfqpoint{1.367895in}{2.013502in}}%
\pgfpathlineto{\pgfqpoint{1.367895in}{2.009751in}}%
\pgfpathlineto{\pgfqpoint{1.365231in}{2.006562in}}%
\pgfpathlineto{\pgfqpoint{1.364761in}{2.005999in}}%
\pgfpathlineto{\pgfqpoint{1.364761in}{2.002248in}}%
\pgfpathlineto{\pgfqpoint{1.364761in}{1.998496in}}%
\pgfpathlineto{\pgfqpoint{1.362096in}{1.995307in}}%
\pgfpathlineto{\pgfqpoint{1.361626in}{1.994745in}}%
\pgfpathlineto{\pgfqpoint{1.361626in}{1.990993in}}%
\pgfpathlineto{\pgfqpoint{1.361626in}{1.987241in}}%
\pgfpathlineto{\pgfqpoint{1.361626in}{1.983490in}}%
\pgfpathlineto{\pgfqpoint{1.358961in}{1.980301in}}%
\pgfpathlineto{\pgfqpoint{1.358491in}{1.979738in}}%
\pgfpathlineto{\pgfqpoint{1.358491in}{1.975987in}}%
\pgfpathlineto{\pgfqpoint{1.358491in}{1.972235in}}%
\pgfpathlineto{\pgfqpoint{1.355827in}{1.969046in}}%
\pgfpathlineto{\pgfqpoint{1.355356in}{1.968484in}}%
\pgfpathlineto{\pgfqpoint{1.355356in}{1.964732in}}%
\pgfpathlineto{\pgfqpoint{1.355356in}{1.960981in}}%
\pgfpathlineto{\pgfqpoint{1.355356in}{1.957229in}}%
\pgfpathlineto{\pgfqpoint{1.352692in}{1.954040in}}%
\pgfpathlineto{\pgfqpoint{1.352222in}{1.953478in}}%
\pgfpathlineto{\pgfqpoint{1.352222in}{1.949726in}}%
\pgfpathlineto{\pgfqpoint{1.352222in}{1.945975in}}%
\pgfpathlineto{\pgfqpoint{1.349557in}{1.942786in}}%
\pgfpathlineto{\pgfqpoint{1.349087in}{1.942223in}}%
\pgfpathlineto{\pgfqpoint{1.349087in}{1.938471in}}%
\pgfpathlineto{\pgfqpoint{1.349087in}{1.934720in}}%
\pgfpathlineto{\pgfqpoint{1.349087in}{1.930968in}}%
\pgfpathlineto{\pgfqpoint{1.346422in}{1.927780in}}%
\pgfpathlineto{\pgfqpoint{1.345952in}{1.927217in}}%
\pgfpathlineto{\pgfqpoint{1.345952in}{1.923465in}}%
\pgfpathlineto{\pgfqpoint{1.345952in}{1.919714in}}%
\pgfpathlineto{\pgfqpoint{1.343288in}{1.916525in}}%
\pgfpathlineto{\pgfqpoint{1.342817in}{1.915962in}}%
\pgfpathlineto{\pgfqpoint{1.342817in}{1.912211in}}%
\pgfpathlineto{\pgfqpoint{1.342817in}{1.908459in}}%
\pgfpathlineto{\pgfqpoint{1.342817in}{1.904708in}}%
\pgfpathlineto{\pgfqpoint{1.340153in}{1.901519in}}%
\pgfpathlineto{\pgfqpoint{1.339683in}{1.900956in}}%
\pgfpathlineto{\pgfqpoint{1.339683in}{1.897205in}}%
\pgfpathlineto{\pgfqpoint{1.339683in}{1.893453in}}%
\pgfpathlineto{\pgfqpoint{1.337018in}{1.890264in}}%
\pgfpathlineto{\pgfqpoint{1.336548in}{1.889702in}}%
\pgfpathlineto{\pgfqpoint{1.336548in}{1.885950in}}%
\pgfpathlineto{\pgfqpoint{1.336548in}{1.882198in}}%
\pgfpathlineto{\pgfqpoint{1.336548in}{1.878447in}}%
\pgfpathlineto{\pgfqpoint{1.333883in}{1.875258in}}%
\pgfpathlineto{\pgfqpoint{1.333413in}{1.874695in}}%
\pgfpathlineto{\pgfqpoint{1.333413in}{1.870944in}}%
\pgfpathlineto{\pgfqpoint{1.333413in}{1.867192in}}%
\pgfpathlineto{\pgfqpoint{1.330749in}{1.864004in}}%
\pgfpathlineto{\pgfqpoint{1.330278in}{1.863441in}}%
\pgfpathlineto{\pgfqpoint{1.330278in}{1.859689in}}%
\pgfpathlineto{\pgfqpoint{1.330278in}{1.855938in}}%
\pgfpathlineto{\pgfqpoint{1.330278in}{1.852186in}}%
\pgfpathlineto{\pgfqpoint{1.327614in}{1.848997in}}%
\pgfpathlineto{\pgfqpoint{1.327144in}{1.848435in}}%
\pgfpathlineto{\pgfqpoint{1.327144in}{1.844683in}}%
\pgfpathlineto{\pgfqpoint{1.327144in}{1.840932in}}%
\pgfpathlineto{\pgfqpoint{1.324479in}{1.837743in}}%
\pgfpathlineto{\pgfqpoint{1.324009in}{1.837180in}}%
\pgfpathlineto{\pgfqpoint{1.324009in}{1.833429in}}%
\pgfpathlineto{\pgfqpoint{1.324009in}{1.829677in}}%
\pgfpathlineto{\pgfqpoint{1.324009in}{1.825925in}}%
\pgfpathlineto{\pgfqpoint{1.321344in}{1.822737in}}%
\pgfpathlineto{\pgfqpoint{1.320874in}{1.822174in}}%
\pgfpathlineto{\pgfqpoint{1.320874in}{1.818422in}}%
\pgfpathlineto{\pgfqpoint{1.320874in}{1.814671in}}%
\pgfpathlineto{\pgfqpoint{1.318210in}{1.811482in}}%
\pgfpathlineto{\pgfqpoint{1.317739in}{1.810919in}}%
\pgfpathlineto{\pgfqpoint{1.317739in}{1.807168in}}%
\pgfpathlineto{\pgfqpoint{1.317739in}{1.803416in}}%
\pgfpathlineto{\pgfqpoint{1.315075in}{1.800227in}}%
\pgfpathlineto{\pgfqpoint{1.314605in}{1.799665in}}%
\pgfpathlineto{\pgfqpoint{1.314605in}{1.795913in}}%
\pgfpathlineto{\pgfqpoint{1.314605in}{1.792162in}}%
\pgfpathlineto{\pgfqpoint{1.314605in}{1.788410in}}%
\pgfpathlineto{\pgfqpoint{1.311940in}{1.785221in}}%
\pgfpathlineto{\pgfqpoint{1.311470in}{1.784659in}}%
\pgfpathlineto{\pgfqpoint{1.311470in}{1.780907in}}%
\pgfpathlineto{\pgfqpoint{1.311470in}{1.777155in}}%
\pgfpathlineto{\pgfqpoint{1.308805in}{1.773967in}}%
\pgfpathlineto{\pgfqpoint{1.308335in}{1.773404in}}%
\pgfpathlineto{\pgfqpoint{1.308335in}{1.769652in}}%
\pgfpathlineto{\pgfqpoint{1.308335in}{1.765901in}}%
\pgfpathlineto{\pgfqpoint{1.308335in}{1.762149in}}%
\pgfpathlineto{\pgfqpoint{1.305671in}{1.758961in}}%
\pgfpathlineto{\pgfqpoint{1.305201in}{1.758398in}}%
\pgfpathlineto{\pgfqpoint{1.305201in}{1.754646in}}%
\pgfpathlineto{\pgfqpoint{1.305201in}{1.750895in}}%
\pgfpathlineto{\pgfqpoint{1.302536in}{1.747706in}}%
\pgfpathlineto{\pgfqpoint{1.302066in}{1.747143in}}%
\pgfpathlineto{\pgfqpoint{1.302066in}{1.743392in}}%
\pgfpathlineto{\pgfqpoint{1.302066in}{1.739640in}}%
\pgfpathlineto{\pgfqpoint{1.302066in}{1.735889in}}%
\pgfpathlineto{\pgfqpoint{1.299401in}{1.732700in}}%
\pgfpathlineto{\pgfqpoint{1.298931in}{1.732137in}}%
\pgfpathlineto{\pgfqpoint{1.298931in}{1.728386in}}%
\pgfpathlineto{\pgfqpoint{1.298931in}{1.724634in}}%
\pgfpathlineto{\pgfqpoint{1.296266in}{1.721445in}}%
\pgfpathlineto{\pgfqpoint{1.295796in}{1.720882in}}%
\pgfpathlineto{\pgfqpoint{1.295796in}{1.717131in}}%
\pgfpathlineto{\pgfqpoint{1.295796in}{1.713379in}}%
\pgfpathlineto{\pgfqpoint{1.295796in}{1.709628in}}%
\pgfpathlineto{\pgfqpoint{1.293132in}{1.706439in}}%
\pgfpathlineto{\pgfqpoint{1.292662in}{1.705876in}}%
\pgfpathlineto{\pgfqpoint{1.292662in}{1.702125in}}%
\pgfpathlineto{\pgfqpoint{1.293132in}{1.701562in}}%
\pgfpathlineto{\pgfqpoint{1.295796in}{1.698373in}}%
\pgfpathlineto{\pgfqpoint{1.296266in}{1.697811in}}%
\pgfpathlineto{\pgfqpoint{1.298931in}{1.694622in}}%
\pgfpathlineto{\pgfqpoint{1.299401in}{1.694059in}}%
\pgfpathlineto{\pgfqpoint{1.302536in}{1.694059in}}%
\pgfpathlineto{\pgfqpoint{1.305201in}{1.690870in}}%
\pgfpathlineto{\pgfqpoint{1.305671in}{1.690307in}}%
\pgfpathlineto{\pgfqpoint{1.308335in}{1.687119in}}%
\pgfpathlineto{\pgfqpoint{1.308805in}{1.686556in}}%
\pgfpathlineto{\pgfqpoint{1.311470in}{1.683367in}}%
\pgfpathlineto{\pgfqpoint{1.311940in}{1.682804in}}%
\pgfpathlineto{\pgfqpoint{1.315075in}{1.682804in}}%
\pgfpathlineto{\pgfqpoint{1.317739in}{1.679616in}}%
\pgfpathlineto{\pgfqpoint{1.318210in}{1.679053in}}%
\pgfpathlineto{\pgfqpoint{1.320874in}{1.675864in}}%
\pgfpathlineto{\pgfqpoint{1.321344in}{1.675301in}}%
\pgfpathlineto{\pgfqpoint{1.324009in}{1.672113in}}%
\pgfpathlineto{\pgfqpoint{1.324479in}{1.671550in}}%
\pgfpathlineto{\pgfqpoint{1.327614in}{1.671550in}}%
\pgfpathlineto{\pgfqpoint{1.330278in}{1.668361in}}%
\pgfpathlineto{\pgfqpoint{1.330749in}{1.667798in}}%
\pgfpathlineto{\pgfqpoint{1.333413in}{1.664609in}}%
\pgfpathlineto{\pgfqpoint{1.333883in}{1.664047in}}%
\pgfpathlineto{\pgfqpoint{1.336548in}{1.660858in}}%
\pgfpathlineto{\pgfqpoint{1.337018in}{1.660295in}}%
\pgfpathlineto{\pgfqpoint{1.339683in}{1.657106in}}%
\pgfpathlineto{\pgfqpoint{1.340153in}{1.656544in}}%
\pgfpathlineto{\pgfqpoint{1.343288in}{1.656544in}}%
\pgfpathlineto{\pgfqpoint{1.345952in}{1.653355in}}%
\pgfpathlineto{\pgfqpoint{1.346422in}{1.652792in}}%
\pgfpathlineto{\pgfqpoint{1.349087in}{1.649603in}}%
\pgfpathlineto{\pgfqpoint{1.349557in}{1.649041in}}%
\pgfpathlineto{\pgfqpoint{1.352222in}{1.645852in}}%
\pgfpathlineto{\pgfqpoint{1.352692in}{1.645289in}}%
\pgfpathlineto{\pgfqpoint{1.355827in}{1.645289in}}%
\pgfpathlineto{\pgfqpoint{1.358491in}{1.642100in}}%
\pgfpathlineto{\pgfqpoint{1.358961in}{1.641538in}}%
\pgfpathlineto{\pgfqpoint{1.361626in}{1.638349in}}%
\pgfpathlineto{\pgfqpoint{1.362096in}{1.637786in}}%
\pgfpathlineto{\pgfqpoint{1.364761in}{1.634597in}}%
\pgfpathlineto{\pgfqpoint{1.365231in}{1.634034in}}%
\pgfpathlineto{\pgfqpoint{1.368366in}{1.634034in}}%
\pgfpathlineto{\pgfqpoint{1.371030in}{1.630846in}}%
\pgfpathlineto{\pgfqpoint{1.371500in}{1.630283in}}%
\pgfpathlineto{\pgfqpoint{1.374165in}{1.627094in}}%
\pgfpathlineto{\pgfqpoint{1.374635in}{1.626531in}}%
\pgfpathlineto{\pgfqpoint{1.377300in}{1.623343in}}%
\pgfpathlineto{\pgfqpoint{1.377770in}{1.622780in}}%
\pgfpathlineto{\pgfqpoint{1.380905in}{1.622780in}}%
\pgfpathlineto{\pgfqpoint{1.383569in}{1.619591in}}%
\pgfpathlineto{\pgfqpoint{1.384039in}{1.619028in}}%
\pgfpathlineto{\pgfqpoint{1.386704in}{1.615840in}}%
\pgfpathlineto{\pgfqpoint{1.387174in}{1.615277in}}%
\pgfpathlineto{\pgfqpoint{1.389839in}{1.612088in}}%
\pgfpathlineto{\pgfqpoint{1.390309in}{1.611525in}}%
\pgfpathlineto{\pgfqpoint{1.393444in}{1.611525in}}%
\pgfpathlineto{\pgfqpoint{1.396108in}{1.608336in}}%
\pgfpathlineto{\pgfqpoint{1.396578in}{1.607774in}}%
\pgfpathlineto{\pgfqpoint{1.399243in}{1.604585in}}%
\pgfpathlineto{\pgfqpoint{1.399713in}{1.604022in}}%
\pgfpathlineto{\pgfqpoint{1.402378in}{1.600833in}}%
\pgfpathlineto{\pgfqpoint{1.402848in}{1.600271in}}%
\pgfpathlineto{\pgfqpoint{1.405982in}{1.600271in}}%
\pgfpathlineto{\pgfqpoint{1.408647in}{1.597082in}}%
\pgfpathlineto{\pgfqpoint{1.409117in}{1.596519in}}%
\pgfpathlineto{\pgfqpoint{1.411782in}{1.593330in}}%
\pgfpathlineto{\pgfqpoint{1.412252in}{1.592768in}}%
\pgfpathlineto{\pgfqpoint{1.414916in}{1.589579in}}%
\pgfpathlineto{\pgfqpoint{1.415387in}{1.589016in}}%
\pgfpathlineto{\pgfqpoint{1.418051in}{1.585827in}}%
\pgfpathlineto{\pgfqpoint{1.418521in}{1.585265in}}%
\pgfpathlineto{\pgfqpoint{1.421656in}{1.585265in}}%
\pgfpathlineto{\pgfqpoint{1.424321in}{1.582076in}}%
\pgfpathlineto{\pgfqpoint{1.424791in}{1.581513in}}%
\pgfpathlineto{\pgfqpoint{1.427455in}{1.578324in}}%
\pgfpathlineto{\pgfqpoint{1.427926in}{1.577761in}}%
\pgfpathlineto{\pgfqpoint{1.430590in}{1.574573in}}%
\pgfpathlineto{\pgfqpoint{1.431060in}{1.574010in}}%
\pgfpathlineto{\pgfqpoint{1.434195in}{1.574010in}}%
\pgfpathlineto{\pgfqpoint{1.436860in}{1.570821in}}%
\pgfpathlineto{\pgfqpoint{1.437330in}{1.570258in}}%
\pgfpathlineto{\pgfqpoint{1.439994in}{1.567070in}}%
\pgfpathlineto{\pgfqpoint{1.440465in}{1.566507in}}%
\pgfpathlineto{\pgfqpoint{1.443129in}{1.563318in}}%
\pgfpathlineto{\pgfqpoint{1.443599in}{1.562755in}}%
\pgfpathlineto{\pgfqpoint{1.446734in}{1.562755in}}%
\pgfpathlineto{\pgfqpoint{1.449399in}{1.559566in}}%
\pgfpathlineto{\pgfqpoint{1.449869in}{1.559004in}}%
\pgfpathlineto{\pgfqpoint{1.452533in}{1.555815in}}%
\pgfpathlineto{\pgfqpoint{1.453004in}{1.555252in}}%
\pgfpathlineto{\pgfqpoint{1.455668in}{1.552063in}}%
\pgfpathlineto{\pgfqpoint{1.456138in}{1.551501in}}%
\pgfpathlineto{\pgfqpoint{1.459273in}{1.551501in}}%
\pgfpathlineto{\pgfqpoint{1.461938in}{1.548312in}}%
\pgfpathlineto{\pgfqpoint{1.462408in}{1.547749in}}%
\pgfpathlineto{\pgfqpoint{1.465072in}{1.544560in}}%
\pgfpathlineto{\pgfqpoint{1.465543in}{1.543998in}}%
\pgfpathlineto{\pgfqpoint{1.468207in}{1.540809in}}%
\pgfpathlineto{\pgfqpoint{1.468677in}{1.540246in}}%
\pgfpathlineto{\pgfqpoint{1.471812in}{1.540246in}}%
\pgfpathlineto{\pgfqpoint{1.474477in}{1.537057in}}%
\pgfpathlineto{\pgfqpoint{1.474947in}{1.536495in}}%
\pgfpathlineto{\pgfqpoint{1.477611in}{1.533306in}}%
\pgfpathlineto{\pgfqpoint{1.478082in}{1.532743in}}%
\pgfpathlineto{\pgfqpoint{1.480746in}{1.529554in}}%
\pgfpathlineto{\pgfqpoint{1.481216in}{1.528991in}}%
\pgfpathlineto{\pgfqpoint{1.484351in}{1.528991in}}%
\pgfpathlineto{\pgfqpoint{1.487016in}{1.525803in}}%
\pgfpathlineto{\pgfqpoint{1.487486in}{1.525240in}}%
\pgfpathlineto{\pgfqpoint{1.490150in}{1.522051in}}%
\pgfpathlineto{\pgfqpoint{1.490621in}{1.521488in}}%
\pgfpathlineto{\pgfqpoint{1.493285in}{1.518300in}}%
\pgfpathlineto{\pgfqpoint{1.493755in}{1.517737in}}%
\pgfpathlineto{\pgfqpoint{1.496420in}{1.514548in}}%
\pgfpathlineto{\pgfqpoint{1.496890in}{1.513985in}}%
\pgfpathlineto{\pgfqpoint{1.500025in}{1.513985in}}%
\pgfpathlineto{\pgfqpoint{1.502689in}{1.510797in}}%
\pgfpathlineto{\pgfqpoint{1.503159in}{1.510234in}}%
\pgfpathlineto{\pgfqpoint{1.505824in}{1.507045in}}%
\pgfpathlineto{\pgfqpoint{1.506294in}{1.506482in}}%
\pgfpathlineto{\pgfqpoint{1.508959in}{1.503293in}}%
\pgfpathlineto{\pgfqpoint{1.509429in}{1.502731in}}%
\pgfpathlineto{\pgfqpoint{1.512564in}{1.502731in}}%
\pgfpathlineto{\pgfqpoint{1.515228in}{1.499542in}}%
\pgfpathlineto{\pgfqpoint{1.515698in}{1.498979in}}%
\pgfpathlineto{\pgfqpoint{1.518363in}{1.495790in}}%
\pgfpathlineto{\pgfqpoint{1.518833in}{1.495228in}}%
\pgfpathlineto{\pgfqpoint{1.521498in}{1.492039in}}%
\pgfpathlineto{\pgfqpoint{1.521968in}{1.491476in}}%
\pgfpathlineto{\pgfqpoint{1.525103in}{1.491476in}}%
\pgfpathlineto{\pgfqpoint{1.527767in}{1.488287in}}%
\pgfpathlineto{\pgfqpoint{1.528237in}{1.487725in}}%
\pgfpathlineto{\pgfqpoint{1.530902in}{1.484536in}}%
\pgfpathlineto{\pgfqpoint{1.531372in}{1.483973in}}%
\pgfpathlineto{\pgfqpoint{1.534037in}{1.480784in}}%
\pgfpathlineto{\pgfqpoint{1.534507in}{1.480222in}}%
\pgfpathlineto{\pgfqpoint{1.537642in}{1.480222in}}%
\pgfpathlineto{\pgfqpoint{1.540306in}{1.477033in}}%
\pgfpathlineto{\pgfqpoint{1.540776in}{1.476470in}}%
\pgfpathlineto{\pgfqpoint{1.543441in}{1.473281in}}%
\pgfpathlineto{\pgfqpoint{1.543911in}{1.472718in}}%
\pgfpathlineto{\pgfqpoint{1.546576in}{1.469530in}}%
\pgfpathlineto{\pgfqpoint{1.547046in}{1.468967in}}%
\pgfpathlineto{\pgfqpoint{1.550181in}{1.468967in}}%
\pgfpathlineto{\pgfqpoint{1.552845in}{1.465778in}}%
\pgfpathlineto{\pgfqpoint{1.553315in}{1.465215in}}%
\pgfpathlineto{\pgfqpoint{1.555980in}{1.462027in}}%
\pgfpathlineto{\pgfqpoint{1.556450in}{1.461464in}}%
\pgfpathlineto{\pgfqpoint{1.559115in}{1.458275in}}%
\pgfpathlineto{\pgfqpoint{1.559585in}{1.457712in}}%
\pgfpathlineto{\pgfqpoint{1.562249in}{1.454524in}}%
\pgfpathlineto{\pgfqpoint{1.562720in}{1.453961in}}%
\pgfpathlineto{\pgfqpoint{1.565854in}{1.453961in}}%
\pgfpathlineto{\pgfqpoint{1.568519in}{1.450772in}}%
\pgfpathlineto{\pgfqpoint{1.568989in}{1.450209in}}%
\pgfpathlineto{\pgfqpoint{1.571654in}{1.447020in}}%
\pgfpathlineto{\pgfqpoint{1.572124in}{1.446458in}}%
\pgfpathlineto{\pgfqpoint{1.574788in}{1.443269in}}%
\pgfpathlineto{\pgfqpoint{1.575259in}{1.442706in}}%
\pgfpathlineto{\pgfqpoint{1.578393in}{1.442706in}}%
\pgfpathlineto{\pgfqpoint{1.581058in}{1.439517in}}%
\pgfpathlineto{\pgfqpoint{1.581528in}{1.438955in}}%
\pgfpathlineto{\pgfqpoint{1.584193in}{1.435766in}}%
\pgfpathlineto{\pgfqpoint{1.584663in}{1.435203in}}%
\pgfpathlineto{\pgfqpoint{1.587327in}{1.432014in}}%
\pgfpathlineto{\pgfqpoint{1.587798in}{1.431452in}}%
\pgfpathlineto{\pgfqpoint{1.590932in}{1.431452in}}%
\pgfpathlineto{\pgfqpoint{1.593597in}{1.428263in}}%
\pgfpathlineto{\pgfqpoint{1.594067in}{1.427700in}}%
\pgfpathlineto{\pgfqpoint{1.596732in}{1.424511in}}%
\pgfpathlineto{\pgfqpoint{1.597202in}{1.423949in}}%
\pgfpathlineto{\pgfqpoint{1.599866in}{1.420760in}}%
\pgfpathlineto{\pgfqpoint{1.600337in}{1.420197in}}%
\pgfpathlineto{\pgfqpoint{1.603471in}{1.420197in}}%
\pgfpathlineto{\pgfqpoint{1.606136in}{1.417008in}}%
\pgfpathlineto{\pgfqpoint{1.606606in}{1.416445in}}%
\pgfpathlineto{\pgfqpoint{1.609271in}{1.413257in}}%
\pgfpathlineto{\pgfqpoint{1.609741in}{1.412694in}}%
\pgfpathlineto{\pgfqpoint{1.612405in}{1.409505in}}%
\pgfpathlineto{\pgfqpoint{1.612875in}{1.408942in}}%
\pgfpathlineto{\pgfqpoint{1.616010in}{1.408942in}}%
\pgfpathlineto{\pgfqpoint{1.618675in}{1.405754in}}%
\pgfpathlineto{\pgfqpoint{1.619145in}{1.405191in}}%
\pgfpathlineto{\pgfqpoint{1.621809in}{1.402002in}}%
\pgfpathlineto{\pgfqpoint{1.622280in}{1.401439in}}%
\pgfpathlineto{\pgfqpoint{1.624944in}{1.398251in}}%
\pgfpathlineto{\pgfqpoint{1.625414in}{1.397688in}}%
\pgfpathlineto{\pgfqpoint{1.628549in}{1.397688in}}%
\pgfpathlineto{\pgfqpoint{1.631214in}{1.394499in}}%
\pgfpathlineto{\pgfqpoint{1.631684in}{1.393936in}}%
\pgfpathlineto{\pgfqpoint{1.634348in}{1.390747in}}%
\pgfpathlineto{\pgfqpoint{1.634819in}{1.390185in}}%
\pgfpathlineto{\pgfqpoint{1.637483in}{1.386996in}}%
\pgfpathlineto{\pgfqpoint{1.637953in}{1.386433in}}%
\pgfpathlineto{\pgfqpoint{1.640618in}{1.383244in}}%
\pgfpathlineto{\pgfqpoint{1.641088in}{1.382682in}}%
\pgfpathlineto{\pgfqpoint{1.644223in}{1.382682in}}%
\pgfpathlineto{\pgfqpoint{1.646887in}{1.379493in}}%
\pgfpathlineto{\pgfqpoint{1.647358in}{1.378930in}}%
\pgfpathlineto{\pgfqpoint{1.650022in}{1.375741in}}%
\pgfpathlineto{\pgfqpoint{1.650492in}{1.375179in}}%
\pgfpathlineto{\pgfqpoint{1.653157in}{1.371990in}}%
\pgfpathlineto{\pgfqpoint{1.653627in}{1.371427in}}%
\pgfpathlineto{\pgfqpoint{1.656762in}{1.371427in}}%
\pgfpathlineto{\pgfqpoint{1.659426in}{1.368238in}}%
\pgfpathlineto{\pgfqpoint{1.659897in}{1.367676in}}%
\pgfpathlineto{\pgfqpoint{1.662561in}{1.364487in}}%
\pgfpathlineto{\pgfqpoint{1.663031in}{1.363924in}}%
\pgfpathlineto{\pgfqpoint{1.665696in}{1.360735in}}%
\pgfpathlineto{\pgfqpoint{1.666166in}{1.360172in}}%
\pgfpathlineto{\pgfqpoint{1.669301in}{1.360172in}}%
\pgfpathlineto{\pgfqpoint{1.671965in}{1.356984in}}%
\pgfpathlineto{\pgfqpoint{1.672436in}{1.356421in}}%
\pgfpathlineto{\pgfqpoint{1.675100in}{1.353232in}}%
\pgfpathlineto{\pgfqpoint{1.675570in}{1.352669in}}%
\pgfpathlineto{\pgfqpoint{1.678235in}{1.349481in}}%
\pgfpathlineto{\pgfqpoint{1.678705in}{1.348918in}}%
\pgfpathlineto{\pgfqpoint{1.681840in}{1.348918in}}%
\pgfpathlineto{\pgfqpoint{1.684504in}{1.345729in}}%
\pgfpathlineto{\pgfqpoint{1.684975in}{1.345166in}}%
\pgfpathlineto{\pgfqpoint{1.687639in}{1.341977in}}%
\pgfpathlineto{\pgfqpoint{1.688109in}{1.341415in}}%
\pgfpathlineto{\pgfqpoint{1.690774in}{1.338226in}}%
\pgfpathlineto{\pgfqpoint{1.691244in}{1.337663in}}%
\pgfpathlineto{\pgfqpoint{1.694379in}{1.337663in}}%
\pgfpathlineto{\pgfqpoint{1.697043in}{1.334474in}}%
\pgfpathlineto{\pgfqpoint{1.697514in}{1.333912in}}%
\pgfpathlineto{\pgfqpoint{1.700178in}{1.330723in}}%
\pgfpathlineto{\pgfqpoint{1.700648in}{1.330160in}}%
\pgfpathlineto{\pgfqpoint{1.703313in}{1.326971in}}%
\pgfpathlineto{\pgfqpoint{1.703783in}{1.326409in}}%
\pgfpathlineto{\pgfqpoint{1.706918in}{1.326409in}}%
\pgfpathlineto{\pgfqpoint{1.709582in}{1.323220in}}%
\pgfpathlineto{\pgfqpoint{1.710052in}{1.322657in}}%
\pgfpathlineto{\pgfqpoint{1.712717in}{1.319468in}}%
\pgfpathlineto{\pgfqpoint{1.713187in}{1.318906in}}%
\pgfpathlineto{\pgfqpoint{1.715852in}{1.315717in}}%
\pgfpathlineto{\pgfqpoint{1.716322in}{1.315154in}}%
\pgfpathlineto{\pgfqpoint{1.718986in}{1.311965in}}%
\pgfpathlineto{\pgfqpoint{1.719457in}{1.311402in}}%
\pgfpathlineto{\pgfqpoint{1.722591in}{1.311402in}}%
\pgfpathlineto{\pgfqpoint{1.725256in}{1.308214in}}%
\pgfpathlineto{\pgfqpoint{1.725726in}{1.307651in}}%
\pgfpathlineto{\pgfqpoint{1.728391in}{1.304462in}}%
\pgfpathlineto{\pgfqpoint{1.728861in}{1.303899in}}%
\pgfpathlineto{\pgfqpoint{1.731525in}{1.300711in}}%
\pgfpathlineto{\pgfqpoint{1.731996in}{1.300148in}}%
\pgfpathlineto{\pgfqpoint{1.735130in}{1.300148in}}%
\pgfpathlineto{\pgfqpoint{1.737795in}{1.296959in}}%
\pgfpathlineto{\pgfqpoint{1.738265in}{1.296396in}}%
\pgfpathlineto{\pgfqpoint{1.740930in}{1.293208in}}%
\pgfpathlineto{\pgfqpoint{1.741400in}{1.292645in}}%
\pgfpathlineto{\pgfqpoint{1.744064in}{1.289456in}}%
\pgfpathlineto{\pgfqpoint{1.744535in}{1.288893in}}%
\pgfpathlineto{\pgfqpoint{1.747669in}{1.288893in}}%
\pgfpathlineto{\pgfqpoint{1.750334in}{1.285704in}}%
\pgfpathlineto{\pgfqpoint{1.750804in}{1.285142in}}%
\pgfpathlineto{\pgfqpoint{1.753469in}{1.281953in}}%
\pgfpathlineto{\pgfqpoint{1.753939in}{1.281390in}}%
\pgfpathlineto{\pgfqpoint{1.756603in}{1.278201in}}%
\pgfpathlineto{\pgfqpoint{1.757074in}{1.277639in}}%
\pgfpathlineto{\pgfqpoint{1.760208in}{1.277639in}}%
\pgfpathlineto{\pgfqpoint{1.762873in}{1.274450in}}%
\pgfpathlineto{\pgfqpoint{1.763343in}{1.273887in}}%
\pgfpathlineto{\pgfqpoint{1.766008in}{1.270698in}}%
\pgfpathlineto{\pgfqpoint{1.766478in}{1.270136in}}%
\pgfpathlineto{\pgfqpoint{1.769142in}{1.266947in}}%
\pgfpathlineto{\pgfqpoint{1.769613in}{1.266384in}}%
\pgfpathlineto{\pgfqpoint{1.772747in}{1.266384in}}%
\pgfpathlineto{\pgfqpoint{1.775412in}{1.263195in}}%
\pgfpathlineto{\pgfqpoint{1.775882in}{1.262633in}}%
\pgfpathlineto{\pgfqpoint{1.778547in}{1.259444in}}%
\pgfpathlineto{\pgfqpoint{1.779017in}{1.258881in}}%
\pgfpathlineto{\pgfqpoint{1.781681in}{1.255692in}}%
\pgfpathlineto{\pgfqpoint{1.782152in}{1.255129in}}%
\pgfpathlineto{\pgfqpoint{1.785286in}{1.255129in}}%
\pgfpathlineto{\pgfqpoint{1.787951in}{1.251941in}}%
\pgfpathlineto{\pgfqpoint{1.788421in}{1.251378in}}%
\pgfpathlineto{\pgfqpoint{1.791086in}{1.248189in}}%
\pgfpathlineto{\pgfqpoint{1.791556in}{1.247626in}}%
\pgfpathlineto{\pgfqpoint{1.794220in}{1.244438in}}%
\pgfpathlineto{\pgfqpoint{1.794691in}{1.243875in}}%
\pgfpathlineto{\pgfqpoint{1.797355in}{1.240686in}}%
\pgfpathlineto{\pgfqpoint{1.797825in}{1.240123in}}%
\pgfpathlineto{\pgfqpoint{1.800960in}{1.240123in}}%
\pgfpathlineto{\pgfqpoint{1.803625in}{1.236935in}}%
\pgfpathlineto{\pgfqpoint{1.804095in}{1.236372in}}%
\pgfpathlineto{\pgfqpoint{1.806759in}{1.233183in}}%
\pgfpathlineto{\pgfqpoint{1.807229in}{1.232620in}}%
\pgfpathlineto{\pgfqpoint{1.809894in}{1.229431in}}%
\pgfpathlineto{\pgfqpoint{1.810364in}{1.228869in}}%
\pgfpathlineto{\pgfqpoint{1.813499in}{1.228869in}}%
\pgfpathlineto{\pgfqpoint{1.816164in}{1.225680in}}%
\pgfpathlineto{\pgfqpoint{1.816634in}{1.225117in}}%
\pgfpathlineto{\pgfqpoint{1.819298in}{1.221928in}}%
\pgfpathlineto{\pgfqpoint{1.819768in}{1.221366in}}%
\pgfpathlineto{\pgfqpoint{1.822433in}{1.218177in}}%
\pgfpathlineto{\pgfqpoint{1.822903in}{1.217614in}}%
\pgfpathlineto{\pgfqpoint{1.826038in}{1.217614in}}%
\pgfpathlineto{\pgfqpoint{1.828702in}{1.214425in}}%
\pgfpathlineto{\pgfqpoint{1.829173in}{1.213863in}}%
\pgfpathlineto{\pgfqpoint{1.831837in}{1.210674in}}%
\pgfpathlineto{\pgfqpoint{1.832307in}{1.210111in}}%
\pgfpathlineto{\pgfqpoint{1.834972in}{1.206922in}}%
\pgfpathlineto{\pgfqpoint{1.835442in}{1.206360in}}%
\pgfpathlineto{\pgfqpoint{1.838577in}{1.206360in}}%
\pgfpathlineto{\pgfqpoint{1.841241in}{1.203171in}}%
\pgfpathlineto{\pgfqpoint{1.841712in}{1.202608in}}%
\pgfpathlineto{\pgfqpoint{1.844376in}{1.199419in}}%
\pgfpathlineto{\pgfqpoint{1.844846in}{1.198856in}}%
\pgfpathlineto{\pgfqpoint{1.847511in}{1.195668in}}%
\pgfpathlineto{\pgfqpoint{1.847981in}{1.195105in}}%
\pgfpathlineto{\pgfqpoint{1.851116in}{1.195105in}}%
\pgfpathlineto{\pgfqpoint{1.853780in}{1.191916in}}%
\pgfpathlineto{\pgfqpoint{1.854251in}{1.191353in}}%
\pgfpathlineto{\pgfqpoint{1.856915in}{1.188165in}}%
\pgfpathlineto{\pgfqpoint{1.857385in}{1.187602in}}%
\pgfpathlineto{\pgfqpoint{1.860050in}{1.184413in}}%
\pgfpathlineto{\pgfqpoint{1.860520in}{1.183850in}}%
\pgfpathlineto{\pgfqpoint{1.863185in}{1.180662in}}%
\pgfpathlineto{\pgfqpoint{1.863655in}{1.180099in}}%
\pgfpathlineto{\pgfqpoint{1.866790in}{1.180099in}}%
\pgfpathlineto{\pgfqpoint{1.869454in}{1.176910in}}%
\pgfpathlineto{\pgfqpoint{1.869924in}{1.176347in}}%
\pgfpathlineto{\pgfqpoint{1.872589in}{1.173158in}}%
\pgfpathlineto{\pgfqpoint{1.873059in}{1.172596in}}%
\pgfpathlineto{\pgfqpoint{1.875724in}{1.169407in}}%
\pgfpathlineto{\pgfqpoint{1.876194in}{1.168844in}}%
\pgfpathlineto{\pgfqpoint{1.879329in}{1.168844in}}%
\pgfpathlineto{\pgfqpoint{1.881993in}{1.165655in}}%
\pgfpathlineto{\pgfqpoint{1.882463in}{1.165093in}}%
\pgfpathlineto{\pgfqpoint{1.885128in}{1.161904in}}%
\pgfpathlineto{\pgfqpoint{1.885598in}{1.161341in}}%
\pgfpathlineto{\pgfqpoint{1.888263in}{1.158152in}}%
\pgfpathlineto{\pgfqpoint{1.888733in}{1.157590in}}%
\pgfpathlineto{\pgfqpoint{1.891868in}{1.157590in}}%
\pgfpathlineto{\pgfqpoint{1.894532in}{1.154401in}}%
\pgfpathlineto{\pgfqpoint{1.895002in}{1.153838in}}%
\pgfpathlineto{\pgfqpoint{1.897667in}{1.150649in}}%
\pgfpathlineto{\pgfqpoint{1.898137in}{1.150087in}}%
\pgfpathlineto{\pgfqpoint{1.900802in}{1.146898in}}%
\pgfpathlineto{\pgfqpoint{1.901272in}{1.146335in}}%
\pgfpathlineto{\pgfqpoint{1.904407in}{1.146335in}}%
\pgfpathlineto{\pgfqpoint{1.907071in}{1.143146in}}%
\pgfpathlineto{\pgfqpoint{1.907541in}{1.142583in}}%
\pgfpathlineto{\pgfqpoint{1.910206in}{1.139395in}}%
\pgfpathlineto{\pgfqpoint{1.910676in}{1.138832in}}%
\pgfpathlineto{\pgfqpoint{1.913341in}{1.135643in}}%
\pgfpathlineto{\pgfqpoint{1.913811in}{1.135080in}}%
\pgfpathlineto{\pgfqpoint{1.916945in}{1.135080in}}%
\pgfpathlineto{\pgfqpoint{1.919610in}{1.131892in}}%
\pgfpathlineto{\pgfqpoint{1.920080in}{1.131329in}}%
\pgfpathlineto{\pgfqpoint{1.922745in}{1.128140in}}%
\pgfpathlineto{\pgfqpoint{1.923215in}{1.127577in}}%
\pgfpathlineto{\pgfqpoint{1.925879in}{1.124388in}}%
\pgfpathlineto{\pgfqpoint{1.926350in}{1.123826in}}%
\pgfpathlineto{\pgfqpoint{1.929484in}{1.123826in}}%
\pgfpathlineto{\pgfqpoint{1.932149in}{1.120637in}}%
\pgfpathlineto{\pgfqpoint{1.932619in}{1.120074in}}%
\pgfpathlineto{\pgfqpoint{1.935284in}{1.116885in}}%
\pgfpathlineto{\pgfqpoint{1.935754in}{1.116323in}}%
\pgfpathlineto{\pgfqpoint{1.938418in}{1.113134in}}%
\pgfpathlineto{\pgfqpoint{1.938889in}{1.112571in}}%
\pgfpathlineto{\pgfqpoint{1.941553in}{1.109382in}}%
\pgfpathlineto{\pgfqpoint{1.942023in}{1.108820in}}%
\pgfpathlineto{\pgfqpoint{1.945158in}{1.108820in}}%
\pgfpathlineto{\pgfqpoint{1.947823in}{1.105631in}}%
\pgfpathlineto{\pgfqpoint{1.948293in}{1.105068in}}%
\pgfpathlineto{\pgfqpoint{1.950957in}{1.101879in}}%
\pgfpathlineto{\pgfqpoint{1.951428in}{1.101317in}}%
\pgfpathlineto{\pgfqpoint{1.954092in}{1.098128in}}%
\pgfpathlineto{\pgfqpoint{1.954562in}{1.097565in}}%
\pgfpathlineto{\pgfqpoint{1.957697in}{1.097565in}}%
\pgfpathlineto{\pgfqpoint{1.960362in}{1.094376in}}%
\pgfpathlineto{\pgfqpoint{1.960832in}{1.093813in}}%
\pgfpathlineto{\pgfqpoint{1.963496in}{1.090625in}}%
\pgfpathlineto{\pgfqpoint{1.963967in}{1.090062in}}%
\pgfpathlineto{\pgfqpoint{1.966631in}{1.086873in}}%
\pgfpathlineto{\pgfqpoint{1.967101in}{1.086310in}}%
\pgfpathlineto{\pgfqpoint{1.970236in}{1.086310in}}%
\pgfpathlineto{\pgfqpoint{1.972901in}{1.083122in}}%
\pgfpathlineto{\pgfqpoint{1.973371in}{1.082559in}}%
\pgfpathlineto{\pgfqpoint{1.976035in}{1.079370in}}%
\pgfpathlineto{\pgfqpoint{1.976506in}{1.078807in}}%
\pgfpathlineto{\pgfqpoint{1.979170in}{1.075619in}}%
\pgfpathlineto{\pgfqpoint{1.979640in}{1.075056in}}%
\pgfpathlineto{\pgfqpoint{1.982775in}{1.075056in}}%
\pgfpathlineto{\pgfqpoint{1.985440in}{1.071867in}}%
\pgfpathlineto{\pgfqpoint{1.985910in}{1.071304in}}%
\pgfpathlineto{\pgfqpoint{1.988574in}{1.068115in}}%
\pgfpathlineto{\pgfqpoint{1.989045in}{1.067553in}}%
\pgfpathlineto{\pgfqpoint{1.991709in}{1.064364in}}%
\pgfpathlineto{\pgfqpoint{1.992179in}{1.063801in}}%
\pgfpathlineto{\pgfqpoint{1.995314in}{1.063801in}}%
\pgfpathlineto{\pgfqpoint{1.997979in}{1.060612in}}%
\pgfpathlineto{\pgfqpoint{1.998449in}{1.060050in}}%
\pgfpathlineto{\pgfqpoint{2.001113in}{1.056861in}}%
\pgfpathlineto{\pgfqpoint{2.001584in}{1.056298in}}%
\pgfpathlineto{\pgfqpoint{2.004248in}{1.053109in}}%
\pgfpathlineto{\pgfqpoint{2.004718in}{1.052547in}}%
\pgfpathlineto{\pgfqpoint{2.007853in}{1.052547in}}%
\pgfpathlineto{\pgfqpoint{2.010518in}{1.049358in}}%
\pgfpathlineto{\pgfqpoint{2.010988in}{1.048795in}}%
\pgfpathlineto{\pgfqpoint{2.013652in}{1.045606in}}%
\pgfpathlineto{\pgfqpoint{2.014122in}{1.045044in}}%
\pgfpathlineto{\pgfqpoint{2.016787in}{1.041855in}}%
\pgfpathlineto{\pgfqpoint{2.017257in}{1.041292in}}%
\pgfpathlineto{\pgfqpoint{2.019922in}{1.038103in}}%
\pgfpathlineto{\pgfqpoint{2.020392in}{1.037540in}}%
\pgfpathlineto{\pgfqpoint{2.023527in}{1.037540in}}%
\pgfpathlineto{\pgfqpoint{2.026191in}{1.034352in}}%
\pgfpathlineto{\pgfqpoint{2.026661in}{1.033789in}}%
\pgfpathlineto{\pgfqpoint{2.029326in}{1.030600in}}%
\pgfpathlineto{\pgfqpoint{2.029796in}{1.030037in}}%
\pgfpathlineto{\pgfqpoint{2.032461in}{1.026849in}}%
\pgfpathlineto{\pgfqpoint{2.032931in}{1.026286in}}%
\pgfpathlineto{\pgfqpoint{2.036066in}{1.026286in}}%
\pgfpathlineto{\pgfqpoint{2.038730in}{1.023097in}}%
\pgfpathlineto{\pgfqpoint{2.039200in}{1.022534in}}%
\pgfpathlineto{\pgfqpoint{2.041865in}{1.019346in}}%
\pgfpathlineto{\pgfqpoint{2.042335in}{1.018783in}}%
\pgfpathlineto{\pgfqpoint{2.045000in}{1.015594in}}%
\pgfpathlineto{\pgfqpoint{2.045470in}{1.015031in}}%
\pgfpathlineto{\pgfqpoint{2.048605in}{1.015031in}}%
\pgfpathlineto{\pgfqpoint{2.051269in}{1.011842in}}%
\pgfpathlineto{\pgfqpoint{2.051739in}{1.011280in}}%
\pgfpathlineto{\pgfqpoint{2.054404in}{1.008091in}}%
\pgfpathlineto{\pgfqpoint{2.054874in}{1.007528in}}%
\pgfpathlineto{\pgfqpoint{2.057539in}{1.004339in}}%
\pgfpathlineto{\pgfqpoint{2.058009in}{1.003777in}}%
\pgfpathlineto{\pgfqpoint{2.061144in}{1.003777in}}%
\pgfpathlineto{\pgfqpoint{2.063808in}{1.000588in}}%
\pgfpathlineto{\pgfqpoint{2.064278in}{1.000025in}}%
\pgfpathlineto{\pgfqpoint{2.066943in}{0.996836in}}%
\pgfpathlineto{\pgfqpoint{2.067413in}{0.996274in}}%
\pgfpathlineto{\pgfqpoint{2.070078in}{0.993085in}}%
\pgfpathlineto{\pgfqpoint{2.070548in}{0.992522in}}%
\pgfpathlineto{\pgfqpoint{2.073683in}{0.992522in}}%
\pgfpathlineto{\pgfqpoint{2.076347in}{0.989333in}}%
\pgfpathlineto{\pgfqpoint{2.076817in}{0.988771in}}%
\pgfpathlineto{\pgfqpoint{2.079482in}{0.985582in}}%
\pgfpathlineto{\pgfqpoint{2.079952in}{0.985019in}}%
\pgfpathlineto{\pgfqpoint{2.082617in}{0.981830in}}%
\pgfpathlineto{\pgfqpoint{2.083087in}{0.981267in}}%
\pgfpathlineto{\pgfqpoint{2.086222in}{0.981267in}}%
\pgfpathlineto{\pgfqpoint{2.088886in}{0.978079in}}%
\pgfpathlineto{\pgfqpoint{2.089356in}{0.977516in}}%
\pgfpathlineto{\pgfqpoint{2.092021in}{0.974327in}}%
\pgfpathlineto{\pgfqpoint{2.092491in}{0.973764in}}%
\pgfpathlineto{\pgfqpoint{2.095156in}{0.970576in}}%
\pgfpathlineto{\pgfqpoint{2.095626in}{0.970013in}}%
\pgfpathlineto{\pgfqpoint{2.098290in}{0.966824in}}%
\pgfpathlineto{\pgfqpoint{2.098761in}{0.966261in}}%
\pgfpathlineto{\pgfqpoint{2.101895in}{0.966261in}}%
\pgfpathlineto{\pgfqpoint{2.104560in}{0.963073in}}%
\pgfpathlineto{\pgfqpoint{2.105030in}{0.962510in}}%
\pgfpathlineto{\pgfqpoint{2.107695in}{0.959321in}}%
\pgfpathlineto{\pgfqpoint{2.108165in}{0.958758in}}%
\pgfpathlineto{\pgfqpoint{2.110829in}{0.955569in}}%
\pgfpathlineto{\pgfqpoint{2.111299in}{0.955007in}}%
\pgfpathlineto{\pgfqpoint{2.114434in}{0.955007in}}%
\pgfpathlineto{\pgfqpoint{2.117099in}{0.951818in}}%
\pgfpathlineto{\pgfqpoint{2.117569in}{0.951255in}}%
\pgfpathlineto{\pgfqpoint{2.120234in}{0.948066in}}%
\pgfpathlineto{\pgfqpoint{2.120704in}{0.947504in}}%
\pgfpathlineto{\pgfqpoint{2.123368in}{0.944315in}}%
\pgfpathlineto{\pgfqpoint{2.123838in}{0.943752in}}%
\pgfpathlineto{\pgfqpoint{2.126973in}{0.943752in}}%
\pgfpathlineto{\pgfqpoint{2.129638in}{0.940563in}}%
\pgfpathlineto{\pgfqpoint{2.130108in}{0.940001in}}%
\pgfpathlineto{\pgfqpoint{2.132772in}{0.936812in}}%
\pgfpathlineto{\pgfqpoint{2.133243in}{0.936249in}}%
\pgfpathlineto{\pgfqpoint{2.135907in}{0.933060in}}%
\pgfpathlineto{\pgfqpoint{2.136377in}{0.932498in}}%
\pgfpathlineto{\pgfqpoint{2.139512in}{0.932498in}}%
\pgfpathlineto{\pgfqpoint{2.142177in}{0.929309in}}%
\pgfpathlineto{\pgfqpoint{2.142647in}{0.928746in}}%
\pgfpathlineto{\pgfqpoint{2.145311in}{0.925557in}}%
\pgfpathlineto{\pgfqpoint{2.145782in}{0.924994in}}%
\pgfpathlineto{\pgfqpoint{2.148446in}{0.921806in}}%
\pgfpathlineto{\pgfqpoint{2.148916in}{0.921243in}}%
\pgfpathlineto{\pgfqpoint{2.152051in}{0.921243in}}%
\pgfpathlineto{\pgfqpoint{2.154716in}{0.918054in}}%
\pgfpathlineto{\pgfqpoint{2.155186in}{0.917491in}}%
\pgfpathlineto{\pgfqpoint{2.157850in}{0.914303in}}%
\pgfpathlineto{\pgfqpoint{2.158321in}{0.913740in}}%
\pgfpathlineto{\pgfqpoint{2.160985in}{0.910551in}}%
\pgfpathlineto{\pgfqpoint{2.161455in}{0.909988in}}%
\pgfpathlineto{\pgfqpoint{2.164120in}{0.906799in}}%
\pgfpathlineto{\pgfqpoint{2.164590in}{0.906237in}}%
\pgfpathlineto{\pgfqpoint{2.167725in}{0.906237in}}%
\pgfpathlineto{\pgfqpoint{2.170389in}{0.903048in}}%
\pgfpathlineto{\pgfqpoint{2.170860in}{0.902485in}}%
\pgfpathlineto{\pgfqpoint{2.173524in}{0.899296in}}%
\pgfpathclose%
\pgfusepath{fill}%
\end{pgfscope}%
\begin{pgfscope}%
\pgfpathrectangle{\pgfqpoint{0.888750in}{0.419100in}}{\pgfqpoint{2.504659in}{2.933700in}} %
\pgfusepath{clip}%
\pgfsetbuttcap%
\pgfsetroundjoin%
\definecolor{currentfill}{rgb}{1.000000,0.647059,0.000000}%
\pgfsetfillcolor{currentfill}%
\pgfsetfillopacity{0.300000}%
\pgfsetlinewidth{0.000000pt}%
\definecolor{currentstroke}{rgb}{0.000000,0.000000,0.000000}%
\pgfsetstrokecolor{currentstroke}%
\pgfsetdash{}{0pt}%
\pgfpathmoveto{\pgfqpoint{2.173994in}{0.898171in}}%
\pgfpathlineto{\pgfqpoint{2.177129in}{0.898171in}}%
\pgfpathlineto{\pgfqpoint{2.180264in}{0.898171in}}%
\pgfpathlineto{\pgfqpoint{2.181204in}{0.899296in}}%
\pgfpathlineto{\pgfqpoint{2.183399in}{0.901922in}}%
\pgfpathlineto{\pgfqpoint{2.186533in}{0.901922in}}%
\pgfpathlineto{\pgfqpoint{2.189668in}{0.901922in}}%
\pgfpathlineto{\pgfqpoint{2.190608in}{0.903048in}}%
\pgfpathlineto{\pgfqpoint{2.192803in}{0.905674in}}%
\pgfpathlineto{\pgfqpoint{2.195938in}{0.905674in}}%
\pgfpathlineto{\pgfqpoint{2.196878in}{0.906799in}}%
\pgfpathlineto{\pgfqpoint{2.199072in}{0.909426in}}%
\pgfpathlineto{\pgfqpoint{2.202207in}{0.909426in}}%
\pgfpathlineto{\pgfqpoint{2.205342in}{0.909426in}}%
\pgfpathlineto{\pgfqpoint{2.206282in}{0.910551in}}%
\pgfpathlineto{\pgfqpoint{2.208477in}{0.913177in}}%
\pgfpathlineto{\pgfqpoint{2.211611in}{0.913177in}}%
\pgfpathlineto{\pgfqpoint{2.214746in}{0.913177in}}%
\pgfpathlineto{\pgfqpoint{2.215686in}{0.914303in}}%
\pgfpathlineto{\pgfqpoint{2.217881in}{0.916929in}}%
\pgfpathlineto{\pgfqpoint{2.221015in}{0.916929in}}%
\pgfpathlineto{\pgfqpoint{2.224150in}{0.916929in}}%
\pgfpathlineto{\pgfqpoint{2.225091in}{0.918054in}}%
\pgfpathlineto{\pgfqpoint{2.227285in}{0.920680in}}%
\pgfpathlineto{\pgfqpoint{2.230420in}{0.920680in}}%
\pgfpathlineto{\pgfqpoint{2.231360in}{0.921806in}}%
\pgfpathlineto{\pgfqpoint{2.233554in}{0.924432in}}%
\pgfpathlineto{\pgfqpoint{2.236689in}{0.924432in}}%
\pgfpathlineto{\pgfqpoint{2.239824in}{0.924432in}}%
\pgfpathlineto{\pgfqpoint{2.240764in}{0.925557in}}%
\pgfpathlineto{\pgfqpoint{2.242959in}{0.928183in}}%
\pgfpathlineto{\pgfqpoint{2.246093in}{0.928183in}}%
\pgfpathlineto{\pgfqpoint{2.249228in}{0.928183in}}%
\pgfpathlineto{\pgfqpoint{2.250169in}{0.929309in}}%
\pgfpathlineto{\pgfqpoint{2.252363in}{0.931935in}}%
\pgfpathlineto{\pgfqpoint{2.255498in}{0.931935in}}%
\pgfpathlineto{\pgfqpoint{2.258632in}{0.931935in}}%
\pgfpathlineto{\pgfqpoint{2.259573in}{0.933060in}}%
\pgfpathlineto{\pgfqpoint{2.261767in}{0.935686in}}%
\pgfpathlineto{\pgfqpoint{2.264902in}{0.935686in}}%
\pgfpathlineto{\pgfqpoint{2.268037in}{0.935686in}}%
\pgfpathlineto{\pgfqpoint{2.268977in}{0.936812in}}%
\pgfpathlineto{\pgfqpoint{2.271171in}{0.939438in}}%
\pgfpathlineto{\pgfqpoint{2.274306in}{0.939438in}}%
\pgfpathlineto{\pgfqpoint{2.275247in}{0.940563in}}%
\pgfpathlineto{\pgfqpoint{2.277441in}{0.943189in}}%
\pgfpathlineto{\pgfqpoint{2.280576in}{0.943189in}}%
\pgfpathlineto{\pgfqpoint{2.283710in}{0.943189in}}%
\pgfpathlineto{\pgfqpoint{2.284651in}{0.944315in}}%
\pgfpathlineto{\pgfqpoint{2.286845in}{0.946941in}}%
\pgfpathlineto{\pgfqpoint{2.289980in}{0.946941in}}%
\pgfpathlineto{\pgfqpoint{2.293115in}{0.946941in}}%
\pgfpathlineto{\pgfqpoint{2.294055in}{0.948066in}}%
\pgfpathlineto{\pgfqpoint{2.296249in}{0.950692in}}%
\pgfpathlineto{\pgfqpoint{2.299384in}{0.950692in}}%
\pgfpathlineto{\pgfqpoint{2.302519in}{0.950692in}}%
\pgfpathlineto{\pgfqpoint{2.303459in}{0.951818in}}%
\pgfpathlineto{\pgfqpoint{2.305654in}{0.954444in}}%
\pgfpathlineto{\pgfqpoint{2.308788in}{0.954444in}}%
\pgfpathlineto{\pgfqpoint{2.309729in}{0.955569in}}%
\pgfpathlineto{\pgfqpoint{2.311923in}{0.958196in}}%
\pgfpathlineto{\pgfqpoint{2.315058in}{0.958196in}}%
\pgfpathlineto{\pgfqpoint{2.318192in}{0.958196in}}%
\pgfpathlineto{\pgfqpoint{2.319133in}{0.959321in}}%
\pgfpathlineto{\pgfqpoint{2.321327in}{0.961947in}}%
\pgfpathlineto{\pgfqpoint{2.324462in}{0.961947in}}%
\pgfpathlineto{\pgfqpoint{2.327597in}{0.961947in}}%
\pgfpathlineto{\pgfqpoint{2.328537in}{0.963073in}}%
\pgfpathlineto{\pgfqpoint{2.330731in}{0.965699in}}%
\pgfpathlineto{\pgfqpoint{2.333866in}{0.965699in}}%
\pgfpathlineto{\pgfqpoint{2.337001in}{0.965699in}}%
\pgfpathlineto{\pgfqpoint{2.337941in}{0.966824in}}%
\pgfpathlineto{\pgfqpoint{2.340136in}{0.969450in}}%
\pgfpathlineto{\pgfqpoint{2.343270in}{0.969450in}}%
\pgfpathlineto{\pgfqpoint{2.344211in}{0.970576in}}%
\pgfpathlineto{\pgfqpoint{2.346405in}{0.973202in}}%
\pgfpathlineto{\pgfqpoint{2.349540in}{0.973202in}}%
\pgfpathlineto{\pgfqpoint{2.352675in}{0.973202in}}%
\pgfpathlineto{\pgfqpoint{2.353615in}{0.974327in}}%
\pgfpathlineto{\pgfqpoint{2.355809in}{0.976953in}}%
\pgfpathlineto{\pgfqpoint{2.358944in}{0.976953in}}%
\pgfpathlineto{\pgfqpoint{2.362079in}{0.976953in}}%
\pgfpathlineto{\pgfqpoint{2.363019in}{0.978079in}}%
\pgfpathlineto{\pgfqpoint{2.365214in}{0.980705in}}%
\pgfpathlineto{\pgfqpoint{2.368348in}{0.980705in}}%
\pgfpathlineto{\pgfqpoint{2.371483in}{0.980705in}}%
\pgfpathlineto{\pgfqpoint{2.372424in}{0.981830in}}%
\pgfpathlineto{\pgfqpoint{2.374618in}{0.984456in}}%
\pgfpathlineto{\pgfqpoint{2.377753in}{0.984456in}}%
\pgfpathlineto{\pgfqpoint{2.378693in}{0.985582in}}%
\pgfpathlineto{\pgfqpoint{2.380887in}{0.988208in}}%
\pgfpathlineto{\pgfqpoint{2.384022in}{0.988208in}}%
\pgfpathlineto{\pgfqpoint{2.387157in}{0.988208in}}%
\pgfpathlineto{\pgfqpoint{2.388097in}{0.989333in}}%
\pgfpathlineto{\pgfqpoint{2.390292in}{0.991959in}}%
\pgfpathlineto{\pgfqpoint{2.393426in}{0.991959in}}%
\pgfpathlineto{\pgfqpoint{2.396561in}{0.991959in}}%
\pgfpathlineto{\pgfqpoint{2.397501in}{0.993085in}}%
\pgfpathlineto{\pgfqpoint{2.399696in}{0.995711in}}%
\pgfpathlineto{\pgfqpoint{2.402831in}{0.995711in}}%
\pgfpathlineto{\pgfqpoint{2.405965in}{0.995711in}}%
\pgfpathlineto{\pgfqpoint{2.406906in}{0.996836in}}%
\pgfpathlineto{\pgfqpoint{2.409100in}{0.999462in}}%
\pgfpathlineto{\pgfqpoint{2.412235in}{0.999462in}}%
\pgfpathlineto{\pgfqpoint{2.415369in}{0.999462in}}%
\pgfpathlineto{\pgfqpoint{2.416310in}{1.000588in}}%
\pgfpathlineto{\pgfqpoint{2.418504in}{1.003214in}}%
\pgfpathlineto{\pgfqpoint{2.421639in}{1.003214in}}%
\pgfpathlineto{\pgfqpoint{2.422579in}{1.004339in}}%
\pgfpathlineto{\pgfqpoint{2.424774in}{1.006965in}}%
\pgfpathlineto{\pgfqpoint{2.427908in}{1.006965in}}%
\pgfpathlineto{\pgfqpoint{2.431043in}{1.006965in}}%
\pgfpathlineto{\pgfqpoint{2.431984in}{1.008091in}}%
\pgfpathlineto{\pgfqpoint{2.434178in}{1.010717in}}%
\pgfpathlineto{\pgfqpoint{2.437313in}{1.010717in}}%
\pgfpathlineto{\pgfqpoint{2.440447in}{1.010717in}}%
\pgfpathlineto{\pgfqpoint{2.441388in}{1.011842in}}%
\pgfpathlineto{\pgfqpoint{2.443582in}{1.014469in}}%
\pgfpathlineto{\pgfqpoint{2.446717in}{1.014469in}}%
\pgfpathlineto{\pgfqpoint{2.449852in}{1.014469in}}%
\pgfpathlineto{\pgfqpoint{2.450792in}{1.015594in}}%
\pgfpathlineto{\pgfqpoint{2.452986in}{1.018220in}}%
\pgfpathlineto{\pgfqpoint{2.456121in}{1.018220in}}%
\pgfpathlineto{\pgfqpoint{2.457062in}{1.019346in}}%
\pgfpathlineto{\pgfqpoint{2.459256in}{1.021972in}}%
\pgfpathlineto{\pgfqpoint{2.462391in}{1.021972in}}%
\pgfpathlineto{\pgfqpoint{2.465525in}{1.021972in}}%
\pgfpathlineto{\pgfqpoint{2.466466in}{1.023097in}}%
\pgfpathlineto{\pgfqpoint{2.468660in}{1.025723in}}%
\pgfpathlineto{\pgfqpoint{2.471795in}{1.025723in}}%
\pgfpathlineto{\pgfqpoint{2.474930in}{1.025723in}}%
\pgfpathlineto{\pgfqpoint{2.475870in}{1.026849in}}%
\pgfpathlineto{\pgfqpoint{2.478064in}{1.029475in}}%
\pgfpathlineto{\pgfqpoint{2.481199in}{1.029475in}}%
\pgfpathlineto{\pgfqpoint{2.484334in}{1.029475in}}%
\pgfpathlineto{\pgfqpoint{2.485274in}{1.030600in}}%
\pgfpathlineto{\pgfqpoint{2.487469in}{1.033226in}}%
\pgfpathlineto{\pgfqpoint{2.490603in}{1.033226in}}%
\pgfpathlineto{\pgfqpoint{2.491544in}{1.034352in}}%
\pgfpathlineto{\pgfqpoint{2.493738in}{1.036978in}}%
\pgfpathlineto{\pgfqpoint{2.496873in}{1.036978in}}%
\pgfpathlineto{\pgfqpoint{2.500008in}{1.036978in}}%
\pgfpathlineto{\pgfqpoint{2.500948in}{1.038103in}}%
\pgfpathlineto{\pgfqpoint{2.503142in}{1.040729in}}%
\pgfpathlineto{\pgfqpoint{2.506277in}{1.040729in}}%
\pgfpathlineto{\pgfqpoint{2.509412in}{1.040729in}}%
\pgfpathlineto{\pgfqpoint{2.510352in}{1.041855in}}%
\pgfpathlineto{\pgfqpoint{2.512547in}{1.044481in}}%
\pgfpathlineto{\pgfqpoint{2.515681in}{1.044481in}}%
\pgfpathlineto{\pgfqpoint{2.518816in}{1.044481in}}%
\pgfpathlineto{\pgfqpoint{2.519756in}{1.045606in}}%
\pgfpathlineto{\pgfqpoint{2.521951in}{1.048232in}}%
\pgfpathlineto{\pgfqpoint{2.525085in}{1.048232in}}%
\pgfpathlineto{\pgfqpoint{2.528220in}{1.048232in}}%
\pgfpathlineto{\pgfqpoint{2.529161in}{1.049358in}}%
\pgfpathlineto{\pgfqpoint{2.531355in}{1.051984in}}%
\pgfpathlineto{\pgfqpoint{2.534490in}{1.051984in}}%
\pgfpathlineto{\pgfqpoint{2.535430in}{1.053109in}}%
\pgfpathlineto{\pgfqpoint{2.537624in}{1.055735in}}%
\pgfpathlineto{\pgfqpoint{2.540759in}{1.055735in}}%
\pgfpathlineto{\pgfqpoint{2.543894in}{1.055735in}}%
\pgfpathlineto{\pgfqpoint{2.544834in}{1.056861in}}%
\pgfpathlineto{\pgfqpoint{2.547029in}{1.059487in}}%
\pgfpathlineto{\pgfqpoint{2.550163in}{1.059487in}}%
\pgfpathlineto{\pgfqpoint{2.553298in}{1.059487in}}%
\pgfpathlineto{\pgfqpoint{2.554239in}{1.060612in}}%
\pgfpathlineto{\pgfqpoint{2.556433in}{1.063238in}}%
\pgfpathlineto{\pgfqpoint{2.559568in}{1.063238in}}%
\pgfpathlineto{\pgfqpoint{2.562702in}{1.063238in}}%
\pgfpathlineto{\pgfqpoint{2.563643in}{1.064364in}}%
\pgfpathlineto{\pgfqpoint{2.565837in}{1.066990in}}%
\pgfpathlineto{\pgfqpoint{2.568972in}{1.066990in}}%
\pgfpathlineto{\pgfqpoint{2.569912in}{1.068115in}}%
\pgfpathlineto{\pgfqpoint{2.572107in}{1.070742in}}%
\pgfpathlineto{\pgfqpoint{2.575241in}{1.070742in}}%
\pgfpathlineto{\pgfqpoint{2.578376in}{1.070742in}}%
\pgfpathlineto{\pgfqpoint{2.579317in}{1.071867in}}%
\pgfpathlineto{\pgfqpoint{2.581511in}{1.074493in}}%
\pgfpathlineto{\pgfqpoint{2.584646in}{1.074493in}}%
\pgfpathlineto{\pgfqpoint{2.587780in}{1.074493in}}%
\pgfpathlineto{\pgfqpoint{2.588721in}{1.075619in}}%
\pgfpathlineto{\pgfqpoint{2.590915in}{1.078245in}}%
\pgfpathlineto{\pgfqpoint{2.594050in}{1.078245in}}%
\pgfpathlineto{\pgfqpoint{2.597185in}{1.078245in}}%
\pgfpathlineto{\pgfqpoint{2.598125in}{1.079370in}}%
\pgfpathlineto{\pgfqpoint{2.600319in}{1.081996in}}%
\pgfpathlineto{\pgfqpoint{2.603454in}{1.081996in}}%
\pgfpathlineto{\pgfqpoint{2.604394in}{1.083122in}}%
\pgfpathlineto{\pgfqpoint{2.606589in}{1.085748in}}%
\pgfpathlineto{\pgfqpoint{2.609724in}{1.085748in}}%
\pgfpathlineto{\pgfqpoint{2.612858in}{1.085748in}}%
\pgfpathlineto{\pgfqpoint{2.613799in}{1.086873in}}%
\pgfpathlineto{\pgfqpoint{2.615993in}{1.089499in}}%
\pgfpathlineto{\pgfqpoint{2.619128in}{1.089499in}}%
\pgfpathlineto{\pgfqpoint{2.622262in}{1.089499in}}%
\pgfpathlineto{\pgfqpoint{2.623203in}{1.090625in}}%
\pgfpathlineto{\pgfqpoint{2.625397in}{1.093251in}}%
\pgfpathlineto{\pgfqpoint{2.628532in}{1.093251in}}%
\pgfpathlineto{\pgfqpoint{2.631667in}{1.093251in}}%
\pgfpathlineto{\pgfqpoint{2.632607in}{1.094376in}}%
\pgfpathlineto{\pgfqpoint{2.634801in}{1.097002in}}%
\pgfpathlineto{\pgfqpoint{2.637936in}{1.097002in}}%
\pgfpathlineto{\pgfqpoint{2.641071in}{1.097002in}}%
\pgfpathlineto{\pgfqpoint{2.642011in}{1.098128in}}%
\pgfpathlineto{\pgfqpoint{2.644206in}{1.100754in}}%
\pgfpathlineto{\pgfqpoint{2.647340in}{1.100754in}}%
\pgfpathlineto{\pgfqpoint{2.648281in}{1.101879in}}%
\pgfpathlineto{\pgfqpoint{2.650475in}{1.104505in}}%
\pgfpathlineto{\pgfqpoint{2.653610in}{1.104505in}}%
\pgfpathlineto{\pgfqpoint{2.656745in}{1.104505in}}%
\pgfpathlineto{\pgfqpoint{2.657685in}{1.105631in}}%
\pgfpathlineto{\pgfqpoint{2.659879in}{1.108257in}}%
\pgfpathlineto{\pgfqpoint{2.663014in}{1.108257in}}%
\pgfpathlineto{\pgfqpoint{2.666149in}{1.108257in}}%
\pgfpathlineto{\pgfqpoint{2.667089in}{1.109382in}}%
\pgfpathlineto{\pgfqpoint{2.669284in}{1.112008in}}%
\pgfpathlineto{\pgfqpoint{2.672418in}{1.112008in}}%
\pgfpathlineto{\pgfqpoint{2.675553in}{1.112008in}}%
\pgfpathlineto{\pgfqpoint{2.676494in}{1.113134in}}%
\pgfpathlineto{\pgfqpoint{2.678688in}{1.115760in}}%
\pgfpathlineto{\pgfqpoint{2.681823in}{1.115760in}}%
\pgfpathlineto{\pgfqpoint{2.682763in}{1.116885in}}%
\pgfpathlineto{\pgfqpoint{2.684957in}{1.119511in}}%
\pgfpathlineto{\pgfqpoint{2.688092in}{1.119511in}}%
\pgfpathlineto{\pgfqpoint{2.691227in}{1.119511in}}%
\pgfpathlineto{\pgfqpoint{2.692167in}{1.120637in}}%
\pgfpathlineto{\pgfqpoint{2.694362in}{1.123263in}}%
\pgfpathlineto{\pgfqpoint{2.697496in}{1.123263in}}%
\pgfpathlineto{\pgfqpoint{2.700631in}{1.123263in}}%
\pgfpathlineto{\pgfqpoint{2.701571in}{1.124388in}}%
\pgfpathlineto{\pgfqpoint{2.703766in}{1.127015in}}%
\pgfpathlineto{\pgfqpoint{2.706901in}{1.127015in}}%
\pgfpathlineto{\pgfqpoint{2.710035in}{1.127015in}}%
\pgfpathlineto{\pgfqpoint{2.710976in}{1.128140in}}%
\pgfpathlineto{\pgfqpoint{2.713170in}{1.130766in}}%
\pgfpathlineto{\pgfqpoint{2.716305in}{1.130766in}}%
\pgfpathlineto{\pgfqpoint{2.717245in}{1.131892in}}%
\pgfpathlineto{\pgfqpoint{2.719439in}{1.134518in}}%
\pgfpathlineto{\pgfqpoint{2.722574in}{1.134518in}}%
\pgfpathlineto{\pgfqpoint{2.725709in}{1.134518in}}%
\pgfpathlineto{\pgfqpoint{2.726649in}{1.135643in}}%
\pgfpathlineto{\pgfqpoint{2.728844in}{1.138269in}}%
\pgfpathlineto{\pgfqpoint{2.731978in}{1.138269in}}%
\pgfpathlineto{\pgfqpoint{2.735113in}{1.138269in}}%
\pgfpathlineto{\pgfqpoint{2.736054in}{1.139395in}}%
\pgfpathlineto{\pgfqpoint{2.738248in}{1.142021in}}%
\pgfpathlineto{\pgfqpoint{2.741383in}{1.142021in}}%
\pgfpathlineto{\pgfqpoint{2.744517in}{1.142021in}}%
\pgfpathlineto{\pgfqpoint{2.745458in}{1.143146in}}%
\pgfpathlineto{\pgfqpoint{2.747652in}{1.145772in}}%
\pgfpathlineto{\pgfqpoint{2.750787in}{1.145772in}}%
\pgfpathlineto{\pgfqpoint{2.753922in}{1.145772in}}%
\pgfpathlineto{\pgfqpoint{2.754862in}{1.146898in}}%
\pgfpathlineto{\pgfqpoint{2.757056in}{1.149524in}}%
\pgfpathlineto{\pgfqpoint{2.760191in}{1.149524in}}%
\pgfpathlineto{\pgfqpoint{2.761132in}{1.150649in}}%
\pgfpathlineto{\pgfqpoint{2.763326in}{1.153275in}}%
\pgfpathlineto{\pgfqpoint{2.766461in}{1.153275in}}%
\pgfpathlineto{\pgfqpoint{2.769595in}{1.153275in}}%
\pgfpathlineto{\pgfqpoint{2.770536in}{1.154401in}}%
\pgfpathlineto{\pgfqpoint{2.772730in}{1.157027in}}%
\pgfpathlineto{\pgfqpoint{2.775865in}{1.157027in}}%
\pgfpathlineto{\pgfqpoint{2.779000in}{1.157027in}}%
\pgfpathlineto{\pgfqpoint{2.779940in}{1.158152in}}%
\pgfpathlineto{\pgfqpoint{2.782134in}{1.160778in}}%
\pgfpathlineto{\pgfqpoint{2.785269in}{1.160778in}}%
\pgfpathlineto{\pgfqpoint{2.788404in}{1.160778in}}%
\pgfpathlineto{\pgfqpoint{2.789344in}{1.161904in}}%
\pgfpathlineto{\pgfqpoint{2.791539in}{1.164530in}}%
\pgfpathlineto{\pgfqpoint{2.794673in}{1.164530in}}%
\pgfpathlineto{\pgfqpoint{2.795614in}{1.165655in}}%
\pgfpathlineto{\pgfqpoint{2.797808in}{1.168281in}}%
\pgfpathlineto{\pgfqpoint{2.800943in}{1.168281in}}%
\pgfpathlineto{\pgfqpoint{2.804078in}{1.168281in}}%
\pgfpathlineto{\pgfqpoint{2.805018in}{1.169407in}}%
\pgfpathlineto{\pgfqpoint{2.807212in}{1.172033in}}%
\pgfpathlineto{\pgfqpoint{2.810347in}{1.172033in}}%
\pgfpathlineto{\pgfqpoint{2.813482in}{1.172033in}}%
\pgfpathlineto{\pgfqpoint{2.814422in}{1.173158in}}%
\pgfpathlineto{\pgfqpoint{2.814422in}{1.176910in}}%
\pgfpathlineto{\pgfqpoint{2.816617in}{1.179536in}}%
\pgfpathlineto{\pgfqpoint{2.817557in}{1.180662in}}%
\pgfpathlineto{\pgfqpoint{2.817557in}{1.184413in}}%
\pgfpathlineto{\pgfqpoint{2.817557in}{1.188165in}}%
\pgfpathlineto{\pgfqpoint{2.819751in}{1.190791in}}%
\pgfpathlineto{\pgfqpoint{2.820692in}{1.191916in}}%
\pgfpathlineto{\pgfqpoint{2.820692in}{1.195668in}}%
\pgfpathlineto{\pgfqpoint{2.820692in}{1.199419in}}%
\pgfpathlineto{\pgfqpoint{2.822886in}{1.202045in}}%
\pgfpathlineto{\pgfqpoint{2.823826in}{1.203171in}}%
\pgfpathlineto{\pgfqpoint{2.823826in}{1.206922in}}%
\pgfpathlineto{\pgfqpoint{2.823826in}{1.210674in}}%
\pgfpathlineto{\pgfqpoint{2.826021in}{1.213300in}}%
\pgfpathlineto{\pgfqpoint{2.826961in}{1.214425in}}%
\pgfpathlineto{\pgfqpoint{2.826961in}{1.218177in}}%
\pgfpathlineto{\pgfqpoint{2.826961in}{1.221928in}}%
\pgfpathlineto{\pgfqpoint{2.829155in}{1.224554in}}%
\pgfpathlineto{\pgfqpoint{2.830096in}{1.225680in}}%
\pgfpathlineto{\pgfqpoint{2.830096in}{1.229431in}}%
\pgfpathlineto{\pgfqpoint{2.830096in}{1.233183in}}%
\pgfpathlineto{\pgfqpoint{2.832290in}{1.235809in}}%
\pgfpathlineto{\pgfqpoint{2.833231in}{1.236935in}}%
\pgfpathlineto{\pgfqpoint{2.833231in}{1.240686in}}%
\pgfpathlineto{\pgfqpoint{2.833231in}{1.244438in}}%
\pgfpathlineto{\pgfqpoint{2.835425in}{1.247064in}}%
\pgfpathlineto{\pgfqpoint{2.836365in}{1.248189in}}%
\pgfpathlineto{\pgfqpoint{2.836365in}{1.251941in}}%
\pgfpathlineto{\pgfqpoint{2.836365in}{1.255692in}}%
\pgfpathlineto{\pgfqpoint{2.836365in}{1.259444in}}%
\pgfpathlineto{\pgfqpoint{2.838560in}{1.262070in}}%
\pgfpathlineto{\pgfqpoint{2.839500in}{1.263195in}}%
\pgfpathlineto{\pgfqpoint{2.839500in}{1.266947in}}%
\pgfpathlineto{\pgfqpoint{2.839500in}{1.270698in}}%
\pgfpathlineto{\pgfqpoint{2.841694in}{1.273324in}}%
\pgfpathlineto{\pgfqpoint{2.842635in}{1.274450in}}%
\pgfpathlineto{\pgfqpoint{2.842635in}{1.278201in}}%
\pgfpathlineto{\pgfqpoint{2.842635in}{1.281953in}}%
\pgfpathlineto{\pgfqpoint{2.844829in}{1.284579in}}%
\pgfpathlineto{\pgfqpoint{2.845770in}{1.285704in}}%
\pgfpathlineto{\pgfqpoint{2.845770in}{1.289456in}}%
\pgfpathlineto{\pgfqpoint{2.845770in}{1.293208in}}%
\pgfpathlineto{\pgfqpoint{2.847964in}{1.295834in}}%
\pgfpathlineto{\pgfqpoint{2.848904in}{1.296959in}}%
\pgfpathlineto{\pgfqpoint{2.848904in}{1.300711in}}%
\pgfpathlineto{\pgfqpoint{2.848904in}{1.304462in}}%
\pgfpathlineto{\pgfqpoint{2.851099in}{1.307088in}}%
\pgfpathlineto{\pgfqpoint{2.852039in}{1.308214in}}%
\pgfpathlineto{\pgfqpoint{2.852039in}{1.311965in}}%
\pgfpathlineto{\pgfqpoint{2.852039in}{1.315717in}}%
\pgfpathlineto{\pgfqpoint{2.854233in}{1.318343in}}%
\pgfpathlineto{\pgfqpoint{2.855174in}{1.319468in}}%
\pgfpathlineto{\pgfqpoint{2.855174in}{1.323220in}}%
\pgfpathlineto{\pgfqpoint{2.855174in}{1.326971in}}%
\pgfpathlineto{\pgfqpoint{2.857368in}{1.329597in}}%
\pgfpathlineto{\pgfqpoint{2.858309in}{1.330723in}}%
\pgfpathlineto{\pgfqpoint{2.858309in}{1.334474in}}%
\pgfpathlineto{\pgfqpoint{2.858309in}{1.338226in}}%
\pgfpathlineto{\pgfqpoint{2.860503in}{1.340852in}}%
\pgfpathlineto{\pgfqpoint{2.861443in}{1.341977in}}%
\pgfpathlineto{\pgfqpoint{2.861443in}{1.345729in}}%
\pgfpathlineto{\pgfqpoint{2.861443in}{1.349481in}}%
\pgfpathlineto{\pgfqpoint{2.863638in}{1.352107in}}%
\pgfpathlineto{\pgfqpoint{2.864578in}{1.353232in}}%
\pgfpathlineto{\pgfqpoint{2.864578in}{1.356984in}}%
\pgfpathlineto{\pgfqpoint{2.864578in}{1.360735in}}%
\pgfpathlineto{\pgfqpoint{2.864578in}{1.364487in}}%
\pgfpathlineto{\pgfqpoint{2.866772in}{1.367113in}}%
\pgfpathlineto{\pgfqpoint{2.867713in}{1.368238in}}%
\pgfpathlineto{\pgfqpoint{2.867713in}{1.371990in}}%
\pgfpathlineto{\pgfqpoint{2.867713in}{1.375741in}}%
\pgfpathlineto{\pgfqpoint{2.869907in}{1.378367in}}%
\pgfpathlineto{\pgfqpoint{2.870848in}{1.379493in}}%
\pgfpathlineto{\pgfqpoint{2.870848in}{1.383244in}}%
\pgfpathlineto{\pgfqpoint{2.870848in}{1.386996in}}%
\pgfpathlineto{\pgfqpoint{2.873042in}{1.389622in}}%
\pgfpathlineto{\pgfqpoint{2.873982in}{1.390747in}}%
\pgfpathlineto{\pgfqpoint{2.873982in}{1.394499in}}%
\pgfpathlineto{\pgfqpoint{2.873982in}{1.398251in}}%
\pgfpathlineto{\pgfqpoint{2.876177in}{1.400877in}}%
\pgfpathlineto{\pgfqpoint{2.877117in}{1.402002in}}%
\pgfpathlineto{\pgfqpoint{2.877117in}{1.405754in}}%
\pgfpathlineto{\pgfqpoint{2.877117in}{1.409505in}}%
\pgfpathlineto{\pgfqpoint{2.879311in}{1.412131in}}%
\pgfpathlineto{\pgfqpoint{2.880252in}{1.413257in}}%
\pgfpathlineto{\pgfqpoint{2.880252in}{1.417008in}}%
\pgfpathlineto{\pgfqpoint{2.880252in}{1.420760in}}%
\pgfpathlineto{\pgfqpoint{2.882446in}{1.423386in}}%
\pgfpathlineto{\pgfqpoint{2.883387in}{1.424511in}}%
\pgfpathlineto{\pgfqpoint{2.883387in}{1.428263in}}%
\pgfpathlineto{\pgfqpoint{2.883387in}{1.432014in}}%
\pgfpathlineto{\pgfqpoint{2.885581in}{1.434640in}}%
\pgfpathlineto{\pgfqpoint{2.886521in}{1.435766in}}%
\pgfpathlineto{\pgfqpoint{2.886521in}{1.439517in}}%
\pgfpathlineto{\pgfqpoint{2.886521in}{1.443269in}}%
\pgfpathlineto{\pgfqpoint{2.888716in}{1.445895in}}%
\pgfpathlineto{\pgfqpoint{2.889656in}{1.447020in}}%
\pgfpathlineto{\pgfqpoint{2.889656in}{1.450772in}}%
\pgfpathlineto{\pgfqpoint{2.889656in}{1.454524in}}%
\pgfpathlineto{\pgfqpoint{2.889656in}{1.458275in}}%
\pgfpathlineto{\pgfqpoint{2.891850in}{1.460901in}}%
\pgfpathlineto{\pgfqpoint{2.892791in}{1.462027in}}%
\pgfpathlineto{\pgfqpoint{2.892791in}{1.465778in}}%
\pgfpathlineto{\pgfqpoint{2.892791in}{1.469530in}}%
\pgfpathlineto{\pgfqpoint{2.894985in}{1.472156in}}%
\pgfpathlineto{\pgfqpoint{2.895925in}{1.473281in}}%
\pgfpathlineto{\pgfqpoint{2.895925in}{1.477033in}}%
\pgfpathlineto{\pgfqpoint{2.895925in}{1.480784in}}%
\pgfpathlineto{\pgfqpoint{2.898120in}{1.483410in}}%
\pgfpathlineto{\pgfqpoint{2.899060in}{1.484536in}}%
\pgfpathlineto{\pgfqpoint{2.899060in}{1.488287in}}%
\pgfpathlineto{\pgfqpoint{2.899060in}{1.492039in}}%
\pgfpathlineto{\pgfqpoint{2.901255in}{1.494665in}}%
\pgfpathlineto{\pgfqpoint{2.902195in}{1.495790in}}%
\pgfpathlineto{\pgfqpoint{2.902195in}{1.499542in}}%
\pgfpathlineto{\pgfqpoint{2.902195in}{1.503293in}}%
\pgfpathlineto{\pgfqpoint{2.904389in}{1.505920in}}%
\pgfpathlineto{\pgfqpoint{2.905330in}{1.507045in}}%
\pgfpathlineto{\pgfqpoint{2.905330in}{1.510797in}}%
\pgfpathlineto{\pgfqpoint{2.905330in}{1.514548in}}%
\pgfpathlineto{\pgfqpoint{2.907524in}{1.517174in}}%
\pgfpathlineto{\pgfqpoint{2.908464in}{1.518300in}}%
\pgfpathlineto{\pgfqpoint{2.908464in}{1.522051in}}%
\pgfpathlineto{\pgfqpoint{2.908464in}{1.525803in}}%
\pgfpathlineto{\pgfqpoint{2.910659in}{1.528429in}}%
\pgfpathlineto{\pgfqpoint{2.911599in}{1.529554in}}%
\pgfpathlineto{\pgfqpoint{2.911599in}{1.533306in}}%
\pgfpathlineto{\pgfqpoint{2.911599in}{1.537057in}}%
\pgfpathlineto{\pgfqpoint{2.913794in}{1.539683in}}%
\pgfpathlineto{\pgfqpoint{2.914734in}{1.540809in}}%
\pgfpathlineto{\pgfqpoint{2.914734in}{1.544560in}}%
\pgfpathlineto{\pgfqpoint{2.914734in}{1.548312in}}%
\pgfpathlineto{\pgfqpoint{2.916928in}{1.550938in}}%
\pgfpathlineto{\pgfqpoint{2.917869in}{1.552063in}}%
\pgfpathlineto{\pgfqpoint{2.917869in}{1.555815in}}%
\pgfpathlineto{\pgfqpoint{2.917869in}{1.559566in}}%
\pgfpathlineto{\pgfqpoint{2.917869in}{1.563318in}}%
\pgfpathlineto{\pgfqpoint{2.920063in}{1.565944in}}%
\pgfpathlineto{\pgfqpoint{2.921003in}{1.567070in}}%
\pgfpathlineto{\pgfqpoint{2.921003in}{1.570821in}}%
\pgfpathlineto{\pgfqpoint{2.921003in}{1.574573in}}%
\pgfpathlineto{\pgfqpoint{2.923198in}{1.577199in}}%
\pgfpathlineto{\pgfqpoint{2.924138in}{1.578324in}}%
\pgfpathlineto{\pgfqpoint{2.924138in}{1.582076in}}%
\pgfpathlineto{\pgfqpoint{2.924138in}{1.585827in}}%
\pgfpathlineto{\pgfqpoint{2.926332in}{1.588453in}}%
\pgfpathlineto{\pgfqpoint{2.927273in}{1.589579in}}%
\pgfpathlineto{\pgfqpoint{2.927273in}{1.593330in}}%
\pgfpathlineto{\pgfqpoint{2.927273in}{1.597082in}}%
\pgfpathlineto{\pgfqpoint{2.929467in}{1.599708in}}%
\pgfpathlineto{\pgfqpoint{2.930408in}{1.600833in}}%
\pgfpathlineto{\pgfqpoint{2.930408in}{1.604585in}}%
\pgfpathlineto{\pgfqpoint{2.930408in}{1.608336in}}%
\pgfpathlineto{\pgfqpoint{2.932602in}{1.610963in}}%
\pgfpathlineto{\pgfqpoint{2.933542in}{1.612088in}}%
\pgfpathlineto{\pgfqpoint{2.933542in}{1.615840in}}%
\pgfpathlineto{\pgfqpoint{2.933542in}{1.619591in}}%
\pgfpathlineto{\pgfqpoint{2.935737in}{1.622217in}}%
\pgfpathlineto{\pgfqpoint{2.936677in}{1.623343in}}%
\pgfpathlineto{\pgfqpoint{2.936677in}{1.627094in}}%
\pgfpathlineto{\pgfqpoint{2.936677in}{1.630846in}}%
\pgfpathlineto{\pgfqpoint{2.938871in}{1.633472in}}%
\pgfpathlineto{\pgfqpoint{2.939812in}{1.634597in}}%
\pgfpathlineto{\pgfqpoint{2.939812in}{1.638349in}}%
\pgfpathlineto{\pgfqpoint{2.939812in}{1.642100in}}%
\pgfpathlineto{\pgfqpoint{2.942006in}{1.644726in}}%
\pgfpathlineto{\pgfqpoint{2.942947in}{1.645852in}}%
\pgfpathlineto{\pgfqpoint{2.942947in}{1.649603in}}%
\pgfpathlineto{\pgfqpoint{2.942947in}{1.653355in}}%
\pgfpathlineto{\pgfqpoint{2.942947in}{1.657106in}}%
\pgfpathlineto{\pgfqpoint{2.945141in}{1.659732in}}%
\pgfpathlineto{\pgfqpoint{2.946081in}{1.660858in}}%
\pgfpathlineto{\pgfqpoint{2.946081in}{1.664609in}}%
\pgfpathlineto{\pgfqpoint{2.946081in}{1.668361in}}%
\pgfpathlineto{\pgfqpoint{2.948276in}{1.670987in}}%
\pgfpathlineto{\pgfqpoint{2.949216in}{1.672113in}}%
\pgfpathlineto{\pgfqpoint{2.949216in}{1.675864in}}%
\pgfpathlineto{\pgfqpoint{2.949216in}{1.679616in}}%
\pgfpathlineto{\pgfqpoint{2.951410in}{1.682242in}}%
\pgfpathlineto{\pgfqpoint{2.952351in}{1.683367in}}%
\pgfpathlineto{\pgfqpoint{2.952351in}{1.687119in}}%
\pgfpathlineto{\pgfqpoint{2.952351in}{1.690870in}}%
\pgfpathlineto{\pgfqpoint{2.954545in}{1.693496in}}%
\pgfpathlineto{\pgfqpoint{2.955486in}{1.694622in}}%
\pgfpathlineto{\pgfqpoint{2.955486in}{1.698373in}}%
\pgfpathlineto{\pgfqpoint{2.955486in}{1.702125in}}%
\pgfpathlineto{\pgfqpoint{2.957680in}{1.704751in}}%
\pgfpathlineto{\pgfqpoint{2.958620in}{1.705876in}}%
\pgfpathlineto{\pgfqpoint{2.958620in}{1.709628in}}%
\pgfpathlineto{\pgfqpoint{2.958620in}{1.713379in}}%
\pgfpathlineto{\pgfqpoint{2.960815in}{1.716005in}}%
\pgfpathlineto{\pgfqpoint{2.961755in}{1.717131in}}%
\pgfpathlineto{\pgfqpoint{2.961755in}{1.720882in}}%
\pgfpathlineto{\pgfqpoint{2.961755in}{1.724634in}}%
\pgfpathlineto{\pgfqpoint{2.963949in}{1.727260in}}%
\pgfpathlineto{\pgfqpoint{2.964890in}{1.728386in}}%
\pgfpathlineto{\pgfqpoint{2.964890in}{1.732137in}}%
\pgfpathlineto{\pgfqpoint{2.964890in}{1.735889in}}%
\pgfpathlineto{\pgfqpoint{2.967084in}{1.738515in}}%
\pgfpathlineto{\pgfqpoint{2.968025in}{1.739640in}}%
\pgfpathlineto{\pgfqpoint{2.968025in}{1.743392in}}%
\pgfpathlineto{\pgfqpoint{2.968025in}{1.747143in}}%
\pgfpathlineto{\pgfqpoint{2.970219in}{1.749769in}}%
\pgfpathlineto{\pgfqpoint{2.971159in}{1.750895in}}%
\pgfpathlineto{\pgfqpoint{2.971159in}{1.754646in}}%
\pgfpathlineto{\pgfqpoint{2.971159in}{1.758398in}}%
\pgfpathlineto{\pgfqpoint{2.971159in}{1.762149in}}%
\pgfpathlineto{\pgfqpoint{2.973354in}{1.764775in}}%
\pgfpathlineto{\pgfqpoint{2.974294in}{1.765901in}}%
\pgfpathlineto{\pgfqpoint{2.974294in}{1.769652in}}%
\pgfpathlineto{\pgfqpoint{2.974294in}{1.773404in}}%
\pgfpathlineto{\pgfqpoint{2.976488in}{1.776030in}}%
\pgfpathlineto{\pgfqpoint{2.977429in}{1.777155in}}%
\pgfpathlineto{\pgfqpoint{2.977429in}{1.780907in}}%
\pgfpathlineto{\pgfqpoint{2.977429in}{1.784659in}}%
\pgfpathlineto{\pgfqpoint{2.979623in}{1.787285in}}%
\pgfpathlineto{\pgfqpoint{2.980564in}{1.788410in}}%
\pgfpathlineto{\pgfqpoint{2.980564in}{1.792162in}}%
\pgfpathlineto{\pgfqpoint{2.980564in}{1.795913in}}%
\pgfpathlineto{\pgfqpoint{2.982758in}{1.798539in}}%
\pgfpathlineto{\pgfqpoint{2.983698in}{1.799665in}}%
\pgfpathlineto{\pgfqpoint{2.983698in}{1.803416in}}%
\pgfpathlineto{\pgfqpoint{2.983698in}{1.807168in}}%
\pgfpathlineto{\pgfqpoint{2.985893in}{1.809794in}}%
\pgfpathlineto{\pgfqpoint{2.986833in}{1.810919in}}%
\pgfpathlineto{\pgfqpoint{2.986833in}{1.814671in}}%
\pgfpathlineto{\pgfqpoint{2.986833in}{1.818422in}}%
\pgfpathlineto{\pgfqpoint{2.989027in}{1.821048in}}%
\pgfpathlineto{\pgfqpoint{2.989968in}{1.822174in}}%
\pgfpathlineto{\pgfqpoint{2.989968in}{1.825925in}}%
\pgfpathlineto{\pgfqpoint{2.989968in}{1.829677in}}%
\pgfpathlineto{\pgfqpoint{2.992162in}{1.832303in}}%
\pgfpathlineto{\pgfqpoint{2.993103in}{1.833429in}}%
\pgfpathlineto{\pgfqpoint{2.993103in}{1.837180in}}%
\pgfpathlineto{\pgfqpoint{2.993103in}{1.840932in}}%
\pgfpathlineto{\pgfqpoint{2.995297in}{1.843558in}}%
\pgfpathlineto{\pgfqpoint{2.996237in}{1.844683in}}%
\pgfpathlineto{\pgfqpoint{2.996237in}{1.848435in}}%
\pgfpathlineto{\pgfqpoint{2.996237in}{1.852186in}}%
\pgfpathlineto{\pgfqpoint{2.996237in}{1.855938in}}%
\pgfpathlineto{\pgfqpoint{2.998432in}{1.858564in}}%
\pgfpathlineto{\pgfqpoint{2.999372in}{1.859689in}}%
\pgfpathlineto{\pgfqpoint{2.999372in}{1.863441in}}%
\pgfpathlineto{\pgfqpoint{2.999372in}{1.867192in}}%
\pgfpathlineto{\pgfqpoint{3.001566in}{1.869818in}}%
\pgfpathlineto{\pgfqpoint{3.002507in}{1.870944in}}%
\pgfpathlineto{\pgfqpoint{3.002507in}{1.874695in}}%
\pgfpathlineto{\pgfqpoint{3.002507in}{1.878447in}}%
\pgfpathlineto{\pgfqpoint{3.004701in}{1.881073in}}%
\pgfpathlineto{\pgfqpoint{3.005641in}{1.882198in}}%
\pgfpathlineto{\pgfqpoint{3.005641in}{1.885950in}}%
\pgfpathlineto{\pgfqpoint{3.005641in}{1.889702in}}%
\pgfpathlineto{\pgfqpoint{3.007836in}{1.892328in}}%
\pgfpathlineto{\pgfqpoint{3.008776in}{1.893453in}}%
\pgfpathlineto{\pgfqpoint{3.008776in}{1.897205in}}%
\pgfpathlineto{\pgfqpoint{3.008776in}{1.900956in}}%
\pgfpathlineto{\pgfqpoint{3.010971in}{1.903582in}}%
\pgfpathlineto{\pgfqpoint{3.011911in}{1.904708in}}%
\pgfpathlineto{\pgfqpoint{3.011911in}{1.908459in}}%
\pgfpathlineto{\pgfqpoint{3.011911in}{1.912211in}}%
\pgfpathlineto{\pgfqpoint{3.014105in}{1.914837in}}%
\pgfpathlineto{\pgfqpoint{3.015046in}{1.915962in}}%
\pgfpathlineto{\pgfqpoint{3.015046in}{1.919714in}}%
\pgfpathlineto{\pgfqpoint{3.015046in}{1.923465in}}%
\pgfpathlineto{\pgfqpoint{3.017240in}{1.926091in}}%
\pgfpathlineto{\pgfqpoint{3.018180in}{1.927217in}}%
\pgfpathlineto{\pgfqpoint{3.018180in}{1.930968in}}%
\pgfpathlineto{\pgfqpoint{3.018180in}{1.934720in}}%
\pgfpathlineto{\pgfqpoint{3.020375in}{1.937346in}}%
\pgfpathlineto{\pgfqpoint{3.021315in}{1.938471in}}%
\pgfpathlineto{\pgfqpoint{3.021315in}{1.942223in}}%
\pgfpathlineto{\pgfqpoint{3.021315in}{1.945975in}}%
\pgfpathlineto{\pgfqpoint{3.023510in}{1.948601in}}%
\pgfpathlineto{\pgfqpoint{3.024450in}{1.949726in}}%
\pgfpathlineto{\pgfqpoint{3.024450in}{1.953478in}}%
\pgfpathlineto{\pgfqpoint{3.024450in}{1.957229in}}%
\pgfpathlineto{\pgfqpoint{3.024450in}{1.960981in}}%
\pgfpathlineto{\pgfqpoint{3.026644in}{1.963607in}}%
\pgfpathlineto{\pgfqpoint{3.027585in}{1.964732in}}%
\pgfpathlineto{\pgfqpoint{3.027585in}{1.968484in}}%
\pgfpathlineto{\pgfqpoint{3.027585in}{1.972235in}}%
\pgfpathlineto{\pgfqpoint{3.029779in}{1.974861in}}%
\pgfpathlineto{\pgfqpoint{3.030719in}{1.975987in}}%
\pgfpathlineto{\pgfqpoint{3.030719in}{1.979738in}}%
\pgfpathlineto{\pgfqpoint{3.030719in}{1.983490in}}%
\pgfpathlineto{\pgfqpoint{3.032914in}{1.986116in}}%
\pgfpathlineto{\pgfqpoint{3.033854in}{1.987241in}}%
\pgfpathlineto{\pgfqpoint{3.033854in}{1.990993in}}%
\pgfpathlineto{\pgfqpoint{3.033854in}{1.994745in}}%
\pgfpathlineto{\pgfqpoint{3.036048in}{1.997371in}}%
\pgfpathlineto{\pgfqpoint{3.036989in}{1.998496in}}%
\pgfpathlineto{\pgfqpoint{3.036989in}{2.002248in}}%
\pgfpathlineto{\pgfqpoint{3.036989in}{2.005999in}}%
\pgfpathlineto{\pgfqpoint{3.039183in}{2.008625in}}%
\pgfpathlineto{\pgfqpoint{3.040124in}{2.009751in}}%
\pgfpathlineto{\pgfqpoint{3.040124in}{2.013502in}}%
\pgfpathlineto{\pgfqpoint{3.040124in}{2.017254in}}%
\pgfpathlineto{\pgfqpoint{3.042318in}{2.019880in}}%
\pgfpathlineto{\pgfqpoint{3.043258in}{2.021005in}}%
\pgfpathlineto{\pgfqpoint{3.043258in}{2.024757in}}%
\pgfpathlineto{\pgfqpoint{3.043258in}{2.028508in}}%
\pgfpathlineto{\pgfqpoint{3.045453in}{2.031134in}}%
\pgfpathlineto{\pgfqpoint{3.046393in}{2.032260in}}%
\pgfpathlineto{\pgfqpoint{3.046393in}{2.036011in}}%
\pgfpathlineto{\pgfqpoint{3.046393in}{2.039763in}}%
\pgfpathlineto{\pgfqpoint{3.048587in}{2.042389in}}%
\pgfpathlineto{\pgfqpoint{3.049528in}{2.043514in}}%
\pgfpathlineto{\pgfqpoint{3.049528in}{2.047266in}}%
\pgfpathlineto{\pgfqpoint{3.049528in}{2.051018in}}%
\pgfpathlineto{\pgfqpoint{3.049528in}{2.054769in}}%
\pgfpathlineto{\pgfqpoint{3.051722in}{2.057395in}}%
\pgfpathlineto{\pgfqpoint{3.052663in}{2.058521in}}%
\pgfpathlineto{\pgfqpoint{3.052663in}{2.062272in}}%
\pgfpathlineto{\pgfqpoint{3.052663in}{2.066024in}}%
\pgfpathlineto{\pgfqpoint{3.054857in}{2.068650in}}%
\pgfpathlineto{\pgfqpoint{3.055797in}{2.069775in}}%
\pgfpathlineto{\pgfqpoint{3.055797in}{2.073527in}}%
\pgfpathlineto{\pgfqpoint{3.055797in}{2.077278in}}%
\pgfpathlineto{\pgfqpoint{3.057992in}{2.079904in}}%
\pgfpathlineto{\pgfqpoint{3.058932in}{2.081030in}}%
\pgfpathlineto{\pgfqpoint{3.058932in}{2.084781in}}%
\pgfpathlineto{\pgfqpoint{3.058932in}{2.088533in}}%
\pgfpathlineto{\pgfqpoint{3.061126in}{2.091159in}}%
\pgfpathlineto{\pgfqpoint{3.062067in}{2.092284in}}%
\pgfpathlineto{\pgfqpoint{3.062067in}{2.096036in}}%
\pgfpathlineto{\pgfqpoint{3.062067in}{2.099787in}}%
\pgfpathlineto{\pgfqpoint{3.064261in}{2.102414in}}%
\pgfpathlineto{\pgfqpoint{3.065202in}{2.103539in}}%
\pgfpathlineto{\pgfqpoint{3.065202in}{2.107291in}}%
\pgfpathlineto{\pgfqpoint{3.065202in}{2.111042in}}%
\pgfpathlineto{\pgfqpoint{3.067396in}{2.113668in}}%
\pgfpathlineto{\pgfqpoint{3.068336in}{2.114794in}}%
\pgfpathlineto{\pgfqpoint{3.068336in}{2.118545in}}%
\pgfpathlineto{\pgfqpoint{3.068336in}{2.122297in}}%
\pgfpathlineto{\pgfqpoint{3.070531in}{2.124923in}}%
\pgfpathlineto{\pgfqpoint{3.071471in}{2.126048in}}%
\pgfpathlineto{\pgfqpoint{3.071471in}{2.129800in}}%
\pgfpathlineto{\pgfqpoint{3.071471in}{2.133551in}}%
\pgfpathlineto{\pgfqpoint{3.073665in}{2.136177in}}%
\pgfpathlineto{\pgfqpoint{3.074606in}{2.137303in}}%
\pgfpathlineto{\pgfqpoint{3.074606in}{2.141054in}}%
\pgfpathlineto{\pgfqpoint{3.074606in}{2.144806in}}%
\pgfpathlineto{\pgfqpoint{3.076800in}{2.147432in}}%
\pgfpathlineto{\pgfqpoint{3.077741in}{2.148557in}}%
\pgfpathlineto{\pgfqpoint{3.077741in}{2.152309in}}%
\pgfpathlineto{\pgfqpoint{3.077741in}{2.156060in}}%
\pgfpathlineto{\pgfqpoint{3.077741in}{2.159812in}}%
\pgfpathlineto{\pgfqpoint{3.079935in}{2.162438in}}%
\pgfpathlineto{\pgfqpoint{3.080875in}{2.163564in}}%
\pgfpathlineto{\pgfqpoint{3.080875in}{2.167315in}}%
\pgfpathlineto{\pgfqpoint{3.080875in}{2.171067in}}%
\pgfpathlineto{\pgfqpoint{3.083070in}{2.173693in}}%
\pgfpathlineto{\pgfqpoint{3.084010in}{2.174818in}}%
\pgfpathlineto{\pgfqpoint{3.084010in}{2.178570in}}%
\pgfpathlineto{\pgfqpoint{3.084010in}{2.182321in}}%
\pgfpathlineto{\pgfqpoint{3.086204in}{2.184947in}}%
\pgfpathlineto{\pgfqpoint{3.087145in}{2.186073in}}%
\pgfpathlineto{\pgfqpoint{3.087145in}{2.189824in}}%
\pgfpathlineto{\pgfqpoint{3.087145in}{2.193576in}}%
\pgfpathlineto{\pgfqpoint{3.089339in}{2.196202in}}%
\pgfpathlineto{\pgfqpoint{3.090280in}{2.197327in}}%
\pgfpathlineto{\pgfqpoint{3.090280in}{2.201079in}}%
\pgfpathlineto{\pgfqpoint{3.090280in}{2.204830in}}%
\pgfpathlineto{\pgfqpoint{3.092474in}{2.207457in}}%
\pgfpathlineto{\pgfqpoint{3.093414in}{2.208582in}}%
\pgfpathlineto{\pgfqpoint{3.093414in}{2.212334in}}%
\pgfpathlineto{\pgfqpoint{3.092474in}{2.213459in}}%
\pgfpathlineto{\pgfqpoint{3.090280in}{2.216085in}}%
\pgfpathlineto{\pgfqpoint{3.089339in}{2.217210in}}%
\pgfpathlineto{\pgfqpoint{3.087145in}{2.219837in}}%
\pgfpathlineto{\pgfqpoint{3.086204in}{2.220962in}}%
\pgfpathlineto{\pgfqpoint{3.084010in}{2.223588in}}%
\pgfpathlineto{\pgfqpoint{3.083070in}{2.224714in}}%
\pgfpathlineto{\pgfqpoint{3.080875in}{2.227340in}}%
\pgfpathlineto{\pgfqpoint{3.079935in}{2.228465in}}%
\pgfpathlineto{\pgfqpoint{3.077741in}{2.231091in}}%
\pgfpathlineto{\pgfqpoint{3.076800in}{2.232217in}}%
\pgfpathlineto{\pgfqpoint{3.074606in}{2.234843in}}%
\pgfpathlineto{\pgfqpoint{3.073665in}{2.235968in}}%
\pgfpathlineto{\pgfqpoint{3.071471in}{2.238594in}}%
\pgfpathlineto{\pgfqpoint{3.070531in}{2.239720in}}%
\pgfpathlineto{\pgfqpoint{3.068336in}{2.242346in}}%
\pgfpathlineto{\pgfqpoint{3.067396in}{2.243471in}}%
\pgfpathlineto{\pgfqpoint{3.065202in}{2.246097in}}%
\pgfpathlineto{\pgfqpoint{3.064261in}{2.247223in}}%
\pgfpathlineto{\pgfqpoint{3.062067in}{2.249849in}}%
\pgfpathlineto{\pgfqpoint{3.061126in}{2.250974in}}%
\pgfpathlineto{\pgfqpoint{3.058932in}{2.253600in}}%
\pgfpathlineto{\pgfqpoint{3.057992in}{2.254726in}}%
\pgfpathlineto{\pgfqpoint{3.055797in}{2.257352in}}%
\pgfpathlineto{\pgfqpoint{3.054857in}{2.258477in}}%
\pgfpathlineto{\pgfqpoint{3.052663in}{2.261103in}}%
\pgfpathlineto{\pgfqpoint{3.052663in}{2.264855in}}%
\pgfpathlineto{\pgfqpoint{3.051722in}{2.265980in}}%
\pgfpathlineto{\pgfqpoint{3.049528in}{2.268607in}}%
\pgfpathlineto{\pgfqpoint{3.048587in}{2.269732in}}%
\pgfpathlineto{\pgfqpoint{3.046393in}{2.272358in}}%
\pgfpathlineto{\pgfqpoint{3.045453in}{2.273484in}}%
\pgfpathlineto{\pgfqpoint{3.043258in}{2.276110in}}%
\pgfpathlineto{\pgfqpoint{3.042318in}{2.277235in}}%
\pgfpathlineto{\pgfqpoint{3.040124in}{2.279861in}}%
\pgfpathlineto{\pgfqpoint{3.039183in}{2.280987in}}%
\pgfpathlineto{\pgfqpoint{3.036989in}{2.283613in}}%
\pgfpathlineto{\pgfqpoint{3.036048in}{2.284738in}}%
\pgfpathlineto{\pgfqpoint{3.033854in}{2.287364in}}%
\pgfpathlineto{\pgfqpoint{3.032914in}{2.288490in}}%
\pgfpathlineto{\pgfqpoint{3.030719in}{2.291116in}}%
\pgfpathlineto{\pgfqpoint{3.029779in}{2.292241in}}%
\pgfpathlineto{\pgfqpoint{3.027585in}{2.294867in}}%
\pgfpathlineto{\pgfqpoint{3.026644in}{2.295993in}}%
\pgfpathlineto{\pgfqpoint{3.024450in}{2.298619in}}%
\pgfpathlineto{\pgfqpoint{3.023510in}{2.299744in}}%
\pgfpathlineto{\pgfqpoint{3.021315in}{2.302370in}}%
\pgfpathlineto{\pgfqpoint{3.020375in}{2.303496in}}%
\pgfpathlineto{\pgfqpoint{3.018180in}{2.306122in}}%
\pgfpathlineto{\pgfqpoint{3.017240in}{2.307247in}}%
\pgfpathlineto{\pgfqpoint{3.015046in}{2.309873in}}%
\pgfpathlineto{\pgfqpoint{3.014105in}{2.310999in}}%
\pgfpathlineto{\pgfqpoint{3.011911in}{2.313625in}}%
\pgfpathlineto{\pgfqpoint{3.010971in}{2.314750in}}%
\pgfpathlineto{\pgfqpoint{3.008776in}{2.317376in}}%
\pgfpathlineto{\pgfqpoint{3.007836in}{2.318502in}}%
\pgfpathlineto{\pgfqpoint{3.005641in}{2.321128in}}%
\pgfpathlineto{\pgfqpoint{3.004701in}{2.322253in}}%
\pgfpathlineto{\pgfqpoint{3.002507in}{2.324880in}}%
\pgfpathlineto{\pgfqpoint{3.002507in}{2.328631in}}%
\pgfpathlineto{\pgfqpoint{3.001566in}{2.329757in}}%
\pgfpathlineto{\pgfqpoint{2.999372in}{2.332383in}}%
\pgfpathlineto{\pgfqpoint{2.998432in}{2.333508in}}%
\pgfpathlineto{\pgfqpoint{2.996237in}{2.336134in}}%
\pgfpathlineto{\pgfqpoint{2.995297in}{2.337260in}}%
\pgfpathlineto{\pgfqpoint{2.993103in}{2.339886in}}%
\pgfpathlineto{\pgfqpoint{2.992162in}{2.341011in}}%
\pgfpathlineto{\pgfqpoint{2.989968in}{2.343637in}}%
\pgfpathlineto{\pgfqpoint{2.989027in}{2.344763in}}%
\pgfpathlineto{\pgfqpoint{2.986833in}{2.347389in}}%
\pgfpathlineto{\pgfqpoint{2.985893in}{2.348514in}}%
\pgfpathlineto{\pgfqpoint{2.983698in}{2.351140in}}%
\pgfpathlineto{\pgfqpoint{2.982758in}{2.352266in}}%
\pgfpathlineto{\pgfqpoint{2.980564in}{2.354892in}}%
\pgfpathlineto{\pgfqpoint{2.979623in}{2.356017in}}%
\pgfpathlineto{\pgfqpoint{2.977429in}{2.358643in}}%
\pgfpathlineto{\pgfqpoint{2.976488in}{2.359769in}}%
\pgfpathlineto{\pgfqpoint{2.974294in}{2.362395in}}%
\pgfpathlineto{\pgfqpoint{2.973354in}{2.363520in}}%
\pgfpathlineto{\pgfqpoint{2.971159in}{2.366146in}}%
\pgfpathlineto{\pgfqpoint{2.970219in}{2.367272in}}%
\pgfpathlineto{\pgfqpoint{2.968025in}{2.369898in}}%
\pgfpathlineto{\pgfqpoint{2.967084in}{2.371023in}}%
\pgfpathlineto{\pgfqpoint{2.964890in}{2.373649in}}%
\pgfpathlineto{\pgfqpoint{2.963949in}{2.374775in}}%
\pgfpathlineto{\pgfqpoint{2.961755in}{2.377401in}}%
\pgfpathlineto{\pgfqpoint{2.960815in}{2.378526in}}%
\pgfpathlineto{\pgfqpoint{2.958620in}{2.381153in}}%
\pgfpathlineto{\pgfqpoint{2.957680in}{2.382278in}}%
\pgfpathlineto{\pgfqpoint{2.955486in}{2.384904in}}%
\pgfpathlineto{\pgfqpoint{2.955486in}{2.388656in}}%
\pgfpathlineto{\pgfqpoint{2.954545in}{2.389781in}}%
\pgfpathlineto{\pgfqpoint{2.952351in}{2.392407in}}%
\pgfpathlineto{\pgfqpoint{2.951410in}{2.393533in}}%
\pgfpathlineto{\pgfqpoint{2.949216in}{2.396159in}}%
\pgfpathlineto{\pgfqpoint{2.948276in}{2.397284in}}%
\pgfpathlineto{\pgfqpoint{2.946081in}{2.399910in}}%
\pgfpathlineto{\pgfqpoint{2.945141in}{2.401036in}}%
\pgfpathlineto{\pgfqpoint{2.942947in}{2.403662in}}%
\pgfpathlineto{\pgfqpoint{2.942006in}{2.404787in}}%
\pgfpathlineto{\pgfqpoint{2.939812in}{2.407413in}}%
\pgfpathlineto{\pgfqpoint{2.938871in}{2.408539in}}%
\pgfpathlineto{\pgfqpoint{2.936677in}{2.411165in}}%
\pgfpathlineto{\pgfqpoint{2.935737in}{2.412290in}}%
\pgfpathlineto{\pgfqpoint{2.933542in}{2.414916in}}%
\pgfpathlineto{\pgfqpoint{2.932602in}{2.416042in}}%
\pgfpathlineto{\pgfqpoint{2.930408in}{2.418668in}}%
\pgfpathlineto{\pgfqpoint{2.929467in}{2.419793in}}%
\pgfpathlineto{\pgfqpoint{2.927273in}{2.422419in}}%
\pgfpathlineto{\pgfqpoint{2.926332in}{2.423545in}}%
\pgfpathlineto{\pgfqpoint{2.924138in}{2.426171in}}%
\pgfpathlineto{\pgfqpoint{2.923198in}{2.427296in}}%
\pgfpathlineto{\pgfqpoint{2.921003in}{2.429923in}}%
\pgfpathlineto{\pgfqpoint{2.920063in}{2.431048in}}%
\pgfpathlineto{\pgfqpoint{2.917869in}{2.433674in}}%
\pgfpathlineto{\pgfqpoint{2.916928in}{2.434800in}}%
\pgfpathlineto{\pgfqpoint{2.914734in}{2.437426in}}%
\pgfpathlineto{\pgfqpoint{2.913794in}{2.438551in}}%
\pgfpathlineto{\pgfqpoint{2.911599in}{2.441177in}}%
\pgfpathlineto{\pgfqpoint{2.910659in}{2.442303in}}%
\pgfpathlineto{\pgfqpoint{2.908464in}{2.444929in}}%
\pgfpathlineto{\pgfqpoint{2.907524in}{2.446054in}}%
\pgfpathlineto{\pgfqpoint{2.905330in}{2.448680in}}%
\pgfpathlineto{\pgfqpoint{2.905330in}{2.452432in}}%
\pgfpathlineto{\pgfqpoint{2.904389in}{2.453557in}}%
\pgfpathlineto{\pgfqpoint{2.902195in}{2.456183in}}%
\pgfpathlineto{\pgfqpoint{2.901255in}{2.457309in}}%
\pgfpathlineto{\pgfqpoint{2.899060in}{2.459935in}}%
\pgfpathlineto{\pgfqpoint{2.898120in}{2.461060in}}%
\pgfpathlineto{\pgfqpoint{2.895925in}{2.463686in}}%
\pgfpathlineto{\pgfqpoint{2.894985in}{2.464812in}}%
\pgfpathlineto{\pgfqpoint{2.892791in}{2.467438in}}%
\pgfpathlineto{\pgfqpoint{2.891850in}{2.468563in}}%
\pgfpathlineto{\pgfqpoint{2.889656in}{2.471189in}}%
\pgfpathlineto{\pgfqpoint{2.888716in}{2.472315in}}%
\pgfpathlineto{\pgfqpoint{2.886521in}{2.474941in}}%
\pgfpathlineto{\pgfqpoint{2.885581in}{2.476066in}}%
\pgfpathlineto{\pgfqpoint{2.883387in}{2.478692in}}%
\pgfpathlineto{\pgfqpoint{2.882446in}{2.479818in}}%
\pgfpathlineto{\pgfqpoint{2.880252in}{2.482444in}}%
\pgfpathlineto{\pgfqpoint{2.879311in}{2.483569in}}%
\pgfpathlineto{\pgfqpoint{2.877117in}{2.486196in}}%
\pgfpathlineto{\pgfqpoint{2.876177in}{2.487321in}}%
\pgfpathlineto{\pgfqpoint{2.873982in}{2.489947in}}%
\pgfpathlineto{\pgfqpoint{2.873042in}{2.491073in}}%
\pgfpathlineto{\pgfqpoint{2.870848in}{2.493699in}}%
\pgfpathlineto{\pgfqpoint{2.869907in}{2.494824in}}%
\pgfpathlineto{\pgfqpoint{2.867713in}{2.497450in}}%
\pgfpathlineto{\pgfqpoint{2.866772in}{2.498576in}}%
\pgfpathlineto{\pgfqpoint{2.864578in}{2.501202in}}%
\pgfpathlineto{\pgfqpoint{2.863638in}{2.502327in}}%
\pgfpathlineto{\pgfqpoint{2.861443in}{2.504953in}}%
\pgfpathlineto{\pgfqpoint{2.860503in}{2.506079in}}%
\pgfpathlineto{\pgfqpoint{2.858309in}{2.508705in}}%
\pgfpathlineto{\pgfqpoint{2.858309in}{2.512456in}}%
\pgfpathlineto{\pgfqpoint{2.857368in}{2.513582in}}%
\pgfpathlineto{\pgfqpoint{2.855174in}{2.516208in}}%
\pgfpathlineto{\pgfqpoint{2.854233in}{2.517333in}}%
\pgfpathlineto{\pgfqpoint{2.852039in}{2.519959in}}%
\pgfpathlineto{\pgfqpoint{2.851099in}{2.521085in}}%
\pgfpathlineto{\pgfqpoint{2.848904in}{2.523711in}}%
\pgfpathlineto{\pgfqpoint{2.847964in}{2.524836in}}%
\pgfpathlineto{\pgfqpoint{2.845770in}{2.527462in}}%
\pgfpathlineto{\pgfqpoint{2.844829in}{2.528588in}}%
\pgfpathlineto{\pgfqpoint{2.842635in}{2.531214in}}%
\pgfpathlineto{\pgfqpoint{2.841694in}{2.532339in}}%
\pgfpathlineto{\pgfqpoint{2.839500in}{2.534965in}}%
\pgfpathlineto{\pgfqpoint{2.838560in}{2.536091in}}%
\pgfpathlineto{\pgfqpoint{2.836365in}{2.538717in}}%
\pgfpathlineto{\pgfqpoint{2.835425in}{2.539842in}}%
\pgfpathlineto{\pgfqpoint{2.833231in}{2.542469in}}%
\pgfpathlineto{\pgfqpoint{2.832290in}{2.543594in}}%
\pgfpathlineto{\pgfqpoint{2.830096in}{2.546220in}}%
\pgfpathlineto{\pgfqpoint{2.829155in}{2.547346in}}%
\pgfpathlineto{\pgfqpoint{2.826961in}{2.549972in}}%
\pgfpathlineto{\pgfqpoint{2.826021in}{2.551097in}}%
\pgfpathlineto{\pgfqpoint{2.823826in}{2.553723in}}%
\pgfpathlineto{\pgfqpoint{2.822886in}{2.554849in}}%
\pgfpathlineto{\pgfqpoint{2.820692in}{2.557475in}}%
\pgfpathlineto{\pgfqpoint{2.819751in}{2.558600in}}%
\pgfpathlineto{\pgfqpoint{2.817557in}{2.561226in}}%
\pgfpathlineto{\pgfqpoint{2.816617in}{2.562352in}}%
\pgfpathlineto{\pgfqpoint{2.814422in}{2.564978in}}%
\pgfpathlineto{\pgfqpoint{2.813482in}{2.566103in}}%
\pgfpathlineto{\pgfqpoint{2.811287in}{2.568729in}}%
\pgfpathlineto{\pgfqpoint{2.810347in}{2.569855in}}%
\pgfpathlineto{\pgfqpoint{2.808153in}{2.572481in}}%
\pgfpathlineto{\pgfqpoint{2.808153in}{2.576232in}}%
\pgfpathlineto{\pgfqpoint{2.807212in}{2.577358in}}%
\pgfpathlineto{\pgfqpoint{2.805018in}{2.579984in}}%
\pgfpathlineto{\pgfqpoint{2.804078in}{2.581109in}}%
\pgfpathlineto{\pgfqpoint{2.801883in}{2.583735in}}%
\pgfpathlineto{\pgfqpoint{2.800943in}{2.584861in}}%
\pgfpathlineto{\pgfqpoint{2.798748in}{2.587487in}}%
\pgfpathlineto{\pgfqpoint{2.797808in}{2.588612in}}%
\pgfpathlineto{\pgfqpoint{2.795614in}{2.591238in}}%
\pgfpathlineto{\pgfqpoint{2.794673in}{2.592364in}}%
\pgfpathlineto{\pgfqpoint{2.792479in}{2.594990in}}%
\pgfpathlineto{\pgfqpoint{2.791539in}{2.596115in}}%
\pgfpathlineto{\pgfqpoint{2.789344in}{2.598742in}}%
\pgfpathlineto{\pgfqpoint{2.788404in}{2.599867in}}%
\pgfpathlineto{\pgfqpoint{2.786210in}{2.602493in}}%
\pgfpathlineto{\pgfqpoint{2.785269in}{2.603619in}}%
\pgfpathlineto{\pgfqpoint{2.783075in}{2.606245in}}%
\pgfpathlineto{\pgfqpoint{2.782134in}{2.607370in}}%
\pgfpathlineto{\pgfqpoint{2.779940in}{2.609996in}}%
\pgfpathlineto{\pgfqpoint{2.779000in}{2.611122in}}%
\pgfpathlineto{\pgfqpoint{2.776805in}{2.613748in}}%
\pgfpathlineto{\pgfqpoint{2.775865in}{2.614873in}}%
\pgfpathlineto{\pgfqpoint{2.773671in}{2.617499in}}%
\pgfpathlineto{\pgfqpoint{2.772730in}{2.618625in}}%
\pgfpathlineto{\pgfqpoint{2.770536in}{2.621251in}}%
\pgfpathlineto{\pgfqpoint{2.769595in}{2.622376in}}%
\pgfpathlineto{\pgfqpoint{2.767401in}{2.625002in}}%
\pgfpathlineto{\pgfqpoint{2.766461in}{2.626128in}}%
\pgfpathlineto{\pgfqpoint{2.764266in}{2.628754in}}%
\pgfpathlineto{\pgfqpoint{2.763326in}{2.629879in}}%
\pgfpathlineto{\pgfqpoint{2.761132in}{2.632505in}}%
\pgfpathlineto{\pgfqpoint{2.761132in}{2.636257in}}%
\pgfpathlineto{\pgfqpoint{2.760191in}{2.637382in}}%
\pgfpathlineto{\pgfqpoint{2.757997in}{2.640008in}}%
\pgfpathlineto{\pgfqpoint{2.757056in}{2.641134in}}%
\pgfpathlineto{\pgfqpoint{2.754862in}{2.643760in}}%
\pgfpathlineto{\pgfqpoint{2.753922in}{2.644885in}}%
\pgfpathlineto{\pgfqpoint{2.751727in}{2.647512in}}%
\pgfpathlineto{\pgfqpoint{2.750787in}{2.648637in}}%
\pgfpathlineto{\pgfqpoint{2.748593in}{2.651263in}}%
\pgfpathlineto{\pgfqpoint{2.747652in}{2.652389in}}%
\pgfpathlineto{\pgfqpoint{2.745458in}{2.655015in}}%
\pgfpathlineto{\pgfqpoint{2.744517in}{2.656140in}}%
\pgfpathlineto{\pgfqpoint{2.742323in}{2.658766in}}%
\pgfpathlineto{\pgfqpoint{2.741383in}{2.659892in}}%
\pgfpathlineto{\pgfqpoint{2.739188in}{2.662518in}}%
\pgfpathlineto{\pgfqpoint{2.738248in}{2.663643in}}%
\pgfpathlineto{\pgfqpoint{2.736054in}{2.666269in}}%
\pgfpathlineto{\pgfqpoint{2.735113in}{2.667395in}}%
\pgfpathlineto{\pgfqpoint{2.732919in}{2.670021in}}%
\pgfpathlineto{\pgfqpoint{2.731978in}{2.671146in}}%
\pgfpathlineto{\pgfqpoint{2.729784in}{2.673772in}}%
\pgfpathlineto{\pgfqpoint{2.728844in}{2.674898in}}%
\pgfpathlineto{\pgfqpoint{2.726649in}{2.677524in}}%
\pgfpathlineto{\pgfqpoint{2.725709in}{2.678649in}}%
\pgfpathlineto{\pgfqpoint{2.723515in}{2.681275in}}%
\pgfpathlineto{\pgfqpoint{2.722574in}{2.682401in}}%
\pgfpathlineto{\pgfqpoint{2.720380in}{2.685027in}}%
\pgfpathlineto{\pgfqpoint{2.719439in}{2.686152in}}%
\pgfpathlineto{\pgfqpoint{2.717245in}{2.688778in}}%
\pgfpathlineto{\pgfqpoint{2.716305in}{2.689904in}}%
\pgfpathlineto{\pgfqpoint{2.714110in}{2.692530in}}%
\pgfpathlineto{\pgfqpoint{2.713170in}{2.693655in}}%
\pgfpathlineto{\pgfqpoint{2.710976in}{2.696281in}}%
\pgfpathlineto{\pgfqpoint{2.710976in}{2.700033in}}%
\pgfpathlineto{\pgfqpoint{2.710035in}{2.701158in}}%
\pgfpathlineto{\pgfqpoint{2.707841in}{2.703785in}}%
\pgfpathlineto{\pgfqpoint{2.706901in}{2.704910in}}%
\pgfpathlineto{\pgfqpoint{2.704706in}{2.707536in}}%
\pgfpathlineto{\pgfqpoint{2.703766in}{2.708662in}}%
\pgfpathlineto{\pgfqpoint{2.701571in}{2.711288in}}%
\pgfpathlineto{\pgfqpoint{2.700631in}{2.712413in}}%
\pgfpathlineto{\pgfqpoint{2.698437in}{2.715039in}}%
\pgfpathlineto{\pgfqpoint{2.697496in}{2.716165in}}%
\pgfpathlineto{\pgfqpoint{2.695302in}{2.718791in}}%
\pgfpathlineto{\pgfqpoint{2.694362in}{2.719916in}}%
\pgfpathlineto{\pgfqpoint{2.692167in}{2.722542in}}%
\pgfpathlineto{\pgfqpoint{2.691227in}{2.723668in}}%
\pgfpathlineto{\pgfqpoint{2.689032in}{2.726294in}}%
\pgfpathlineto{\pgfqpoint{2.688092in}{2.727419in}}%
\pgfpathlineto{\pgfqpoint{2.685898in}{2.730045in}}%
\pgfpathlineto{\pgfqpoint{2.684957in}{2.731171in}}%
\pgfpathlineto{\pgfqpoint{2.682763in}{2.733797in}}%
\pgfpathlineto{\pgfqpoint{2.681823in}{2.734922in}}%
\pgfpathlineto{\pgfqpoint{2.679628in}{2.737548in}}%
\pgfpathlineto{\pgfqpoint{2.678688in}{2.738674in}}%
\pgfpathlineto{\pgfqpoint{2.676494in}{2.741300in}}%
\pgfpathlineto{\pgfqpoint{2.675553in}{2.742425in}}%
\pgfpathlineto{\pgfqpoint{2.673359in}{2.745051in}}%
\pgfpathlineto{\pgfqpoint{2.672418in}{2.746177in}}%
\pgfpathlineto{\pgfqpoint{2.669284in}{2.746177in}}%
\pgfpathlineto{\pgfqpoint{2.666149in}{2.746177in}}%
\pgfpathlineto{\pgfqpoint{2.663014in}{2.746177in}}%
\pgfpathlineto{\pgfqpoint{2.659879in}{2.746177in}}%
\pgfpathlineto{\pgfqpoint{2.658939in}{2.745051in}}%
\pgfpathlineto{\pgfqpoint{2.656745in}{2.742425in}}%
\pgfpathlineto{\pgfqpoint{2.653610in}{2.742425in}}%
\pgfpathlineto{\pgfqpoint{2.650475in}{2.742425in}}%
\pgfpathlineto{\pgfqpoint{2.647340in}{2.742425in}}%
\pgfpathlineto{\pgfqpoint{2.646400in}{2.741300in}}%
\pgfpathlineto{\pgfqpoint{2.644206in}{2.738674in}}%
\pgfpathlineto{\pgfqpoint{2.641071in}{2.738674in}}%
\pgfpathlineto{\pgfqpoint{2.637936in}{2.738674in}}%
\pgfpathlineto{\pgfqpoint{2.634801in}{2.738674in}}%
\pgfpathlineto{\pgfqpoint{2.633861in}{2.737548in}}%
\pgfpathlineto{\pgfqpoint{2.631667in}{2.734922in}}%
\pgfpathlineto{\pgfqpoint{2.628532in}{2.734922in}}%
\pgfpathlineto{\pgfqpoint{2.625397in}{2.734922in}}%
\pgfpathlineto{\pgfqpoint{2.622262in}{2.734922in}}%
\pgfpathlineto{\pgfqpoint{2.619128in}{2.734922in}}%
\pgfpathlineto{\pgfqpoint{2.618187in}{2.733797in}}%
\pgfpathlineto{\pgfqpoint{2.615993in}{2.731171in}}%
\pgfpathlineto{\pgfqpoint{2.612858in}{2.731171in}}%
\pgfpathlineto{\pgfqpoint{2.609724in}{2.731171in}}%
\pgfpathlineto{\pgfqpoint{2.606589in}{2.731171in}}%
\pgfpathlineto{\pgfqpoint{2.605648in}{2.730045in}}%
\pgfpathlineto{\pgfqpoint{2.603454in}{2.727419in}}%
\pgfpathlineto{\pgfqpoint{2.600319in}{2.727419in}}%
\pgfpathlineto{\pgfqpoint{2.597185in}{2.727419in}}%
\pgfpathlineto{\pgfqpoint{2.594050in}{2.727419in}}%
\pgfpathlineto{\pgfqpoint{2.593109in}{2.726294in}}%
\pgfpathlineto{\pgfqpoint{2.590915in}{2.723668in}}%
\pgfpathlineto{\pgfqpoint{2.587780in}{2.723668in}}%
\pgfpathlineto{\pgfqpoint{2.584646in}{2.723668in}}%
\pgfpathlineto{\pgfqpoint{2.581511in}{2.723668in}}%
\pgfpathlineto{\pgfqpoint{2.580570in}{2.722542in}}%
\pgfpathlineto{\pgfqpoint{2.578376in}{2.719916in}}%
\pgfpathlineto{\pgfqpoint{2.575241in}{2.719916in}}%
\pgfpathlineto{\pgfqpoint{2.572107in}{2.719916in}}%
\pgfpathlineto{\pgfqpoint{2.568972in}{2.719916in}}%
\pgfpathlineto{\pgfqpoint{2.568031in}{2.718791in}}%
\pgfpathlineto{\pgfqpoint{2.565837in}{2.716165in}}%
\pgfpathlineto{\pgfqpoint{2.562702in}{2.716165in}}%
\pgfpathlineto{\pgfqpoint{2.559568in}{2.716165in}}%
\pgfpathlineto{\pgfqpoint{2.556433in}{2.716165in}}%
\pgfpathlineto{\pgfqpoint{2.553298in}{2.716165in}}%
\pgfpathlineto{\pgfqpoint{2.552358in}{2.715039in}}%
\pgfpathlineto{\pgfqpoint{2.550163in}{2.712413in}}%
\pgfpathlineto{\pgfqpoint{2.547029in}{2.712413in}}%
\pgfpathlineto{\pgfqpoint{2.543894in}{2.712413in}}%
\pgfpathlineto{\pgfqpoint{2.540759in}{2.712413in}}%
\pgfpathlineto{\pgfqpoint{2.539819in}{2.711288in}}%
\pgfpathlineto{\pgfqpoint{2.537624in}{2.708662in}}%
\pgfpathlineto{\pgfqpoint{2.534490in}{2.708662in}}%
\pgfpathlineto{\pgfqpoint{2.531355in}{2.708662in}}%
\pgfpathlineto{\pgfqpoint{2.528220in}{2.708662in}}%
\pgfpathlineto{\pgfqpoint{2.527280in}{2.707536in}}%
\pgfpathlineto{\pgfqpoint{2.525085in}{2.704910in}}%
\pgfpathlineto{\pgfqpoint{2.521951in}{2.704910in}}%
\pgfpathlineto{\pgfqpoint{2.518816in}{2.704910in}}%
\pgfpathlineto{\pgfqpoint{2.515681in}{2.704910in}}%
\pgfpathlineto{\pgfqpoint{2.514741in}{2.703785in}}%
\pgfpathlineto{\pgfqpoint{2.512547in}{2.701158in}}%
\pgfpathlineto{\pgfqpoint{2.509412in}{2.701158in}}%
\pgfpathlineto{\pgfqpoint{2.506277in}{2.701158in}}%
\pgfpathlineto{\pgfqpoint{2.503142in}{2.701158in}}%
\pgfpathlineto{\pgfqpoint{2.500008in}{2.701158in}}%
\pgfpathlineto{\pgfqpoint{2.499067in}{2.700033in}}%
\pgfpathlineto{\pgfqpoint{2.496873in}{2.697407in}}%
\pgfpathlineto{\pgfqpoint{2.493738in}{2.697407in}}%
\pgfpathlineto{\pgfqpoint{2.490603in}{2.697407in}}%
\pgfpathlineto{\pgfqpoint{2.487469in}{2.697407in}}%
\pgfpathlineto{\pgfqpoint{2.486528in}{2.696281in}}%
\pgfpathlineto{\pgfqpoint{2.484334in}{2.693655in}}%
\pgfpathlineto{\pgfqpoint{2.481199in}{2.693655in}}%
\pgfpathlineto{\pgfqpoint{2.478064in}{2.693655in}}%
\pgfpathlineto{\pgfqpoint{2.474930in}{2.693655in}}%
\pgfpathlineto{\pgfqpoint{2.473989in}{2.692530in}}%
\pgfpathlineto{\pgfqpoint{2.471795in}{2.689904in}}%
\pgfpathlineto{\pgfqpoint{2.468660in}{2.689904in}}%
\pgfpathlineto{\pgfqpoint{2.465525in}{2.689904in}}%
\pgfpathlineto{\pgfqpoint{2.462391in}{2.689904in}}%
\pgfpathlineto{\pgfqpoint{2.461450in}{2.688778in}}%
\pgfpathlineto{\pgfqpoint{2.459256in}{2.686152in}}%
\pgfpathlineto{\pgfqpoint{2.456121in}{2.686152in}}%
\pgfpathlineto{\pgfqpoint{2.452986in}{2.686152in}}%
\pgfpathlineto{\pgfqpoint{2.449852in}{2.686152in}}%
\pgfpathlineto{\pgfqpoint{2.448911in}{2.685027in}}%
\pgfpathlineto{\pgfqpoint{2.446717in}{2.682401in}}%
\pgfpathlineto{\pgfqpoint{2.443582in}{2.682401in}}%
\pgfpathlineto{\pgfqpoint{2.440447in}{2.682401in}}%
\pgfpathlineto{\pgfqpoint{2.437313in}{2.682401in}}%
\pgfpathlineto{\pgfqpoint{2.434178in}{2.682401in}}%
\pgfpathlineto{\pgfqpoint{2.433238in}{2.681275in}}%
\pgfpathlineto{\pgfqpoint{2.431043in}{2.678649in}}%
\pgfpathlineto{\pgfqpoint{2.427908in}{2.678649in}}%
\pgfpathlineto{\pgfqpoint{2.424774in}{2.678649in}}%
\pgfpathlineto{\pgfqpoint{2.421639in}{2.678649in}}%
\pgfpathlineto{\pgfqpoint{2.420699in}{2.677524in}}%
\pgfpathlineto{\pgfqpoint{2.418504in}{2.674898in}}%
\pgfpathlineto{\pgfqpoint{2.415369in}{2.674898in}}%
\pgfpathlineto{\pgfqpoint{2.412235in}{2.674898in}}%
\pgfpathlineto{\pgfqpoint{2.409100in}{2.674898in}}%
\pgfpathlineto{\pgfqpoint{2.408160in}{2.673772in}}%
\pgfpathlineto{\pgfqpoint{2.405965in}{2.671146in}}%
\pgfpathlineto{\pgfqpoint{2.402831in}{2.671146in}}%
\pgfpathlineto{\pgfqpoint{2.399696in}{2.671146in}}%
\pgfpathlineto{\pgfqpoint{2.396561in}{2.671146in}}%
\pgfpathlineto{\pgfqpoint{2.395621in}{2.670021in}}%
\pgfpathlineto{\pgfqpoint{2.393426in}{2.667395in}}%
\pgfpathlineto{\pgfqpoint{2.390292in}{2.667395in}}%
\pgfpathlineto{\pgfqpoint{2.387157in}{2.667395in}}%
\pgfpathlineto{\pgfqpoint{2.384022in}{2.667395in}}%
\pgfpathlineto{\pgfqpoint{2.380887in}{2.667395in}}%
\pgfpathlineto{\pgfqpoint{2.379947in}{2.666269in}}%
\pgfpathlineto{\pgfqpoint{2.377753in}{2.663643in}}%
\pgfpathlineto{\pgfqpoint{2.374618in}{2.663643in}}%
\pgfpathlineto{\pgfqpoint{2.371483in}{2.663643in}}%
\pgfpathlineto{\pgfqpoint{2.368348in}{2.663643in}}%
\pgfpathlineto{\pgfqpoint{2.367408in}{2.662518in}}%
\pgfpathlineto{\pgfqpoint{2.365214in}{2.659892in}}%
\pgfpathlineto{\pgfqpoint{2.362079in}{2.659892in}}%
\pgfpathlineto{\pgfqpoint{2.358944in}{2.659892in}}%
\pgfpathlineto{\pgfqpoint{2.355809in}{2.659892in}}%
\pgfpathlineto{\pgfqpoint{2.354869in}{2.658766in}}%
\pgfpathlineto{\pgfqpoint{2.352675in}{2.656140in}}%
\pgfpathlineto{\pgfqpoint{2.349540in}{2.656140in}}%
\pgfpathlineto{\pgfqpoint{2.346405in}{2.656140in}}%
\pgfpathlineto{\pgfqpoint{2.343270in}{2.656140in}}%
\pgfpathlineto{\pgfqpoint{2.342330in}{2.655015in}}%
\pgfpathlineto{\pgfqpoint{2.340136in}{2.652389in}}%
\pgfpathlineto{\pgfqpoint{2.337001in}{2.652389in}}%
\pgfpathlineto{\pgfqpoint{2.333866in}{2.652389in}}%
\pgfpathlineto{\pgfqpoint{2.330731in}{2.652389in}}%
\pgfpathlineto{\pgfqpoint{2.327597in}{2.652389in}}%
\pgfpathlineto{\pgfqpoint{2.326656in}{2.651263in}}%
\pgfpathlineto{\pgfqpoint{2.324462in}{2.648637in}}%
\pgfpathlineto{\pgfqpoint{2.321327in}{2.648637in}}%
\pgfpathlineto{\pgfqpoint{2.318192in}{2.648637in}}%
\pgfpathlineto{\pgfqpoint{2.315058in}{2.648637in}}%
\pgfpathlineto{\pgfqpoint{2.314117in}{2.647512in}}%
\pgfpathlineto{\pgfqpoint{2.311923in}{2.644885in}}%
\pgfpathlineto{\pgfqpoint{2.308788in}{2.644885in}}%
\pgfpathlineto{\pgfqpoint{2.305654in}{2.644885in}}%
\pgfpathlineto{\pgfqpoint{2.302519in}{2.644885in}}%
\pgfpathlineto{\pgfqpoint{2.301578in}{2.643760in}}%
\pgfpathlineto{\pgfqpoint{2.299384in}{2.641134in}}%
\pgfpathlineto{\pgfqpoint{2.296249in}{2.641134in}}%
\pgfpathlineto{\pgfqpoint{2.293115in}{2.641134in}}%
\pgfpathlineto{\pgfqpoint{2.289980in}{2.641134in}}%
\pgfpathlineto{\pgfqpoint{2.289039in}{2.640008in}}%
\pgfpathlineto{\pgfqpoint{2.286845in}{2.637382in}}%
\pgfpathlineto{\pgfqpoint{2.283710in}{2.637382in}}%
\pgfpathlineto{\pgfqpoint{2.280576in}{2.637382in}}%
\pgfpathlineto{\pgfqpoint{2.277441in}{2.637382in}}%
\pgfpathlineto{\pgfqpoint{2.276500in}{2.636257in}}%
\pgfpathlineto{\pgfqpoint{2.274306in}{2.633631in}}%
\pgfpathlineto{\pgfqpoint{2.271171in}{2.633631in}}%
\pgfpathlineto{\pgfqpoint{2.268037in}{2.633631in}}%
\pgfpathlineto{\pgfqpoint{2.264902in}{2.633631in}}%
\pgfpathlineto{\pgfqpoint{2.261767in}{2.633631in}}%
\pgfpathlineto{\pgfqpoint{2.260827in}{2.632505in}}%
\pgfpathlineto{\pgfqpoint{2.258632in}{2.629879in}}%
\pgfpathlineto{\pgfqpoint{2.255498in}{2.629879in}}%
\pgfpathlineto{\pgfqpoint{2.252363in}{2.629879in}}%
\pgfpathlineto{\pgfqpoint{2.249228in}{2.629879in}}%
\pgfpathlineto{\pgfqpoint{2.248288in}{2.628754in}}%
\pgfpathlineto{\pgfqpoint{2.246093in}{2.626128in}}%
\pgfpathlineto{\pgfqpoint{2.242959in}{2.626128in}}%
\pgfpathlineto{\pgfqpoint{2.239824in}{2.626128in}}%
\pgfpathlineto{\pgfqpoint{2.236689in}{2.626128in}}%
\pgfpathlineto{\pgfqpoint{2.235749in}{2.625002in}}%
\pgfpathlineto{\pgfqpoint{2.233554in}{2.622376in}}%
\pgfpathlineto{\pgfqpoint{2.230420in}{2.622376in}}%
\pgfpathlineto{\pgfqpoint{2.227285in}{2.622376in}}%
\pgfpathlineto{\pgfqpoint{2.224150in}{2.622376in}}%
\pgfpathlineto{\pgfqpoint{2.223210in}{2.621251in}}%
\pgfpathlineto{\pgfqpoint{2.221015in}{2.618625in}}%
\pgfpathlineto{\pgfqpoint{2.217881in}{2.618625in}}%
\pgfpathlineto{\pgfqpoint{2.214746in}{2.618625in}}%
\pgfpathlineto{\pgfqpoint{2.211611in}{2.618625in}}%
\pgfpathlineto{\pgfqpoint{2.208477in}{2.618625in}}%
\pgfpathlineto{\pgfqpoint{2.207536in}{2.617499in}}%
\pgfpathlineto{\pgfqpoint{2.205342in}{2.614873in}}%
\pgfpathlineto{\pgfqpoint{2.202207in}{2.614873in}}%
\pgfpathlineto{\pgfqpoint{2.199072in}{2.614873in}}%
\pgfpathlineto{\pgfqpoint{2.195938in}{2.614873in}}%
\pgfpathlineto{\pgfqpoint{2.194997in}{2.613748in}}%
\pgfpathlineto{\pgfqpoint{2.192803in}{2.611122in}}%
\pgfpathlineto{\pgfqpoint{2.189668in}{2.611122in}}%
\pgfpathlineto{\pgfqpoint{2.186533in}{2.611122in}}%
\pgfpathlineto{\pgfqpoint{2.183399in}{2.611122in}}%
\pgfpathlineto{\pgfqpoint{2.182458in}{2.609996in}}%
\pgfpathlineto{\pgfqpoint{2.180264in}{2.607370in}}%
\pgfpathlineto{\pgfqpoint{2.177129in}{2.607370in}}%
\pgfpathlineto{\pgfqpoint{2.173994in}{2.607370in}}%
\pgfpathlineto{\pgfqpoint{2.170860in}{2.607370in}}%
\pgfpathlineto{\pgfqpoint{2.169919in}{2.606245in}}%
\pgfpathlineto{\pgfqpoint{2.167725in}{2.603619in}}%
\pgfpathlineto{\pgfqpoint{2.164590in}{2.603619in}}%
\pgfpathlineto{\pgfqpoint{2.161455in}{2.603619in}}%
\pgfpathlineto{\pgfqpoint{2.158321in}{2.603619in}}%
\pgfpathlineto{\pgfqpoint{2.155186in}{2.603619in}}%
\pgfpathlineto{\pgfqpoint{2.154245in}{2.602493in}}%
\pgfpathlineto{\pgfqpoint{2.152051in}{2.599867in}}%
\pgfpathlineto{\pgfqpoint{2.148916in}{2.599867in}}%
\pgfpathlineto{\pgfqpoint{2.145782in}{2.599867in}}%
\pgfpathlineto{\pgfqpoint{2.142647in}{2.599867in}}%
\pgfpathlineto{\pgfqpoint{2.141706in}{2.598742in}}%
\pgfpathlineto{\pgfqpoint{2.139512in}{2.596115in}}%
\pgfpathlineto{\pgfqpoint{2.136377in}{2.596115in}}%
\pgfpathlineto{\pgfqpoint{2.133243in}{2.596115in}}%
\pgfpathlineto{\pgfqpoint{2.130108in}{2.596115in}}%
\pgfpathlineto{\pgfqpoint{2.129168in}{2.594990in}}%
\pgfpathlineto{\pgfqpoint{2.126973in}{2.592364in}}%
\pgfpathlineto{\pgfqpoint{2.123838in}{2.592364in}}%
\pgfpathlineto{\pgfqpoint{2.120704in}{2.592364in}}%
\pgfpathlineto{\pgfqpoint{2.117569in}{2.592364in}}%
\pgfpathlineto{\pgfqpoint{2.116629in}{2.591238in}}%
\pgfpathlineto{\pgfqpoint{2.114434in}{2.588612in}}%
\pgfpathlineto{\pgfqpoint{2.111299in}{2.588612in}}%
\pgfpathlineto{\pgfqpoint{2.108165in}{2.588612in}}%
\pgfpathlineto{\pgfqpoint{2.105030in}{2.588612in}}%
\pgfpathlineto{\pgfqpoint{2.104090in}{2.587487in}}%
\pgfpathlineto{\pgfqpoint{2.101895in}{2.584861in}}%
\pgfpathlineto{\pgfqpoint{2.098761in}{2.584861in}}%
\pgfpathlineto{\pgfqpoint{2.095626in}{2.584861in}}%
\pgfpathlineto{\pgfqpoint{2.092491in}{2.584861in}}%
\pgfpathlineto{\pgfqpoint{2.089356in}{2.584861in}}%
\pgfpathlineto{\pgfqpoint{2.088416in}{2.583735in}}%
\pgfpathlineto{\pgfqpoint{2.086222in}{2.581109in}}%
\pgfpathlineto{\pgfqpoint{2.083087in}{2.581109in}}%
\pgfpathlineto{\pgfqpoint{2.079952in}{2.581109in}}%
\pgfpathlineto{\pgfqpoint{2.076817in}{2.581109in}}%
\pgfpathlineto{\pgfqpoint{2.075877in}{2.579984in}}%
\pgfpathlineto{\pgfqpoint{2.073683in}{2.577358in}}%
\pgfpathlineto{\pgfqpoint{2.070548in}{2.577358in}}%
\pgfpathlineto{\pgfqpoint{2.067413in}{2.577358in}}%
\pgfpathlineto{\pgfqpoint{2.064278in}{2.577358in}}%
\pgfpathlineto{\pgfqpoint{2.063338in}{2.576232in}}%
\pgfpathlineto{\pgfqpoint{2.061144in}{2.573606in}}%
\pgfpathlineto{\pgfqpoint{2.058009in}{2.573606in}}%
\pgfpathlineto{\pgfqpoint{2.054874in}{2.573606in}}%
\pgfpathlineto{\pgfqpoint{2.051739in}{2.573606in}}%
\pgfpathlineto{\pgfqpoint{2.050799in}{2.572481in}}%
\pgfpathlineto{\pgfqpoint{2.048605in}{2.569855in}}%
\pgfpathlineto{\pgfqpoint{2.045470in}{2.569855in}}%
\pgfpathlineto{\pgfqpoint{2.042335in}{2.569855in}}%
\pgfpathlineto{\pgfqpoint{2.039200in}{2.569855in}}%
\pgfpathlineto{\pgfqpoint{2.036066in}{2.569855in}}%
\pgfpathlineto{\pgfqpoint{2.035125in}{2.568729in}}%
\pgfpathlineto{\pgfqpoint{2.032931in}{2.566103in}}%
\pgfpathlineto{\pgfqpoint{2.029796in}{2.566103in}}%
\pgfpathlineto{\pgfqpoint{2.026661in}{2.566103in}}%
\pgfpathlineto{\pgfqpoint{2.023527in}{2.566103in}}%
\pgfpathlineto{\pgfqpoint{2.022586in}{2.564978in}}%
\pgfpathlineto{\pgfqpoint{2.020392in}{2.562352in}}%
\pgfpathlineto{\pgfqpoint{2.017257in}{2.562352in}}%
\pgfpathlineto{\pgfqpoint{2.014122in}{2.562352in}}%
\pgfpathlineto{\pgfqpoint{2.010988in}{2.562352in}}%
\pgfpathlineto{\pgfqpoint{2.010047in}{2.561226in}}%
\pgfpathlineto{\pgfqpoint{2.007853in}{2.558600in}}%
\pgfpathlineto{\pgfqpoint{2.004718in}{2.558600in}}%
\pgfpathlineto{\pgfqpoint{2.001584in}{2.558600in}}%
\pgfpathlineto{\pgfqpoint{1.998449in}{2.558600in}}%
\pgfpathlineto{\pgfqpoint{1.997508in}{2.557475in}}%
\pgfpathlineto{\pgfqpoint{1.995314in}{2.554849in}}%
\pgfpathlineto{\pgfqpoint{1.992179in}{2.554849in}}%
\pgfpathlineto{\pgfqpoint{1.989045in}{2.554849in}}%
\pgfpathlineto{\pgfqpoint{1.985910in}{2.554849in}}%
\pgfpathlineto{\pgfqpoint{1.982775in}{2.554849in}}%
\pgfpathlineto{\pgfqpoint{1.981835in}{2.553723in}}%
\pgfpathlineto{\pgfqpoint{1.979640in}{2.551097in}}%
\pgfpathlineto{\pgfqpoint{1.976506in}{2.551097in}}%
\pgfpathlineto{\pgfqpoint{1.973371in}{2.551097in}}%
\pgfpathlineto{\pgfqpoint{1.970236in}{2.551097in}}%
\pgfpathlineto{\pgfqpoint{1.969296in}{2.549972in}}%
\pgfpathlineto{\pgfqpoint{1.967101in}{2.547346in}}%
\pgfpathlineto{\pgfqpoint{1.963967in}{2.547346in}}%
\pgfpathlineto{\pgfqpoint{1.960832in}{2.547346in}}%
\pgfpathlineto{\pgfqpoint{1.957697in}{2.547346in}}%
\pgfpathlineto{\pgfqpoint{1.956757in}{2.546220in}}%
\pgfpathlineto{\pgfqpoint{1.954562in}{2.543594in}}%
\pgfpathlineto{\pgfqpoint{1.951428in}{2.543594in}}%
\pgfpathlineto{\pgfqpoint{1.948293in}{2.543594in}}%
\pgfpathlineto{\pgfqpoint{1.945158in}{2.543594in}}%
\pgfpathlineto{\pgfqpoint{1.944218in}{2.542469in}}%
\pgfpathlineto{\pgfqpoint{1.942023in}{2.539842in}}%
\pgfpathlineto{\pgfqpoint{1.938889in}{2.539842in}}%
\pgfpathlineto{\pgfqpoint{1.935754in}{2.539842in}}%
\pgfpathlineto{\pgfqpoint{1.932619in}{2.539842in}}%
\pgfpathlineto{\pgfqpoint{1.931679in}{2.538717in}}%
\pgfpathlineto{\pgfqpoint{1.929484in}{2.536091in}}%
\pgfpathlineto{\pgfqpoint{1.926350in}{2.536091in}}%
\pgfpathlineto{\pgfqpoint{1.923215in}{2.536091in}}%
\pgfpathlineto{\pgfqpoint{1.920080in}{2.536091in}}%
\pgfpathlineto{\pgfqpoint{1.916945in}{2.536091in}}%
\pgfpathlineto{\pgfqpoint{1.916005in}{2.534965in}}%
\pgfpathlineto{\pgfqpoint{1.913811in}{2.532339in}}%
\pgfpathlineto{\pgfqpoint{1.910676in}{2.532339in}}%
\pgfpathlineto{\pgfqpoint{1.907541in}{2.532339in}}%
\pgfpathlineto{\pgfqpoint{1.904407in}{2.532339in}}%
\pgfpathlineto{\pgfqpoint{1.903466in}{2.531214in}}%
\pgfpathlineto{\pgfqpoint{1.901272in}{2.528588in}}%
\pgfpathlineto{\pgfqpoint{1.898137in}{2.528588in}}%
\pgfpathlineto{\pgfqpoint{1.895002in}{2.528588in}}%
\pgfpathlineto{\pgfqpoint{1.891868in}{2.528588in}}%
\pgfpathlineto{\pgfqpoint{1.890927in}{2.527462in}}%
\pgfpathlineto{\pgfqpoint{1.888733in}{2.524836in}}%
\pgfpathlineto{\pgfqpoint{1.885598in}{2.524836in}}%
\pgfpathlineto{\pgfqpoint{1.882463in}{2.524836in}}%
\pgfpathlineto{\pgfqpoint{1.879329in}{2.524836in}}%
\pgfpathlineto{\pgfqpoint{1.878388in}{2.523711in}}%
\pgfpathlineto{\pgfqpoint{1.876194in}{2.521085in}}%
\pgfpathlineto{\pgfqpoint{1.873059in}{2.521085in}}%
\pgfpathlineto{\pgfqpoint{1.869924in}{2.521085in}}%
\pgfpathlineto{\pgfqpoint{1.866790in}{2.521085in}}%
\pgfpathlineto{\pgfqpoint{1.863655in}{2.521085in}}%
\pgfpathlineto{\pgfqpoint{1.862714in}{2.519959in}}%
\pgfpathlineto{\pgfqpoint{1.860520in}{2.517333in}}%
\pgfpathlineto{\pgfqpoint{1.857385in}{2.517333in}}%
\pgfpathlineto{\pgfqpoint{1.854251in}{2.517333in}}%
\pgfpathlineto{\pgfqpoint{1.851116in}{2.517333in}}%
\pgfpathlineto{\pgfqpoint{1.850175in}{2.516208in}}%
\pgfpathlineto{\pgfqpoint{1.847981in}{2.513582in}}%
\pgfpathlineto{\pgfqpoint{1.844846in}{2.513582in}}%
\pgfpathlineto{\pgfqpoint{1.841712in}{2.513582in}}%
\pgfpathlineto{\pgfqpoint{1.838577in}{2.513582in}}%
\pgfpathlineto{\pgfqpoint{1.837636in}{2.512456in}}%
\pgfpathlineto{\pgfqpoint{1.835442in}{2.509830in}}%
\pgfpathlineto{\pgfqpoint{1.832307in}{2.509830in}}%
\pgfpathlineto{\pgfqpoint{1.829173in}{2.509830in}}%
\pgfpathlineto{\pgfqpoint{1.826038in}{2.509830in}}%
\pgfpathlineto{\pgfqpoint{1.825098in}{2.508705in}}%
\pgfpathlineto{\pgfqpoint{1.822903in}{2.506079in}}%
\pgfpathlineto{\pgfqpoint{1.819768in}{2.506079in}}%
\pgfpathlineto{\pgfqpoint{1.816634in}{2.506079in}}%
\pgfpathlineto{\pgfqpoint{1.813499in}{2.506079in}}%
\pgfpathlineto{\pgfqpoint{1.812559in}{2.504953in}}%
\pgfpathlineto{\pgfqpoint{1.810364in}{2.502327in}}%
\pgfpathlineto{\pgfqpoint{1.807229in}{2.502327in}}%
\pgfpathlineto{\pgfqpoint{1.804095in}{2.502327in}}%
\pgfpathlineto{\pgfqpoint{1.800960in}{2.502327in}}%
\pgfpathlineto{\pgfqpoint{1.797825in}{2.502327in}}%
\pgfpathlineto{\pgfqpoint{1.796885in}{2.501202in}}%
\pgfpathlineto{\pgfqpoint{1.794691in}{2.498576in}}%
\pgfpathlineto{\pgfqpoint{1.791556in}{2.498576in}}%
\pgfpathlineto{\pgfqpoint{1.788421in}{2.498576in}}%
\pgfpathlineto{\pgfqpoint{1.785286in}{2.498576in}}%
\pgfpathlineto{\pgfqpoint{1.784346in}{2.497450in}}%
\pgfpathlineto{\pgfqpoint{1.782152in}{2.494824in}}%
\pgfpathlineto{\pgfqpoint{1.779017in}{2.494824in}}%
\pgfpathlineto{\pgfqpoint{1.775882in}{2.494824in}}%
\pgfpathlineto{\pgfqpoint{1.772747in}{2.494824in}}%
\pgfpathlineto{\pgfqpoint{1.771807in}{2.493699in}}%
\pgfpathlineto{\pgfqpoint{1.769613in}{2.491073in}}%
\pgfpathlineto{\pgfqpoint{1.766478in}{2.491073in}}%
\pgfpathlineto{\pgfqpoint{1.763343in}{2.491073in}}%
\pgfpathlineto{\pgfqpoint{1.760208in}{2.491073in}}%
\pgfpathlineto{\pgfqpoint{1.759268in}{2.489947in}}%
\pgfpathlineto{\pgfqpoint{1.757074in}{2.487321in}}%
\pgfpathlineto{\pgfqpoint{1.753939in}{2.487321in}}%
\pgfpathlineto{\pgfqpoint{1.750804in}{2.487321in}}%
\pgfpathlineto{\pgfqpoint{1.747669in}{2.487321in}}%
\pgfpathlineto{\pgfqpoint{1.744535in}{2.487321in}}%
\pgfpathlineto{\pgfqpoint{1.743594in}{2.486196in}}%
\pgfpathlineto{\pgfqpoint{1.741400in}{2.483569in}}%
\pgfpathlineto{\pgfqpoint{1.738265in}{2.483569in}}%
\pgfpathlineto{\pgfqpoint{1.735130in}{2.483569in}}%
\pgfpathlineto{\pgfqpoint{1.731996in}{2.483569in}}%
\pgfpathlineto{\pgfqpoint{1.731055in}{2.482444in}}%
\pgfpathlineto{\pgfqpoint{1.728861in}{2.479818in}}%
\pgfpathlineto{\pgfqpoint{1.725726in}{2.479818in}}%
\pgfpathlineto{\pgfqpoint{1.722591in}{2.479818in}}%
\pgfpathlineto{\pgfqpoint{1.719457in}{2.479818in}}%
\pgfpathlineto{\pgfqpoint{1.718516in}{2.478692in}}%
\pgfpathlineto{\pgfqpoint{1.716322in}{2.476066in}}%
\pgfpathlineto{\pgfqpoint{1.713187in}{2.476066in}}%
\pgfpathlineto{\pgfqpoint{1.710052in}{2.476066in}}%
\pgfpathlineto{\pgfqpoint{1.706918in}{2.476066in}}%
\pgfpathlineto{\pgfqpoint{1.705977in}{2.474941in}}%
\pgfpathlineto{\pgfqpoint{1.703783in}{2.472315in}}%
\pgfpathlineto{\pgfqpoint{1.700648in}{2.472315in}}%
\pgfpathlineto{\pgfqpoint{1.697514in}{2.472315in}}%
\pgfpathlineto{\pgfqpoint{1.694379in}{2.472315in}}%
\pgfpathlineto{\pgfqpoint{1.691244in}{2.472315in}}%
\pgfpathlineto{\pgfqpoint{1.690304in}{2.471189in}}%
\pgfpathlineto{\pgfqpoint{1.688109in}{2.468563in}}%
\pgfpathlineto{\pgfqpoint{1.684975in}{2.468563in}}%
\pgfpathlineto{\pgfqpoint{1.681840in}{2.468563in}}%
\pgfpathlineto{\pgfqpoint{1.678705in}{2.468563in}}%
\pgfpathlineto{\pgfqpoint{1.677765in}{2.467438in}}%
\pgfpathlineto{\pgfqpoint{1.675570in}{2.464812in}}%
\pgfpathlineto{\pgfqpoint{1.672436in}{2.464812in}}%
\pgfpathlineto{\pgfqpoint{1.669301in}{2.464812in}}%
\pgfpathlineto{\pgfqpoint{1.666166in}{2.464812in}}%
\pgfpathlineto{\pgfqpoint{1.665226in}{2.463686in}}%
\pgfpathlineto{\pgfqpoint{1.663031in}{2.461060in}}%
\pgfpathlineto{\pgfqpoint{1.659897in}{2.461060in}}%
\pgfpathlineto{\pgfqpoint{1.656762in}{2.461060in}}%
\pgfpathlineto{\pgfqpoint{1.653627in}{2.461060in}}%
\pgfpathlineto{\pgfqpoint{1.652687in}{2.459935in}}%
\pgfpathlineto{\pgfqpoint{1.650492in}{2.457309in}}%
\pgfpathlineto{\pgfqpoint{1.647358in}{2.457309in}}%
\pgfpathlineto{\pgfqpoint{1.644223in}{2.457309in}}%
\pgfpathlineto{\pgfqpoint{1.641088in}{2.457309in}}%
\pgfpathlineto{\pgfqpoint{1.640148in}{2.456183in}}%
\pgfpathlineto{\pgfqpoint{1.637953in}{2.453557in}}%
\pgfpathlineto{\pgfqpoint{1.634819in}{2.453557in}}%
\pgfpathlineto{\pgfqpoint{1.631684in}{2.453557in}}%
\pgfpathlineto{\pgfqpoint{1.628549in}{2.453557in}}%
\pgfpathlineto{\pgfqpoint{1.625414in}{2.453557in}}%
\pgfpathlineto{\pgfqpoint{1.624474in}{2.452432in}}%
\pgfpathlineto{\pgfqpoint{1.622280in}{2.449806in}}%
\pgfpathlineto{\pgfqpoint{1.619145in}{2.449806in}}%
\pgfpathlineto{\pgfqpoint{1.616010in}{2.449806in}}%
\pgfpathlineto{\pgfqpoint{1.612875in}{2.449806in}}%
\pgfpathlineto{\pgfqpoint{1.611935in}{2.448680in}}%
\pgfpathlineto{\pgfqpoint{1.609741in}{2.446054in}}%
\pgfpathlineto{\pgfqpoint{1.606606in}{2.446054in}}%
\pgfpathlineto{\pgfqpoint{1.603471in}{2.446054in}}%
\pgfpathlineto{\pgfqpoint{1.600337in}{2.446054in}}%
\pgfpathlineto{\pgfqpoint{1.599396in}{2.444929in}}%
\pgfpathlineto{\pgfqpoint{1.597202in}{2.442303in}}%
\pgfpathlineto{\pgfqpoint{1.594067in}{2.442303in}}%
\pgfpathlineto{\pgfqpoint{1.590932in}{2.442303in}}%
\pgfpathlineto{\pgfqpoint{1.587798in}{2.442303in}}%
\pgfpathlineto{\pgfqpoint{1.586857in}{2.441177in}}%
\pgfpathlineto{\pgfqpoint{1.584663in}{2.438551in}}%
\pgfpathlineto{\pgfqpoint{1.581528in}{2.438551in}}%
\pgfpathlineto{\pgfqpoint{1.578393in}{2.438551in}}%
\pgfpathlineto{\pgfqpoint{1.575259in}{2.438551in}}%
\pgfpathlineto{\pgfqpoint{1.572124in}{2.438551in}}%
\pgfpathlineto{\pgfqpoint{1.571183in}{2.437426in}}%
\pgfpathlineto{\pgfqpoint{1.568989in}{2.434800in}}%
\pgfpathlineto{\pgfqpoint{1.565854in}{2.434800in}}%
\pgfpathlineto{\pgfqpoint{1.562720in}{2.434800in}}%
\pgfpathlineto{\pgfqpoint{1.559585in}{2.434800in}}%
\pgfpathlineto{\pgfqpoint{1.558644in}{2.433674in}}%
\pgfpathlineto{\pgfqpoint{1.556450in}{2.431048in}}%
\pgfpathlineto{\pgfqpoint{1.553315in}{2.431048in}}%
\pgfpathlineto{\pgfqpoint{1.550181in}{2.431048in}}%
\pgfpathlineto{\pgfqpoint{1.547046in}{2.431048in}}%
\pgfpathlineto{\pgfqpoint{1.546105in}{2.429923in}}%
\pgfpathlineto{\pgfqpoint{1.543911in}{2.427296in}}%
\pgfpathlineto{\pgfqpoint{1.540776in}{2.427296in}}%
\pgfpathlineto{\pgfqpoint{1.537642in}{2.427296in}}%
\pgfpathlineto{\pgfqpoint{1.534507in}{2.427296in}}%
\pgfpathlineto{\pgfqpoint{1.533566in}{2.426171in}}%
\pgfpathlineto{\pgfqpoint{1.531372in}{2.423545in}}%
\pgfpathlineto{\pgfqpoint{1.528237in}{2.423545in}}%
\pgfpathlineto{\pgfqpoint{1.525103in}{2.423545in}}%
\pgfpathlineto{\pgfqpoint{1.521968in}{2.423545in}}%
\pgfpathlineto{\pgfqpoint{1.518833in}{2.423545in}}%
\pgfpathlineto{\pgfqpoint{1.517893in}{2.422419in}}%
\pgfpathlineto{\pgfqpoint{1.515698in}{2.419793in}}%
\pgfpathlineto{\pgfqpoint{1.512564in}{2.419793in}}%
\pgfpathlineto{\pgfqpoint{1.509429in}{2.419793in}}%
\pgfpathlineto{\pgfqpoint{1.506294in}{2.419793in}}%
\pgfpathlineto{\pgfqpoint{1.505354in}{2.418668in}}%
\pgfpathlineto{\pgfqpoint{1.503159in}{2.416042in}}%
\pgfpathlineto{\pgfqpoint{1.500025in}{2.416042in}}%
\pgfpathlineto{\pgfqpoint{1.496890in}{2.416042in}}%
\pgfpathlineto{\pgfqpoint{1.493755in}{2.416042in}}%
\pgfpathlineto{\pgfqpoint{1.492815in}{2.414916in}}%
\pgfpathlineto{\pgfqpoint{1.490621in}{2.412290in}}%
\pgfpathlineto{\pgfqpoint{1.487486in}{2.412290in}}%
\pgfpathlineto{\pgfqpoint{1.484351in}{2.412290in}}%
\pgfpathlineto{\pgfqpoint{1.481216in}{2.412290in}}%
\pgfpathlineto{\pgfqpoint{1.480276in}{2.411165in}}%
\pgfpathlineto{\pgfqpoint{1.478082in}{2.408539in}}%
\pgfpathlineto{\pgfqpoint{1.474947in}{2.408539in}}%
\pgfpathlineto{\pgfqpoint{1.471812in}{2.408539in}}%
\pgfpathlineto{\pgfqpoint{1.468677in}{2.408539in}}%
\pgfpathlineto{\pgfqpoint{1.467737in}{2.407413in}}%
\pgfpathlineto{\pgfqpoint{1.465543in}{2.404787in}}%
\pgfpathlineto{\pgfqpoint{1.462408in}{2.404787in}}%
\pgfpathlineto{\pgfqpoint{1.461467in}{2.403662in}}%
\pgfpathlineto{\pgfqpoint{1.461467in}{2.399910in}}%
\pgfpathlineto{\pgfqpoint{1.459273in}{2.397284in}}%
\pgfpathlineto{\pgfqpoint{1.458333in}{2.396159in}}%
\pgfpathlineto{\pgfqpoint{1.458333in}{2.392407in}}%
\pgfpathlineto{\pgfqpoint{1.458333in}{2.388656in}}%
\pgfpathlineto{\pgfqpoint{1.456138in}{2.386030in}}%
\pgfpathlineto{\pgfqpoint{1.455198in}{2.384904in}}%
\pgfpathlineto{\pgfqpoint{1.455198in}{2.381153in}}%
\pgfpathlineto{\pgfqpoint{1.455198in}{2.377401in}}%
\pgfpathlineto{\pgfqpoint{1.455198in}{2.373649in}}%
\pgfpathlineto{\pgfqpoint{1.453004in}{2.371023in}}%
\pgfpathlineto{\pgfqpoint{1.452063in}{2.369898in}}%
\pgfpathlineto{\pgfqpoint{1.452063in}{2.366146in}}%
\pgfpathlineto{\pgfqpoint{1.452063in}{2.362395in}}%
\pgfpathlineto{\pgfqpoint{1.449869in}{2.359769in}}%
\pgfpathlineto{\pgfqpoint{1.448928in}{2.358643in}}%
\pgfpathlineto{\pgfqpoint{1.448928in}{2.354892in}}%
\pgfpathlineto{\pgfqpoint{1.448928in}{2.351140in}}%
\pgfpathlineto{\pgfqpoint{1.448928in}{2.347389in}}%
\pgfpathlineto{\pgfqpoint{1.446734in}{2.344763in}}%
\pgfpathlineto{\pgfqpoint{1.445794in}{2.343637in}}%
\pgfpathlineto{\pgfqpoint{1.445794in}{2.339886in}}%
\pgfpathlineto{\pgfqpoint{1.445794in}{2.336134in}}%
\pgfpathlineto{\pgfqpoint{1.443599in}{2.333508in}}%
\pgfpathlineto{\pgfqpoint{1.442659in}{2.332383in}}%
\pgfpathlineto{\pgfqpoint{1.442659in}{2.328631in}}%
\pgfpathlineto{\pgfqpoint{1.442659in}{2.324880in}}%
\pgfpathlineto{\pgfqpoint{1.442659in}{2.321128in}}%
\pgfpathlineto{\pgfqpoint{1.440465in}{2.318502in}}%
\pgfpathlineto{\pgfqpoint{1.439524in}{2.317376in}}%
\pgfpathlineto{\pgfqpoint{1.439524in}{2.313625in}}%
\pgfpathlineto{\pgfqpoint{1.439524in}{2.309873in}}%
\pgfpathlineto{\pgfqpoint{1.437330in}{2.307247in}}%
\pgfpathlineto{\pgfqpoint{1.436389in}{2.306122in}}%
\pgfpathlineto{\pgfqpoint{1.436389in}{2.302370in}}%
\pgfpathlineto{\pgfqpoint{1.436389in}{2.298619in}}%
\pgfpathlineto{\pgfqpoint{1.436389in}{2.294867in}}%
\pgfpathlineto{\pgfqpoint{1.434195in}{2.292241in}}%
\pgfpathlineto{\pgfqpoint{1.433255in}{2.291116in}}%
\pgfpathlineto{\pgfqpoint{1.433255in}{2.287364in}}%
\pgfpathlineto{\pgfqpoint{1.433255in}{2.283613in}}%
\pgfpathlineto{\pgfqpoint{1.431060in}{2.280987in}}%
\pgfpathlineto{\pgfqpoint{1.430120in}{2.279861in}}%
\pgfpathlineto{\pgfqpoint{1.430120in}{2.276110in}}%
\pgfpathlineto{\pgfqpoint{1.430120in}{2.272358in}}%
\pgfpathlineto{\pgfqpoint{1.430120in}{2.268607in}}%
\pgfpathlineto{\pgfqpoint{1.427926in}{2.265980in}}%
\pgfpathlineto{\pgfqpoint{1.426985in}{2.264855in}}%
\pgfpathlineto{\pgfqpoint{1.426985in}{2.261103in}}%
\pgfpathlineto{\pgfqpoint{1.426985in}{2.257352in}}%
\pgfpathlineto{\pgfqpoint{1.424791in}{2.254726in}}%
\pgfpathlineto{\pgfqpoint{1.423851in}{2.253600in}}%
\pgfpathlineto{\pgfqpoint{1.423851in}{2.249849in}}%
\pgfpathlineto{\pgfqpoint{1.423851in}{2.246097in}}%
\pgfpathlineto{\pgfqpoint{1.421656in}{2.243471in}}%
\pgfpathlineto{\pgfqpoint{1.420716in}{2.242346in}}%
\pgfpathlineto{\pgfqpoint{1.420716in}{2.238594in}}%
\pgfpathlineto{\pgfqpoint{1.420716in}{2.234843in}}%
\pgfpathlineto{\pgfqpoint{1.420716in}{2.231091in}}%
\pgfpathlineto{\pgfqpoint{1.418521in}{2.228465in}}%
\pgfpathlineto{\pgfqpoint{1.417581in}{2.227340in}}%
\pgfpathlineto{\pgfqpoint{1.417581in}{2.223588in}}%
\pgfpathlineto{\pgfqpoint{1.417581in}{2.219837in}}%
\pgfpathlineto{\pgfqpoint{1.415387in}{2.217210in}}%
\pgfpathlineto{\pgfqpoint{1.414446in}{2.216085in}}%
\pgfpathlineto{\pgfqpoint{1.414446in}{2.212334in}}%
\pgfpathlineto{\pgfqpoint{1.414446in}{2.208582in}}%
\pgfpathlineto{\pgfqpoint{1.414446in}{2.204830in}}%
\pgfpathlineto{\pgfqpoint{1.412252in}{2.202204in}}%
\pgfpathlineto{\pgfqpoint{1.411312in}{2.201079in}}%
\pgfpathlineto{\pgfqpoint{1.411312in}{2.197327in}}%
\pgfpathlineto{\pgfqpoint{1.411312in}{2.193576in}}%
\pgfpathlineto{\pgfqpoint{1.409117in}{2.190950in}}%
\pgfpathlineto{\pgfqpoint{1.408177in}{2.189824in}}%
\pgfpathlineto{\pgfqpoint{1.408177in}{2.186073in}}%
\pgfpathlineto{\pgfqpoint{1.408177in}{2.182321in}}%
\pgfpathlineto{\pgfqpoint{1.408177in}{2.178570in}}%
\pgfpathlineto{\pgfqpoint{1.405982in}{2.175944in}}%
\pgfpathlineto{\pgfqpoint{1.405042in}{2.174818in}}%
\pgfpathlineto{\pgfqpoint{1.405042in}{2.171067in}}%
\pgfpathlineto{\pgfqpoint{1.405042in}{2.167315in}}%
\pgfpathlineto{\pgfqpoint{1.402848in}{2.164689in}}%
\pgfpathlineto{\pgfqpoint{1.401907in}{2.163564in}}%
\pgfpathlineto{\pgfqpoint{1.401907in}{2.159812in}}%
\pgfpathlineto{\pgfqpoint{1.401907in}{2.156060in}}%
\pgfpathlineto{\pgfqpoint{1.401907in}{2.152309in}}%
\pgfpathlineto{\pgfqpoint{1.399713in}{2.149683in}}%
\pgfpathlineto{\pgfqpoint{1.398773in}{2.148557in}}%
\pgfpathlineto{\pgfqpoint{1.398773in}{2.144806in}}%
\pgfpathlineto{\pgfqpoint{1.398773in}{2.141054in}}%
\pgfpathlineto{\pgfqpoint{1.396578in}{2.138428in}}%
\pgfpathlineto{\pgfqpoint{1.395638in}{2.137303in}}%
\pgfpathlineto{\pgfqpoint{1.395638in}{2.133551in}}%
\pgfpathlineto{\pgfqpoint{1.395638in}{2.129800in}}%
\pgfpathlineto{\pgfqpoint{1.395638in}{2.126048in}}%
\pgfpathlineto{\pgfqpoint{1.393444in}{2.123422in}}%
\pgfpathlineto{\pgfqpoint{1.392503in}{2.122297in}}%
\pgfpathlineto{\pgfqpoint{1.392503in}{2.118545in}}%
\pgfpathlineto{\pgfqpoint{1.392503in}{2.114794in}}%
\pgfpathlineto{\pgfqpoint{1.390309in}{2.112168in}}%
\pgfpathlineto{\pgfqpoint{1.389368in}{2.111042in}}%
\pgfpathlineto{\pgfqpoint{1.389368in}{2.107291in}}%
\pgfpathlineto{\pgfqpoint{1.389368in}{2.103539in}}%
\pgfpathlineto{\pgfqpoint{1.389368in}{2.099787in}}%
\pgfpathlineto{\pgfqpoint{1.387174in}{2.097161in}}%
\pgfpathlineto{\pgfqpoint{1.386234in}{2.096036in}}%
\pgfpathlineto{\pgfqpoint{1.386234in}{2.092284in}}%
\pgfpathlineto{\pgfqpoint{1.386234in}{2.088533in}}%
\pgfpathlineto{\pgfqpoint{1.384039in}{2.085907in}}%
\pgfpathlineto{\pgfqpoint{1.383099in}{2.084781in}}%
\pgfpathlineto{\pgfqpoint{1.383099in}{2.081030in}}%
\pgfpathlineto{\pgfqpoint{1.383099in}{2.077278in}}%
\pgfpathlineto{\pgfqpoint{1.383099in}{2.073527in}}%
\pgfpathlineto{\pgfqpoint{1.380905in}{2.070901in}}%
\pgfpathlineto{\pgfqpoint{1.379964in}{2.069775in}}%
\pgfpathlineto{\pgfqpoint{1.379964in}{2.066024in}}%
\pgfpathlineto{\pgfqpoint{1.379964in}{2.062272in}}%
\pgfpathlineto{\pgfqpoint{1.377770in}{2.059646in}}%
\pgfpathlineto{\pgfqpoint{1.376829in}{2.058521in}}%
\pgfpathlineto{\pgfqpoint{1.376829in}{2.054769in}}%
\pgfpathlineto{\pgfqpoint{1.376829in}{2.051018in}}%
\pgfpathlineto{\pgfqpoint{1.376829in}{2.047266in}}%
\pgfpathlineto{\pgfqpoint{1.374635in}{2.044640in}}%
\pgfpathlineto{\pgfqpoint{1.373695in}{2.043514in}}%
\pgfpathlineto{\pgfqpoint{1.373695in}{2.039763in}}%
\pgfpathlineto{\pgfqpoint{1.373695in}{2.036011in}}%
\pgfpathlineto{\pgfqpoint{1.371500in}{2.033385in}}%
\pgfpathlineto{\pgfqpoint{1.370560in}{2.032260in}}%
\pgfpathlineto{\pgfqpoint{1.370560in}{2.028508in}}%
\pgfpathlineto{\pgfqpoint{1.370560in}{2.024757in}}%
\pgfpathlineto{\pgfqpoint{1.368366in}{2.022131in}}%
\pgfpathlineto{\pgfqpoint{1.367425in}{2.021005in}}%
\pgfpathlineto{\pgfqpoint{1.367425in}{2.017254in}}%
\pgfpathlineto{\pgfqpoint{1.367425in}{2.013502in}}%
\pgfpathlineto{\pgfqpoint{1.367425in}{2.009751in}}%
\pgfpathlineto{\pgfqpoint{1.365231in}{2.007125in}}%
\pgfpathlineto{\pgfqpoint{1.364290in}{2.005999in}}%
\pgfpathlineto{\pgfqpoint{1.364290in}{2.002248in}}%
\pgfpathlineto{\pgfqpoint{1.364290in}{1.998496in}}%
\pgfpathlineto{\pgfqpoint{1.362096in}{1.995870in}}%
\pgfpathlineto{\pgfqpoint{1.361156in}{1.994745in}}%
\pgfpathlineto{\pgfqpoint{1.361156in}{1.990993in}}%
\pgfpathlineto{\pgfqpoint{1.361156in}{1.987241in}}%
\pgfpathlineto{\pgfqpoint{1.361156in}{1.983490in}}%
\pgfpathlineto{\pgfqpoint{1.358961in}{1.980864in}}%
\pgfpathlineto{\pgfqpoint{1.358021in}{1.979738in}}%
\pgfpathlineto{\pgfqpoint{1.358021in}{1.975987in}}%
\pgfpathlineto{\pgfqpoint{1.358021in}{1.972235in}}%
\pgfpathlineto{\pgfqpoint{1.355827in}{1.969609in}}%
\pgfpathlineto{\pgfqpoint{1.354886in}{1.968484in}}%
\pgfpathlineto{\pgfqpoint{1.354886in}{1.964732in}}%
\pgfpathlineto{\pgfqpoint{1.354886in}{1.960981in}}%
\pgfpathlineto{\pgfqpoint{1.354886in}{1.957229in}}%
\pgfpathlineto{\pgfqpoint{1.352692in}{1.954603in}}%
\pgfpathlineto{\pgfqpoint{1.351751in}{1.953478in}}%
\pgfpathlineto{\pgfqpoint{1.351751in}{1.949726in}}%
\pgfpathlineto{\pgfqpoint{1.351751in}{1.945975in}}%
\pgfpathlineto{\pgfqpoint{1.349557in}{1.943348in}}%
\pgfpathlineto{\pgfqpoint{1.348617in}{1.942223in}}%
\pgfpathlineto{\pgfqpoint{1.348617in}{1.938471in}}%
\pgfpathlineto{\pgfqpoint{1.348617in}{1.934720in}}%
\pgfpathlineto{\pgfqpoint{1.348617in}{1.930968in}}%
\pgfpathlineto{\pgfqpoint{1.346422in}{1.928342in}}%
\pgfpathlineto{\pgfqpoint{1.345482in}{1.927217in}}%
\pgfpathlineto{\pgfqpoint{1.345482in}{1.923465in}}%
\pgfpathlineto{\pgfqpoint{1.345482in}{1.919714in}}%
\pgfpathlineto{\pgfqpoint{1.343288in}{1.917088in}}%
\pgfpathlineto{\pgfqpoint{1.342347in}{1.915962in}}%
\pgfpathlineto{\pgfqpoint{1.342347in}{1.912211in}}%
\pgfpathlineto{\pgfqpoint{1.342347in}{1.908459in}}%
\pgfpathlineto{\pgfqpoint{1.342347in}{1.904708in}}%
\pgfpathlineto{\pgfqpoint{1.340153in}{1.902082in}}%
\pgfpathlineto{\pgfqpoint{1.339212in}{1.900956in}}%
\pgfpathlineto{\pgfqpoint{1.339212in}{1.897205in}}%
\pgfpathlineto{\pgfqpoint{1.339212in}{1.893453in}}%
\pgfpathlineto{\pgfqpoint{1.337018in}{1.890827in}}%
\pgfpathlineto{\pgfqpoint{1.336078in}{1.889702in}}%
\pgfpathlineto{\pgfqpoint{1.336078in}{1.885950in}}%
\pgfpathlineto{\pgfqpoint{1.336078in}{1.882198in}}%
\pgfpathlineto{\pgfqpoint{1.336078in}{1.878447in}}%
\pgfpathlineto{\pgfqpoint{1.333883in}{1.875821in}}%
\pgfpathlineto{\pgfqpoint{1.332943in}{1.874695in}}%
\pgfpathlineto{\pgfqpoint{1.332943in}{1.870944in}}%
\pgfpathlineto{\pgfqpoint{1.332943in}{1.867192in}}%
\pgfpathlineto{\pgfqpoint{1.330749in}{1.864566in}}%
\pgfpathlineto{\pgfqpoint{1.329808in}{1.863441in}}%
\pgfpathlineto{\pgfqpoint{1.329808in}{1.859689in}}%
\pgfpathlineto{\pgfqpoint{1.329808in}{1.855938in}}%
\pgfpathlineto{\pgfqpoint{1.329808in}{1.852186in}}%
\pgfpathlineto{\pgfqpoint{1.327614in}{1.849560in}}%
\pgfpathlineto{\pgfqpoint{1.326673in}{1.848435in}}%
\pgfpathlineto{\pgfqpoint{1.326673in}{1.844683in}}%
\pgfpathlineto{\pgfqpoint{1.326673in}{1.840932in}}%
\pgfpathlineto{\pgfqpoint{1.324479in}{1.838306in}}%
\pgfpathlineto{\pgfqpoint{1.323539in}{1.837180in}}%
\pgfpathlineto{\pgfqpoint{1.323539in}{1.833429in}}%
\pgfpathlineto{\pgfqpoint{1.323539in}{1.829677in}}%
\pgfpathlineto{\pgfqpoint{1.323539in}{1.825925in}}%
\pgfpathlineto{\pgfqpoint{1.321344in}{1.823299in}}%
\pgfpathlineto{\pgfqpoint{1.320404in}{1.822174in}}%
\pgfpathlineto{\pgfqpoint{1.320404in}{1.818422in}}%
\pgfpathlineto{\pgfqpoint{1.320404in}{1.814671in}}%
\pgfpathlineto{\pgfqpoint{1.318210in}{1.812045in}}%
\pgfpathlineto{\pgfqpoint{1.317269in}{1.810919in}}%
\pgfpathlineto{\pgfqpoint{1.317269in}{1.807168in}}%
\pgfpathlineto{\pgfqpoint{1.317269in}{1.803416in}}%
\pgfpathlineto{\pgfqpoint{1.315075in}{1.800790in}}%
\pgfpathlineto{\pgfqpoint{1.314135in}{1.799665in}}%
\pgfpathlineto{\pgfqpoint{1.314135in}{1.795913in}}%
\pgfpathlineto{\pgfqpoint{1.314135in}{1.792162in}}%
\pgfpathlineto{\pgfqpoint{1.314135in}{1.788410in}}%
\pgfpathlineto{\pgfqpoint{1.311940in}{1.785784in}}%
\pgfpathlineto{\pgfqpoint{1.311000in}{1.784659in}}%
\pgfpathlineto{\pgfqpoint{1.311000in}{1.780907in}}%
\pgfpathlineto{\pgfqpoint{1.311000in}{1.777155in}}%
\pgfpathlineto{\pgfqpoint{1.308805in}{1.774529in}}%
\pgfpathlineto{\pgfqpoint{1.307865in}{1.773404in}}%
\pgfpathlineto{\pgfqpoint{1.307865in}{1.769652in}}%
\pgfpathlineto{\pgfqpoint{1.307865in}{1.765901in}}%
\pgfpathlineto{\pgfqpoint{1.307865in}{1.762149in}}%
\pgfpathlineto{\pgfqpoint{1.305671in}{1.759523in}}%
\pgfpathlineto{\pgfqpoint{1.304730in}{1.758398in}}%
\pgfpathlineto{\pgfqpoint{1.304730in}{1.754646in}}%
\pgfpathlineto{\pgfqpoint{1.304730in}{1.750895in}}%
\pgfpathlineto{\pgfqpoint{1.302536in}{1.748269in}}%
\pgfpathlineto{\pgfqpoint{1.301596in}{1.747143in}}%
\pgfpathlineto{\pgfqpoint{1.301596in}{1.743392in}}%
\pgfpathlineto{\pgfqpoint{1.301596in}{1.739640in}}%
\pgfpathlineto{\pgfqpoint{1.301596in}{1.735889in}}%
\pgfpathlineto{\pgfqpoint{1.299401in}{1.733263in}}%
\pgfpathlineto{\pgfqpoint{1.298461in}{1.732137in}}%
\pgfpathlineto{\pgfqpoint{1.298461in}{1.728386in}}%
\pgfpathlineto{\pgfqpoint{1.298461in}{1.724634in}}%
\pgfpathlineto{\pgfqpoint{1.296266in}{1.722008in}}%
\pgfpathlineto{\pgfqpoint{1.295326in}{1.720882in}}%
\pgfpathlineto{\pgfqpoint{1.295326in}{1.717131in}}%
\pgfpathlineto{\pgfqpoint{1.295326in}{1.713379in}}%
\pgfpathlineto{\pgfqpoint{1.295326in}{1.709628in}}%
\pgfpathlineto{\pgfqpoint{1.293132in}{1.707002in}}%
\pgfpathlineto{\pgfqpoint{1.292191in}{1.705876in}}%
\pgfpathlineto{\pgfqpoint{1.292191in}{1.702125in}}%
\pgfpathlineto{\pgfqpoint{1.293132in}{1.700999in}}%
\pgfpathlineto{\pgfqpoint{1.295326in}{1.698373in}}%
\pgfpathlineto{\pgfqpoint{1.296266in}{1.697248in}}%
\pgfpathlineto{\pgfqpoint{1.298461in}{1.694622in}}%
\pgfpathlineto{\pgfqpoint{1.299401in}{1.693496in}}%
\pgfpathlineto{\pgfqpoint{1.302536in}{1.693496in}}%
\pgfpathlineto{\pgfqpoint{1.304730in}{1.690870in}}%
\pgfpathlineto{\pgfqpoint{1.305671in}{1.689745in}}%
\pgfpathlineto{\pgfqpoint{1.307865in}{1.687119in}}%
\pgfpathlineto{\pgfqpoint{1.308805in}{1.685993in}}%
\pgfpathlineto{\pgfqpoint{1.311000in}{1.683367in}}%
\pgfpathlineto{\pgfqpoint{1.311940in}{1.682242in}}%
\pgfpathlineto{\pgfqpoint{1.315075in}{1.682242in}}%
\pgfpathlineto{\pgfqpoint{1.317269in}{1.679616in}}%
\pgfpathlineto{\pgfqpoint{1.318210in}{1.678490in}}%
\pgfpathlineto{\pgfqpoint{1.320404in}{1.675864in}}%
\pgfpathlineto{\pgfqpoint{1.321344in}{1.674739in}}%
\pgfpathlineto{\pgfqpoint{1.323539in}{1.672113in}}%
\pgfpathlineto{\pgfqpoint{1.324479in}{1.670987in}}%
\pgfpathlineto{\pgfqpoint{1.327614in}{1.670987in}}%
\pgfpathlineto{\pgfqpoint{1.329808in}{1.668361in}}%
\pgfpathlineto{\pgfqpoint{1.330749in}{1.667236in}}%
\pgfpathlineto{\pgfqpoint{1.332943in}{1.664609in}}%
\pgfpathlineto{\pgfqpoint{1.333883in}{1.663484in}}%
\pgfpathlineto{\pgfqpoint{1.336078in}{1.660858in}}%
\pgfpathlineto{\pgfqpoint{1.337018in}{1.659732in}}%
\pgfpathlineto{\pgfqpoint{1.339212in}{1.657106in}}%
\pgfpathlineto{\pgfqpoint{1.340153in}{1.655981in}}%
\pgfpathlineto{\pgfqpoint{1.343288in}{1.655981in}}%
\pgfpathlineto{\pgfqpoint{1.345482in}{1.653355in}}%
\pgfpathlineto{\pgfqpoint{1.346422in}{1.652229in}}%
\pgfpathlineto{\pgfqpoint{1.348617in}{1.649603in}}%
\pgfpathlineto{\pgfqpoint{1.349557in}{1.648478in}}%
\pgfpathlineto{\pgfqpoint{1.351751in}{1.645852in}}%
\pgfpathlineto{\pgfqpoint{1.352692in}{1.644726in}}%
\pgfpathlineto{\pgfqpoint{1.355827in}{1.644726in}}%
\pgfpathlineto{\pgfqpoint{1.358021in}{1.642100in}}%
\pgfpathlineto{\pgfqpoint{1.358961in}{1.640975in}}%
\pgfpathlineto{\pgfqpoint{1.361156in}{1.638349in}}%
\pgfpathlineto{\pgfqpoint{1.362096in}{1.637223in}}%
\pgfpathlineto{\pgfqpoint{1.364290in}{1.634597in}}%
\pgfpathlineto{\pgfqpoint{1.365231in}{1.633472in}}%
\pgfpathlineto{\pgfqpoint{1.368366in}{1.633472in}}%
\pgfpathlineto{\pgfqpoint{1.370560in}{1.630846in}}%
\pgfpathlineto{\pgfqpoint{1.371500in}{1.629720in}}%
\pgfpathlineto{\pgfqpoint{1.373695in}{1.627094in}}%
\pgfpathlineto{\pgfqpoint{1.374635in}{1.625969in}}%
\pgfpathlineto{\pgfqpoint{1.376829in}{1.623343in}}%
\pgfpathlineto{\pgfqpoint{1.377770in}{1.622217in}}%
\pgfpathlineto{\pgfqpoint{1.380905in}{1.622217in}}%
\pgfpathlineto{\pgfqpoint{1.383099in}{1.619591in}}%
\pgfpathlineto{\pgfqpoint{1.384039in}{1.618466in}}%
\pgfpathlineto{\pgfqpoint{1.386234in}{1.615840in}}%
\pgfpathlineto{\pgfqpoint{1.387174in}{1.614714in}}%
\pgfpathlineto{\pgfqpoint{1.389368in}{1.612088in}}%
\pgfpathlineto{\pgfqpoint{1.390309in}{1.610963in}}%
\pgfpathlineto{\pgfqpoint{1.393444in}{1.610963in}}%
\pgfpathlineto{\pgfqpoint{1.395638in}{1.608336in}}%
\pgfpathlineto{\pgfqpoint{1.396578in}{1.607211in}}%
\pgfpathlineto{\pgfqpoint{1.398773in}{1.604585in}}%
\pgfpathlineto{\pgfqpoint{1.399713in}{1.603459in}}%
\pgfpathlineto{\pgfqpoint{1.401907in}{1.600833in}}%
\pgfpathlineto{\pgfqpoint{1.402848in}{1.599708in}}%
\pgfpathlineto{\pgfqpoint{1.405982in}{1.599708in}}%
\pgfpathlineto{\pgfqpoint{1.408177in}{1.597082in}}%
\pgfpathlineto{\pgfqpoint{1.409117in}{1.595956in}}%
\pgfpathlineto{\pgfqpoint{1.411312in}{1.593330in}}%
\pgfpathlineto{\pgfqpoint{1.412252in}{1.592205in}}%
\pgfpathlineto{\pgfqpoint{1.414446in}{1.589579in}}%
\pgfpathlineto{\pgfqpoint{1.415387in}{1.588453in}}%
\pgfpathlineto{\pgfqpoint{1.417581in}{1.585827in}}%
\pgfpathlineto{\pgfqpoint{1.418521in}{1.584702in}}%
\pgfpathlineto{\pgfqpoint{1.421656in}{1.584702in}}%
\pgfpathlineto{\pgfqpoint{1.423851in}{1.582076in}}%
\pgfpathlineto{\pgfqpoint{1.424791in}{1.580950in}}%
\pgfpathlineto{\pgfqpoint{1.426985in}{1.578324in}}%
\pgfpathlineto{\pgfqpoint{1.427926in}{1.577199in}}%
\pgfpathlineto{\pgfqpoint{1.430120in}{1.574573in}}%
\pgfpathlineto{\pgfqpoint{1.431060in}{1.573447in}}%
\pgfpathlineto{\pgfqpoint{1.434195in}{1.573447in}}%
\pgfpathlineto{\pgfqpoint{1.436389in}{1.570821in}}%
\pgfpathlineto{\pgfqpoint{1.437330in}{1.569696in}}%
\pgfpathlineto{\pgfqpoint{1.439524in}{1.567070in}}%
\pgfpathlineto{\pgfqpoint{1.440465in}{1.565944in}}%
\pgfpathlineto{\pgfqpoint{1.442659in}{1.563318in}}%
\pgfpathlineto{\pgfqpoint{1.443599in}{1.562193in}}%
\pgfpathlineto{\pgfqpoint{1.446734in}{1.562193in}}%
\pgfpathlineto{\pgfqpoint{1.448928in}{1.559566in}}%
\pgfpathlineto{\pgfqpoint{1.449869in}{1.558441in}}%
\pgfpathlineto{\pgfqpoint{1.452063in}{1.555815in}}%
\pgfpathlineto{\pgfqpoint{1.453004in}{1.554690in}}%
\pgfpathlineto{\pgfqpoint{1.455198in}{1.552063in}}%
\pgfpathlineto{\pgfqpoint{1.456138in}{1.550938in}}%
\pgfpathlineto{\pgfqpoint{1.459273in}{1.550938in}}%
\pgfpathlineto{\pgfqpoint{1.461467in}{1.548312in}}%
\pgfpathlineto{\pgfqpoint{1.462408in}{1.547186in}}%
\pgfpathlineto{\pgfqpoint{1.464602in}{1.544560in}}%
\pgfpathlineto{\pgfqpoint{1.465543in}{1.543435in}}%
\pgfpathlineto{\pgfqpoint{1.467737in}{1.540809in}}%
\pgfpathlineto{\pgfqpoint{1.468677in}{1.539683in}}%
\pgfpathlineto{\pgfqpoint{1.471812in}{1.539683in}}%
\pgfpathlineto{\pgfqpoint{1.474006in}{1.537057in}}%
\pgfpathlineto{\pgfqpoint{1.474947in}{1.535932in}}%
\pgfpathlineto{\pgfqpoint{1.477141in}{1.533306in}}%
\pgfpathlineto{\pgfqpoint{1.478082in}{1.532180in}}%
\pgfpathlineto{\pgfqpoint{1.480276in}{1.529554in}}%
\pgfpathlineto{\pgfqpoint{1.481216in}{1.528429in}}%
\pgfpathlineto{\pgfqpoint{1.484351in}{1.528429in}}%
\pgfpathlineto{\pgfqpoint{1.486545in}{1.525803in}}%
\pgfpathlineto{\pgfqpoint{1.487486in}{1.524677in}}%
\pgfpathlineto{\pgfqpoint{1.489680in}{1.522051in}}%
\pgfpathlineto{\pgfqpoint{1.490621in}{1.520926in}}%
\pgfpathlineto{\pgfqpoint{1.492815in}{1.518300in}}%
\pgfpathlineto{\pgfqpoint{1.493755in}{1.517174in}}%
\pgfpathlineto{\pgfqpoint{1.495950in}{1.514548in}}%
\pgfpathlineto{\pgfqpoint{1.496890in}{1.513423in}}%
\pgfpathlineto{\pgfqpoint{1.500025in}{1.513423in}}%
\pgfpathlineto{\pgfqpoint{1.502219in}{1.510797in}}%
\pgfpathlineto{\pgfqpoint{1.503159in}{1.509671in}}%
\pgfpathlineto{\pgfqpoint{1.505354in}{1.507045in}}%
\pgfpathlineto{\pgfqpoint{1.506294in}{1.505920in}}%
\pgfpathlineto{\pgfqpoint{1.508489in}{1.503293in}}%
\pgfpathlineto{\pgfqpoint{1.509429in}{1.502168in}}%
\pgfpathlineto{\pgfqpoint{1.512564in}{1.502168in}}%
\pgfpathlineto{\pgfqpoint{1.514758in}{1.499542in}}%
\pgfpathlineto{\pgfqpoint{1.515698in}{1.498416in}}%
\pgfpathlineto{\pgfqpoint{1.517893in}{1.495790in}}%
\pgfpathlineto{\pgfqpoint{1.518833in}{1.494665in}}%
\pgfpathlineto{\pgfqpoint{1.521028in}{1.492039in}}%
\pgfpathlineto{\pgfqpoint{1.521968in}{1.490913in}}%
\pgfpathlineto{\pgfqpoint{1.525103in}{1.490913in}}%
\pgfpathlineto{\pgfqpoint{1.527297in}{1.488287in}}%
\pgfpathlineto{\pgfqpoint{1.528237in}{1.487162in}}%
\pgfpathlineto{\pgfqpoint{1.530432in}{1.484536in}}%
\pgfpathlineto{\pgfqpoint{1.531372in}{1.483410in}}%
\pgfpathlineto{\pgfqpoint{1.533566in}{1.480784in}}%
\pgfpathlineto{\pgfqpoint{1.534507in}{1.479659in}}%
\pgfpathlineto{\pgfqpoint{1.537642in}{1.479659in}}%
\pgfpathlineto{\pgfqpoint{1.539836in}{1.477033in}}%
\pgfpathlineto{\pgfqpoint{1.540776in}{1.475907in}}%
\pgfpathlineto{\pgfqpoint{1.542971in}{1.473281in}}%
\pgfpathlineto{\pgfqpoint{1.543911in}{1.472156in}}%
\pgfpathlineto{\pgfqpoint{1.546105in}{1.469530in}}%
\pgfpathlineto{\pgfqpoint{1.547046in}{1.468404in}}%
\pgfpathlineto{\pgfqpoint{1.550181in}{1.468404in}}%
\pgfpathlineto{\pgfqpoint{1.552375in}{1.465778in}}%
\pgfpathlineto{\pgfqpoint{1.553315in}{1.464653in}}%
\pgfpathlineto{\pgfqpoint{1.555510in}{1.462027in}}%
\pgfpathlineto{\pgfqpoint{1.556450in}{1.460901in}}%
\pgfpathlineto{\pgfqpoint{1.558644in}{1.458275in}}%
\pgfpathlineto{\pgfqpoint{1.559585in}{1.457150in}}%
\pgfpathlineto{\pgfqpoint{1.561779in}{1.454524in}}%
\pgfpathlineto{\pgfqpoint{1.562720in}{1.453398in}}%
\pgfpathlineto{\pgfqpoint{1.565854in}{1.453398in}}%
\pgfpathlineto{\pgfqpoint{1.568049in}{1.450772in}}%
\pgfpathlineto{\pgfqpoint{1.568989in}{1.449647in}}%
\pgfpathlineto{\pgfqpoint{1.571183in}{1.447020in}}%
\pgfpathlineto{\pgfqpoint{1.572124in}{1.445895in}}%
\pgfpathlineto{\pgfqpoint{1.574318in}{1.443269in}}%
\pgfpathlineto{\pgfqpoint{1.575259in}{1.442143in}}%
\pgfpathlineto{\pgfqpoint{1.578393in}{1.442143in}}%
\pgfpathlineto{\pgfqpoint{1.580588in}{1.439517in}}%
\pgfpathlineto{\pgfqpoint{1.581528in}{1.438392in}}%
\pgfpathlineto{\pgfqpoint{1.583722in}{1.435766in}}%
\pgfpathlineto{\pgfqpoint{1.584663in}{1.434640in}}%
\pgfpathlineto{\pgfqpoint{1.586857in}{1.432014in}}%
\pgfpathlineto{\pgfqpoint{1.587798in}{1.430889in}}%
\pgfpathlineto{\pgfqpoint{1.590932in}{1.430889in}}%
\pgfpathlineto{\pgfqpoint{1.593127in}{1.428263in}}%
\pgfpathlineto{\pgfqpoint{1.594067in}{1.427137in}}%
\pgfpathlineto{\pgfqpoint{1.596261in}{1.424511in}}%
\pgfpathlineto{\pgfqpoint{1.597202in}{1.423386in}}%
\pgfpathlineto{\pgfqpoint{1.599396in}{1.420760in}}%
\pgfpathlineto{\pgfqpoint{1.600337in}{1.419634in}}%
\pgfpathlineto{\pgfqpoint{1.603471in}{1.419634in}}%
\pgfpathlineto{\pgfqpoint{1.605666in}{1.417008in}}%
\pgfpathlineto{\pgfqpoint{1.606606in}{1.415883in}}%
\pgfpathlineto{\pgfqpoint{1.608800in}{1.413257in}}%
\pgfpathlineto{\pgfqpoint{1.609741in}{1.412131in}}%
\pgfpathlineto{\pgfqpoint{1.611935in}{1.409505in}}%
\pgfpathlineto{\pgfqpoint{1.612875in}{1.408380in}}%
\pgfpathlineto{\pgfqpoint{1.616010in}{1.408380in}}%
\pgfpathlineto{\pgfqpoint{1.618205in}{1.405754in}}%
\pgfpathlineto{\pgfqpoint{1.619145in}{1.404628in}}%
\pgfpathlineto{\pgfqpoint{1.621339in}{1.402002in}}%
\pgfpathlineto{\pgfqpoint{1.622280in}{1.400877in}}%
\pgfpathlineto{\pgfqpoint{1.624474in}{1.398251in}}%
\pgfpathlineto{\pgfqpoint{1.625414in}{1.397125in}}%
\pgfpathlineto{\pgfqpoint{1.628549in}{1.397125in}}%
\pgfpathlineto{\pgfqpoint{1.630744in}{1.394499in}}%
\pgfpathlineto{\pgfqpoint{1.631684in}{1.393374in}}%
\pgfpathlineto{\pgfqpoint{1.633878in}{1.390747in}}%
\pgfpathlineto{\pgfqpoint{1.634819in}{1.389622in}}%
\pgfpathlineto{\pgfqpoint{1.637013in}{1.386996in}}%
\pgfpathlineto{\pgfqpoint{1.637953in}{1.385870in}}%
\pgfpathlineto{\pgfqpoint{1.640148in}{1.383244in}}%
\pgfpathlineto{\pgfqpoint{1.641088in}{1.382119in}}%
\pgfpathlineto{\pgfqpoint{1.644223in}{1.382119in}}%
\pgfpathlineto{\pgfqpoint{1.646417in}{1.379493in}}%
\pgfpathlineto{\pgfqpoint{1.647358in}{1.378367in}}%
\pgfpathlineto{\pgfqpoint{1.649552in}{1.375741in}}%
\pgfpathlineto{\pgfqpoint{1.650492in}{1.374616in}}%
\pgfpathlineto{\pgfqpoint{1.652687in}{1.371990in}}%
\pgfpathlineto{\pgfqpoint{1.653627in}{1.370864in}}%
\pgfpathlineto{\pgfqpoint{1.656762in}{1.370864in}}%
\pgfpathlineto{\pgfqpoint{1.658956in}{1.368238in}}%
\pgfpathlineto{\pgfqpoint{1.659897in}{1.367113in}}%
\pgfpathlineto{\pgfqpoint{1.662091in}{1.364487in}}%
\pgfpathlineto{\pgfqpoint{1.663031in}{1.363361in}}%
\pgfpathlineto{\pgfqpoint{1.665226in}{1.360735in}}%
\pgfpathlineto{\pgfqpoint{1.666166in}{1.359610in}}%
\pgfpathlineto{\pgfqpoint{1.669301in}{1.359610in}}%
\pgfpathlineto{\pgfqpoint{1.671495in}{1.356984in}}%
\pgfpathlineto{\pgfqpoint{1.672436in}{1.355858in}}%
\pgfpathlineto{\pgfqpoint{1.674630in}{1.353232in}}%
\pgfpathlineto{\pgfqpoint{1.675570in}{1.352107in}}%
\pgfpathlineto{\pgfqpoint{1.677765in}{1.349481in}}%
\pgfpathlineto{\pgfqpoint{1.678705in}{1.348355in}}%
\pgfpathlineto{\pgfqpoint{1.681840in}{1.348355in}}%
\pgfpathlineto{\pgfqpoint{1.684034in}{1.345729in}}%
\pgfpathlineto{\pgfqpoint{1.684975in}{1.344604in}}%
\pgfpathlineto{\pgfqpoint{1.687169in}{1.341977in}}%
\pgfpathlineto{\pgfqpoint{1.688109in}{1.340852in}}%
\pgfpathlineto{\pgfqpoint{1.690304in}{1.338226in}}%
\pgfpathlineto{\pgfqpoint{1.691244in}{1.337100in}}%
\pgfpathlineto{\pgfqpoint{1.694379in}{1.337100in}}%
\pgfpathlineto{\pgfqpoint{1.696573in}{1.334474in}}%
\pgfpathlineto{\pgfqpoint{1.697514in}{1.333349in}}%
\pgfpathlineto{\pgfqpoint{1.699708in}{1.330723in}}%
\pgfpathlineto{\pgfqpoint{1.700648in}{1.329597in}}%
\pgfpathlineto{\pgfqpoint{1.702843in}{1.326971in}}%
\pgfpathlineto{\pgfqpoint{1.703783in}{1.325846in}}%
\pgfpathlineto{\pgfqpoint{1.706918in}{1.325846in}}%
\pgfpathlineto{\pgfqpoint{1.709112in}{1.323220in}}%
\pgfpathlineto{\pgfqpoint{1.710052in}{1.322094in}}%
\pgfpathlineto{\pgfqpoint{1.712247in}{1.319468in}}%
\pgfpathlineto{\pgfqpoint{1.713187in}{1.318343in}}%
\pgfpathlineto{\pgfqpoint{1.715382in}{1.315717in}}%
\pgfpathlineto{\pgfqpoint{1.716322in}{1.314591in}}%
\pgfpathlineto{\pgfqpoint{1.718516in}{1.311965in}}%
\pgfpathlineto{\pgfqpoint{1.719457in}{1.310840in}}%
\pgfpathlineto{\pgfqpoint{1.722591in}{1.310840in}}%
\pgfpathlineto{\pgfqpoint{1.724786in}{1.308214in}}%
\pgfpathlineto{\pgfqpoint{1.725726in}{1.307088in}}%
\pgfpathlineto{\pgfqpoint{1.727921in}{1.304462in}}%
\pgfpathlineto{\pgfqpoint{1.728861in}{1.303337in}}%
\pgfpathlineto{\pgfqpoint{1.731055in}{1.300711in}}%
\pgfpathlineto{\pgfqpoint{1.731996in}{1.299585in}}%
\pgfpathlineto{\pgfqpoint{1.735130in}{1.299585in}}%
\pgfpathlineto{\pgfqpoint{1.737325in}{1.296959in}}%
\pgfpathlineto{\pgfqpoint{1.738265in}{1.295834in}}%
\pgfpathlineto{\pgfqpoint{1.740459in}{1.293208in}}%
\pgfpathlineto{\pgfqpoint{1.741400in}{1.292082in}}%
\pgfpathlineto{\pgfqpoint{1.743594in}{1.289456in}}%
\pgfpathlineto{\pgfqpoint{1.744535in}{1.288331in}}%
\pgfpathlineto{\pgfqpoint{1.747669in}{1.288331in}}%
\pgfpathlineto{\pgfqpoint{1.749864in}{1.285704in}}%
\pgfpathlineto{\pgfqpoint{1.750804in}{1.284579in}}%
\pgfpathlineto{\pgfqpoint{1.752998in}{1.281953in}}%
\pgfpathlineto{\pgfqpoint{1.753939in}{1.280827in}}%
\pgfpathlineto{\pgfqpoint{1.756133in}{1.278201in}}%
\pgfpathlineto{\pgfqpoint{1.757074in}{1.277076in}}%
\pgfpathlineto{\pgfqpoint{1.760208in}{1.277076in}}%
\pgfpathlineto{\pgfqpoint{1.762403in}{1.274450in}}%
\pgfpathlineto{\pgfqpoint{1.763343in}{1.273324in}}%
\pgfpathlineto{\pgfqpoint{1.765537in}{1.270698in}}%
\pgfpathlineto{\pgfqpoint{1.766478in}{1.269573in}}%
\pgfpathlineto{\pgfqpoint{1.768672in}{1.266947in}}%
\pgfpathlineto{\pgfqpoint{1.769613in}{1.265821in}}%
\pgfpathlineto{\pgfqpoint{1.772747in}{1.265821in}}%
\pgfpathlineto{\pgfqpoint{1.774942in}{1.263195in}}%
\pgfpathlineto{\pgfqpoint{1.775882in}{1.262070in}}%
\pgfpathlineto{\pgfqpoint{1.778076in}{1.259444in}}%
\pgfpathlineto{\pgfqpoint{1.779017in}{1.258318in}}%
\pgfpathlineto{\pgfqpoint{1.781211in}{1.255692in}}%
\pgfpathlineto{\pgfqpoint{1.782152in}{1.254567in}}%
\pgfpathlineto{\pgfqpoint{1.785286in}{1.254567in}}%
\pgfpathlineto{\pgfqpoint{1.787481in}{1.251941in}}%
\pgfpathlineto{\pgfqpoint{1.788421in}{1.250815in}}%
\pgfpathlineto{\pgfqpoint{1.790615in}{1.248189in}}%
\pgfpathlineto{\pgfqpoint{1.791556in}{1.247064in}}%
\pgfpathlineto{\pgfqpoint{1.793750in}{1.244438in}}%
\pgfpathlineto{\pgfqpoint{1.794691in}{1.243312in}}%
\pgfpathlineto{\pgfqpoint{1.796885in}{1.240686in}}%
\pgfpathlineto{\pgfqpoint{1.797825in}{1.239561in}}%
\pgfpathlineto{\pgfqpoint{1.800960in}{1.239561in}}%
\pgfpathlineto{\pgfqpoint{1.803154in}{1.236935in}}%
\pgfpathlineto{\pgfqpoint{1.804095in}{1.235809in}}%
\pgfpathlineto{\pgfqpoint{1.806289in}{1.233183in}}%
\pgfpathlineto{\pgfqpoint{1.807229in}{1.232058in}}%
\pgfpathlineto{\pgfqpoint{1.809424in}{1.229431in}}%
\pgfpathlineto{\pgfqpoint{1.810364in}{1.228306in}}%
\pgfpathlineto{\pgfqpoint{1.813499in}{1.228306in}}%
\pgfpathlineto{\pgfqpoint{1.815693in}{1.225680in}}%
\pgfpathlineto{\pgfqpoint{1.816634in}{1.224554in}}%
\pgfpathlineto{\pgfqpoint{1.818828in}{1.221928in}}%
\pgfpathlineto{\pgfqpoint{1.819768in}{1.220803in}}%
\pgfpathlineto{\pgfqpoint{1.821963in}{1.218177in}}%
\pgfpathlineto{\pgfqpoint{1.822903in}{1.217051in}}%
\pgfpathlineto{\pgfqpoint{1.826038in}{1.217051in}}%
\pgfpathlineto{\pgfqpoint{1.828232in}{1.214425in}}%
\pgfpathlineto{\pgfqpoint{1.829173in}{1.213300in}}%
\pgfpathlineto{\pgfqpoint{1.831367in}{1.210674in}}%
\pgfpathlineto{\pgfqpoint{1.832307in}{1.209548in}}%
\pgfpathlineto{\pgfqpoint{1.834502in}{1.206922in}}%
\pgfpathlineto{\pgfqpoint{1.835442in}{1.205797in}}%
\pgfpathlineto{\pgfqpoint{1.838577in}{1.205797in}}%
\pgfpathlineto{\pgfqpoint{1.840771in}{1.203171in}}%
\pgfpathlineto{\pgfqpoint{1.841712in}{1.202045in}}%
\pgfpathlineto{\pgfqpoint{1.843906in}{1.199419in}}%
\pgfpathlineto{\pgfqpoint{1.844846in}{1.198294in}}%
\pgfpathlineto{\pgfqpoint{1.847041in}{1.195668in}}%
\pgfpathlineto{\pgfqpoint{1.847981in}{1.194542in}}%
\pgfpathlineto{\pgfqpoint{1.851116in}{1.194542in}}%
\pgfpathlineto{\pgfqpoint{1.853310in}{1.191916in}}%
\pgfpathlineto{\pgfqpoint{1.854251in}{1.190791in}}%
\pgfpathlineto{\pgfqpoint{1.856445in}{1.188165in}}%
\pgfpathlineto{\pgfqpoint{1.857385in}{1.187039in}}%
\pgfpathlineto{\pgfqpoint{1.859580in}{1.184413in}}%
\pgfpathlineto{\pgfqpoint{1.860520in}{1.183288in}}%
\pgfpathlineto{\pgfqpoint{1.862714in}{1.180662in}}%
\pgfpathlineto{\pgfqpoint{1.863655in}{1.179536in}}%
\pgfpathlineto{\pgfqpoint{1.866790in}{1.179536in}}%
\pgfpathlineto{\pgfqpoint{1.868984in}{1.176910in}}%
\pgfpathlineto{\pgfqpoint{1.869924in}{1.175785in}}%
\pgfpathlineto{\pgfqpoint{1.872119in}{1.173158in}}%
\pgfpathlineto{\pgfqpoint{1.873059in}{1.172033in}}%
\pgfpathlineto{\pgfqpoint{1.875253in}{1.169407in}}%
\pgfpathlineto{\pgfqpoint{1.876194in}{1.168281in}}%
\pgfpathlineto{\pgfqpoint{1.879329in}{1.168281in}}%
\pgfpathlineto{\pgfqpoint{1.881523in}{1.165655in}}%
\pgfpathlineto{\pgfqpoint{1.882463in}{1.164530in}}%
\pgfpathlineto{\pgfqpoint{1.884658in}{1.161904in}}%
\pgfpathlineto{\pgfqpoint{1.885598in}{1.160778in}}%
\pgfpathlineto{\pgfqpoint{1.887792in}{1.158152in}}%
\pgfpathlineto{\pgfqpoint{1.888733in}{1.157027in}}%
\pgfpathlineto{\pgfqpoint{1.891868in}{1.157027in}}%
\pgfpathlineto{\pgfqpoint{1.894062in}{1.154401in}}%
\pgfpathlineto{\pgfqpoint{1.895002in}{1.153275in}}%
\pgfpathlineto{\pgfqpoint{1.897197in}{1.150649in}}%
\pgfpathlineto{\pgfqpoint{1.898137in}{1.149524in}}%
\pgfpathlineto{\pgfqpoint{1.900331in}{1.146898in}}%
\pgfpathlineto{\pgfqpoint{1.901272in}{1.145772in}}%
\pgfpathlineto{\pgfqpoint{1.904407in}{1.145772in}}%
\pgfpathlineto{\pgfqpoint{1.906601in}{1.143146in}}%
\pgfpathlineto{\pgfqpoint{1.907541in}{1.142021in}}%
\pgfpathlineto{\pgfqpoint{1.909736in}{1.139395in}}%
\pgfpathlineto{\pgfqpoint{1.910676in}{1.138269in}}%
\pgfpathlineto{\pgfqpoint{1.912870in}{1.135643in}}%
\pgfpathlineto{\pgfqpoint{1.913811in}{1.134518in}}%
\pgfpathlineto{\pgfqpoint{1.916945in}{1.134518in}}%
\pgfpathlineto{\pgfqpoint{1.919140in}{1.131892in}}%
\pgfpathlineto{\pgfqpoint{1.920080in}{1.130766in}}%
\pgfpathlineto{\pgfqpoint{1.922275in}{1.128140in}}%
\pgfpathlineto{\pgfqpoint{1.923215in}{1.127015in}}%
\pgfpathlineto{\pgfqpoint{1.925409in}{1.124388in}}%
\pgfpathlineto{\pgfqpoint{1.926350in}{1.123263in}}%
\pgfpathlineto{\pgfqpoint{1.929484in}{1.123263in}}%
\pgfpathlineto{\pgfqpoint{1.931679in}{1.120637in}}%
\pgfpathlineto{\pgfqpoint{1.932619in}{1.119511in}}%
\pgfpathlineto{\pgfqpoint{1.934814in}{1.116885in}}%
\pgfpathlineto{\pgfqpoint{1.935754in}{1.115760in}}%
\pgfpathlineto{\pgfqpoint{1.937948in}{1.113134in}}%
\pgfpathlineto{\pgfqpoint{1.938889in}{1.112008in}}%
\pgfpathlineto{\pgfqpoint{1.941083in}{1.109382in}}%
\pgfpathlineto{\pgfqpoint{1.942023in}{1.108257in}}%
\pgfpathlineto{\pgfqpoint{1.945158in}{1.108257in}}%
\pgfpathlineto{\pgfqpoint{1.947352in}{1.105631in}}%
\pgfpathlineto{\pgfqpoint{1.948293in}{1.104505in}}%
\pgfpathlineto{\pgfqpoint{1.950487in}{1.101879in}}%
\pgfpathlineto{\pgfqpoint{1.951428in}{1.100754in}}%
\pgfpathlineto{\pgfqpoint{1.953622in}{1.098128in}}%
\pgfpathlineto{\pgfqpoint{1.954562in}{1.097002in}}%
\pgfpathlineto{\pgfqpoint{1.957697in}{1.097002in}}%
\pgfpathlineto{\pgfqpoint{1.959891in}{1.094376in}}%
\pgfpathlineto{\pgfqpoint{1.960832in}{1.093251in}}%
\pgfpathlineto{\pgfqpoint{1.963026in}{1.090625in}}%
\pgfpathlineto{\pgfqpoint{1.963967in}{1.089499in}}%
\pgfpathlineto{\pgfqpoint{1.966161in}{1.086873in}}%
\pgfpathlineto{\pgfqpoint{1.967101in}{1.085748in}}%
\pgfpathlineto{\pgfqpoint{1.970236in}{1.085748in}}%
\pgfpathlineto{\pgfqpoint{1.972430in}{1.083122in}}%
\pgfpathlineto{\pgfqpoint{1.973371in}{1.081996in}}%
\pgfpathlineto{\pgfqpoint{1.975565in}{1.079370in}}%
\pgfpathlineto{\pgfqpoint{1.976506in}{1.078245in}}%
\pgfpathlineto{\pgfqpoint{1.978700in}{1.075619in}}%
\pgfpathlineto{\pgfqpoint{1.979640in}{1.074493in}}%
\pgfpathlineto{\pgfqpoint{1.982775in}{1.074493in}}%
\pgfpathlineto{\pgfqpoint{1.984969in}{1.071867in}}%
\pgfpathlineto{\pgfqpoint{1.985910in}{1.070742in}}%
\pgfpathlineto{\pgfqpoint{1.988104in}{1.068115in}}%
\pgfpathlineto{\pgfqpoint{1.989045in}{1.066990in}}%
\pgfpathlineto{\pgfqpoint{1.991239in}{1.064364in}}%
\pgfpathlineto{\pgfqpoint{1.992179in}{1.063238in}}%
\pgfpathlineto{\pgfqpoint{1.995314in}{1.063238in}}%
\pgfpathlineto{\pgfqpoint{1.997508in}{1.060612in}}%
\pgfpathlineto{\pgfqpoint{1.998449in}{1.059487in}}%
\pgfpathlineto{\pgfqpoint{2.000643in}{1.056861in}}%
\pgfpathlineto{\pgfqpoint{2.001584in}{1.055735in}}%
\pgfpathlineto{\pgfqpoint{2.003778in}{1.053109in}}%
\pgfpathlineto{\pgfqpoint{2.004718in}{1.051984in}}%
\pgfpathlineto{\pgfqpoint{2.007853in}{1.051984in}}%
\pgfpathlineto{\pgfqpoint{2.010047in}{1.049358in}}%
\pgfpathlineto{\pgfqpoint{2.010988in}{1.048232in}}%
\pgfpathlineto{\pgfqpoint{2.013182in}{1.045606in}}%
\pgfpathlineto{\pgfqpoint{2.014122in}{1.044481in}}%
\pgfpathlineto{\pgfqpoint{2.016317in}{1.041855in}}%
\pgfpathlineto{\pgfqpoint{2.017257in}{1.040729in}}%
\pgfpathlineto{\pgfqpoint{2.019452in}{1.038103in}}%
\pgfpathlineto{\pgfqpoint{2.020392in}{1.036978in}}%
\pgfpathlineto{\pgfqpoint{2.023527in}{1.036978in}}%
\pgfpathlineto{\pgfqpoint{2.025721in}{1.034352in}}%
\pgfpathlineto{\pgfqpoint{2.026661in}{1.033226in}}%
\pgfpathlineto{\pgfqpoint{2.028856in}{1.030600in}}%
\pgfpathlineto{\pgfqpoint{2.029796in}{1.029475in}}%
\pgfpathlineto{\pgfqpoint{2.031991in}{1.026849in}}%
\pgfpathlineto{\pgfqpoint{2.032931in}{1.025723in}}%
\pgfpathlineto{\pgfqpoint{2.036066in}{1.025723in}}%
\pgfpathlineto{\pgfqpoint{2.038260in}{1.023097in}}%
\pgfpathlineto{\pgfqpoint{2.039200in}{1.021972in}}%
\pgfpathlineto{\pgfqpoint{2.041395in}{1.019346in}}%
\pgfpathlineto{\pgfqpoint{2.042335in}{1.018220in}}%
\pgfpathlineto{\pgfqpoint{2.044529in}{1.015594in}}%
\pgfpathlineto{\pgfqpoint{2.045470in}{1.014469in}}%
\pgfpathlineto{\pgfqpoint{2.048605in}{1.014469in}}%
\pgfpathlineto{\pgfqpoint{2.050799in}{1.011842in}}%
\pgfpathlineto{\pgfqpoint{2.051739in}{1.010717in}}%
\pgfpathlineto{\pgfqpoint{2.053934in}{1.008091in}}%
\pgfpathlineto{\pgfqpoint{2.054874in}{1.006965in}}%
\pgfpathlineto{\pgfqpoint{2.057068in}{1.004339in}}%
\pgfpathlineto{\pgfqpoint{2.058009in}{1.003214in}}%
\pgfpathlineto{\pgfqpoint{2.061144in}{1.003214in}}%
\pgfpathlineto{\pgfqpoint{2.063338in}{1.000588in}}%
\pgfpathlineto{\pgfqpoint{2.064278in}{0.999462in}}%
\pgfpathlineto{\pgfqpoint{2.066473in}{0.996836in}}%
\pgfpathlineto{\pgfqpoint{2.067413in}{0.995711in}}%
\pgfpathlineto{\pgfqpoint{2.069607in}{0.993085in}}%
\pgfpathlineto{\pgfqpoint{2.070548in}{0.991959in}}%
\pgfpathlineto{\pgfqpoint{2.073683in}{0.991959in}}%
\pgfpathlineto{\pgfqpoint{2.075877in}{0.989333in}}%
\pgfpathlineto{\pgfqpoint{2.076817in}{0.988208in}}%
\pgfpathlineto{\pgfqpoint{2.079012in}{0.985582in}}%
\pgfpathlineto{\pgfqpoint{2.079952in}{0.984456in}}%
\pgfpathlineto{\pgfqpoint{2.082146in}{0.981830in}}%
\pgfpathlineto{\pgfqpoint{2.083087in}{0.980705in}}%
\pgfpathlineto{\pgfqpoint{2.086222in}{0.980705in}}%
\pgfpathlineto{\pgfqpoint{2.088416in}{0.978079in}}%
\pgfpathlineto{\pgfqpoint{2.089356in}{0.976953in}}%
\pgfpathlineto{\pgfqpoint{2.091551in}{0.974327in}}%
\pgfpathlineto{\pgfqpoint{2.092491in}{0.973202in}}%
\pgfpathlineto{\pgfqpoint{2.094685in}{0.970576in}}%
\pgfpathlineto{\pgfqpoint{2.095626in}{0.969450in}}%
\pgfpathlineto{\pgfqpoint{2.097820in}{0.966824in}}%
\pgfpathlineto{\pgfqpoint{2.098761in}{0.965699in}}%
\pgfpathlineto{\pgfqpoint{2.101895in}{0.965699in}}%
\pgfpathlineto{\pgfqpoint{2.104090in}{0.963073in}}%
\pgfpathlineto{\pgfqpoint{2.105030in}{0.961947in}}%
\pgfpathlineto{\pgfqpoint{2.107224in}{0.959321in}}%
\pgfpathlineto{\pgfqpoint{2.108165in}{0.958196in}}%
\pgfpathlineto{\pgfqpoint{2.110359in}{0.955569in}}%
\pgfpathlineto{\pgfqpoint{2.111299in}{0.954444in}}%
\pgfpathlineto{\pgfqpoint{2.114434in}{0.954444in}}%
\pgfpathlineto{\pgfqpoint{2.116629in}{0.951818in}}%
\pgfpathlineto{\pgfqpoint{2.117569in}{0.950692in}}%
\pgfpathlineto{\pgfqpoint{2.119763in}{0.948066in}}%
\pgfpathlineto{\pgfqpoint{2.120704in}{0.946941in}}%
\pgfpathlineto{\pgfqpoint{2.122898in}{0.944315in}}%
\pgfpathlineto{\pgfqpoint{2.123838in}{0.943189in}}%
\pgfpathlineto{\pgfqpoint{2.126973in}{0.943189in}}%
\pgfpathlineto{\pgfqpoint{2.129168in}{0.940563in}}%
\pgfpathlineto{\pgfqpoint{2.130108in}{0.939438in}}%
\pgfpathlineto{\pgfqpoint{2.132302in}{0.936812in}}%
\pgfpathlineto{\pgfqpoint{2.133243in}{0.935686in}}%
\pgfpathlineto{\pgfqpoint{2.135437in}{0.933060in}}%
\pgfpathlineto{\pgfqpoint{2.136377in}{0.931935in}}%
\pgfpathlineto{\pgfqpoint{2.139512in}{0.931935in}}%
\pgfpathlineto{\pgfqpoint{2.141706in}{0.929309in}}%
\pgfpathlineto{\pgfqpoint{2.142647in}{0.928183in}}%
\pgfpathlineto{\pgfqpoint{2.144841in}{0.925557in}}%
\pgfpathlineto{\pgfqpoint{2.145782in}{0.924432in}}%
\pgfpathlineto{\pgfqpoint{2.147976in}{0.921806in}}%
\pgfpathlineto{\pgfqpoint{2.148916in}{0.920680in}}%
\pgfpathlineto{\pgfqpoint{2.152051in}{0.920680in}}%
\pgfpathlineto{\pgfqpoint{2.154245in}{0.918054in}}%
\pgfpathlineto{\pgfqpoint{2.155186in}{0.916929in}}%
\pgfpathlineto{\pgfqpoint{2.157380in}{0.914303in}}%
\pgfpathlineto{\pgfqpoint{2.158321in}{0.913177in}}%
\pgfpathlineto{\pgfqpoint{2.160515in}{0.910551in}}%
\pgfpathlineto{\pgfqpoint{2.161455in}{0.909426in}}%
\pgfpathlineto{\pgfqpoint{2.163650in}{0.906799in}}%
\pgfpathlineto{\pgfqpoint{2.164590in}{0.905674in}}%
\pgfpathlineto{\pgfqpoint{2.167725in}{0.905674in}}%
\pgfpathlineto{\pgfqpoint{2.169919in}{0.903048in}}%
\pgfpathlineto{\pgfqpoint{2.170860in}{0.901922in}}%
\pgfpathlineto{\pgfqpoint{2.173054in}{0.899296in}}%
\pgfpathclose%
\pgfpathmoveto{\pgfqpoint{2.173524in}{0.899296in}}%
\pgfpathlineto{\pgfqpoint{2.170860in}{0.902485in}}%
\pgfpathlineto{\pgfqpoint{2.170389in}{0.903048in}}%
\pgfpathlineto{\pgfqpoint{2.167725in}{0.906237in}}%
\pgfpathlineto{\pgfqpoint{2.164590in}{0.906237in}}%
\pgfpathlineto{\pgfqpoint{2.164120in}{0.906799in}}%
\pgfpathlineto{\pgfqpoint{2.161455in}{0.909988in}}%
\pgfpathlineto{\pgfqpoint{2.160985in}{0.910551in}}%
\pgfpathlineto{\pgfqpoint{2.158321in}{0.913740in}}%
\pgfpathlineto{\pgfqpoint{2.157850in}{0.914303in}}%
\pgfpathlineto{\pgfqpoint{2.155186in}{0.917491in}}%
\pgfpathlineto{\pgfqpoint{2.154716in}{0.918054in}}%
\pgfpathlineto{\pgfqpoint{2.152051in}{0.921243in}}%
\pgfpathlineto{\pgfqpoint{2.148916in}{0.921243in}}%
\pgfpathlineto{\pgfqpoint{2.148446in}{0.921806in}}%
\pgfpathlineto{\pgfqpoint{2.145782in}{0.924994in}}%
\pgfpathlineto{\pgfqpoint{2.145311in}{0.925557in}}%
\pgfpathlineto{\pgfqpoint{2.142647in}{0.928746in}}%
\pgfpathlineto{\pgfqpoint{2.142177in}{0.929309in}}%
\pgfpathlineto{\pgfqpoint{2.139512in}{0.932498in}}%
\pgfpathlineto{\pgfqpoint{2.136377in}{0.932498in}}%
\pgfpathlineto{\pgfqpoint{2.135907in}{0.933060in}}%
\pgfpathlineto{\pgfqpoint{2.133243in}{0.936249in}}%
\pgfpathlineto{\pgfqpoint{2.132772in}{0.936812in}}%
\pgfpathlineto{\pgfqpoint{2.130108in}{0.940001in}}%
\pgfpathlineto{\pgfqpoint{2.129638in}{0.940563in}}%
\pgfpathlineto{\pgfqpoint{2.126973in}{0.943752in}}%
\pgfpathlineto{\pgfqpoint{2.123838in}{0.943752in}}%
\pgfpathlineto{\pgfqpoint{2.123368in}{0.944315in}}%
\pgfpathlineto{\pgfqpoint{2.120704in}{0.947504in}}%
\pgfpathlineto{\pgfqpoint{2.120234in}{0.948066in}}%
\pgfpathlineto{\pgfqpoint{2.117569in}{0.951255in}}%
\pgfpathlineto{\pgfqpoint{2.117099in}{0.951818in}}%
\pgfpathlineto{\pgfqpoint{2.114434in}{0.955007in}}%
\pgfpathlineto{\pgfqpoint{2.111299in}{0.955007in}}%
\pgfpathlineto{\pgfqpoint{2.110829in}{0.955569in}}%
\pgfpathlineto{\pgfqpoint{2.108165in}{0.958758in}}%
\pgfpathlineto{\pgfqpoint{2.107695in}{0.959321in}}%
\pgfpathlineto{\pgfqpoint{2.105030in}{0.962510in}}%
\pgfpathlineto{\pgfqpoint{2.104560in}{0.963073in}}%
\pgfpathlineto{\pgfqpoint{2.101895in}{0.966261in}}%
\pgfpathlineto{\pgfqpoint{2.098761in}{0.966261in}}%
\pgfpathlineto{\pgfqpoint{2.098290in}{0.966824in}}%
\pgfpathlineto{\pgfqpoint{2.095626in}{0.970013in}}%
\pgfpathlineto{\pgfqpoint{2.095156in}{0.970576in}}%
\pgfpathlineto{\pgfqpoint{2.092491in}{0.973764in}}%
\pgfpathlineto{\pgfqpoint{2.092021in}{0.974327in}}%
\pgfpathlineto{\pgfqpoint{2.089356in}{0.977516in}}%
\pgfpathlineto{\pgfqpoint{2.088886in}{0.978079in}}%
\pgfpathlineto{\pgfqpoint{2.086222in}{0.981267in}}%
\pgfpathlineto{\pgfqpoint{2.083087in}{0.981267in}}%
\pgfpathlineto{\pgfqpoint{2.082617in}{0.981830in}}%
\pgfpathlineto{\pgfqpoint{2.079952in}{0.985019in}}%
\pgfpathlineto{\pgfqpoint{2.079482in}{0.985582in}}%
\pgfpathlineto{\pgfqpoint{2.076817in}{0.988771in}}%
\pgfpathlineto{\pgfqpoint{2.076347in}{0.989333in}}%
\pgfpathlineto{\pgfqpoint{2.073683in}{0.992522in}}%
\pgfpathlineto{\pgfqpoint{2.070548in}{0.992522in}}%
\pgfpathlineto{\pgfqpoint{2.070078in}{0.993085in}}%
\pgfpathlineto{\pgfqpoint{2.067413in}{0.996274in}}%
\pgfpathlineto{\pgfqpoint{2.066943in}{0.996836in}}%
\pgfpathlineto{\pgfqpoint{2.064278in}{1.000025in}}%
\pgfpathlineto{\pgfqpoint{2.063808in}{1.000588in}}%
\pgfpathlineto{\pgfqpoint{2.061144in}{1.003777in}}%
\pgfpathlineto{\pgfqpoint{2.058009in}{1.003777in}}%
\pgfpathlineto{\pgfqpoint{2.057539in}{1.004339in}}%
\pgfpathlineto{\pgfqpoint{2.054874in}{1.007528in}}%
\pgfpathlineto{\pgfqpoint{2.054404in}{1.008091in}}%
\pgfpathlineto{\pgfqpoint{2.051739in}{1.011280in}}%
\pgfpathlineto{\pgfqpoint{2.051269in}{1.011842in}}%
\pgfpathlineto{\pgfqpoint{2.048605in}{1.015031in}}%
\pgfpathlineto{\pgfqpoint{2.045470in}{1.015031in}}%
\pgfpathlineto{\pgfqpoint{2.045000in}{1.015594in}}%
\pgfpathlineto{\pgfqpoint{2.042335in}{1.018783in}}%
\pgfpathlineto{\pgfqpoint{2.041865in}{1.019346in}}%
\pgfpathlineto{\pgfqpoint{2.039200in}{1.022534in}}%
\pgfpathlineto{\pgfqpoint{2.038730in}{1.023097in}}%
\pgfpathlineto{\pgfqpoint{2.036066in}{1.026286in}}%
\pgfpathlineto{\pgfqpoint{2.032931in}{1.026286in}}%
\pgfpathlineto{\pgfqpoint{2.032461in}{1.026849in}}%
\pgfpathlineto{\pgfqpoint{2.029796in}{1.030037in}}%
\pgfpathlineto{\pgfqpoint{2.029326in}{1.030600in}}%
\pgfpathlineto{\pgfqpoint{2.026661in}{1.033789in}}%
\pgfpathlineto{\pgfqpoint{2.026191in}{1.034352in}}%
\pgfpathlineto{\pgfqpoint{2.023527in}{1.037540in}}%
\pgfpathlineto{\pgfqpoint{2.020392in}{1.037540in}}%
\pgfpathlineto{\pgfqpoint{2.019922in}{1.038103in}}%
\pgfpathlineto{\pgfqpoint{2.017257in}{1.041292in}}%
\pgfpathlineto{\pgfqpoint{2.016787in}{1.041855in}}%
\pgfpathlineto{\pgfqpoint{2.014122in}{1.045044in}}%
\pgfpathlineto{\pgfqpoint{2.013652in}{1.045606in}}%
\pgfpathlineto{\pgfqpoint{2.010988in}{1.048795in}}%
\pgfpathlineto{\pgfqpoint{2.010518in}{1.049358in}}%
\pgfpathlineto{\pgfqpoint{2.007853in}{1.052547in}}%
\pgfpathlineto{\pgfqpoint{2.004718in}{1.052547in}}%
\pgfpathlineto{\pgfqpoint{2.004248in}{1.053109in}}%
\pgfpathlineto{\pgfqpoint{2.001584in}{1.056298in}}%
\pgfpathlineto{\pgfqpoint{2.001113in}{1.056861in}}%
\pgfpathlineto{\pgfqpoint{1.998449in}{1.060050in}}%
\pgfpathlineto{\pgfqpoint{1.997979in}{1.060612in}}%
\pgfpathlineto{\pgfqpoint{1.995314in}{1.063801in}}%
\pgfpathlineto{\pgfqpoint{1.992179in}{1.063801in}}%
\pgfpathlineto{\pgfqpoint{1.991709in}{1.064364in}}%
\pgfpathlineto{\pgfqpoint{1.989045in}{1.067553in}}%
\pgfpathlineto{\pgfqpoint{1.988574in}{1.068115in}}%
\pgfpathlineto{\pgfqpoint{1.985910in}{1.071304in}}%
\pgfpathlineto{\pgfqpoint{1.985440in}{1.071867in}}%
\pgfpathlineto{\pgfqpoint{1.982775in}{1.075056in}}%
\pgfpathlineto{\pgfqpoint{1.979640in}{1.075056in}}%
\pgfpathlineto{\pgfqpoint{1.979170in}{1.075619in}}%
\pgfpathlineto{\pgfqpoint{1.976506in}{1.078807in}}%
\pgfpathlineto{\pgfqpoint{1.976035in}{1.079370in}}%
\pgfpathlineto{\pgfqpoint{1.973371in}{1.082559in}}%
\pgfpathlineto{\pgfqpoint{1.972901in}{1.083122in}}%
\pgfpathlineto{\pgfqpoint{1.970236in}{1.086310in}}%
\pgfpathlineto{\pgfqpoint{1.967101in}{1.086310in}}%
\pgfpathlineto{\pgfqpoint{1.966631in}{1.086873in}}%
\pgfpathlineto{\pgfqpoint{1.963967in}{1.090062in}}%
\pgfpathlineto{\pgfqpoint{1.963496in}{1.090625in}}%
\pgfpathlineto{\pgfqpoint{1.960832in}{1.093813in}}%
\pgfpathlineto{\pgfqpoint{1.960362in}{1.094376in}}%
\pgfpathlineto{\pgfqpoint{1.957697in}{1.097565in}}%
\pgfpathlineto{\pgfqpoint{1.954562in}{1.097565in}}%
\pgfpathlineto{\pgfqpoint{1.954092in}{1.098128in}}%
\pgfpathlineto{\pgfqpoint{1.951428in}{1.101317in}}%
\pgfpathlineto{\pgfqpoint{1.950957in}{1.101879in}}%
\pgfpathlineto{\pgfqpoint{1.948293in}{1.105068in}}%
\pgfpathlineto{\pgfqpoint{1.947823in}{1.105631in}}%
\pgfpathlineto{\pgfqpoint{1.945158in}{1.108820in}}%
\pgfpathlineto{\pgfqpoint{1.942023in}{1.108820in}}%
\pgfpathlineto{\pgfqpoint{1.941553in}{1.109382in}}%
\pgfpathlineto{\pgfqpoint{1.938889in}{1.112571in}}%
\pgfpathlineto{\pgfqpoint{1.938418in}{1.113134in}}%
\pgfpathlineto{\pgfqpoint{1.935754in}{1.116323in}}%
\pgfpathlineto{\pgfqpoint{1.935284in}{1.116885in}}%
\pgfpathlineto{\pgfqpoint{1.932619in}{1.120074in}}%
\pgfpathlineto{\pgfqpoint{1.932149in}{1.120637in}}%
\pgfpathlineto{\pgfqpoint{1.929484in}{1.123826in}}%
\pgfpathlineto{\pgfqpoint{1.926350in}{1.123826in}}%
\pgfpathlineto{\pgfqpoint{1.925879in}{1.124388in}}%
\pgfpathlineto{\pgfqpoint{1.923215in}{1.127577in}}%
\pgfpathlineto{\pgfqpoint{1.922745in}{1.128140in}}%
\pgfpathlineto{\pgfqpoint{1.920080in}{1.131329in}}%
\pgfpathlineto{\pgfqpoint{1.919610in}{1.131892in}}%
\pgfpathlineto{\pgfqpoint{1.916945in}{1.135080in}}%
\pgfpathlineto{\pgfqpoint{1.913811in}{1.135080in}}%
\pgfpathlineto{\pgfqpoint{1.913341in}{1.135643in}}%
\pgfpathlineto{\pgfqpoint{1.910676in}{1.138832in}}%
\pgfpathlineto{\pgfqpoint{1.910206in}{1.139395in}}%
\pgfpathlineto{\pgfqpoint{1.907541in}{1.142583in}}%
\pgfpathlineto{\pgfqpoint{1.907071in}{1.143146in}}%
\pgfpathlineto{\pgfqpoint{1.904407in}{1.146335in}}%
\pgfpathlineto{\pgfqpoint{1.901272in}{1.146335in}}%
\pgfpathlineto{\pgfqpoint{1.900802in}{1.146898in}}%
\pgfpathlineto{\pgfqpoint{1.898137in}{1.150087in}}%
\pgfpathlineto{\pgfqpoint{1.897667in}{1.150649in}}%
\pgfpathlineto{\pgfqpoint{1.895002in}{1.153838in}}%
\pgfpathlineto{\pgfqpoint{1.894532in}{1.154401in}}%
\pgfpathlineto{\pgfqpoint{1.891868in}{1.157590in}}%
\pgfpathlineto{\pgfqpoint{1.888733in}{1.157590in}}%
\pgfpathlineto{\pgfqpoint{1.888263in}{1.158152in}}%
\pgfpathlineto{\pgfqpoint{1.885598in}{1.161341in}}%
\pgfpathlineto{\pgfqpoint{1.885128in}{1.161904in}}%
\pgfpathlineto{\pgfqpoint{1.882463in}{1.165093in}}%
\pgfpathlineto{\pgfqpoint{1.881993in}{1.165655in}}%
\pgfpathlineto{\pgfqpoint{1.879329in}{1.168844in}}%
\pgfpathlineto{\pgfqpoint{1.876194in}{1.168844in}}%
\pgfpathlineto{\pgfqpoint{1.875724in}{1.169407in}}%
\pgfpathlineto{\pgfqpoint{1.873059in}{1.172596in}}%
\pgfpathlineto{\pgfqpoint{1.872589in}{1.173158in}}%
\pgfpathlineto{\pgfqpoint{1.869924in}{1.176347in}}%
\pgfpathlineto{\pgfqpoint{1.869454in}{1.176910in}}%
\pgfpathlineto{\pgfqpoint{1.866790in}{1.180099in}}%
\pgfpathlineto{\pgfqpoint{1.863655in}{1.180099in}}%
\pgfpathlineto{\pgfqpoint{1.863185in}{1.180662in}}%
\pgfpathlineto{\pgfqpoint{1.860520in}{1.183850in}}%
\pgfpathlineto{\pgfqpoint{1.860050in}{1.184413in}}%
\pgfpathlineto{\pgfqpoint{1.857385in}{1.187602in}}%
\pgfpathlineto{\pgfqpoint{1.856915in}{1.188165in}}%
\pgfpathlineto{\pgfqpoint{1.854251in}{1.191353in}}%
\pgfpathlineto{\pgfqpoint{1.853780in}{1.191916in}}%
\pgfpathlineto{\pgfqpoint{1.851116in}{1.195105in}}%
\pgfpathlineto{\pgfqpoint{1.847981in}{1.195105in}}%
\pgfpathlineto{\pgfqpoint{1.847511in}{1.195668in}}%
\pgfpathlineto{\pgfqpoint{1.844846in}{1.198856in}}%
\pgfpathlineto{\pgfqpoint{1.844376in}{1.199419in}}%
\pgfpathlineto{\pgfqpoint{1.841712in}{1.202608in}}%
\pgfpathlineto{\pgfqpoint{1.841241in}{1.203171in}}%
\pgfpathlineto{\pgfqpoint{1.838577in}{1.206360in}}%
\pgfpathlineto{\pgfqpoint{1.835442in}{1.206360in}}%
\pgfpathlineto{\pgfqpoint{1.834972in}{1.206922in}}%
\pgfpathlineto{\pgfqpoint{1.832307in}{1.210111in}}%
\pgfpathlineto{\pgfqpoint{1.831837in}{1.210674in}}%
\pgfpathlineto{\pgfqpoint{1.829173in}{1.213863in}}%
\pgfpathlineto{\pgfqpoint{1.828702in}{1.214425in}}%
\pgfpathlineto{\pgfqpoint{1.826038in}{1.217614in}}%
\pgfpathlineto{\pgfqpoint{1.822903in}{1.217614in}}%
\pgfpathlineto{\pgfqpoint{1.822433in}{1.218177in}}%
\pgfpathlineto{\pgfqpoint{1.819768in}{1.221366in}}%
\pgfpathlineto{\pgfqpoint{1.819298in}{1.221928in}}%
\pgfpathlineto{\pgfqpoint{1.816634in}{1.225117in}}%
\pgfpathlineto{\pgfqpoint{1.816164in}{1.225680in}}%
\pgfpathlineto{\pgfqpoint{1.813499in}{1.228869in}}%
\pgfpathlineto{\pgfqpoint{1.810364in}{1.228869in}}%
\pgfpathlineto{\pgfqpoint{1.809894in}{1.229431in}}%
\pgfpathlineto{\pgfqpoint{1.807229in}{1.232620in}}%
\pgfpathlineto{\pgfqpoint{1.806759in}{1.233183in}}%
\pgfpathlineto{\pgfqpoint{1.804095in}{1.236372in}}%
\pgfpathlineto{\pgfqpoint{1.803625in}{1.236935in}}%
\pgfpathlineto{\pgfqpoint{1.800960in}{1.240123in}}%
\pgfpathlineto{\pgfqpoint{1.797825in}{1.240123in}}%
\pgfpathlineto{\pgfqpoint{1.797355in}{1.240686in}}%
\pgfpathlineto{\pgfqpoint{1.794691in}{1.243875in}}%
\pgfpathlineto{\pgfqpoint{1.794220in}{1.244438in}}%
\pgfpathlineto{\pgfqpoint{1.791556in}{1.247626in}}%
\pgfpathlineto{\pgfqpoint{1.791086in}{1.248189in}}%
\pgfpathlineto{\pgfqpoint{1.788421in}{1.251378in}}%
\pgfpathlineto{\pgfqpoint{1.787951in}{1.251941in}}%
\pgfpathlineto{\pgfqpoint{1.785286in}{1.255129in}}%
\pgfpathlineto{\pgfqpoint{1.782152in}{1.255129in}}%
\pgfpathlineto{\pgfqpoint{1.781681in}{1.255692in}}%
\pgfpathlineto{\pgfqpoint{1.779017in}{1.258881in}}%
\pgfpathlineto{\pgfqpoint{1.778547in}{1.259444in}}%
\pgfpathlineto{\pgfqpoint{1.775882in}{1.262633in}}%
\pgfpathlineto{\pgfqpoint{1.775412in}{1.263195in}}%
\pgfpathlineto{\pgfqpoint{1.772747in}{1.266384in}}%
\pgfpathlineto{\pgfqpoint{1.769613in}{1.266384in}}%
\pgfpathlineto{\pgfqpoint{1.769142in}{1.266947in}}%
\pgfpathlineto{\pgfqpoint{1.766478in}{1.270136in}}%
\pgfpathlineto{\pgfqpoint{1.766008in}{1.270698in}}%
\pgfpathlineto{\pgfqpoint{1.763343in}{1.273887in}}%
\pgfpathlineto{\pgfqpoint{1.762873in}{1.274450in}}%
\pgfpathlineto{\pgfqpoint{1.760208in}{1.277639in}}%
\pgfpathlineto{\pgfqpoint{1.757074in}{1.277639in}}%
\pgfpathlineto{\pgfqpoint{1.756603in}{1.278201in}}%
\pgfpathlineto{\pgfqpoint{1.753939in}{1.281390in}}%
\pgfpathlineto{\pgfqpoint{1.753469in}{1.281953in}}%
\pgfpathlineto{\pgfqpoint{1.750804in}{1.285142in}}%
\pgfpathlineto{\pgfqpoint{1.750334in}{1.285704in}}%
\pgfpathlineto{\pgfqpoint{1.747669in}{1.288893in}}%
\pgfpathlineto{\pgfqpoint{1.744535in}{1.288893in}}%
\pgfpathlineto{\pgfqpoint{1.744064in}{1.289456in}}%
\pgfpathlineto{\pgfqpoint{1.741400in}{1.292645in}}%
\pgfpathlineto{\pgfqpoint{1.740930in}{1.293208in}}%
\pgfpathlineto{\pgfqpoint{1.738265in}{1.296396in}}%
\pgfpathlineto{\pgfqpoint{1.737795in}{1.296959in}}%
\pgfpathlineto{\pgfqpoint{1.735130in}{1.300148in}}%
\pgfpathlineto{\pgfqpoint{1.731996in}{1.300148in}}%
\pgfpathlineto{\pgfqpoint{1.731525in}{1.300711in}}%
\pgfpathlineto{\pgfqpoint{1.728861in}{1.303899in}}%
\pgfpathlineto{\pgfqpoint{1.728391in}{1.304462in}}%
\pgfpathlineto{\pgfqpoint{1.725726in}{1.307651in}}%
\pgfpathlineto{\pgfqpoint{1.725256in}{1.308214in}}%
\pgfpathlineto{\pgfqpoint{1.722591in}{1.311402in}}%
\pgfpathlineto{\pgfqpoint{1.719457in}{1.311402in}}%
\pgfpathlineto{\pgfqpoint{1.718986in}{1.311965in}}%
\pgfpathlineto{\pgfqpoint{1.716322in}{1.315154in}}%
\pgfpathlineto{\pgfqpoint{1.715852in}{1.315717in}}%
\pgfpathlineto{\pgfqpoint{1.713187in}{1.318906in}}%
\pgfpathlineto{\pgfqpoint{1.712717in}{1.319468in}}%
\pgfpathlineto{\pgfqpoint{1.710052in}{1.322657in}}%
\pgfpathlineto{\pgfqpoint{1.709582in}{1.323220in}}%
\pgfpathlineto{\pgfqpoint{1.706918in}{1.326409in}}%
\pgfpathlineto{\pgfqpoint{1.703783in}{1.326409in}}%
\pgfpathlineto{\pgfqpoint{1.703313in}{1.326971in}}%
\pgfpathlineto{\pgfqpoint{1.700648in}{1.330160in}}%
\pgfpathlineto{\pgfqpoint{1.700178in}{1.330723in}}%
\pgfpathlineto{\pgfqpoint{1.697514in}{1.333912in}}%
\pgfpathlineto{\pgfqpoint{1.697043in}{1.334474in}}%
\pgfpathlineto{\pgfqpoint{1.694379in}{1.337663in}}%
\pgfpathlineto{\pgfqpoint{1.691244in}{1.337663in}}%
\pgfpathlineto{\pgfqpoint{1.690774in}{1.338226in}}%
\pgfpathlineto{\pgfqpoint{1.688109in}{1.341415in}}%
\pgfpathlineto{\pgfqpoint{1.687639in}{1.341977in}}%
\pgfpathlineto{\pgfqpoint{1.684975in}{1.345166in}}%
\pgfpathlineto{\pgfqpoint{1.684504in}{1.345729in}}%
\pgfpathlineto{\pgfqpoint{1.681840in}{1.348918in}}%
\pgfpathlineto{\pgfqpoint{1.678705in}{1.348918in}}%
\pgfpathlineto{\pgfqpoint{1.678235in}{1.349481in}}%
\pgfpathlineto{\pgfqpoint{1.675570in}{1.352669in}}%
\pgfpathlineto{\pgfqpoint{1.675100in}{1.353232in}}%
\pgfpathlineto{\pgfqpoint{1.672436in}{1.356421in}}%
\pgfpathlineto{\pgfqpoint{1.671965in}{1.356984in}}%
\pgfpathlineto{\pgfqpoint{1.669301in}{1.360172in}}%
\pgfpathlineto{\pgfqpoint{1.666166in}{1.360172in}}%
\pgfpathlineto{\pgfqpoint{1.665696in}{1.360735in}}%
\pgfpathlineto{\pgfqpoint{1.663031in}{1.363924in}}%
\pgfpathlineto{\pgfqpoint{1.662561in}{1.364487in}}%
\pgfpathlineto{\pgfqpoint{1.659897in}{1.367676in}}%
\pgfpathlineto{\pgfqpoint{1.659426in}{1.368238in}}%
\pgfpathlineto{\pgfqpoint{1.656762in}{1.371427in}}%
\pgfpathlineto{\pgfqpoint{1.653627in}{1.371427in}}%
\pgfpathlineto{\pgfqpoint{1.653157in}{1.371990in}}%
\pgfpathlineto{\pgfqpoint{1.650492in}{1.375179in}}%
\pgfpathlineto{\pgfqpoint{1.650022in}{1.375741in}}%
\pgfpathlineto{\pgfqpoint{1.647358in}{1.378930in}}%
\pgfpathlineto{\pgfqpoint{1.646887in}{1.379493in}}%
\pgfpathlineto{\pgfqpoint{1.644223in}{1.382682in}}%
\pgfpathlineto{\pgfqpoint{1.641088in}{1.382682in}}%
\pgfpathlineto{\pgfqpoint{1.640618in}{1.383244in}}%
\pgfpathlineto{\pgfqpoint{1.637953in}{1.386433in}}%
\pgfpathlineto{\pgfqpoint{1.637483in}{1.386996in}}%
\pgfpathlineto{\pgfqpoint{1.634819in}{1.390185in}}%
\pgfpathlineto{\pgfqpoint{1.634348in}{1.390747in}}%
\pgfpathlineto{\pgfqpoint{1.631684in}{1.393936in}}%
\pgfpathlineto{\pgfqpoint{1.631214in}{1.394499in}}%
\pgfpathlineto{\pgfqpoint{1.628549in}{1.397688in}}%
\pgfpathlineto{\pgfqpoint{1.625414in}{1.397688in}}%
\pgfpathlineto{\pgfqpoint{1.624944in}{1.398251in}}%
\pgfpathlineto{\pgfqpoint{1.622280in}{1.401439in}}%
\pgfpathlineto{\pgfqpoint{1.621809in}{1.402002in}}%
\pgfpathlineto{\pgfqpoint{1.619145in}{1.405191in}}%
\pgfpathlineto{\pgfqpoint{1.618675in}{1.405754in}}%
\pgfpathlineto{\pgfqpoint{1.616010in}{1.408942in}}%
\pgfpathlineto{\pgfqpoint{1.612875in}{1.408942in}}%
\pgfpathlineto{\pgfqpoint{1.612405in}{1.409505in}}%
\pgfpathlineto{\pgfqpoint{1.609741in}{1.412694in}}%
\pgfpathlineto{\pgfqpoint{1.609271in}{1.413257in}}%
\pgfpathlineto{\pgfqpoint{1.606606in}{1.416445in}}%
\pgfpathlineto{\pgfqpoint{1.606136in}{1.417008in}}%
\pgfpathlineto{\pgfqpoint{1.603471in}{1.420197in}}%
\pgfpathlineto{\pgfqpoint{1.600337in}{1.420197in}}%
\pgfpathlineto{\pgfqpoint{1.599866in}{1.420760in}}%
\pgfpathlineto{\pgfqpoint{1.597202in}{1.423949in}}%
\pgfpathlineto{\pgfqpoint{1.596732in}{1.424511in}}%
\pgfpathlineto{\pgfqpoint{1.594067in}{1.427700in}}%
\pgfpathlineto{\pgfqpoint{1.593597in}{1.428263in}}%
\pgfpathlineto{\pgfqpoint{1.590932in}{1.431452in}}%
\pgfpathlineto{\pgfqpoint{1.587798in}{1.431452in}}%
\pgfpathlineto{\pgfqpoint{1.587327in}{1.432014in}}%
\pgfpathlineto{\pgfqpoint{1.584663in}{1.435203in}}%
\pgfpathlineto{\pgfqpoint{1.584193in}{1.435766in}}%
\pgfpathlineto{\pgfqpoint{1.581528in}{1.438955in}}%
\pgfpathlineto{\pgfqpoint{1.581058in}{1.439517in}}%
\pgfpathlineto{\pgfqpoint{1.578393in}{1.442706in}}%
\pgfpathlineto{\pgfqpoint{1.575259in}{1.442706in}}%
\pgfpathlineto{\pgfqpoint{1.574788in}{1.443269in}}%
\pgfpathlineto{\pgfqpoint{1.572124in}{1.446458in}}%
\pgfpathlineto{\pgfqpoint{1.571654in}{1.447020in}}%
\pgfpathlineto{\pgfqpoint{1.568989in}{1.450209in}}%
\pgfpathlineto{\pgfqpoint{1.568519in}{1.450772in}}%
\pgfpathlineto{\pgfqpoint{1.565854in}{1.453961in}}%
\pgfpathlineto{\pgfqpoint{1.562720in}{1.453961in}}%
\pgfpathlineto{\pgfqpoint{1.562249in}{1.454524in}}%
\pgfpathlineto{\pgfqpoint{1.559585in}{1.457712in}}%
\pgfpathlineto{\pgfqpoint{1.559115in}{1.458275in}}%
\pgfpathlineto{\pgfqpoint{1.556450in}{1.461464in}}%
\pgfpathlineto{\pgfqpoint{1.555980in}{1.462027in}}%
\pgfpathlineto{\pgfqpoint{1.553315in}{1.465215in}}%
\pgfpathlineto{\pgfqpoint{1.552845in}{1.465778in}}%
\pgfpathlineto{\pgfqpoint{1.550181in}{1.468967in}}%
\pgfpathlineto{\pgfqpoint{1.547046in}{1.468967in}}%
\pgfpathlineto{\pgfqpoint{1.546576in}{1.469530in}}%
\pgfpathlineto{\pgfqpoint{1.543911in}{1.472718in}}%
\pgfpathlineto{\pgfqpoint{1.543441in}{1.473281in}}%
\pgfpathlineto{\pgfqpoint{1.540776in}{1.476470in}}%
\pgfpathlineto{\pgfqpoint{1.540306in}{1.477033in}}%
\pgfpathlineto{\pgfqpoint{1.537642in}{1.480222in}}%
\pgfpathlineto{\pgfqpoint{1.534507in}{1.480222in}}%
\pgfpathlineto{\pgfqpoint{1.534037in}{1.480784in}}%
\pgfpathlineto{\pgfqpoint{1.531372in}{1.483973in}}%
\pgfpathlineto{\pgfqpoint{1.530902in}{1.484536in}}%
\pgfpathlineto{\pgfqpoint{1.528237in}{1.487725in}}%
\pgfpathlineto{\pgfqpoint{1.527767in}{1.488287in}}%
\pgfpathlineto{\pgfqpoint{1.525103in}{1.491476in}}%
\pgfpathlineto{\pgfqpoint{1.521968in}{1.491476in}}%
\pgfpathlineto{\pgfqpoint{1.521498in}{1.492039in}}%
\pgfpathlineto{\pgfqpoint{1.518833in}{1.495228in}}%
\pgfpathlineto{\pgfqpoint{1.518363in}{1.495790in}}%
\pgfpathlineto{\pgfqpoint{1.515698in}{1.498979in}}%
\pgfpathlineto{\pgfqpoint{1.515228in}{1.499542in}}%
\pgfpathlineto{\pgfqpoint{1.512564in}{1.502731in}}%
\pgfpathlineto{\pgfqpoint{1.509429in}{1.502731in}}%
\pgfpathlineto{\pgfqpoint{1.508959in}{1.503293in}}%
\pgfpathlineto{\pgfqpoint{1.506294in}{1.506482in}}%
\pgfpathlineto{\pgfqpoint{1.505824in}{1.507045in}}%
\pgfpathlineto{\pgfqpoint{1.503159in}{1.510234in}}%
\pgfpathlineto{\pgfqpoint{1.502689in}{1.510797in}}%
\pgfpathlineto{\pgfqpoint{1.500025in}{1.513985in}}%
\pgfpathlineto{\pgfqpoint{1.496890in}{1.513985in}}%
\pgfpathlineto{\pgfqpoint{1.496420in}{1.514548in}}%
\pgfpathlineto{\pgfqpoint{1.493755in}{1.517737in}}%
\pgfpathlineto{\pgfqpoint{1.493285in}{1.518300in}}%
\pgfpathlineto{\pgfqpoint{1.490621in}{1.521488in}}%
\pgfpathlineto{\pgfqpoint{1.490150in}{1.522051in}}%
\pgfpathlineto{\pgfqpoint{1.487486in}{1.525240in}}%
\pgfpathlineto{\pgfqpoint{1.487016in}{1.525803in}}%
\pgfpathlineto{\pgfqpoint{1.484351in}{1.528991in}}%
\pgfpathlineto{\pgfqpoint{1.481216in}{1.528991in}}%
\pgfpathlineto{\pgfqpoint{1.480746in}{1.529554in}}%
\pgfpathlineto{\pgfqpoint{1.478082in}{1.532743in}}%
\pgfpathlineto{\pgfqpoint{1.477611in}{1.533306in}}%
\pgfpathlineto{\pgfqpoint{1.474947in}{1.536495in}}%
\pgfpathlineto{\pgfqpoint{1.474477in}{1.537057in}}%
\pgfpathlineto{\pgfqpoint{1.471812in}{1.540246in}}%
\pgfpathlineto{\pgfqpoint{1.468677in}{1.540246in}}%
\pgfpathlineto{\pgfqpoint{1.468207in}{1.540809in}}%
\pgfpathlineto{\pgfqpoint{1.465543in}{1.543998in}}%
\pgfpathlineto{\pgfqpoint{1.465072in}{1.544560in}}%
\pgfpathlineto{\pgfqpoint{1.462408in}{1.547749in}}%
\pgfpathlineto{\pgfqpoint{1.461938in}{1.548312in}}%
\pgfpathlineto{\pgfqpoint{1.459273in}{1.551501in}}%
\pgfpathlineto{\pgfqpoint{1.456138in}{1.551501in}}%
\pgfpathlineto{\pgfqpoint{1.455668in}{1.552063in}}%
\pgfpathlineto{\pgfqpoint{1.453004in}{1.555252in}}%
\pgfpathlineto{\pgfqpoint{1.452533in}{1.555815in}}%
\pgfpathlineto{\pgfqpoint{1.449869in}{1.559004in}}%
\pgfpathlineto{\pgfqpoint{1.449399in}{1.559566in}}%
\pgfpathlineto{\pgfqpoint{1.446734in}{1.562755in}}%
\pgfpathlineto{\pgfqpoint{1.443599in}{1.562755in}}%
\pgfpathlineto{\pgfqpoint{1.443129in}{1.563318in}}%
\pgfpathlineto{\pgfqpoint{1.440465in}{1.566507in}}%
\pgfpathlineto{\pgfqpoint{1.439994in}{1.567070in}}%
\pgfpathlineto{\pgfqpoint{1.437330in}{1.570258in}}%
\pgfpathlineto{\pgfqpoint{1.436860in}{1.570821in}}%
\pgfpathlineto{\pgfqpoint{1.434195in}{1.574010in}}%
\pgfpathlineto{\pgfqpoint{1.431060in}{1.574010in}}%
\pgfpathlineto{\pgfqpoint{1.430590in}{1.574573in}}%
\pgfpathlineto{\pgfqpoint{1.427926in}{1.577761in}}%
\pgfpathlineto{\pgfqpoint{1.427455in}{1.578324in}}%
\pgfpathlineto{\pgfqpoint{1.424791in}{1.581513in}}%
\pgfpathlineto{\pgfqpoint{1.424321in}{1.582076in}}%
\pgfpathlineto{\pgfqpoint{1.421656in}{1.585265in}}%
\pgfpathlineto{\pgfqpoint{1.418521in}{1.585265in}}%
\pgfpathlineto{\pgfqpoint{1.418051in}{1.585827in}}%
\pgfpathlineto{\pgfqpoint{1.415387in}{1.589016in}}%
\pgfpathlineto{\pgfqpoint{1.414916in}{1.589579in}}%
\pgfpathlineto{\pgfqpoint{1.412252in}{1.592768in}}%
\pgfpathlineto{\pgfqpoint{1.411782in}{1.593330in}}%
\pgfpathlineto{\pgfqpoint{1.409117in}{1.596519in}}%
\pgfpathlineto{\pgfqpoint{1.408647in}{1.597082in}}%
\pgfpathlineto{\pgfqpoint{1.405982in}{1.600271in}}%
\pgfpathlineto{\pgfqpoint{1.402848in}{1.600271in}}%
\pgfpathlineto{\pgfqpoint{1.402378in}{1.600833in}}%
\pgfpathlineto{\pgfqpoint{1.399713in}{1.604022in}}%
\pgfpathlineto{\pgfqpoint{1.399243in}{1.604585in}}%
\pgfpathlineto{\pgfqpoint{1.396578in}{1.607774in}}%
\pgfpathlineto{\pgfqpoint{1.396108in}{1.608336in}}%
\pgfpathlineto{\pgfqpoint{1.393444in}{1.611525in}}%
\pgfpathlineto{\pgfqpoint{1.390309in}{1.611525in}}%
\pgfpathlineto{\pgfqpoint{1.389839in}{1.612088in}}%
\pgfpathlineto{\pgfqpoint{1.387174in}{1.615277in}}%
\pgfpathlineto{\pgfqpoint{1.386704in}{1.615840in}}%
\pgfpathlineto{\pgfqpoint{1.384039in}{1.619028in}}%
\pgfpathlineto{\pgfqpoint{1.383569in}{1.619591in}}%
\pgfpathlineto{\pgfqpoint{1.380905in}{1.622780in}}%
\pgfpathlineto{\pgfqpoint{1.377770in}{1.622780in}}%
\pgfpathlineto{\pgfqpoint{1.377300in}{1.623343in}}%
\pgfpathlineto{\pgfqpoint{1.374635in}{1.626531in}}%
\pgfpathlineto{\pgfqpoint{1.374165in}{1.627094in}}%
\pgfpathlineto{\pgfqpoint{1.371500in}{1.630283in}}%
\pgfpathlineto{\pgfqpoint{1.371030in}{1.630846in}}%
\pgfpathlineto{\pgfqpoint{1.368366in}{1.634034in}}%
\pgfpathlineto{\pgfqpoint{1.365231in}{1.634034in}}%
\pgfpathlineto{\pgfqpoint{1.364761in}{1.634597in}}%
\pgfpathlineto{\pgfqpoint{1.362096in}{1.637786in}}%
\pgfpathlineto{\pgfqpoint{1.361626in}{1.638349in}}%
\pgfpathlineto{\pgfqpoint{1.358961in}{1.641538in}}%
\pgfpathlineto{\pgfqpoint{1.358491in}{1.642100in}}%
\pgfpathlineto{\pgfqpoint{1.355827in}{1.645289in}}%
\pgfpathlineto{\pgfqpoint{1.352692in}{1.645289in}}%
\pgfpathlineto{\pgfqpoint{1.352222in}{1.645852in}}%
\pgfpathlineto{\pgfqpoint{1.349557in}{1.649041in}}%
\pgfpathlineto{\pgfqpoint{1.349087in}{1.649603in}}%
\pgfpathlineto{\pgfqpoint{1.346422in}{1.652792in}}%
\pgfpathlineto{\pgfqpoint{1.345952in}{1.653355in}}%
\pgfpathlineto{\pgfqpoint{1.343288in}{1.656544in}}%
\pgfpathlineto{\pgfqpoint{1.340153in}{1.656544in}}%
\pgfpathlineto{\pgfqpoint{1.339683in}{1.657106in}}%
\pgfpathlineto{\pgfqpoint{1.337018in}{1.660295in}}%
\pgfpathlineto{\pgfqpoint{1.336548in}{1.660858in}}%
\pgfpathlineto{\pgfqpoint{1.333883in}{1.664047in}}%
\pgfpathlineto{\pgfqpoint{1.333413in}{1.664609in}}%
\pgfpathlineto{\pgfqpoint{1.330749in}{1.667798in}}%
\pgfpathlineto{\pgfqpoint{1.330278in}{1.668361in}}%
\pgfpathlineto{\pgfqpoint{1.327614in}{1.671550in}}%
\pgfpathlineto{\pgfqpoint{1.324479in}{1.671550in}}%
\pgfpathlineto{\pgfqpoint{1.324009in}{1.672113in}}%
\pgfpathlineto{\pgfqpoint{1.321344in}{1.675301in}}%
\pgfpathlineto{\pgfqpoint{1.320874in}{1.675864in}}%
\pgfpathlineto{\pgfqpoint{1.318210in}{1.679053in}}%
\pgfpathlineto{\pgfqpoint{1.317739in}{1.679616in}}%
\pgfpathlineto{\pgfqpoint{1.315075in}{1.682804in}}%
\pgfpathlineto{\pgfqpoint{1.311940in}{1.682804in}}%
\pgfpathlineto{\pgfqpoint{1.311470in}{1.683367in}}%
\pgfpathlineto{\pgfqpoint{1.308805in}{1.686556in}}%
\pgfpathlineto{\pgfqpoint{1.308335in}{1.687119in}}%
\pgfpathlineto{\pgfqpoint{1.305671in}{1.690307in}}%
\pgfpathlineto{\pgfqpoint{1.305201in}{1.690870in}}%
\pgfpathlineto{\pgfqpoint{1.302536in}{1.694059in}}%
\pgfpathlineto{\pgfqpoint{1.299401in}{1.694059in}}%
\pgfpathlineto{\pgfqpoint{1.298931in}{1.694622in}}%
\pgfpathlineto{\pgfqpoint{1.296266in}{1.697811in}}%
\pgfpathlineto{\pgfqpoint{1.295796in}{1.698373in}}%
\pgfpathlineto{\pgfqpoint{1.293132in}{1.701562in}}%
\pgfpathlineto{\pgfqpoint{1.292662in}{1.702125in}}%
\pgfpathlineto{\pgfqpoint{1.292662in}{1.705876in}}%
\pgfpathlineto{\pgfqpoint{1.293132in}{1.706439in}}%
\pgfpathlineto{\pgfqpoint{1.295796in}{1.709628in}}%
\pgfpathlineto{\pgfqpoint{1.295796in}{1.713379in}}%
\pgfpathlineto{\pgfqpoint{1.295796in}{1.717131in}}%
\pgfpathlineto{\pgfqpoint{1.295796in}{1.720882in}}%
\pgfpathlineto{\pgfqpoint{1.296266in}{1.721445in}}%
\pgfpathlineto{\pgfqpoint{1.298931in}{1.724634in}}%
\pgfpathlineto{\pgfqpoint{1.298931in}{1.728386in}}%
\pgfpathlineto{\pgfqpoint{1.298931in}{1.732137in}}%
\pgfpathlineto{\pgfqpoint{1.299401in}{1.732700in}}%
\pgfpathlineto{\pgfqpoint{1.302066in}{1.735889in}}%
\pgfpathlineto{\pgfqpoint{1.302066in}{1.739640in}}%
\pgfpathlineto{\pgfqpoint{1.302066in}{1.743392in}}%
\pgfpathlineto{\pgfqpoint{1.302066in}{1.747143in}}%
\pgfpathlineto{\pgfqpoint{1.302536in}{1.747706in}}%
\pgfpathlineto{\pgfqpoint{1.305201in}{1.750895in}}%
\pgfpathlineto{\pgfqpoint{1.305201in}{1.754646in}}%
\pgfpathlineto{\pgfqpoint{1.305201in}{1.758398in}}%
\pgfpathlineto{\pgfqpoint{1.305671in}{1.758961in}}%
\pgfpathlineto{\pgfqpoint{1.308335in}{1.762149in}}%
\pgfpathlineto{\pgfqpoint{1.308335in}{1.765901in}}%
\pgfpathlineto{\pgfqpoint{1.308335in}{1.769652in}}%
\pgfpathlineto{\pgfqpoint{1.308335in}{1.773404in}}%
\pgfpathlineto{\pgfqpoint{1.308805in}{1.773967in}}%
\pgfpathlineto{\pgfqpoint{1.311470in}{1.777155in}}%
\pgfpathlineto{\pgfqpoint{1.311470in}{1.780907in}}%
\pgfpathlineto{\pgfqpoint{1.311470in}{1.784659in}}%
\pgfpathlineto{\pgfqpoint{1.311940in}{1.785221in}}%
\pgfpathlineto{\pgfqpoint{1.314605in}{1.788410in}}%
\pgfpathlineto{\pgfqpoint{1.314605in}{1.792162in}}%
\pgfpathlineto{\pgfqpoint{1.314605in}{1.795913in}}%
\pgfpathlineto{\pgfqpoint{1.314605in}{1.799665in}}%
\pgfpathlineto{\pgfqpoint{1.315075in}{1.800227in}}%
\pgfpathlineto{\pgfqpoint{1.317739in}{1.803416in}}%
\pgfpathlineto{\pgfqpoint{1.317739in}{1.807168in}}%
\pgfpathlineto{\pgfqpoint{1.317739in}{1.810919in}}%
\pgfpathlineto{\pgfqpoint{1.318210in}{1.811482in}}%
\pgfpathlineto{\pgfqpoint{1.320874in}{1.814671in}}%
\pgfpathlineto{\pgfqpoint{1.320874in}{1.818422in}}%
\pgfpathlineto{\pgfqpoint{1.320874in}{1.822174in}}%
\pgfpathlineto{\pgfqpoint{1.321344in}{1.822737in}}%
\pgfpathlineto{\pgfqpoint{1.324009in}{1.825925in}}%
\pgfpathlineto{\pgfqpoint{1.324009in}{1.829677in}}%
\pgfpathlineto{\pgfqpoint{1.324009in}{1.833429in}}%
\pgfpathlineto{\pgfqpoint{1.324009in}{1.837180in}}%
\pgfpathlineto{\pgfqpoint{1.324479in}{1.837743in}}%
\pgfpathlineto{\pgfqpoint{1.327144in}{1.840932in}}%
\pgfpathlineto{\pgfqpoint{1.327144in}{1.844683in}}%
\pgfpathlineto{\pgfqpoint{1.327144in}{1.848435in}}%
\pgfpathlineto{\pgfqpoint{1.327614in}{1.848997in}}%
\pgfpathlineto{\pgfqpoint{1.330278in}{1.852186in}}%
\pgfpathlineto{\pgfqpoint{1.330278in}{1.855938in}}%
\pgfpathlineto{\pgfqpoint{1.330278in}{1.859689in}}%
\pgfpathlineto{\pgfqpoint{1.330278in}{1.863441in}}%
\pgfpathlineto{\pgfqpoint{1.330749in}{1.864004in}}%
\pgfpathlineto{\pgfqpoint{1.333413in}{1.867192in}}%
\pgfpathlineto{\pgfqpoint{1.333413in}{1.870944in}}%
\pgfpathlineto{\pgfqpoint{1.333413in}{1.874695in}}%
\pgfpathlineto{\pgfqpoint{1.333883in}{1.875258in}}%
\pgfpathlineto{\pgfqpoint{1.336548in}{1.878447in}}%
\pgfpathlineto{\pgfqpoint{1.336548in}{1.882198in}}%
\pgfpathlineto{\pgfqpoint{1.336548in}{1.885950in}}%
\pgfpathlineto{\pgfqpoint{1.336548in}{1.889702in}}%
\pgfpathlineto{\pgfqpoint{1.337018in}{1.890264in}}%
\pgfpathlineto{\pgfqpoint{1.339683in}{1.893453in}}%
\pgfpathlineto{\pgfqpoint{1.339683in}{1.897205in}}%
\pgfpathlineto{\pgfqpoint{1.339683in}{1.900956in}}%
\pgfpathlineto{\pgfqpoint{1.340153in}{1.901519in}}%
\pgfpathlineto{\pgfqpoint{1.342817in}{1.904708in}}%
\pgfpathlineto{\pgfqpoint{1.342817in}{1.908459in}}%
\pgfpathlineto{\pgfqpoint{1.342817in}{1.912211in}}%
\pgfpathlineto{\pgfqpoint{1.342817in}{1.915962in}}%
\pgfpathlineto{\pgfqpoint{1.343288in}{1.916525in}}%
\pgfpathlineto{\pgfqpoint{1.345952in}{1.919714in}}%
\pgfpathlineto{\pgfqpoint{1.345952in}{1.923465in}}%
\pgfpathlineto{\pgfqpoint{1.345952in}{1.927217in}}%
\pgfpathlineto{\pgfqpoint{1.346422in}{1.927780in}}%
\pgfpathlineto{\pgfqpoint{1.349087in}{1.930968in}}%
\pgfpathlineto{\pgfqpoint{1.349087in}{1.934720in}}%
\pgfpathlineto{\pgfqpoint{1.349087in}{1.938471in}}%
\pgfpathlineto{\pgfqpoint{1.349087in}{1.942223in}}%
\pgfpathlineto{\pgfqpoint{1.349557in}{1.942786in}}%
\pgfpathlineto{\pgfqpoint{1.352222in}{1.945975in}}%
\pgfpathlineto{\pgfqpoint{1.352222in}{1.949726in}}%
\pgfpathlineto{\pgfqpoint{1.352222in}{1.953478in}}%
\pgfpathlineto{\pgfqpoint{1.352692in}{1.954040in}}%
\pgfpathlineto{\pgfqpoint{1.355356in}{1.957229in}}%
\pgfpathlineto{\pgfqpoint{1.355356in}{1.960981in}}%
\pgfpathlineto{\pgfqpoint{1.355356in}{1.964732in}}%
\pgfpathlineto{\pgfqpoint{1.355356in}{1.968484in}}%
\pgfpathlineto{\pgfqpoint{1.355827in}{1.969046in}}%
\pgfpathlineto{\pgfqpoint{1.358491in}{1.972235in}}%
\pgfpathlineto{\pgfqpoint{1.358491in}{1.975987in}}%
\pgfpathlineto{\pgfqpoint{1.358491in}{1.979738in}}%
\pgfpathlineto{\pgfqpoint{1.358961in}{1.980301in}}%
\pgfpathlineto{\pgfqpoint{1.361626in}{1.983490in}}%
\pgfpathlineto{\pgfqpoint{1.361626in}{1.987241in}}%
\pgfpathlineto{\pgfqpoint{1.361626in}{1.990993in}}%
\pgfpathlineto{\pgfqpoint{1.361626in}{1.994745in}}%
\pgfpathlineto{\pgfqpoint{1.362096in}{1.995307in}}%
\pgfpathlineto{\pgfqpoint{1.364761in}{1.998496in}}%
\pgfpathlineto{\pgfqpoint{1.364761in}{2.002248in}}%
\pgfpathlineto{\pgfqpoint{1.364761in}{2.005999in}}%
\pgfpathlineto{\pgfqpoint{1.365231in}{2.006562in}}%
\pgfpathlineto{\pgfqpoint{1.367895in}{2.009751in}}%
\pgfpathlineto{\pgfqpoint{1.367895in}{2.013502in}}%
\pgfpathlineto{\pgfqpoint{1.367895in}{2.017254in}}%
\pgfpathlineto{\pgfqpoint{1.367895in}{2.021005in}}%
\pgfpathlineto{\pgfqpoint{1.368366in}{2.021568in}}%
\pgfpathlineto{\pgfqpoint{1.371030in}{2.024757in}}%
\pgfpathlineto{\pgfqpoint{1.371030in}{2.028508in}}%
\pgfpathlineto{\pgfqpoint{1.371030in}{2.032260in}}%
\pgfpathlineto{\pgfqpoint{1.371500in}{2.032823in}}%
\pgfpathlineto{\pgfqpoint{1.374165in}{2.036011in}}%
\pgfpathlineto{\pgfqpoint{1.374165in}{2.039763in}}%
\pgfpathlineto{\pgfqpoint{1.374165in}{2.043514in}}%
\pgfpathlineto{\pgfqpoint{1.374635in}{2.044077in}}%
\pgfpathlineto{\pgfqpoint{1.377300in}{2.047266in}}%
\pgfpathlineto{\pgfqpoint{1.377300in}{2.051018in}}%
\pgfpathlineto{\pgfqpoint{1.377300in}{2.054769in}}%
\pgfpathlineto{\pgfqpoint{1.377300in}{2.058521in}}%
\pgfpathlineto{\pgfqpoint{1.377770in}{2.059083in}}%
\pgfpathlineto{\pgfqpoint{1.380434in}{2.062272in}}%
\pgfpathlineto{\pgfqpoint{1.380434in}{2.066024in}}%
\pgfpathlineto{\pgfqpoint{1.380434in}{2.069775in}}%
\pgfpathlineto{\pgfqpoint{1.380905in}{2.070338in}}%
\pgfpathlineto{\pgfqpoint{1.383569in}{2.073527in}}%
\pgfpathlineto{\pgfqpoint{1.383569in}{2.077278in}}%
\pgfpathlineto{\pgfqpoint{1.383569in}{2.081030in}}%
\pgfpathlineto{\pgfqpoint{1.383569in}{2.084781in}}%
\pgfpathlineto{\pgfqpoint{1.384039in}{2.085344in}}%
\pgfpathlineto{\pgfqpoint{1.386704in}{2.088533in}}%
\pgfpathlineto{\pgfqpoint{1.386704in}{2.092284in}}%
\pgfpathlineto{\pgfqpoint{1.386704in}{2.096036in}}%
\pgfpathlineto{\pgfqpoint{1.387174in}{2.096599in}}%
\pgfpathlineto{\pgfqpoint{1.389839in}{2.099787in}}%
\pgfpathlineto{\pgfqpoint{1.389839in}{2.103539in}}%
\pgfpathlineto{\pgfqpoint{1.389839in}{2.107291in}}%
\pgfpathlineto{\pgfqpoint{1.389839in}{2.111042in}}%
\pgfpathlineto{\pgfqpoint{1.390309in}{2.111605in}}%
\pgfpathlineto{\pgfqpoint{1.392973in}{2.114794in}}%
\pgfpathlineto{\pgfqpoint{1.392973in}{2.118545in}}%
\pgfpathlineto{\pgfqpoint{1.392973in}{2.122297in}}%
\pgfpathlineto{\pgfqpoint{1.393444in}{2.122859in}}%
\pgfpathlineto{\pgfqpoint{1.396108in}{2.126048in}}%
\pgfpathlineto{\pgfqpoint{1.396108in}{2.129800in}}%
\pgfpathlineto{\pgfqpoint{1.396108in}{2.133551in}}%
\pgfpathlineto{\pgfqpoint{1.396108in}{2.137303in}}%
\pgfpathlineto{\pgfqpoint{1.396578in}{2.137866in}}%
\pgfpathlineto{\pgfqpoint{1.399243in}{2.141054in}}%
\pgfpathlineto{\pgfqpoint{1.399243in}{2.144806in}}%
\pgfpathlineto{\pgfqpoint{1.399243in}{2.148557in}}%
\pgfpathlineto{\pgfqpoint{1.399713in}{2.149120in}}%
\pgfpathlineto{\pgfqpoint{1.402378in}{2.152309in}}%
\pgfpathlineto{\pgfqpoint{1.402378in}{2.156060in}}%
\pgfpathlineto{\pgfqpoint{1.402378in}{2.159812in}}%
\pgfpathlineto{\pgfqpoint{1.402378in}{2.163564in}}%
\pgfpathlineto{\pgfqpoint{1.402848in}{2.164126in}}%
\pgfpathlineto{\pgfqpoint{1.405512in}{2.167315in}}%
\pgfpathlineto{\pgfqpoint{1.405512in}{2.171067in}}%
\pgfpathlineto{\pgfqpoint{1.405512in}{2.174818in}}%
\pgfpathlineto{\pgfqpoint{1.405982in}{2.175381in}}%
\pgfpathlineto{\pgfqpoint{1.408647in}{2.178570in}}%
\pgfpathlineto{\pgfqpoint{1.408647in}{2.182321in}}%
\pgfpathlineto{\pgfqpoint{1.408647in}{2.186073in}}%
\pgfpathlineto{\pgfqpoint{1.408647in}{2.189824in}}%
\pgfpathlineto{\pgfqpoint{1.409117in}{2.190387in}}%
\pgfpathlineto{\pgfqpoint{1.411782in}{2.193576in}}%
\pgfpathlineto{\pgfqpoint{1.411782in}{2.197327in}}%
\pgfpathlineto{\pgfqpoint{1.411782in}{2.201079in}}%
\pgfpathlineto{\pgfqpoint{1.412252in}{2.201642in}}%
\pgfpathlineto{\pgfqpoint{1.414916in}{2.204830in}}%
\pgfpathlineto{\pgfqpoint{1.414916in}{2.208582in}}%
\pgfpathlineto{\pgfqpoint{1.414916in}{2.212334in}}%
\pgfpathlineto{\pgfqpoint{1.414916in}{2.216085in}}%
\pgfpathlineto{\pgfqpoint{1.415387in}{2.216648in}}%
\pgfpathlineto{\pgfqpoint{1.418051in}{2.219837in}}%
\pgfpathlineto{\pgfqpoint{1.418051in}{2.223588in}}%
\pgfpathlineto{\pgfqpoint{1.418051in}{2.227340in}}%
\pgfpathlineto{\pgfqpoint{1.418521in}{2.227902in}}%
\pgfpathlineto{\pgfqpoint{1.421186in}{2.231091in}}%
\pgfpathlineto{\pgfqpoint{1.421186in}{2.234843in}}%
\pgfpathlineto{\pgfqpoint{1.421186in}{2.238594in}}%
\pgfpathlineto{\pgfqpoint{1.421186in}{2.242346in}}%
\pgfpathlineto{\pgfqpoint{1.421656in}{2.242909in}}%
\pgfpathlineto{\pgfqpoint{1.424321in}{2.246097in}}%
\pgfpathlineto{\pgfqpoint{1.424321in}{2.249849in}}%
\pgfpathlineto{\pgfqpoint{1.424321in}{2.253600in}}%
\pgfpathlineto{\pgfqpoint{1.424791in}{2.254163in}}%
\pgfpathlineto{\pgfqpoint{1.427455in}{2.257352in}}%
\pgfpathlineto{\pgfqpoint{1.427455in}{2.261103in}}%
\pgfpathlineto{\pgfqpoint{1.427455in}{2.264855in}}%
\pgfpathlineto{\pgfqpoint{1.427926in}{2.265418in}}%
\pgfpathlineto{\pgfqpoint{1.430590in}{2.268607in}}%
\pgfpathlineto{\pgfqpoint{1.430590in}{2.272358in}}%
\pgfpathlineto{\pgfqpoint{1.430590in}{2.276110in}}%
\pgfpathlineto{\pgfqpoint{1.430590in}{2.279861in}}%
\pgfpathlineto{\pgfqpoint{1.431060in}{2.280424in}}%
\pgfpathlineto{\pgfqpoint{1.433725in}{2.283613in}}%
\pgfpathlineto{\pgfqpoint{1.433725in}{2.287364in}}%
\pgfpathlineto{\pgfqpoint{1.433725in}{2.291116in}}%
\pgfpathlineto{\pgfqpoint{1.434195in}{2.291678in}}%
\pgfpathlineto{\pgfqpoint{1.436860in}{2.294867in}}%
\pgfpathlineto{\pgfqpoint{1.436860in}{2.298619in}}%
\pgfpathlineto{\pgfqpoint{1.436860in}{2.302370in}}%
\pgfpathlineto{\pgfqpoint{1.436860in}{2.306122in}}%
\pgfpathlineto{\pgfqpoint{1.437330in}{2.306685in}}%
\pgfpathlineto{\pgfqpoint{1.439994in}{2.309873in}}%
\pgfpathlineto{\pgfqpoint{1.439994in}{2.313625in}}%
\pgfpathlineto{\pgfqpoint{1.439994in}{2.317376in}}%
\pgfpathlineto{\pgfqpoint{1.440465in}{2.317939in}}%
\pgfpathlineto{\pgfqpoint{1.443129in}{2.321128in}}%
\pgfpathlineto{\pgfqpoint{1.443129in}{2.324880in}}%
\pgfpathlineto{\pgfqpoint{1.443129in}{2.328631in}}%
\pgfpathlineto{\pgfqpoint{1.443129in}{2.332383in}}%
\pgfpathlineto{\pgfqpoint{1.443599in}{2.332945in}}%
\pgfpathlineto{\pgfqpoint{1.446264in}{2.336134in}}%
\pgfpathlineto{\pgfqpoint{1.446264in}{2.339886in}}%
\pgfpathlineto{\pgfqpoint{1.446264in}{2.343637in}}%
\pgfpathlineto{\pgfqpoint{1.446734in}{2.344200in}}%
\pgfpathlineto{\pgfqpoint{1.449399in}{2.347389in}}%
\pgfpathlineto{\pgfqpoint{1.449399in}{2.351140in}}%
\pgfpathlineto{\pgfqpoint{1.449399in}{2.354892in}}%
\pgfpathlineto{\pgfqpoint{1.449399in}{2.358643in}}%
\pgfpathlineto{\pgfqpoint{1.449869in}{2.359206in}}%
\pgfpathlineto{\pgfqpoint{1.452533in}{2.362395in}}%
\pgfpathlineto{\pgfqpoint{1.452533in}{2.366146in}}%
\pgfpathlineto{\pgfqpoint{1.452533in}{2.369898in}}%
\pgfpathlineto{\pgfqpoint{1.453004in}{2.370461in}}%
\pgfpathlineto{\pgfqpoint{1.455668in}{2.373649in}}%
\pgfpathlineto{\pgfqpoint{1.455668in}{2.377401in}}%
\pgfpathlineto{\pgfqpoint{1.455668in}{2.381153in}}%
\pgfpathlineto{\pgfqpoint{1.455668in}{2.384904in}}%
\pgfpathlineto{\pgfqpoint{1.456138in}{2.385467in}}%
\pgfpathlineto{\pgfqpoint{1.458803in}{2.388656in}}%
\pgfpathlineto{\pgfqpoint{1.458803in}{2.392407in}}%
\pgfpathlineto{\pgfqpoint{1.458803in}{2.396159in}}%
\pgfpathlineto{\pgfqpoint{1.459273in}{2.396721in}}%
\pgfpathlineto{\pgfqpoint{1.461938in}{2.399910in}}%
\pgfpathlineto{\pgfqpoint{1.461938in}{2.403662in}}%
\pgfpathlineto{\pgfqpoint{1.462408in}{2.404224in}}%
\pgfpathlineto{\pgfqpoint{1.465543in}{2.404224in}}%
\pgfpathlineto{\pgfqpoint{1.468207in}{2.407413in}}%
\pgfpathlineto{\pgfqpoint{1.468677in}{2.407976in}}%
\pgfpathlineto{\pgfqpoint{1.471812in}{2.407976in}}%
\pgfpathlineto{\pgfqpoint{1.474947in}{2.407976in}}%
\pgfpathlineto{\pgfqpoint{1.478082in}{2.407976in}}%
\pgfpathlineto{\pgfqpoint{1.480746in}{2.411165in}}%
\pgfpathlineto{\pgfqpoint{1.481216in}{2.411728in}}%
\pgfpathlineto{\pgfqpoint{1.484351in}{2.411728in}}%
\pgfpathlineto{\pgfqpoint{1.487486in}{2.411728in}}%
\pgfpathlineto{\pgfqpoint{1.490621in}{2.411728in}}%
\pgfpathlineto{\pgfqpoint{1.493285in}{2.414916in}}%
\pgfpathlineto{\pgfqpoint{1.493755in}{2.415479in}}%
\pgfpathlineto{\pgfqpoint{1.496890in}{2.415479in}}%
\pgfpathlineto{\pgfqpoint{1.500025in}{2.415479in}}%
\pgfpathlineto{\pgfqpoint{1.503159in}{2.415479in}}%
\pgfpathlineto{\pgfqpoint{1.505824in}{2.418668in}}%
\pgfpathlineto{\pgfqpoint{1.506294in}{2.419231in}}%
\pgfpathlineto{\pgfqpoint{1.509429in}{2.419231in}}%
\pgfpathlineto{\pgfqpoint{1.512564in}{2.419231in}}%
\pgfpathlineto{\pgfqpoint{1.515698in}{2.419231in}}%
\pgfpathlineto{\pgfqpoint{1.518363in}{2.422419in}}%
\pgfpathlineto{\pgfqpoint{1.518833in}{2.422982in}}%
\pgfpathlineto{\pgfqpoint{1.521968in}{2.422982in}}%
\pgfpathlineto{\pgfqpoint{1.525103in}{2.422982in}}%
\pgfpathlineto{\pgfqpoint{1.528237in}{2.422982in}}%
\pgfpathlineto{\pgfqpoint{1.531372in}{2.422982in}}%
\pgfpathlineto{\pgfqpoint{1.534037in}{2.426171in}}%
\pgfpathlineto{\pgfqpoint{1.534507in}{2.426734in}}%
\pgfpathlineto{\pgfqpoint{1.537642in}{2.426734in}}%
\pgfpathlineto{\pgfqpoint{1.540776in}{2.426734in}}%
\pgfpathlineto{\pgfqpoint{1.543911in}{2.426734in}}%
\pgfpathlineto{\pgfqpoint{1.546576in}{2.429923in}}%
\pgfpathlineto{\pgfqpoint{1.547046in}{2.430485in}}%
\pgfpathlineto{\pgfqpoint{1.550181in}{2.430485in}}%
\pgfpathlineto{\pgfqpoint{1.553315in}{2.430485in}}%
\pgfpathlineto{\pgfqpoint{1.556450in}{2.430485in}}%
\pgfpathlineto{\pgfqpoint{1.559115in}{2.433674in}}%
\pgfpathlineto{\pgfqpoint{1.559585in}{2.434237in}}%
\pgfpathlineto{\pgfqpoint{1.562720in}{2.434237in}}%
\pgfpathlineto{\pgfqpoint{1.565854in}{2.434237in}}%
\pgfpathlineto{\pgfqpoint{1.568989in}{2.434237in}}%
\pgfpathlineto{\pgfqpoint{1.571654in}{2.437426in}}%
\pgfpathlineto{\pgfqpoint{1.572124in}{2.437988in}}%
\pgfpathlineto{\pgfqpoint{1.575259in}{2.437988in}}%
\pgfpathlineto{\pgfqpoint{1.578393in}{2.437988in}}%
\pgfpathlineto{\pgfqpoint{1.581528in}{2.437988in}}%
\pgfpathlineto{\pgfqpoint{1.584663in}{2.437988in}}%
\pgfpathlineto{\pgfqpoint{1.587327in}{2.441177in}}%
\pgfpathlineto{\pgfqpoint{1.587798in}{2.441740in}}%
\pgfpathlineto{\pgfqpoint{1.590932in}{2.441740in}}%
\pgfpathlineto{\pgfqpoint{1.594067in}{2.441740in}}%
\pgfpathlineto{\pgfqpoint{1.597202in}{2.441740in}}%
\pgfpathlineto{\pgfqpoint{1.599866in}{2.444929in}}%
\pgfpathlineto{\pgfqpoint{1.600337in}{2.445491in}}%
\pgfpathlineto{\pgfqpoint{1.603471in}{2.445491in}}%
\pgfpathlineto{\pgfqpoint{1.606606in}{2.445491in}}%
\pgfpathlineto{\pgfqpoint{1.609741in}{2.445491in}}%
\pgfpathlineto{\pgfqpoint{1.612405in}{2.448680in}}%
\pgfpathlineto{\pgfqpoint{1.612875in}{2.449243in}}%
\pgfpathlineto{\pgfqpoint{1.616010in}{2.449243in}}%
\pgfpathlineto{\pgfqpoint{1.619145in}{2.449243in}}%
\pgfpathlineto{\pgfqpoint{1.622280in}{2.449243in}}%
\pgfpathlineto{\pgfqpoint{1.624944in}{2.452432in}}%
\pgfpathlineto{\pgfqpoint{1.625414in}{2.452994in}}%
\pgfpathlineto{\pgfqpoint{1.628549in}{2.452994in}}%
\pgfpathlineto{\pgfqpoint{1.631684in}{2.452994in}}%
\pgfpathlineto{\pgfqpoint{1.634819in}{2.452994in}}%
\pgfpathlineto{\pgfqpoint{1.637953in}{2.452994in}}%
\pgfpathlineto{\pgfqpoint{1.640618in}{2.456183in}}%
\pgfpathlineto{\pgfqpoint{1.641088in}{2.456746in}}%
\pgfpathlineto{\pgfqpoint{1.644223in}{2.456746in}}%
\pgfpathlineto{\pgfqpoint{1.647358in}{2.456746in}}%
\pgfpathlineto{\pgfqpoint{1.650492in}{2.456746in}}%
\pgfpathlineto{\pgfqpoint{1.653157in}{2.459935in}}%
\pgfpathlineto{\pgfqpoint{1.653627in}{2.460498in}}%
\pgfpathlineto{\pgfqpoint{1.656762in}{2.460498in}}%
\pgfpathlineto{\pgfqpoint{1.659897in}{2.460498in}}%
\pgfpathlineto{\pgfqpoint{1.663031in}{2.460498in}}%
\pgfpathlineto{\pgfqpoint{1.665696in}{2.463686in}}%
\pgfpathlineto{\pgfqpoint{1.666166in}{2.464249in}}%
\pgfpathlineto{\pgfqpoint{1.669301in}{2.464249in}}%
\pgfpathlineto{\pgfqpoint{1.672436in}{2.464249in}}%
\pgfpathlineto{\pgfqpoint{1.675570in}{2.464249in}}%
\pgfpathlineto{\pgfqpoint{1.678235in}{2.467438in}}%
\pgfpathlineto{\pgfqpoint{1.678705in}{2.468001in}}%
\pgfpathlineto{\pgfqpoint{1.681840in}{2.468001in}}%
\pgfpathlineto{\pgfqpoint{1.684975in}{2.468001in}}%
\pgfpathlineto{\pgfqpoint{1.688109in}{2.468001in}}%
\pgfpathlineto{\pgfqpoint{1.690774in}{2.471189in}}%
\pgfpathlineto{\pgfqpoint{1.691244in}{2.471752in}}%
\pgfpathlineto{\pgfqpoint{1.694379in}{2.471752in}}%
\pgfpathlineto{\pgfqpoint{1.697514in}{2.471752in}}%
\pgfpathlineto{\pgfqpoint{1.700648in}{2.471752in}}%
\pgfpathlineto{\pgfqpoint{1.703783in}{2.471752in}}%
\pgfpathlineto{\pgfqpoint{1.706448in}{2.474941in}}%
\pgfpathlineto{\pgfqpoint{1.706918in}{2.475504in}}%
\pgfpathlineto{\pgfqpoint{1.710052in}{2.475504in}}%
\pgfpathlineto{\pgfqpoint{1.713187in}{2.475504in}}%
\pgfpathlineto{\pgfqpoint{1.716322in}{2.475504in}}%
\pgfpathlineto{\pgfqpoint{1.718986in}{2.478692in}}%
\pgfpathlineto{\pgfqpoint{1.719457in}{2.479255in}}%
\pgfpathlineto{\pgfqpoint{1.722591in}{2.479255in}}%
\pgfpathlineto{\pgfqpoint{1.725726in}{2.479255in}}%
\pgfpathlineto{\pgfqpoint{1.728861in}{2.479255in}}%
\pgfpathlineto{\pgfqpoint{1.731525in}{2.482444in}}%
\pgfpathlineto{\pgfqpoint{1.731996in}{2.483007in}}%
\pgfpathlineto{\pgfqpoint{1.735130in}{2.483007in}}%
\pgfpathlineto{\pgfqpoint{1.738265in}{2.483007in}}%
\pgfpathlineto{\pgfqpoint{1.741400in}{2.483007in}}%
\pgfpathlineto{\pgfqpoint{1.744064in}{2.486196in}}%
\pgfpathlineto{\pgfqpoint{1.744535in}{2.486758in}}%
\pgfpathlineto{\pgfqpoint{1.747669in}{2.486758in}}%
\pgfpathlineto{\pgfqpoint{1.750804in}{2.486758in}}%
\pgfpathlineto{\pgfqpoint{1.753939in}{2.486758in}}%
\pgfpathlineto{\pgfqpoint{1.757074in}{2.486758in}}%
\pgfpathlineto{\pgfqpoint{1.759738in}{2.489947in}}%
\pgfpathlineto{\pgfqpoint{1.760208in}{2.490510in}}%
\pgfpathlineto{\pgfqpoint{1.763343in}{2.490510in}}%
\pgfpathlineto{\pgfqpoint{1.766478in}{2.490510in}}%
\pgfpathlineto{\pgfqpoint{1.769613in}{2.490510in}}%
\pgfpathlineto{\pgfqpoint{1.772277in}{2.493699in}}%
\pgfpathlineto{\pgfqpoint{1.772747in}{2.494261in}}%
\pgfpathlineto{\pgfqpoint{1.775882in}{2.494261in}}%
\pgfpathlineto{\pgfqpoint{1.779017in}{2.494261in}}%
\pgfpathlineto{\pgfqpoint{1.782152in}{2.494261in}}%
\pgfpathlineto{\pgfqpoint{1.784816in}{2.497450in}}%
\pgfpathlineto{\pgfqpoint{1.785286in}{2.498013in}}%
\pgfpathlineto{\pgfqpoint{1.788421in}{2.498013in}}%
\pgfpathlineto{\pgfqpoint{1.791556in}{2.498013in}}%
\pgfpathlineto{\pgfqpoint{1.794691in}{2.498013in}}%
\pgfpathlineto{\pgfqpoint{1.797355in}{2.501202in}}%
\pgfpathlineto{\pgfqpoint{1.797825in}{2.501764in}}%
\pgfpathlineto{\pgfqpoint{1.800960in}{2.501764in}}%
\pgfpathlineto{\pgfqpoint{1.804095in}{2.501764in}}%
\pgfpathlineto{\pgfqpoint{1.807229in}{2.501764in}}%
\pgfpathlineto{\pgfqpoint{1.810364in}{2.501764in}}%
\pgfpathlineto{\pgfqpoint{1.813029in}{2.504953in}}%
\pgfpathlineto{\pgfqpoint{1.813499in}{2.505516in}}%
\pgfpathlineto{\pgfqpoint{1.816634in}{2.505516in}}%
\pgfpathlineto{\pgfqpoint{1.819768in}{2.505516in}}%
\pgfpathlineto{\pgfqpoint{1.822903in}{2.505516in}}%
\pgfpathlineto{\pgfqpoint{1.825568in}{2.508705in}}%
\pgfpathlineto{\pgfqpoint{1.826038in}{2.509267in}}%
\pgfpathlineto{\pgfqpoint{1.829173in}{2.509267in}}%
\pgfpathlineto{\pgfqpoint{1.832307in}{2.509267in}}%
\pgfpathlineto{\pgfqpoint{1.835442in}{2.509267in}}%
\pgfpathlineto{\pgfqpoint{1.838107in}{2.512456in}}%
\pgfpathlineto{\pgfqpoint{1.838577in}{2.513019in}}%
\pgfpathlineto{\pgfqpoint{1.841712in}{2.513019in}}%
\pgfpathlineto{\pgfqpoint{1.844846in}{2.513019in}}%
\pgfpathlineto{\pgfqpoint{1.847981in}{2.513019in}}%
\pgfpathlineto{\pgfqpoint{1.850646in}{2.516208in}}%
\pgfpathlineto{\pgfqpoint{1.851116in}{2.516771in}}%
\pgfpathlineto{\pgfqpoint{1.854251in}{2.516771in}}%
\pgfpathlineto{\pgfqpoint{1.857385in}{2.516771in}}%
\pgfpathlineto{\pgfqpoint{1.860520in}{2.516771in}}%
\pgfpathlineto{\pgfqpoint{1.863185in}{2.519959in}}%
\pgfpathlineto{\pgfqpoint{1.863655in}{2.520522in}}%
\pgfpathlineto{\pgfqpoint{1.866790in}{2.520522in}}%
\pgfpathlineto{\pgfqpoint{1.869924in}{2.520522in}}%
\pgfpathlineto{\pgfqpoint{1.873059in}{2.520522in}}%
\pgfpathlineto{\pgfqpoint{1.876194in}{2.520522in}}%
\pgfpathlineto{\pgfqpoint{1.878858in}{2.523711in}}%
\pgfpathlineto{\pgfqpoint{1.879329in}{2.524274in}}%
\pgfpathlineto{\pgfqpoint{1.882463in}{2.524274in}}%
\pgfpathlineto{\pgfqpoint{1.885598in}{2.524274in}}%
\pgfpathlineto{\pgfqpoint{1.888733in}{2.524274in}}%
\pgfpathlineto{\pgfqpoint{1.891397in}{2.527462in}}%
\pgfpathlineto{\pgfqpoint{1.891868in}{2.528025in}}%
\pgfpathlineto{\pgfqpoint{1.895002in}{2.528025in}}%
\pgfpathlineto{\pgfqpoint{1.898137in}{2.528025in}}%
\pgfpathlineto{\pgfqpoint{1.901272in}{2.528025in}}%
\pgfpathlineto{\pgfqpoint{1.903936in}{2.531214in}}%
\pgfpathlineto{\pgfqpoint{1.904407in}{2.531777in}}%
\pgfpathlineto{\pgfqpoint{1.907541in}{2.531777in}}%
\pgfpathlineto{\pgfqpoint{1.910676in}{2.531777in}}%
\pgfpathlineto{\pgfqpoint{1.913811in}{2.531777in}}%
\pgfpathlineto{\pgfqpoint{1.916475in}{2.534965in}}%
\pgfpathlineto{\pgfqpoint{1.916945in}{2.535528in}}%
\pgfpathlineto{\pgfqpoint{1.920080in}{2.535528in}}%
\pgfpathlineto{\pgfqpoint{1.923215in}{2.535528in}}%
\pgfpathlineto{\pgfqpoint{1.926350in}{2.535528in}}%
\pgfpathlineto{\pgfqpoint{1.929484in}{2.535528in}}%
\pgfpathlineto{\pgfqpoint{1.932149in}{2.538717in}}%
\pgfpathlineto{\pgfqpoint{1.932619in}{2.539280in}}%
\pgfpathlineto{\pgfqpoint{1.935754in}{2.539280in}}%
\pgfpathlineto{\pgfqpoint{1.938889in}{2.539280in}}%
\pgfpathlineto{\pgfqpoint{1.942023in}{2.539280in}}%
\pgfpathlineto{\pgfqpoint{1.944688in}{2.542469in}}%
\pgfpathlineto{\pgfqpoint{1.945158in}{2.543031in}}%
\pgfpathlineto{\pgfqpoint{1.948293in}{2.543031in}}%
\pgfpathlineto{\pgfqpoint{1.951428in}{2.543031in}}%
\pgfpathlineto{\pgfqpoint{1.954562in}{2.543031in}}%
\pgfpathlineto{\pgfqpoint{1.957227in}{2.546220in}}%
\pgfpathlineto{\pgfqpoint{1.957697in}{2.546783in}}%
\pgfpathlineto{\pgfqpoint{1.960832in}{2.546783in}}%
\pgfpathlineto{\pgfqpoint{1.963967in}{2.546783in}}%
\pgfpathlineto{\pgfqpoint{1.967101in}{2.546783in}}%
\pgfpathlineto{\pgfqpoint{1.969766in}{2.549972in}}%
\pgfpathlineto{\pgfqpoint{1.970236in}{2.550534in}}%
\pgfpathlineto{\pgfqpoint{1.973371in}{2.550534in}}%
\pgfpathlineto{\pgfqpoint{1.976506in}{2.550534in}}%
\pgfpathlineto{\pgfqpoint{1.979640in}{2.550534in}}%
\pgfpathlineto{\pgfqpoint{1.982305in}{2.553723in}}%
\pgfpathlineto{\pgfqpoint{1.982775in}{2.554286in}}%
\pgfpathlineto{\pgfqpoint{1.985910in}{2.554286in}}%
\pgfpathlineto{\pgfqpoint{1.989045in}{2.554286in}}%
\pgfpathlineto{\pgfqpoint{1.992179in}{2.554286in}}%
\pgfpathlineto{\pgfqpoint{1.995314in}{2.554286in}}%
\pgfpathlineto{\pgfqpoint{1.997979in}{2.557475in}}%
\pgfpathlineto{\pgfqpoint{1.998449in}{2.558037in}}%
\pgfpathlineto{\pgfqpoint{2.001584in}{2.558037in}}%
\pgfpathlineto{\pgfqpoint{2.004718in}{2.558037in}}%
\pgfpathlineto{\pgfqpoint{2.007853in}{2.558037in}}%
\pgfpathlineto{\pgfqpoint{2.010518in}{2.561226in}}%
\pgfpathlineto{\pgfqpoint{2.010988in}{2.561789in}}%
\pgfpathlineto{\pgfqpoint{2.014122in}{2.561789in}}%
\pgfpathlineto{\pgfqpoint{2.017257in}{2.561789in}}%
\pgfpathlineto{\pgfqpoint{2.020392in}{2.561789in}}%
\pgfpathlineto{\pgfqpoint{2.023056in}{2.564978in}}%
\pgfpathlineto{\pgfqpoint{2.023527in}{2.565540in}}%
\pgfpathlineto{\pgfqpoint{2.026661in}{2.565540in}}%
\pgfpathlineto{\pgfqpoint{2.029796in}{2.565540in}}%
\pgfpathlineto{\pgfqpoint{2.032931in}{2.565540in}}%
\pgfpathlineto{\pgfqpoint{2.035595in}{2.568729in}}%
\pgfpathlineto{\pgfqpoint{2.036066in}{2.569292in}}%
\pgfpathlineto{\pgfqpoint{2.039200in}{2.569292in}}%
\pgfpathlineto{\pgfqpoint{2.042335in}{2.569292in}}%
\pgfpathlineto{\pgfqpoint{2.045470in}{2.569292in}}%
\pgfpathlineto{\pgfqpoint{2.048605in}{2.569292in}}%
\pgfpathlineto{\pgfqpoint{2.051269in}{2.572481in}}%
\pgfpathlineto{\pgfqpoint{2.051739in}{2.573044in}}%
\pgfpathlineto{\pgfqpoint{2.054874in}{2.573044in}}%
\pgfpathlineto{\pgfqpoint{2.058009in}{2.573044in}}%
\pgfpathlineto{\pgfqpoint{2.061144in}{2.573044in}}%
\pgfpathlineto{\pgfqpoint{2.063808in}{2.576232in}}%
\pgfpathlineto{\pgfqpoint{2.064278in}{2.576795in}}%
\pgfpathlineto{\pgfqpoint{2.067413in}{2.576795in}}%
\pgfpathlineto{\pgfqpoint{2.070548in}{2.576795in}}%
\pgfpathlineto{\pgfqpoint{2.073683in}{2.576795in}}%
\pgfpathlineto{\pgfqpoint{2.076347in}{2.579984in}}%
\pgfpathlineto{\pgfqpoint{2.076817in}{2.580547in}}%
\pgfpathlineto{\pgfqpoint{2.079952in}{2.580547in}}%
\pgfpathlineto{\pgfqpoint{2.083087in}{2.580547in}}%
\pgfpathlineto{\pgfqpoint{2.086222in}{2.580547in}}%
\pgfpathlineto{\pgfqpoint{2.088886in}{2.583735in}}%
\pgfpathlineto{\pgfqpoint{2.089356in}{2.584298in}}%
\pgfpathlineto{\pgfqpoint{2.092491in}{2.584298in}}%
\pgfpathlineto{\pgfqpoint{2.095626in}{2.584298in}}%
\pgfpathlineto{\pgfqpoint{2.098761in}{2.584298in}}%
\pgfpathlineto{\pgfqpoint{2.101895in}{2.584298in}}%
\pgfpathlineto{\pgfqpoint{2.104560in}{2.587487in}}%
\pgfpathlineto{\pgfqpoint{2.105030in}{2.588050in}}%
\pgfpathlineto{\pgfqpoint{2.108165in}{2.588050in}}%
\pgfpathlineto{\pgfqpoint{2.111299in}{2.588050in}}%
\pgfpathlineto{\pgfqpoint{2.114434in}{2.588050in}}%
\pgfpathlineto{\pgfqpoint{2.117099in}{2.591238in}}%
\pgfpathlineto{\pgfqpoint{2.117569in}{2.591801in}}%
\pgfpathlineto{\pgfqpoint{2.120704in}{2.591801in}}%
\pgfpathlineto{\pgfqpoint{2.123838in}{2.591801in}}%
\pgfpathlineto{\pgfqpoint{2.126973in}{2.591801in}}%
\pgfpathlineto{\pgfqpoint{2.129638in}{2.594990in}}%
\pgfpathlineto{\pgfqpoint{2.130108in}{2.595553in}}%
\pgfpathlineto{\pgfqpoint{2.133243in}{2.595553in}}%
\pgfpathlineto{\pgfqpoint{2.136377in}{2.595553in}}%
\pgfpathlineto{\pgfqpoint{2.139512in}{2.595553in}}%
\pgfpathlineto{\pgfqpoint{2.142177in}{2.598742in}}%
\pgfpathlineto{\pgfqpoint{2.142647in}{2.599304in}}%
\pgfpathlineto{\pgfqpoint{2.145782in}{2.599304in}}%
\pgfpathlineto{\pgfqpoint{2.148916in}{2.599304in}}%
\pgfpathlineto{\pgfqpoint{2.152051in}{2.599304in}}%
\pgfpathlineto{\pgfqpoint{2.154716in}{2.602493in}}%
\pgfpathlineto{\pgfqpoint{2.155186in}{2.603056in}}%
\pgfpathlineto{\pgfqpoint{2.158321in}{2.603056in}}%
\pgfpathlineto{\pgfqpoint{2.161455in}{2.603056in}}%
\pgfpathlineto{\pgfqpoint{2.164590in}{2.603056in}}%
\pgfpathlineto{\pgfqpoint{2.167725in}{2.603056in}}%
\pgfpathlineto{\pgfqpoint{2.170389in}{2.606245in}}%
\pgfpathlineto{\pgfqpoint{2.170860in}{2.606807in}}%
\pgfpathlineto{\pgfqpoint{2.173994in}{2.606807in}}%
\pgfpathlineto{\pgfqpoint{2.177129in}{2.606807in}}%
\pgfpathlineto{\pgfqpoint{2.180264in}{2.606807in}}%
\pgfpathlineto{\pgfqpoint{2.182928in}{2.609996in}}%
\pgfpathlineto{\pgfqpoint{2.183399in}{2.610559in}}%
\pgfpathlineto{\pgfqpoint{2.186533in}{2.610559in}}%
\pgfpathlineto{\pgfqpoint{2.189668in}{2.610559in}}%
\pgfpathlineto{\pgfqpoint{2.192803in}{2.610559in}}%
\pgfpathlineto{\pgfqpoint{2.195467in}{2.613748in}}%
\pgfpathlineto{\pgfqpoint{2.195938in}{2.614310in}}%
\pgfpathlineto{\pgfqpoint{2.199072in}{2.614310in}}%
\pgfpathlineto{\pgfqpoint{2.202207in}{2.614310in}}%
\pgfpathlineto{\pgfqpoint{2.205342in}{2.614310in}}%
\pgfpathlineto{\pgfqpoint{2.208006in}{2.617499in}}%
\pgfpathlineto{\pgfqpoint{2.208477in}{2.618062in}}%
\pgfpathlineto{\pgfqpoint{2.211611in}{2.618062in}}%
\pgfpathlineto{\pgfqpoint{2.214746in}{2.618062in}}%
\pgfpathlineto{\pgfqpoint{2.217881in}{2.618062in}}%
\pgfpathlineto{\pgfqpoint{2.221015in}{2.618062in}}%
\pgfpathlineto{\pgfqpoint{2.223680in}{2.621251in}}%
\pgfpathlineto{\pgfqpoint{2.224150in}{2.621813in}}%
\pgfpathlineto{\pgfqpoint{2.227285in}{2.621813in}}%
\pgfpathlineto{\pgfqpoint{2.230420in}{2.621813in}}%
\pgfpathlineto{\pgfqpoint{2.233554in}{2.621813in}}%
\pgfpathlineto{\pgfqpoint{2.236219in}{2.625002in}}%
\pgfpathlineto{\pgfqpoint{2.236689in}{2.625565in}}%
\pgfpathlineto{\pgfqpoint{2.239824in}{2.625565in}}%
\pgfpathlineto{\pgfqpoint{2.242959in}{2.625565in}}%
\pgfpathlineto{\pgfqpoint{2.246093in}{2.625565in}}%
\pgfpathlineto{\pgfqpoint{2.248758in}{2.628754in}}%
\pgfpathlineto{\pgfqpoint{2.249228in}{2.629317in}}%
\pgfpathlineto{\pgfqpoint{2.252363in}{2.629317in}}%
\pgfpathlineto{\pgfqpoint{2.255498in}{2.629317in}}%
\pgfpathlineto{\pgfqpoint{2.258632in}{2.629317in}}%
\pgfpathlineto{\pgfqpoint{2.261297in}{2.632505in}}%
\pgfpathlineto{\pgfqpoint{2.261767in}{2.633068in}}%
\pgfpathlineto{\pgfqpoint{2.264902in}{2.633068in}}%
\pgfpathlineto{\pgfqpoint{2.268037in}{2.633068in}}%
\pgfpathlineto{\pgfqpoint{2.271171in}{2.633068in}}%
\pgfpathlineto{\pgfqpoint{2.274306in}{2.633068in}}%
\pgfpathlineto{\pgfqpoint{2.276971in}{2.636257in}}%
\pgfpathlineto{\pgfqpoint{2.277441in}{2.636820in}}%
\pgfpathlineto{\pgfqpoint{2.280576in}{2.636820in}}%
\pgfpathlineto{\pgfqpoint{2.283710in}{2.636820in}}%
\pgfpathlineto{\pgfqpoint{2.286845in}{2.636820in}}%
\pgfpathlineto{\pgfqpoint{2.289510in}{2.640008in}}%
\pgfpathlineto{\pgfqpoint{2.289980in}{2.640571in}}%
\pgfpathlineto{\pgfqpoint{2.293115in}{2.640571in}}%
\pgfpathlineto{\pgfqpoint{2.296249in}{2.640571in}}%
\pgfpathlineto{\pgfqpoint{2.299384in}{2.640571in}}%
\pgfpathlineto{\pgfqpoint{2.302049in}{2.643760in}}%
\pgfpathlineto{\pgfqpoint{2.302519in}{2.644323in}}%
\pgfpathlineto{\pgfqpoint{2.305654in}{2.644323in}}%
\pgfpathlineto{\pgfqpoint{2.308788in}{2.644323in}}%
\pgfpathlineto{\pgfqpoint{2.311923in}{2.644323in}}%
\pgfpathlineto{\pgfqpoint{2.314588in}{2.647512in}}%
\pgfpathlineto{\pgfqpoint{2.315058in}{2.648074in}}%
\pgfpathlineto{\pgfqpoint{2.318192in}{2.648074in}}%
\pgfpathlineto{\pgfqpoint{2.321327in}{2.648074in}}%
\pgfpathlineto{\pgfqpoint{2.324462in}{2.648074in}}%
\pgfpathlineto{\pgfqpoint{2.327127in}{2.651263in}}%
\pgfpathlineto{\pgfqpoint{2.327597in}{2.651826in}}%
\pgfpathlineto{\pgfqpoint{2.330731in}{2.651826in}}%
\pgfpathlineto{\pgfqpoint{2.333866in}{2.651826in}}%
\pgfpathlineto{\pgfqpoint{2.337001in}{2.651826in}}%
\pgfpathlineto{\pgfqpoint{2.340136in}{2.651826in}}%
\pgfpathlineto{\pgfqpoint{2.342800in}{2.655015in}}%
\pgfpathlineto{\pgfqpoint{2.343270in}{2.655577in}}%
\pgfpathlineto{\pgfqpoint{2.346405in}{2.655577in}}%
\pgfpathlineto{\pgfqpoint{2.349540in}{2.655577in}}%
\pgfpathlineto{\pgfqpoint{2.352675in}{2.655577in}}%
\pgfpathlineto{\pgfqpoint{2.355339in}{2.658766in}}%
\pgfpathlineto{\pgfqpoint{2.355809in}{2.659329in}}%
\pgfpathlineto{\pgfqpoint{2.358944in}{2.659329in}}%
\pgfpathlineto{\pgfqpoint{2.362079in}{2.659329in}}%
\pgfpathlineto{\pgfqpoint{2.365214in}{2.659329in}}%
\pgfpathlineto{\pgfqpoint{2.367878in}{2.662518in}}%
\pgfpathlineto{\pgfqpoint{2.368348in}{2.663080in}}%
\pgfpathlineto{\pgfqpoint{2.371483in}{2.663080in}}%
\pgfpathlineto{\pgfqpoint{2.374618in}{2.663080in}}%
\pgfpathlineto{\pgfqpoint{2.377753in}{2.663080in}}%
\pgfpathlineto{\pgfqpoint{2.380417in}{2.666269in}}%
\pgfpathlineto{\pgfqpoint{2.380887in}{2.666832in}}%
\pgfpathlineto{\pgfqpoint{2.384022in}{2.666832in}}%
\pgfpathlineto{\pgfqpoint{2.387157in}{2.666832in}}%
\pgfpathlineto{\pgfqpoint{2.390292in}{2.666832in}}%
\pgfpathlineto{\pgfqpoint{2.393426in}{2.666832in}}%
\pgfpathlineto{\pgfqpoint{2.396091in}{2.670021in}}%
\pgfpathlineto{\pgfqpoint{2.396561in}{2.670583in}}%
\pgfpathlineto{\pgfqpoint{2.399696in}{2.670583in}}%
\pgfpathlineto{\pgfqpoint{2.402831in}{2.670583in}}%
\pgfpathlineto{\pgfqpoint{2.405965in}{2.670583in}}%
\pgfpathlineto{\pgfqpoint{2.408630in}{2.673772in}}%
\pgfpathlineto{\pgfqpoint{2.409100in}{2.674335in}}%
\pgfpathlineto{\pgfqpoint{2.412235in}{2.674335in}}%
\pgfpathlineto{\pgfqpoint{2.415369in}{2.674335in}}%
\pgfpathlineto{\pgfqpoint{2.418504in}{2.674335in}}%
\pgfpathlineto{\pgfqpoint{2.421169in}{2.677524in}}%
\pgfpathlineto{\pgfqpoint{2.421639in}{2.678087in}}%
\pgfpathlineto{\pgfqpoint{2.424774in}{2.678087in}}%
\pgfpathlineto{\pgfqpoint{2.427908in}{2.678087in}}%
\pgfpathlineto{\pgfqpoint{2.431043in}{2.678087in}}%
\pgfpathlineto{\pgfqpoint{2.433708in}{2.681275in}}%
\pgfpathlineto{\pgfqpoint{2.434178in}{2.681838in}}%
\pgfpathlineto{\pgfqpoint{2.437313in}{2.681838in}}%
\pgfpathlineto{\pgfqpoint{2.440447in}{2.681838in}}%
\pgfpathlineto{\pgfqpoint{2.443582in}{2.681838in}}%
\pgfpathlineto{\pgfqpoint{2.446717in}{2.681838in}}%
\pgfpathlineto{\pgfqpoint{2.449381in}{2.685027in}}%
\pgfpathlineto{\pgfqpoint{2.449852in}{2.685590in}}%
\pgfpathlineto{\pgfqpoint{2.452986in}{2.685590in}}%
\pgfpathlineto{\pgfqpoint{2.456121in}{2.685590in}}%
\pgfpathlineto{\pgfqpoint{2.459256in}{2.685590in}}%
\pgfpathlineto{\pgfqpoint{2.461920in}{2.688778in}}%
\pgfpathlineto{\pgfqpoint{2.462391in}{2.689341in}}%
\pgfpathlineto{\pgfqpoint{2.465525in}{2.689341in}}%
\pgfpathlineto{\pgfqpoint{2.468660in}{2.689341in}}%
\pgfpathlineto{\pgfqpoint{2.471795in}{2.689341in}}%
\pgfpathlineto{\pgfqpoint{2.474459in}{2.692530in}}%
\pgfpathlineto{\pgfqpoint{2.474930in}{2.693093in}}%
\pgfpathlineto{\pgfqpoint{2.478064in}{2.693093in}}%
\pgfpathlineto{\pgfqpoint{2.481199in}{2.693093in}}%
\pgfpathlineto{\pgfqpoint{2.484334in}{2.693093in}}%
\pgfpathlineto{\pgfqpoint{2.486998in}{2.696281in}}%
\pgfpathlineto{\pgfqpoint{2.487469in}{2.696844in}}%
\pgfpathlineto{\pgfqpoint{2.490603in}{2.696844in}}%
\pgfpathlineto{\pgfqpoint{2.493738in}{2.696844in}}%
\pgfpathlineto{\pgfqpoint{2.496873in}{2.696844in}}%
\pgfpathlineto{\pgfqpoint{2.499537in}{2.700033in}}%
\pgfpathlineto{\pgfqpoint{2.500008in}{2.700596in}}%
\pgfpathlineto{\pgfqpoint{2.503142in}{2.700596in}}%
\pgfpathlineto{\pgfqpoint{2.506277in}{2.700596in}}%
\pgfpathlineto{\pgfqpoint{2.509412in}{2.700596in}}%
\pgfpathlineto{\pgfqpoint{2.512547in}{2.700596in}}%
\pgfpathlineto{\pgfqpoint{2.515211in}{2.703785in}}%
\pgfpathlineto{\pgfqpoint{2.515681in}{2.704347in}}%
\pgfpathlineto{\pgfqpoint{2.518816in}{2.704347in}}%
\pgfpathlineto{\pgfqpoint{2.521951in}{2.704347in}}%
\pgfpathlineto{\pgfqpoint{2.525085in}{2.704347in}}%
\pgfpathlineto{\pgfqpoint{2.527750in}{2.707536in}}%
\pgfpathlineto{\pgfqpoint{2.528220in}{2.708099in}}%
\pgfpathlineto{\pgfqpoint{2.531355in}{2.708099in}}%
\pgfpathlineto{\pgfqpoint{2.534490in}{2.708099in}}%
\pgfpathlineto{\pgfqpoint{2.537624in}{2.708099in}}%
\pgfpathlineto{\pgfqpoint{2.540289in}{2.711288in}}%
\pgfpathlineto{\pgfqpoint{2.540759in}{2.711850in}}%
\pgfpathlineto{\pgfqpoint{2.543894in}{2.711850in}}%
\pgfpathlineto{\pgfqpoint{2.547029in}{2.711850in}}%
\pgfpathlineto{\pgfqpoint{2.550163in}{2.711850in}}%
\pgfpathlineto{\pgfqpoint{2.552828in}{2.715039in}}%
\pgfpathlineto{\pgfqpoint{2.553298in}{2.715602in}}%
\pgfpathlineto{\pgfqpoint{2.556433in}{2.715602in}}%
\pgfpathlineto{\pgfqpoint{2.559568in}{2.715602in}}%
\pgfpathlineto{\pgfqpoint{2.562702in}{2.715602in}}%
\pgfpathlineto{\pgfqpoint{2.565837in}{2.715602in}}%
\pgfpathlineto{\pgfqpoint{2.568502in}{2.718791in}}%
\pgfpathlineto{\pgfqpoint{2.568972in}{2.719353in}}%
\pgfpathlineto{\pgfqpoint{2.572107in}{2.719353in}}%
\pgfpathlineto{\pgfqpoint{2.575241in}{2.719353in}}%
\pgfpathlineto{\pgfqpoint{2.578376in}{2.719353in}}%
\pgfpathlineto{\pgfqpoint{2.581041in}{2.722542in}}%
\pgfpathlineto{\pgfqpoint{2.581511in}{2.723105in}}%
\pgfpathlineto{\pgfqpoint{2.584646in}{2.723105in}}%
\pgfpathlineto{\pgfqpoint{2.587780in}{2.723105in}}%
\pgfpathlineto{\pgfqpoint{2.590915in}{2.723105in}}%
\pgfpathlineto{\pgfqpoint{2.593580in}{2.726294in}}%
\pgfpathlineto{\pgfqpoint{2.594050in}{2.726856in}}%
\pgfpathlineto{\pgfqpoint{2.597185in}{2.726856in}}%
\pgfpathlineto{\pgfqpoint{2.600319in}{2.726856in}}%
\pgfpathlineto{\pgfqpoint{2.603454in}{2.726856in}}%
\pgfpathlineto{\pgfqpoint{2.606119in}{2.730045in}}%
\pgfpathlineto{\pgfqpoint{2.606589in}{2.730608in}}%
\pgfpathlineto{\pgfqpoint{2.609724in}{2.730608in}}%
\pgfpathlineto{\pgfqpoint{2.612858in}{2.730608in}}%
\pgfpathlineto{\pgfqpoint{2.615993in}{2.730608in}}%
\pgfpathlineto{\pgfqpoint{2.618658in}{2.733797in}}%
\pgfpathlineto{\pgfqpoint{2.619128in}{2.734360in}}%
\pgfpathlineto{\pgfqpoint{2.622262in}{2.734360in}}%
\pgfpathlineto{\pgfqpoint{2.625397in}{2.734360in}}%
\pgfpathlineto{\pgfqpoint{2.628532in}{2.734360in}}%
\pgfpathlineto{\pgfqpoint{2.631667in}{2.734360in}}%
\pgfpathlineto{\pgfqpoint{2.634331in}{2.737548in}}%
\pgfpathlineto{\pgfqpoint{2.634801in}{2.738111in}}%
\pgfpathlineto{\pgfqpoint{2.637936in}{2.738111in}}%
\pgfpathlineto{\pgfqpoint{2.641071in}{2.738111in}}%
\pgfpathlineto{\pgfqpoint{2.644206in}{2.738111in}}%
\pgfpathlineto{\pgfqpoint{2.646870in}{2.741300in}}%
\pgfpathlineto{\pgfqpoint{2.647340in}{2.741863in}}%
\pgfpathlineto{\pgfqpoint{2.650475in}{2.741863in}}%
\pgfpathlineto{\pgfqpoint{2.653610in}{2.741863in}}%
\pgfpathlineto{\pgfqpoint{2.656745in}{2.741863in}}%
\pgfpathlineto{\pgfqpoint{2.659409in}{2.745051in}}%
\pgfpathlineto{\pgfqpoint{2.659879in}{2.745614in}}%
\pgfpathlineto{\pgfqpoint{2.663014in}{2.745614in}}%
\pgfpathlineto{\pgfqpoint{2.666149in}{2.745614in}}%
\pgfpathlineto{\pgfqpoint{2.669284in}{2.745614in}}%
\pgfpathlineto{\pgfqpoint{2.672418in}{2.745614in}}%
\pgfpathlineto{\pgfqpoint{2.672889in}{2.745051in}}%
\pgfpathlineto{\pgfqpoint{2.675553in}{2.741863in}}%
\pgfpathlineto{\pgfqpoint{2.676023in}{2.741300in}}%
\pgfpathlineto{\pgfqpoint{2.678688in}{2.738111in}}%
\pgfpathlineto{\pgfqpoint{2.679158in}{2.737548in}}%
\pgfpathlineto{\pgfqpoint{2.681823in}{2.734360in}}%
\pgfpathlineto{\pgfqpoint{2.682293in}{2.733797in}}%
\pgfpathlineto{\pgfqpoint{2.684957in}{2.730608in}}%
\pgfpathlineto{\pgfqpoint{2.685428in}{2.730045in}}%
\pgfpathlineto{\pgfqpoint{2.688092in}{2.726856in}}%
\pgfpathlineto{\pgfqpoint{2.688562in}{2.726294in}}%
\pgfpathlineto{\pgfqpoint{2.691227in}{2.723105in}}%
\pgfpathlineto{\pgfqpoint{2.691697in}{2.722542in}}%
\pgfpathlineto{\pgfqpoint{2.694362in}{2.719353in}}%
\pgfpathlineto{\pgfqpoint{2.694832in}{2.718791in}}%
\pgfpathlineto{\pgfqpoint{2.697496in}{2.715602in}}%
\pgfpathlineto{\pgfqpoint{2.697967in}{2.715039in}}%
\pgfpathlineto{\pgfqpoint{2.700631in}{2.711850in}}%
\pgfpathlineto{\pgfqpoint{2.701101in}{2.711288in}}%
\pgfpathlineto{\pgfqpoint{2.703766in}{2.708099in}}%
\pgfpathlineto{\pgfqpoint{2.704236in}{2.707536in}}%
\pgfpathlineto{\pgfqpoint{2.706901in}{2.704347in}}%
\pgfpathlineto{\pgfqpoint{2.707371in}{2.703785in}}%
\pgfpathlineto{\pgfqpoint{2.710035in}{2.700596in}}%
\pgfpathlineto{\pgfqpoint{2.710505in}{2.700033in}}%
\pgfpathlineto{\pgfqpoint{2.710505in}{2.696281in}}%
\pgfpathlineto{\pgfqpoint{2.713170in}{2.693093in}}%
\pgfpathlineto{\pgfqpoint{2.713640in}{2.692530in}}%
\pgfpathlineto{\pgfqpoint{2.716305in}{2.689341in}}%
\pgfpathlineto{\pgfqpoint{2.716775in}{2.688778in}}%
\pgfpathlineto{\pgfqpoint{2.719439in}{2.685590in}}%
\pgfpathlineto{\pgfqpoint{2.719910in}{2.685027in}}%
\pgfpathlineto{\pgfqpoint{2.722574in}{2.681838in}}%
\pgfpathlineto{\pgfqpoint{2.723044in}{2.681275in}}%
\pgfpathlineto{\pgfqpoint{2.725709in}{2.678087in}}%
\pgfpathlineto{\pgfqpoint{2.726179in}{2.677524in}}%
\pgfpathlineto{\pgfqpoint{2.728844in}{2.674335in}}%
\pgfpathlineto{\pgfqpoint{2.729314in}{2.673772in}}%
\pgfpathlineto{\pgfqpoint{2.731978in}{2.670583in}}%
\pgfpathlineto{\pgfqpoint{2.732449in}{2.670021in}}%
\pgfpathlineto{\pgfqpoint{2.735113in}{2.666832in}}%
\pgfpathlineto{\pgfqpoint{2.735583in}{2.666269in}}%
\pgfpathlineto{\pgfqpoint{2.738248in}{2.663080in}}%
\pgfpathlineto{\pgfqpoint{2.738718in}{2.662518in}}%
\pgfpathlineto{\pgfqpoint{2.741383in}{2.659329in}}%
\pgfpathlineto{\pgfqpoint{2.741853in}{2.658766in}}%
\pgfpathlineto{\pgfqpoint{2.744517in}{2.655577in}}%
\pgfpathlineto{\pgfqpoint{2.744988in}{2.655015in}}%
\pgfpathlineto{\pgfqpoint{2.747652in}{2.651826in}}%
\pgfpathlineto{\pgfqpoint{2.748122in}{2.651263in}}%
\pgfpathlineto{\pgfqpoint{2.750787in}{2.648074in}}%
\pgfpathlineto{\pgfqpoint{2.751257in}{2.647512in}}%
\pgfpathlineto{\pgfqpoint{2.753922in}{2.644323in}}%
\pgfpathlineto{\pgfqpoint{2.754392in}{2.643760in}}%
\pgfpathlineto{\pgfqpoint{2.757056in}{2.640571in}}%
\pgfpathlineto{\pgfqpoint{2.757527in}{2.640008in}}%
\pgfpathlineto{\pgfqpoint{2.760191in}{2.636820in}}%
\pgfpathlineto{\pgfqpoint{2.760661in}{2.636257in}}%
\pgfpathlineto{\pgfqpoint{2.760661in}{2.632505in}}%
\pgfpathlineto{\pgfqpoint{2.763326in}{2.629317in}}%
\pgfpathlineto{\pgfqpoint{2.763796in}{2.628754in}}%
\pgfpathlineto{\pgfqpoint{2.766461in}{2.625565in}}%
\pgfpathlineto{\pgfqpoint{2.766931in}{2.625002in}}%
\pgfpathlineto{\pgfqpoint{2.769595in}{2.621813in}}%
\pgfpathlineto{\pgfqpoint{2.770066in}{2.621251in}}%
\pgfpathlineto{\pgfqpoint{2.772730in}{2.618062in}}%
\pgfpathlineto{\pgfqpoint{2.773200in}{2.617499in}}%
\pgfpathlineto{\pgfqpoint{2.775865in}{2.614310in}}%
\pgfpathlineto{\pgfqpoint{2.776335in}{2.613748in}}%
\pgfpathlineto{\pgfqpoint{2.779000in}{2.610559in}}%
\pgfpathlineto{\pgfqpoint{2.779470in}{2.609996in}}%
\pgfpathlineto{\pgfqpoint{2.782134in}{2.606807in}}%
\pgfpathlineto{\pgfqpoint{2.782605in}{2.606245in}}%
\pgfpathlineto{\pgfqpoint{2.785269in}{2.603056in}}%
\pgfpathlineto{\pgfqpoint{2.785739in}{2.602493in}}%
\pgfpathlineto{\pgfqpoint{2.788404in}{2.599304in}}%
\pgfpathlineto{\pgfqpoint{2.788874in}{2.598742in}}%
\pgfpathlineto{\pgfqpoint{2.791539in}{2.595553in}}%
\pgfpathlineto{\pgfqpoint{2.792009in}{2.594990in}}%
\pgfpathlineto{\pgfqpoint{2.794673in}{2.591801in}}%
\pgfpathlineto{\pgfqpoint{2.795144in}{2.591238in}}%
\pgfpathlineto{\pgfqpoint{2.797808in}{2.588050in}}%
\pgfpathlineto{\pgfqpoint{2.798278in}{2.587487in}}%
\pgfpathlineto{\pgfqpoint{2.800943in}{2.584298in}}%
\pgfpathlineto{\pgfqpoint{2.801413in}{2.583735in}}%
\pgfpathlineto{\pgfqpoint{2.804078in}{2.580547in}}%
\pgfpathlineto{\pgfqpoint{2.804548in}{2.579984in}}%
\pgfpathlineto{\pgfqpoint{2.807212in}{2.576795in}}%
\pgfpathlineto{\pgfqpoint{2.807682in}{2.576232in}}%
\pgfpathlineto{\pgfqpoint{2.807682in}{2.572481in}}%
\pgfpathlineto{\pgfqpoint{2.810347in}{2.569292in}}%
\pgfpathlineto{\pgfqpoint{2.810817in}{2.568729in}}%
\pgfpathlineto{\pgfqpoint{2.813482in}{2.565540in}}%
\pgfpathlineto{\pgfqpoint{2.813952in}{2.564978in}}%
\pgfpathlineto{\pgfqpoint{2.816617in}{2.561789in}}%
\pgfpathlineto{\pgfqpoint{2.817087in}{2.561226in}}%
\pgfpathlineto{\pgfqpoint{2.819751in}{2.558037in}}%
\pgfpathlineto{\pgfqpoint{2.820221in}{2.557475in}}%
\pgfpathlineto{\pgfqpoint{2.822886in}{2.554286in}}%
\pgfpathlineto{\pgfqpoint{2.823356in}{2.553723in}}%
\pgfpathlineto{\pgfqpoint{2.826021in}{2.550534in}}%
\pgfpathlineto{\pgfqpoint{2.826491in}{2.549972in}}%
\pgfpathlineto{\pgfqpoint{2.829155in}{2.546783in}}%
\pgfpathlineto{\pgfqpoint{2.829626in}{2.546220in}}%
\pgfpathlineto{\pgfqpoint{2.832290in}{2.543031in}}%
\pgfpathlineto{\pgfqpoint{2.832760in}{2.542469in}}%
\pgfpathlineto{\pgfqpoint{2.835425in}{2.539280in}}%
\pgfpathlineto{\pgfqpoint{2.835895in}{2.538717in}}%
\pgfpathlineto{\pgfqpoint{2.838560in}{2.535528in}}%
\pgfpathlineto{\pgfqpoint{2.839030in}{2.534965in}}%
\pgfpathlineto{\pgfqpoint{2.841694in}{2.531777in}}%
\pgfpathlineto{\pgfqpoint{2.842165in}{2.531214in}}%
\pgfpathlineto{\pgfqpoint{2.844829in}{2.528025in}}%
\pgfpathlineto{\pgfqpoint{2.845299in}{2.527462in}}%
\pgfpathlineto{\pgfqpoint{2.847964in}{2.524274in}}%
\pgfpathlineto{\pgfqpoint{2.848434in}{2.523711in}}%
\pgfpathlineto{\pgfqpoint{2.851099in}{2.520522in}}%
\pgfpathlineto{\pgfqpoint{2.851569in}{2.519959in}}%
\pgfpathlineto{\pgfqpoint{2.854233in}{2.516771in}}%
\pgfpathlineto{\pgfqpoint{2.854704in}{2.516208in}}%
\pgfpathlineto{\pgfqpoint{2.857368in}{2.513019in}}%
\pgfpathlineto{\pgfqpoint{2.857838in}{2.512456in}}%
\pgfpathlineto{\pgfqpoint{2.857838in}{2.508705in}}%
\pgfpathlineto{\pgfqpoint{2.860503in}{2.505516in}}%
\pgfpathlineto{\pgfqpoint{2.860973in}{2.504953in}}%
\pgfpathlineto{\pgfqpoint{2.863638in}{2.501764in}}%
\pgfpathlineto{\pgfqpoint{2.864108in}{2.501202in}}%
\pgfpathlineto{\pgfqpoint{2.866772in}{2.498013in}}%
\pgfpathlineto{\pgfqpoint{2.867243in}{2.497450in}}%
\pgfpathlineto{\pgfqpoint{2.869907in}{2.494261in}}%
\pgfpathlineto{\pgfqpoint{2.870377in}{2.493699in}}%
\pgfpathlineto{\pgfqpoint{2.873042in}{2.490510in}}%
\pgfpathlineto{\pgfqpoint{2.873512in}{2.489947in}}%
\pgfpathlineto{\pgfqpoint{2.876177in}{2.486758in}}%
\pgfpathlineto{\pgfqpoint{2.876647in}{2.486196in}}%
\pgfpathlineto{\pgfqpoint{2.879311in}{2.483007in}}%
\pgfpathlineto{\pgfqpoint{2.879782in}{2.482444in}}%
\pgfpathlineto{\pgfqpoint{2.882446in}{2.479255in}}%
\pgfpathlineto{\pgfqpoint{2.882916in}{2.478692in}}%
\pgfpathlineto{\pgfqpoint{2.885581in}{2.475504in}}%
\pgfpathlineto{\pgfqpoint{2.886051in}{2.474941in}}%
\pgfpathlineto{\pgfqpoint{2.888716in}{2.471752in}}%
\pgfpathlineto{\pgfqpoint{2.889186in}{2.471189in}}%
\pgfpathlineto{\pgfqpoint{2.891850in}{2.468001in}}%
\pgfpathlineto{\pgfqpoint{2.892321in}{2.467438in}}%
\pgfpathlineto{\pgfqpoint{2.894985in}{2.464249in}}%
\pgfpathlineto{\pgfqpoint{2.895455in}{2.463686in}}%
\pgfpathlineto{\pgfqpoint{2.898120in}{2.460498in}}%
\pgfpathlineto{\pgfqpoint{2.898590in}{2.459935in}}%
\pgfpathlineto{\pgfqpoint{2.901255in}{2.456746in}}%
\pgfpathlineto{\pgfqpoint{2.901725in}{2.456183in}}%
\pgfpathlineto{\pgfqpoint{2.904389in}{2.452994in}}%
\pgfpathlineto{\pgfqpoint{2.904860in}{2.452432in}}%
\pgfpathlineto{\pgfqpoint{2.904860in}{2.448680in}}%
\pgfpathlineto{\pgfqpoint{2.907524in}{2.445491in}}%
\pgfpathlineto{\pgfqpoint{2.907994in}{2.444929in}}%
\pgfpathlineto{\pgfqpoint{2.910659in}{2.441740in}}%
\pgfpathlineto{\pgfqpoint{2.911129in}{2.441177in}}%
\pgfpathlineto{\pgfqpoint{2.913794in}{2.437988in}}%
\pgfpathlineto{\pgfqpoint{2.914264in}{2.437426in}}%
\pgfpathlineto{\pgfqpoint{2.916928in}{2.434237in}}%
\pgfpathlineto{\pgfqpoint{2.917398in}{2.433674in}}%
\pgfpathlineto{\pgfqpoint{2.920063in}{2.430485in}}%
\pgfpathlineto{\pgfqpoint{2.920533in}{2.429923in}}%
\pgfpathlineto{\pgfqpoint{2.923198in}{2.426734in}}%
\pgfpathlineto{\pgfqpoint{2.923668in}{2.426171in}}%
\pgfpathlineto{\pgfqpoint{2.926332in}{2.422982in}}%
\pgfpathlineto{\pgfqpoint{2.926803in}{2.422419in}}%
\pgfpathlineto{\pgfqpoint{2.929467in}{2.419231in}}%
\pgfpathlineto{\pgfqpoint{2.929937in}{2.418668in}}%
\pgfpathlineto{\pgfqpoint{2.932602in}{2.415479in}}%
\pgfpathlineto{\pgfqpoint{2.933072in}{2.414916in}}%
\pgfpathlineto{\pgfqpoint{2.935737in}{2.411728in}}%
\pgfpathlineto{\pgfqpoint{2.936207in}{2.411165in}}%
\pgfpathlineto{\pgfqpoint{2.938871in}{2.407976in}}%
\pgfpathlineto{\pgfqpoint{2.939342in}{2.407413in}}%
\pgfpathlineto{\pgfqpoint{2.942006in}{2.404224in}}%
\pgfpathlineto{\pgfqpoint{2.942476in}{2.403662in}}%
\pgfpathlineto{\pgfqpoint{2.945141in}{2.400473in}}%
\pgfpathlineto{\pgfqpoint{2.945611in}{2.399910in}}%
\pgfpathlineto{\pgfqpoint{2.948276in}{2.396721in}}%
\pgfpathlineto{\pgfqpoint{2.948746in}{2.396159in}}%
\pgfpathlineto{\pgfqpoint{2.951410in}{2.392970in}}%
\pgfpathlineto{\pgfqpoint{2.951881in}{2.392407in}}%
\pgfpathlineto{\pgfqpoint{2.954545in}{2.389218in}}%
\pgfpathlineto{\pgfqpoint{2.955015in}{2.388656in}}%
\pgfpathlineto{\pgfqpoint{2.955015in}{2.384904in}}%
\pgfpathlineto{\pgfqpoint{2.957680in}{2.381715in}}%
\pgfpathlineto{\pgfqpoint{2.958150in}{2.381153in}}%
\pgfpathlineto{\pgfqpoint{2.960815in}{2.377964in}}%
\pgfpathlineto{\pgfqpoint{2.961285in}{2.377401in}}%
\pgfpathlineto{\pgfqpoint{2.963949in}{2.374212in}}%
\pgfpathlineto{\pgfqpoint{2.964420in}{2.373649in}}%
\pgfpathlineto{\pgfqpoint{2.967084in}{2.370461in}}%
\pgfpathlineto{\pgfqpoint{2.967554in}{2.369898in}}%
\pgfpathlineto{\pgfqpoint{2.970219in}{2.366709in}}%
\pgfpathlineto{\pgfqpoint{2.970689in}{2.366146in}}%
\pgfpathlineto{\pgfqpoint{2.973354in}{2.362958in}}%
\pgfpathlineto{\pgfqpoint{2.973824in}{2.362395in}}%
\pgfpathlineto{\pgfqpoint{2.976488in}{2.359206in}}%
\pgfpathlineto{\pgfqpoint{2.976959in}{2.358643in}}%
\pgfpathlineto{\pgfqpoint{2.979623in}{2.355455in}}%
\pgfpathlineto{\pgfqpoint{2.980093in}{2.354892in}}%
\pgfpathlineto{\pgfqpoint{2.982758in}{2.351703in}}%
\pgfpathlineto{\pgfqpoint{2.983228in}{2.351140in}}%
\pgfpathlineto{\pgfqpoint{2.985893in}{2.347951in}}%
\pgfpathlineto{\pgfqpoint{2.986363in}{2.347389in}}%
\pgfpathlineto{\pgfqpoint{2.989027in}{2.344200in}}%
\pgfpathlineto{\pgfqpoint{2.989498in}{2.343637in}}%
\pgfpathlineto{\pgfqpoint{2.992162in}{2.340448in}}%
\pgfpathlineto{\pgfqpoint{2.992632in}{2.339886in}}%
\pgfpathlineto{\pgfqpoint{2.995297in}{2.336697in}}%
\pgfpathlineto{\pgfqpoint{2.995767in}{2.336134in}}%
\pgfpathlineto{\pgfqpoint{2.998432in}{2.332945in}}%
\pgfpathlineto{\pgfqpoint{2.998902in}{2.332383in}}%
\pgfpathlineto{\pgfqpoint{3.001566in}{2.329194in}}%
\pgfpathlineto{\pgfqpoint{3.002037in}{2.328631in}}%
\pgfpathlineto{\pgfqpoint{3.002037in}{2.324880in}}%
\pgfpathlineto{\pgfqpoint{3.004701in}{2.321691in}}%
\pgfpathlineto{\pgfqpoint{3.005171in}{2.321128in}}%
\pgfpathlineto{\pgfqpoint{3.007836in}{2.317939in}}%
\pgfpathlineto{\pgfqpoint{3.008306in}{2.317376in}}%
\pgfpathlineto{\pgfqpoint{3.010971in}{2.314188in}}%
\pgfpathlineto{\pgfqpoint{3.011441in}{2.313625in}}%
\pgfpathlineto{\pgfqpoint{3.014105in}{2.310436in}}%
\pgfpathlineto{\pgfqpoint{3.014575in}{2.309873in}}%
\pgfpathlineto{\pgfqpoint{3.017240in}{2.306685in}}%
\pgfpathlineto{\pgfqpoint{3.017710in}{2.306122in}}%
\pgfpathlineto{\pgfqpoint{3.020375in}{2.302933in}}%
\pgfpathlineto{\pgfqpoint{3.020845in}{2.302370in}}%
\pgfpathlineto{\pgfqpoint{3.023510in}{2.299182in}}%
\pgfpathlineto{\pgfqpoint{3.023980in}{2.298619in}}%
\pgfpathlineto{\pgfqpoint{3.026644in}{2.295430in}}%
\pgfpathlineto{\pgfqpoint{3.027114in}{2.294867in}}%
\pgfpathlineto{\pgfqpoint{3.029779in}{2.291678in}}%
\pgfpathlineto{\pgfqpoint{3.030249in}{2.291116in}}%
\pgfpathlineto{\pgfqpoint{3.032914in}{2.287927in}}%
\pgfpathlineto{\pgfqpoint{3.033384in}{2.287364in}}%
\pgfpathlineto{\pgfqpoint{3.036048in}{2.284175in}}%
\pgfpathlineto{\pgfqpoint{3.036519in}{2.283613in}}%
\pgfpathlineto{\pgfqpoint{3.039183in}{2.280424in}}%
\pgfpathlineto{\pgfqpoint{3.039653in}{2.279861in}}%
\pgfpathlineto{\pgfqpoint{3.042318in}{2.276672in}}%
\pgfpathlineto{\pgfqpoint{3.042788in}{2.276110in}}%
\pgfpathlineto{\pgfqpoint{3.045453in}{2.272921in}}%
\pgfpathlineto{\pgfqpoint{3.045923in}{2.272358in}}%
\pgfpathlineto{\pgfqpoint{3.048587in}{2.269169in}}%
\pgfpathlineto{\pgfqpoint{3.049058in}{2.268607in}}%
\pgfpathlineto{\pgfqpoint{3.051722in}{2.265418in}}%
\pgfpathlineto{\pgfqpoint{3.052192in}{2.264855in}}%
\pgfpathlineto{\pgfqpoint{3.052192in}{2.261103in}}%
\pgfpathlineto{\pgfqpoint{3.054857in}{2.257915in}}%
\pgfpathlineto{\pgfqpoint{3.055327in}{2.257352in}}%
\pgfpathlineto{\pgfqpoint{3.057992in}{2.254163in}}%
\pgfpathlineto{\pgfqpoint{3.058462in}{2.253600in}}%
\pgfpathlineto{\pgfqpoint{3.061126in}{2.250412in}}%
\pgfpathlineto{\pgfqpoint{3.061597in}{2.249849in}}%
\pgfpathlineto{\pgfqpoint{3.064261in}{2.246660in}}%
\pgfpathlineto{\pgfqpoint{3.064731in}{2.246097in}}%
\pgfpathlineto{\pgfqpoint{3.067396in}{2.242909in}}%
\pgfpathlineto{\pgfqpoint{3.067866in}{2.242346in}}%
\pgfpathlineto{\pgfqpoint{3.070531in}{2.239157in}}%
\pgfpathlineto{\pgfqpoint{3.071001in}{2.238594in}}%
\pgfpathlineto{\pgfqpoint{3.073665in}{2.235405in}}%
\pgfpathlineto{\pgfqpoint{3.074136in}{2.234843in}}%
\pgfpathlineto{\pgfqpoint{3.076800in}{2.231654in}}%
\pgfpathlineto{\pgfqpoint{3.077270in}{2.231091in}}%
\pgfpathlineto{\pgfqpoint{3.079935in}{2.227902in}}%
\pgfpathlineto{\pgfqpoint{3.080405in}{2.227340in}}%
\pgfpathlineto{\pgfqpoint{3.083070in}{2.224151in}}%
\pgfpathlineto{\pgfqpoint{3.083540in}{2.223588in}}%
\pgfpathlineto{\pgfqpoint{3.086204in}{2.220399in}}%
\pgfpathlineto{\pgfqpoint{3.086675in}{2.219837in}}%
\pgfpathlineto{\pgfqpoint{3.089339in}{2.216648in}}%
\pgfpathlineto{\pgfqpoint{3.089809in}{2.216085in}}%
\pgfpathlineto{\pgfqpoint{3.092474in}{2.212896in}}%
\pgfpathlineto{\pgfqpoint{3.092944in}{2.212334in}}%
\pgfpathlineto{\pgfqpoint{3.092944in}{2.208582in}}%
\pgfpathlineto{\pgfqpoint{3.092474in}{2.208019in}}%
\pgfpathlineto{\pgfqpoint{3.089809in}{2.204830in}}%
\pgfpathlineto{\pgfqpoint{3.089809in}{2.201079in}}%
\pgfpathlineto{\pgfqpoint{3.089809in}{2.197327in}}%
\pgfpathlineto{\pgfqpoint{3.089339in}{2.196765in}}%
\pgfpathlineto{\pgfqpoint{3.086675in}{2.193576in}}%
\pgfpathlineto{\pgfqpoint{3.086675in}{2.189824in}}%
\pgfpathlineto{\pgfqpoint{3.086675in}{2.186073in}}%
\pgfpathlineto{\pgfqpoint{3.086204in}{2.185510in}}%
\pgfpathlineto{\pgfqpoint{3.083540in}{2.182321in}}%
\pgfpathlineto{\pgfqpoint{3.083540in}{2.178570in}}%
\pgfpathlineto{\pgfqpoint{3.083540in}{2.174818in}}%
\pgfpathlineto{\pgfqpoint{3.083070in}{2.174255in}}%
\pgfpathlineto{\pgfqpoint{3.080405in}{2.171067in}}%
\pgfpathlineto{\pgfqpoint{3.080405in}{2.167315in}}%
\pgfpathlineto{\pgfqpoint{3.080405in}{2.163564in}}%
\pgfpathlineto{\pgfqpoint{3.079935in}{2.163001in}}%
\pgfpathlineto{\pgfqpoint{3.077270in}{2.159812in}}%
\pgfpathlineto{\pgfqpoint{3.077270in}{2.156060in}}%
\pgfpathlineto{\pgfqpoint{3.077270in}{2.152309in}}%
\pgfpathlineto{\pgfqpoint{3.077270in}{2.148557in}}%
\pgfpathlineto{\pgfqpoint{3.076800in}{2.147995in}}%
\pgfpathlineto{\pgfqpoint{3.074136in}{2.144806in}}%
\pgfpathlineto{\pgfqpoint{3.074136in}{2.141054in}}%
\pgfpathlineto{\pgfqpoint{3.074136in}{2.137303in}}%
\pgfpathlineto{\pgfqpoint{3.073665in}{2.136740in}}%
\pgfpathlineto{\pgfqpoint{3.071001in}{2.133551in}}%
\pgfpathlineto{\pgfqpoint{3.071001in}{2.129800in}}%
\pgfpathlineto{\pgfqpoint{3.071001in}{2.126048in}}%
\pgfpathlineto{\pgfqpoint{3.070531in}{2.125485in}}%
\pgfpathlineto{\pgfqpoint{3.067866in}{2.122297in}}%
\pgfpathlineto{\pgfqpoint{3.067866in}{2.118545in}}%
\pgfpathlineto{\pgfqpoint{3.067866in}{2.114794in}}%
\pgfpathlineto{\pgfqpoint{3.067396in}{2.114231in}}%
\pgfpathlineto{\pgfqpoint{3.064731in}{2.111042in}}%
\pgfpathlineto{\pgfqpoint{3.064731in}{2.107291in}}%
\pgfpathlineto{\pgfqpoint{3.064731in}{2.103539in}}%
\pgfpathlineto{\pgfqpoint{3.064261in}{2.102976in}}%
\pgfpathlineto{\pgfqpoint{3.061597in}{2.099787in}}%
\pgfpathlineto{\pgfqpoint{3.061597in}{2.096036in}}%
\pgfpathlineto{\pgfqpoint{3.061597in}{2.092284in}}%
\pgfpathlineto{\pgfqpoint{3.061126in}{2.091722in}}%
\pgfpathlineto{\pgfqpoint{3.058462in}{2.088533in}}%
\pgfpathlineto{\pgfqpoint{3.058462in}{2.084781in}}%
\pgfpathlineto{\pgfqpoint{3.058462in}{2.081030in}}%
\pgfpathlineto{\pgfqpoint{3.057992in}{2.080467in}}%
\pgfpathlineto{\pgfqpoint{3.055327in}{2.077278in}}%
\pgfpathlineto{\pgfqpoint{3.055327in}{2.073527in}}%
\pgfpathlineto{\pgfqpoint{3.055327in}{2.069775in}}%
\pgfpathlineto{\pgfqpoint{3.054857in}{2.069212in}}%
\pgfpathlineto{\pgfqpoint{3.052192in}{2.066024in}}%
\pgfpathlineto{\pgfqpoint{3.052192in}{2.062272in}}%
\pgfpathlineto{\pgfqpoint{3.052192in}{2.058521in}}%
\pgfpathlineto{\pgfqpoint{3.051722in}{2.057958in}}%
\pgfpathlineto{\pgfqpoint{3.049058in}{2.054769in}}%
\pgfpathlineto{\pgfqpoint{3.049058in}{2.051018in}}%
\pgfpathlineto{\pgfqpoint{3.049058in}{2.047266in}}%
\pgfpathlineto{\pgfqpoint{3.049058in}{2.043514in}}%
\pgfpathlineto{\pgfqpoint{3.048587in}{2.042952in}}%
\pgfpathlineto{\pgfqpoint{3.045923in}{2.039763in}}%
\pgfpathlineto{\pgfqpoint{3.045923in}{2.036011in}}%
\pgfpathlineto{\pgfqpoint{3.045923in}{2.032260in}}%
\pgfpathlineto{\pgfqpoint{3.045453in}{2.031697in}}%
\pgfpathlineto{\pgfqpoint{3.042788in}{2.028508in}}%
\pgfpathlineto{\pgfqpoint{3.042788in}{2.024757in}}%
\pgfpathlineto{\pgfqpoint{3.042788in}{2.021005in}}%
\pgfpathlineto{\pgfqpoint{3.042318in}{2.020443in}}%
\pgfpathlineto{\pgfqpoint{3.039653in}{2.017254in}}%
\pgfpathlineto{\pgfqpoint{3.039653in}{2.013502in}}%
\pgfpathlineto{\pgfqpoint{3.039653in}{2.009751in}}%
\pgfpathlineto{\pgfqpoint{3.039183in}{2.009188in}}%
\pgfpathlineto{\pgfqpoint{3.036519in}{2.005999in}}%
\pgfpathlineto{\pgfqpoint{3.036519in}{2.002248in}}%
\pgfpathlineto{\pgfqpoint{3.036519in}{1.998496in}}%
\pgfpathlineto{\pgfqpoint{3.036048in}{1.997933in}}%
\pgfpathlineto{\pgfqpoint{3.033384in}{1.994745in}}%
\pgfpathlineto{\pgfqpoint{3.033384in}{1.990993in}}%
\pgfpathlineto{\pgfqpoint{3.033384in}{1.987241in}}%
\pgfpathlineto{\pgfqpoint{3.032914in}{1.986679in}}%
\pgfpathlineto{\pgfqpoint{3.030249in}{1.983490in}}%
\pgfpathlineto{\pgfqpoint{3.030249in}{1.979738in}}%
\pgfpathlineto{\pgfqpoint{3.030249in}{1.975987in}}%
\pgfpathlineto{\pgfqpoint{3.029779in}{1.975424in}}%
\pgfpathlineto{\pgfqpoint{3.027114in}{1.972235in}}%
\pgfpathlineto{\pgfqpoint{3.027114in}{1.968484in}}%
\pgfpathlineto{\pgfqpoint{3.027114in}{1.964732in}}%
\pgfpathlineto{\pgfqpoint{3.026644in}{1.964169in}}%
\pgfpathlineto{\pgfqpoint{3.023980in}{1.960981in}}%
\pgfpathlineto{\pgfqpoint{3.023980in}{1.957229in}}%
\pgfpathlineto{\pgfqpoint{3.023980in}{1.953478in}}%
\pgfpathlineto{\pgfqpoint{3.023980in}{1.949726in}}%
\pgfpathlineto{\pgfqpoint{3.023510in}{1.949163in}}%
\pgfpathlineto{\pgfqpoint{3.020845in}{1.945975in}}%
\pgfpathlineto{\pgfqpoint{3.020845in}{1.942223in}}%
\pgfpathlineto{\pgfqpoint{3.020845in}{1.938471in}}%
\pgfpathlineto{\pgfqpoint{3.020375in}{1.937909in}}%
\pgfpathlineto{\pgfqpoint{3.017710in}{1.934720in}}%
\pgfpathlineto{\pgfqpoint{3.017710in}{1.930968in}}%
\pgfpathlineto{\pgfqpoint{3.017710in}{1.927217in}}%
\pgfpathlineto{\pgfqpoint{3.017240in}{1.926654in}}%
\pgfpathlineto{\pgfqpoint{3.014575in}{1.923465in}}%
\pgfpathlineto{\pgfqpoint{3.014575in}{1.919714in}}%
\pgfpathlineto{\pgfqpoint{3.014575in}{1.915962in}}%
\pgfpathlineto{\pgfqpoint{3.014105in}{1.915400in}}%
\pgfpathlineto{\pgfqpoint{3.011441in}{1.912211in}}%
\pgfpathlineto{\pgfqpoint{3.011441in}{1.908459in}}%
\pgfpathlineto{\pgfqpoint{3.011441in}{1.904708in}}%
\pgfpathlineto{\pgfqpoint{3.010971in}{1.904145in}}%
\pgfpathlineto{\pgfqpoint{3.008306in}{1.900956in}}%
\pgfpathlineto{\pgfqpoint{3.008306in}{1.897205in}}%
\pgfpathlineto{\pgfqpoint{3.008306in}{1.893453in}}%
\pgfpathlineto{\pgfqpoint{3.007836in}{1.892890in}}%
\pgfpathlineto{\pgfqpoint{3.005171in}{1.889702in}}%
\pgfpathlineto{\pgfqpoint{3.005171in}{1.885950in}}%
\pgfpathlineto{\pgfqpoint{3.005171in}{1.882198in}}%
\pgfpathlineto{\pgfqpoint{3.004701in}{1.881636in}}%
\pgfpathlineto{\pgfqpoint{3.002037in}{1.878447in}}%
\pgfpathlineto{\pgfqpoint{3.002037in}{1.874695in}}%
\pgfpathlineto{\pgfqpoint{3.002037in}{1.870944in}}%
\pgfpathlineto{\pgfqpoint{3.001566in}{1.870381in}}%
\pgfpathlineto{\pgfqpoint{2.998902in}{1.867192in}}%
\pgfpathlineto{\pgfqpoint{2.998902in}{1.863441in}}%
\pgfpathlineto{\pgfqpoint{2.998902in}{1.859689in}}%
\pgfpathlineto{\pgfqpoint{2.998432in}{1.859127in}}%
\pgfpathlineto{\pgfqpoint{2.995767in}{1.855938in}}%
\pgfpathlineto{\pgfqpoint{2.995767in}{1.852186in}}%
\pgfpathlineto{\pgfqpoint{2.995767in}{1.848435in}}%
\pgfpathlineto{\pgfqpoint{2.995767in}{1.844683in}}%
\pgfpathlineto{\pgfqpoint{2.995297in}{1.844120in}}%
\pgfpathlineto{\pgfqpoint{2.992632in}{1.840932in}}%
\pgfpathlineto{\pgfqpoint{2.992632in}{1.837180in}}%
\pgfpathlineto{\pgfqpoint{2.992632in}{1.833429in}}%
\pgfpathlineto{\pgfqpoint{2.992162in}{1.832866in}}%
\pgfpathlineto{\pgfqpoint{2.989498in}{1.829677in}}%
\pgfpathlineto{\pgfqpoint{2.989498in}{1.825925in}}%
\pgfpathlineto{\pgfqpoint{2.989498in}{1.822174in}}%
\pgfpathlineto{\pgfqpoint{2.989027in}{1.821611in}}%
\pgfpathlineto{\pgfqpoint{2.986363in}{1.818422in}}%
\pgfpathlineto{\pgfqpoint{2.986363in}{1.814671in}}%
\pgfpathlineto{\pgfqpoint{2.986363in}{1.810919in}}%
\pgfpathlineto{\pgfqpoint{2.985893in}{1.810357in}}%
\pgfpathlineto{\pgfqpoint{2.983228in}{1.807168in}}%
\pgfpathlineto{\pgfqpoint{2.983228in}{1.803416in}}%
\pgfpathlineto{\pgfqpoint{2.983228in}{1.799665in}}%
\pgfpathlineto{\pgfqpoint{2.982758in}{1.799102in}}%
\pgfpathlineto{\pgfqpoint{2.980093in}{1.795913in}}%
\pgfpathlineto{\pgfqpoint{2.980093in}{1.792162in}}%
\pgfpathlineto{\pgfqpoint{2.980093in}{1.788410in}}%
\pgfpathlineto{\pgfqpoint{2.979623in}{1.787847in}}%
\pgfpathlineto{\pgfqpoint{2.976959in}{1.784659in}}%
\pgfpathlineto{\pgfqpoint{2.976959in}{1.780907in}}%
\pgfpathlineto{\pgfqpoint{2.976959in}{1.777155in}}%
\pgfpathlineto{\pgfqpoint{2.976488in}{1.776593in}}%
\pgfpathlineto{\pgfqpoint{2.973824in}{1.773404in}}%
\pgfpathlineto{\pgfqpoint{2.973824in}{1.769652in}}%
\pgfpathlineto{\pgfqpoint{2.973824in}{1.765901in}}%
\pgfpathlineto{\pgfqpoint{2.973354in}{1.765338in}}%
\pgfpathlineto{\pgfqpoint{2.970689in}{1.762149in}}%
\pgfpathlineto{\pgfqpoint{2.970689in}{1.758398in}}%
\pgfpathlineto{\pgfqpoint{2.970689in}{1.754646in}}%
\pgfpathlineto{\pgfqpoint{2.970689in}{1.750895in}}%
\pgfpathlineto{\pgfqpoint{2.970219in}{1.750332in}}%
\pgfpathlineto{\pgfqpoint{2.967554in}{1.747143in}}%
\pgfpathlineto{\pgfqpoint{2.967554in}{1.743392in}}%
\pgfpathlineto{\pgfqpoint{2.967554in}{1.739640in}}%
\pgfpathlineto{\pgfqpoint{2.967084in}{1.739077in}}%
\pgfpathlineto{\pgfqpoint{2.964420in}{1.735889in}}%
\pgfpathlineto{\pgfqpoint{2.964420in}{1.732137in}}%
\pgfpathlineto{\pgfqpoint{2.964420in}{1.728386in}}%
\pgfpathlineto{\pgfqpoint{2.963949in}{1.727823in}}%
\pgfpathlineto{\pgfqpoint{2.961285in}{1.724634in}}%
\pgfpathlineto{\pgfqpoint{2.961285in}{1.720882in}}%
\pgfpathlineto{\pgfqpoint{2.961285in}{1.717131in}}%
\pgfpathlineto{\pgfqpoint{2.960815in}{1.716568in}}%
\pgfpathlineto{\pgfqpoint{2.958150in}{1.713379in}}%
\pgfpathlineto{\pgfqpoint{2.958150in}{1.709628in}}%
\pgfpathlineto{\pgfqpoint{2.958150in}{1.705876in}}%
\pgfpathlineto{\pgfqpoint{2.957680in}{1.705314in}}%
\pgfpathlineto{\pgfqpoint{2.955015in}{1.702125in}}%
\pgfpathlineto{\pgfqpoint{2.955015in}{1.698373in}}%
\pgfpathlineto{\pgfqpoint{2.955015in}{1.694622in}}%
\pgfpathlineto{\pgfqpoint{2.954545in}{1.694059in}}%
\pgfpathlineto{\pgfqpoint{2.951881in}{1.690870in}}%
\pgfpathlineto{\pgfqpoint{2.951881in}{1.687119in}}%
\pgfpathlineto{\pgfqpoint{2.951881in}{1.683367in}}%
\pgfpathlineto{\pgfqpoint{2.951410in}{1.682804in}}%
\pgfpathlineto{\pgfqpoint{2.948746in}{1.679616in}}%
\pgfpathlineto{\pgfqpoint{2.948746in}{1.675864in}}%
\pgfpathlineto{\pgfqpoint{2.948746in}{1.672113in}}%
\pgfpathlineto{\pgfqpoint{2.948276in}{1.671550in}}%
\pgfpathlineto{\pgfqpoint{2.945611in}{1.668361in}}%
\pgfpathlineto{\pgfqpoint{2.945611in}{1.664609in}}%
\pgfpathlineto{\pgfqpoint{2.945611in}{1.660858in}}%
\pgfpathlineto{\pgfqpoint{2.945141in}{1.660295in}}%
\pgfpathlineto{\pgfqpoint{2.942476in}{1.657106in}}%
\pgfpathlineto{\pgfqpoint{2.942476in}{1.653355in}}%
\pgfpathlineto{\pgfqpoint{2.942476in}{1.649603in}}%
\pgfpathlineto{\pgfqpoint{2.942476in}{1.645852in}}%
\pgfpathlineto{\pgfqpoint{2.942006in}{1.645289in}}%
\pgfpathlineto{\pgfqpoint{2.939342in}{1.642100in}}%
\pgfpathlineto{\pgfqpoint{2.939342in}{1.638349in}}%
\pgfpathlineto{\pgfqpoint{2.939342in}{1.634597in}}%
\pgfpathlineto{\pgfqpoint{2.938871in}{1.634034in}}%
\pgfpathlineto{\pgfqpoint{2.936207in}{1.630846in}}%
\pgfpathlineto{\pgfqpoint{2.936207in}{1.627094in}}%
\pgfpathlineto{\pgfqpoint{2.936207in}{1.623343in}}%
\pgfpathlineto{\pgfqpoint{2.935737in}{1.622780in}}%
\pgfpathlineto{\pgfqpoint{2.933072in}{1.619591in}}%
\pgfpathlineto{\pgfqpoint{2.933072in}{1.615840in}}%
\pgfpathlineto{\pgfqpoint{2.933072in}{1.612088in}}%
\pgfpathlineto{\pgfqpoint{2.932602in}{1.611525in}}%
\pgfpathlineto{\pgfqpoint{2.929937in}{1.608336in}}%
\pgfpathlineto{\pgfqpoint{2.929937in}{1.604585in}}%
\pgfpathlineto{\pgfqpoint{2.929937in}{1.600833in}}%
\pgfpathlineto{\pgfqpoint{2.929467in}{1.600271in}}%
\pgfpathlineto{\pgfqpoint{2.926803in}{1.597082in}}%
\pgfpathlineto{\pgfqpoint{2.926803in}{1.593330in}}%
\pgfpathlineto{\pgfqpoint{2.926803in}{1.589579in}}%
\pgfpathlineto{\pgfqpoint{2.926332in}{1.589016in}}%
\pgfpathlineto{\pgfqpoint{2.923668in}{1.585827in}}%
\pgfpathlineto{\pgfqpoint{2.923668in}{1.582076in}}%
\pgfpathlineto{\pgfqpoint{2.923668in}{1.578324in}}%
\pgfpathlineto{\pgfqpoint{2.923198in}{1.577761in}}%
\pgfpathlineto{\pgfqpoint{2.920533in}{1.574573in}}%
\pgfpathlineto{\pgfqpoint{2.920533in}{1.570821in}}%
\pgfpathlineto{\pgfqpoint{2.920533in}{1.567070in}}%
\pgfpathlineto{\pgfqpoint{2.920063in}{1.566507in}}%
\pgfpathlineto{\pgfqpoint{2.917398in}{1.563318in}}%
\pgfpathlineto{\pgfqpoint{2.917398in}{1.559566in}}%
\pgfpathlineto{\pgfqpoint{2.917398in}{1.555815in}}%
\pgfpathlineto{\pgfqpoint{2.917398in}{1.552063in}}%
\pgfpathlineto{\pgfqpoint{2.916928in}{1.551501in}}%
\pgfpathlineto{\pgfqpoint{2.914264in}{1.548312in}}%
\pgfpathlineto{\pgfqpoint{2.914264in}{1.544560in}}%
\pgfpathlineto{\pgfqpoint{2.914264in}{1.540809in}}%
\pgfpathlineto{\pgfqpoint{2.913794in}{1.540246in}}%
\pgfpathlineto{\pgfqpoint{2.911129in}{1.537057in}}%
\pgfpathlineto{\pgfqpoint{2.911129in}{1.533306in}}%
\pgfpathlineto{\pgfqpoint{2.911129in}{1.529554in}}%
\pgfpathlineto{\pgfqpoint{2.910659in}{1.528991in}}%
\pgfpathlineto{\pgfqpoint{2.907994in}{1.525803in}}%
\pgfpathlineto{\pgfqpoint{2.907994in}{1.522051in}}%
\pgfpathlineto{\pgfqpoint{2.907994in}{1.518300in}}%
\pgfpathlineto{\pgfqpoint{2.907524in}{1.517737in}}%
\pgfpathlineto{\pgfqpoint{2.904860in}{1.514548in}}%
\pgfpathlineto{\pgfqpoint{2.904860in}{1.510797in}}%
\pgfpathlineto{\pgfqpoint{2.904860in}{1.507045in}}%
\pgfpathlineto{\pgfqpoint{2.904389in}{1.506482in}}%
\pgfpathlineto{\pgfqpoint{2.901725in}{1.503293in}}%
\pgfpathlineto{\pgfqpoint{2.901725in}{1.499542in}}%
\pgfpathlineto{\pgfqpoint{2.901725in}{1.495790in}}%
\pgfpathlineto{\pgfqpoint{2.901255in}{1.495228in}}%
\pgfpathlineto{\pgfqpoint{2.898590in}{1.492039in}}%
\pgfpathlineto{\pgfqpoint{2.898590in}{1.488287in}}%
\pgfpathlineto{\pgfqpoint{2.898590in}{1.484536in}}%
\pgfpathlineto{\pgfqpoint{2.898120in}{1.483973in}}%
\pgfpathlineto{\pgfqpoint{2.895455in}{1.480784in}}%
\pgfpathlineto{\pgfqpoint{2.895455in}{1.477033in}}%
\pgfpathlineto{\pgfqpoint{2.895455in}{1.473281in}}%
\pgfpathlineto{\pgfqpoint{2.894985in}{1.472718in}}%
\pgfpathlineto{\pgfqpoint{2.892321in}{1.469530in}}%
\pgfpathlineto{\pgfqpoint{2.892321in}{1.465778in}}%
\pgfpathlineto{\pgfqpoint{2.892321in}{1.462027in}}%
\pgfpathlineto{\pgfqpoint{2.891850in}{1.461464in}}%
\pgfpathlineto{\pgfqpoint{2.889186in}{1.458275in}}%
\pgfpathlineto{\pgfqpoint{2.889186in}{1.454524in}}%
\pgfpathlineto{\pgfqpoint{2.889186in}{1.450772in}}%
\pgfpathlineto{\pgfqpoint{2.889186in}{1.447020in}}%
\pgfpathlineto{\pgfqpoint{2.888716in}{1.446458in}}%
\pgfpathlineto{\pgfqpoint{2.886051in}{1.443269in}}%
\pgfpathlineto{\pgfqpoint{2.886051in}{1.439517in}}%
\pgfpathlineto{\pgfqpoint{2.886051in}{1.435766in}}%
\pgfpathlineto{\pgfqpoint{2.885581in}{1.435203in}}%
\pgfpathlineto{\pgfqpoint{2.882916in}{1.432014in}}%
\pgfpathlineto{\pgfqpoint{2.882916in}{1.428263in}}%
\pgfpathlineto{\pgfqpoint{2.882916in}{1.424511in}}%
\pgfpathlineto{\pgfqpoint{2.882446in}{1.423949in}}%
\pgfpathlineto{\pgfqpoint{2.879782in}{1.420760in}}%
\pgfpathlineto{\pgfqpoint{2.879782in}{1.417008in}}%
\pgfpathlineto{\pgfqpoint{2.879782in}{1.413257in}}%
\pgfpathlineto{\pgfqpoint{2.879311in}{1.412694in}}%
\pgfpathlineto{\pgfqpoint{2.876647in}{1.409505in}}%
\pgfpathlineto{\pgfqpoint{2.876647in}{1.405754in}}%
\pgfpathlineto{\pgfqpoint{2.876647in}{1.402002in}}%
\pgfpathlineto{\pgfqpoint{2.876177in}{1.401439in}}%
\pgfpathlineto{\pgfqpoint{2.873512in}{1.398251in}}%
\pgfpathlineto{\pgfqpoint{2.873512in}{1.394499in}}%
\pgfpathlineto{\pgfqpoint{2.873512in}{1.390747in}}%
\pgfpathlineto{\pgfqpoint{2.873042in}{1.390185in}}%
\pgfpathlineto{\pgfqpoint{2.870377in}{1.386996in}}%
\pgfpathlineto{\pgfqpoint{2.870377in}{1.383244in}}%
\pgfpathlineto{\pgfqpoint{2.870377in}{1.379493in}}%
\pgfpathlineto{\pgfqpoint{2.869907in}{1.378930in}}%
\pgfpathlineto{\pgfqpoint{2.867243in}{1.375741in}}%
\pgfpathlineto{\pgfqpoint{2.867243in}{1.371990in}}%
\pgfpathlineto{\pgfqpoint{2.867243in}{1.368238in}}%
\pgfpathlineto{\pgfqpoint{2.866772in}{1.367676in}}%
\pgfpathlineto{\pgfqpoint{2.864108in}{1.364487in}}%
\pgfpathlineto{\pgfqpoint{2.864108in}{1.360735in}}%
\pgfpathlineto{\pgfqpoint{2.864108in}{1.356984in}}%
\pgfpathlineto{\pgfqpoint{2.864108in}{1.353232in}}%
\pgfpathlineto{\pgfqpoint{2.863638in}{1.352669in}}%
\pgfpathlineto{\pgfqpoint{2.860973in}{1.349481in}}%
\pgfpathlineto{\pgfqpoint{2.860973in}{1.345729in}}%
\pgfpathlineto{\pgfqpoint{2.860973in}{1.341977in}}%
\pgfpathlineto{\pgfqpoint{2.860503in}{1.341415in}}%
\pgfpathlineto{\pgfqpoint{2.857838in}{1.338226in}}%
\pgfpathlineto{\pgfqpoint{2.857838in}{1.334474in}}%
\pgfpathlineto{\pgfqpoint{2.857838in}{1.330723in}}%
\pgfpathlineto{\pgfqpoint{2.857368in}{1.330160in}}%
\pgfpathlineto{\pgfqpoint{2.854704in}{1.326971in}}%
\pgfpathlineto{\pgfqpoint{2.854704in}{1.323220in}}%
\pgfpathlineto{\pgfqpoint{2.854704in}{1.319468in}}%
\pgfpathlineto{\pgfqpoint{2.854233in}{1.318906in}}%
\pgfpathlineto{\pgfqpoint{2.851569in}{1.315717in}}%
\pgfpathlineto{\pgfqpoint{2.851569in}{1.311965in}}%
\pgfpathlineto{\pgfqpoint{2.851569in}{1.308214in}}%
\pgfpathlineto{\pgfqpoint{2.851099in}{1.307651in}}%
\pgfpathlineto{\pgfqpoint{2.848434in}{1.304462in}}%
\pgfpathlineto{\pgfqpoint{2.848434in}{1.300711in}}%
\pgfpathlineto{\pgfqpoint{2.848434in}{1.296959in}}%
\pgfpathlineto{\pgfqpoint{2.847964in}{1.296396in}}%
\pgfpathlineto{\pgfqpoint{2.845299in}{1.293208in}}%
\pgfpathlineto{\pgfqpoint{2.845299in}{1.289456in}}%
\pgfpathlineto{\pgfqpoint{2.845299in}{1.285704in}}%
\pgfpathlineto{\pgfqpoint{2.844829in}{1.285142in}}%
\pgfpathlineto{\pgfqpoint{2.842165in}{1.281953in}}%
\pgfpathlineto{\pgfqpoint{2.842165in}{1.278201in}}%
\pgfpathlineto{\pgfqpoint{2.842165in}{1.274450in}}%
\pgfpathlineto{\pgfqpoint{2.841694in}{1.273887in}}%
\pgfpathlineto{\pgfqpoint{2.839030in}{1.270698in}}%
\pgfpathlineto{\pgfqpoint{2.839030in}{1.266947in}}%
\pgfpathlineto{\pgfqpoint{2.839030in}{1.263195in}}%
\pgfpathlineto{\pgfqpoint{2.838560in}{1.262633in}}%
\pgfpathlineto{\pgfqpoint{2.835895in}{1.259444in}}%
\pgfpathlineto{\pgfqpoint{2.835895in}{1.255692in}}%
\pgfpathlineto{\pgfqpoint{2.835895in}{1.251941in}}%
\pgfpathlineto{\pgfqpoint{2.835895in}{1.248189in}}%
\pgfpathlineto{\pgfqpoint{2.835425in}{1.247626in}}%
\pgfpathlineto{\pgfqpoint{2.832760in}{1.244438in}}%
\pgfpathlineto{\pgfqpoint{2.832760in}{1.240686in}}%
\pgfpathlineto{\pgfqpoint{2.832760in}{1.236935in}}%
\pgfpathlineto{\pgfqpoint{2.832290in}{1.236372in}}%
\pgfpathlineto{\pgfqpoint{2.829626in}{1.233183in}}%
\pgfpathlineto{\pgfqpoint{2.829626in}{1.229431in}}%
\pgfpathlineto{\pgfqpoint{2.829626in}{1.225680in}}%
\pgfpathlineto{\pgfqpoint{2.829155in}{1.225117in}}%
\pgfpathlineto{\pgfqpoint{2.826491in}{1.221928in}}%
\pgfpathlineto{\pgfqpoint{2.826491in}{1.218177in}}%
\pgfpathlineto{\pgfqpoint{2.826491in}{1.214425in}}%
\pgfpathlineto{\pgfqpoint{2.826021in}{1.213863in}}%
\pgfpathlineto{\pgfqpoint{2.823356in}{1.210674in}}%
\pgfpathlineto{\pgfqpoint{2.823356in}{1.206922in}}%
\pgfpathlineto{\pgfqpoint{2.823356in}{1.203171in}}%
\pgfpathlineto{\pgfqpoint{2.822886in}{1.202608in}}%
\pgfpathlineto{\pgfqpoint{2.820221in}{1.199419in}}%
\pgfpathlineto{\pgfqpoint{2.820221in}{1.195668in}}%
\pgfpathlineto{\pgfqpoint{2.820221in}{1.191916in}}%
\pgfpathlineto{\pgfqpoint{2.819751in}{1.191353in}}%
\pgfpathlineto{\pgfqpoint{2.817087in}{1.188165in}}%
\pgfpathlineto{\pgfqpoint{2.817087in}{1.184413in}}%
\pgfpathlineto{\pgfqpoint{2.817087in}{1.180662in}}%
\pgfpathlineto{\pgfqpoint{2.816617in}{1.180099in}}%
\pgfpathlineto{\pgfqpoint{2.813952in}{1.176910in}}%
\pgfpathlineto{\pgfqpoint{2.813952in}{1.173158in}}%
\pgfpathlineto{\pgfqpoint{2.813482in}{1.172596in}}%
\pgfpathlineto{\pgfqpoint{2.810347in}{1.172596in}}%
\pgfpathlineto{\pgfqpoint{2.807212in}{1.172596in}}%
\pgfpathlineto{\pgfqpoint{2.804548in}{1.169407in}}%
\pgfpathlineto{\pgfqpoint{2.804078in}{1.168844in}}%
\pgfpathlineto{\pgfqpoint{2.800943in}{1.168844in}}%
\pgfpathlineto{\pgfqpoint{2.797808in}{1.168844in}}%
\pgfpathlineto{\pgfqpoint{2.795144in}{1.165655in}}%
\pgfpathlineto{\pgfqpoint{2.794673in}{1.165093in}}%
\pgfpathlineto{\pgfqpoint{2.791539in}{1.165093in}}%
\pgfpathlineto{\pgfqpoint{2.788874in}{1.161904in}}%
\pgfpathlineto{\pgfqpoint{2.788404in}{1.161341in}}%
\pgfpathlineto{\pgfqpoint{2.785269in}{1.161341in}}%
\pgfpathlineto{\pgfqpoint{2.782134in}{1.161341in}}%
\pgfpathlineto{\pgfqpoint{2.779470in}{1.158152in}}%
\pgfpathlineto{\pgfqpoint{2.779000in}{1.157590in}}%
\pgfpathlineto{\pgfqpoint{2.775865in}{1.157590in}}%
\pgfpathlineto{\pgfqpoint{2.772730in}{1.157590in}}%
\pgfpathlineto{\pgfqpoint{2.770066in}{1.154401in}}%
\pgfpathlineto{\pgfqpoint{2.769595in}{1.153838in}}%
\pgfpathlineto{\pgfqpoint{2.766461in}{1.153838in}}%
\pgfpathlineto{\pgfqpoint{2.763326in}{1.153838in}}%
\pgfpathlineto{\pgfqpoint{2.760661in}{1.150649in}}%
\pgfpathlineto{\pgfqpoint{2.760191in}{1.150087in}}%
\pgfpathlineto{\pgfqpoint{2.757056in}{1.150087in}}%
\pgfpathlineto{\pgfqpoint{2.754392in}{1.146898in}}%
\pgfpathlineto{\pgfqpoint{2.753922in}{1.146335in}}%
\pgfpathlineto{\pgfqpoint{2.750787in}{1.146335in}}%
\pgfpathlineto{\pgfqpoint{2.747652in}{1.146335in}}%
\pgfpathlineto{\pgfqpoint{2.744988in}{1.143146in}}%
\pgfpathlineto{\pgfqpoint{2.744517in}{1.142583in}}%
\pgfpathlineto{\pgfqpoint{2.741383in}{1.142583in}}%
\pgfpathlineto{\pgfqpoint{2.738248in}{1.142583in}}%
\pgfpathlineto{\pgfqpoint{2.735583in}{1.139395in}}%
\pgfpathlineto{\pgfqpoint{2.735113in}{1.138832in}}%
\pgfpathlineto{\pgfqpoint{2.731978in}{1.138832in}}%
\pgfpathlineto{\pgfqpoint{2.728844in}{1.138832in}}%
\pgfpathlineto{\pgfqpoint{2.726179in}{1.135643in}}%
\pgfpathlineto{\pgfqpoint{2.725709in}{1.135080in}}%
\pgfpathlineto{\pgfqpoint{2.722574in}{1.135080in}}%
\pgfpathlineto{\pgfqpoint{2.719439in}{1.135080in}}%
\pgfpathlineto{\pgfqpoint{2.716775in}{1.131892in}}%
\pgfpathlineto{\pgfqpoint{2.716305in}{1.131329in}}%
\pgfpathlineto{\pgfqpoint{2.713170in}{1.131329in}}%
\pgfpathlineto{\pgfqpoint{2.710505in}{1.128140in}}%
\pgfpathlineto{\pgfqpoint{2.710035in}{1.127577in}}%
\pgfpathlineto{\pgfqpoint{2.706901in}{1.127577in}}%
\pgfpathlineto{\pgfqpoint{2.703766in}{1.127577in}}%
\pgfpathlineto{\pgfqpoint{2.701101in}{1.124388in}}%
\pgfpathlineto{\pgfqpoint{2.700631in}{1.123826in}}%
\pgfpathlineto{\pgfqpoint{2.697496in}{1.123826in}}%
\pgfpathlineto{\pgfqpoint{2.694362in}{1.123826in}}%
\pgfpathlineto{\pgfqpoint{2.691697in}{1.120637in}}%
\pgfpathlineto{\pgfqpoint{2.691227in}{1.120074in}}%
\pgfpathlineto{\pgfqpoint{2.688092in}{1.120074in}}%
\pgfpathlineto{\pgfqpoint{2.684957in}{1.120074in}}%
\pgfpathlineto{\pgfqpoint{2.682293in}{1.116885in}}%
\pgfpathlineto{\pgfqpoint{2.681823in}{1.116323in}}%
\pgfpathlineto{\pgfqpoint{2.678688in}{1.116323in}}%
\pgfpathlineto{\pgfqpoint{2.676023in}{1.113134in}}%
\pgfpathlineto{\pgfqpoint{2.675553in}{1.112571in}}%
\pgfpathlineto{\pgfqpoint{2.672418in}{1.112571in}}%
\pgfpathlineto{\pgfqpoint{2.669284in}{1.112571in}}%
\pgfpathlineto{\pgfqpoint{2.666619in}{1.109382in}}%
\pgfpathlineto{\pgfqpoint{2.666149in}{1.108820in}}%
\pgfpathlineto{\pgfqpoint{2.663014in}{1.108820in}}%
\pgfpathlineto{\pgfqpoint{2.659879in}{1.108820in}}%
\pgfpathlineto{\pgfqpoint{2.657215in}{1.105631in}}%
\pgfpathlineto{\pgfqpoint{2.656745in}{1.105068in}}%
\pgfpathlineto{\pgfqpoint{2.653610in}{1.105068in}}%
\pgfpathlineto{\pgfqpoint{2.650475in}{1.105068in}}%
\pgfpathlineto{\pgfqpoint{2.647811in}{1.101879in}}%
\pgfpathlineto{\pgfqpoint{2.647340in}{1.101317in}}%
\pgfpathlineto{\pgfqpoint{2.644206in}{1.101317in}}%
\pgfpathlineto{\pgfqpoint{2.641541in}{1.098128in}}%
\pgfpathlineto{\pgfqpoint{2.641071in}{1.097565in}}%
\pgfpathlineto{\pgfqpoint{2.637936in}{1.097565in}}%
\pgfpathlineto{\pgfqpoint{2.634801in}{1.097565in}}%
\pgfpathlineto{\pgfqpoint{2.632137in}{1.094376in}}%
\pgfpathlineto{\pgfqpoint{2.631667in}{1.093813in}}%
\pgfpathlineto{\pgfqpoint{2.628532in}{1.093813in}}%
\pgfpathlineto{\pgfqpoint{2.625397in}{1.093813in}}%
\pgfpathlineto{\pgfqpoint{2.622733in}{1.090625in}}%
\pgfpathlineto{\pgfqpoint{2.622262in}{1.090062in}}%
\pgfpathlineto{\pgfqpoint{2.619128in}{1.090062in}}%
\pgfpathlineto{\pgfqpoint{2.615993in}{1.090062in}}%
\pgfpathlineto{\pgfqpoint{2.613328in}{1.086873in}}%
\pgfpathlineto{\pgfqpoint{2.612858in}{1.086310in}}%
\pgfpathlineto{\pgfqpoint{2.609724in}{1.086310in}}%
\pgfpathlineto{\pgfqpoint{2.606589in}{1.086310in}}%
\pgfpathlineto{\pgfqpoint{2.603924in}{1.083122in}}%
\pgfpathlineto{\pgfqpoint{2.603454in}{1.082559in}}%
\pgfpathlineto{\pgfqpoint{2.600319in}{1.082559in}}%
\pgfpathlineto{\pgfqpoint{2.597655in}{1.079370in}}%
\pgfpathlineto{\pgfqpoint{2.597185in}{1.078807in}}%
\pgfpathlineto{\pgfqpoint{2.594050in}{1.078807in}}%
\pgfpathlineto{\pgfqpoint{2.590915in}{1.078807in}}%
\pgfpathlineto{\pgfqpoint{2.588251in}{1.075619in}}%
\pgfpathlineto{\pgfqpoint{2.587780in}{1.075056in}}%
\pgfpathlineto{\pgfqpoint{2.584646in}{1.075056in}}%
\pgfpathlineto{\pgfqpoint{2.581511in}{1.075056in}}%
\pgfpathlineto{\pgfqpoint{2.578846in}{1.071867in}}%
\pgfpathlineto{\pgfqpoint{2.578376in}{1.071304in}}%
\pgfpathlineto{\pgfqpoint{2.575241in}{1.071304in}}%
\pgfpathlineto{\pgfqpoint{2.572107in}{1.071304in}}%
\pgfpathlineto{\pgfqpoint{2.569442in}{1.068115in}}%
\pgfpathlineto{\pgfqpoint{2.568972in}{1.067553in}}%
\pgfpathlineto{\pgfqpoint{2.565837in}{1.067553in}}%
\pgfpathlineto{\pgfqpoint{2.563173in}{1.064364in}}%
\pgfpathlineto{\pgfqpoint{2.562702in}{1.063801in}}%
\pgfpathlineto{\pgfqpoint{2.559568in}{1.063801in}}%
\pgfpathlineto{\pgfqpoint{2.556433in}{1.063801in}}%
\pgfpathlineto{\pgfqpoint{2.553768in}{1.060612in}}%
\pgfpathlineto{\pgfqpoint{2.553298in}{1.060050in}}%
\pgfpathlineto{\pgfqpoint{2.550163in}{1.060050in}}%
\pgfpathlineto{\pgfqpoint{2.547029in}{1.060050in}}%
\pgfpathlineto{\pgfqpoint{2.544364in}{1.056861in}}%
\pgfpathlineto{\pgfqpoint{2.543894in}{1.056298in}}%
\pgfpathlineto{\pgfqpoint{2.540759in}{1.056298in}}%
\pgfpathlineto{\pgfqpoint{2.537624in}{1.056298in}}%
\pgfpathlineto{\pgfqpoint{2.534960in}{1.053109in}}%
\pgfpathlineto{\pgfqpoint{2.534490in}{1.052547in}}%
\pgfpathlineto{\pgfqpoint{2.531355in}{1.052547in}}%
\pgfpathlineto{\pgfqpoint{2.528690in}{1.049358in}}%
\pgfpathlineto{\pgfqpoint{2.528220in}{1.048795in}}%
\pgfpathlineto{\pgfqpoint{2.525085in}{1.048795in}}%
\pgfpathlineto{\pgfqpoint{2.521951in}{1.048795in}}%
\pgfpathlineto{\pgfqpoint{2.519286in}{1.045606in}}%
\pgfpathlineto{\pgfqpoint{2.518816in}{1.045044in}}%
\pgfpathlineto{\pgfqpoint{2.515681in}{1.045044in}}%
\pgfpathlineto{\pgfqpoint{2.512547in}{1.045044in}}%
\pgfpathlineto{\pgfqpoint{2.509882in}{1.041855in}}%
\pgfpathlineto{\pgfqpoint{2.509412in}{1.041292in}}%
\pgfpathlineto{\pgfqpoint{2.506277in}{1.041292in}}%
\pgfpathlineto{\pgfqpoint{2.503142in}{1.041292in}}%
\pgfpathlineto{\pgfqpoint{2.500478in}{1.038103in}}%
\pgfpathlineto{\pgfqpoint{2.500008in}{1.037540in}}%
\pgfpathlineto{\pgfqpoint{2.496873in}{1.037540in}}%
\pgfpathlineto{\pgfqpoint{2.493738in}{1.037540in}}%
\pgfpathlineto{\pgfqpoint{2.491074in}{1.034352in}}%
\pgfpathlineto{\pgfqpoint{2.490603in}{1.033789in}}%
\pgfpathlineto{\pgfqpoint{2.487469in}{1.033789in}}%
\pgfpathlineto{\pgfqpoint{2.484804in}{1.030600in}}%
\pgfpathlineto{\pgfqpoint{2.484334in}{1.030037in}}%
\pgfpathlineto{\pgfqpoint{2.481199in}{1.030037in}}%
\pgfpathlineto{\pgfqpoint{2.478064in}{1.030037in}}%
\pgfpathlineto{\pgfqpoint{2.475400in}{1.026849in}}%
\pgfpathlineto{\pgfqpoint{2.474930in}{1.026286in}}%
\pgfpathlineto{\pgfqpoint{2.471795in}{1.026286in}}%
\pgfpathlineto{\pgfqpoint{2.468660in}{1.026286in}}%
\pgfpathlineto{\pgfqpoint{2.465996in}{1.023097in}}%
\pgfpathlineto{\pgfqpoint{2.465525in}{1.022534in}}%
\pgfpathlineto{\pgfqpoint{2.462391in}{1.022534in}}%
\pgfpathlineto{\pgfqpoint{2.459256in}{1.022534in}}%
\pgfpathlineto{\pgfqpoint{2.456591in}{1.019346in}}%
\pgfpathlineto{\pgfqpoint{2.456121in}{1.018783in}}%
\pgfpathlineto{\pgfqpoint{2.452986in}{1.018783in}}%
\pgfpathlineto{\pgfqpoint{2.450322in}{1.015594in}}%
\pgfpathlineto{\pgfqpoint{2.449852in}{1.015031in}}%
\pgfpathlineto{\pgfqpoint{2.446717in}{1.015031in}}%
\pgfpathlineto{\pgfqpoint{2.443582in}{1.015031in}}%
\pgfpathlineto{\pgfqpoint{2.440918in}{1.011842in}}%
\pgfpathlineto{\pgfqpoint{2.440447in}{1.011280in}}%
\pgfpathlineto{\pgfqpoint{2.437313in}{1.011280in}}%
\pgfpathlineto{\pgfqpoint{2.434178in}{1.011280in}}%
\pgfpathlineto{\pgfqpoint{2.431513in}{1.008091in}}%
\pgfpathlineto{\pgfqpoint{2.431043in}{1.007528in}}%
\pgfpathlineto{\pgfqpoint{2.427908in}{1.007528in}}%
\pgfpathlineto{\pgfqpoint{2.424774in}{1.007528in}}%
\pgfpathlineto{\pgfqpoint{2.422109in}{1.004339in}}%
\pgfpathlineto{\pgfqpoint{2.421639in}{1.003777in}}%
\pgfpathlineto{\pgfqpoint{2.418504in}{1.003777in}}%
\pgfpathlineto{\pgfqpoint{2.415840in}{1.000588in}}%
\pgfpathlineto{\pgfqpoint{2.415369in}{1.000025in}}%
\pgfpathlineto{\pgfqpoint{2.412235in}{1.000025in}}%
\pgfpathlineto{\pgfqpoint{2.409100in}{1.000025in}}%
\pgfpathlineto{\pgfqpoint{2.406435in}{0.996836in}}%
\pgfpathlineto{\pgfqpoint{2.405965in}{0.996274in}}%
\pgfpathlineto{\pgfqpoint{2.402831in}{0.996274in}}%
\pgfpathlineto{\pgfqpoint{2.399696in}{0.996274in}}%
\pgfpathlineto{\pgfqpoint{2.397031in}{0.993085in}}%
\pgfpathlineto{\pgfqpoint{2.396561in}{0.992522in}}%
\pgfpathlineto{\pgfqpoint{2.393426in}{0.992522in}}%
\pgfpathlineto{\pgfqpoint{2.390292in}{0.992522in}}%
\pgfpathlineto{\pgfqpoint{2.387627in}{0.989333in}}%
\pgfpathlineto{\pgfqpoint{2.387157in}{0.988771in}}%
\pgfpathlineto{\pgfqpoint{2.384022in}{0.988771in}}%
\pgfpathlineto{\pgfqpoint{2.380887in}{0.988771in}}%
\pgfpathlineto{\pgfqpoint{2.378223in}{0.985582in}}%
\pgfpathlineto{\pgfqpoint{2.377753in}{0.985019in}}%
\pgfpathlineto{\pgfqpoint{2.374618in}{0.985019in}}%
\pgfpathlineto{\pgfqpoint{2.371953in}{0.981830in}}%
\pgfpathlineto{\pgfqpoint{2.371483in}{0.981267in}}%
\pgfpathlineto{\pgfqpoint{2.368348in}{0.981267in}}%
\pgfpathlineto{\pgfqpoint{2.365214in}{0.981267in}}%
\pgfpathlineto{\pgfqpoint{2.362549in}{0.978079in}}%
\pgfpathlineto{\pgfqpoint{2.362079in}{0.977516in}}%
\pgfpathlineto{\pgfqpoint{2.358944in}{0.977516in}}%
\pgfpathlineto{\pgfqpoint{2.355809in}{0.977516in}}%
\pgfpathlineto{\pgfqpoint{2.353145in}{0.974327in}}%
\pgfpathlineto{\pgfqpoint{2.352675in}{0.973764in}}%
\pgfpathlineto{\pgfqpoint{2.349540in}{0.973764in}}%
\pgfpathlineto{\pgfqpoint{2.346405in}{0.973764in}}%
\pgfpathlineto{\pgfqpoint{2.343741in}{0.970576in}}%
\pgfpathlineto{\pgfqpoint{2.343270in}{0.970013in}}%
\pgfpathlineto{\pgfqpoint{2.340136in}{0.970013in}}%
\pgfpathlineto{\pgfqpoint{2.337471in}{0.966824in}}%
\pgfpathlineto{\pgfqpoint{2.337001in}{0.966261in}}%
\pgfpathlineto{\pgfqpoint{2.333866in}{0.966261in}}%
\pgfpathlineto{\pgfqpoint{2.330731in}{0.966261in}}%
\pgfpathlineto{\pgfqpoint{2.328067in}{0.963073in}}%
\pgfpathlineto{\pgfqpoint{2.327597in}{0.962510in}}%
\pgfpathlineto{\pgfqpoint{2.324462in}{0.962510in}}%
\pgfpathlineto{\pgfqpoint{2.321327in}{0.962510in}}%
\pgfpathlineto{\pgfqpoint{2.318663in}{0.959321in}}%
\pgfpathlineto{\pgfqpoint{2.318192in}{0.958758in}}%
\pgfpathlineto{\pgfqpoint{2.315058in}{0.958758in}}%
\pgfpathlineto{\pgfqpoint{2.311923in}{0.958758in}}%
\pgfpathlineto{\pgfqpoint{2.309258in}{0.955569in}}%
\pgfpathlineto{\pgfqpoint{2.308788in}{0.955007in}}%
\pgfpathlineto{\pgfqpoint{2.305654in}{0.955007in}}%
\pgfpathlineto{\pgfqpoint{2.302989in}{0.951818in}}%
\pgfpathlineto{\pgfqpoint{2.302519in}{0.951255in}}%
\pgfpathlineto{\pgfqpoint{2.299384in}{0.951255in}}%
\pgfpathlineto{\pgfqpoint{2.296249in}{0.951255in}}%
\pgfpathlineto{\pgfqpoint{2.293585in}{0.948066in}}%
\pgfpathlineto{\pgfqpoint{2.293115in}{0.947504in}}%
\pgfpathlineto{\pgfqpoint{2.289980in}{0.947504in}}%
\pgfpathlineto{\pgfqpoint{2.286845in}{0.947504in}}%
\pgfpathlineto{\pgfqpoint{2.284181in}{0.944315in}}%
\pgfpathlineto{\pgfqpoint{2.283710in}{0.943752in}}%
\pgfpathlineto{\pgfqpoint{2.280576in}{0.943752in}}%
\pgfpathlineto{\pgfqpoint{2.277441in}{0.943752in}}%
\pgfpathlineto{\pgfqpoint{2.274776in}{0.940563in}}%
\pgfpathlineto{\pgfqpoint{2.274306in}{0.940001in}}%
\pgfpathlineto{\pgfqpoint{2.271171in}{0.940001in}}%
\pgfpathlineto{\pgfqpoint{2.268507in}{0.936812in}}%
\pgfpathlineto{\pgfqpoint{2.268037in}{0.936249in}}%
\pgfpathlineto{\pgfqpoint{2.264902in}{0.936249in}}%
\pgfpathlineto{\pgfqpoint{2.261767in}{0.936249in}}%
\pgfpathlineto{\pgfqpoint{2.259103in}{0.933060in}}%
\pgfpathlineto{\pgfqpoint{2.258632in}{0.932498in}}%
\pgfpathlineto{\pgfqpoint{2.255498in}{0.932498in}}%
\pgfpathlineto{\pgfqpoint{2.252363in}{0.932498in}}%
\pgfpathlineto{\pgfqpoint{2.249698in}{0.929309in}}%
\pgfpathlineto{\pgfqpoint{2.249228in}{0.928746in}}%
\pgfpathlineto{\pgfqpoint{2.246093in}{0.928746in}}%
\pgfpathlineto{\pgfqpoint{2.242959in}{0.928746in}}%
\pgfpathlineto{\pgfqpoint{2.240294in}{0.925557in}}%
\pgfpathlineto{\pgfqpoint{2.239824in}{0.924994in}}%
\pgfpathlineto{\pgfqpoint{2.236689in}{0.924994in}}%
\pgfpathlineto{\pgfqpoint{2.233554in}{0.924994in}}%
\pgfpathlineto{\pgfqpoint{2.230890in}{0.921806in}}%
\pgfpathlineto{\pgfqpoint{2.230420in}{0.921243in}}%
\pgfpathlineto{\pgfqpoint{2.227285in}{0.921243in}}%
\pgfpathlineto{\pgfqpoint{2.224620in}{0.918054in}}%
\pgfpathlineto{\pgfqpoint{2.224150in}{0.917491in}}%
\pgfpathlineto{\pgfqpoint{2.221015in}{0.917491in}}%
\pgfpathlineto{\pgfqpoint{2.217881in}{0.917491in}}%
\pgfpathlineto{\pgfqpoint{2.215216in}{0.914303in}}%
\pgfpathlineto{\pgfqpoint{2.214746in}{0.913740in}}%
\pgfpathlineto{\pgfqpoint{2.211611in}{0.913740in}}%
\pgfpathlineto{\pgfqpoint{2.208477in}{0.913740in}}%
\pgfpathlineto{\pgfqpoint{2.205812in}{0.910551in}}%
\pgfpathlineto{\pgfqpoint{2.205342in}{0.909988in}}%
\pgfpathlineto{\pgfqpoint{2.202207in}{0.909988in}}%
\pgfpathlineto{\pgfqpoint{2.199072in}{0.909988in}}%
\pgfpathlineto{\pgfqpoint{2.196408in}{0.906799in}}%
\pgfpathlineto{\pgfqpoint{2.195938in}{0.906237in}}%
\pgfpathlineto{\pgfqpoint{2.192803in}{0.906237in}}%
\pgfpathlineto{\pgfqpoint{2.190138in}{0.903048in}}%
\pgfpathlineto{\pgfqpoint{2.189668in}{0.902485in}}%
\pgfpathlineto{\pgfqpoint{2.186533in}{0.902485in}}%
\pgfpathlineto{\pgfqpoint{2.183399in}{0.902485in}}%
\pgfpathlineto{\pgfqpoint{2.180734in}{0.899296in}}%
\pgfpathlineto{\pgfqpoint{2.180264in}{0.898734in}}%
\pgfpathlineto{\pgfqpoint{2.177129in}{0.898734in}}%
\pgfpathlineto{\pgfqpoint{2.173994in}{0.898734in}}%
\pgfpathclose%
\pgfusepath{fill}%
\end{pgfscope}%
\begin{pgfscope}%
\pgfpathrectangle{\pgfqpoint{0.888750in}{0.419100in}}{\pgfqpoint{2.504659in}{2.933700in}} %
\pgfusepath{clip}%
\pgfsetbuttcap%
\pgfsetroundjoin%
\definecolor{currentfill}{rgb}{0.632651,1.000000,0.000000}%
\pgfsetfillcolor{currentfill}%
\pgfsetfillopacity{0.300000}%
\pgfsetlinewidth{0.000000pt}%
\definecolor{currentstroke}{rgb}{0.000000,0.000000,0.000000}%
\pgfsetstrokecolor{currentstroke}%
\pgfsetdash{}{0pt}%
\pgfpathmoveto{\pgfqpoint{2.173994in}{0.897608in}}%
\pgfpathlineto{\pgfqpoint{2.177129in}{0.897608in}}%
\pgfpathlineto{\pgfqpoint{2.180264in}{0.897608in}}%
\pgfpathlineto{\pgfqpoint{2.181674in}{0.899296in}}%
\pgfpathlineto{\pgfqpoint{2.183399in}{0.901360in}}%
\pgfpathlineto{\pgfqpoint{2.186533in}{0.901360in}}%
\pgfpathlineto{\pgfqpoint{2.189668in}{0.901360in}}%
\pgfpathlineto{\pgfqpoint{2.191079in}{0.903048in}}%
\pgfpathlineto{\pgfqpoint{2.192803in}{0.905111in}}%
\pgfpathlineto{\pgfqpoint{2.195938in}{0.905111in}}%
\pgfpathlineto{\pgfqpoint{2.197348in}{0.906799in}}%
\pgfpathlineto{\pgfqpoint{2.199072in}{0.908863in}}%
\pgfpathlineto{\pgfqpoint{2.202207in}{0.908863in}}%
\pgfpathlineto{\pgfqpoint{2.205342in}{0.908863in}}%
\pgfpathlineto{\pgfqpoint{2.206752in}{0.910551in}}%
\pgfpathlineto{\pgfqpoint{2.208477in}{0.912614in}}%
\pgfpathlineto{\pgfqpoint{2.211611in}{0.912614in}}%
\pgfpathlineto{\pgfqpoint{2.214746in}{0.912614in}}%
\pgfpathlineto{\pgfqpoint{2.216157in}{0.914303in}}%
\pgfpathlineto{\pgfqpoint{2.217881in}{0.916366in}}%
\pgfpathlineto{\pgfqpoint{2.221015in}{0.916366in}}%
\pgfpathlineto{\pgfqpoint{2.224150in}{0.916366in}}%
\pgfpathlineto{\pgfqpoint{2.225561in}{0.918054in}}%
\pgfpathlineto{\pgfqpoint{2.227285in}{0.920117in}}%
\pgfpathlineto{\pgfqpoint{2.230420in}{0.920117in}}%
\pgfpathlineto{\pgfqpoint{2.231830in}{0.921806in}}%
\pgfpathlineto{\pgfqpoint{2.233554in}{0.923869in}}%
\pgfpathlineto{\pgfqpoint{2.236689in}{0.923869in}}%
\pgfpathlineto{\pgfqpoint{2.239824in}{0.923869in}}%
\pgfpathlineto{\pgfqpoint{2.241235in}{0.925557in}}%
\pgfpathlineto{\pgfqpoint{2.242959in}{0.927621in}}%
\pgfpathlineto{\pgfqpoint{2.246093in}{0.927621in}}%
\pgfpathlineto{\pgfqpoint{2.249228in}{0.927621in}}%
\pgfpathlineto{\pgfqpoint{2.250639in}{0.929309in}}%
\pgfpathlineto{\pgfqpoint{2.252363in}{0.931372in}}%
\pgfpathlineto{\pgfqpoint{2.255498in}{0.931372in}}%
\pgfpathlineto{\pgfqpoint{2.258632in}{0.931372in}}%
\pgfpathlineto{\pgfqpoint{2.260043in}{0.933060in}}%
\pgfpathlineto{\pgfqpoint{2.261767in}{0.935124in}}%
\pgfpathlineto{\pgfqpoint{2.264902in}{0.935124in}}%
\pgfpathlineto{\pgfqpoint{2.268037in}{0.935124in}}%
\pgfpathlineto{\pgfqpoint{2.269447in}{0.936812in}}%
\pgfpathlineto{\pgfqpoint{2.271171in}{0.938875in}}%
\pgfpathlineto{\pgfqpoint{2.274306in}{0.938875in}}%
\pgfpathlineto{\pgfqpoint{2.275717in}{0.940563in}}%
\pgfpathlineto{\pgfqpoint{2.277441in}{0.942627in}}%
\pgfpathlineto{\pgfqpoint{2.280576in}{0.942627in}}%
\pgfpathlineto{\pgfqpoint{2.283710in}{0.942627in}}%
\pgfpathlineto{\pgfqpoint{2.285121in}{0.944315in}}%
\pgfpathlineto{\pgfqpoint{2.286845in}{0.946378in}}%
\pgfpathlineto{\pgfqpoint{2.289980in}{0.946378in}}%
\pgfpathlineto{\pgfqpoint{2.293115in}{0.946378in}}%
\pgfpathlineto{\pgfqpoint{2.294525in}{0.948066in}}%
\pgfpathlineto{\pgfqpoint{2.296249in}{0.950130in}}%
\pgfpathlineto{\pgfqpoint{2.299384in}{0.950130in}}%
\pgfpathlineto{\pgfqpoint{2.302519in}{0.950130in}}%
\pgfpathlineto{\pgfqpoint{2.303929in}{0.951818in}}%
\pgfpathlineto{\pgfqpoint{2.305654in}{0.953881in}}%
\pgfpathlineto{\pgfqpoint{2.308788in}{0.953881in}}%
\pgfpathlineto{\pgfqpoint{2.310199in}{0.955569in}}%
\pgfpathlineto{\pgfqpoint{2.311923in}{0.957633in}}%
\pgfpathlineto{\pgfqpoint{2.315058in}{0.957633in}}%
\pgfpathlineto{\pgfqpoint{2.318192in}{0.957633in}}%
\pgfpathlineto{\pgfqpoint{2.319603in}{0.959321in}}%
\pgfpathlineto{\pgfqpoint{2.321327in}{0.961384in}}%
\pgfpathlineto{\pgfqpoint{2.324462in}{0.961384in}}%
\pgfpathlineto{\pgfqpoint{2.327597in}{0.961384in}}%
\pgfpathlineto{\pgfqpoint{2.329007in}{0.963073in}}%
\pgfpathlineto{\pgfqpoint{2.330731in}{0.965136in}}%
\pgfpathlineto{\pgfqpoint{2.333866in}{0.965136in}}%
\pgfpathlineto{\pgfqpoint{2.337001in}{0.965136in}}%
\pgfpathlineto{\pgfqpoint{2.338412in}{0.966824in}}%
\pgfpathlineto{\pgfqpoint{2.340136in}{0.968887in}}%
\pgfpathlineto{\pgfqpoint{2.343270in}{0.968887in}}%
\pgfpathlineto{\pgfqpoint{2.344681in}{0.970576in}}%
\pgfpathlineto{\pgfqpoint{2.346405in}{0.972639in}}%
\pgfpathlineto{\pgfqpoint{2.349540in}{0.972639in}}%
\pgfpathlineto{\pgfqpoint{2.352675in}{0.972639in}}%
\pgfpathlineto{\pgfqpoint{2.354085in}{0.974327in}}%
\pgfpathlineto{\pgfqpoint{2.355809in}{0.976390in}}%
\pgfpathlineto{\pgfqpoint{2.358944in}{0.976390in}}%
\pgfpathlineto{\pgfqpoint{2.362079in}{0.976390in}}%
\pgfpathlineto{\pgfqpoint{2.363490in}{0.978079in}}%
\pgfpathlineto{\pgfqpoint{2.365214in}{0.980142in}}%
\pgfpathlineto{\pgfqpoint{2.368348in}{0.980142in}}%
\pgfpathlineto{\pgfqpoint{2.371483in}{0.980142in}}%
\pgfpathlineto{\pgfqpoint{2.372894in}{0.981830in}}%
\pgfpathlineto{\pgfqpoint{2.374618in}{0.983894in}}%
\pgfpathlineto{\pgfqpoint{2.377753in}{0.983894in}}%
\pgfpathlineto{\pgfqpoint{2.379163in}{0.985582in}}%
\pgfpathlineto{\pgfqpoint{2.380887in}{0.987645in}}%
\pgfpathlineto{\pgfqpoint{2.384022in}{0.987645in}}%
\pgfpathlineto{\pgfqpoint{2.387157in}{0.987645in}}%
\pgfpathlineto{\pgfqpoint{2.388567in}{0.989333in}}%
\pgfpathlineto{\pgfqpoint{2.390292in}{0.991397in}}%
\pgfpathlineto{\pgfqpoint{2.393426in}{0.991397in}}%
\pgfpathlineto{\pgfqpoint{2.396561in}{0.991397in}}%
\pgfpathlineto{\pgfqpoint{2.397972in}{0.993085in}}%
\pgfpathlineto{\pgfqpoint{2.399696in}{0.995148in}}%
\pgfpathlineto{\pgfqpoint{2.402831in}{0.995148in}}%
\pgfpathlineto{\pgfqpoint{2.405965in}{0.995148in}}%
\pgfpathlineto{\pgfqpoint{2.407376in}{0.996836in}}%
\pgfpathlineto{\pgfqpoint{2.409100in}{0.998900in}}%
\pgfpathlineto{\pgfqpoint{2.412235in}{0.998900in}}%
\pgfpathlineto{\pgfqpoint{2.415369in}{0.998900in}}%
\pgfpathlineto{\pgfqpoint{2.416780in}{1.000588in}}%
\pgfpathlineto{\pgfqpoint{2.418504in}{1.002651in}}%
\pgfpathlineto{\pgfqpoint{2.421639in}{1.002651in}}%
\pgfpathlineto{\pgfqpoint{2.423050in}{1.004339in}}%
\pgfpathlineto{\pgfqpoint{2.424774in}{1.006403in}}%
\pgfpathlineto{\pgfqpoint{2.427908in}{1.006403in}}%
\pgfpathlineto{\pgfqpoint{2.431043in}{1.006403in}}%
\pgfpathlineto{\pgfqpoint{2.432454in}{1.008091in}}%
\pgfpathlineto{\pgfqpoint{2.434178in}{1.010154in}}%
\pgfpathlineto{\pgfqpoint{2.437313in}{1.010154in}}%
\pgfpathlineto{\pgfqpoint{2.440447in}{1.010154in}}%
\pgfpathlineto{\pgfqpoint{2.441858in}{1.011842in}}%
\pgfpathlineto{\pgfqpoint{2.443582in}{1.013906in}}%
\pgfpathlineto{\pgfqpoint{2.446717in}{1.013906in}}%
\pgfpathlineto{\pgfqpoint{2.449852in}{1.013906in}}%
\pgfpathlineto{\pgfqpoint{2.451262in}{1.015594in}}%
\pgfpathlineto{\pgfqpoint{2.452986in}{1.017657in}}%
\pgfpathlineto{\pgfqpoint{2.456121in}{1.017657in}}%
\pgfpathlineto{\pgfqpoint{2.457532in}{1.019346in}}%
\pgfpathlineto{\pgfqpoint{2.459256in}{1.021409in}}%
\pgfpathlineto{\pgfqpoint{2.462391in}{1.021409in}}%
\pgfpathlineto{\pgfqpoint{2.465525in}{1.021409in}}%
\pgfpathlineto{\pgfqpoint{2.466936in}{1.023097in}}%
\pgfpathlineto{\pgfqpoint{2.468660in}{1.025160in}}%
\pgfpathlineto{\pgfqpoint{2.471795in}{1.025160in}}%
\pgfpathlineto{\pgfqpoint{2.474930in}{1.025160in}}%
\pgfpathlineto{\pgfqpoint{2.476340in}{1.026849in}}%
\pgfpathlineto{\pgfqpoint{2.478064in}{1.028912in}}%
\pgfpathlineto{\pgfqpoint{2.481199in}{1.028912in}}%
\pgfpathlineto{\pgfqpoint{2.484334in}{1.028912in}}%
\pgfpathlineto{\pgfqpoint{2.485744in}{1.030600in}}%
\pgfpathlineto{\pgfqpoint{2.487469in}{1.032663in}}%
\pgfpathlineto{\pgfqpoint{2.490603in}{1.032663in}}%
\pgfpathlineto{\pgfqpoint{2.492014in}{1.034352in}}%
\pgfpathlineto{\pgfqpoint{2.493738in}{1.036415in}}%
\pgfpathlineto{\pgfqpoint{2.496873in}{1.036415in}}%
\pgfpathlineto{\pgfqpoint{2.500008in}{1.036415in}}%
\pgfpathlineto{\pgfqpoint{2.501418in}{1.038103in}}%
\pgfpathlineto{\pgfqpoint{2.503142in}{1.040167in}}%
\pgfpathlineto{\pgfqpoint{2.506277in}{1.040167in}}%
\pgfpathlineto{\pgfqpoint{2.509412in}{1.040167in}}%
\pgfpathlineto{\pgfqpoint{2.510822in}{1.041855in}}%
\pgfpathlineto{\pgfqpoint{2.512547in}{1.043918in}}%
\pgfpathlineto{\pgfqpoint{2.515681in}{1.043918in}}%
\pgfpathlineto{\pgfqpoint{2.518816in}{1.043918in}}%
\pgfpathlineto{\pgfqpoint{2.520227in}{1.045606in}}%
\pgfpathlineto{\pgfqpoint{2.521951in}{1.047670in}}%
\pgfpathlineto{\pgfqpoint{2.525085in}{1.047670in}}%
\pgfpathlineto{\pgfqpoint{2.528220in}{1.047670in}}%
\pgfpathlineto{\pgfqpoint{2.529631in}{1.049358in}}%
\pgfpathlineto{\pgfqpoint{2.531355in}{1.051421in}}%
\pgfpathlineto{\pgfqpoint{2.534490in}{1.051421in}}%
\pgfpathlineto{\pgfqpoint{2.535900in}{1.053109in}}%
\pgfpathlineto{\pgfqpoint{2.537624in}{1.055173in}}%
\pgfpathlineto{\pgfqpoint{2.540759in}{1.055173in}}%
\pgfpathlineto{\pgfqpoint{2.543894in}{1.055173in}}%
\pgfpathlineto{\pgfqpoint{2.545305in}{1.056861in}}%
\pgfpathlineto{\pgfqpoint{2.547029in}{1.058924in}}%
\pgfpathlineto{\pgfqpoint{2.550163in}{1.058924in}}%
\pgfpathlineto{\pgfqpoint{2.553298in}{1.058924in}}%
\pgfpathlineto{\pgfqpoint{2.554709in}{1.060612in}}%
\pgfpathlineto{\pgfqpoint{2.556433in}{1.062676in}}%
\pgfpathlineto{\pgfqpoint{2.559568in}{1.062676in}}%
\pgfpathlineto{\pgfqpoint{2.562702in}{1.062676in}}%
\pgfpathlineto{\pgfqpoint{2.564113in}{1.064364in}}%
\pgfpathlineto{\pgfqpoint{2.565837in}{1.066427in}}%
\pgfpathlineto{\pgfqpoint{2.568972in}{1.066427in}}%
\pgfpathlineto{\pgfqpoint{2.570383in}{1.068115in}}%
\pgfpathlineto{\pgfqpoint{2.572107in}{1.070179in}}%
\pgfpathlineto{\pgfqpoint{2.575241in}{1.070179in}}%
\pgfpathlineto{\pgfqpoint{2.578376in}{1.070179in}}%
\pgfpathlineto{\pgfqpoint{2.579787in}{1.071867in}}%
\pgfpathlineto{\pgfqpoint{2.581511in}{1.073930in}}%
\pgfpathlineto{\pgfqpoint{2.584646in}{1.073930in}}%
\pgfpathlineto{\pgfqpoint{2.587780in}{1.073930in}}%
\pgfpathlineto{\pgfqpoint{2.589191in}{1.075619in}}%
\pgfpathlineto{\pgfqpoint{2.590915in}{1.077682in}}%
\pgfpathlineto{\pgfqpoint{2.594050in}{1.077682in}}%
\pgfpathlineto{\pgfqpoint{2.597185in}{1.077682in}}%
\pgfpathlineto{\pgfqpoint{2.598595in}{1.079370in}}%
\pgfpathlineto{\pgfqpoint{2.600319in}{1.081433in}}%
\pgfpathlineto{\pgfqpoint{2.603454in}{1.081433in}}%
\pgfpathlineto{\pgfqpoint{2.604865in}{1.083122in}}%
\pgfpathlineto{\pgfqpoint{2.606589in}{1.085185in}}%
\pgfpathlineto{\pgfqpoint{2.609724in}{1.085185in}}%
\pgfpathlineto{\pgfqpoint{2.612858in}{1.085185in}}%
\pgfpathlineto{\pgfqpoint{2.614269in}{1.086873in}}%
\pgfpathlineto{\pgfqpoint{2.615993in}{1.088936in}}%
\pgfpathlineto{\pgfqpoint{2.619128in}{1.088936in}}%
\pgfpathlineto{\pgfqpoint{2.622262in}{1.088936in}}%
\pgfpathlineto{\pgfqpoint{2.623673in}{1.090625in}}%
\pgfpathlineto{\pgfqpoint{2.625397in}{1.092688in}}%
\pgfpathlineto{\pgfqpoint{2.628532in}{1.092688in}}%
\pgfpathlineto{\pgfqpoint{2.631667in}{1.092688in}}%
\pgfpathlineto{\pgfqpoint{2.633077in}{1.094376in}}%
\pgfpathlineto{\pgfqpoint{2.634801in}{1.096440in}}%
\pgfpathlineto{\pgfqpoint{2.637936in}{1.096440in}}%
\pgfpathlineto{\pgfqpoint{2.641071in}{1.096440in}}%
\pgfpathlineto{\pgfqpoint{2.642482in}{1.098128in}}%
\pgfpathlineto{\pgfqpoint{2.644206in}{1.100191in}}%
\pgfpathlineto{\pgfqpoint{2.647340in}{1.100191in}}%
\pgfpathlineto{\pgfqpoint{2.648751in}{1.101879in}}%
\pgfpathlineto{\pgfqpoint{2.650475in}{1.103943in}}%
\pgfpathlineto{\pgfqpoint{2.653610in}{1.103943in}}%
\pgfpathlineto{\pgfqpoint{2.656745in}{1.103943in}}%
\pgfpathlineto{\pgfqpoint{2.658155in}{1.105631in}}%
\pgfpathlineto{\pgfqpoint{2.659879in}{1.107694in}}%
\pgfpathlineto{\pgfqpoint{2.663014in}{1.107694in}}%
\pgfpathlineto{\pgfqpoint{2.666149in}{1.107694in}}%
\pgfpathlineto{\pgfqpoint{2.667560in}{1.109382in}}%
\pgfpathlineto{\pgfqpoint{2.669284in}{1.111446in}}%
\pgfpathlineto{\pgfqpoint{2.672418in}{1.111446in}}%
\pgfpathlineto{\pgfqpoint{2.675553in}{1.111446in}}%
\pgfpathlineto{\pgfqpoint{2.676964in}{1.113134in}}%
\pgfpathlineto{\pgfqpoint{2.678688in}{1.115197in}}%
\pgfpathlineto{\pgfqpoint{2.681823in}{1.115197in}}%
\pgfpathlineto{\pgfqpoint{2.683233in}{1.116885in}}%
\pgfpathlineto{\pgfqpoint{2.684957in}{1.118949in}}%
\pgfpathlineto{\pgfqpoint{2.688092in}{1.118949in}}%
\pgfpathlineto{\pgfqpoint{2.691227in}{1.118949in}}%
\pgfpathlineto{\pgfqpoint{2.692637in}{1.120637in}}%
\pgfpathlineto{\pgfqpoint{2.694362in}{1.122700in}}%
\pgfpathlineto{\pgfqpoint{2.697496in}{1.122700in}}%
\pgfpathlineto{\pgfqpoint{2.700631in}{1.122700in}}%
\pgfpathlineto{\pgfqpoint{2.702042in}{1.124388in}}%
\pgfpathlineto{\pgfqpoint{2.703766in}{1.126452in}}%
\pgfpathlineto{\pgfqpoint{2.706901in}{1.126452in}}%
\pgfpathlineto{\pgfqpoint{2.710035in}{1.126452in}}%
\pgfpathlineto{\pgfqpoint{2.711446in}{1.128140in}}%
\pgfpathlineto{\pgfqpoint{2.713170in}{1.130203in}}%
\pgfpathlineto{\pgfqpoint{2.716305in}{1.130203in}}%
\pgfpathlineto{\pgfqpoint{2.717715in}{1.131892in}}%
\pgfpathlineto{\pgfqpoint{2.719439in}{1.133955in}}%
\pgfpathlineto{\pgfqpoint{2.722574in}{1.133955in}}%
\pgfpathlineto{\pgfqpoint{2.725709in}{1.133955in}}%
\pgfpathlineto{\pgfqpoint{2.727120in}{1.135643in}}%
\pgfpathlineto{\pgfqpoint{2.728844in}{1.137706in}}%
\pgfpathlineto{\pgfqpoint{2.731978in}{1.137706in}}%
\pgfpathlineto{\pgfqpoint{2.735113in}{1.137706in}}%
\pgfpathlineto{\pgfqpoint{2.736524in}{1.139395in}}%
\pgfpathlineto{\pgfqpoint{2.738248in}{1.141458in}}%
\pgfpathlineto{\pgfqpoint{2.741383in}{1.141458in}}%
\pgfpathlineto{\pgfqpoint{2.744517in}{1.141458in}}%
\pgfpathlineto{\pgfqpoint{2.745928in}{1.143146in}}%
\pgfpathlineto{\pgfqpoint{2.747652in}{1.145210in}}%
\pgfpathlineto{\pgfqpoint{2.750787in}{1.145210in}}%
\pgfpathlineto{\pgfqpoint{2.753922in}{1.145210in}}%
\pgfpathlineto{\pgfqpoint{2.755332in}{1.146898in}}%
\pgfpathlineto{\pgfqpoint{2.757056in}{1.148961in}}%
\pgfpathlineto{\pgfqpoint{2.760191in}{1.148961in}}%
\pgfpathlineto{\pgfqpoint{2.761602in}{1.150649in}}%
\pgfpathlineto{\pgfqpoint{2.763326in}{1.152713in}}%
\pgfpathlineto{\pgfqpoint{2.766461in}{1.152713in}}%
\pgfpathlineto{\pgfqpoint{2.769595in}{1.152713in}}%
\pgfpathlineto{\pgfqpoint{2.771006in}{1.154401in}}%
\pgfpathlineto{\pgfqpoint{2.772730in}{1.156464in}}%
\pgfpathlineto{\pgfqpoint{2.775865in}{1.156464in}}%
\pgfpathlineto{\pgfqpoint{2.779000in}{1.156464in}}%
\pgfpathlineto{\pgfqpoint{2.780410in}{1.158152in}}%
\pgfpathlineto{\pgfqpoint{2.782134in}{1.160216in}}%
\pgfpathlineto{\pgfqpoint{2.785269in}{1.160216in}}%
\pgfpathlineto{\pgfqpoint{2.788404in}{1.160216in}}%
\pgfpathlineto{\pgfqpoint{2.789814in}{1.161904in}}%
\pgfpathlineto{\pgfqpoint{2.791539in}{1.163967in}}%
\pgfpathlineto{\pgfqpoint{2.794673in}{1.163967in}}%
\pgfpathlineto{\pgfqpoint{2.796084in}{1.165655in}}%
\pgfpathlineto{\pgfqpoint{2.797808in}{1.167719in}}%
\pgfpathlineto{\pgfqpoint{2.800943in}{1.167719in}}%
\pgfpathlineto{\pgfqpoint{2.804078in}{1.167719in}}%
\pgfpathlineto{\pgfqpoint{2.805488in}{1.169407in}}%
\pgfpathlineto{\pgfqpoint{2.807212in}{1.171470in}}%
\pgfpathlineto{\pgfqpoint{2.810347in}{1.171470in}}%
\pgfpathlineto{\pgfqpoint{2.813482in}{1.171470in}}%
\pgfpathlineto{\pgfqpoint{2.814892in}{1.173158in}}%
\pgfpathlineto{\pgfqpoint{2.814892in}{1.176910in}}%
\pgfpathlineto{\pgfqpoint{2.816617in}{1.178973in}}%
\pgfpathlineto{\pgfqpoint{2.818027in}{1.180662in}}%
\pgfpathlineto{\pgfqpoint{2.818027in}{1.184413in}}%
\pgfpathlineto{\pgfqpoint{2.818027in}{1.188165in}}%
\pgfpathlineto{\pgfqpoint{2.819751in}{1.190228in}}%
\pgfpathlineto{\pgfqpoint{2.821162in}{1.191916in}}%
\pgfpathlineto{\pgfqpoint{2.821162in}{1.195668in}}%
\pgfpathlineto{\pgfqpoint{2.821162in}{1.199419in}}%
\pgfpathlineto{\pgfqpoint{2.822886in}{1.201483in}}%
\pgfpathlineto{\pgfqpoint{2.824297in}{1.203171in}}%
\pgfpathlineto{\pgfqpoint{2.824297in}{1.206922in}}%
\pgfpathlineto{\pgfqpoint{2.824297in}{1.210674in}}%
\pgfpathlineto{\pgfqpoint{2.826021in}{1.212737in}}%
\pgfpathlineto{\pgfqpoint{2.827431in}{1.214425in}}%
\pgfpathlineto{\pgfqpoint{2.827431in}{1.218177in}}%
\pgfpathlineto{\pgfqpoint{2.827431in}{1.221928in}}%
\pgfpathlineto{\pgfqpoint{2.829155in}{1.223992in}}%
\pgfpathlineto{\pgfqpoint{2.830566in}{1.225680in}}%
\pgfpathlineto{\pgfqpoint{2.830566in}{1.229431in}}%
\pgfpathlineto{\pgfqpoint{2.830566in}{1.233183in}}%
\pgfpathlineto{\pgfqpoint{2.832290in}{1.235246in}}%
\pgfpathlineto{\pgfqpoint{2.833701in}{1.236935in}}%
\pgfpathlineto{\pgfqpoint{2.833701in}{1.240686in}}%
\pgfpathlineto{\pgfqpoint{2.833701in}{1.244438in}}%
\pgfpathlineto{\pgfqpoint{2.835425in}{1.246501in}}%
\pgfpathlineto{\pgfqpoint{2.836836in}{1.248189in}}%
\pgfpathlineto{\pgfqpoint{2.836836in}{1.251941in}}%
\pgfpathlineto{\pgfqpoint{2.836836in}{1.255692in}}%
\pgfpathlineto{\pgfqpoint{2.836836in}{1.259444in}}%
\pgfpathlineto{\pgfqpoint{2.838560in}{1.261507in}}%
\pgfpathlineto{\pgfqpoint{2.839970in}{1.263195in}}%
\pgfpathlineto{\pgfqpoint{2.839970in}{1.266947in}}%
\pgfpathlineto{\pgfqpoint{2.839970in}{1.270698in}}%
\pgfpathlineto{\pgfqpoint{2.841694in}{1.272762in}}%
\pgfpathlineto{\pgfqpoint{2.843105in}{1.274450in}}%
\pgfpathlineto{\pgfqpoint{2.843105in}{1.278201in}}%
\pgfpathlineto{\pgfqpoint{2.843105in}{1.281953in}}%
\pgfpathlineto{\pgfqpoint{2.844829in}{1.284016in}}%
\pgfpathlineto{\pgfqpoint{2.846240in}{1.285704in}}%
\pgfpathlineto{\pgfqpoint{2.846240in}{1.289456in}}%
\pgfpathlineto{\pgfqpoint{2.846240in}{1.293208in}}%
\pgfpathlineto{\pgfqpoint{2.847964in}{1.295271in}}%
\pgfpathlineto{\pgfqpoint{2.849375in}{1.296959in}}%
\pgfpathlineto{\pgfqpoint{2.849375in}{1.300711in}}%
\pgfpathlineto{\pgfqpoint{2.849375in}{1.304462in}}%
\pgfpathlineto{\pgfqpoint{2.851099in}{1.306525in}}%
\pgfpathlineto{\pgfqpoint{2.852509in}{1.308214in}}%
\pgfpathlineto{\pgfqpoint{2.852509in}{1.311965in}}%
\pgfpathlineto{\pgfqpoint{2.852509in}{1.315717in}}%
\pgfpathlineto{\pgfqpoint{2.854233in}{1.317780in}}%
\pgfpathlineto{\pgfqpoint{2.855644in}{1.319468in}}%
\pgfpathlineto{\pgfqpoint{2.855644in}{1.323220in}}%
\pgfpathlineto{\pgfqpoint{2.855644in}{1.326971in}}%
\pgfpathlineto{\pgfqpoint{2.857368in}{1.329035in}}%
\pgfpathlineto{\pgfqpoint{2.858779in}{1.330723in}}%
\pgfpathlineto{\pgfqpoint{2.858779in}{1.334474in}}%
\pgfpathlineto{\pgfqpoint{2.858779in}{1.338226in}}%
\pgfpathlineto{\pgfqpoint{2.860503in}{1.340289in}}%
\pgfpathlineto{\pgfqpoint{2.861914in}{1.341977in}}%
\pgfpathlineto{\pgfqpoint{2.861914in}{1.345729in}}%
\pgfpathlineto{\pgfqpoint{2.861914in}{1.349481in}}%
\pgfpathlineto{\pgfqpoint{2.863638in}{1.351544in}}%
\pgfpathlineto{\pgfqpoint{2.865048in}{1.353232in}}%
\pgfpathlineto{\pgfqpoint{2.865048in}{1.356984in}}%
\pgfpathlineto{\pgfqpoint{2.865048in}{1.360735in}}%
\pgfpathlineto{\pgfqpoint{2.865048in}{1.364487in}}%
\pgfpathlineto{\pgfqpoint{2.866772in}{1.366550in}}%
\pgfpathlineto{\pgfqpoint{2.868183in}{1.368238in}}%
\pgfpathlineto{\pgfqpoint{2.868183in}{1.371990in}}%
\pgfpathlineto{\pgfqpoint{2.868183in}{1.375741in}}%
\pgfpathlineto{\pgfqpoint{2.869907in}{1.377805in}}%
\pgfpathlineto{\pgfqpoint{2.871318in}{1.379493in}}%
\pgfpathlineto{\pgfqpoint{2.871318in}{1.383244in}}%
\pgfpathlineto{\pgfqpoint{2.871318in}{1.386996in}}%
\pgfpathlineto{\pgfqpoint{2.873042in}{1.389059in}}%
\pgfpathlineto{\pgfqpoint{2.874453in}{1.390747in}}%
\pgfpathlineto{\pgfqpoint{2.874453in}{1.394499in}}%
\pgfpathlineto{\pgfqpoint{2.874453in}{1.398251in}}%
\pgfpathlineto{\pgfqpoint{2.876177in}{1.400314in}}%
\pgfpathlineto{\pgfqpoint{2.877587in}{1.402002in}}%
\pgfpathlineto{\pgfqpoint{2.877587in}{1.405754in}}%
\pgfpathlineto{\pgfqpoint{2.877587in}{1.409505in}}%
\pgfpathlineto{\pgfqpoint{2.879311in}{1.411568in}}%
\pgfpathlineto{\pgfqpoint{2.880722in}{1.413257in}}%
\pgfpathlineto{\pgfqpoint{2.880722in}{1.417008in}}%
\pgfpathlineto{\pgfqpoint{2.880722in}{1.420760in}}%
\pgfpathlineto{\pgfqpoint{2.882446in}{1.422823in}}%
\pgfpathlineto{\pgfqpoint{2.883857in}{1.424511in}}%
\pgfpathlineto{\pgfqpoint{2.883857in}{1.428263in}}%
\pgfpathlineto{\pgfqpoint{2.883857in}{1.432014in}}%
\pgfpathlineto{\pgfqpoint{2.885581in}{1.434078in}}%
\pgfpathlineto{\pgfqpoint{2.886991in}{1.435766in}}%
\pgfpathlineto{\pgfqpoint{2.886991in}{1.439517in}}%
\pgfpathlineto{\pgfqpoint{2.886991in}{1.443269in}}%
\pgfpathlineto{\pgfqpoint{2.888716in}{1.445332in}}%
\pgfpathlineto{\pgfqpoint{2.890126in}{1.447020in}}%
\pgfpathlineto{\pgfqpoint{2.890126in}{1.450772in}}%
\pgfpathlineto{\pgfqpoint{2.890126in}{1.454524in}}%
\pgfpathlineto{\pgfqpoint{2.890126in}{1.458275in}}%
\pgfpathlineto{\pgfqpoint{2.891850in}{1.460338in}}%
\pgfpathlineto{\pgfqpoint{2.893261in}{1.462027in}}%
\pgfpathlineto{\pgfqpoint{2.893261in}{1.465778in}}%
\pgfpathlineto{\pgfqpoint{2.893261in}{1.469530in}}%
\pgfpathlineto{\pgfqpoint{2.894985in}{1.471593in}}%
\pgfpathlineto{\pgfqpoint{2.896396in}{1.473281in}}%
\pgfpathlineto{\pgfqpoint{2.896396in}{1.477033in}}%
\pgfpathlineto{\pgfqpoint{2.896396in}{1.480784in}}%
\pgfpathlineto{\pgfqpoint{2.898120in}{1.482848in}}%
\pgfpathlineto{\pgfqpoint{2.899530in}{1.484536in}}%
\pgfpathlineto{\pgfqpoint{2.899530in}{1.488287in}}%
\pgfpathlineto{\pgfqpoint{2.899530in}{1.492039in}}%
\pgfpathlineto{\pgfqpoint{2.901255in}{1.494102in}}%
\pgfpathlineto{\pgfqpoint{2.902665in}{1.495790in}}%
\pgfpathlineto{\pgfqpoint{2.902665in}{1.499542in}}%
\pgfpathlineto{\pgfqpoint{2.902665in}{1.503293in}}%
\pgfpathlineto{\pgfqpoint{2.904389in}{1.505357in}}%
\pgfpathlineto{\pgfqpoint{2.905800in}{1.507045in}}%
\pgfpathlineto{\pgfqpoint{2.905800in}{1.510797in}}%
\pgfpathlineto{\pgfqpoint{2.905800in}{1.514548in}}%
\pgfpathlineto{\pgfqpoint{2.907524in}{1.516611in}}%
\pgfpathlineto{\pgfqpoint{2.908935in}{1.518300in}}%
\pgfpathlineto{\pgfqpoint{2.908935in}{1.522051in}}%
\pgfpathlineto{\pgfqpoint{2.908935in}{1.525803in}}%
\pgfpathlineto{\pgfqpoint{2.910659in}{1.527866in}}%
\pgfpathlineto{\pgfqpoint{2.912069in}{1.529554in}}%
\pgfpathlineto{\pgfqpoint{2.912069in}{1.533306in}}%
\pgfpathlineto{\pgfqpoint{2.912069in}{1.537057in}}%
\pgfpathlineto{\pgfqpoint{2.913794in}{1.539121in}}%
\pgfpathlineto{\pgfqpoint{2.915204in}{1.540809in}}%
\pgfpathlineto{\pgfqpoint{2.915204in}{1.544560in}}%
\pgfpathlineto{\pgfqpoint{2.915204in}{1.548312in}}%
\pgfpathlineto{\pgfqpoint{2.916928in}{1.550375in}}%
\pgfpathlineto{\pgfqpoint{2.918339in}{1.552063in}}%
\pgfpathlineto{\pgfqpoint{2.918339in}{1.555815in}}%
\pgfpathlineto{\pgfqpoint{2.918339in}{1.559566in}}%
\pgfpathlineto{\pgfqpoint{2.918339in}{1.563318in}}%
\pgfpathlineto{\pgfqpoint{2.920063in}{1.565381in}}%
\pgfpathlineto{\pgfqpoint{2.921474in}{1.567070in}}%
\pgfpathlineto{\pgfqpoint{2.921474in}{1.570821in}}%
\pgfpathlineto{\pgfqpoint{2.921474in}{1.574573in}}%
\pgfpathlineto{\pgfqpoint{2.923198in}{1.576636in}}%
\pgfpathlineto{\pgfqpoint{2.924608in}{1.578324in}}%
\pgfpathlineto{\pgfqpoint{2.924608in}{1.582076in}}%
\pgfpathlineto{\pgfqpoint{2.924608in}{1.585827in}}%
\pgfpathlineto{\pgfqpoint{2.926332in}{1.587891in}}%
\pgfpathlineto{\pgfqpoint{2.927743in}{1.589579in}}%
\pgfpathlineto{\pgfqpoint{2.927743in}{1.593330in}}%
\pgfpathlineto{\pgfqpoint{2.927743in}{1.597082in}}%
\pgfpathlineto{\pgfqpoint{2.929467in}{1.599145in}}%
\pgfpathlineto{\pgfqpoint{2.930878in}{1.600833in}}%
\pgfpathlineto{\pgfqpoint{2.930878in}{1.604585in}}%
\pgfpathlineto{\pgfqpoint{2.930878in}{1.608336in}}%
\pgfpathlineto{\pgfqpoint{2.932602in}{1.610400in}}%
\pgfpathlineto{\pgfqpoint{2.934013in}{1.612088in}}%
\pgfpathlineto{\pgfqpoint{2.934013in}{1.615840in}}%
\pgfpathlineto{\pgfqpoint{2.934013in}{1.619591in}}%
\pgfpathlineto{\pgfqpoint{2.935737in}{1.621654in}}%
\pgfpathlineto{\pgfqpoint{2.937147in}{1.623343in}}%
\pgfpathlineto{\pgfqpoint{2.937147in}{1.627094in}}%
\pgfpathlineto{\pgfqpoint{2.937147in}{1.630846in}}%
\pgfpathlineto{\pgfqpoint{2.938871in}{1.632909in}}%
\pgfpathlineto{\pgfqpoint{2.940282in}{1.634597in}}%
\pgfpathlineto{\pgfqpoint{2.940282in}{1.638349in}}%
\pgfpathlineto{\pgfqpoint{2.940282in}{1.642100in}}%
\pgfpathlineto{\pgfqpoint{2.942006in}{1.644164in}}%
\pgfpathlineto{\pgfqpoint{2.943417in}{1.645852in}}%
\pgfpathlineto{\pgfqpoint{2.943417in}{1.649603in}}%
\pgfpathlineto{\pgfqpoint{2.943417in}{1.653355in}}%
\pgfpathlineto{\pgfqpoint{2.943417in}{1.657106in}}%
\pgfpathlineto{\pgfqpoint{2.945141in}{1.659170in}}%
\pgfpathlineto{\pgfqpoint{2.946552in}{1.660858in}}%
\pgfpathlineto{\pgfqpoint{2.946552in}{1.664609in}}%
\pgfpathlineto{\pgfqpoint{2.946552in}{1.668361in}}%
\pgfpathlineto{\pgfqpoint{2.948276in}{1.670424in}}%
\pgfpathlineto{\pgfqpoint{2.949686in}{1.672113in}}%
\pgfpathlineto{\pgfqpoint{2.949686in}{1.675864in}}%
\pgfpathlineto{\pgfqpoint{2.949686in}{1.679616in}}%
\pgfpathlineto{\pgfqpoint{2.951410in}{1.681679in}}%
\pgfpathlineto{\pgfqpoint{2.952821in}{1.683367in}}%
\pgfpathlineto{\pgfqpoint{2.952821in}{1.687119in}}%
\pgfpathlineto{\pgfqpoint{2.952821in}{1.690870in}}%
\pgfpathlineto{\pgfqpoint{2.954545in}{1.692934in}}%
\pgfpathlineto{\pgfqpoint{2.955956in}{1.694622in}}%
\pgfpathlineto{\pgfqpoint{2.955956in}{1.698373in}}%
\pgfpathlineto{\pgfqpoint{2.955956in}{1.702125in}}%
\pgfpathlineto{\pgfqpoint{2.957680in}{1.704188in}}%
\pgfpathlineto{\pgfqpoint{2.959091in}{1.705876in}}%
\pgfpathlineto{\pgfqpoint{2.959091in}{1.709628in}}%
\pgfpathlineto{\pgfqpoint{2.959091in}{1.713379in}}%
\pgfpathlineto{\pgfqpoint{2.960815in}{1.715443in}}%
\pgfpathlineto{\pgfqpoint{2.962225in}{1.717131in}}%
\pgfpathlineto{\pgfqpoint{2.962225in}{1.720882in}}%
\pgfpathlineto{\pgfqpoint{2.962225in}{1.724634in}}%
\pgfpathlineto{\pgfqpoint{2.963949in}{1.726697in}}%
\pgfpathlineto{\pgfqpoint{2.965360in}{1.728386in}}%
\pgfpathlineto{\pgfqpoint{2.965360in}{1.732137in}}%
\pgfpathlineto{\pgfqpoint{2.965360in}{1.735889in}}%
\pgfpathlineto{\pgfqpoint{2.967084in}{1.737952in}}%
\pgfpathlineto{\pgfqpoint{2.968495in}{1.739640in}}%
\pgfpathlineto{\pgfqpoint{2.968495in}{1.743392in}}%
\pgfpathlineto{\pgfqpoint{2.968495in}{1.747143in}}%
\pgfpathlineto{\pgfqpoint{2.970219in}{1.749207in}}%
\pgfpathlineto{\pgfqpoint{2.971630in}{1.750895in}}%
\pgfpathlineto{\pgfqpoint{2.971630in}{1.754646in}}%
\pgfpathlineto{\pgfqpoint{2.971630in}{1.758398in}}%
\pgfpathlineto{\pgfqpoint{2.971630in}{1.762149in}}%
\pgfpathlineto{\pgfqpoint{2.973354in}{1.764213in}}%
\pgfpathlineto{\pgfqpoint{2.974764in}{1.765901in}}%
\pgfpathlineto{\pgfqpoint{2.974764in}{1.769652in}}%
\pgfpathlineto{\pgfqpoint{2.974764in}{1.773404in}}%
\pgfpathlineto{\pgfqpoint{2.976488in}{1.775467in}}%
\pgfpathlineto{\pgfqpoint{2.977899in}{1.777155in}}%
\pgfpathlineto{\pgfqpoint{2.977899in}{1.780907in}}%
\pgfpathlineto{\pgfqpoint{2.977899in}{1.784659in}}%
\pgfpathlineto{\pgfqpoint{2.979623in}{1.786722in}}%
\pgfpathlineto{\pgfqpoint{2.981034in}{1.788410in}}%
\pgfpathlineto{\pgfqpoint{2.981034in}{1.792162in}}%
\pgfpathlineto{\pgfqpoint{2.981034in}{1.795913in}}%
\pgfpathlineto{\pgfqpoint{2.982758in}{1.797977in}}%
\pgfpathlineto{\pgfqpoint{2.984168in}{1.799665in}}%
\pgfpathlineto{\pgfqpoint{2.984168in}{1.803416in}}%
\pgfpathlineto{\pgfqpoint{2.984168in}{1.807168in}}%
\pgfpathlineto{\pgfqpoint{2.985893in}{1.809231in}}%
\pgfpathlineto{\pgfqpoint{2.987303in}{1.810919in}}%
\pgfpathlineto{\pgfqpoint{2.987303in}{1.814671in}}%
\pgfpathlineto{\pgfqpoint{2.987303in}{1.818422in}}%
\pgfpathlineto{\pgfqpoint{2.989027in}{1.820486in}}%
\pgfpathlineto{\pgfqpoint{2.990438in}{1.822174in}}%
\pgfpathlineto{\pgfqpoint{2.990438in}{1.825925in}}%
\pgfpathlineto{\pgfqpoint{2.990438in}{1.829677in}}%
\pgfpathlineto{\pgfqpoint{2.992162in}{1.831740in}}%
\pgfpathlineto{\pgfqpoint{2.993573in}{1.833429in}}%
\pgfpathlineto{\pgfqpoint{2.993573in}{1.837180in}}%
\pgfpathlineto{\pgfqpoint{2.993573in}{1.840932in}}%
\pgfpathlineto{\pgfqpoint{2.995297in}{1.842995in}}%
\pgfpathlineto{\pgfqpoint{2.996707in}{1.844683in}}%
\pgfpathlineto{\pgfqpoint{2.996707in}{1.848435in}}%
\pgfpathlineto{\pgfqpoint{2.996707in}{1.852186in}}%
\pgfpathlineto{\pgfqpoint{2.996707in}{1.855938in}}%
\pgfpathlineto{\pgfqpoint{2.998432in}{1.858001in}}%
\pgfpathlineto{\pgfqpoint{2.999842in}{1.859689in}}%
\pgfpathlineto{\pgfqpoint{2.999842in}{1.863441in}}%
\pgfpathlineto{\pgfqpoint{2.999842in}{1.867192in}}%
\pgfpathlineto{\pgfqpoint{3.001566in}{1.869256in}}%
\pgfpathlineto{\pgfqpoint{3.002977in}{1.870944in}}%
\pgfpathlineto{\pgfqpoint{3.002977in}{1.874695in}}%
\pgfpathlineto{\pgfqpoint{3.002977in}{1.878447in}}%
\pgfpathlineto{\pgfqpoint{3.004701in}{1.880510in}}%
\pgfpathlineto{\pgfqpoint{3.006112in}{1.882198in}}%
\pgfpathlineto{\pgfqpoint{3.006112in}{1.885950in}}%
\pgfpathlineto{\pgfqpoint{3.006112in}{1.889702in}}%
\pgfpathlineto{\pgfqpoint{3.007836in}{1.891765in}}%
\pgfpathlineto{\pgfqpoint{3.009246in}{1.893453in}}%
\pgfpathlineto{\pgfqpoint{3.009246in}{1.897205in}}%
\pgfpathlineto{\pgfqpoint{3.009246in}{1.900956in}}%
\pgfpathlineto{\pgfqpoint{3.010971in}{1.903019in}}%
\pgfpathlineto{\pgfqpoint{3.012381in}{1.904708in}}%
\pgfpathlineto{\pgfqpoint{3.012381in}{1.908459in}}%
\pgfpathlineto{\pgfqpoint{3.012381in}{1.912211in}}%
\pgfpathlineto{\pgfqpoint{3.014105in}{1.914274in}}%
\pgfpathlineto{\pgfqpoint{3.015516in}{1.915962in}}%
\pgfpathlineto{\pgfqpoint{3.015516in}{1.919714in}}%
\pgfpathlineto{\pgfqpoint{3.015516in}{1.923465in}}%
\pgfpathlineto{\pgfqpoint{3.017240in}{1.925529in}}%
\pgfpathlineto{\pgfqpoint{3.018651in}{1.927217in}}%
\pgfpathlineto{\pgfqpoint{3.018651in}{1.930968in}}%
\pgfpathlineto{\pgfqpoint{3.018651in}{1.934720in}}%
\pgfpathlineto{\pgfqpoint{3.020375in}{1.936783in}}%
\pgfpathlineto{\pgfqpoint{3.021785in}{1.938471in}}%
\pgfpathlineto{\pgfqpoint{3.021785in}{1.942223in}}%
\pgfpathlineto{\pgfqpoint{3.021785in}{1.945975in}}%
\pgfpathlineto{\pgfqpoint{3.023510in}{1.948038in}}%
\pgfpathlineto{\pgfqpoint{3.024920in}{1.949726in}}%
\pgfpathlineto{\pgfqpoint{3.024920in}{1.953478in}}%
\pgfpathlineto{\pgfqpoint{3.024920in}{1.957229in}}%
\pgfpathlineto{\pgfqpoint{3.024920in}{1.960981in}}%
\pgfpathlineto{\pgfqpoint{3.026644in}{1.963044in}}%
\pgfpathlineto{\pgfqpoint{3.028055in}{1.964732in}}%
\pgfpathlineto{\pgfqpoint{3.028055in}{1.968484in}}%
\pgfpathlineto{\pgfqpoint{3.028055in}{1.972235in}}%
\pgfpathlineto{\pgfqpoint{3.029779in}{1.974299in}}%
\pgfpathlineto{\pgfqpoint{3.031190in}{1.975987in}}%
\pgfpathlineto{\pgfqpoint{3.031190in}{1.979738in}}%
\pgfpathlineto{\pgfqpoint{3.031190in}{1.983490in}}%
\pgfpathlineto{\pgfqpoint{3.032914in}{1.985553in}}%
\pgfpathlineto{\pgfqpoint{3.034324in}{1.987241in}}%
\pgfpathlineto{\pgfqpoint{3.034324in}{1.990993in}}%
\pgfpathlineto{\pgfqpoint{3.034324in}{1.994745in}}%
\pgfpathlineto{\pgfqpoint{3.036048in}{1.996808in}}%
\pgfpathlineto{\pgfqpoint{3.037459in}{1.998496in}}%
\pgfpathlineto{\pgfqpoint{3.037459in}{2.002248in}}%
\pgfpathlineto{\pgfqpoint{3.037459in}{2.005999in}}%
\pgfpathlineto{\pgfqpoint{3.039183in}{2.008062in}}%
\pgfpathlineto{\pgfqpoint{3.040594in}{2.009751in}}%
\pgfpathlineto{\pgfqpoint{3.040594in}{2.013502in}}%
\pgfpathlineto{\pgfqpoint{3.040594in}{2.017254in}}%
\pgfpathlineto{\pgfqpoint{3.042318in}{2.019317in}}%
\pgfpathlineto{\pgfqpoint{3.043729in}{2.021005in}}%
\pgfpathlineto{\pgfqpoint{3.043729in}{2.024757in}}%
\pgfpathlineto{\pgfqpoint{3.043729in}{2.028508in}}%
\pgfpathlineto{\pgfqpoint{3.045453in}{2.030572in}}%
\pgfpathlineto{\pgfqpoint{3.046863in}{2.032260in}}%
\pgfpathlineto{\pgfqpoint{3.046863in}{2.036011in}}%
\pgfpathlineto{\pgfqpoint{3.046863in}{2.039763in}}%
\pgfpathlineto{\pgfqpoint{3.048587in}{2.041826in}}%
\pgfpathlineto{\pgfqpoint{3.049998in}{2.043514in}}%
\pgfpathlineto{\pgfqpoint{3.049998in}{2.047266in}}%
\pgfpathlineto{\pgfqpoint{3.049998in}{2.051018in}}%
\pgfpathlineto{\pgfqpoint{3.049998in}{2.054769in}}%
\pgfpathlineto{\pgfqpoint{3.051722in}{2.056832in}}%
\pgfpathlineto{\pgfqpoint{3.053133in}{2.058521in}}%
\pgfpathlineto{\pgfqpoint{3.053133in}{2.062272in}}%
\pgfpathlineto{\pgfqpoint{3.053133in}{2.066024in}}%
\pgfpathlineto{\pgfqpoint{3.054857in}{2.068087in}}%
\pgfpathlineto{\pgfqpoint{3.056268in}{2.069775in}}%
\pgfpathlineto{\pgfqpoint{3.056268in}{2.073527in}}%
\pgfpathlineto{\pgfqpoint{3.056268in}{2.077278in}}%
\pgfpathlineto{\pgfqpoint{3.057992in}{2.079342in}}%
\pgfpathlineto{\pgfqpoint{3.059402in}{2.081030in}}%
\pgfpathlineto{\pgfqpoint{3.059402in}{2.084781in}}%
\pgfpathlineto{\pgfqpoint{3.059402in}{2.088533in}}%
\pgfpathlineto{\pgfqpoint{3.061126in}{2.090596in}}%
\pgfpathlineto{\pgfqpoint{3.062537in}{2.092284in}}%
\pgfpathlineto{\pgfqpoint{3.062537in}{2.096036in}}%
\pgfpathlineto{\pgfqpoint{3.062537in}{2.099787in}}%
\pgfpathlineto{\pgfqpoint{3.064261in}{2.101851in}}%
\pgfpathlineto{\pgfqpoint{3.065672in}{2.103539in}}%
\pgfpathlineto{\pgfqpoint{3.065672in}{2.107291in}}%
\pgfpathlineto{\pgfqpoint{3.065672in}{2.111042in}}%
\pgfpathlineto{\pgfqpoint{3.067396in}{2.113105in}}%
\pgfpathlineto{\pgfqpoint{3.068807in}{2.114794in}}%
\pgfpathlineto{\pgfqpoint{3.068807in}{2.118545in}}%
\pgfpathlineto{\pgfqpoint{3.068807in}{2.122297in}}%
\pgfpathlineto{\pgfqpoint{3.070531in}{2.124360in}}%
\pgfpathlineto{\pgfqpoint{3.071941in}{2.126048in}}%
\pgfpathlineto{\pgfqpoint{3.071941in}{2.129800in}}%
\pgfpathlineto{\pgfqpoint{3.071941in}{2.133551in}}%
\pgfpathlineto{\pgfqpoint{3.073665in}{2.135615in}}%
\pgfpathlineto{\pgfqpoint{3.075076in}{2.137303in}}%
\pgfpathlineto{\pgfqpoint{3.075076in}{2.141054in}}%
\pgfpathlineto{\pgfqpoint{3.075076in}{2.144806in}}%
\pgfpathlineto{\pgfqpoint{3.076800in}{2.146869in}}%
\pgfpathlineto{\pgfqpoint{3.078211in}{2.148557in}}%
\pgfpathlineto{\pgfqpoint{3.078211in}{2.152309in}}%
\pgfpathlineto{\pgfqpoint{3.078211in}{2.156060in}}%
\pgfpathlineto{\pgfqpoint{3.078211in}{2.159812in}}%
\pgfpathlineto{\pgfqpoint{3.079935in}{2.161875in}}%
\pgfpathlineto{\pgfqpoint{3.081345in}{2.163564in}}%
\pgfpathlineto{\pgfqpoint{3.081345in}{2.167315in}}%
\pgfpathlineto{\pgfqpoint{3.081345in}{2.171067in}}%
\pgfpathlineto{\pgfqpoint{3.083070in}{2.173130in}}%
\pgfpathlineto{\pgfqpoint{3.084480in}{2.174818in}}%
\pgfpathlineto{\pgfqpoint{3.084480in}{2.178570in}}%
\pgfpathlineto{\pgfqpoint{3.084480in}{2.182321in}}%
\pgfpathlineto{\pgfqpoint{3.086204in}{2.184385in}}%
\pgfpathlineto{\pgfqpoint{3.087615in}{2.186073in}}%
\pgfpathlineto{\pgfqpoint{3.087615in}{2.189824in}}%
\pgfpathlineto{\pgfqpoint{3.087615in}{2.193576in}}%
\pgfpathlineto{\pgfqpoint{3.089339in}{2.195639in}}%
\pgfpathlineto{\pgfqpoint{3.090750in}{2.197327in}}%
\pgfpathlineto{\pgfqpoint{3.090750in}{2.201079in}}%
\pgfpathlineto{\pgfqpoint{3.090750in}{2.204830in}}%
\pgfpathlineto{\pgfqpoint{3.092474in}{2.206894in}}%
\pgfpathlineto{\pgfqpoint{3.093884in}{2.208582in}}%
\pgfpathlineto{\pgfqpoint{3.093884in}{2.212334in}}%
\pgfpathlineto{\pgfqpoint{3.092474in}{2.214022in}}%
\pgfpathlineto{\pgfqpoint{3.090750in}{2.216085in}}%
\pgfpathlineto{\pgfqpoint{3.089339in}{2.217773in}}%
\pgfpathlineto{\pgfqpoint{3.087615in}{2.219837in}}%
\pgfpathlineto{\pgfqpoint{3.086204in}{2.221525in}}%
\pgfpathlineto{\pgfqpoint{3.084480in}{2.223588in}}%
\pgfpathlineto{\pgfqpoint{3.083070in}{2.225276in}}%
\pgfpathlineto{\pgfqpoint{3.081345in}{2.227340in}}%
\pgfpathlineto{\pgfqpoint{3.079935in}{2.229028in}}%
\pgfpathlineto{\pgfqpoint{3.078211in}{2.231091in}}%
\pgfpathlineto{\pgfqpoint{3.076800in}{2.232779in}}%
\pgfpathlineto{\pgfqpoint{3.075076in}{2.234843in}}%
\pgfpathlineto{\pgfqpoint{3.073665in}{2.236531in}}%
\pgfpathlineto{\pgfqpoint{3.071941in}{2.238594in}}%
\pgfpathlineto{\pgfqpoint{3.070531in}{2.240282in}}%
\pgfpathlineto{\pgfqpoint{3.068807in}{2.242346in}}%
\pgfpathlineto{\pgfqpoint{3.067396in}{2.244034in}}%
\pgfpathlineto{\pgfqpoint{3.065672in}{2.246097in}}%
\pgfpathlineto{\pgfqpoint{3.064261in}{2.247786in}}%
\pgfpathlineto{\pgfqpoint{3.062537in}{2.249849in}}%
\pgfpathlineto{\pgfqpoint{3.061126in}{2.251537in}}%
\pgfpathlineto{\pgfqpoint{3.059402in}{2.253600in}}%
\pgfpathlineto{\pgfqpoint{3.057992in}{2.255289in}}%
\pgfpathlineto{\pgfqpoint{3.056268in}{2.257352in}}%
\pgfpathlineto{\pgfqpoint{3.054857in}{2.259040in}}%
\pgfpathlineto{\pgfqpoint{3.053133in}{2.261103in}}%
\pgfpathlineto{\pgfqpoint{3.053133in}{2.264855in}}%
\pgfpathlineto{\pgfqpoint{3.051722in}{2.266543in}}%
\pgfpathlineto{\pgfqpoint{3.049998in}{2.268607in}}%
\pgfpathlineto{\pgfqpoint{3.048587in}{2.270295in}}%
\pgfpathlineto{\pgfqpoint{3.046863in}{2.272358in}}%
\pgfpathlineto{\pgfqpoint{3.045453in}{2.274046in}}%
\pgfpathlineto{\pgfqpoint{3.043729in}{2.276110in}}%
\pgfpathlineto{\pgfqpoint{3.042318in}{2.277798in}}%
\pgfpathlineto{\pgfqpoint{3.040594in}{2.279861in}}%
\pgfpathlineto{\pgfqpoint{3.039183in}{2.281549in}}%
\pgfpathlineto{\pgfqpoint{3.037459in}{2.283613in}}%
\pgfpathlineto{\pgfqpoint{3.036048in}{2.285301in}}%
\pgfpathlineto{\pgfqpoint{3.034324in}{2.287364in}}%
\pgfpathlineto{\pgfqpoint{3.032914in}{2.289052in}}%
\pgfpathlineto{\pgfqpoint{3.031190in}{2.291116in}}%
\pgfpathlineto{\pgfqpoint{3.029779in}{2.292804in}}%
\pgfpathlineto{\pgfqpoint{3.028055in}{2.294867in}}%
\pgfpathlineto{\pgfqpoint{3.026644in}{2.296555in}}%
\pgfpathlineto{\pgfqpoint{3.024920in}{2.298619in}}%
\pgfpathlineto{\pgfqpoint{3.023510in}{2.300307in}}%
\pgfpathlineto{\pgfqpoint{3.021785in}{2.302370in}}%
\pgfpathlineto{\pgfqpoint{3.020375in}{2.304059in}}%
\pgfpathlineto{\pgfqpoint{3.018651in}{2.306122in}}%
\pgfpathlineto{\pgfqpoint{3.017240in}{2.307810in}}%
\pgfpathlineto{\pgfqpoint{3.015516in}{2.309873in}}%
\pgfpathlineto{\pgfqpoint{3.014105in}{2.311562in}}%
\pgfpathlineto{\pgfqpoint{3.012381in}{2.313625in}}%
\pgfpathlineto{\pgfqpoint{3.010971in}{2.315313in}}%
\pgfpathlineto{\pgfqpoint{3.009246in}{2.317376in}}%
\pgfpathlineto{\pgfqpoint{3.007836in}{2.319065in}}%
\pgfpathlineto{\pgfqpoint{3.006112in}{2.321128in}}%
\pgfpathlineto{\pgfqpoint{3.004701in}{2.322816in}}%
\pgfpathlineto{\pgfqpoint{3.002977in}{2.324880in}}%
\pgfpathlineto{\pgfqpoint{3.002977in}{2.328631in}}%
\pgfpathlineto{\pgfqpoint{3.001566in}{2.330319in}}%
\pgfpathlineto{\pgfqpoint{2.999842in}{2.332383in}}%
\pgfpathlineto{\pgfqpoint{2.998432in}{2.334071in}}%
\pgfpathlineto{\pgfqpoint{2.996707in}{2.336134in}}%
\pgfpathlineto{\pgfqpoint{2.995297in}{2.337822in}}%
\pgfpathlineto{\pgfqpoint{2.993573in}{2.339886in}}%
\pgfpathlineto{\pgfqpoint{2.992162in}{2.341574in}}%
\pgfpathlineto{\pgfqpoint{2.990438in}{2.343637in}}%
\pgfpathlineto{\pgfqpoint{2.989027in}{2.345325in}}%
\pgfpathlineto{\pgfqpoint{2.987303in}{2.347389in}}%
\pgfpathlineto{\pgfqpoint{2.985893in}{2.349077in}}%
\pgfpathlineto{\pgfqpoint{2.984168in}{2.351140in}}%
\pgfpathlineto{\pgfqpoint{2.982758in}{2.352828in}}%
\pgfpathlineto{\pgfqpoint{2.981034in}{2.354892in}}%
\pgfpathlineto{\pgfqpoint{2.979623in}{2.356580in}}%
\pgfpathlineto{\pgfqpoint{2.977899in}{2.358643in}}%
\pgfpathlineto{\pgfqpoint{2.976488in}{2.360332in}}%
\pgfpathlineto{\pgfqpoint{2.974764in}{2.362395in}}%
\pgfpathlineto{\pgfqpoint{2.973354in}{2.364083in}}%
\pgfpathlineto{\pgfqpoint{2.971630in}{2.366146in}}%
\pgfpathlineto{\pgfqpoint{2.970219in}{2.367835in}}%
\pgfpathlineto{\pgfqpoint{2.968495in}{2.369898in}}%
\pgfpathlineto{\pgfqpoint{2.967084in}{2.371586in}}%
\pgfpathlineto{\pgfqpoint{2.965360in}{2.373649in}}%
\pgfpathlineto{\pgfqpoint{2.963949in}{2.375338in}}%
\pgfpathlineto{\pgfqpoint{2.962225in}{2.377401in}}%
\pgfpathlineto{\pgfqpoint{2.960815in}{2.379089in}}%
\pgfpathlineto{\pgfqpoint{2.959091in}{2.381153in}}%
\pgfpathlineto{\pgfqpoint{2.957680in}{2.382841in}}%
\pgfpathlineto{\pgfqpoint{2.955956in}{2.384904in}}%
\pgfpathlineto{\pgfqpoint{2.955956in}{2.388656in}}%
\pgfpathlineto{\pgfqpoint{2.954545in}{2.390344in}}%
\pgfpathlineto{\pgfqpoint{2.952821in}{2.392407in}}%
\pgfpathlineto{\pgfqpoint{2.951410in}{2.394095in}}%
\pgfpathlineto{\pgfqpoint{2.949686in}{2.396159in}}%
\pgfpathlineto{\pgfqpoint{2.948276in}{2.397847in}}%
\pgfpathlineto{\pgfqpoint{2.946552in}{2.399910in}}%
\pgfpathlineto{\pgfqpoint{2.945141in}{2.401598in}}%
\pgfpathlineto{\pgfqpoint{2.943417in}{2.403662in}}%
\pgfpathlineto{\pgfqpoint{2.942006in}{2.405350in}}%
\pgfpathlineto{\pgfqpoint{2.940282in}{2.407413in}}%
\pgfpathlineto{\pgfqpoint{2.938871in}{2.409101in}}%
\pgfpathlineto{\pgfqpoint{2.937147in}{2.411165in}}%
\pgfpathlineto{\pgfqpoint{2.935737in}{2.412853in}}%
\pgfpathlineto{\pgfqpoint{2.934013in}{2.414916in}}%
\pgfpathlineto{\pgfqpoint{2.932602in}{2.416605in}}%
\pgfpathlineto{\pgfqpoint{2.930878in}{2.418668in}}%
\pgfpathlineto{\pgfqpoint{2.929467in}{2.420356in}}%
\pgfpathlineto{\pgfqpoint{2.927743in}{2.422419in}}%
\pgfpathlineto{\pgfqpoint{2.926332in}{2.424108in}}%
\pgfpathlineto{\pgfqpoint{2.924608in}{2.426171in}}%
\pgfpathlineto{\pgfqpoint{2.923198in}{2.427859in}}%
\pgfpathlineto{\pgfqpoint{2.921474in}{2.429923in}}%
\pgfpathlineto{\pgfqpoint{2.920063in}{2.431611in}}%
\pgfpathlineto{\pgfqpoint{2.918339in}{2.433674in}}%
\pgfpathlineto{\pgfqpoint{2.916928in}{2.435362in}}%
\pgfpathlineto{\pgfqpoint{2.915204in}{2.437426in}}%
\pgfpathlineto{\pgfqpoint{2.913794in}{2.439114in}}%
\pgfpathlineto{\pgfqpoint{2.912069in}{2.441177in}}%
\pgfpathlineto{\pgfqpoint{2.910659in}{2.442865in}}%
\pgfpathlineto{\pgfqpoint{2.908935in}{2.444929in}}%
\pgfpathlineto{\pgfqpoint{2.907524in}{2.446617in}}%
\pgfpathlineto{\pgfqpoint{2.905800in}{2.448680in}}%
\pgfpathlineto{\pgfqpoint{2.905800in}{2.452432in}}%
\pgfpathlineto{\pgfqpoint{2.904389in}{2.454120in}}%
\pgfpathlineto{\pgfqpoint{2.902665in}{2.456183in}}%
\pgfpathlineto{\pgfqpoint{2.901255in}{2.457871in}}%
\pgfpathlineto{\pgfqpoint{2.899530in}{2.459935in}}%
\pgfpathlineto{\pgfqpoint{2.898120in}{2.461623in}}%
\pgfpathlineto{\pgfqpoint{2.896396in}{2.463686in}}%
\pgfpathlineto{\pgfqpoint{2.894985in}{2.465375in}}%
\pgfpathlineto{\pgfqpoint{2.893261in}{2.467438in}}%
\pgfpathlineto{\pgfqpoint{2.891850in}{2.469126in}}%
\pgfpathlineto{\pgfqpoint{2.890126in}{2.471189in}}%
\pgfpathlineto{\pgfqpoint{2.888716in}{2.472878in}}%
\pgfpathlineto{\pgfqpoint{2.886991in}{2.474941in}}%
\pgfpathlineto{\pgfqpoint{2.885581in}{2.476629in}}%
\pgfpathlineto{\pgfqpoint{2.883857in}{2.478692in}}%
\pgfpathlineto{\pgfqpoint{2.882446in}{2.480381in}}%
\pgfpathlineto{\pgfqpoint{2.880722in}{2.482444in}}%
\pgfpathlineto{\pgfqpoint{2.879311in}{2.484132in}}%
\pgfpathlineto{\pgfqpoint{2.877587in}{2.486196in}}%
\pgfpathlineto{\pgfqpoint{2.876177in}{2.487884in}}%
\pgfpathlineto{\pgfqpoint{2.874453in}{2.489947in}}%
\pgfpathlineto{\pgfqpoint{2.873042in}{2.491635in}}%
\pgfpathlineto{\pgfqpoint{2.871318in}{2.493699in}}%
\pgfpathlineto{\pgfqpoint{2.869907in}{2.495387in}}%
\pgfpathlineto{\pgfqpoint{2.868183in}{2.497450in}}%
\pgfpathlineto{\pgfqpoint{2.866772in}{2.499138in}}%
\pgfpathlineto{\pgfqpoint{2.865048in}{2.501202in}}%
\pgfpathlineto{\pgfqpoint{2.863638in}{2.502890in}}%
\pgfpathlineto{\pgfqpoint{2.861914in}{2.504953in}}%
\pgfpathlineto{\pgfqpoint{2.860503in}{2.506641in}}%
\pgfpathlineto{\pgfqpoint{2.858779in}{2.508705in}}%
\pgfpathlineto{\pgfqpoint{2.858779in}{2.512456in}}%
\pgfpathlineto{\pgfqpoint{2.857368in}{2.514144in}}%
\pgfpathlineto{\pgfqpoint{2.855644in}{2.516208in}}%
\pgfpathlineto{\pgfqpoint{2.854233in}{2.517896in}}%
\pgfpathlineto{\pgfqpoint{2.852509in}{2.519959in}}%
\pgfpathlineto{\pgfqpoint{2.851099in}{2.521648in}}%
\pgfpathlineto{\pgfqpoint{2.849375in}{2.523711in}}%
\pgfpathlineto{\pgfqpoint{2.847964in}{2.525399in}}%
\pgfpathlineto{\pgfqpoint{2.846240in}{2.527462in}}%
\pgfpathlineto{\pgfqpoint{2.844829in}{2.529151in}}%
\pgfpathlineto{\pgfqpoint{2.843105in}{2.531214in}}%
\pgfpathlineto{\pgfqpoint{2.841694in}{2.532902in}}%
\pgfpathlineto{\pgfqpoint{2.839970in}{2.534965in}}%
\pgfpathlineto{\pgfqpoint{2.838560in}{2.536654in}}%
\pgfpathlineto{\pgfqpoint{2.836836in}{2.538717in}}%
\pgfpathlineto{\pgfqpoint{2.835425in}{2.540405in}}%
\pgfpathlineto{\pgfqpoint{2.833701in}{2.542469in}}%
\pgfpathlineto{\pgfqpoint{2.832290in}{2.544157in}}%
\pgfpathlineto{\pgfqpoint{2.830566in}{2.546220in}}%
\pgfpathlineto{\pgfqpoint{2.829155in}{2.547908in}}%
\pgfpathlineto{\pgfqpoint{2.827431in}{2.549972in}}%
\pgfpathlineto{\pgfqpoint{2.826021in}{2.551660in}}%
\pgfpathlineto{\pgfqpoint{2.824297in}{2.553723in}}%
\pgfpathlineto{\pgfqpoint{2.822886in}{2.555411in}}%
\pgfpathlineto{\pgfqpoint{2.821162in}{2.557475in}}%
\pgfpathlineto{\pgfqpoint{2.819751in}{2.559163in}}%
\pgfpathlineto{\pgfqpoint{2.818027in}{2.561226in}}%
\pgfpathlineto{\pgfqpoint{2.816617in}{2.562914in}}%
\pgfpathlineto{\pgfqpoint{2.814892in}{2.564978in}}%
\pgfpathlineto{\pgfqpoint{2.813482in}{2.566666in}}%
\pgfpathlineto{\pgfqpoint{2.811758in}{2.568729in}}%
\pgfpathlineto{\pgfqpoint{2.810347in}{2.570417in}}%
\pgfpathlineto{\pgfqpoint{2.808623in}{2.572481in}}%
\pgfpathlineto{\pgfqpoint{2.808623in}{2.576232in}}%
\pgfpathlineto{\pgfqpoint{2.807212in}{2.577921in}}%
\pgfpathlineto{\pgfqpoint{2.805488in}{2.579984in}}%
\pgfpathlineto{\pgfqpoint{2.804078in}{2.581672in}}%
\pgfpathlineto{\pgfqpoint{2.802353in}{2.583735in}}%
\pgfpathlineto{\pgfqpoint{2.800943in}{2.585424in}}%
\pgfpathlineto{\pgfqpoint{2.799219in}{2.587487in}}%
\pgfpathlineto{\pgfqpoint{2.797808in}{2.589175in}}%
\pgfpathlineto{\pgfqpoint{2.796084in}{2.591238in}}%
\pgfpathlineto{\pgfqpoint{2.794673in}{2.592927in}}%
\pgfpathlineto{\pgfqpoint{2.792949in}{2.594990in}}%
\pgfpathlineto{\pgfqpoint{2.791539in}{2.596678in}}%
\pgfpathlineto{\pgfqpoint{2.789814in}{2.598742in}}%
\pgfpathlineto{\pgfqpoint{2.788404in}{2.600430in}}%
\pgfpathlineto{\pgfqpoint{2.786680in}{2.602493in}}%
\pgfpathlineto{\pgfqpoint{2.785269in}{2.604181in}}%
\pgfpathlineto{\pgfqpoint{2.783545in}{2.606245in}}%
\pgfpathlineto{\pgfqpoint{2.782134in}{2.607933in}}%
\pgfpathlineto{\pgfqpoint{2.780410in}{2.609996in}}%
\pgfpathlineto{\pgfqpoint{2.779000in}{2.611684in}}%
\pgfpathlineto{\pgfqpoint{2.777275in}{2.613748in}}%
\pgfpathlineto{\pgfqpoint{2.775865in}{2.615436in}}%
\pgfpathlineto{\pgfqpoint{2.774141in}{2.617499in}}%
\pgfpathlineto{\pgfqpoint{2.772730in}{2.619187in}}%
\pgfpathlineto{\pgfqpoint{2.771006in}{2.621251in}}%
\pgfpathlineto{\pgfqpoint{2.769595in}{2.622939in}}%
\pgfpathlineto{\pgfqpoint{2.767871in}{2.625002in}}%
\pgfpathlineto{\pgfqpoint{2.766461in}{2.626690in}}%
\pgfpathlineto{\pgfqpoint{2.764737in}{2.628754in}}%
\pgfpathlineto{\pgfqpoint{2.763326in}{2.630442in}}%
\pgfpathlineto{\pgfqpoint{2.761602in}{2.632505in}}%
\pgfpathlineto{\pgfqpoint{2.761602in}{2.636257in}}%
\pgfpathlineto{\pgfqpoint{2.760191in}{2.637945in}}%
\pgfpathlineto{\pgfqpoint{2.758467in}{2.640008in}}%
\pgfpathlineto{\pgfqpoint{2.757056in}{2.641697in}}%
\pgfpathlineto{\pgfqpoint{2.755332in}{2.643760in}}%
\pgfpathlineto{\pgfqpoint{2.753922in}{2.645448in}}%
\pgfpathlineto{\pgfqpoint{2.752198in}{2.647512in}}%
\pgfpathlineto{\pgfqpoint{2.750787in}{2.649200in}}%
\pgfpathlineto{\pgfqpoint{2.749063in}{2.651263in}}%
\pgfpathlineto{\pgfqpoint{2.747652in}{2.652951in}}%
\pgfpathlineto{\pgfqpoint{2.745928in}{2.655015in}}%
\pgfpathlineto{\pgfqpoint{2.744517in}{2.656703in}}%
\pgfpathlineto{\pgfqpoint{2.742793in}{2.658766in}}%
\pgfpathlineto{\pgfqpoint{2.741383in}{2.660454in}}%
\pgfpathlineto{\pgfqpoint{2.739659in}{2.662518in}}%
\pgfpathlineto{\pgfqpoint{2.738248in}{2.664206in}}%
\pgfpathlineto{\pgfqpoint{2.736524in}{2.666269in}}%
\pgfpathlineto{\pgfqpoint{2.735113in}{2.667957in}}%
\pgfpathlineto{\pgfqpoint{2.733389in}{2.670021in}}%
\pgfpathlineto{\pgfqpoint{2.731978in}{2.671709in}}%
\pgfpathlineto{\pgfqpoint{2.730254in}{2.673772in}}%
\pgfpathlineto{\pgfqpoint{2.728844in}{2.675460in}}%
\pgfpathlineto{\pgfqpoint{2.727120in}{2.677524in}}%
\pgfpathlineto{\pgfqpoint{2.725709in}{2.679212in}}%
\pgfpathlineto{\pgfqpoint{2.723985in}{2.681275in}}%
\pgfpathlineto{\pgfqpoint{2.722574in}{2.682964in}}%
\pgfpathlineto{\pgfqpoint{2.720850in}{2.685027in}}%
\pgfpathlineto{\pgfqpoint{2.719439in}{2.686715in}}%
\pgfpathlineto{\pgfqpoint{2.717715in}{2.688778in}}%
\pgfpathlineto{\pgfqpoint{2.716305in}{2.690467in}}%
\pgfpathlineto{\pgfqpoint{2.714581in}{2.692530in}}%
\pgfpathlineto{\pgfqpoint{2.713170in}{2.694218in}}%
\pgfpathlineto{\pgfqpoint{2.711446in}{2.696281in}}%
\pgfpathlineto{\pgfqpoint{2.711446in}{2.700033in}}%
\pgfpathlineto{\pgfqpoint{2.710035in}{2.701721in}}%
\pgfpathlineto{\pgfqpoint{2.708311in}{2.703785in}}%
\pgfpathlineto{\pgfqpoint{2.706901in}{2.705473in}}%
\pgfpathlineto{\pgfqpoint{2.705176in}{2.707536in}}%
\pgfpathlineto{\pgfqpoint{2.703766in}{2.709224in}}%
\pgfpathlineto{\pgfqpoint{2.702042in}{2.711288in}}%
\pgfpathlineto{\pgfqpoint{2.700631in}{2.712976in}}%
\pgfpathlineto{\pgfqpoint{2.698907in}{2.715039in}}%
\pgfpathlineto{\pgfqpoint{2.697496in}{2.716727in}}%
\pgfpathlineto{\pgfqpoint{2.695772in}{2.718791in}}%
\pgfpathlineto{\pgfqpoint{2.694362in}{2.720479in}}%
\pgfpathlineto{\pgfqpoint{2.692637in}{2.722542in}}%
\pgfpathlineto{\pgfqpoint{2.691227in}{2.724230in}}%
\pgfpathlineto{\pgfqpoint{2.689503in}{2.726294in}}%
\pgfpathlineto{\pgfqpoint{2.688092in}{2.727982in}}%
\pgfpathlineto{\pgfqpoint{2.686368in}{2.730045in}}%
\pgfpathlineto{\pgfqpoint{2.684957in}{2.731733in}}%
\pgfpathlineto{\pgfqpoint{2.683233in}{2.733797in}}%
\pgfpathlineto{\pgfqpoint{2.681823in}{2.735485in}}%
\pgfpathlineto{\pgfqpoint{2.680098in}{2.737548in}}%
\pgfpathlineto{\pgfqpoint{2.678688in}{2.739237in}}%
\pgfpathlineto{\pgfqpoint{2.676964in}{2.741300in}}%
\pgfpathlineto{\pgfqpoint{2.675553in}{2.742988in}}%
\pgfpathlineto{\pgfqpoint{2.673829in}{2.745051in}}%
\pgfpathlineto{\pgfqpoint{2.672418in}{2.746740in}}%
\pgfpathlineto{\pgfqpoint{2.669284in}{2.746740in}}%
\pgfpathlineto{\pgfqpoint{2.666149in}{2.746740in}}%
\pgfpathlineto{\pgfqpoint{2.663014in}{2.746740in}}%
\pgfpathlineto{\pgfqpoint{2.659879in}{2.746740in}}%
\pgfpathlineto{\pgfqpoint{2.658469in}{2.745051in}}%
\pgfpathlineto{\pgfqpoint{2.656745in}{2.742988in}}%
\pgfpathlineto{\pgfqpoint{2.653610in}{2.742988in}}%
\pgfpathlineto{\pgfqpoint{2.650475in}{2.742988in}}%
\pgfpathlineto{\pgfqpoint{2.647340in}{2.742988in}}%
\pgfpathlineto{\pgfqpoint{2.645930in}{2.741300in}}%
\pgfpathlineto{\pgfqpoint{2.644206in}{2.739237in}}%
\pgfpathlineto{\pgfqpoint{2.641071in}{2.739237in}}%
\pgfpathlineto{\pgfqpoint{2.637936in}{2.739237in}}%
\pgfpathlineto{\pgfqpoint{2.634801in}{2.739237in}}%
\pgfpathlineto{\pgfqpoint{2.633391in}{2.737548in}}%
\pgfpathlineto{\pgfqpoint{2.631667in}{2.735485in}}%
\pgfpathlineto{\pgfqpoint{2.628532in}{2.735485in}}%
\pgfpathlineto{\pgfqpoint{2.625397in}{2.735485in}}%
\pgfpathlineto{\pgfqpoint{2.622262in}{2.735485in}}%
\pgfpathlineto{\pgfqpoint{2.619128in}{2.735485in}}%
\pgfpathlineto{\pgfqpoint{2.617717in}{2.733797in}}%
\pgfpathlineto{\pgfqpoint{2.615993in}{2.731733in}}%
\pgfpathlineto{\pgfqpoint{2.612858in}{2.731733in}}%
\pgfpathlineto{\pgfqpoint{2.609724in}{2.731733in}}%
\pgfpathlineto{\pgfqpoint{2.606589in}{2.731733in}}%
\pgfpathlineto{\pgfqpoint{2.605178in}{2.730045in}}%
\pgfpathlineto{\pgfqpoint{2.603454in}{2.727982in}}%
\pgfpathlineto{\pgfqpoint{2.600319in}{2.727982in}}%
\pgfpathlineto{\pgfqpoint{2.597185in}{2.727982in}}%
\pgfpathlineto{\pgfqpoint{2.594050in}{2.727982in}}%
\pgfpathlineto{\pgfqpoint{2.592639in}{2.726294in}}%
\pgfpathlineto{\pgfqpoint{2.590915in}{2.724230in}}%
\pgfpathlineto{\pgfqpoint{2.587780in}{2.724230in}}%
\pgfpathlineto{\pgfqpoint{2.584646in}{2.724230in}}%
\pgfpathlineto{\pgfqpoint{2.581511in}{2.724230in}}%
\pgfpathlineto{\pgfqpoint{2.580100in}{2.722542in}}%
\pgfpathlineto{\pgfqpoint{2.578376in}{2.720479in}}%
\pgfpathlineto{\pgfqpoint{2.575241in}{2.720479in}}%
\pgfpathlineto{\pgfqpoint{2.572107in}{2.720479in}}%
\pgfpathlineto{\pgfqpoint{2.568972in}{2.720479in}}%
\pgfpathlineto{\pgfqpoint{2.567561in}{2.718791in}}%
\pgfpathlineto{\pgfqpoint{2.565837in}{2.716727in}}%
\pgfpathlineto{\pgfqpoint{2.562702in}{2.716727in}}%
\pgfpathlineto{\pgfqpoint{2.559568in}{2.716727in}}%
\pgfpathlineto{\pgfqpoint{2.556433in}{2.716727in}}%
\pgfpathlineto{\pgfqpoint{2.553298in}{2.716727in}}%
\pgfpathlineto{\pgfqpoint{2.551888in}{2.715039in}}%
\pgfpathlineto{\pgfqpoint{2.550163in}{2.712976in}}%
\pgfpathlineto{\pgfqpoint{2.547029in}{2.712976in}}%
\pgfpathlineto{\pgfqpoint{2.543894in}{2.712976in}}%
\pgfpathlineto{\pgfqpoint{2.540759in}{2.712976in}}%
\pgfpathlineto{\pgfqpoint{2.539349in}{2.711288in}}%
\pgfpathlineto{\pgfqpoint{2.537624in}{2.709224in}}%
\pgfpathlineto{\pgfqpoint{2.534490in}{2.709224in}}%
\pgfpathlineto{\pgfqpoint{2.531355in}{2.709224in}}%
\pgfpathlineto{\pgfqpoint{2.528220in}{2.709224in}}%
\pgfpathlineto{\pgfqpoint{2.526810in}{2.707536in}}%
\pgfpathlineto{\pgfqpoint{2.525085in}{2.705473in}}%
\pgfpathlineto{\pgfqpoint{2.521951in}{2.705473in}}%
\pgfpathlineto{\pgfqpoint{2.518816in}{2.705473in}}%
\pgfpathlineto{\pgfqpoint{2.515681in}{2.705473in}}%
\pgfpathlineto{\pgfqpoint{2.514271in}{2.703785in}}%
\pgfpathlineto{\pgfqpoint{2.512547in}{2.701721in}}%
\pgfpathlineto{\pgfqpoint{2.509412in}{2.701721in}}%
\pgfpathlineto{\pgfqpoint{2.506277in}{2.701721in}}%
\pgfpathlineto{\pgfqpoint{2.503142in}{2.701721in}}%
\pgfpathlineto{\pgfqpoint{2.500008in}{2.701721in}}%
\pgfpathlineto{\pgfqpoint{2.498597in}{2.700033in}}%
\pgfpathlineto{\pgfqpoint{2.496873in}{2.697970in}}%
\pgfpathlineto{\pgfqpoint{2.493738in}{2.697970in}}%
\pgfpathlineto{\pgfqpoint{2.490603in}{2.697970in}}%
\pgfpathlineto{\pgfqpoint{2.487469in}{2.697970in}}%
\pgfpathlineto{\pgfqpoint{2.486058in}{2.696281in}}%
\pgfpathlineto{\pgfqpoint{2.484334in}{2.694218in}}%
\pgfpathlineto{\pgfqpoint{2.481199in}{2.694218in}}%
\pgfpathlineto{\pgfqpoint{2.478064in}{2.694218in}}%
\pgfpathlineto{\pgfqpoint{2.474930in}{2.694218in}}%
\pgfpathlineto{\pgfqpoint{2.473519in}{2.692530in}}%
\pgfpathlineto{\pgfqpoint{2.471795in}{2.690467in}}%
\pgfpathlineto{\pgfqpoint{2.468660in}{2.690467in}}%
\pgfpathlineto{\pgfqpoint{2.465525in}{2.690467in}}%
\pgfpathlineto{\pgfqpoint{2.462391in}{2.690467in}}%
\pgfpathlineto{\pgfqpoint{2.460980in}{2.688778in}}%
\pgfpathlineto{\pgfqpoint{2.459256in}{2.686715in}}%
\pgfpathlineto{\pgfqpoint{2.456121in}{2.686715in}}%
\pgfpathlineto{\pgfqpoint{2.452986in}{2.686715in}}%
\pgfpathlineto{\pgfqpoint{2.449852in}{2.686715in}}%
\pgfpathlineto{\pgfqpoint{2.448441in}{2.685027in}}%
\pgfpathlineto{\pgfqpoint{2.446717in}{2.682964in}}%
\pgfpathlineto{\pgfqpoint{2.443582in}{2.682964in}}%
\pgfpathlineto{\pgfqpoint{2.440447in}{2.682964in}}%
\pgfpathlineto{\pgfqpoint{2.437313in}{2.682964in}}%
\pgfpathlineto{\pgfqpoint{2.434178in}{2.682964in}}%
\pgfpathlineto{\pgfqpoint{2.432767in}{2.681275in}}%
\pgfpathlineto{\pgfqpoint{2.431043in}{2.679212in}}%
\pgfpathlineto{\pgfqpoint{2.427908in}{2.679212in}}%
\pgfpathlineto{\pgfqpoint{2.424774in}{2.679212in}}%
\pgfpathlineto{\pgfqpoint{2.421639in}{2.679212in}}%
\pgfpathlineto{\pgfqpoint{2.420228in}{2.677524in}}%
\pgfpathlineto{\pgfqpoint{2.418504in}{2.675460in}}%
\pgfpathlineto{\pgfqpoint{2.415369in}{2.675460in}}%
\pgfpathlineto{\pgfqpoint{2.412235in}{2.675460in}}%
\pgfpathlineto{\pgfqpoint{2.409100in}{2.675460in}}%
\pgfpathlineto{\pgfqpoint{2.407689in}{2.673772in}}%
\pgfpathlineto{\pgfqpoint{2.405965in}{2.671709in}}%
\pgfpathlineto{\pgfqpoint{2.402831in}{2.671709in}}%
\pgfpathlineto{\pgfqpoint{2.399696in}{2.671709in}}%
\pgfpathlineto{\pgfqpoint{2.396561in}{2.671709in}}%
\pgfpathlineto{\pgfqpoint{2.395150in}{2.670021in}}%
\pgfpathlineto{\pgfqpoint{2.393426in}{2.667957in}}%
\pgfpathlineto{\pgfqpoint{2.390292in}{2.667957in}}%
\pgfpathlineto{\pgfqpoint{2.387157in}{2.667957in}}%
\pgfpathlineto{\pgfqpoint{2.384022in}{2.667957in}}%
\pgfpathlineto{\pgfqpoint{2.380887in}{2.667957in}}%
\pgfpathlineto{\pgfqpoint{2.379477in}{2.666269in}}%
\pgfpathlineto{\pgfqpoint{2.377753in}{2.664206in}}%
\pgfpathlineto{\pgfqpoint{2.374618in}{2.664206in}}%
\pgfpathlineto{\pgfqpoint{2.371483in}{2.664206in}}%
\pgfpathlineto{\pgfqpoint{2.368348in}{2.664206in}}%
\pgfpathlineto{\pgfqpoint{2.366938in}{2.662518in}}%
\pgfpathlineto{\pgfqpoint{2.365214in}{2.660454in}}%
\pgfpathlineto{\pgfqpoint{2.362079in}{2.660454in}}%
\pgfpathlineto{\pgfqpoint{2.358944in}{2.660454in}}%
\pgfpathlineto{\pgfqpoint{2.355809in}{2.660454in}}%
\pgfpathlineto{\pgfqpoint{2.354399in}{2.658766in}}%
\pgfpathlineto{\pgfqpoint{2.352675in}{2.656703in}}%
\pgfpathlineto{\pgfqpoint{2.349540in}{2.656703in}}%
\pgfpathlineto{\pgfqpoint{2.346405in}{2.656703in}}%
\pgfpathlineto{\pgfqpoint{2.343270in}{2.656703in}}%
\pgfpathlineto{\pgfqpoint{2.341860in}{2.655015in}}%
\pgfpathlineto{\pgfqpoint{2.340136in}{2.652951in}}%
\pgfpathlineto{\pgfqpoint{2.337001in}{2.652951in}}%
\pgfpathlineto{\pgfqpoint{2.333866in}{2.652951in}}%
\pgfpathlineto{\pgfqpoint{2.330731in}{2.652951in}}%
\pgfpathlineto{\pgfqpoint{2.327597in}{2.652951in}}%
\pgfpathlineto{\pgfqpoint{2.326186in}{2.651263in}}%
\pgfpathlineto{\pgfqpoint{2.324462in}{2.649200in}}%
\pgfpathlineto{\pgfqpoint{2.321327in}{2.649200in}}%
\pgfpathlineto{\pgfqpoint{2.318192in}{2.649200in}}%
\pgfpathlineto{\pgfqpoint{2.315058in}{2.649200in}}%
\pgfpathlineto{\pgfqpoint{2.313647in}{2.647512in}}%
\pgfpathlineto{\pgfqpoint{2.311923in}{2.645448in}}%
\pgfpathlineto{\pgfqpoint{2.308788in}{2.645448in}}%
\pgfpathlineto{\pgfqpoint{2.305654in}{2.645448in}}%
\pgfpathlineto{\pgfqpoint{2.302519in}{2.645448in}}%
\pgfpathlineto{\pgfqpoint{2.301108in}{2.643760in}}%
\pgfpathlineto{\pgfqpoint{2.299384in}{2.641697in}}%
\pgfpathlineto{\pgfqpoint{2.296249in}{2.641697in}}%
\pgfpathlineto{\pgfqpoint{2.293115in}{2.641697in}}%
\pgfpathlineto{\pgfqpoint{2.289980in}{2.641697in}}%
\pgfpathlineto{\pgfqpoint{2.288569in}{2.640008in}}%
\pgfpathlineto{\pgfqpoint{2.286845in}{2.637945in}}%
\pgfpathlineto{\pgfqpoint{2.283710in}{2.637945in}}%
\pgfpathlineto{\pgfqpoint{2.280576in}{2.637945in}}%
\pgfpathlineto{\pgfqpoint{2.277441in}{2.637945in}}%
\pgfpathlineto{\pgfqpoint{2.276030in}{2.636257in}}%
\pgfpathlineto{\pgfqpoint{2.274306in}{2.634194in}}%
\pgfpathlineto{\pgfqpoint{2.271171in}{2.634194in}}%
\pgfpathlineto{\pgfqpoint{2.268037in}{2.634194in}}%
\pgfpathlineto{\pgfqpoint{2.264902in}{2.634194in}}%
\pgfpathlineto{\pgfqpoint{2.261767in}{2.634194in}}%
\pgfpathlineto{\pgfqpoint{2.260356in}{2.632505in}}%
\pgfpathlineto{\pgfqpoint{2.258632in}{2.630442in}}%
\pgfpathlineto{\pgfqpoint{2.255498in}{2.630442in}}%
\pgfpathlineto{\pgfqpoint{2.252363in}{2.630442in}}%
\pgfpathlineto{\pgfqpoint{2.249228in}{2.630442in}}%
\pgfpathlineto{\pgfqpoint{2.247818in}{2.628754in}}%
\pgfpathlineto{\pgfqpoint{2.246093in}{2.626690in}}%
\pgfpathlineto{\pgfqpoint{2.242959in}{2.626690in}}%
\pgfpathlineto{\pgfqpoint{2.239824in}{2.626690in}}%
\pgfpathlineto{\pgfqpoint{2.236689in}{2.626690in}}%
\pgfpathlineto{\pgfqpoint{2.235279in}{2.625002in}}%
\pgfpathlineto{\pgfqpoint{2.233554in}{2.622939in}}%
\pgfpathlineto{\pgfqpoint{2.230420in}{2.622939in}}%
\pgfpathlineto{\pgfqpoint{2.227285in}{2.622939in}}%
\pgfpathlineto{\pgfqpoint{2.224150in}{2.622939in}}%
\pgfpathlineto{\pgfqpoint{2.222740in}{2.621251in}}%
\pgfpathlineto{\pgfqpoint{2.221015in}{2.619187in}}%
\pgfpathlineto{\pgfqpoint{2.217881in}{2.619187in}}%
\pgfpathlineto{\pgfqpoint{2.214746in}{2.619187in}}%
\pgfpathlineto{\pgfqpoint{2.211611in}{2.619187in}}%
\pgfpathlineto{\pgfqpoint{2.208477in}{2.619187in}}%
\pgfpathlineto{\pgfqpoint{2.207066in}{2.617499in}}%
\pgfpathlineto{\pgfqpoint{2.205342in}{2.615436in}}%
\pgfpathlineto{\pgfqpoint{2.202207in}{2.615436in}}%
\pgfpathlineto{\pgfqpoint{2.199072in}{2.615436in}}%
\pgfpathlineto{\pgfqpoint{2.195938in}{2.615436in}}%
\pgfpathlineto{\pgfqpoint{2.194527in}{2.613748in}}%
\pgfpathlineto{\pgfqpoint{2.192803in}{2.611684in}}%
\pgfpathlineto{\pgfqpoint{2.189668in}{2.611684in}}%
\pgfpathlineto{\pgfqpoint{2.186533in}{2.611684in}}%
\pgfpathlineto{\pgfqpoint{2.183399in}{2.611684in}}%
\pgfpathlineto{\pgfqpoint{2.181988in}{2.609996in}}%
\pgfpathlineto{\pgfqpoint{2.180264in}{2.607933in}}%
\pgfpathlineto{\pgfqpoint{2.177129in}{2.607933in}}%
\pgfpathlineto{\pgfqpoint{2.173994in}{2.607933in}}%
\pgfpathlineto{\pgfqpoint{2.170860in}{2.607933in}}%
\pgfpathlineto{\pgfqpoint{2.169449in}{2.606245in}}%
\pgfpathlineto{\pgfqpoint{2.167725in}{2.604181in}}%
\pgfpathlineto{\pgfqpoint{2.164590in}{2.604181in}}%
\pgfpathlineto{\pgfqpoint{2.161455in}{2.604181in}}%
\pgfpathlineto{\pgfqpoint{2.158321in}{2.604181in}}%
\pgfpathlineto{\pgfqpoint{2.155186in}{2.604181in}}%
\pgfpathlineto{\pgfqpoint{2.153775in}{2.602493in}}%
\pgfpathlineto{\pgfqpoint{2.152051in}{2.600430in}}%
\pgfpathlineto{\pgfqpoint{2.148916in}{2.600430in}}%
\pgfpathlineto{\pgfqpoint{2.145782in}{2.600430in}}%
\pgfpathlineto{\pgfqpoint{2.142647in}{2.600430in}}%
\pgfpathlineto{\pgfqpoint{2.141236in}{2.598742in}}%
\pgfpathlineto{\pgfqpoint{2.139512in}{2.596678in}}%
\pgfpathlineto{\pgfqpoint{2.136377in}{2.596678in}}%
\pgfpathlineto{\pgfqpoint{2.133243in}{2.596678in}}%
\pgfpathlineto{\pgfqpoint{2.130108in}{2.596678in}}%
\pgfpathlineto{\pgfqpoint{2.128697in}{2.594990in}}%
\pgfpathlineto{\pgfqpoint{2.126973in}{2.592927in}}%
\pgfpathlineto{\pgfqpoint{2.123838in}{2.592927in}}%
\pgfpathlineto{\pgfqpoint{2.120704in}{2.592927in}}%
\pgfpathlineto{\pgfqpoint{2.117569in}{2.592927in}}%
\pgfpathlineto{\pgfqpoint{2.116158in}{2.591238in}}%
\pgfpathlineto{\pgfqpoint{2.114434in}{2.589175in}}%
\pgfpathlineto{\pgfqpoint{2.111299in}{2.589175in}}%
\pgfpathlineto{\pgfqpoint{2.108165in}{2.589175in}}%
\pgfpathlineto{\pgfqpoint{2.105030in}{2.589175in}}%
\pgfpathlineto{\pgfqpoint{2.103619in}{2.587487in}}%
\pgfpathlineto{\pgfqpoint{2.101895in}{2.585424in}}%
\pgfpathlineto{\pgfqpoint{2.098761in}{2.585424in}}%
\pgfpathlineto{\pgfqpoint{2.095626in}{2.585424in}}%
\pgfpathlineto{\pgfqpoint{2.092491in}{2.585424in}}%
\pgfpathlineto{\pgfqpoint{2.089356in}{2.585424in}}%
\pgfpathlineto{\pgfqpoint{2.087946in}{2.583735in}}%
\pgfpathlineto{\pgfqpoint{2.086222in}{2.581672in}}%
\pgfpathlineto{\pgfqpoint{2.083087in}{2.581672in}}%
\pgfpathlineto{\pgfqpoint{2.079952in}{2.581672in}}%
\pgfpathlineto{\pgfqpoint{2.076817in}{2.581672in}}%
\pgfpathlineto{\pgfqpoint{2.075407in}{2.579984in}}%
\pgfpathlineto{\pgfqpoint{2.073683in}{2.577921in}}%
\pgfpathlineto{\pgfqpoint{2.070548in}{2.577921in}}%
\pgfpathlineto{\pgfqpoint{2.067413in}{2.577921in}}%
\pgfpathlineto{\pgfqpoint{2.064278in}{2.577921in}}%
\pgfpathlineto{\pgfqpoint{2.062868in}{2.576232in}}%
\pgfpathlineto{\pgfqpoint{2.061144in}{2.574169in}}%
\pgfpathlineto{\pgfqpoint{2.058009in}{2.574169in}}%
\pgfpathlineto{\pgfqpoint{2.054874in}{2.574169in}}%
\pgfpathlineto{\pgfqpoint{2.051739in}{2.574169in}}%
\pgfpathlineto{\pgfqpoint{2.050329in}{2.572481in}}%
\pgfpathlineto{\pgfqpoint{2.048605in}{2.570417in}}%
\pgfpathlineto{\pgfqpoint{2.045470in}{2.570417in}}%
\pgfpathlineto{\pgfqpoint{2.042335in}{2.570417in}}%
\pgfpathlineto{\pgfqpoint{2.039200in}{2.570417in}}%
\pgfpathlineto{\pgfqpoint{2.036066in}{2.570417in}}%
\pgfpathlineto{\pgfqpoint{2.034655in}{2.568729in}}%
\pgfpathlineto{\pgfqpoint{2.032931in}{2.566666in}}%
\pgfpathlineto{\pgfqpoint{2.029796in}{2.566666in}}%
\pgfpathlineto{\pgfqpoint{2.026661in}{2.566666in}}%
\pgfpathlineto{\pgfqpoint{2.023527in}{2.566666in}}%
\pgfpathlineto{\pgfqpoint{2.022116in}{2.564978in}}%
\pgfpathlineto{\pgfqpoint{2.020392in}{2.562914in}}%
\pgfpathlineto{\pgfqpoint{2.017257in}{2.562914in}}%
\pgfpathlineto{\pgfqpoint{2.014122in}{2.562914in}}%
\pgfpathlineto{\pgfqpoint{2.010988in}{2.562914in}}%
\pgfpathlineto{\pgfqpoint{2.009577in}{2.561226in}}%
\pgfpathlineto{\pgfqpoint{2.007853in}{2.559163in}}%
\pgfpathlineto{\pgfqpoint{2.004718in}{2.559163in}}%
\pgfpathlineto{\pgfqpoint{2.001584in}{2.559163in}}%
\pgfpathlineto{\pgfqpoint{1.998449in}{2.559163in}}%
\pgfpathlineto{\pgfqpoint{1.997038in}{2.557475in}}%
\pgfpathlineto{\pgfqpoint{1.995314in}{2.555411in}}%
\pgfpathlineto{\pgfqpoint{1.992179in}{2.555411in}}%
\pgfpathlineto{\pgfqpoint{1.989045in}{2.555411in}}%
\pgfpathlineto{\pgfqpoint{1.985910in}{2.555411in}}%
\pgfpathlineto{\pgfqpoint{1.982775in}{2.555411in}}%
\pgfpathlineto{\pgfqpoint{1.981364in}{2.553723in}}%
\pgfpathlineto{\pgfqpoint{1.979640in}{2.551660in}}%
\pgfpathlineto{\pgfqpoint{1.976506in}{2.551660in}}%
\pgfpathlineto{\pgfqpoint{1.973371in}{2.551660in}}%
\pgfpathlineto{\pgfqpoint{1.970236in}{2.551660in}}%
\pgfpathlineto{\pgfqpoint{1.968825in}{2.549972in}}%
\pgfpathlineto{\pgfqpoint{1.967101in}{2.547908in}}%
\pgfpathlineto{\pgfqpoint{1.963967in}{2.547908in}}%
\pgfpathlineto{\pgfqpoint{1.960832in}{2.547908in}}%
\pgfpathlineto{\pgfqpoint{1.957697in}{2.547908in}}%
\pgfpathlineto{\pgfqpoint{1.956286in}{2.546220in}}%
\pgfpathlineto{\pgfqpoint{1.954562in}{2.544157in}}%
\pgfpathlineto{\pgfqpoint{1.951428in}{2.544157in}}%
\pgfpathlineto{\pgfqpoint{1.948293in}{2.544157in}}%
\pgfpathlineto{\pgfqpoint{1.945158in}{2.544157in}}%
\pgfpathlineto{\pgfqpoint{1.943748in}{2.542469in}}%
\pgfpathlineto{\pgfqpoint{1.942023in}{2.540405in}}%
\pgfpathlineto{\pgfqpoint{1.938889in}{2.540405in}}%
\pgfpathlineto{\pgfqpoint{1.935754in}{2.540405in}}%
\pgfpathlineto{\pgfqpoint{1.932619in}{2.540405in}}%
\pgfpathlineto{\pgfqpoint{1.931209in}{2.538717in}}%
\pgfpathlineto{\pgfqpoint{1.929484in}{2.536654in}}%
\pgfpathlineto{\pgfqpoint{1.926350in}{2.536654in}}%
\pgfpathlineto{\pgfqpoint{1.923215in}{2.536654in}}%
\pgfpathlineto{\pgfqpoint{1.920080in}{2.536654in}}%
\pgfpathlineto{\pgfqpoint{1.916945in}{2.536654in}}%
\pgfpathlineto{\pgfqpoint{1.915535in}{2.534965in}}%
\pgfpathlineto{\pgfqpoint{1.913811in}{2.532902in}}%
\pgfpathlineto{\pgfqpoint{1.910676in}{2.532902in}}%
\pgfpathlineto{\pgfqpoint{1.907541in}{2.532902in}}%
\pgfpathlineto{\pgfqpoint{1.904407in}{2.532902in}}%
\pgfpathlineto{\pgfqpoint{1.902996in}{2.531214in}}%
\pgfpathlineto{\pgfqpoint{1.901272in}{2.529151in}}%
\pgfpathlineto{\pgfqpoint{1.898137in}{2.529151in}}%
\pgfpathlineto{\pgfqpoint{1.895002in}{2.529151in}}%
\pgfpathlineto{\pgfqpoint{1.891868in}{2.529151in}}%
\pgfpathlineto{\pgfqpoint{1.890457in}{2.527462in}}%
\pgfpathlineto{\pgfqpoint{1.888733in}{2.525399in}}%
\pgfpathlineto{\pgfqpoint{1.885598in}{2.525399in}}%
\pgfpathlineto{\pgfqpoint{1.882463in}{2.525399in}}%
\pgfpathlineto{\pgfqpoint{1.879329in}{2.525399in}}%
\pgfpathlineto{\pgfqpoint{1.877918in}{2.523711in}}%
\pgfpathlineto{\pgfqpoint{1.876194in}{2.521648in}}%
\pgfpathlineto{\pgfqpoint{1.873059in}{2.521648in}}%
\pgfpathlineto{\pgfqpoint{1.869924in}{2.521648in}}%
\pgfpathlineto{\pgfqpoint{1.866790in}{2.521648in}}%
\pgfpathlineto{\pgfqpoint{1.863655in}{2.521648in}}%
\pgfpathlineto{\pgfqpoint{1.862244in}{2.519959in}}%
\pgfpathlineto{\pgfqpoint{1.860520in}{2.517896in}}%
\pgfpathlineto{\pgfqpoint{1.857385in}{2.517896in}}%
\pgfpathlineto{\pgfqpoint{1.854251in}{2.517896in}}%
\pgfpathlineto{\pgfqpoint{1.851116in}{2.517896in}}%
\pgfpathlineto{\pgfqpoint{1.849705in}{2.516208in}}%
\pgfpathlineto{\pgfqpoint{1.847981in}{2.514144in}}%
\pgfpathlineto{\pgfqpoint{1.844846in}{2.514144in}}%
\pgfpathlineto{\pgfqpoint{1.841712in}{2.514144in}}%
\pgfpathlineto{\pgfqpoint{1.838577in}{2.514144in}}%
\pgfpathlineto{\pgfqpoint{1.837166in}{2.512456in}}%
\pgfpathlineto{\pgfqpoint{1.835442in}{2.510393in}}%
\pgfpathlineto{\pgfqpoint{1.832307in}{2.510393in}}%
\pgfpathlineto{\pgfqpoint{1.829173in}{2.510393in}}%
\pgfpathlineto{\pgfqpoint{1.826038in}{2.510393in}}%
\pgfpathlineto{\pgfqpoint{1.824627in}{2.508705in}}%
\pgfpathlineto{\pgfqpoint{1.822903in}{2.506641in}}%
\pgfpathlineto{\pgfqpoint{1.819768in}{2.506641in}}%
\pgfpathlineto{\pgfqpoint{1.816634in}{2.506641in}}%
\pgfpathlineto{\pgfqpoint{1.813499in}{2.506641in}}%
\pgfpathlineto{\pgfqpoint{1.812088in}{2.504953in}}%
\pgfpathlineto{\pgfqpoint{1.810364in}{2.502890in}}%
\pgfpathlineto{\pgfqpoint{1.807229in}{2.502890in}}%
\pgfpathlineto{\pgfqpoint{1.804095in}{2.502890in}}%
\pgfpathlineto{\pgfqpoint{1.800960in}{2.502890in}}%
\pgfpathlineto{\pgfqpoint{1.797825in}{2.502890in}}%
\pgfpathlineto{\pgfqpoint{1.796415in}{2.501202in}}%
\pgfpathlineto{\pgfqpoint{1.794691in}{2.499138in}}%
\pgfpathlineto{\pgfqpoint{1.791556in}{2.499138in}}%
\pgfpathlineto{\pgfqpoint{1.788421in}{2.499138in}}%
\pgfpathlineto{\pgfqpoint{1.785286in}{2.499138in}}%
\pgfpathlineto{\pgfqpoint{1.783876in}{2.497450in}}%
\pgfpathlineto{\pgfqpoint{1.782152in}{2.495387in}}%
\pgfpathlineto{\pgfqpoint{1.779017in}{2.495387in}}%
\pgfpathlineto{\pgfqpoint{1.775882in}{2.495387in}}%
\pgfpathlineto{\pgfqpoint{1.772747in}{2.495387in}}%
\pgfpathlineto{\pgfqpoint{1.771337in}{2.493699in}}%
\pgfpathlineto{\pgfqpoint{1.769613in}{2.491635in}}%
\pgfpathlineto{\pgfqpoint{1.766478in}{2.491635in}}%
\pgfpathlineto{\pgfqpoint{1.763343in}{2.491635in}}%
\pgfpathlineto{\pgfqpoint{1.760208in}{2.491635in}}%
\pgfpathlineto{\pgfqpoint{1.758798in}{2.489947in}}%
\pgfpathlineto{\pgfqpoint{1.757074in}{2.487884in}}%
\pgfpathlineto{\pgfqpoint{1.753939in}{2.487884in}}%
\pgfpathlineto{\pgfqpoint{1.750804in}{2.487884in}}%
\pgfpathlineto{\pgfqpoint{1.747669in}{2.487884in}}%
\pgfpathlineto{\pgfqpoint{1.744535in}{2.487884in}}%
\pgfpathlineto{\pgfqpoint{1.743124in}{2.486196in}}%
\pgfpathlineto{\pgfqpoint{1.741400in}{2.484132in}}%
\pgfpathlineto{\pgfqpoint{1.738265in}{2.484132in}}%
\pgfpathlineto{\pgfqpoint{1.735130in}{2.484132in}}%
\pgfpathlineto{\pgfqpoint{1.731996in}{2.484132in}}%
\pgfpathlineto{\pgfqpoint{1.730585in}{2.482444in}}%
\pgfpathlineto{\pgfqpoint{1.728861in}{2.480381in}}%
\pgfpathlineto{\pgfqpoint{1.725726in}{2.480381in}}%
\pgfpathlineto{\pgfqpoint{1.722591in}{2.480381in}}%
\pgfpathlineto{\pgfqpoint{1.719457in}{2.480381in}}%
\pgfpathlineto{\pgfqpoint{1.718046in}{2.478692in}}%
\pgfpathlineto{\pgfqpoint{1.716322in}{2.476629in}}%
\pgfpathlineto{\pgfqpoint{1.713187in}{2.476629in}}%
\pgfpathlineto{\pgfqpoint{1.710052in}{2.476629in}}%
\pgfpathlineto{\pgfqpoint{1.706918in}{2.476629in}}%
\pgfpathlineto{\pgfqpoint{1.705507in}{2.474941in}}%
\pgfpathlineto{\pgfqpoint{1.703783in}{2.472878in}}%
\pgfpathlineto{\pgfqpoint{1.700648in}{2.472878in}}%
\pgfpathlineto{\pgfqpoint{1.697514in}{2.472878in}}%
\pgfpathlineto{\pgfqpoint{1.694379in}{2.472878in}}%
\pgfpathlineto{\pgfqpoint{1.691244in}{2.472878in}}%
\pgfpathlineto{\pgfqpoint{1.689833in}{2.471189in}}%
\pgfpathlineto{\pgfqpoint{1.688109in}{2.469126in}}%
\pgfpathlineto{\pgfqpoint{1.684975in}{2.469126in}}%
\pgfpathlineto{\pgfqpoint{1.681840in}{2.469126in}}%
\pgfpathlineto{\pgfqpoint{1.678705in}{2.469126in}}%
\pgfpathlineto{\pgfqpoint{1.677294in}{2.467438in}}%
\pgfpathlineto{\pgfqpoint{1.675570in}{2.465375in}}%
\pgfpathlineto{\pgfqpoint{1.672436in}{2.465375in}}%
\pgfpathlineto{\pgfqpoint{1.669301in}{2.465375in}}%
\pgfpathlineto{\pgfqpoint{1.666166in}{2.465375in}}%
\pgfpathlineto{\pgfqpoint{1.664755in}{2.463686in}}%
\pgfpathlineto{\pgfqpoint{1.663031in}{2.461623in}}%
\pgfpathlineto{\pgfqpoint{1.659897in}{2.461623in}}%
\pgfpathlineto{\pgfqpoint{1.656762in}{2.461623in}}%
\pgfpathlineto{\pgfqpoint{1.653627in}{2.461623in}}%
\pgfpathlineto{\pgfqpoint{1.652216in}{2.459935in}}%
\pgfpathlineto{\pgfqpoint{1.650492in}{2.457871in}}%
\pgfpathlineto{\pgfqpoint{1.647358in}{2.457871in}}%
\pgfpathlineto{\pgfqpoint{1.644223in}{2.457871in}}%
\pgfpathlineto{\pgfqpoint{1.641088in}{2.457871in}}%
\pgfpathlineto{\pgfqpoint{1.639678in}{2.456183in}}%
\pgfpathlineto{\pgfqpoint{1.637953in}{2.454120in}}%
\pgfpathlineto{\pgfqpoint{1.634819in}{2.454120in}}%
\pgfpathlineto{\pgfqpoint{1.631684in}{2.454120in}}%
\pgfpathlineto{\pgfqpoint{1.628549in}{2.454120in}}%
\pgfpathlineto{\pgfqpoint{1.625414in}{2.454120in}}%
\pgfpathlineto{\pgfqpoint{1.624004in}{2.452432in}}%
\pgfpathlineto{\pgfqpoint{1.622280in}{2.450368in}}%
\pgfpathlineto{\pgfqpoint{1.619145in}{2.450368in}}%
\pgfpathlineto{\pgfqpoint{1.616010in}{2.450368in}}%
\pgfpathlineto{\pgfqpoint{1.612875in}{2.450368in}}%
\pgfpathlineto{\pgfqpoint{1.611465in}{2.448680in}}%
\pgfpathlineto{\pgfqpoint{1.609741in}{2.446617in}}%
\pgfpathlineto{\pgfqpoint{1.606606in}{2.446617in}}%
\pgfpathlineto{\pgfqpoint{1.603471in}{2.446617in}}%
\pgfpathlineto{\pgfqpoint{1.600337in}{2.446617in}}%
\pgfpathlineto{\pgfqpoint{1.598926in}{2.444929in}}%
\pgfpathlineto{\pgfqpoint{1.597202in}{2.442865in}}%
\pgfpathlineto{\pgfqpoint{1.594067in}{2.442865in}}%
\pgfpathlineto{\pgfqpoint{1.590932in}{2.442865in}}%
\pgfpathlineto{\pgfqpoint{1.587798in}{2.442865in}}%
\pgfpathlineto{\pgfqpoint{1.586387in}{2.441177in}}%
\pgfpathlineto{\pgfqpoint{1.584663in}{2.439114in}}%
\pgfpathlineto{\pgfqpoint{1.581528in}{2.439114in}}%
\pgfpathlineto{\pgfqpoint{1.578393in}{2.439114in}}%
\pgfpathlineto{\pgfqpoint{1.575259in}{2.439114in}}%
\pgfpathlineto{\pgfqpoint{1.572124in}{2.439114in}}%
\pgfpathlineto{\pgfqpoint{1.570713in}{2.437426in}}%
\pgfpathlineto{\pgfqpoint{1.568989in}{2.435362in}}%
\pgfpathlineto{\pgfqpoint{1.565854in}{2.435362in}}%
\pgfpathlineto{\pgfqpoint{1.562720in}{2.435362in}}%
\pgfpathlineto{\pgfqpoint{1.559585in}{2.435362in}}%
\pgfpathlineto{\pgfqpoint{1.558174in}{2.433674in}}%
\pgfpathlineto{\pgfqpoint{1.556450in}{2.431611in}}%
\pgfpathlineto{\pgfqpoint{1.553315in}{2.431611in}}%
\pgfpathlineto{\pgfqpoint{1.550181in}{2.431611in}}%
\pgfpathlineto{\pgfqpoint{1.547046in}{2.431611in}}%
\pgfpathlineto{\pgfqpoint{1.545635in}{2.429923in}}%
\pgfpathlineto{\pgfqpoint{1.543911in}{2.427859in}}%
\pgfpathlineto{\pgfqpoint{1.540776in}{2.427859in}}%
\pgfpathlineto{\pgfqpoint{1.537642in}{2.427859in}}%
\pgfpathlineto{\pgfqpoint{1.534507in}{2.427859in}}%
\pgfpathlineto{\pgfqpoint{1.533096in}{2.426171in}}%
\pgfpathlineto{\pgfqpoint{1.531372in}{2.424108in}}%
\pgfpathlineto{\pgfqpoint{1.528237in}{2.424108in}}%
\pgfpathlineto{\pgfqpoint{1.525103in}{2.424108in}}%
\pgfpathlineto{\pgfqpoint{1.521968in}{2.424108in}}%
\pgfpathlineto{\pgfqpoint{1.518833in}{2.424108in}}%
\pgfpathlineto{\pgfqpoint{1.517423in}{2.422419in}}%
\pgfpathlineto{\pgfqpoint{1.515698in}{2.420356in}}%
\pgfpathlineto{\pgfqpoint{1.512564in}{2.420356in}}%
\pgfpathlineto{\pgfqpoint{1.509429in}{2.420356in}}%
\pgfpathlineto{\pgfqpoint{1.506294in}{2.420356in}}%
\pgfpathlineto{\pgfqpoint{1.504884in}{2.418668in}}%
\pgfpathlineto{\pgfqpoint{1.503159in}{2.416605in}}%
\pgfpathlineto{\pgfqpoint{1.500025in}{2.416605in}}%
\pgfpathlineto{\pgfqpoint{1.496890in}{2.416605in}}%
\pgfpathlineto{\pgfqpoint{1.493755in}{2.416605in}}%
\pgfpathlineto{\pgfqpoint{1.492345in}{2.414916in}}%
\pgfpathlineto{\pgfqpoint{1.490621in}{2.412853in}}%
\pgfpathlineto{\pgfqpoint{1.487486in}{2.412853in}}%
\pgfpathlineto{\pgfqpoint{1.484351in}{2.412853in}}%
\pgfpathlineto{\pgfqpoint{1.481216in}{2.412853in}}%
\pgfpathlineto{\pgfqpoint{1.479806in}{2.411165in}}%
\pgfpathlineto{\pgfqpoint{1.478082in}{2.409101in}}%
\pgfpathlineto{\pgfqpoint{1.474947in}{2.409101in}}%
\pgfpathlineto{\pgfqpoint{1.471812in}{2.409101in}}%
\pgfpathlineto{\pgfqpoint{1.468677in}{2.409101in}}%
\pgfpathlineto{\pgfqpoint{1.467267in}{2.407413in}}%
\pgfpathlineto{\pgfqpoint{1.465543in}{2.405350in}}%
\pgfpathlineto{\pgfqpoint{1.462408in}{2.405350in}}%
\pgfpathlineto{\pgfqpoint{1.460997in}{2.403662in}}%
\pgfpathlineto{\pgfqpoint{1.460997in}{2.399910in}}%
\pgfpathlineto{\pgfqpoint{1.459273in}{2.397847in}}%
\pgfpathlineto{\pgfqpoint{1.457862in}{2.396159in}}%
\pgfpathlineto{\pgfqpoint{1.457862in}{2.392407in}}%
\pgfpathlineto{\pgfqpoint{1.457862in}{2.388656in}}%
\pgfpathlineto{\pgfqpoint{1.456138in}{2.386592in}}%
\pgfpathlineto{\pgfqpoint{1.454728in}{2.384904in}}%
\pgfpathlineto{\pgfqpoint{1.454728in}{2.381153in}}%
\pgfpathlineto{\pgfqpoint{1.454728in}{2.377401in}}%
\pgfpathlineto{\pgfqpoint{1.454728in}{2.373649in}}%
\pgfpathlineto{\pgfqpoint{1.453004in}{2.371586in}}%
\pgfpathlineto{\pgfqpoint{1.451593in}{2.369898in}}%
\pgfpathlineto{\pgfqpoint{1.451593in}{2.366146in}}%
\pgfpathlineto{\pgfqpoint{1.451593in}{2.362395in}}%
\pgfpathlineto{\pgfqpoint{1.449869in}{2.360332in}}%
\pgfpathlineto{\pgfqpoint{1.448458in}{2.358643in}}%
\pgfpathlineto{\pgfqpoint{1.448458in}{2.354892in}}%
\pgfpathlineto{\pgfqpoint{1.448458in}{2.351140in}}%
\pgfpathlineto{\pgfqpoint{1.448458in}{2.347389in}}%
\pgfpathlineto{\pgfqpoint{1.446734in}{2.345325in}}%
\pgfpathlineto{\pgfqpoint{1.445323in}{2.343637in}}%
\pgfpathlineto{\pgfqpoint{1.445323in}{2.339886in}}%
\pgfpathlineto{\pgfqpoint{1.445323in}{2.336134in}}%
\pgfpathlineto{\pgfqpoint{1.443599in}{2.334071in}}%
\pgfpathlineto{\pgfqpoint{1.442189in}{2.332383in}}%
\pgfpathlineto{\pgfqpoint{1.442189in}{2.328631in}}%
\pgfpathlineto{\pgfqpoint{1.442189in}{2.324880in}}%
\pgfpathlineto{\pgfqpoint{1.442189in}{2.321128in}}%
\pgfpathlineto{\pgfqpoint{1.440465in}{2.319065in}}%
\pgfpathlineto{\pgfqpoint{1.439054in}{2.317376in}}%
\pgfpathlineto{\pgfqpoint{1.439054in}{2.313625in}}%
\pgfpathlineto{\pgfqpoint{1.439054in}{2.309873in}}%
\pgfpathlineto{\pgfqpoint{1.437330in}{2.307810in}}%
\pgfpathlineto{\pgfqpoint{1.435919in}{2.306122in}}%
\pgfpathlineto{\pgfqpoint{1.435919in}{2.302370in}}%
\pgfpathlineto{\pgfqpoint{1.435919in}{2.298619in}}%
\pgfpathlineto{\pgfqpoint{1.435919in}{2.294867in}}%
\pgfpathlineto{\pgfqpoint{1.434195in}{2.292804in}}%
\pgfpathlineto{\pgfqpoint{1.432785in}{2.291116in}}%
\pgfpathlineto{\pgfqpoint{1.432785in}{2.287364in}}%
\pgfpathlineto{\pgfqpoint{1.432785in}{2.283613in}}%
\pgfpathlineto{\pgfqpoint{1.431060in}{2.281549in}}%
\pgfpathlineto{\pgfqpoint{1.429650in}{2.279861in}}%
\pgfpathlineto{\pgfqpoint{1.429650in}{2.276110in}}%
\pgfpathlineto{\pgfqpoint{1.429650in}{2.272358in}}%
\pgfpathlineto{\pgfqpoint{1.429650in}{2.268607in}}%
\pgfpathlineto{\pgfqpoint{1.427926in}{2.266543in}}%
\pgfpathlineto{\pgfqpoint{1.426515in}{2.264855in}}%
\pgfpathlineto{\pgfqpoint{1.426515in}{2.261103in}}%
\pgfpathlineto{\pgfqpoint{1.426515in}{2.257352in}}%
\pgfpathlineto{\pgfqpoint{1.424791in}{2.255289in}}%
\pgfpathlineto{\pgfqpoint{1.423380in}{2.253600in}}%
\pgfpathlineto{\pgfqpoint{1.423380in}{2.249849in}}%
\pgfpathlineto{\pgfqpoint{1.423380in}{2.246097in}}%
\pgfpathlineto{\pgfqpoint{1.421656in}{2.244034in}}%
\pgfpathlineto{\pgfqpoint{1.420246in}{2.242346in}}%
\pgfpathlineto{\pgfqpoint{1.420246in}{2.238594in}}%
\pgfpathlineto{\pgfqpoint{1.420246in}{2.234843in}}%
\pgfpathlineto{\pgfqpoint{1.420246in}{2.231091in}}%
\pgfpathlineto{\pgfqpoint{1.418521in}{2.229028in}}%
\pgfpathlineto{\pgfqpoint{1.417111in}{2.227340in}}%
\pgfpathlineto{\pgfqpoint{1.417111in}{2.223588in}}%
\pgfpathlineto{\pgfqpoint{1.417111in}{2.219837in}}%
\pgfpathlineto{\pgfqpoint{1.415387in}{2.217773in}}%
\pgfpathlineto{\pgfqpoint{1.413976in}{2.216085in}}%
\pgfpathlineto{\pgfqpoint{1.413976in}{2.212334in}}%
\pgfpathlineto{\pgfqpoint{1.413976in}{2.208582in}}%
\pgfpathlineto{\pgfqpoint{1.413976in}{2.204830in}}%
\pgfpathlineto{\pgfqpoint{1.412252in}{2.202767in}}%
\pgfpathlineto{\pgfqpoint{1.410841in}{2.201079in}}%
\pgfpathlineto{\pgfqpoint{1.410841in}{2.197327in}}%
\pgfpathlineto{\pgfqpoint{1.410841in}{2.193576in}}%
\pgfpathlineto{\pgfqpoint{1.409117in}{2.191512in}}%
\pgfpathlineto{\pgfqpoint{1.407707in}{2.189824in}}%
\pgfpathlineto{\pgfqpoint{1.407707in}{2.186073in}}%
\pgfpathlineto{\pgfqpoint{1.407707in}{2.182321in}}%
\pgfpathlineto{\pgfqpoint{1.407707in}{2.178570in}}%
\pgfpathlineto{\pgfqpoint{1.405982in}{2.176506in}}%
\pgfpathlineto{\pgfqpoint{1.404572in}{2.174818in}}%
\pgfpathlineto{\pgfqpoint{1.404572in}{2.171067in}}%
\pgfpathlineto{\pgfqpoint{1.404572in}{2.167315in}}%
\pgfpathlineto{\pgfqpoint{1.402848in}{2.165252in}}%
\pgfpathlineto{\pgfqpoint{1.401437in}{2.163564in}}%
\pgfpathlineto{\pgfqpoint{1.401437in}{2.159812in}}%
\pgfpathlineto{\pgfqpoint{1.401437in}{2.156060in}}%
\pgfpathlineto{\pgfqpoint{1.401437in}{2.152309in}}%
\pgfpathlineto{\pgfqpoint{1.399713in}{2.150246in}}%
\pgfpathlineto{\pgfqpoint{1.398302in}{2.148557in}}%
\pgfpathlineto{\pgfqpoint{1.398302in}{2.144806in}}%
\pgfpathlineto{\pgfqpoint{1.398302in}{2.141054in}}%
\pgfpathlineto{\pgfqpoint{1.396578in}{2.138991in}}%
\pgfpathlineto{\pgfqpoint{1.395168in}{2.137303in}}%
\pgfpathlineto{\pgfqpoint{1.395168in}{2.133551in}}%
\pgfpathlineto{\pgfqpoint{1.395168in}{2.129800in}}%
\pgfpathlineto{\pgfqpoint{1.395168in}{2.126048in}}%
\pgfpathlineto{\pgfqpoint{1.393444in}{2.123985in}}%
\pgfpathlineto{\pgfqpoint{1.392033in}{2.122297in}}%
\pgfpathlineto{\pgfqpoint{1.392033in}{2.118545in}}%
\pgfpathlineto{\pgfqpoint{1.392033in}{2.114794in}}%
\pgfpathlineto{\pgfqpoint{1.390309in}{2.112730in}}%
\pgfpathlineto{\pgfqpoint{1.388898in}{2.111042in}}%
\pgfpathlineto{\pgfqpoint{1.388898in}{2.107291in}}%
\pgfpathlineto{\pgfqpoint{1.388898in}{2.103539in}}%
\pgfpathlineto{\pgfqpoint{1.388898in}{2.099787in}}%
\pgfpathlineto{\pgfqpoint{1.387174in}{2.097724in}}%
\pgfpathlineto{\pgfqpoint{1.385763in}{2.096036in}}%
\pgfpathlineto{\pgfqpoint{1.385763in}{2.092284in}}%
\pgfpathlineto{\pgfqpoint{1.385763in}{2.088533in}}%
\pgfpathlineto{\pgfqpoint{1.384039in}{2.086470in}}%
\pgfpathlineto{\pgfqpoint{1.382629in}{2.084781in}}%
\pgfpathlineto{\pgfqpoint{1.382629in}{2.081030in}}%
\pgfpathlineto{\pgfqpoint{1.382629in}{2.077278in}}%
\pgfpathlineto{\pgfqpoint{1.382629in}{2.073527in}}%
\pgfpathlineto{\pgfqpoint{1.380905in}{2.071463in}}%
\pgfpathlineto{\pgfqpoint{1.379494in}{2.069775in}}%
\pgfpathlineto{\pgfqpoint{1.379494in}{2.066024in}}%
\pgfpathlineto{\pgfqpoint{1.379494in}{2.062272in}}%
\pgfpathlineto{\pgfqpoint{1.377770in}{2.060209in}}%
\pgfpathlineto{\pgfqpoint{1.376359in}{2.058521in}}%
\pgfpathlineto{\pgfqpoint{1.376359in}{2.054769in}}%
\pgfpathlineto{\pgfqpoint{1.376359in}{2.051018in}}%
\pgfpathlineto{\pgfqpoint{1.376359in}{2.047266in}}%
\pgfpathlineto{\pgfqpoint{1.374635in}{2.045203in}}%
\pgfpathlineto{\pgfqpoint{1.373224in}{2.043514in}}%
\pgfpathlineto{\pgfqpoint{1.373224in}{2.039763in}}%
\pgfpathlineto{\pgfqpoint{1.373224in}{2.036011in}}%
\pgfpathlineto{\pgfqpoint{1.371500in}{2.033948in}}%
\pgfpathlineto{\pgfqpoint{1.370090in}{2.032260in}}%
\pgfpathlineto{\pgfqpoint{1.370090in}{2.028508in}}%
\pgfpathlineto{\pgfqpoint{1.370090in}{2.024757in}}%
\pgfpathlineto{\pgfqpoint{1.368366in}{2.022693in}}%
\pgfpathlineto{\pgfqpoint{1.366955in}{2.021005in}}%
\pgfpathlineto{\pgfqpoint{1.366955in}{2.017254in}}%
\pgfpathlineto{\pgfqpoint{1.366955in}{2.013502in}}%
\pgfpathlineto{\pgfqpoint{1.366955in}{2.009751in}}%
\pgfpathlineto{\pgfqpoint{1.365231in}{2.007687in}}%
\pgfpathlineto{\pgfqpoint{1.363820in}{2.005999in}}%
\pgfpathlineto{\pgfqpoint{1.363820in}{2.002248in}}%
\pgfpathlineto{\pgfqpoint{1.363820in}{1.998496in}}%
\pgfpathlineto{\pgfqpoint{1.362096in}{1.996433in}}%
\pgfpathlineto{\pgfqpoint{1.360685in}{1.994745in}}%
\pgfpathlineto{\pgfqpoint{1.360685in}{1.990993in}}%
\pgfpathlineto{\pgfqpoint{1.360685in}{1.987241in}}%
\pgfpathlineto{\pgfqpoint{1.360685in}{1.983490in}}%
\pgfpathlineto{\pgfqpoint{1.358961in}{1.981427in}}%
\pgfpathlineto{\pgfqpoint{1.357551in}{1.979738in}}%
\pgfpathlineto{\pgfqpoint{1.357551in}{1.975987in}}%
\pgfpathlineto{\pgfqpoint{1.357551in}{1.972235in}}%
\pgfpathlineto{\pgfqpoint{1.355827in}{1.970172in}}%
\pgfpathlineto{\pgfqpoint{1.354416in}{1.968484in}}%
\pgfpathlineto{\pgfqpoint{1.354416in}{1.964732in}}%
\pgfpathlineto{\pgfqpoint{1.354416in}{1.960981in}}%
\pgfpathlineto{\pgfqpoint{1.354416in}{1.957229in}}%
\pgfpathlineto{\pgfqpoint{1.352692in}{1.955166in}}%
\pgfpathlineto{\pgfqpoint{1.351281in}{1.953478in}}%
\pgfpathlineto{\pgfqpoint{1.351281in}{1.949726in}}%
\pgfpathlineto{\pgfqpoint{1.351281in}{1.945975in}}%
\pgfpathlineto{\pgfqpoint{1.349557in}{1.943911in}}%
\pgfpathlineto{\pgfqpoint{1.348146in}{1.942223in}}%
\pgfpathlineto{\pgfqpoint{1.348146in}{1.938471in}}%
\pgfpathlineto{\pgfqpoint{1.348146in}{1.934720in}}%
\pgfpathlineto{\pgfqpoint{1.348146in}{1.930968in}}%
\pgfpathlineto{\pgfqpoint{1.346422in}{1.928905in}}%
\pgfpathlineto{\pgfqpoint{1.345012in}{1.927217in}}%
\pgfpathlineto{\pgfqpoint{1.345012in}{1.923465in}}%
\pgfpathlineto{\pgfqpoint{1.345012in}{1.919714in}}%
\pgfpathlineto{\pgfqpoint{1.343288in}{1.917650in}}%
\pgfpathlineto{\pgfqpoint{1.341877in}{1.915962in}}%
\pgfpathlineto{\pgfqpoint{1.341877in}{1.912211in}}%
\pgfpathlineto{\pgfqpoint{1.341877in}{1.908459in}}%
\pgfpathlineto{\pgfqpoint{1.341877in}{1.904708in}}%
\pgfpathlineto{\pgfqpoint{1.340153in}{1.902644in}}%
\pgfpathlineto{\pgfqpoint{1.338742in}{1.900956in}}%
\pgfpathlineto{\pgfqpoint{1.338742in}{1.897205in}}%
\pgfpathlineto{\pgfqpoint{1.338742in}{1.893453in}}%
\pgfpathlineto{\pgfqpoint{1.337018in}{1.891390in}}%
\pgfpathlineto{\pgfqpoint{1.335608in}{1.889702in}}%
\pgfpathlineto{\pgfqpoint{1.335608in}{1.885950in}}%
\pgfpathlineto{\pgfqpoint{1.335608in}{1.882198in}}%
\pgfpathlineto{\pgfqpoint{1.335608in}{1.878447in}}%
\pgfpathlineto{\pgfqpoint{1.333883in}{1.876384in}}%
\pgfpathlineto{\pgfqpoint{1.332473in}{1.874695in}}%
\pgfpathlineto{\pgfqpoint{1.332473in}{1.870944in}}%
\pgfpathlineto{\pgfqpoint{1.332473in}{1.867192in}}%
\pgfpathlineto{\pgfqpoint{1.330749in}{1.865129in}}%
\pgfpathlineto{\pgfqpoint{1.329338in}{1.863441in}}%
\pgfpathlineto{\pgfqpoint{1.329338in}{1.859689in}}%
\pgfpathlineto{\pgfqpoint{1.329338in}{1.855938in}}%
\pgfpathlineto{\pgfqpoint{1.329338in}{1.852186in}}%
\pgfpathlineto{\pgfqpoint{1.327614in}{1.850123in}}%
\pgfpathlineto{\pgfqpoint{1.326203in}{1.848435in}}%
\pgfpathlineto{\pgfqpoint{1.326203in}{1.844683in}}%
\pgfpathlineto{\pgfqpoint{1.326203in}{1.840932in}}%
\pgfpathlineto{\pgfqpoint{1.324479in}{1.838868in}}%
\pgfpathlineto{\pgfqpoint{1.323069in}{1.837180in}}%
\pgfpathlineto{\pgfqpoint{1.323069in}{1.833429in}}%
\pgfpathlineto{\pgfqpoint{1.323069in}{1.829677in}}%
\pgfpathlineto{\pgfqpoint{1.323069in}{1.825925in}}%
\pgfpathlineto{\pgfqpoint{1.321344in}{1.823862in}}%
\pgfpathlineto{\pgfqpoint{1.319934in}{1.822174in}}%
\pgfpathlineto{\pgfqpoint{1.319934in}{1.818422in}}%
\pgfpathlineto{\pgfqpoint{1.319934in}{1.814671in}}%
\pgfpathlineto{\pgfqpoint{1.318210in}{1.812608in}}%
\pgfpathlineto{\pgfqpoint{1.316799in}{1.810919in}}%
\pgfpathlineto{\pgfqpoint{1.316799in}{1.807168in}}%
\pgfpathlineto{\pgfqpoint{1.316799in}{1.803416in}}%
\pgfpathlineto{\pgfqpoint{1.315075in}{1.801353in}}%
\pgfpathlineto{\pgfqpoint{1.313664in}{1.799665in}}%
\pgfpathlineto{\pgfqpoint{1.313664in}{1.795913in}}%
\pgfpathlineto{\pgfqpoint{1.313664in}{1.792162in}}%
\pgfpathlineto{\pgfqpoint{1.313664in}{1.788410in}}%
\pgfpathlineto{\pgfqpoint{1.311940in}{1.786347in}}%
\pgfpathlineto{\pgfqpoint{1.310530in}{1.784659in}}%
\pgfpathlineto{\pgfqpoint{1.310530in}{1.780907in}}%
\pgfpathlineto{\pgfqpoint{1.310530in}{1.777155in}}%
\pgfpathlineto{\pgfqpoint{1.308805in}{1.775092in}}%
\pgfpathlineto{\pgfqpoint{1.307395in}{1.773404in}}%
\pgfpathlineto{\pgfqpoint{1.307395in}{1.769652in}}%
\pgfpathlineto{\pgfqpoint{1.307395in}{1.765901in}}%
\pgfpathlineto{\pgfqpoint{1.307395in}{1.762149in}}%
\pgfpathlineto{\pgfqpoint{1.305671in}{1.760086in}}%
\pgfpathlineto{\pgfqpoint{1.304260in}{1.758398in}}%
\pgfpathlineto{\pgfqpoint{1.304260in}{1.754646in}}%
\pgfpathlineto{\pgfqpoint{1.304260in}{1.750895in}}%
\pgfpathlineto{\pgfqpoint{1.302536in}{1.748831in}}%
\pgfpathlineto{\pgfqpoint{1.301125in}{1.747143in}}%
\pgfpathlineto{\pgfqpoint{1.301125in}{1.743392in}}%
\pgfpathlineto{\pgfqpoint{1.301125in}{1.739640in}}%
\pgfpathlineto{\pgfqpoint{1.301125in}{1.735889in}}%
\pgfpathlineto{\pgfqpoint{1.299401in}{1.733825in}}%
\pgfpathlineto{\pgfqpoint{1.297991in}{1.732137in}}%
\pgfpathlineto{\pgfqpoint{1.297991in}{1.728386in}}%
\pgfpathlineto{\pgfqpoint{1.297991in}{1.724634in}}%
\pgfpathlineto{\pgfqpoint{1.296266in}{1.722571in}}%
\pgfpathlineto{\pgfqpoint{1.294856in}{1.720882in}}%
\pgfpathlineto{\pgfqpoint{1.294856in}{1.717131in}}%
\pgfpathlineto{\pgfqpoint{1.294856in}{1.713379in}}%
\pgfpathlineto{\pgfqpoint{1.294856in}{1.709628in}}%
\pgfpathlineto{\pgfqpoint{1.293132in}{1.707565in}}%
\pgfpathlineto{\pgfqpoint{1.291721in}{1.705876in}}%
\pgfpathlineto{\pgfqpoint{1.291721in}{1.702125in}}%
\pgfpathlineto{\pgfqpoint{1.293132in}{1.700437in}}%
\pgfpathlineto{\pgfqpoint{1.294856in}{1.698373in}}%
\pgfpathlineto{\pgfqpoint{1.296266in}{1.696685in}}%
\pgfpathlineto{\pgfqpoint{1.297991in}{1.694622in}}%
\pgfpathlineto{\pgfqpoint{1.299401in}{1.692934in}}%
\pgfpathlineto{\pgfqpoint{1.302536in}{1.692934in}}%
\pgfpathlineto{\pgfqpoint{1.304260in}{1.690870in}}%
\pgfpathlineto{\pgfqpoint{1.305671in}{1.689182in}}%
\pgfpathlineto{\pgfqpoint{1.307395in}{1.687119in}}%
\pgfpathlineto{\pgfqpoint{1.308805in}{1.685430in}}%
\pgfpathlineto{\pgfqpoint{1.310530in}{1.683367in}}%
\pgfpathlineto{\pgfqpoint{1.311940in}{1.681679in}}%
\pgfpathlineto{\pgfqpoint{1.315075in}{1.681679in}}%
\pgfpathlineto{\pgfqpoint{1.316799in}{1.679616in}}%
\pgfpathlineto{\pgfqpoint{1.318210in}{1.677927in}}%
\pgfpathlineto{\pgfqpoint{1.319934in}{1.675864in}}%
\pgfpathlineto{\pgfqpoint{1.321344in}{1.674176in}}%
\pgfpathlineto{\pgfqpoint{1.323069in}{1.672113in}}%
\pgfpathlineto{\pgfqpoint{1.324479in}{1.670424in}}%
\pgfpathlineto{\pgfqpoint{1.327614in}{1.670424in}}%
\pgfpathlineto{\pgfqpoint{1.329338in}{1.668361in}}%
\pgfpathlineto{\pgfqpoint{1.330749in}{1.666673in}}%
\pgfpathlineto{\pgfqpoint{1.332473in}{1.664609in}}%
\pgfpathlineto{\pgfqpoint{1.333883in}{1.662921in}}%
\pgfpathlineto{\pgfqpoint{1.335608in}{1.660858in}}%
\pgfpathlineto{\pgfqpoint{1.337018in}{1.659170in}}%
\pgfpathlineto{\pgfqpoint{1.338742in}{1.657106in}}%
\pgfpathlineto{\pgfqpoint{1.340153in}{1.655418in}}%
\pgfpathlineto{\pgfqpoint{1.343288in}{1.655418in}}%
\pgfpathlineto{\pgfqpoint{1.345012in}{1.653355in}}%
\pgfpathlineto{\pgfqpoint{1.346422in}{1.651667in}}%
\pgfpathlineto{\pgfqpoint{1.348146in}{1.649603in}}%
\pgfpathlineto{\pgfqpoint{1.349557in}{1.647915in}}%
\pgfpathlineto{\pgfqpoint{1.351281in}{1.645852in}}%
\pgfpathlineto{\pgfqpoint{1.352692in}{1.644164in}}%
\pgfpathlineto{\pgfqpoint{1.355827in}{1.644164in}}%
\pgfpathlineto{\pgfqpoint{1.357551in}{1.642100in}}%
\pgfpathlineto{\pgfqpoint{1.358961in}{1.640412in}}%
\pgfpathlineto{\pgfqpoint{1.360685in}{1.638349in}}%
\pgfpathlineto{\pgfqpoint{1.362096in}{1.636661in}}%
\pgfpathlineto{\pgfqpoint{1.363820in}{1.634597in}}%
\pgfpathlineto{\pgfqpoint{1.365231in}{1.632909in}}%
\pgfpathlineto{\pgfqpoint{1.368366in}{1.632909in}}%
\pgfpathlineto{\pgfqpoint{1.370090in}{1.630846in}}%
\pgfpathlineto{\pgfqpoint{1.371500in}{1.629157in}}%
\pgfpathlineto{\pgfqpoint{1.373224in}{1.627094in}}%
\pgfpathlineto{\pgfqpoint{1.374635in}{1.625406in}}%
\pgfpathlineto{\pgfqpoint{1.376359in}{1.623343in}}%
\pgfpathlineto{\pgfqpoint{1.377770in}{1.621654in}}%
\pgfpathlineto{\pgfqpoint{1.380905in}{1.621654in}}%
\pgfpathlineto{\pgfqpoint{1.382629in}{1.619591in}}%
\pgfpathlineto{\pgfqpoint{1.384039in}{1.617903in}}%
\pgfpathlineto{\pgfqpoint{1.385763in}{1.615840in}}%
\pgfpathlineto{\pgfqpoint{1.387174in}{1.614151in}}%
\pgfpathlineto{\pgfqpoint{1.388898in}{1.612088in}}%
\pgfpathlineto{\pgfqpoint{1.390309in}{1.610400in}}%
\pgfpathlineto{\pgfqpoint{1.393444in}{1.610400in}}%
\pgfpathlineto{\pgfqpoint{1.395168in}{1.608336in}}%
\pgfpathlineto{\pgfqpoint{1.396578in}{1.606648in}}%
\pgfpathlineto{\pgfqpoint{1.398302in}{1.604585in}}%
\pgfpathlineto{\pgfqpoint{1.399713in}{1.602897in}}%
\pgfpathlineto{\pgfqpoint{1.401437in}{1.600833in}}%
\pgfpathlineto{\pgfqpoint{1.402848in}{1.599145in}}%
\pgfpathlineto{\pgfqpoint{1.405982in}{1.599145in}}%
\pgfpathlineto{\pgfqpoint{1.407707in}{1.597082in}}%
\pgfpathlineto{\pgfqpoint{1.409117in}{1.595394in}}%
\pgfpathlineto{\pgfqpoint{1.410841in}{1.593330in}}%
\pgfpathlineto{\pgfqpoint{1.412252in}{1.591642in}}%
\pgfpathlineto{\pgfqpoint{1.413976in}{1.589579in}}%
\pgfpathlineto{\pgfqpoint{1.415387in}{1.587891in}}%
\pgfpathlineto{\pgfqpoint{1.417111in}{1.585827in}}%
\pgfpathlineto{\pgfqpoint{1.418521in}{1.584139in}}%
\pgfpathlineto{\pgfqpoint{1.421656in}{1.584139in}}%
\pgfpathlineto{\pgfqpoint{1.423380in}{1.582076in}}%
\pgfpathlineto{\pgfqpoint{1.424791in}{1.580388in}}%
\pgfpathlineto{\pgfqpoint{1.426515in}{1.578324in}}%
\pgfpathlineto{\pgfqpoint{1.427926in}{1.576636in}}%
\pgfpathlineto{\pgfqpoint{1.429650in}{1.574573in}}%
\pgfpathlineto{\pgfqpoint{1.431060in}{1.572884in}}%
\pgfpathlineto{\pgfqpoint{1.434195in}{1.572884in}}%
\pgfpathlineto{\pgfqpoint{1.435919in}{1.570821in}}%
\pgfpathlineto{\pgfqpoint{1.437330in}{1.569133in}}%
\pgfpathlineto{\pgfqpoint{1.439054in}{1.567070in}}%
\pgfpathlineto{\pgfqpoint{1.440465in}{1.565381in}}%
\pgfpathlineto{\pgfqpoint{1.442189in}{1.563318in}}%
\pgfpathlineto{\pgfqpoint{1.443599in}{1.561630in}}%
\pgfpathlineto{\pgfqpoint{1.446734in}{1.561630in}}%
\pgfpathlineto{\pgfqpoint{1.448458in}{1.559566in}}%
\pgfpathlineto{\pgfqpoint{1.449869in}{1.557878in}}%
\pgfpathlineto{\pgfqpoint{1.451593in}{1.555815in}}%
\pgfpathlineto{\pgfqpoint{1.453004in}{1.554127in}}%
\pgfpathlineto{\pgfqpoint{1.454728in}{1.552063in}}%
\pgfpathlineto{\pgfqpoint{1.456138in}{1.550375in}}%
\pgfpathlineto{\pgfqpoint{1.459273in}{1.550375in}}%
\pgfpathlineto{\pgfqpoint{1.460997in}{1.548312in}}%
\pgfpathlineto{\pgfqpoint{1.462408in}{1.546624in}}%
\pgfpathlineto{\pgfqpoint{1.464132in}{1.544560in}}%
\pgfpathlineto{\pgfqpoint{1.465543in}{1.542872in}}%
\pgfpathlineto{\pgfqpoint{1.467267in}{1.540809in}}%
\pgfpathlineto{\pgfqpoint{1.468677in}{1.539121in}}%
\pgfpathlineto{\pgfqpoint{1.471812in}{1.539121in}}%
\pgfpathlineto{\pgfqpoint{1.473536in}{1.537057in}}%
\pgfpathlineto{\pgfqpoint{1.474947in}{1.535369in}}%
\pgfpathlineto{\pgfqpoint{1.476671in}{1.533306in}}%
\pgfpathlineto{\pgfqpoint{1.478082in}{1.531618in}}%
\pgfpathlineto{\pgfqpoint{1.479806in}{1.529554in}}%
\pgfpathlineto{\pgfqpoint{1.481216in}{1.527866in}}%
\pgfpathlineto{\pgfqpoint{1.484351in}{1.527866in}}%
\pgfpathlineto{\pgfqpoint{1.486075in}{1.525803in}}%
\pgfpathlineto{\pgfqpoint{1.487486in}{1.524114in}}%
\pgfpathlineto{\pgfqpoint{1.489210in}{1.522051in}}%
\pgfpathlineto{\pgfqpoint{1.490621in}{1.520363in}}%
\pgfpathlineto{\pgfqpoint{1.492345in}{1.518300in}}%
\pgfpathlineto{\pgfqpoint{1.493755in}{1.516611in}}%
\pgfpathlineto{\pgfqpoint{1.495479in}{1.514548in}}%
\pgfpathlineto{\pgfqpoint{1.496890in}{1.512860in}}%
\pgfpathlineto{\pgfqpoint{1.500025in}{1.512860in}}%
\pgfpathlineto{\pgfqpoint{1.501749in}{1.510797in}}%
\pgfpathlineto{\pgfqpoint{1.503159in}{1.509108in}}%
\pgfpathlineto{\pgfqpoint{1.504884in}{1.507045in}}%
\pgfpathlineto{\pgfqpoint{1.506294in}{1.505357in}}%
\pgfpathlineto{\pgfqpoint{1.508018in}{1.503293in}}%
\pgfpathlineto{\pgfqpoint{1.509429in}{1.501605in}}%
\pgfpathlineto{\pgfqpoint{1.512564in}{1.501605in}}%
\pgfpathlineto{\pgfqpoint{1.514288in}{1.499542in}}%
\pgfpathlineto{\pgfqpoint{1.515698in}{1.497854in}}%
\pgfpathlineto{\pgfqpoint{1.517423in}{1.495790in}}%
\pgfpathlineto{\pgfqpoint{1.518833in}{1.494102in}}%
\pgfpathlineto{\pgfqpoint{1.520557in}{1.492039in}}%
\pgfpathlineto{\pgfqpoint{1.521968in}{1.490351in}}%
\pgfpathlineto{\pgfqpoint{1.525103in}{1.490351in}}%
\pgfpathlineto{\pgfqpoint{1.526827in}{1.488287in}}%
\pgfpathlineto{\pgfqpoint{1.528237in}{1.486599in}}%
\pgfpathlineto{\pgfqpoint{1.529962in}{1.484536in}}%
\pgfpathlineto{\pgfqpoint{1.531372in}{1.482848in}}%
\pgfpathlineto{\pgfqpoint{1.533096in}{1.480784in}}%
\pgfpathlineto{\pgfqpoint{1.534507in}{1.479096in}}%
\pgfpathlineto{\pgfqpoint{1.537642in}{1.479096in}}%
\pgfpathlineto{\pgfqpoint{1.539366in}{1.477033in}}%
\pgfpathlineto{\pgfqpoint{1.540776in}{1.475345in}}%
\pgfpathlineto{\pgfqpoint{1.542501in}{1.473281in}}%
\pgfpathlineto{\pgfqpoint{1.543911in}{1.471593in}}%
\pgfpathlineto{\pgfqpoint{1.545635in}{1.469530in}}%
\pgfpathlineto{\pgfqpoint{1.547046in}{1.467841in}}%
\pgfpathlineto{\pgfqpoint{1.550181in}{1.467841in}}%
\pgfpathlineto{\pgfqpoint{1.551905in}{1.465778in}}%
\pgfpathlineto{\pgfqpoint{1.553315in}{1.464090in}}%
\pgfpathlineto{\pgfqpoint{1.555039in}{1.462027in}}%
\pgfpathlineto{\pgfqpoint{1.556450in}{1.460338in}}%
\pgfpathlineto{\pgfqpoint{1.558174in}{1.458275in}}%
\pgfpathlineto{\pgfqpoint{1.559585in}{1.456587in}}%
\pgfpathlineto{\pgfqpoint{1.561309in}{1.454524in}}%
\pgfpathlineto{\pgfqpoint{1.562720in}{1.452835in}}%
\pgfpathlineto{\pgfqpoint{1.565854in}{1.452835in}}%
\pgfpathlineto{\pgfqpoint{1.567578in}{1.450772in}}%
\pgfpathlineto{\pgfqpoint{1.568989in}{1.449084in}}%
\pgfpathlineto{\pgfqpoint{1.570713in}{1.447020in}}%
\pgfpathlineto{\pgfqpoint{1.572124in}{1.445332in}}%
\pgfpathlineto{\pgfqpoint{1.573848in}{1.443269in}}%
\pgfpathlineto{\pgfqpoint{1.575259in}{1.441581in}}%
\pgfpathlineto{\pgfqpoint{1.578393in}{1.441581in}}%
\pgfpathlineto{\pgfqpoint{1.580117in}{1.439517in}}%
\pgfpathlineto{\pgfqpoint{1.581528in}{1.437829in}}%
\pgfpathlineto{\pgfqpoint{1.583252in}{1.435766in}}%
\pgfpathlineto{\pgfqpoint{1.584663in}{1.434078in}}%
\pgfpathlineto{\pgfqpoint{1.586387in}{1.432014in}}%
\pgfpathlineto{\pgfqpoint{1.587798in}{1.430326in}}%
\pgfpathlineto{\pgfqpoint{1.590932in}{1.430326in}}%
\pgfpathlineto{\pgfqpoint{1.592656in}{1.428263in}}%
\pgfpathlineto{\pgfqpoint{1.594067in}{1.426575in}}%
\pgfpathlineto{\pgfqpoint{1.595791in}{1.424511in}}%
\pgfpathlineto{\pgfqpoint{1.597202in}{1.422823in}}%
\pgfpathlineto{\pgfqpoint{1.598926in}{1.420760in}}%
\pgfpathlineto{\pgfqpoint{1.600337in}{1.419072in}}%
\pgfpathlineto{\pgfqpoint{1.603471in}{1.419072in}}%
\pgfpathlineto{\pgfqpoint{1.605195in}{1.417008in}}%
\pgfpathlineto{\pgfqpoint{1.606606in}{1.415320in}}%
\pgfpathlineto{\pgfqpoint{1.608330in}{1.413257in}}%
\pgfpathlineto{\pgfqpoint{1.609741in}{1.411568in}}%
\pgfpathlineto{\pgfqpoint{1.611465in}{1.409505in}}%
\pgfpathlineto{\pgfqpoint{1.612875in}{1.407817in}}%
\pgfpathlineto{\pgfqpoint{1.616010in}{1.407817in}}%
\pgfpathlineto{\pgfqpoint{1.617734in}{1.405754in}}%
\pgfpathlineto{\pgfqpoint{1.619145in}{1.404065in}}%
\pgfpathlineto{\pgfqpoint{1.620869in}{1.402002in}}%
\pgfpathlineto{\pgfqpoint{1.622280in}{1.400314in}}%
\pgfpathlineto{\pgfqpoint{1.624004in}{1.398251in}}%
\pgfpathlineto{\pgfqpoint{1.625414in}{1.396562in}}%
\pgfpathlineto{\pgfqpoint{1.628549in}{1.396562in}}%
\pgfpathlineto{\pgfqpoint{1.630273in}{1.394499in}}%
\pgfpathlineto{\pgfqpoint{1.631684in}{1.392811in}}%
\pgfpathlineto{\pgfqpoint{1.633408in}{1.390747in}}%
\pgfpathlineto{\pgfqpoint{1.634819in}{1.389059in}}%
\pgfpathlineto{\pgfqpoint{1.636543in}{1.386996in}}%
\pgfpathlineto{\pgfqpoint{1.637953in}{1.385308in}}%
\pgfpathlineto{\pgfqpoint{1.639678in}{1.383244in}}%
\pgfpathlineto{\pgfqpoint{1.641088in}{1.381556in}}%
\pgfpathlineto{\pgfqpoint{1.644223in}{1.381556in}}%
\pgfpathlineto{\pgfqpoint{1.645947in}{1.379493in}}%
\pgfpathlineto{\pgfqpoint{1.647358in}{1.377805in}}%
\pgfpathlineto{\pgfqpoint{1.649082in}{1.375741in}}%
\pgfpathlineto{\pgfqpoint{1.650492in}{1.374053in}}%
\pgfpathlineto{\pgfqpoint{1.652216in}{1.371990in}}%
\pgfpathlineto{\pgfqpoint{1.653627in}{1.370302in}}%
\pgfpathlineto{\pgfqpoint{1.656762in}{1.370302in}}%
\pgfpathlineto{\pgfqpoint{1.658486in}{1.368238in}}%
\pgfpathlineto{\pgfqpoint{1.659897in}{1.366550in}}%
\pgfpathlineto{\pgfqpoint{1.661621in}{1.364487in}}%
\pgfpathlineto{\pgfqpoint{1.663031in}{1.362799in}}%
\pgfpathlineto{\pgfqpoint{1.664755in}{1.360735in}}%
\pgfpathlineto{\pgfqpoint{1.666166in}{1.359047in}}%
\pgfpathlineto{\pgfqpoint{1.669301in}{1.359047in}}%
\pgfpathlineto{\pgfqpoint{1.671025in}{1.356984in}}%
\pgfpathlineto{\pgfqpoint{1.672436in}{1.355295in}}%
\pgfpathlineto{\pgfqpoint{1.674160in}{1.353232in}}%
\pgfpathlineto{\pgfqpoint{1.675570in}{1.351544in}}%
\pgfpathlineto{\pgfqpoint{1.677294in}{1.349481in}}%
\pgfpathlineto{\pgfqpoint{1.678705in}{1.347792in}}%
\pgfpathlineto{\pgfqpoint{1.681840in}{1.347792in}}%
\pgfpathlineto{\pgfqpoint{1.683564in}{1.345729in}}%
\pgfpathlineto{\pgfqpoint{1.684975in}{1.344041in}}%
\pgfpathlineto{\pgfqpoint{1.686699in}{1.341977in}}%
\pgfpathlineto{\pgfqpoint{1.688109in}{1.340289in}}%
\pgfpathlineto{\pgfqpoint{1.689833in}{1.338226in}}%
\pgfpathlineto{\pgfqpoint{1.691244in}{1.336538in}}%
\pgfpathlineto{\pgfqpoint{1.694379in}{1.336538in}}%
\pgfpathlineto{\pgfqpoint{1.696103in}{1.334474in}}%
\pgfpathlineto{\pgfqpoint{1.697514in}{1.332786in}}%
\pgfpathlineto{\pgfqpoint{1.699238in}{1.330723in}}%
\pgfpathlineto{\pgfqpoint{1.700648in}{1.329035in}}%
\pgfpathlineto{\pgfqpoint{1.702372in}{1.326971in}}%
\pgfpathlineto{\pgfqpoint{1.703783in}{1.325283in}}%
\pgfpathlineto{\pgfqpoint{1.706918in}{1.325283in}}%
\pgfpathlineto{\pgfqpoint{1.708642in}{1.323220in}}%
\pgfpathlineto{\pgfqpoint{1.710052in}{1.321532in}}%
\pgfpathlineto{\pgfqpoint{1.711777in}{1.319468in}}%
\pgfpathlineto{\pgfqpoint{1.713187in}{1.317780in}}%
\pgfpathlineto{\pgfqpoint{1.714911in}{1.315717in}}%
\pgfpathlineto{\pgfqpoint{1.716322in}{1.314029in}}%
\pgfpathlineto{\pgfqpoint{1.718046in}{1.311965in}}%
\pgfpathlineto{\pgfqpoint{1.719457in}{1.310277in}}%
\pgfpathlineto{\pgfqpoint{1.722591in}{1.310277in}}%
\pgfpathlineto{\pgfqpoint{1.724316in}{1.308214in}}%
\pgfpathlineto{\pgfqpoint{1.725726in}{1.306525in}}%
\pgfpathlineto{\pgfqpoint{1.727450in}{1.304462in}}%
\pgfpathlineto{\pgfqpoint{1.728861in}{1.302774in}}%
\pgfpathlineto{\pgfqpoint{1.730585in}{1.300711in}}%
\pgfpathlineto{\pgfqpoint{1.731996in}{1.299022in}}%
\pgfpathlineto{\pgfqpoint{1.735130in}{1.299022in}}%
\pgfpathlineto{\pgfqpoint{1.736855in}{1.296959in}}%
\pgfpathlineto{\pgfqpoint{1.738265in}{1.295271in}}%
\pgfpathlineto{\pgfqpoint{1.739989in}{1.293208in}}%
\pgfpathlineto{\pgfqpoint{1.741400in}{1.291519in}}%
\pgfpathlineto{\pgfqpoint{1.743124in}{1.289456in}}%
\pgfpathlineto{\pgfqpoint{1.744535in}{1.287768in}}%
\pgfpathlineto{\pgfqpoint{1.747669in}{1.287768in}}%
\pgfpathlineto{\pgfqpoint{1.749393in}{1.285704in}}%
\pgfpathlineto{\pgfqpoint{1.750804in}{1.284016in}}%
\pgfpathlineto{\pgfqpoint{1.752528in}{1.281953in}}%
\pgfpathlineto{\pgfqpoint{1.753939in}{1.280265in}}%
\pgfpathlineto{\pgfqpoint{1.755663in}{1.278201in}}%
\pgfpathlineto{\pgfqpoint{1.757074in}{1.276513in}}%
\pgfpathlineto{\pgfqpoint{1.760208in}{1.276513in}}%
\pgfpathlineto{\pgfqpoint{1.761932in}{1.274450in}}%
\pgfpathlineto{\pgfqpoint{1.763343in}{1.272762in}}%
\pgfpathlineto{\pgfqpoint{1.765067in}{1.270698in}}%
\pgfpathlineto{\pgfqpoint{1.766478in}{1.269010in}}%
\pgfpathlineto{\pgfqpoint{1.768202in}{1.266947in}}%
\pgfpathlineto{\pgfqpoint{1.769613in}{1.265259in}}%
\pgfpathlineto{\pgfqpoint{1.772747in}{1.265259in}}%
\pgfpathlineto{\pgfqpoint{1.774471in}{1.263195in}}%
\pgfpathlineto{\pgfqpoint{1.775882in}{1.261507in}}%
\pgfpathlineto{\pgfqpoint{1.777606in}{1.259444in}}%
\pgfpathlineto{\pgfqpoint{1.779017in}{1.257756in}}%
\pgfpathlineto{\pgfqpoint{1.780741in}{1.255692in}}%
\pgfpathlineto{\pgfqpoint{1.782152in}{1.254004in}}%
\pgfpathlineto{\pgfqpoint{1.785286in}{1.254004in}}%
\pgfpathlineto{\pgfqpoint{1.787010in}{1.251941in}}%
\pgfpathlineto{\pgfqpoint{1.788421in}{1.250252in}}%
\pgfpathlineto{\pgfqpoint{1.790145in}{1.248189in}}%
\pgfpathlineto{\pgfqpoint{1.791556in}{1.246501in}}%
\pgfpathlineto{\pgfqpoint{1.793280in}{1.244438in}}%
\pgfpathlineto{\pgfqpoint{1.794691in}{1.242749in}}%
\pgfpathlineto{\pgfqpoint{1.796415in}{1.240686in}}%
\pgfpathlineto{\pgfqpoint{1.797825in}{1.238998in}}%
\pgfpathlineto{\pgfqpoint{1.800960in}{1.238998in}}%
\pgfpathlineto{\pgfqpoint{1.802684in}{1.236935in}}%
\pgfpathlineto{\pgfqpoint{1.804095in}{1.235246in}}%
\pgfpathlineto{\pgfqpoint{1.805819in}{1.233183in}}%
\pgfpathlineto{\pgfqpoint{1.807229in}{1.231495in}}%
\pgfpathlineto{\pgfqpoint{1.808954in}{1.229431in}}%
\pgfpathlineto{\pgfqpoint{1.810364in}{1.227743in}}%
\pgfpathlineto{\pgfqpoint{1.813499in}{1.227743in}}%
\pgfpathlineto{\pgfqpoint{1.815223in}{1.225680in}}%
\pgfpathlineto{\pgfqpoint{1.816634in}{1.223992in}}%
\pgfpathlineto{\pgfqpoint{1.818358in}{1.221928in}}%
\pgfpathlineto{\pgfqpoint{1.819768in}{1.220240in}}%
\pgfpathlineto{\pgfqpoint{1.821493in}{1.218177in}}%
\pgfpathlineto{\pgfqpoint{1.822903in}{1.216489in}}%
\pgfpathlineto{\pgfqpoint{1.826038in}{1.216489in}}%
\pgfpathlineto{\pgfqpoint{1.827762in}{1.214425in}}%
\pgfpathlineto{\pgfqpoint{1.829173in}{1.212737in}}%
\pgfpathlineto{\pgfqpoint{1.830897in}{1.210674in}}%
\pgfpathlineto{\pgfqpoint{1.832307in}{1.208986in}}%
\pgfpathlineto{\pgfqpoint{1.834032in}{1.206922in}}%
\pgfpathlineto{\pgfqpoint{1.835442in}{1.205234in}}%
\pgfpathlineto{\pgfqpoint{1.838577in}{1.205234in}}%
\pgfpathlineto{\pgfqpoint{1.840301in}{1.203171in}}%
\pgfpathlineto{\pgfqpoint{1.841712in}{1.201483in}}%
\pgfpathlineto{\pgfqpoint{1.843436in}{1.199419in}}%
\pgfpathlineto{\pgfqpoint{1.844846in}{1.197731in}}%
\pgfpathlineto{\pgfqpoint{1.846571in}{1.195668in}}%
\pgfpathlineto{\pgfqpoint{1.847981in}{1.193979in}}%
\pgfpathlineto{\pgfqpoint{1.851116in}{1.193979in}}%
\pgfpathlineto{\pgfqpoint{1.852840in}{1.191916in}}%
\pgfpathlineto{\pgfqpoint{1.854251in}{1.190228in}}%
\pgfpathlineto{\pgfqpoint{1.855975in}{1.188165in}}%
\pgfpathlineto{\pgfqpoint{1.857385in}{1.186476in}}%
\pgfpathlineto{\pgfqpoint{1.859109in}{1.184413in}}%
\pgfpathlineto{\pgfqpoint{1.860520in}{1.182725in}}%
\pgfpathlineto{\pgfqpoint{1.862244in}{1.180662in}}%
\pgfpathlineto{\pgfqpoint{1.863655in}{1.178973in}}%
\pgfpathlineto{\pgfqpoint{1.866790in}{1.178973in}}%
\pgfpathlineto{\pgfqpoint{1.868514in}{1.176910in}}%
\pgfpathlineto{\pgfqpoint{1.869924in}{1.175222in}}%
\pgfpathlineto{\pgfqpoint{1.871648in}{1.173158in}}%
\pgfpathlineto{\pgfqpoint{1.873059in}{1.171470in}}%
\pgfpathlineto{\pgfqpoint{1.874783in}{1.169407in}}%
\pgfpathlineto{\pgfqpoint{1.876194in}{1.167719in}}%
\pgfpathlineto{\pgfqpoint{1.879329in}{1.167719in}}%
\pgfpathlineto{\pgfqpoint{1.881053in}{1.165655in}}%
\pgfpathlineto{\pgfqpoint{1.882463in}{1.163967in}}%
\pgfpathlineto{\pgfqpoint{1.884187in}{1.161904in}}%
\pgfpathlineto{\pgfqpoint{1.885598in}{1.160216in}}%
\pgfpathlineto{\pgfqpoint{1.887322in}{1.158152in}}%
\pgfpathlineto{\pgfqpoint{1.888733in}{1.156464in}}%
\pgfpathlineto{\pgfqpoint{1.891868in}{1.156464in}}%
\pgfpathlineto{\pgfqpoint{1.893592in}{1.154401in}}%
\pgfpathlineto{\pgfqpoint{1.895002in}{1.152713in}}%
\pgfpathlineto{\pgfqpoint{1.896726in}{1.150649in}}%
\pgfpathlineto{\pgfqpoint{1.898137in}{1.148961in}}%
\pgfpathlineto{\pgfqpoint{1.899861in}{1.146898in}}%
\pgfpathlineto{\pgfqpoint{1.901272in}{1.145210in}}%
\pgfpathlineto{\pgfqpoint{1.904407in}{1.145210in}}%
\pgfpathlineto{\pgfqpoint{1.906131in}{1.143146in}}%
\pgfpathlineto{\pgfqpoint{1.907541in}{1.141458in}}%
\pgfpathlineto{\pgfqpoint{1.909265in}{1.139395in}}%
\pgfpathlineto{\pgfqpoint{1.910676in}{1.137706in}}%
\pgfpathlineto{\pgfqpoint{1.912400in}{1.135643in}}%
\pgfpathlineto{\pgfqpoint{1.913811in}{1.133955in}}%
\pgfpathlineto{\pgfqpoint{1.916945in}{1.133955in}}%
\pgfpathlineto{\pgfqpoint{1.918670in}{1.131892in}}%
\pgfpathlineto{\pgfqpoint{1.920080in}{1.130203in}}%
\pgfpathlineto{\pgfqpoint{1.921804in}{1.128140in}}%
\pgfpathlineto{\pgfqpoint{1.923215in}{1.126452in}}%
\pgfpathlineto{\pgfqpoint{1.924939in}{1.124388in}}%
\pgfpathlineto{\pgfqpoint{1.926350in}{1.122700in}}%
\pgfpathlineto{\pgfqpoint{1.929484in}{1.122700in}}%
\pgfpathlineto{\pgfqpoint{1.931209in}{1.120637in}}%
\pgfpathlineto{\pgfqpoint{1.932619in}{1.118949in}}%
\pgfpathlineto{\pgfqpoint{1.934343in}{1.116885in}}%
\pgfpathlineto{\pgfqpoint{1.935754in}{1.115197in}}%
\pgfpathlineto{\pgfqpoint{1.937478in}{1.113134in}}%
\pgfpathlineto{\pgfqpoint{1.938889in}{1.111446in}}%
\pgfpathlineto{\pgfqpoint{1.940613in}{1.109382in}}%
\pgfpathlineto{\pgfqpoint{1.942023in}{1.107694in}}%
\pgfpathlineto{\pgfqpoint{1.945158in}{1.107694in}}%
\pgfpathlineto{\pgfqpoint{1.946882in}{1.105631in}}%
\pgfpathlineto{\pgfqpoint{1.948293in}{1.103943in}}%
\pgfpathlineto{\pgfqpoint{1.950017in}{1.101879in}}%
\pgfpathlineto{\pgfqpoint{1.951428in}{1.100191in}}%
\pgfpathlineto{\pgfqpoint{1.953152in}{1.098128in}}%
\pgfpathlineto{\pgfqpoint{1.954562in}{1.096440in}}%
\pgfpathlineto{\pgfqpoint{1.957697in}{1.096440in}}%
\pgfpathlineto{\pgfqpoint{1.959421in}{1.094376in}}%
\pgfpathlineto{\pgfqpoint{1.960832in}{1.092688in}}%
\pgfpathlineto{\pgfqpoint{1.962556in}{1.090625in}}%
\pgfpathlineto{\pgfqpoint{1.963967in}{1.088936in}}%
\pgfpathlineto{\pgfqpoint{1.965691in}{1.086873in}}%
\pgfpathlineto{\pgfqpoint{1.967101in}{1.085185in}}%
\pgfpathlineto{\pgfqpoint{1.970236in}{1.085185in}}%
\pgfpathlineto{\pgfqpoint{1.971960in}{1.083122in}}%
\pgfpathlineto{\pgfqpoint{1.973371in}{1.081433in}}%
\pgfpathlineto{\pgfqpoint{1.975095in}{1.079370in}}%
\pgfpathlineto{\pgfqpoint{1.976506in}{1.077682in}}%
\pgfpathlineto{\pgfqpoint{1.978230in}{1.075619in}}%
\pgfpathlineto{\pgfqpoint{1.979640in}{1.073930in}}%
\pgfpathlineto{\pgfqpoint{1.982775in}{1.073930in}}%
\pgfpathlineto{\pgfqpoint{1.984499in}{1.071867in}}%
\pgfpathlineto{\pgfqpoint{1.985910in}{1.070179in}}%
\pgfpathlineto{\pgfqpoint{1.987634in}{1.068115in}}%
\pgfpathlineto{\pgfqpoint{1.989045in}{1.066427in}}%
\pgfpathlineto{\pgfqpoint{1.990769in}{1.064364in}}%
\pgfpathlineto{\pgfqpoint{1.992179in}{1.062676in}}%
\pgfpathlineto{\pgfqpoint{1.995314in}{1.062676in}}%
\pgfpathlineto{\pgfqpoint{1.997038in}{1.060612in}}%
\pgfpathlineto{\pgfqpoint{1.998449in}{1.058924in}}%
\pgfpathlineto{\pgfqpoint{2.000173in}{1.056861in}}%
\pgfpathlineto{\pgfqpoint{2.001584in}{1.055173in}}%
\pgfpathlineto{\pgfqpoint{2.003308in}{1.053109in}}%
\pgfpathlineto{\pgfqpoint{2.004718in}{1.051421in}}%
\pgfpathlineto{\pgfqpoint{2.007853in}{1.051421in}}%
\pgfpathlineto{\pgfqpoint{2.009577in}{1.049358in}}%
\pgfpathlineto{\pgfqpoint{2.010988in}{1.047670in}}%
\pgfpathlineto{\pgfqpoint{2.012712in}{1.045606in}}%
\pgfpathlineto{\pgfqpoint{2.014122in}{1.043918in}}%
\pgfpathlineto{\pgfqpoint{2.015847in}{1.041855in}}%
\pgfpathlineto{\pgfqpoint{2.017257in}{1.040167in}}%
\pgfpathlineto{\pgfqpoint{2.018981in}{1.038103in}}%
\pgfpathlineto{\pgfqpoint{2.020392in}{1.036415in}}%
\pgfpathlineto{\pgfqpoint{2.023527in}{1.036415in}}%
\pgfpathlineto{\pgfqpoint{2.025251in}{1.034352in}}%
\pgfpathlineto{\pgfqpoint{2.026661in}{1.032663in}}%
\pgfpathlineto{\pgfqpoint{2.028386in}{1.030600in}}%
\pgfpathlineto{\pgfqpoint{2.029796in}{1.028912in}}%
\pgfpathlineto{\pgfqpoint{2.031520in}{1.026849in}}%
\pgfpathlineto{\pgfqpoint{2.032931in}{1.025160in}}%
\pgfpathlineto{\pgfqpoint{2.036066in}{1.025160in}}%
\pgfpathlineto{\pgfqpoint{2.037790in}{1.023097in}}%
\pgfpathlineto{\pgfqpoint{2.039200in}{1.021409in}}%
\pgfpathlineto{\pgfqpoint{2.040925in}{1.019346in}}%
\pgfpathlineto{\pgfqpoint{2.042335in}{1.017657in}}%
\pgfpathlineto{\pgfqpoint{2.044059in}{1.015594in}}%
\pgfpathlineto{\pgfqpoint{2.045470in}{1.013906in}}%
\pgfpathlineto{\pgfqpoint{2.048605in}{1.013906in}}%
\pgfpathlineto{\pgfqpoint{2.050329in}{1.011842in}}%
\pgfpathlineto{\pgfqpoint{2.051739in}{1.010154in}}%
\pgfpathlineto{\pgfqpoint{2.053463in}{1.008091in}}%
\pgfpathlineto{\pgfqpoint{2.054874in}{1.006403in}}%
\pgfpathlineto{\pgfqpoint{2.056598in}{1.004339in}}%
\pgfpathlineto{\pgfqpoint{2.058009in}{1.002651in}}%
\pgfpathlineto{\pgfqpoint{2.061144in}{1.002651in}}%
\pgfpathlineto{\pgfqpoint{2.062868in}{1.000588in}}%
\pgfpathlineto{\pgfqpoint{2.064278in}{0.998900in}}%
\pgfpathlineto{\pgfqpoint{2.066002in}{0.996836in}}%
\pgfpathlineto{\pgfqpoint{2.067413in}{0.995148in}}%
\pgfpathlineto{\pgfqpoint{2.069137in}{0.993085in}}%
\pgfpathlineto{\pgfqpoint{2.070548in}{0.991397in}}%
\pgfpathlineto{\pgfqpoint{2.073683in}{0.991397in}}%
\pgfpathlineto{\pgfqpoint{2.075407in}{0.989333in}}%
\pgfpathlineto{\pgfqpoint{2.076817in}{0.987645in}}%
\pgfpathlineto{\pgfqpoint{2.078541in}{0.985582in}}%
\pgfpathlineto{\pgfqpoint{2.079952in}{0.983894in}}%
\pgfpathlineto{\pgfqpoint{2.081676in}{0.981830in}}%
\pgfpathlineto{\pgfqpoint{2.083087in}{0.980142in}}%
\pgfpathlineto{\pgfqpoint{2.086222in}{0.980142in}}%
\pgfpathlineto{\pgfqpoint{2.087946in}{0.978079in}}%
\pgfpathlineto{\pgfqpoint{2.089356in}{0.976390in}}%
\pgfpathlineto{\pgfqpoint{2.091080in}{0.974327in}}%
\pgfpathlineto{\pgfqpoint{2.092491in}{0.972639in}}%
\pgfpathlineto{\pgfqpoint{2.094215in}{0.970576in}}%
\pgfpathlineto{\pgfqpoint{2.095626in}{0.968887in}}%
\pgfpathlineto{\pgfqpoint{2.097350in}{0.966824in}}%
\pgfpathlineto{\pgfqpoint{2.098761in}{0.965136in}}%
\pgfpathlineto{\pgfqpoint{2.101895in}{0.965136in}}%
\pgfpathlineto{\pgfqpoint{2.103619in}{0.963073in}}%
\pgfpathlineto{\pgfqpoint{2.105030in}{0.961384in}}%
\pgfpathlineto{\pgfqpoint{2.106754in}{0.959321in}}%
\pgfpathlineto{\pgfqpoint{2.108165in}{0.957633in}}%
\pgfpathlineto{\pgfqpoint{2.109889in}{0.955569in}}%
\pgfpathlineto{\pgfqpoint{2.111299in}{0.953881in}}%
\pgfpathlineto{\pgfqpoint{2.114434in}{0.953881in}}%
\pgfpathlineto{\pgfqpoint{2.116158in}{0.951818in}}%
\pgfpathlineto{\pgfqpoint{2.117569in}{0.950130in}}%
\pgfpathlineto{\pgfqpoint{2.119293in}{0.948066in}}%
\pgfpathlineto{\pgfqpoint{2.120704in}{0.946378in}}%
\pgfpathlineto{\pgfqpoint{2.122428in}{0.944315in}}%
\pgfpathlineto{\pgfqpoint{2.123838in}{0.942627in}}%
\pgfpathlineto{\pgfqpoint{2.126973in}{0.942627in}}%
\pgfpathlineto{\pgfqpoint{2.128697in}{0.940563in}}%
\pgfpathlineto{\pgfqpoint{2.130108in}{0.938875in}}%
\pgfpathlineto{\pgfqpoint{2.131832in}{0.936812in}}%
\pgfpathlineto{\pgfqpoint{2.133243in}{0.935124in}}%
\pgfpathlineto{\pgfqpoint{2.134967in}{0.933060in}}%
\pgfpathlineto{\pgfqpoint{2.136377in}{0.931372in}}%
\pgfpathlineto{\pgfqpoint{2.139512in}{0.931372in}}%
\pgfpathlineto{\pgfqpoint{2.141236in}{0.929309in}}%
\pgfpathlineto{\pgfqpoint{2.142647in}{0.927621in}}%
\pgfpathlineto{\pgfqpoint{2.144371in}{0.925557in}}%
\pgfpathlineto{\pgfqpoint{2.145782in}{0.923869in}}%
\pgfpathlineto{\pgfqpoint{2.147506in}{0.921806in}}%
\pgfpathlineto{\pgfqpoint{2.148916in}{0.920117in}}%
\pgfpathlineto{\pgfqpoint{2.152051in}{0.920117in}}%
\pgfpathlineto{\pgfqpoint{2.153775in}{0.918054in}}%
\pgfpathlineto{\pgfqpoint{2.155186in}{0.916366in}}%
\pgfpathlineto{\pgfqpoint{2.156910in}{0.914303in}}%
\pgfpathlineto{\pgfqpoint{2.158321in}{0.912614in}}%
\pgfpathlineto{\pgfqpoint{2.160045in}{0.910551in}}%
\pgfpathlineto{\pgfqpoint{2.161455in}{0.908863in}}%
\pgfpathlineto{\pgfqpoint{2.163179in}{0.906799in}}%
\pgfpathlineto{\pgfqpoint{2.164590in}{0.905111in}}%
\pgfpathlineto{\pgfqpoint{2.167725in}{0.905111in}}%
\pgfpathlineto{\pgfqpoint{2.169449in}{0.903048in}}%
\pgfpathlineto{\pgfqpoint{2.170860in}{0.901360in}}%
\pgfpathlineto{\pgfqpoint{2.172584in}{0.899296in}}%
\pgfpathclose%
\pgfpathmoveto{\pgfqpoint{2.173054in}{0.899296in}}%
\pgfpathlineto{\pgfqpoint{2.170860in}{0.901922in}}%
\pgfpathlineto{\pgfqpoint{2.169919in}{0.903048in}}%
\pgfpathlineto{\pgfqpoint{2.167725in}{0.905674in}}%
\pgfpathlineto{\pgfqpoint{2.164590in}{0.905674in}}%
\pgfpathlineto{\pgfqpoint{2.163650in}{0.906799in}}%
\pgfpathlineto{\pgfqpoint{2.161455in}{0.909426in}}%
\pgfpathlineto{\pgfqpoint{2.160515in}{0.910551in}}%
\pgfpathlineto{\pgfqpoint{2.158321in}{0.913177in}}%
\pgfpathlineto{\pgfqpoint{2.157380in}{0.914303in}}%
\pgfpathlineto{\pgfqpoint{2.155186in}{0.916929in}}%
\pgfpathlineto{\pgfqpoint{2.154245in}{0.918054in}}%
\pgfpathlineto{\pgfqpoint{2.152051in}{0.920680in}}%
\pgfpathlineto{\pgfqpoint{2.148916in}{0.920680in}}%
\pgfpathlineto{\pgfqpoint{2.147976in}{0.921806in}}%
\pgfpathlineto{\pgfqpoint{2.145782in}{0.924432in}}%
\pgfpathlineto{\pgfqpoint{2.144841in}{0.925557in}}%
\pgfpathlineto{\pgfqpoint{2.142647in}{0.928183in}}%
\pgfpathlineto{\pgfqpoint{2.141706in}{0.929309in}}%
\pgfpathlineto{\pgfqpoint{2.139512in}{0.931935in}}%
\pgfpathlineto{\pgfqpoint{2.136377in}{0.931935in}}%
\pgfpathlineto{\pgfqpoint{2.135437in}{0.933060in}}%
\pgfpathlineto{\pgfqpoint{2.133243in}{0.935686in}}%
\pgfpathlineto{\pgfqpoint{2.132302in}{0.936812in}}%
\pgfpathlineto{\pgfqpoint{2.130108in}{0.939438in}}%
\pgfpathlineto{\pgfqpoint{2.129168in}{0.940563in}}%
\pgfpathlineto{\pgfqpoint{2.126973in}{0.943189in}}%
\pgfpathlineto{\pgfqpoint{2.123838in}{0.943189in}}%
\pgfpathlineto{\pgfqpoint{2.122898in}{0.944315in}}%
\pgfpathlineto{\pgfqpoint{2.120704in}{0.946941in}}%
\pgfpathlineto{\pgfqpoint{2.119763in}{0.948066in}}%
\pgfpathlineto{\pgfqpoint{2.117569in}{0.950692in}}%
\pgfpathlineto{\pgfqpoint{2.116629in}{0.951818in}}%
\pgfpathlineto{\pgfqpoint{2.114434in}{0.954444in}}%
\pgfpathlineto{\pgfqpoint{2.111299in}{0.954444in}}%
\pgfpathlineto{\pgfqpoint{2.110359in}{0.955569in}}%
\pgfpathlineto{\pgfqpoint{2.108165in}{0.958196in}}%
\pgfpathlineto{\pgfqpoint{2.107224in}{0.959321in}}%
\pgfpathlineto{\pgfqpoint{2.105030in}{0.961947in}}%
\pgfpathlineto{\pgfqpoint{2.104090in}{0.963073in}}%
\pgfpathlineto{\pgfqpoint{2.101895in}{0.965699in}}%
\pgfpathlineto{\pgfqpoint{2.098761in}{0.965699in}}%
\pgfpathlineto{\pgfqpoint{2.097820in}{0.966824in}}%
\pgfpathlineto{\pgfqpoint{2.095626in}{0.969450in}}%
\pgfpathlineto{\pgfqpoint{2.094685in}{0.970576in}}%
\pgfpathlineto{\pgfqpoint{2.092491in}{0.973202in}}%
\pgfpathlineto{\pgfqpoint{2.091551in}{0.974327in}}%
\pgfpathlineto{\pgfqpoint{2.089356in}{0.976953in}}%
\pgfpathlineto{\pgfqpoint{2.088416in}{0.978079in}}%
\pgfpathlineto{\pgfqpoint{2.086222in}{0.980705in}}%
\pgfpathlineto{\pgfqpoint{2.083087in}{0.980705in}}%
\pgfpathlineto{\pgfqpoint{2.082146in}{0.981830in}}%
\pgfpathlineto{\pgfqpoint{2.079952in}{0.984456in}}%
\pgfpathlineto{\pgfqpoint{2.079012in}{0.985582in}}%
\pgfpathlineto{\pgfqpoint{2.076817in}{0.988208in}}%
\pgfpathlineto{\pgfqpoint{2.075877in}{0.989333in}}%
\pgfpathlineto{\pgfqpoint{2.073683in}{0.991959in}}%
\pgfpathlineto{\pgfqpoint{2.070548in}{0.991959in}}%
\pgfpathlineto{\pgfqpoint{2.069607in}{0.993085in}}%
\pgfpathlineto{\pgfqpoint{2.067413in}{0.995711in}}%
\pgfpathlineto{\pgfqpoint{2.066473in}{0.996836in}}%
\pgfpathlineto{\pgfqpoint{2.064278in}{0.999462in}}%
\pgfpathlineto{\pgfqpoint{2.063338in}{1.000588in}}%
\pgfpathlineto{\pgfqpoint{2.061144in}{1.003214in}}%
\pgfpathlineto{\pgfqpoint{2.058009in}{1.003214in}}%
\pgfpathlineto{\pgfqpoint{2.057068in}{1.004339in}}%
\pgfpathlineto{\pgfqpoint{2.054874in}{1.006965in}}%
\pgfpathlineto{\pgfqpoint{2.053934in}{1.008091in}}%
\pgfpathlineto{\pgfqpoint{2.051739in}{1.010717in}}%
\pgfpathlineto{\pgfqpoint{2.050799in}{1.011842in}}%
\pgfpathlineto{\pgfqpoint{2.048605in}{1.014469in}}%
\pgfpathlineto{\pgfqpoint{2.045470in}{1.014469in}}%
\pgfpathlineto{\pgfqpoint{2.044529in}{1.015594in}}%
\pgfpathlineto{\pgfqpoint{2.042335in}{1.018220in}}%
\pgfpathlineto{\pgfqpoint{2.041395in}{1.019346in}}%
\pgfpathlineto{\pgfqpoint{2.039200in}{1.021972in}}%
\pgfpathlineto{\pgfqpoint{2.038260in}{1.023097in}}%
\pgfpathlineto{\pgfqpoint{2.036066in}{1.025723in}}%
\pgfpathlineto{\pgfqpoint{2.032931in}{1.025723in}}%
\pgfpathlineto{\pgfqpoint{2.031991in}{1.026849in}}%
\pgfpathlineto{\pgfqpoint{2.029796in}{1.029475in}}%
\pgfpathlineto{\pgfqpoint{2.028856in}{1.030600in}}%
\pgfpathlineto{\pgfqpoint{2.026661in}{1.033226in}}%
\pgfpathlineto{\pgfqpoint{2.025721in}{1.034352in}}%
\pgfpathlineto{\pgfqpoint{2.023527in}{1.036978in}}%
\pgfpathlineto{\pgfqpoint{2.020392in}{1.036978in}}%
\pgfpathlineto{\pgfqpoint{2.019452in}{1.038103in}}%
\pgfpathlineto{\pgfqpoint{2.017257in}{1.040729in}}%
\pgfpathlineto{\pgfqpoint{2.016317in}{1.041855in}}%
\pgfpathlineto{\pgfqpoint{2.014122in}{1.044481in}}%
\pgfpathlineto{\pgfqpoint{2.013182in}{1.045606in}}%
\pgfpathlineto{\pgfqpoint{2.010988in}{1.048232in}}%
\pgfpathlineto{\pgfqpoint{2.010047in}{1.049358in}}%
\pgfpathlineto{\pgfqpoint{2.007853in}{1.051984in}}%
\pgfpathlineto{\pgfqpoint{2.004718in}{1.051984in}}%
\pgfpathlineto{\pgfqpoint{2.003778in}{1.053109in}}%
\pgfpathlineto{\pgfqpoint{2.001584in}{1.055735in}}%
\pgfpathlineto{\pgfqpoint{2.000643in}{1.056861in}}%
\pgfpathlineto{\pgfqpoint{1.998449in}{1.059487in}}%
\pgfpathlineto{\pgfqpoint{1.997508in}{1.060612in}}%
\pgfpathlineto{\pgfqpoint{1.995314in}{1.063238in}}%
\pgfpathlineto{\pgfqpoint{1.992179in}{1.063238in}}%
\pgfpathlineto{\pgfqpoint{1.991239in}{1.064364in}}%
\pgfpathlineto{\pgfqpoint{1.989045in}{1.066990in}}%
\pgfpathlineto{\pgfqpoint{1.988104in}{1.068115in}}%
\pgfpathlineto{\pgfqpoint{1.985910in}{1.070742in}}%
\pgfpathlineto{\pgfqpoint{1.984969in}{1.071867in}}%
\pgfpathlineto{\pgfqpoint{1.982775in}{1.074493in}}%
\pgfpathlineto{\pgfqpoint{1.979640in}{1.074493in}}%
\pgfpathlineto{\pgfqpoint{1.978700in}{1.075619in}}%
\pgfpathlineto{\pgfqpoint{1.976506in}{1.078245in}}%
\pgfpathlineto{\pgfqpoint{1.975565in}{1.079370in}}%
\pgfpathlineto{\pgfqpoint{1.973371in}{1.081996in}}%
\pgfpathlineto{\pgfqpoint{1.972430in}{1.083122in}}%
\pgfpathlineto{\pgfqpoint{1.970236in}{1.085748in}}%
\pgfpathlineto{\pgfqpoint{1.967101in}{1.085748in}}%
\pgfpathlineto{\pgfqpoint{1.966161in}{1.086873in}}%
\pgfpathlineto{\pgfqpoint{1.963967in}{1.089499in}}%
\pgfpathlineto{\pgfqpoint{1.963026in}{1.090625in}}%
\pgfpathlineto{\pgfqpoint{1.960832in}{1.093251in}}%
\pgfpathlineto{\pgfqpoint{1.959891in}{1.094376in}}%
\pgfpathlineto{\pgfqpoint{1.957697in}{1.097002in}}%
\pgfpathlineto{\pgfqpoint{1.954562in}{1.097002in}}%
\pgfpathlineto{\pgfqpoint{1.953622in}{1.098128in}}%
\pgfpathlineto{\pgfqpoint{1.951428in}{1.100754in}}%
\pgfpathlineto{\pgfqpoint{1.950487in}{1.101879in}}%
\pgfpathlineto{\pgfqpoint{1.948293in}{1.104505in}}%
\pgfpathlineto{\pgfqpoint{1.947352in}{1.105631in}}%
\pgfpathlineto{\pgfqpoint{1.945158in}{1.108257in}}%
\pgfpathlineto{\pgfqpoint{1.942023in}{1.108257in}}%
\pgfpathlineto{\pgfqpoint{1.941083in}{1.109382in}}%
\pgfpathlineto{\pgfqpoint{1.938889in}{1.112008in}}%
\pgfpathlineto{\pgfqpoint{1.937948in}{1.113134in}}%
\pgfpathlineto{\pgfqpoint{1.935754in}{1.115760in}}%
\pgfpathlineto{\pgfqpoint{1.934814in}{1.116885in}}%
\pgfpathlineto{\pgfqpoint{1.932619in}{1.119511in}}%
\pgfpathlineto{\pgfqpoint{1.931679in}{1.120637in}}%
\pgfpathlineto{\pgfqpoint{1.929484in}{1.123263in}}%
\pgfpathlineto{\pgfqpoint{1.926350in}{1.123263in}}%
\pgfpathlineto{\pgfqpoint{1.925409in}{1.124388in}}%
\pgfpathlineto{\pgfqpoint{1.923215in}{1.127015in}}%
\pgfpathlineto{\pgfqpoint{1.922275in}{1.128140in}}%
\pgfpathlineto{\pgfqpoint{1.920080in}{1.130766in}}%
\pgfpathlineto{\pgfqpoint{1.919140in}{1.131892in}}%
\pgfpathlineto{\pgfqpoint{1.916945in}{1.134518in}}%
\pgfpathlineto{\pgfqpoint{1.913811in}{1.134518in}}%
\pgfpathlineto{\pgfqpoint{1.912870in}{1.135643in}}%
\pgfpathlineto{\pgfqpoint{1.910676in}{1.138269in}}%
\pgfpathlineto{\pgfqpoint{1.909736in}{1.139395in}}%
\pgfpathlineto{\pgfqpoint{1.907541in}{1.142021in}}%
\pgfpathlineto{\pgfqpoint{1.906601in}{1.143146in}}%
\pgfpathlineto{\pgfqpoint{1.904407in}{1.145772in}}%
\pgfpathlineto{\pgfqpoint{1.901272in}{1.145772in}}%
\pgfpathlineto{\pgfqpoint{1.900331in}{1.146898in}}%
\pgfpathlineto{\pgfqpoint{1.898137in}{1.149524in}}%
\pgfpathlineto{\pgfqpoint{1.897197in}{1.150649in}}%
\pgfpathlineto{\pgfqpoint{1.895002in}{1.153275in}}%
\pgfpathlineto{\pgfqpoint{1.894062in}{1.154401in}}%
\pgfpathlineto{\pgfqpoint{1.891868in}{1.157027in}}%
\pgfpathlineto{\pgfqpoint{1.888733in}{1.157027in}}%
\pgfpathlineto{\pgfqpoint{1.887792in}{1.158152in}}%
\pgfpathlineto{\pgfqpoint{1.885598in}{1.160778in}}%
\pgfpathlineto{\pgfqpoint{1.884658in}{1.161904in}}%
\pgfpathlineto{\pgfqpoint{1.882463in}{1.164530in}}%
\pgfpathlineto{\pgfqpoint{1.881523in}{1.165655in}}%
\pgfpathlineto{\pgfqpoint{1.879329in}{1.168281in}}%
\pgfpathlineto{\pgfqpoint{1.876194in}{1.168281in}}%
\pgfpathlineto{\pgfqpoint{1.875253in}{1.169407in}}%
\pgfpathlineto{\pgfqpoint{1.873059in}{1.172033in}}%
\pgfpathlineto{\pgfqpoint{1.872119in}{1.173158in}}%
\pgfpathlineto{\pgfqpoint{1.869924in}{1.175785in}}%
\pgfpathlineto{\pgfqpoint{1.868984in}{1.176910in}}%
\pgfpathlineto{\pgfqpoint{1.866790in}{1.179536in}}%
\pgfpathlineto{\pgfqpoint{1.863655in}{1.179536in}}%
\pgfpathlineto{\pgfqpoint{1.862714in}{1.180662in}}%
\pgfpathlineto{\pgfqpoint{1.860520in}{1.183288in}}%
\pgfpathlineto{\pgfqpoint{1.859580in}{1.184413in}}%
\pgfpathlineto{\pgfqpoint{1.857385in}{1.187039in}}%
\pgfpathlineto{\pgfqpoint{1.856445in}{1.188165in}}%
\pgfpathlineto{\pgfqpoint{1.854251in}{1.190791in}}%
\pgfpathlineto{\pgfqpoint{1.853310in}{1.191916in}}%
\pgfpathlineto{\pgfqpoint{1.851116in}{1.194542in}}%
\pgfpathlineto{\pgfqpoint{1.847981in}{1.194542in}}%
\pgfpathlineto{\pgfqpoint{1.847041in}{1.195668in}}%
\pgfpathlineto{\pgfqpoint{1.844846in}{1.198294in}}%
\pgfpathlineto{\pgfqpoint{1.843906in}{1.199419in}}%
\pgfpathlineto{\pgfqpoint{1.841712in}{1.202045in}}%
\pgfpathlineto{\pgfqpoint{1.840771in}{1.203171in}}%
\pgfpathlineto{\pgfqpoint{1.838577in}{1.205797in}}%
\pgfpathlineto{\pgfqpoint{1.835442in}{1.205797in}}%
\pgfpathlineto{\pgfqpoint{1.834502in}{1.206922in}}%
\pgfpathlineto{\pgfqpoint{1.832307in}{1.209548in}}%
\pgfpathlineto{\pgfqpoint{1.831367in}{1.210674in}}%
\pgfpathlineto{\pgfqpoint{1.829173in}{1.213300in}}%
\pgfpathlineto{\pgfqpoint{1.828232in}{1.214425in}}%
\pgfpathlineto{\pgfqpoint{1.826038in}{1.217051in}}%
\pgfpathlineto{\pgfqpoint{1.822903in}{1.217051in}}%
\pgfpathlineto{\pgfqpoint{1.821963in}{1.218177in}}%
\pgfpathlineto{\pgfqpoint{1.819768in}{1.220803in}}%
\pgfpathlineto{\pgfqpoint{1.818828in}{1.221928in}}%
\pgfpathlineto{\pgfqpoint{1.816634in}{1.224554in}}%
\pgfpathlineto{\pgfqpoint{1.815693in}{1.225680in}}%
\pgfpathlineto{\pgfqpoint{1.813499in}{1.228306in}}%
\pgfpathlineto{\pgfqpoint{1.810364in}{1.228306in}}%
\pgfpathlineto{\pgfqpoint{1.809424in}{1.229431in}}%
\pgfpathlineto{\pgfqpoint{1.807229in}{1.232058in}}%
\pgfpathlineto{\pgfqpoint{1.806289in}{1.233183in}}%
\pgfpathlineto{\pgfqpoint{1.804095in}{1.235809in}}%
\pgfpathlineto{\pgfqpoint{1.803154in}{1.236935in}}%
\pgfpathlineto{\pgfqpoint{1.800960in}{1.239561in}}%
\pgfpathlineto{\pgfqpoint{1.797825in}{1.239561in}}%
\pgfpathlineto{\pgfqpoint{1.796885in}{1.240686in}}%
\pgfpathlineto{\pgfqpoint{1.794691in}{1.243312in}}%
\pgfpathlineto{\pgfqpoint{1.793750in}{1.244438in}}%
\pgfpathlineto{\pgfqpoint{1.791556in}{1.247064in}}%
\pgfpathlineto{\pgfqpoint{1.790615in}{1.248189in}}%
\pgfpathlineto{\pgfqpoint{1.788421in}{1.250815in}}%
\pgfpathlineto{\pgfqpoint{1.787481in}{1.251941in}}%
\pgfpathlineto{\pgfqpoint{1.785286in}{1.254567in}}%
\pgfpathlineto{\pgfqpoint{1.782152in}{1.254567in}}%
\pgfpathlineto{\pgfqpoint{1.781211in}{1.255692in}}%
\pgfpathlineto{\pgfqpoint{1.779017in}{1.258318in}}%
\pgfpathlineto{\pgfqpoint{1.778076in}{1.259444in}}%
\pgfpathlineto{\pgfqpoint{1.775882in}{1.262070in}}%
\pgfpathlineto{\pgfqpoint{1.774942in}{1.263195in}}%
\pgfpathlineto{\pgfqpoint{1.772747in}{1.265821in}}%
\pgfpathlineto{\pgfqpoint{1.769613in}{1.265821in}}%
\pgfpathlineto{\pgfqpoint{1.768672in}{1.266947in}}%
\pgfpathlineto{\pgfqpoint{1.766478in}{1.269573in}}%
\pgfpathlineto{\pgfqpoint{1.765537in}{1.270698in}}%
\pgfpathlineto{\pgfqpoint{1.763343in}{1.273324in}}%
\pgfpathlineto{\pgfqpoint{1.762403in}{1.274450in}}%
\pgfpathlineto{\pgfqpoint{1.760208in}{1.277076in}}%
\pgfpathlineto{\pgfqpoint{1.757074in}{1.277076in}}%
\pgfpathlineto{\pgfqpoint{1.756133in}{1.278201in}}%
\pgfpathlineto{\pgfqpoint{1.753939in}{1.280827in}}%
\pgfpathlineto{\pgfqpoint{1.752998in}{1.281953in}}%
\pgfpathlineto{\pgfqpoint{1.750804in}{1.284579in}}%
\pgfpathlineto{\pgfqpoint{1.749864in}{1.285704in}}%
\pgfpathlineto{\pgfqpoint{1.747669in}{1.288331in}}%
\pgfpathlineto{\pgfqpoint{1.744535in}{1.288331in}}%
\pgfpathlineto{\pgfqpoint{1.743594in}{1.289456in}}%
\pgfpathlineto{\pgfqpoint{1.741400in}{1.292082in}}%
\pgfpathlineto{\pgfqpoint{1.740459in}{1.293208in}}%
\pgfpathlineto{\pgfqpoint{1.738265in}{1.295834in}}%
\pgfpathlineto{\pgfqpoint{1.737325in}{1.296959in}}%
\pgfpathlineto{\pgfqpoint{1.735130in}{1.299585in}}%
\pgfpathlineto{\pgfqpoint{1.731996in}{1.299585in}}%
\pgfpathlineto{\pgfqpoint{1.731055in}{1.300711in}}%
\pgfpathlineto{\pgfqpoint{1.728861in}{1.303337in}}%
\pgfpathlineto{\pgfqpoint{1.727921in}{1.304462in}}%
\pgfpathlineto{\pgfqpoint{1.725726in}{1.307088in}}%
\pgfpathlineto{\pgfqpoint{1.724786in}{1.308214in}}%
\pgfpathlineto{\pgfqpoint{1.722591in}{1.310840in}}%
\pgfpathlineto{\pgfqpoint{1.719457in}{1.310840in}}%
\pgfpathlineto{\pgfqpoint{1.718516in}{1.311965in}}%
\pgfpathlineto{\pgfqpoint{1.716322in}{1.314591in}}%
\pgfpathlineto{\pgfqpoint{1.715382in}{1.315717in}}%
\pgfpathlineto{\pgfqpoint{1.713187in}{1.318343in}}%
\pgfpathlineto{\pgfqpoint{1.712247in}{1.319468in}}%
\pgfpathlineto{\pgfqpoint{1.710052in}{1.322094in}}%
\pgfpathlineto{\pgfqpoint{1.709112in}{1.323220in}}%
\pgfpathlineto{\pgfqpoint{1.706918in}{1.325846in}}%
\pgfpathlineto{\pgfqpoint{1.703783in}{1.325846in}}%
\pgfpathlineto{\pgfqpoint{1.702843in}{1.326971in}}%
\pgfpathlineto{\pgfqpoint{1.700648in}{1.329597in}}%
\pgfpathlineto{\pgfqpoint{1.699708in}{1.330723in}}%
\pgfpathlineto{\pgfqpoint{1.697514in}{1.333349in}}%
\pgfpathlineto{\pgfqpoint{1.696573in}{1.334474in}}%
\pgfpathlineto{\pgfqpoint{1.694379in}{1.337100in}}%
\pgfpathlineto{\pgfqpoint{1.691244in}{1.337100in}}%
\pgfpathlineto{\pgfqpoint{1.690304in}{1.338226in}}%
\pgfpathlineto{\pgfqpoint{1.688109in}{1.340852in}}%
\pgfpathlineto{\pgfqpoint{1.687169in}{1.341977in}}%
\pgfpathlineto{\pgfqpoint{1.684975in}{1.344604in}}%
\pgfpathlineto{\pgfqpoint{1.684034in}{1.345729in}}%
\pgfpathlineto{\pgfqpoint{1.681840in}{1.348355in}}%
\pgfpathlineto{\pgfqpoint{1.678705in}{1.348355in}}%
\pgfpathlineto{\pgfqpoint{1.677765in}{1.349481in}}%
\pgfpathlineto{\pgfqpoint{1.675570in}{1.352107in}}%
\pgfpathlineto{\pgfqpoint{1.674630in}{1.353232in}}%
\pgfpathlineto{\pgfqpoint{1.672436in}{1.355858in}}%
\pgfpathlineto{\pgfqpoint{1.671495in}{1.356984in}}%
\pgfpathlineto{\pgfqpoint{1.669301in}{1.359610in}}%
\pgfpathlineto{\pgfqpoint{1.666166in}{1.359610in}}%
\pgfpathlineto{\pgfqpoint{1.665226in}{1.360735in}}%
\pgfpathlineto{\pgfqpoint{1.663031in}{1.363361in}}%
\pgfpathlineto{\pgfqpoint{1.662091in}{1.364487in}}%
\pgfpathlineto{\pgfqpoint{1.659897in}{1.367113in}}%
\pgfpathlineto{\pgfqpoint{1.658956in}{1.368238in}}%
\pgfpathlineto{\pgfqpoint{1.656762in}{1.370864in}}%
\pgfpathlineto{\pgfqpoint{1.653627in}{1.370864in}}%
\pgfpathlineto{\pgfqpoint{1.652687in}{1.371990in}}%
\pgfpathlineto{\pgfqpoint{1.650492in}{1.374616in}}%
\pgfpathlineto{\pgfqpoint{1.649552in}{1.375741in}}%
\pgfpathlineto{\pgfqpoint{1.647358in}{1.378367in}}%
\pgfpathlineto{\pgfqpoint{1.646417in}{1.379493in}}%
\pgfpathlineto{\pgfqpoint{1.644223in}{1.382119in}}%
\pgfpathlineto{\pgfqpoint{1.641088in}{1.382119in}}%
\pgfpathlineto{\pgfqpoint{1.640148in}{1.383244in}}%
\pgfpathlineto{\pgfqpoint{1.637953in}{1.385870in}}%
\pgfpathlineto{\pgfqpoint{1.637013in}{1.386996in}}%
\pgfpathlineto{\pgfqpoint{1.634819in}{1.389622in}}%
\pgfpathlineto{\pgfqpoint{1.633878in}{1.390747in}}%
\pgfpathlineto{\pgfqpoint{1.631684in}{1.393374in}}%
\pgfpathlineto{\pgfqpoint{1.630744in}{1.394499in}}%
\pgfpathlineto{\pgfqpoint{1.628549in}{1.397125in}}%
\pgfpathlineto{\pgfqpoint{1.625414in}{1.397125in}}%
\pgfpathlineto{\pgfqpoint{1.624474in}{1.398251in}}%
\pgfpathlineto{\pgfqpoint{1.622280in}{1.400877in}}%
\pgfpathlineto{\pgfqpoint{1.621339in}{1.402002in}}%
\pgfpathlineto{\pgfqpoint{1.619145in}{1.404628in}}%
\pgfpathlineto{\pgfqpoint{1.618205in}{1.405754in}}%
\pgfpathlineto{\pgfqpoint{1.616010in}{1.408380in}}%
\pgfpathlineto{\pgfqpoint{1.612875in}{1.408380in}}%
\pgfpathlineto{\pgfqpoint{1.611935in}{1.409505in}}%
\pgfpathlineto{\pgfqpoint{1.609741in}{1.412131in}}%
\pgfpathlineto{\pgfqpoint{1.608800in}{1.413257in}}%
\pgfpathlineto{\pgfqpoint{1.606606in}{1.415883in}}%
\pgfpathlineto{\pgfqpoint{1.605666in}{1.417008in}}%
\pgfpathlineto{\pgfqpoint{1.603471in}{1.419634in}}%
\pgfpathlineto{\pgfqpoint{1.600337in}{1.419634in}}%
\pgfpathlineto{\pgfqpoint{1.599396in}{1.420760in}}%
\pgfpathlineto{\pgfqpoint{1.597202in}{1.423386in}}%
\pgfpathlineto{\pgfqpoint{1.596261in}{1.424511in}}%
\pgfpathlineto{\pgfqpoint{1.594067in}{1.427137in}}%
\pgfpathlineto{\pgfqpoint{1.593127in}{1.428263in}}%
\pgfpathlineto{\pgfqpoint{1.590932in}{1.430889in}}%
\pgfpathlineto{\pgfqpoint{1.587798in}{1.430889in}}%
\pgfpathlineto{\pgfqpoint{1.586857in}{1.432014in}}%
\pgfpathlineto{\pgfqpoint{1.584663in}{1.434640in}}%
\pgfpathlineto{\pgfqpoint{1.583722in}{1.435766in}}%
\pgfpathlineto{\pgfqpoint{1.581528in}{1.438392in}}%
\pgfpathlineto{\pgfqpoint{1.580588in}{1.439517in}}%
\pgfpathlineto{\pgfqpoint{1.578393in}{1.442143in}}%
\pgfpathlineto{\pgfqpoint{1.575259in}{1.442143in}}%
\pgfpathlineto{\pgfqpoint{1.574318in}{1.443269in}}%
\pgfpathlineto{\pgfqpoint{1.572124in}{1.445895in}}%
\pgfpathlineto{\pgfqpoint{1.571183in}{1.447020in}}%
\pgfpathlineto{\pgfqpoint{1.568989in}{1.449647in}}%
\pgfpathlineto{\pgfqpoint{1.568049in}{1.450772in}}%
\pgfpathlineto{\pgfqpoint{1.565854in}{1.453398in}}%
\pgfpathlineto{\pgfqpoint{1.562720in}{1.453398in}}%
\pgfpathlineto{\pgfqpoint{1.561779in}{1.454524in}}%
\pgfpathlineto{\pgfqpoint{1.559585in}{1.457150in}}%
\pgfpathlineto{\pgfqpoint{1.558644in}{1.458275in}}%
\pgfpathlineto{\pgfqpoint{1.556450in}{1.460901in}}%
\pgfpathlineto{\pgfqpoint{1.555510in}{1.462027in}}%
\pgfpathlineto{\pgfqpoint{1.553315in}{1.464653in}}%
\pgfpathlineto{\pgfqpoint{1.552375in}{1.465778in}}%
\pgfpathlineto{\pgfqpoint{1.550181in}{1.468404in}}%
\pgfpathlineto{\pgfqpoint{1.547046in}{1.468404in}}%
\pgfpathlineto{\pgfqpoint{1.546105in}{1.469530in}}%
\pgfpathlineto{\pgfqpoint{1.543911in}{1.472156in}}%
\pgfpathlineto{\pgfqpoint{1.542971in}{1.473281in}}%
\pgfpathlineto{\pgfqpoint{1.540776in}{1.475907in}}%
\pgfpathlineto{\pgfqpoint{1.539836in}{1.477033in}}%
\pgfpathlineto{\pgfqpoint{1.537642in}{1.479659in}}%
\pgfpathlineto{\pgfqpoint{1.534507in}{1.479659in}}%
\pgfpathlineto{\pgfqpoint{1.533566in}{1.480784in}}%
\pgfpathlineto{\pgfqpoint{1.531372in}{1.483410in}}%
\pgfpathlineto{\pgfqpoint{1.530432in}{1.484536in}}%
\pgfpathlineto{\pgfqpoint{1.528237in}{1.487162in}}%
\pgfpathlineto{\pgfqpoint{1.527297in}{1.488287in}}%
\pgfpathlineto{\pgfqpoint{1.525103in}{1.490913in}}%
\pgfpathlineto{\pgfqpoint{1.521968in}{1.490913in}}%
\pgfpathlineto{\pgfqpoint{1.521028in}{1.492039in}}%
\pgfpathlineto{\pgfqpoint{1.518833in}{1.494665in}}%
\pgfpathlineto{\pgfqpoint{1.517893in}{1.495790in}}%
\pgfpathlineto{\pgfqpoint{1.515698in}{1.498416in}}%
\pgfpathlineto{\pgfqpoint{1.514758in}{1.499542in}}%
\pgfpathlineto{\pgfqpoint{1.512564in}{1.502168in}}%
\pgfpathlineto{\pgfqpoint{1.509429in}{1.502168in}}%
\pgfpathlineto{\pgfqpoint{1.508489in}{1.503293in}}%
\pgfpathlineto{\pgfqpoint{1.506294in}{1.505920in}}%
\pgfpathlineto{\pgfqpoint{1.505354in}{1.507045in}}%
\pgfpathlineto{\pgfqpoint{1.503159in}{1.509671in}}%
\pgfpathlineto{\pgfqpoint{1.502219in}{1.510797in}}%
\pgfpathlineto{\pgfqpoint{1.500025in}{1.513423in}}%
\pgfpathlineto{\pgfqpoint{1.496890in}{1.513423in}}%
\pgfpathlineto{\pgfqpoint{1.495950in}{1.514548in}}%
\pgfpathlineto{\pgfqpoint{1.493755in}{1.517174in}}%
\pgfpathlineto{\pgfqpoint{1.492815in}{1.518300in}}%
\pgfpathlineto{\pgfqpoint{1.490621in}{1.520926in}}%
\pgfpathlineto{\pgfqpoint{1.489680in}{1.522051in}}%
\pgfpathlineto{\pgfqpoint{1.487486in}{1.524677in}}%
\pgfpathlineto{\pgfqpoint{1.486545in}{1.525803in}}%
\pgfpathlineto{\pgfqpoint{1.484351in}{1.528429in}}%
\pgfpathlineto{\pgfqpoint{1.481216in}{1.528429in}}%
\pgfpathlineto{\pgfqpoint{1.480276in}{1.529554in}}%
\pgfpathlineto{\pgfqpoint{1.478082in}{1.532180in}}%
\pgfpathlineto{\pgfqpoint{1.477141in}{1.533306in}}%
\pgfpathlineto{\pgfqpoint{1.474947in}{1.535932in}}%
\pgfpathlineto{\pgfqpoint{1.474006in}{1.537057in}}%
\pgfpathlineto{\pgfqpoint{1.471812in}{1.539683in}}%
\pgfpathlineto{\pgfqpoint{1.468677in}{1.539683in}}%
\pgfpathlineto{\pgfqpoint{1.467737in}{1.540809in}}%
\pgfpathlineto{\pgfqpoint{1.465543in}{1.543435in}}%
\pgfpathlineto{\pgfqpoint{1.464602in}{1.544560in}}%
\pgfpathlineto{\pgfqpoint{1.462408in}{1.547186in}}%
\pgfpathlineto{\pgfqpoint{1.461467in}{1.548312in}}%
\pgfpathlineto{\pgfqpoint{1.459273in}{1.550938in}}%
\pgfpathlineto{\pgfqpoint{1.456138in}{1.550938in}}%
\pgfpathlineto{\pgfqpoint{1.455198in}{1.552063in}}%
\pgfpathlineto{\pgfqpoint{1.453004in}{1.554690in}}%
\pgfpathlineto{\pgfqpoint{1.452063in}{1.555815in}}%
\pgfpathlineto{\pgfqpoint{1.449869in}{1.558441in}}%
\pgfpathlineto{\pgfqpoint{1.448928in}{1.559566in}}%
\pgfpathlineto{\pgfqpoint{1.446734in}{1.562193in}}%
\pgfpathlineto{\pgfqpoint{1.443599in}{1.562193in}}%
\pgfpathlineto{\pgfqpoint{1.442659in}{1.563318in}}%
\pgfpathlineto{\pgfqpoint{1.440465in}{1.565944in}}%
\pgfpathlineto{\pgfqpoint{1.439524in}{1.567070in}}%
\pgfpathlineto{\pgfqpoint{1.437330in}{1.569696in}}%
\pgfpathlineto{\pgfqpoint{1.436389in}{1.570821in}}%
\pgfpathlineto{\pgfqpoint{1.434195in}{1.573447in}}%
\pgfpathlineto{\pgfqpoint{1.431060in}{1.573447in}}%
\pgfpathlineto{\pgfqpoint{1.430120in}{1.574573in}}%
\pgfpathlineto{\pgfqpoint{1.427926in}{1.577199in}}%
\pgfpathlineto{\pgfqpoint{1.426985in}{1.578324in}}%
\pgfpathlineto{\pgfqpoint{1.424791in}{1.580950in}}%
\pgfpathlineto{\pgfqpoint{1.423851in}{1.582076in}}%
\pgfpathlineto{\pgfqpoint{1.421656in}{1.584702in}}%
\pgfpathlineto{\pgfqpoint{1.418521in}{1.584702in}}%
\pgfpathlineto{\pgfqpoint{1.417581in}{1.585827in}}%
\pgfpathlineto{\pgfqpoint{1.415387in}{1.588453in}}%
\pgfpathlineto{\pgfqpoint{1.414446in}{1.589579in}}%
\pgfpathlineto{\pgfqpoint{1.412252in}{1.592205in}}%
\pgfpathlineto{\pgfqpoint{1.411312in}{1.593330in}}%
\pgfpathlineto{\pgfqpoint{1.409117in}{1.595956in}}%
\pgfpathlineto{\pgfqpoint{1.408177in}{1.597082in}}%
\pgfpathlineto{\pgfqpoint{1.405982in}{1.599708in}}%
\pgfpathlineto{\pgfqpoint{1.402848in}{1.599708in}}%
\pgfpathlineto{\pgfqpoint{1.401907in}{1.600833in}}%
\pgfpathlineto{\pgfqpoint{1.399713in}{1.603459in}}%
\pgfpathlineto{\pgfqpoint{1.398773in}{1.604585in}}%
\pgfpathlineto{\pgfqpoint{1.396578in}{1.607211in}}%
\pgfpathlineto{\pgfqpoint{1.395638in}{1.608336in}}%
\pgfpathlineto{\pgfqpoint{1.393444in}{1.610963in}}%
\pgfpathlineto{\pgfqpoint{1.390309in}{1.610963in}}%
\pgfpathlineto{\pgfqpoint{1.389368in}{1.612088in}}%
\pgfpathlineto{\pgfqpoint{1.387174in}{1.614714in}}%
\pgfpathlineto{\pgfqpoint{1.386234in}{1.615840in}}%
\pgfpathlineto{\pgfqpoint{1.384039in}{1.618466in}}%
\pgfpathlineto{\pgfqpoint{1.383099in}{1.619591in}}%
\pgfpathlineto{\pgfqpoint{1.380905in}{1.622217in}}%
\pgfpathlineto{\pgfqpoint{1.377770in}{1.622217in}}%
\pgfpathlineto{\pgfqpoint{1.376829in}{1.623343in}}%
\pgfpathlineto{\pgfqpoint{1.374635in}{1.625969in}}%
\pgfpathlineto{\pgfqpoint{1.373695in}{1.627094in}}%
\pgfpathlineto{\pgfqpoint{1.371500in}{1.629720in}}%
\pgfpathlineto{\pgfqpoint{1.370560in}{1.630846in}}%
\pgfpathlineto{\pgfqpoint{1.368366in}{1.633472in}}%
\pgfpathlineto{\pgfqpoint{1.365231in}{1.633472in}}%
\pgfpathlineto{\pgfqpoint{1.364290in}{1.634597in}}%
\pgfpathlineto{\pgfqpoint{1.362096in}{1.637223in}}%
\pgfpathlineto{\pgfqpoint{1.361156in}{1.638349in}}%
\pgfpathlineto{\pgfqpoint{1.358961in}{1.640975in}}%
\pgfpathlineto{\pgfqpoint{1.358021in}{1.642100in}}%
\pgfpathlineto{\pgfqpoint{1.355827in}{1.644726in}}%
\pgfpathlineto{\pgfqpoint{1.352692in}{1.644726in}}%
\pgfpathlineto{\pgfqpoint{1.351751in}{1.645852in}}%
\pgfpathlineto{\pgfqpoint{1.349557in}{1.648478in}}%
\pgfpathlineto{\pgfqpoint{1.348617in}{1.649603in}}%
\pgfpathlineto{\pgfqpoint{1.346422in}{1.652229in}}%
\pgfpathlineto{\pgfqpoint{1.345482in}{1.653355in}}%
\pgfpathlineto{\pgfqpoint{1.343288in}{1.655981in}}%
\pgfpathlineto{\pgfqpoint{1.340153in}{1.655981in}}%
\pgfpathlineto{\pgfqpoint{1.339212in}{1.657106in}}%
\pgfpathlineto{\pgfqpoint{1.337018in}{1.659732in}}%
\pgfpathlineto{\pgfqpoint{1.336078in}{1.660858in}}%
\pgfpathlineto{\pgfqpoint{1.333883in}{1.663484in}}%
\pgfpathlineto{\pgfqpoint{1.332943in}{1.664609in}}%
\pgfpathlineto{\pgfqpoint{1.330749in}{1.667236in}}%
\pgfpathlineto{\pgfqpoint{1.329808in}{1.668361in}}%
\pgfpathlineto{\pgfqpoint{1.327614in}{1.670987in}}%
\pgfpathlineto{\pgfqpoint{1.324479in}{1.670987in}}%
\pgfpathlineto{\pgfqpoint{1.323539in}{1.672113in}}%
\pgfpathlineto{\pgfqpoint{1.321344in}{1.674739in}}%
\pgfpathlineto{\pgfqpoint{1.320404in}{1.675864in}}%
\pgfpathlineto{\pgfqpoint{1.318210in}{1.678490in}}%
\pgfpathlineto{\pgfqpoint{1.317269in}{1.679616in}}%
\pgfpathlineto{\pgfqpoint{1.315075in}{1.682242in}}%
\pgfpathlineto{\pgfqpoint{1.311940in}{1.682242in}}%
\pgfpathlineto{\pgfqpoint{1.311000in}{1.683367in}}%
\pgfpathlineto{\pgfqpoint{1.308805in}{1.685993in}}%
\pgfpathlineto{\pgfqpoint{1.307865in}{1.687119in}}%
\pgfpathlineto{\pgfqpoint{1.305671in}{1.689745in}}%
\pgfpathlineto{\pgfqpoint{1.304730in}{1.690870in}}%
\pgfpathlineto{\pgfqpoint{1.302536in}{1.693496in}}%
\pgfpathlineto{\pgfqpoint{1.299401in}{1.693496in}}%
\pgfpathlineto{\pgfqpoint{1.298461in}{1.694622in}}%
\pgfpathlineto{\pgfqpoint{1.296266in}{1.697248in}}%
\pgfpathlineto{\pgfqpoint{1.295326in}{1.698373in}}%
\pgfpathlineto{\pgfqpoint{1.293132in}{1.700999in}}%
\pgfpathlineto{\pgfqpoint{1.292191in}{1.702125in}}%
\pgfpathlineto{\pgfqpoint{1.292191in}{1.705876in}}%
\pgfpathlineto{\pgfqpoint{1.293132in}{1.707002in}}%
\pgfpathlineto{\pgfqpoint{1.295326in}{1.709628in}}%
\pgfpathlineto{\pgfqpoint{1.295326in}{1.713379in}}%
\pgfpathlineto{\pgfqpoint{1.295326in}{1.717131in}}%
\pgfpathlineto{\pgfqpoint{1.295326in}{1.720882in}}%
\pgfpathlineto{\pgfqpoint{1.296266in}{1.722008in}}%
\pgfpathlineto{\pgfqpoint{1.298461in}{1.724634in}}%
\pgfpathlineto{\pgfqpoint{1.298461in}{1.728386in}}%
\pgfpathlineto{\pgfqpoint{1.298461in}{1.732137in}}%
\pgfpathlineto{\pgfqpoint{1.299401in}{1.733263in}}%
\pgfpathlineto{\pgfqpoint{1.301596in}{1.735889in}}%
\pgfpathlineto{\pgfqpoint{1.301596in}{1.739640in}}%
\pgfpathlineto{\pgfqpoint{1.301596in}{1.743392in}}%
\pgfpathlineto{\pgfqpoint{1.301596in}{1.747143in}}%
\pgfpathlineto{\pgfqpoint{1.302536in}{1.748269in}}%
\pgfpathlineto{\pgfqpoint{1.304730in}{1.750895in}}%
\pgfpathlineto{\pgfqpoint{1.304730in}{1.754646in}}%
\pgfpathlineto{\pgfqpoint{1.304730in}{1.758398in}}%
\pgfpathlineto{\pgfqpoint{1.305671in}{1.759523in}}%
\pgfpathlineto{\pgfqpoint{1.307865in}{1.762149in}}%
\pgfpathlineto{\pgfqpoint{1.307865in}{1.765901in}}%
\pgfpathlineto{\pgfqpoint{1.307865in}{1.769652in}}%
\pgfpathlineto{\pgfqpoint{1.307865in}{1.773404in}}%
\pgfpathlineto{\pgfqpoint{1.308805in}{1.774529in}}%
\pgfpathlineto{\pgfqpoint{1.311000in}{1.777155in}}%
\pgfpathlineto{\pgfqpoint{1.311000in}{1.780907in}}%
\pgfpathlineto{\pgfqpoint{1.311000in}{1.784659in}}%
\pgfpathlineto{\pgfqpoint{1.311940in}{1.785784in}}%
\pgfpathlineto{\pgfqpoint{1.314135in}{1.788410in}}%
\pgfpathlineto{\pgfqpoint{1.314135in}{1.792162in}}%
\pgfpathlineto{\pgfqpoint{1.314135in}{1.795913in}}%
\pgfpathlineto{\pgfqpoint{1.314135in}{1.799665in}}%
\pgfpathlineto{\pgfqpoint{1.315075in}{1.800790in}}%
\pgfpathlineto{\pgfqpoint{1.317269in}{1.803416in}}%
\pgfpathlineto{\pgfqpoint{1.317269in}{1.807168in}}%
\pgfpathlineto{\pgfqpoint{1.317269in}{1.810919in}}%
\pgfpathlineto{\pgfqpoint{1.318210in}{1.812045in}}%
\pgfpathlineto{\pgfqpoint{1.320404in}{1.814671in}}%
\pgfpathlineto{\pgfqpoint{1.320404in}{1.818422in}}%
\pgfpathlineto{\pgfqpoint{1.320404in}{1.822174in}}%
\pgfpathlineto{\pgfqpoint{1.321344in}{1.823299in}}%
\pgfpathlineto{\pgfqpoint{1.323539in}{1.825925in}}%
\pgfpathlineto{\pgfqpoint{1.323539in}{1.829677in}}%
\pgfpathlineto{\pgfqpoint{1.323539in}{1.833429in}}%
\pgfpathlineto{\pgfqpoint{1.323539in}{1.837180in}}%
\pgfpathlineto{\pgfqpoint{1.324479in}{1.838306in}}%
\pgfpathlineto{\pgfqpoint{1.326673in}{1.840932in}}%
\pgfpathlineto{\pgfqpoint{1.326673in}{1.844683in}}%
\pgfpathlineto{\pgfqpoint{1.326673in}{1.848435in}}%
\pgfpathlineto{\pgfqpoint{1.327614in}{1.849560in}}%
\pgfpathlineto{\pgfqpoint{1.329808in}{1.852186in}}%
\pgfpathlineto{\pgfqpoint{1.329808in}{1.855938in}}%
\pgfpathlineto{\pgfqpoint{1.329808in}{1.859689in}}%
\pgfpathlineto{\pgfqpoint{1.329808in}{1.863441in}}%
\pgfpathlineto{\pgfqpoint{1.330749in}{1.864566in}}%
\pgfpathlineto{\pgfqpoint{1.332943in}{1.867192in}}%
\pgfpathlineto{\pgfqpoint{1.332943in}{1.870944in}}%
\pgfpathlineto{\pgfqpoint{1.332943in}{1.874695in}}%
\pgfpathlineto{\pgfqpoint{1.333883in}{1.875821in}}%
\pgfpathlineto{\pgfqpoint{1.336078in}{1.878447in}}%
\pgfpathlineto{\pgfqpoint{1.336078in}{1.882198in}}%
\pgfpathlineto{\pgfqpoint{1.336078in}{1.885950in}}%
\pgfpathlineto{\pgfqpoint{1.336078in}{1.889702in}}%
\pgfpathlineto{\pgfqpoint{1.337018in}{1.890827in}}%
\pgfpathlineto{\pgfqpoint{1.339212in}{1.893453in}}%
\pgfpathlineto{\pgfqpoint{1.339212in}{1.897205in}}%
\pgfpathlineto{\pgfqpoint{1.339212in}{1.900956in}}%
\pgfpathlineto{\pgfqpoint{1.340153in}{1.902082in}}%
\pgfpathlineto{\pgfqpoint{1.342347in}{1.904708in}}%
\pgfpathlineto{\pgfqpoint{1.342347in}{1.908459in}}%
\pgfpathlineto{\pgfqpoint{1.342347in}{1.912211in}}%
\pgfpathlineto{\pgfqpoint{1.342347in}{1.915962in}}%
\pgfpathlineto{\pgfqpoint{1.343288in}{1.917088in}}%
\pgfpathlineto{\pgfqpoint{1.345482in}{1.919714in}}%
\pgfpathlineto{\pgfqpoint{1.345482in}{1.923465in}}%
\pgfpathlineto{\pgfqpoint{1.345482in}{1.927217in}}%
\pgfpathlineto{\pgfqpoint{1.346422in}{1.928342in}}%
\pgfpathlineto{\pgfqpoint{1.348617in}{1.930968in}}%
\pgfpathlineto{\pgfqpoint{1.348617in}{1.934720in}}%
\pgfpathlineto{\pgfqpoint{1.348617in}{1.938471in}}%
\pgfpathlineto{\pgfqpoint{1.348617in}{1.942223in}}%
\pgfpathlineto{\pgfqpoint{1.349557in}{1.943348in}}%
\pgfpathlineto{\pgfqpoint{1.351751in}{1.945975in}}%
\pgfpathlineto{\pgfqpoint{1.351751in}{1.949726in}}%
\pgfpathlineto{\pgfqpoint{1.351751in}{1.953478in}}%
\pgfpathlineto{\pgfqpoint{1.352692in}{1.954603in}}%
\pgfpathlineto{\pgfqpoint{1.354886in}{1.957229in}}%
\pgfpathlineto{\pgfqpoint{1.354886in}{1.960981in}}%
\pgfpathlineto{\pgfqpoint{1.354886in}{1.964732in}}%
\pgfpathlineto{\pgfqpoint{1.354886in}{1.968484in}}%
\pgfpathlineto{\pgfqpoint{1.355827in}{1.969609in}}%
\pgfpathlineto{\pgfqpoint{1.358021in}{1.972235in}}%
\pgfpathlineto{\pgfqpoint{1.358021in}{1.975987in}}%
\pgfpathlineto{\pgfqpoint{1.358021in}{1.979738in}}%
\pgfpathlineto{\pgfqpoint{1.358961in}{1.980864in}}%
\pgfpathlineto{\pgfqpoint{1.361156in}{1.983490in}}%
\pgfpathlineto{\pgfqpoint{1.361156in}{1.987241in}}%
\pgfpathlineto{\pgfqpoint{1.361156in}{1.990993in}}%
\pgfpathlineto{\pgfqpoint{1.361156in}{1.994745in}}%
\pgfpathlineto{\pgfqpoint{1.362096in}{1.995870in}}%
\pgfpathlineto{\pgfqpoint{1.364290in}{1.998496in}}%
\pgfpathlineto{\pgfqpoint{1.364290in}{2.002248in}}%
\pgfpathlineto{\pgfqpoint{1.364290in}{2.005999in}}%
\pgfpathlineto{\pgfqpoint{1.365231in}{2.007125in}}%
\pgfpathlineto{\pgfqpoint{1.367425in}{2.009751in}}%
\pgfpathlineto{\pgfqpoint{1.367425in}{2.013502in}}%
\pgfpathlineto{\pgfqpoint{1.367425in}{2.017254in}}%
\pgfpathlineto{\pgfqpoint{1.367425in}{2.021005in}}%
\pgfpathlineto{\pgfqpoint{1.368366in}{2.022131in}}%
\pgfpathlineto{\pgfqpoint{1.370560in}{2.024757in}}%
\pgfpathlineto{\pgfqpoint{1.370560in}{2.028508in}}%
\pgfpathlineto{\pgfqpoint{1.370560in}{2.032260in}}%
\pgfpathlineto{\pgfqpoint{1.371500in}{2.033385in}}%
\pgfpathlineto{\pgfqpoint{1.373695in}{2.036011in}}%
\pgfpathlineto{\pgfqpoint{1.373695in}{2.039763in}}%
\pgfpathlineto{\pgfqpoint{1.373695in}{2.043514in}}%
\pgfpathlineto{\pgfqpoint{1.374635in}{2.044640in}}%
\pgfpathlineto{\pgfqpoint{1.376829in}{2.047266in}}%
\pgfpathlineto{\pgfqpoint{1.376829in}{2.051018in}}%
\pgfpathlineto{\pgfqpoint{1.376829in}{2.054769in}}%
\pgfpathlineto{\pgfqpoint{1.376829in}{2.058521in}}%
\pgfpathlineto{\pgfqpoint{1.377770in}{2.059646in}}%
\pgfpathlineto{\pgfqpoint{1.379964in}{2.062272in}}%
\pgfpathlineto{\pgfqpoint{1.379964in}{2.066024in}}%
\pgfpathlineto{\pgfqpoint{1.379964in}{2.069775in}}%
\pgfpathlineto{\pgfqpoint{1.380905in}{2.070901in}}%
\pgfpathlineto{\pgfqpoint{1.383099in}{2.073527in}}%
\pgfpathlineto{\pgfqpoint{1.383099in}{2.077278in}}%
\pgfpathlineto{\pgfqpoint{1.383099in}{2.081030in}}%
\pgfpathlineto{\pgfqpoint{1.383099in}{2.084781in}}%
\pgfpathlineto{\pgfqpoint{1.384039in}{2.085907in}}%
\pgfpathlineto{\pgfqpoint{1.386234in}{2.088533in}}%
\pgfpathlineto{\pgfqpoint{1.386234in}{2.092284in}}%
\pgfpathlineto{\pgfqpoint{1.386234in}{2.096036in}}%
\pgfpathlineto{\pgfqpoint{1.387174in}{2.097161in}}%
\pgfpathlineto{\pgfqpoint{1.389368in}{2.099787in}}%
\pgfpathlineto{\pgfqpoint{1.389368in}{2.103539in}}%
\pgfpathlineto{\pgfqpoint{1.389368in}{2.107291in}}%
\pgfpathlineto{\pgfqpoint{1.389368in}{2.111042in}}%
\pgfpathlineto{\pgfqpoint{1.390309in}{2.112168in}}%
\pgfpathlineto{\pgfqpoint{1.392503in}{2.114794in}}%
\pgfpathlineto{\pgfqpoint{1.392503in}{2.118545in}}%
\pgfpathlineto{\pgfqpoint{1.392503in}{2.122297in}}%
\pgfpathlineto{\pgfqpoint{1.393444in}{2.123422in}}%
\pgfpathlineto{\pgfqpoint{1.395638in}{2.126048in}}%
\pgfpathlineto{\pgfqpoint{1.395638in}{2.129800in}}%
\pgfpathlineto{\pgfqpoint{1.395638in}{2.133551in}}%
\pgfpathlineto{\pgfqpoint{1.395638in}{2.137303in}}%
\pgfpathlineto{\pgfqpoint{1.396578in}{2.138428in}}%
\pgfpathlineto{\pgfqpoint{1.398773in}{2.141054in}}%
\pgfpathlineto{\pgfqpoint{1.398773in}{2.144806in}}%
\pgfpathlineto{\pgfqpoint{1.398773in}{2.148557in}}%
\pgfpathlineto{\pgfqpoint{1.399713in}{2.149683in}}%
\pgfpathlineto{\pgfqpoint{1.401907in}{2.152309in}}%
\pgfpathlineto{\pgfqpoint{1.401907in}{2.156060in}}%
\pgfpathlineto{\pgfqpoint{1.401907in}{2.159812in}}%
\pgfpathlineto{\pgfqpoint{1.401907in}{2.163564in}}%
\pgfpathlineto{\pgfqpoint{1.402848in}{2.164689in}}%
\pgfpathlineto{\pgfqpoint{1.405042in}{2.167315in}}%
\pgfpathlineto{\pgfqpoint{1.405042in}{2.171067in}}%
\pgfpathlineto{\pgfqpoint{1.405042in}{2.174818in}}%
\pgfpathlineto{\pgfqpoint{1.405982in}{2.175944in}}%
\pgfpathlineto{\pgfqpoint{1.408177in}{2.178570in}}%
\pgfpathlineto{\pgfqpoint{1.408177in}{2.182321in}}%
\pgfpathlineto{\pgfqpoint{1.408177in}{2.186073in}}%
\pgfpathlineto{\pgfqpoint{1.408177in}{2.189824in}}%
\pgfpathlineto{\pgfqpoint{1.409117in}{2.190950in}}%
\pgfpathlineto{\pgfqpoint{1.411312in}{2.193576in}}%
\pgfpathlineto{\pgfqpoint{1.411312in}{2.197327in}}%
\pgfpathlineto{\pgfqpoint{1.411312in}{2.201079in}}%
\pgfpathlineto{\pgfqpoint{1.412252in}{2.202204in}}%
\pgfpathlineto{\pgfqpoint{1.414446in}{2.204830in}}%
\pgfpathlineto{\pgfqpoint{1.414446in}{2.208582in}}%
\pgfpathlineto{\pgfqpoint{1.414446in}{2.212334in}}%
\pgfpathlineto{\pgfqpoint{1.414446in}{2.216085in}}%
\pgfpathlineto{\pgfqpoint{1.415387in}{2.217210in}}%
\pgfpathlineto{\pgfqpoint{1.417581in}{2.219837in}}%
\pgfpathlineto{\pgfqpoint{1.417581in}{2.223588in}}%
\pgfpathlineto{\pgfqpoint{1.417581in}{2.227340in}}%
\pgfpathlineto{\pgfqpoint{1.418521in}{2.228465in}}%
\pgfpathlineto{\pgfqpoint{1.420716in}{2.231091in}}%
\pgfpathlineto{\pgfqpoint{1.420716in}{2.234843in}}%
\pgfpathlineto{\pgfqpoint{1.420716in}{2.238594in}}%
\pgfpathlineto{\pgfqpoint{1.420716in}{2.242346in}}%
\pgfpathlineto{\pgfqpoint{1.421656in}{2.243471in}}%
\pgfpathlineto{\pgfqpoint{1.423851in}{2.246097in}}%
\pgfpathlineto{\pgfqpoint{1.423851in}{2.249849in}}%
\pgfpathlineto{\pgfqpoint{1.423851in}{2.253600in}}%
\pgfpathlineto{\pgfqpoint{1.424791in}{2.254726in}}%
\pgfpathlineto{\pgfqpoint{1.426985in}{2.257352in}}%
\pgfpathlineto{\pgfqpoint{1.426985in}{2.261103in}}%
\pgfpathlineto{\pgfqpoint{1.426985in}{2.264855in}}%
\pgfpathlineto{\pgfqpoint{1.427926in}{2.265980in}}%
\pgfpathlineto{\pgfqpoint{1.430120in}{2.268607in}}%
\pgfpathlineto{\pgfqpoint{1.430120in}{2.272358in}}%
\pgfpathlineto{\pgfqpoint{1.430120in}{2.276110in}}%
\pgfpathlineto{\pgfqpoint{1.430120in}{2.279861in}}%
\pgfpathlineto{\pgfqpoint{1.431060in}{2.280987in}}%
\pgfpathlineto{\pgfqpoint{1.433255in}{2.283613in}}%
\pgfpathlineto{\pgfqpoint{1.433255in}{2.287364in}}%
\pgfpathlineto{\pgfqpoint{1.433255in}{2.291116in}}%
\pgfpathlineto{\pgfqpoint{1.434195in}{2.292241in}}%
\pgfpathlineto{\pgfqpoint{1.436389in}{2.294867in}}%
\pgfpathlineto{\pgfqpoint{1.436389in}{2.298619in}}%
\pgfpathlineto{\pgfqpoint{1.436389in}{2.302370in}}%
\pgfpathlineto{\pgfqpoint{1.436389in}{2.306122in}}%
\pgfpathlineto{\pgfqpoint{1.437330in}{2.307247in}}%
\pgfpathlineto{\pgfqpoint{1.439524in}{2.309873in}}%
\pgfpathlineto{\pgfqpoint{1.439524in}{2.313625in}}%
\pgfpathlineto{\pgfqpoint{1.439524in}{2.317376in}}%
\pgfpathlineto{\pgfqpoint{1.440465in}{2.318502in}}%
\pgfpathlineto{\pgfqpoint{1.442659in}{2.321128in}}%
\pgfpathlineto{\pgfqpoint{1.442659in}{2.324880in}}%
\pgfpathlineto{\pgfqpoint{1.442659in}{2.328631in}}%
\pgfpathlineto{\pgfqpoint{1.442659in}{2.332383in}}%
\pgfpathlineto{\pgfqpoint{1.443599in}{2.333508in}}%
\pgfpathlineto{\pgfqpoint{1.445794in}{2.336134in}}%
\pgfpathlineto{\pgfqpoint{1.445794in}{2.339886in}}%
\pgfpathlineto{\pgfqpoint{1.445794in}{2.343637in}}%
\pgfpathlineto{\pgfqpoint{1.446734in}{2.344763in}}%
\pgfpathlineto{\pgfqpoint{1.448928in}{2.347389in}}%
\pgfpathlineto{\pgfqpoint{1.448928in}{2.351140in}}%
\pgfpathlineto{\pgfqpoint{1.448928in}{2.354892in}}%
\pgfpathlineto{\pgfqpoint{1.448928in}{2.358643in}}%
\pgfpathlineto{\pgfqpoint{1.449869in}{2.359769in}}%
\pgfpathlineto{\pgfqpoint{1.452063in}{2.362395in}}%
\pgfpathlineto{\pgfqpoint{1.452063in}{2.366146in}}%
\pgfpathlineto{\pgfqpoint{1.452063in}{2.369898in}}%
\pgfpathlineto{\pgfqpoint{1.453004in}{2.371023in}}%
\pgfpathlineto{\pgfqpoint{1.455198in}{2.373649in}}%
\pgfpathlineto{\pgfqpoint{1.455198in}{2.377401in}}%
\pgfpathlineto{\pgfqpoint{1.455198in}{2.381153in}}%
\pgfpathlineto{\pgfqpoint{1.455198in}{2.384904in}}%
\pgfpathlineto{\pgfqpoint{1.456138in}{2.386030in}}%
\pgfpathlineto{\pgfqpoint{1.458333in}{2.388656in}}%
\pgfpathlineto{\pgfqpoint{1.458333in}{2.392407in}}%
\pgfpathlineto{\pgfqpoint{1.458333in}{2.396159in}}%
\pgfpathlineto{\pgfqpoint{1.459273in}{2.397284in}}%
\pgfpathlineto{\pgfqpoint{1.461467in}{2.399910in}}%
\pgfpathlineto{\pgfqpoint{1.461467in}{2.403662in}}%
\pgfpathlineto{\pgfqpoint{1.462408in}{2.404787in}}%
\pgfpathlineto{\pgfqpoint{1.465543in}{2.404787in}}%
\pgfpathlineto{\pgfqpoint{1.467737in}{2.407413in}}%
\pgfpathlineto{\pgfqpoint{1.468677in}{2.408539in}}%
\pgfpathlineto{\pgfqpoint{1.471812in}{2.408539in}}%
\pgfpathlineto{\pgfqpoint{1.474947in}{2.408539in}}%
\pgfpathlineto{\pgfqpoint{1.478082in}{2.408539in}}%
\pgfpathlineto{\pgfqpoint{1.480276in}{2.411165in}}%
\pgfpathlineto{\pgfqpoint{1.481216in}{2.412290in}}%
\pgfpathlineto{\pgfqpoint{1.484351in}{2.412290in}}%
\pgfpathlineto{\pgfqpoint{1.487486in}{2.412290in}}%
\pgfpathlineto{\pgfqpoint{1.490621in}{2.412290in}}%
\pgfpathlineto{\pgfqpoint{1.492815in}{2.414916in}}%
\pgfpathlineto{\pgfqpoint{1.493755in}{2.416042in}}%
\pgfpathlineto{\pgfqpoint{1.496890in}{2.416042in}}%
\pgfpathlineto{\pgfqpoint{1.500025in}{2.416042in}}%
\pgfpathlineto{\pgfqpoint{1.503159in}{2.416042in}}%
\pgfpathlineto{\pgfqpoint{1.505354in}{2.418668in}}%
\pgfpathlineto{\pgfqpoint{1.506294in}{2.419793in}}%
\pgfpathlineto{\pgfqpoint{1.509429in}{2.419793in}}%
\pgfpathlineto{\pgfqpoint{1.512564in}{2.419793in}}%
\pgfpathlineto{\pgfqpoint{1.515698in}{2.419793in}}%
\pgfpathlineto{\pgfqpoint{1.517893in}{2.422419in}}%
\pgfpathlineto{\pgfqpoint{1.518833in}{2.423545in}}%
\pgfpathlineto{\pgfqpoint{1.521968in}{2.423545in}}%
\pgfpathlineto{\pgfqpoint{1.525103in}{2.423545in}}%
\pgfpathlineto{\pgfqpoint{1.528237in}{2.423545in}}%
\pgfpathlineto{\pgfqpoint{1.531372in}{2.423545in}}%
\pgfpathlineto{\pgfqpoint{1.533566in}{2.426171in}}%
\pgfpathlineto{\pgfqpoint{1.534507in}{2.427296in}}%
\pgfpathlineto{\pgfqpoint{1.537642in}{2.427296in}}%
\pgfpathlineto{\pgfqpoint{1.540776in}{2.427296in}}%
\pgfpathlineto{\pgfqpoint{1.543911in}{2.427296in}}%
\pgfpathlineto{\pgfqpoint{1.546105in}{2.429923in}}%
\pgfpathlineto{\pgfqpoint{1.547046in}{2.431048in}}%
\pgfpathlineto{\pgfqpoint{1.550181in}{2.431048in}}%
\pgfpathlineto{\pgfqpoint{1.553315in}{2.431048in}}%
\pgfpathlineto{\pgfqpoint{1.556450in}{2.431048in}}%
\pgfpathlineto{\pgfqpoint{1.558644in}{2.433674in}}%
\pgfpathlineto{\pgfqpoint{1.559585in}{2.434800in}}%
\pgfpathlineto{\pgfqpoint{1.562720in}{2.434800in}}%
\pgfpathlineto{\pgfqpoint{1.565854in}{2.434800in}}%
\pgfpathlineto{\pgfqpoint{1.568989in}{2.434800in}}%
\pgfpathlineto{\pgfqpoint{1.571183in}{2.437426in}}%
\pgfpathlineto{\pgfqpoint{1.572124in}{2.438551in}}%
\pgfpathlineto{\pgfqpoint{1.575259in}{2.438551in}}%
\pgfpathlineto{\pgfqpoint{1.578393in}{2.438551in}}%
\pgfpathlineto{\pgfqpoint{1.581528in}{2.438551in}}%
\pgfpathlineto{\pgfqpoint{1.584663in}{2.438551in}}%
\pgfpathlineto{\pgfqpoint{1.586857in}{2.441177in}}%
\pgfpathlineto{\pgfqpoint{1.587798in}{2.442303in}}%
\pgfpathlineto{\pgfqpoint{1.590932in}{2.442303in}}%
\pgfpathlineto{\pgfqpoint{1.594067in}{2.442303in}}%
\pgfpathlineto{\pgfqpoint{1.597202in}{2.442303in}}%
\pgfpathlineto{\pgfqpoint{1.599396in}{2.444929in}}%
\pgfpathlineto{\pgfqpoint{1.600337in}{2.446054in}}%
\pgfpathlineto{\pgfqpoint{1.603471in}{2.446054in}}%
\pgfpathlineto{\pgfqpoint{1.606606in}{2.446054in}}%
\pgfpathlineto{\pgfqpoint{1.609741in}{2.446054in}}%
\pgfpathlineto{\pgfqpoint{1.611935in}{2.448680in}}%
\pgfpathlineto{\pgfqpoint{1.612875in}{2.449806in}}%
\pgfpathlineto{\pgfqpoint{1.616010in}{2.449806in}}%
\pgfpathlineto{\pgfqpoint{1.619145in}{2.449806in}}%
\pgfpathlineto{\pgfqpoint{1.622280in}{2.449806in}}%
\pgfpathlineto{\pgfqpoint{1.624474in}{2.452432in}}%
\pgfpathlineto{\pgfqpoint{1.625414in}{2.453557in}}%
\pgfpathlineto{\pgfqpoint{1.628549in}{2.453557in}}%
\pgfpathlineto{\pgfqpoint{1.631684in}{2.453557in}}%
\pgfpathlineto{\pgfqpoint{1.634819in}{2.453557in}}%
\pgfpathlineto{\pgfqpoint{1.637953in}{2.453557in}}%
\pgfpathlineto{\pgfqpoint{1.640148in}{2.456183in}}%
\pgfpathlineto{\pgfqpoint{1.641088in}{2.457309in}}%
\pgfpathlineto{\pgfqpoint{1.644223in}{2.457309in}}%
\pgfpathlineto{\pgfqpoint{1.647358in}{2.457309in}}%
\pgfpathlineto{\pgfqpoint{1.650492in}{2.457309in}}%
\pgfpathlineto{\pgfqpoint{1.652687in}{2.459935in}}%
\pgfpathlineto{\pgfqpoint{1.653627in}{2.461060in}}%
\pgfpathlineto{\pgfqpoint{1.656762in}{2.461060in}}%
\pgfpathlineto{\pgfqpoint{1.659897in}{2.461060in}}%
\pgfpathlineto{\pgfqpoint{1.663031in}{2.461060in}}%
\pgfpathlineto{\pgfqpoint{1.665226in}{2.463686in}}%
\pgfpathlineto{\pgfqpoint{1.666166in}{2.464812in}}%
\pgfpathlineto{\pgfqpoint{1.669301in}{2.464812in}}%
\pgfpathlineto{\pgfqpoint{1.672436in}{2.464812in}}%
\pgfpathlineto{\pgfqpoint{1.675570in}{2.464812in}}%
\pgfpathlineto{\pgfqpoint{1.677765in}{2.467438in}}%
\pgfpathlineto{\pgfqpoint{1.678705in}{2.468563in}}%
\pgfpathlineto{\pgfqpoint{1.681840in}{2.468563in}}%
\pgfpathlineto{\pgfqpoint{1.684975in}{2.468563in}}%
\pgfpathlineto{\pgfqpoint{1.688109in}{2.468563in}}%
\pgfpathlineto{\pgfqpoint{1.690304in}{2.471189in}}%
\pgfpathlineto{\pgfqpoint{1.691244in}{2.472315in}}%
\pgfpathlineto{\pgfqpoint{1.694379in}{2.472315in}}%
\pgfpathlineto{\pgfqpoint{1.697514in}{2.472315in}}%
\pgfpathlineto{\pgfqpoint{1.700648in}{2.472315in}}%
\pgfpathlineto{\pgfqpoint{1.703783in}{2.472315in}}%
\pgfpathlineto{\pgfqpoint{1.705977in}{2.474941in}}%
\pgfpathlineto{\pgfqpoint{1.706918in}{2.476066in}}%
\pgfpathlineto{\pgfqpoint{1.710052in}{2.476066in}}%
\pgfpathlineto{\pgfqpoint{1.713187in}{2.476066in}}%
\pgfpathlineto{\pgfqpoint{1.716322in}{2.476066in}}%
\pgfpathlineto{\pgfqpoint{1.718516in}{2.478692in}}%
\pgfpathlineto{\pgfqpoint{1.719457in}{2.479818in}}%
\pgfpathlineto{\pgfqpoint{1.722591in}{2.479818in}}%
\pgfpathlineto{\pgfqpoint{1.725726in}{2.479818in}}%
\pgfpathlineto{\pgfqpoint{1.728861in}{2.479818in}}%
\pgfpathlineto{\pgfqpoint{1.731055in}{2.482444in}}%
\pgfpathlineto{\pgfqpoint{1.731996in}{2.483569in}}%
\pgfpathlineto{\pgfqpoint{1.735130in}{2.483569in}}%
\pgfpathlineto{\pgfqpoint{1.738265in}{2.483569in}}%
\pgfpathlineto{\pgfqpoint{1.741400in}{2.483569in}}%
\pgfpathlineto{\pgfqpoint{1.743594in}{2.486196in}}%
\pgfpathlineto{\pgfqpoint{1.744535in}{2.487321in}}%
\pgfpathlineto{\pgfqpoint{1.747669in}{2.487321in}}%
\pgfpathlineto{\pgfqpoint{1.750804in}{2.487321in}}%
\pgfpathlineto{\pgfqpoint{1.753939in}{2.487321in}}%
\pgfpathlineto{\pgfqpoint{1.757074in}{2.487321in}}%
\pgfpathlineto{\pgfqpoint{1.759268in}{2.489947in}}%
\pgfpathlineto{\pgfqpoint{1.760208in}{2.491073in}}%
\pgfpathlineto{\pgfqpoint{1.763343in}{2.491073in}}%
\pgfpathlineto{\pgfqpoint{1.766478in}{2.491073in}}%
\pgfpathlineto{\pgfqpoint{1.769613in}{2.491073in}}%
\pgfpathlineto{\pgfqpoint{1.771807in}{2.493699in}}%
\pgfpathlineto{\pgfqpoint{1.772747in}{2.494824in}}%
\pgfpathlineto{\pgfqpoint{1.775882in}{2.494824in}}%
\pgfpathlineto{\pgfqpoint{1.779017in}{2.494824in}}%
\pgfpathlineto{\pgfqpoint{1.782152in}{2.494824in}}%
\pgfpathlineto{\pgfqpoint{1.784346in}{2.497450in}}%
\pgfpathlineto{\pgfqpoint{1.785286in}{2.498576in}}%
\pgfpathlineto{\pgfqpoint{1.788421in}{2.498576in}}%
\pgfpathlineto{\pgfqpoint{1.791556in}{2.498576in}}%
\pgfpathlineto{\pgfqpoint{1.794691in}{2.498576in}}%
\pgfpathlineto{\pgfqpoint{1.796885in}{2.501202in}}%
\pgfpathlineto{\pgfqpoint{1.797825in}{2.502327in}}%
\pgfpathlineto{\pgfqpoint{1.800960in}{2.502327in}}%
\pgfpathlineto{\pgfqpoint{1.804095in}{2.502327in}}%
\pgfpathlineto{\pgfqpoint{1.807229in}{2.502327in}}%
\pgfpathlineto{\pgfqpoint{1.810364in}{2.502327in}}%
\pgfpathlineto{\pgfqpoint{1.812559in}{2.504953in}}%
\pgfpathlineto{\pgfqpoint{1.813499in}{2.506079in}}%
\pgfpathlineto{\pgfqpoint{1.816634in}{2.506079in}}%
\pgfpathlineto{\pgfqpoint{1.819768in}{2.506079in}}%
\pgfpathlineto{\pgfqpoint{1.822903in}{2.506079in}}%
\pgfpathlineto{\pgfqpoint{1.825098in}{2.508705in}}%
\pgfpathlineto{\pgfqpoint{1.826038in}{2.509830in}}%
\pgfpathlineto{\pgfqpoint{1.829173in}{2.509830in}}%
\pgfpathlineto{\pgfqpoint{1.832307in}{2.509830in}}%
\pgfpathlineto{\pgfqpoint{1.835442in}{2.509830in}}%
\pgfpathlineto{\pgfqpoint{1.837636in}{2.512456in}}%
\pgfpathlineto{\pgfqpoint{1.838577in}{2.513582in}}%
\pgfpathlineto{\pgfqpoint{1.841712in}{2.513582in}}%
\pgfpathlineto{\pgfqpoint{1.844846in}{2.513582in}}%
\pgfpathlineto{\pgfqpoint{1.847981in}{2.513582in}}%
\pgfpathlineto{\pgfqpoint{1.850175in}{2.516208in}}%
\pgfpathlineto{\pgfqpoint{1.851116in}{2.517333in}}%
\pgfpathlineto{\pgfqpoint{1.854251in}{2.517333in}}%
\pgfpathlineto{\pgfqpoint{1.857385in}{2.517333in}}%
\pgfpathlineto{\pgfqpoint{1.860520in}{2.517333in}}%
\pgfpathlineto{\pgfqpoint{1.862714in}{2.519959in}}%
\pgfpathlineto{\pgfqpoint{1.863655in}{2.521085in}}%
\pgfpathlineto{\pgfqpoint{1.866790in}{2.521085in}}%
\pgfpathlineto{\pgfqpoint{1.869924in}{2.521085in}}%
\pgfpathlineto{\pgfqpoint{1.873059in}{2.521085in}}%
\pgfpathlineto{\pgfqpoint{1.876194in}{2.521085in}}%
\pgfpathlineto{\pgfqpoint{1.878388in}{2.523711in}}%
\pgfpathlineto{\pgfqpoint{1.879329in}{2.524836in}}%
\pgfpathlineto{\pgfqpoint{1.882463in}{2.524836in}}%
\pgfpathlineto{\pgfqpoint{1.885598in}{2.524836in}}%
\pgfpathlineto{\pgfqpoint{1.888733in}{2.524836in}}%
\pgfpathlineto{\pgfqpoint{1.890927in}{2.527462in}}%
\pgfpathlineto{\pgfqpoint{1.891868in}{2.528588in}}%
\pgfpathlineto{\pgfqpoint{1.895002in}{2.528588in}}%
\pgfpathlineto{\pgfqpoint{1.898137in}{2.528588in}}%
\pgfpathlineto{\pgfqpoint{1.901272in}{2.528588in}}%
\pgfpathlineto{\pgfqpoint{1.903466in}{2.531214in}}%
\pgfpathlineto{\pgfqpoint{1.904407in}{2.532339in}}%
\pgfpathlineto{\pgfqpoint{1.907541in}{2.532339in}}%
\pgfpathlineto{\pgfqpoint{1.910676in}{2.532339in}}%
\pgfpathlineto{\pgfqpoint{1.913811in}{2.532339in}}%
\pgfpathlineto{\pgfqpoint{1.916005in}{2.534965in}}%
\pgfpathlineto{\pgfqpoint{1.916945in}{2.536091in}}%
\pgfpathlineto{\pgfqpoint{1.920080in}{2.536091in}}%
\pgfpathlineto{\pgfqpoint{1.923215in}{2.536091in}}%
\pgfpathlineto{\pgfqpoint{1.926350in}{2.536091in}}%
\pgfpathlineto{\pgfqpoint{1.929484in}{2.536091in}}%
\pgfpathlineto{\pgfqpoint{1.931679in}{2.538717in}}%
\pgfpathlineto{\pgfqpoint{1.932619in}{2.539842in}}%
\pgfpathlineto{\pgfqpoint{1.935754in}{2.539842in}}%
\pgfpathlineto{\pgfqpoint{1.938889in}{2.539842in}}%
\pgfpathlineto{\pgfqpoint{1.942023in}{2.539842in}}%
\pgfpathlineto{\pgfqpoint{1.944218in}{2.542469in}}%
\pgfpathlineto{\pgfqpoint{1.945158in}{2.543594in}}%
\pgfpathlineto{\pgfqpoint{1.948293in}{2.543594in}}%
\pgfpathlineto{\pgfqpoint{1.951428in}{2.543594in}}%
\pgfpathlineto{\pgfqpoint{1.954562in}{2.543594in}}%
\pgfpathlineto{\pgfqpoint{1.956757in}{2.546220in}}%
\pgfpathlineto{\pgfqpoint{1.957697in}{2.547346in}}%
\pgfpathlineto{\pgfqpoint{1.960832in}{2.547346in}}%
\pgfpathlineto{\pgfqpoint{1.963967in}{2.547346in}}%
\pgfpathlineto{\pgfqpoint{1.967101in}{2.547346in}}%
\pgfpathlineto{\pgfqpoint{1.969296in}{2.549972in}}%
\pgfpathlineto{\pgfqpoint{1.970236in}{2.551097in}}%
\pgfpathlineto{\pgfqpoint{1.973371in}{2.551097in}}%
\pgfpathlineto{\pgfqpoint{1.976506in}{2.551097in}}%
\pgfpathlineto{\pgfqpoint{1.979640in}{2.551097in}}%
\pgfpathlineto{\pgfqpoint{1.981835in}{2.553723in}}%
\pgfpathlineto{\pgfqpoint{1.982775in}{2.554849in}}%
\pgfpathlineto{\pgfqpoint{1.985910in}{2.554849in}}%
\pgfpathlineto{\pgfqpoint{1.989045in}{2.554849in}}%
\pgfpathlineto{\pgfqpoint{1.992179in}{2.554849in}}%
\pgfpathlineto{\pgfqpoint{1.995314in}{2.554849in}}%
\pgfpathlineto{\pgfqpoint{1.997508in}{2.557475in}}%
\pgfpathlineto{\pgfqpoint{1.998449in}{2.558600in}}%
\pgfpathlineto{\pgfqpoint{2.001584in}{2.558600in}}%
\pgfpathlineto{\pgfqpoint{2.004718in}{2.558600in}}%
\pgfpathlineto{\pgfqpoint{2.007853in}{2.558600in}}%
\pgfpathlineto{\pgfqpoint{2.010047in}{2.561226in}}%
\pgfpathlineto{\pgfqpoint{2.010988in}{2.562352in}}%
\pgfpathlineto{\pgfqpoint{2.014122in}{2.562352in}}%
\pgfpathlineto{\pgfqpoint{2.017257in}{2.562352in}}%
\pgfpathlineto{\pgfqpoint{2.020392in}{2.562352in}}%
\pgfpathlineto{\pgfqpoint{2.022586in}{2.564978in}}%
\pgfpathlineto{\pgfqpoint{2.023527in}{2.566103in}}%
\pgfpathlineto{\pgfqpoint{2.026661in}{2.566103in}}%
\pgfpathlineto{\pgfqpoint{2.029796in}{2.566103in}}%
\pgfpathlineto{\pgfqpoint{2.032931in}{2.566103in}}%
\pgfpathlineto{\pgfqpoint{2.035125in}{2.568729in}}%
\pgfpathlineto{\pgfqpoint{2.036066in}{2.569855in}}%
\pgfpathlineto{\pgfqpoint{2.039200in}{2.569855in}}%
\pgfpathlineto{\pgfqpoint{2.042335in}{2.569855in}}%
\pgfpathlineto{\pgfqpoint{2.045470in}{2.569855in}}%
\pgfpathlineto{\pgfqpoint{2.048605in}{2.569855in}}%
\pgfpathlineto{\pgfqpoint{2.050799in}{2.572481in}}%
\pgfpathlineto{\pgfqpoint{2.051739in}{2.573606in}}%
\pgfpathlineto{\pgfqpoint{2.054874in}{2.573606in}}%
\pgfpathlineto{\pgfqpoint{2.058009in}{2.573606in}}%
\pgfpathlineto{\pgfqpoint{2.061144in}{2.573606in}}%
\pgfpathlineto{\pgfqpoint{2.063338in}{2.576232in}}%
\pgfpathlineto{\pgfqpoint{2.064278in}{2.577358in}}%
\pgfpathlineto{\pgfqpoint{2.067413in}{2.577358in}}%
\pgfpathlineto{\pgfqpoint{2.070548in}{2.577358in}}%
\pgfpathlineto{\pgfqpoint{2.073683in}{2.577358in}}%
\pgfpathlineto{\pgfqpoint{2.075877in}{2.579984in}}%
\pgfpathlineto{\pgfqpoint{2.076817in}{2.581109in}}%
\pgfpathlineto{\pgfqpoint{2.079952in}{2.581109in}}%
\pgfpathlineto{\pgfqpoint{2.083087in}{2.581109in}}%
\pgfpathlineto{\pgfqpoint{2.086222in}{2.581109in}}%
\pgfpathlineto{\pgfqpoint{2.088416in}{2.583735in}}%
\pgfpathlineto{\pgfqpoint{2.089356in}{2.584861in}}%
\pgfpathlineto{\pgfqpoint{2.092491in}{2.584861in}}%
\pgfpathlineto{\pgfqpoint{2.095626in}{2.584861in}}%
\pgfpathlineto{\pgfqpoint{2.098761in}{2.584861in}}%
\pgfpathlineto{\pgfqpoint{2.101895in}{2.584861in}}%
\pgfpathlineto{\pgfqpoint{2.104090in}{2.587487in}}%
\pgfpathlineto{\pgfqpoint{2.105030in}{2.588612in}}%
\pgfpathlineto{\pgfqpoint{2.108165in}{2.588612in}}%
\pgfpathlineto{\pgfqpoint{2.111299in}{2.588612in}}%
\pgfpathlineto{\pgfqpoint{2.114434in}{2.588612in}}%
\pgfpathlineto{\pgfqpoint{2.116629in}{2.591238in}}%
\pgfpathlineto{\pgfqpoint{2.117569in}{2.592364in}}%
\pgfpathlineto{\pgfqpoint{2.120704in}{2.592364in}}%
\pgfpathlineto{\pgfqpoint{2.123838in}{2.592364in}}%
\pgfpathlineto{\pgfqpoint{2.126973in}{2.592364in}}%
\pgfpathlineto{\pgfqpoint{2.129168in}{2.594990in}}%
\pgfpathlineto{\pgfqpoint{2.130108in}{2.596115in}}%
\pgfpathlineto{\pgfqpoint{2.133243in}{2.596115in}}%
\pgfpathlineto{\pgfqpoint{2.136377in}{2.596115in}}%
\pgfpathlineto{\pgfqpoint{2.139512in}{2.596115in}}%
\pgfpathlineto{\pgfqpoint{2.141706in}{2.598742in}}%
\pgfpathlineto{\pgfqpoint{2.142647in}{2.599867in}}%
\pgfpathlineto{\pgfqpoint{2.145782in}{2.599867in}}%
\pgfpathlineto{\pgfqpoint{2.148916in}{2.599867in}}%
\pgfpathlineto{\pgfqpoint{2.152051in}{2.599867in}}%
\pgfpathlineto{\pgfqpoint{2.154245in}{2.602493in}}%
\pgfpathlineto{\pgfqpoint{2.155186in}{2.603619in}}%
\pgfpathlineto{\pgfqpoint{2.158321in}{2.603619in}}%
\pgfpathlineto{\pgfqpoint{2.161455in}{2.603619in}}%
\pgfpathlineto{\pgfqpoint{2.164590in}{2.603619in}}%
\pgfpathlineto{\pgfqpoint{2.167725in}{2.603619in}}%
\pgfpathlineto{\pgfqpoint{2.169919in}{2.606245in}}%
\pgfpathlineto{\pgfqpoint{2.170860in}{2.607370in}}%
\pgfpathlineto{\pgfqpoint{2.173994in}{2.607370in}}%
\pgfpathlineto{\pgfqpoint{2.177129in}{2.607370in}}%
\pgfpathlineto{\pgfqpoint{2.180264in}{2.607370in}}%
\pgfpathlineto{\pgfqpoint{2.182458in}{2.609996in}}%
\pgfpathlineto{\pgfqpoint{2.183399in}{2.611122in}}%
\pgfpathlineto{\pgfqpoint{2.186533in}{2.611122in}}%
\pgfpathlineto{\pgfqpoint{2.189668in}{2.611122in}}%
\pgfpathlineto{\pgfqpoint{2.192803in}{2.611122in}}%
\pgfpathlineto{\pgfqpoint{2.194997in}{2.613748in}}%
\pgfpathlineto{\pgfqpoint{2.195938in}{2.614873in}}%
\pgfpathlineto{\pgfqpoint{2.199072in}{2.614873in}}%
\pgfpathlineto{\pgfqpoint{2.202207in}{2.614873in}}%
\pgfpathlineto{\pgfqpoint{2.205342in}{2.614873in}}%
\pgfpathlineto{\pgfqpoint{2.207536in}{2.617499in}}%
\pgfpathlineto{\pgfqpoint{2.208477in}{2.618625in}}%
\pgfpathlineto{\pgfqpoint{2.211611in}{2.618625in}}%
\pgfpathlineto{\pgfqpoint{2.214746in}{2.618625in}}%
\pgfpathlineto{\pgfqpoint{2.217881in}{2.618625in}}%
\pgfpathlineto{\pgfqpoint{2.221015in}{2.618625in}}%
\pgfpathlineto{\pgfqpoint{2.223210in}{2.621251in}}%
\pgfpathlineto{\pgfqpoint{2.224150in}{2.622376in}}%
\pgfpathlineto{\pgfqpoint{2.227285in}{2.622376in}}%
\pgfpathlineto{\pgfqpoint{2.230420in}{2.622376in}}%
\pgfpathlineto{\pgfqpoint{2.233554in}{2.622376in}}%
\pgfpathlineto{\pgfqpoint{2.235749in}{2.625002in}}%
\pgfpathlineto{\pgfqpoint{2.236689in}{2.626128in}}%
\pgfpathlineto{\pgfqpoint{2.239824in}{2.626128in}}%
\pgfpathlineto{\pgfqpoint{2.242959in}{2.626128in}}%
\pgfpathlineto{\pgfqpoint{2.246093in}{2.626128in}}%
\pgfpathlineto{\pgfqpoint{2.248288in}{2.628754in}}%
\pgfpathlineto{\pgfqpoint{2.249228in}{2.629879in}}%
\pgfpathlineto{\pgfqpoint{2.252363in}{2.629879in}}%
\pgfpathlineto{\pgfqpoint{2.255498in}{2.629879in}}%
\pgfpathlineto{\pgfqpoint{2.258632in}{2.629879in}}%
\pgfpathlineto{\pgfqpoint{2.260827in}{2.632505in}}%
\pgfpathlineto{\pgfqpoint{2.261767in}{2.633631in}}%
\pgfpathlineto{\pgfqpoint{2.264902in}{2.633631in}}%
\pgfpathlineto{\pgfqpoint{2.268037in}{2.633631in}}%
\pgfpathlineto{\pgfqpoint{2.271171in}{2.633631in}}%
\pgfpathlineto{\pgfqpoint{2.274306in}{2.633631in}}%
\pgfpathlineto{\pgfqpoint{2.276500in}{2.636257in}}%
\pgfpathlineto{\pgfqpoint{2.277441in}{2.637382in}}%
\pgfpathlineto{\pgfqpoint{2.280576in}{2.637382in}}%
\pgfpathlineto{\pgfqpoint{2.283710in}{2.637382in}}%
\pgfpathlineto{\pgfqpoint{2.286845in}{2.637382in}}%
\pgfpathlineto{\pgfqpoint{2.289039in}{2.640008in}}%
\pgfpathlineto{\pgfqpoint{2.289980in}{2.641134in}}%
\pgfpathlineto{\pgfqpoint{2.293115in}{2.641134in}}%
\pgfpathlineto{\pgfqpoint{2.296249in}{2.641134in}}%
\pgfpathlineto{\pgfqpoint{2.299384in}{2.641134in}}%
\pgfpathlineto{\pgfqpoint{2.301578in}{2.643760in}}%
\pgfpathlineto{\pgfqpoint{2.302519in}{2.644885in}}%
\pgfpathlineto{\pgfqpoint{2.305654in}{2.644885in}}%
\pgfpathlineto{\pgfqpoint{2.308788in}{2.644885in}}%
\pgfpathlineto{\pgfqpoint{2.311923in}{2.644885in}}%
\pgfpathlineto{\pgfqpoint{2.314117in}{2.647512in}}%
\pgfpathlineto{\pgfqpoint{2.315058in}{2.648637in}}%
\pgfpathlineto{\pgfqpoint{2.318192in}{2.648637in}}%
\pgfpathlineto{\pgfqpoint{2.321327in}{2.648637in}}%
\pgfpathlineto{\pgfqpoint{2.324462in}{2.648637in}}%
\pgfpathlineto{\pgfqpoint{2.326656in}{2.651263in}}%
\pgfpathlineto{\pgfqpoint{2.327597in}{2.652389in}}%
\pgfpathlineto{\pgfqpoint{2.330731in}{2.652389in}}%
\pgfpathlineto{\pgfqpoint{2.333866in}{2.652389in}}%
\pgfpathlineto{\pgfqpoint{2.337001in}{2.652389in}}%
\pgfpathlineto{\pgfqpoint{2.340136in}{2.652389in}}%
\pgfpathlineto{\pgfqpoint{2.342330in}{2.655015in}}%
\pgfpathlineto{\pgfqpoint{2.343270in}{2.656140in}}%
\pgfpathlineto{\pgfqpoint{2.346405in}{2.656140in}}%
\pgfpathlineto{\pgfqpoint{2.349540in}{2.656140in}}%
\pgfpathlineto{\pgfqpoint{2.352675in}{2.656140in}}%
\pgfpathlineto{\pgfqpoint{2.354869in}{2.658766in}}%
\pgfpathlineto{\pgfqpoint{2.355809in}{2.659892in}}%
\pgfpathlineto{\pgfqpoint{2.358944in}{2.659892in}}%
\pgfpathlineto{\pgfqpoint{2.362079in}{2.659892in}}%
\pgfpathlineto{\pgfqpoint{2.365214in}{2.659892in}}%
\pgfpathlineto{\pgfqpoint{2.367408in}{2.662518in}}%
\pgfpathlineto{\pgfqpoint{2.368348in}{2.663643in}}%
\pgfpathlineto{\pgfqpoint{2.371483in}{2.663643in}}%
\pgfpathlineto{\pgfqpoint{2.374618in}{2.663643in}}%
\pgfpathlineto{\pgfqpoint{2.377753in}{2.663643in}}%
\pgfpathlineto{\pgfqpoint{2.379947in}{2.666269in}}%
\pgfpathlineto{\pgfqpoint{2.380887in}{2.667395in}}%
\pgfpathlineto{\pgfqpoint{2.384022in}{2.667395in}}%
\pgfpathlineto{\pgfqpoint{2.387157in}{2.667395in}}%
\pgfpathlineto{\pgfqpoint{2.390292in}{2.667395in}}%
\pgfpathlineto{\pgfqpoint{2.393426in}{2.667395in}}%
\pgfpathlineto{\pgfqpoint{2.395621in}{2.670021in}}%
\pgfpathlineto{\pgfqpoint{2.396561in}{2.671146in}}%
\pgfpathlineto{\pgfqpoint{2.399696in}{2.671146in}}%
\pgfpathlineto{\pgfqpoint{2.402831in}{2.671146in}}%
\pgfpathlineto{\pgfqpoint{2.405965in}{2.671146in}}%
\pgfpathlineto{\pgfqpoint{2.408160in}{2.673772in}}%
\pgfpathlineto{\pgfqpoint{2.409100in}{2.674898in}}%
\pgfpathlineto{\pgfqpoint{2.412235in}{2.674898in}}%
\pgfpathlineto{\pgfqpoint{2.415369in}{2.674898in}}%
\pgfpathlineto{\pgfqpoint{2.418504in}{2.674898in}}%
\pgfpathlineto{\pgfqpoint{2.420699in}{2.677524in}}%
\pgfpathlineto{\pgfqpoint{2.421639in}{2.678649in}}%
\pgfpathlineto{\pgfqpoint{2.424774in}{2.678649in}}%
\pgfpathlineto{\pgfqpoint{2.427908in}{2.678649in}}%
\pgfpathlineto{\pgfqpoint{2.431043in}{2.678649in}}%
\pgfpathlineto{\pgfqpoint{2.433238in}{2.681275in}}%
\pgfpathlineto{\pgfqpoint{2.434178in}{2.682401in}}%
\pgfpathlineto{\pgfqpoint{2.437313in}{2.682401in}}%
\pgfpathlineto{\pgfqpoint{2.440447in}{2.682401in}}%
\pgfpathlineto{\pgfqpoint{2.443582in}{2.682401in}}%
\pgfpathlineto{\pgfqpoint{2.446717in}{2.682401in}}%
\pgfpathlineto{\pgfqpoint{2.448911in}{2.685027in}}%
\pgfpathlineto{\pgfqpoint{2.449852in}{2.686152in}}%
\pgfpathlineto{\pgfqpoint{2.452986in}{2.686152in}}%
\pgfpathlineto{\pgfqpoint{2.456121in}{2.686152in}}%
\pgfpathlineto{\pgfqpoint{2.459256in}{2.686152in}}%
\pgfpathlineto{\pgfqpoint{2.461450in}{2.688778in}}%
\pgfpathlineto{\pgfqpoint{2.462391in}{2.689904in}}%
\pgfpathlineto{\pgfqpoint{2.465525in}{2.689904in}}%
\pgfpathlineto{\pgfqpoint{2.468660in}{2.689904in}}%
\pgfpathlineto{\pgfqpoint{2.471795in}{2.689904in}}%
\pgfpathlineto{\pgfqpoint{2.473989in}{2.692530in}}%
\pgfpathlineto{\pgfqpoint{2.474930in}{2.693655in}}%
\pgfpathlineto{\pgfqpoint{2.478064in}{2.693655in}}%
\pgfpathlineto{\pgfqpoint{2.481199in}{2.693655in}}%
\pgfpathlineto{\pgfqpoint{2.484334in}{2.693655in}}%
\pgfpathlineto{\pgfqpoint{2.486528in}{2.696281in}}%
\pgfpathlineto{\pgfqpoint{2.487469in}{2.697407in}}%
\pgfpathlineto{\pgfqpoint{2.490603in}{2.697407in}}%
\pgfpathlineto{\pgfqpoint{2.493738in}{2.697407in}}%
\pgfpathlineto{\pgfqpoint{2.496873in}{2.697407in}}%
\pgfpathlineto{\pgfqpoint{2.499067in}{2.700033in}}%
\pgfpathlineto{\pgfqpoint{2.500008in}{2.701158in}}%
\pgfpathlineto{\pgfqpoint{2.503142in}{2.701158in}}%
\pgfpathlineto{\pgfqpoint{2.506277in}{2.701158in}}%
\pgfpathlineto{\pgfqpoint{2.509412in}{2.701158in}}%
\pgfpathlineto{\pgfqpoint{2.512547in}{2.701158in}}%
\pgfpathlineto{\pgfqpoint{2.514741in}{2.703785in}}%
\pgfpathlineto{\pgfqpoint{2.515681in}{2.704910in}}%
\pgfpathlineto{\pgfqpoint{2.518816in}{2.704910in}}%
\pgfpathlineto{\pgfqpoint{2.521951in}{2.704910in}}%
\pgfpathlineto{\pgfqpoint{2.525085in}{2.704910in}}%
\pgfpathlineto{\pgfqpoint{2.527280in}{2.707536in}}%
\pgfpathlineto{\pgfqpoint{2.528220in}{2.708662in}}%
\pgfpathlineto{\pgfqpoint{2.531355in}{2.708662in}}%
\pgfpathlineto{\pgfqpoint{2.534490in}{2.708662in}}%
\pgfpathlineto{\pgfqpoint{2.537624in}{2.708662in}}%
\pgfpathlineto{\pgfqpoint{2.539819in}{2.711288in}}%
\pgfpathlineto{\pgfqpoint{2.540759in}{2.712413in}}%
\pgfpathlineto{\pgfqpoint{2.543894in}{2.712413in}}%
\pgfpathlineto{\pgfqpoint{2.547029in}{2.712413in}}%
\pgfpathlineto{\pgfqpoint{2.550163in}{2.712413in}}%
\pgfpathlineto{\pgfqpoint{2.552358in}{2.715039in}}%
\pgfpathlineto{\pgfqpoint{2.553298in}{2.716165in}}%
\pgfpathlineto{\pgfqpoint{2.556433in}{2.716165in}}%
\pgfpathlineto{\pgfqpoint{2.559568in}{2.716165in}}%
\pgfpathlineto{\pgfqpoint{2.562702in}{2.716165in}}%
\pgfpathlineto{\pgfqpoint{2.565837in}{2.716165in}}%
\pgfpathlineto{\pgfqpoint{2.568031in}{2.718791in}}%
\pgfpathlineto{\pgfqpoint{2.568972in}{2.719916in}}%
\pgfpathlineto{\pgfqpoint{2.572107in}{2.719916in}}%
\pgfpathlineto{\pgfqpoint{2.575241in}{2.719916in}}%
\pgfpathlineto{\pgfqpoint{2.578376in}{2.719916in}}%
\pgfpathlineto{\pgfqpoint{2.580570in}{2.722542in}}%
\pgfpathlineto{\pgfqpoint{2.581511in}{2.723668in}}%
\pgfpathlineto{\pgfqpoint{2.584646in}{2.723668in}}%
\pgfpathlineto{\pgfqpoint{2.587780in}{2.723668in}}%
\pgfpathlineto{\pgfqpoint{2.590915in}{2.723668in}}%
\pgfpathlineto{\pgfqpoint{2.593109in}{2.726294in}}%
\pgfpathlineto{\pgfqpoint{2.594050in}{2.727419in}}%
\pgfpathlineto{\pgfqpoint{2.597185in}{2.727419in}}%
\pgfpathlineto{\pgfqpoint{2.600319in}{2.727419in}}%
\pgfpathlineto{\pgfqpoint{2.603454in}{2.727419in}}%
\pgfpathlineto{\pgfqpoint{2.605648in}{2.730045in}}%
\pgfpathlineto{\pgfqpoint{2.606589in}{2.731171in}}%
\pgfpathlineto{\pgfqpoint{2.609724in}{2.731171in}}%
\pgfpathlineto{\pgfqpoint{2.612858in}{2.731171in}}%
\pgfpathlineto{\pgfqpoint{2.615993in}{2.731171in}}%
\pgfpathlineto{\pgfqpoint{2.618187in}{2.733797in}}%
\pgfpathlineto{\pgfqpoint{2.619128in}{2.734922in}}%
\pgfpathlineto{\pgfqpoint{2.622262in}{2.734922in}}%
\pgfpathlineto{\pgfqpoint{2.625397in}{2.734922in}}%
\pgfpathlineto{\pgfqpoint{2.628532in}{2.734922in}}%
\pgfpathlineto{\pgfqpoint{2.631667in}{2.734922in}}%
\pgfpathlineto{\pgfqpoint{2.633861in}{2.737548in}}%
\pgfpathlineto{\pgfqpoint{2.634801in}{2.738674in}}%
\pgfpathlineto{\pgfqpoint{2.637936in}{2.738674in}}%
\pgfpathlineto{\pgfqpoint{2.641071in}{2.738674in}}%
\pgfpathlineto{\pgfqpoint{2.644206in}{2.738674in}}%
\pgfpathlineto{\pgfqpoint{2.646400in}{2.741300in}}%
\pgfpathlineto{\pgfqpoint{2.647340in}{2.742425in}}%
\pgfpathlineto{\pgfqpoint{2.650475in}{2.742425in}}%
\pgfpathlineto{\pgfqpoint{2.653610in}{2.742425in}}%
\pgfpathlineto{\pgfqpoint{2.656745in}{2.742425in}}%
\pgfpathlineto{\pgfqpoint{2.658939in}{2.745051in}}%
\pgfpathlineto{\pgfqpoint{2.659879in}{2.746177in}}%
\pgfpathlineto{\pgfqpoint{2.663014in}{2.746177in}}%
\pgfpathlineto{\pgfqpoint{2.666149in}{2.746177in}}%
\pgfpathlineto{\pgfqpoint{2.669284in}{2.746177in}}%
\pgfpathlineto{\pgfqpoint{2.672418in}{2.746177in}}%
\pgfpathlineto{\pgfqpoint{2.673359in}{2.745051in}}%
\pgfpathlineto{\pgfqpoint{2.675553in}{2.742425in}}%
\pgfpathlineto{\pgfqpoint{2.676494in}{2.741300in}}%
\pgfpathlineto{\pgfqpoint{2.678688in}{2.738674in}}%
\pgfpathlineto{\pgfqpoint{2.679628in}{2.737548in}}%
\pgfpathlineto{\pgfqpoint{2.681823in}{2.734922in}}%
\pgfpathlineto{\pgfqpoint{2.682763in}{2.733797in}}%
\pgfpathlineto{\pgfqpoint{2.684957in}{2.731171in}}%
\pgfpathlineto{\pgfqpoint{2.685898in}{2.730045in}}%
\pgfpathlineto{\pgfqpoint{2.688092in}{2.727419in}}%
\pgfpathlineto{\pgfqpoint{2.689032in}{2.726294in}}%
\pgfpathlineto{\pgfqpoint{2.691227in}{2.723668in}}%
\pgfpathlineto{\pgfqpoint{2.692167in}{2.722542in}}%
\pgfpathlineto{\pgfqpoint{2.694362in}{2.719916in}}%
\pgfpathlineto{\pgfqpoint{2.695302in}{2.718791in}}%
\pgfpathlineto{\pgfqpoint{2.697496in}{2.716165in}}%
\pgfpathlineto{\pgfqpoint{2.698437in}{2.715039in}}%
\pgfpathlineto{\pgfqpoint{2.700631in}{2.712413in}}%
\pgfpathlineto{\pgfqpoint{2.701571in}{2.711288in}}%
\pgfpathlineto{\pgfqpoint{2.703766in}{2.708662in}}%
\pgfpathlineto{\pgfqpoint{2.704706in}{2.707536in}}%
\pgfpathlineto{\pgfqpoint{2.706901in}{2.704910in}}%
\pgfpathlineto{\pgfqpoint{2.707841in}{2.703785in}}%
\pgfpathlineto{\pgfqpoint{2.710035in}{2.701158in}}%
\pgfpathlineto{\pgfqpoint{2.710976in}{2.700033in}}%
\pgfpathlineto{\pgfqpoint{2.710976in}{2.696281in}}%
\pgfpathlineto{\pgfqpoint{2.713170in}{2.693655in}}%
\pgfpathlineto{\pgfqpoint{2.714110in}{2.692530in}}%
\pgfpathlineto{\pgfqpoint{2.716305in}{2.689904in}}%
\pgfpathlineto{\pgfqpoint{2.717245in}{2.688778in}}%
\pgfpathlineto{\pgfqpoint{2.719439in}{2.686152in}}%
\pgfpathlineto{\pgfqpoint{2.720380in}{2.685027in}}%
\pgfpathlineto{\pgfqpoint{2.722574in}{2.682401in}}%
\pgfpathlineto{\pgfqpoint{2.723515in}{2.681275in}}%
\pgfpathlineto{\pgfqpoint{2.725709in}{2.678649in}}%
\pgfpathlineto{\pgfqpoint{2.726649in}{2.677524in}}%
\pgfpathlineto{\pgfqpoint{2.728844in}{2.674898in}}%
\pgfpathlineto{\pgfqpoint{2.729784in}{2.673772in}}%
\pgfpathlineto{\pgfqpoint{2.731978in}{2.671146in}}%
\pgfpathlineto{\pgfqpoint{2.732919in}{2.670021in}}%
\pgfpathlineto{\pgfqpoint{2.735113in}{2.667395in}}%
\pgfpathlineto{\pgfqpoint{2.736054in}{2.666269in}}%
\pgfpathlineto{\pgfqpoint{2.738248in}{2.663643in}}%
\pgfpathlineto{\pgfqpoint{2.739188in}{2.662518in}}%
\pgfpathlineto{\pgfqpoint{2.741383in}{2.659892in}}%
\pgfpathlineto{\pgfqpoint{2.742323in}{2.658766in}}%
\pgfpathlineto{\pgfqpoint{2.744517in}{2.656140in}}%
\pgfpathlineto{\pgfqpoint{2.745458in}{2.655015in}}%
\pgfpathlineto{\pgfqpoint{2.747652in}{2.652389in}}%
\pgfpathlineto{\pgfqpoint{2.748593in}{2.651263in}}%
\pgfpathlineto{\pgfqpoint{2.750787in}{2.648637in}}%
\pgfpathlineto{\pgfqpoint{2.751727in}{2.647512in}}%
\pgfpathlineto{\pgfqpoint{2.753922in}{2.644885in}}%
\pgfpathlineto{\pgfqpoint{2.754862in}{2.643760in}}%
\pgfpathlineto{\pgfqpoint{2.757056in}{2.641134in}}%
\pgfpathlineto{\pgfqpoint{2.757997in}{2.640008in}}%
\pgfpathlineto{\pgfqpoint{2.760191in}{2.637382in}}%
\pgfpathlineto{\pgfqpoint{2.761132in}{2.636257in}}%
\pgfpathlineto{\pgfqpoint{2.761132in}{2.632505in}}%
\pgfpathlineto{\pgfqpoint{2.763326in}{2.629879in}}%
\pgfpathlineto{\pgfqpoint{2.764266in}{2.628754in}}%
\pgfpathlineto{\pgfqpoint{2.766461in}{2.626128in}}%
\pgfpathlineto{\pgfqpoint{2.767401in}{2.625002in}}%
\pgfpathlineto{\pgfqpoint{2.769595in}{2.622376in}}%
\pgfpathlineto{\pgfqpoint{2.770536in}{2.621251in}}%
\pgfpathlineto{\pgfqpoint{2.772730in}{2.618625in}}%
\pgfpathlineto{\pgfqpoint{2.773671in}{2.617499in}}%
\pgfpathlineto{\pgfqpoint{2.775865in}{2.614873in}}%
\pgfpathlineto{\pgfqpoint{2.776805in}{2.613748in}}%
\pgfpathlineto{\pgfqpoint{2.779000in}{2.611122in}}%
\pgfpathlineto{\pgfqpoint{2.779940in}{2.609996in}}%
\pgfpathlineto{\pgfqpoint{2.782134in}{2.607370in}}%
\pgfpathlineto{\pgfqpoint{2.783075in}{2.606245in}}%
\pgfpathlineto{\pgfqpoint{2.785269in}{2.603619in}}%
\pgfpathlineto{\pgfqpoint{2.786210in}{2.602493in}}%
\pgfpathlineto{\pgfqpoint{2.788404in}{2.599867in}}%
\pgfpathlineto{\pgfqpoint{2.789344in}{2.598742in}}%
\pgfpathlineto{\pgfqpoint{2.791539in}{2.596115in}}%
\pgfpathlineto{\pgfqpoint{2.792479in}{2.594990in}}%
\pgfpathlineto{\pgfqpoint{2.794673in}{2.592364in}}%
\pgfpathlineto{\pgfqpoint{2.795614in}{2.591238in}}%
\pgfpathlineto{\pgfqpoint{2.797808in}{2.588612in}}%
\pgfpathlineto{\pgfqpoint{2.798748in}{2.587487in}}%
\pgfpathlineto{\pgfqpoint{2.800943in}{2.584861in}}%
\pgfpathlineto{\pgfqpoint{2.801883in}{2.583735in}}%
\pgfpathlineto{\pgfqpoint{2.804078in}{2.581109in}}%
\pgfpathlineto{\pgfqpoint{2.805018in}{2.579984in}}%
\pgfpathlineto{\pgfqpoint{2.807212in}{2.577358in}}%
\pgfpathlineto{\pgfqpoint{2.808153in}{2.576232in}}%
\pgfpathlineto{\pgfqpoint{2.808153in}{2.572481in}}%
\pgfpathlineto{\pgfqpoint{2.810347in}{2.569855in}}%
\pgfpathlineto{\pgfqpoint{2.811287in}{2.568729in}}%
\pgfpathlineto{\pgfqpoint{2.813482in}{2.566103in}}%
\pgfpathlineto{\pgfqpoint{2.814422in}{2.564978in}}%
\pgfpathlineto{\pgfqpoint{2.816617in}{2.562352in}}%
\pgfpathlineto{\pgfqpoint{2.817557in}{2.561226in}}%
\pgfpathlineto{\pgfqpoint{2.819751in}{2.558600in}}%
\pgfpathlineto{\pgfqpoint{2.820692in}{2.557475in}}%
\pgfpathlineto{\pgfqpoint{2.822886in}{2.554849in}}%
\pgfpathlineto{\pgfqpoint{2.823826in}{2.553723in}}%
\pgfpathlineto{\pgfqpoint{2.826021in}{2.551097in}}%
\pgfpathlineto{\pgfqpoint{2.826961in}{2.549972in}}%
\pgfpathlineto{\pgfqpoint{2.829155in}{2.547346in}}%
\pgfpathlineto{\pgfqpoint{2.830096in}{2.546220in}}%
\pgfpathlineto{\pgfqpoint{2.832290in}{2.543594in}}%
\pgfpathlineto{\pgfqpoint{2.833231in}{2.542469in}}%
\pgfpathlineto{\pgfqpoint{2.835425in}{2.539842in}}%
\pgfpathlineto{\pgfqpoint{2.836365in}{2.538717in}}%
\pgfpathlineto{\pgfqpoint{2.838560in}{2.536091in}}%
\pgfpathlineto{\pgfqpoint{2.839500in}{2.534965in}}%
\pgfpathlineto{\pgfqpoint{2.841694in}{2.532339in}}%
\pgfpathlineto{\pgfqpoint{2.842635in}{2.531214in}}%
\pgfpathlineto{\pgfqpoint{2.844829in}{2.528588in}}%
\pgfpathlineto{\pgfqpoint{2.845770in}{2.527462in}}%
\pgfpathlineto{\pgfqpoint{2.847964in}{2.524836in}}%
\pgfpathlineto{\pgfqpoint{2.848904in}{2.523711in}}%
\pgfpathlineto{\pgfqpoint{2.851099in}{2.521085in}}%
\pgfpathlineto{\pgfqpoint{2.852039in}{2.519959in}}%
\pgfpathlineto{\pgfqpoint{2.854233in}{2.517333in}}%
\pgfpathlineto{\pgfqpoint{2.855174in}{2.516208in}}%
\pgfpathlineto{\pgfqpoint{2.857368in}{2.513582in}}%
\pgfpathlineto{\pgfqpoint{2.858309in}{2.512456in}}%
\pgfpathlineto{\pgfqpoint{2.858309in}{2.508705in}}%
\pgfpathlineto{\pgfqpoint{2.860503in}{2.506079in}}%
\pgfpathlineto{\pgfqpoint{2.861443in}{2.504953in}}%
\pgfpathlineto{\pgfqpoint{2.863638in}{2.502327in}}%
\pgfpathlineto{\pgfqpoint{2.864578in}{2.501202in}}%
\pgfpathlineto{\pgfqpoint{2.866772in}{2.498576in}}%
\pgfpathlineto{\pgfqpoint{2.867713in}{2.497450in}}%
\pgfpathlineto{\pgfqpoint{2.869907in}{2.494824in}}%
\pgfpathlineto{\pgfqpoint{2.870848in}{2.493699in}}%
\pgfpathlineto{\pgfqpoint{2.873042in}{2.491073in}}%
\pgfpathlineto{\pgfqpoint{2.873982in}{2.489947in}}%
\pgfpathlineto{\pgfqpoint{2.876177in}{2.487321in}}%
\pgfpathlineto{\pgfqpoint{2.877117in}{2.486196in}}%
\pgfpathlineto{\pgfqpoint{2.879311in}{2.483569in}}%
\pgfpathlineto{\pgfqpoint{2.880252in}{2.482444in}}%
\pgfpathlineto{\pgfqpoint{2.882446in}{2.479818in}}%
\pgfpathlineto{\pgfqpoint{2.883387in}{2.478692in}}%
\pgfpathlineto{\pgfqpoint{2.885581in}{2.476066in}}%
\pgfpathlineto{\pgfqpoint{2.886521in}{2.474941in}}%
\pgfpathlineto{\pgfqpoint{2.888716in}{2.472315in}}%
\pgfpathlineto{\pgfqpoint{2.889656in}{2.471189in}}%
\pgfpathlineto{\pgfqpoint{2.891850in}{2.468563in}}%
\pgfpathlineto{\pgfqpoint{2.892791in}{2.467438in}}%
\pgfpathlineto{\pgfqpoint{2.894985in}{2.464812in}}%
\pgfpathlineto{\pgfqpoint{2.895925in}{2.463686in}}%
\pgfpathlineto{\pgfqpoint{2.898120in}{2.461060in}}%
\pgfpathlineto{\pgfqpoint{2.899060in}{2.459935in}}%
\pgfpathlineto{\pgfqpoint{2.901255in}{2.457309in}}%
\pgfpathlineto{\pgfqpoint{2.902195in}{2.456183in}}%
\pgfpathlineto{\pgfqpoint{2.904389in}{2.453557in}}%
\pgfpathlineto{\pgfqpoint{2.905330in}{2.452432in}}%
\pgfpathlineto{\pgfqpoint{2.905330in}{2.448680in}}%
\pgfpathlineto{\pgfqpoint{2.907524in}{2.446054in}}%
\pgfpathlineto{\pgfqpoint{2.908464in}{2.444929in}}%
\pgfpathlineto{\pgfqpoint{2.910659in}{2.442303in}}%
\pgfpathlineto{\pgfqpoint{2.911599in}{2.441177in}}%
\pgfpathlineto{\pgfqpoint{2.913794in}{2.438551in}}%
\pgfpathlineto{\pgfqpoint{2.914734in}{2.437426in}}%
\pgfpathlineto{\pgfqpoint{2.916928in}{2.434800in}}%
\pgfpathlineto{\pgfqpoint{2.917869in}{2.433674in}}%
\pgfpathlineto{\pgfqpoint{2.920063in}{2.431048in}}%
\pgfpathlineto{\pgfqpoint{2.921003in}{2.429923in}}%
\pgfpathlineto{\pgfqpoint{2.923198in}{2.427296in}}%
\pgfpathlineto{\pgfqpoint{2.924138in}{2.426171in}}%
\pgfpathlineto{\pgfqpoint{2.926332in}{2.423545in}}%
\pgfpathlineto{\pgfqpoint{2.927273in}{2.422419in}}%
\pgfpathlineto{\pgfqpoint{2.929467in}{2.419793in}}%
\pgfpathlineto{\pgfqpoint{2.930408in}{2.418668in}}%
\pgfpathlineto{\pgfqpoint{2.932602in}{2.416042in}}%
\pgfpathlineto{\pgfqpoint{2.933542in}{2.414916in}}%
\pgfpathlineto{\pgfqpoint{2.935737in}{2.412290in}}%
\pgfpathlineto{\pgfqpoint{2.936677in}{2.411165in}}%
\pgfpathlineto{\pgfqpoint{2.938871in}{2.408539in}}%
\pgfpathlineto{\pgfqpoint{2.939812in}{2.407413in}}%
\pgfpathlineto{\pgfqpoint{2.942006in}{2.404787in}}%
\pgfpathlineto{\pgfqpoint{2.942947in}{2.403662in}}%
\pgfpathlineto{\pgfqpoint{2.945141in}{2.401036in}}%
\pgfpathlineto{\pgfqpoint{2.946081in}{2.399910in}}%
\pgfpathlineto{\pgfqpoint{2.948276in}{2.397284in}}%
\pgfpathlineto{\pgfqpoint{2.949216in}{2.396159in}}%
\pgfpathlineto{\pgfqpoint{2.951410in}{2.393533in}}%
\pgfpathlineto{\pgfqpoint{2.952351in}{2.392407in}}%
\pgfpathlineto{\pgfqpoint{2.954545in}{2.389781in}}%
\pgfpathlineto{\pgfqpoint{2.955486in}{2.388656in}}%
\pgfpathlineto{\pgfqpoint{2.955486in}{2.384904in}}%
\pgfpathlineto{\pgfqpoint{2.957680in}{2.382278in}}%
\pgfpathlineto{\pgfqpoint{2.958620in}{2.381153in}}%
\pgfpathlineto{\pgfqpoint{2.960815in}{2.378526in}}%
\pgfpathlineto{\pgfqpoint{2.961755in}{2.377401in}}%
\pgfpathlineto{\pgfqpoint{2.963949in}{2.374775in}}%
\pgfpathlineto{\pgfqpoint{2.964890in}{2.373649in}}%
\pgfpathlineto{\pgfqpoint{2.967084in}{2.371023in}}%
\pgfpathlineto{\pgfqpoint{2.968025in}{2.369898in}}%
\pgfpathlineto{\pgfqpoint{2.970219in}{2.367272in}}%
\pgfpathlineto{\pgfqpoint{2.971159in}{2.366146in}}%
\pgfpathlineto{\pgfqpoint{2.973354in}{2.363520in}}%
\pgfpathlineto{\pgfqpoint{2.974294in}{2.362395in}}%
\pgfpathlineto{\pgfqpoint{2.976488in}{2.359769in}}%
\pgfpathlineto{\pgfqpoint{2.977429in}{2.358643in}}%
\pgfpathlineto{\pgfqpoint{2.979623in}{2.356017in}}%
\pgfpathlineto{\pgfqpoint{2.980564in}{2.354892in}}%
\pgfpathlineto{\pgfqpoint{2.982758in}{2.352266in}}%
\pgfpathlineto{\pgfqpoint{2.983698in}{2.351140in}}%
\pgfpathlineto{\pgfqpoint{2.985893in}{2.348514in}}%
\pgfpathlineto{\pgfqpoint{2.986833in}{2.347389in}}%
\pgfpathlineto{\pgfqpoint{2.989027in}{2.344763in}}%
\pgfpathlineto{\pgfqpoint{2.989968in}{2.343637in}}%
\pgfpathlineto{\pgfqpoint{2.992162in}{2.341011in}}%
\pgfpathlineto{\pgfqpoint{2.993103in}{2.339886in}}%
\pgfpathlineto{\pgfqpoint{2.995297in}{2.337260in}}%
\pgfpathlineto{\pgfqpoint{2.996237in}{2.336134in}}%
\pgfpathlineto{\pgfqpoint{2.998432in}{2.333508in}}%
\pgfpathlineto{\pgfqpoint{2.999372in}{2.332383in}}%
\pgfpathlineto{\pgfqpoint{3.001566in}{2.329757in}}%
\pgfpathlineto{\pgfqpoint{3.002507in}{2.328631in}}%
\pgfpathlineto{\pgfqpoint{3.002507in}{2.324880in}}%
\pgfpathlineto{\pgfqpoint{3.004701in}{2.322253in}}%
\pgfpathlineto{\pgfqpoint{3.005641in}{2.321128in}}%
\pgfpathlineto{\pgfqpoint{3.007836in}{2.318502in}}%
\pgfpathlineto{\pgfqpoint{3.008776in}{2.317376in}}%
\pgfpathlineto{\pgfqpoint{3.010971in}{2.314750in}}%
\pgfpathlineto{\pgfqpoint{3.011911in}{2.313625in}}%
\pgfpathlineto{\pgfqpoint{3.014105in}{2.310999in}}%
\pgfpathlineto{\pgfqpoint{3.015046in}{2.309873in}}%
\pgfpathlineto{\pgfqpoint{3.017240in}{2.307247in}}%
\pgfpathlineto{\pgfqpoint{3.018180in}{2.306122in}}%
\pgfpathlineto{\pgfqpoint{3.020375in}{2.303496in}}%
\pgfpathlineto{\pgfqpoint{3.021315in}{2.302370in}}%
\pgfpathlineto{\pgfqpoint{3.023510in}{2.299744in}}%
\pgfpathlineto{\pgfqpoint{3.024450in}{2.298619in}}%
\pgfpathlineto{\pgfqpoint{3.026644in}{2.295993in}}%
\pgfpathlineto{\pgfqpoint{3.027585in}{2.294867in}}%
\pgfpathlineto{\pgfqpoint{3.029779in}{2.292241in}}%
\pgfpathlineto{\pgfqpoint{3.030719in}{2.291116in}}%
\pgfpathlineto{\pgfqpoint{3.032914in}{2.288490in}}%
\pgfpathlineto{\pgfqpoint{3.033854in}{2.287364in}}%
\pgfpathlineto{\pgfqpoint{3.036048in}{2.284738in}}%
\pgfpathlineto{\pgfqpoint{3.036989in}{2.283613in}}%
\pgfpathlineto{\pgfqpoint{3.039183in}{2.280987in}}%
\pgfpathlineto{\pgfqpoint{3.040124in}{2.279861in}}%
\pgfpathlineto{\pgfqpoint{3.042318in}{2.277235in}}%
\pgfpathlineto{\pgfqpoint{3.043258in}{2.276110in}}%
\pgfpathlineto{\pgfqpoint{3.045453in}{2.273484in}}%
\pgfpathlineto{\pgfqpoint{3.046393in}{2.272358in}}%
\pgfpathlineto{\pgfqpoint{3.048587in}{2.269732in}}%
\pgfpathlineto{\pgfqpoint{3.049528in}{2.268607in}}%
\pgfpathlineto{\pgfqpoint{3.051722in}{2.265980in}}%
\pgfpathlineto{\pgfqpoint{3.052663in}{2.264855in}}%
\pgfpathlineto{\pgfqpoint{3.052663in}{2.261103in}}%
\pgfpathlineto{\pgfqpoint{3.054857in}{2.258477in}}%
\pgfpathlineto{\pgfqpoint{3.055797in}{2.257352in}}%
\pgfpathlineto{\pgfqpoint{3.057992in}{2.254726in}}%
\pgfpathlineto{\pgfqpoint{3.058932in}{2.253600in}}%
\pgfpathlineto{\pgfqpoint{3.061126in}{2.250974in}}%
\pgfpathlineto{\pgfqpoint{3.062067in}{2.249849in}}%
\pgfpathlineto{\pgfqpoint{3.064261in}{2.247223in}}%
\pgfpathlineto{\pgfqpoint{3.065202in}{2.246097in}}%
\pgfpathlineto{\pgfqpoint{3.067396in}{2.243471in}}%
\pgfpathlineto{\pgfqpoint{3.068336in}{2.242346in}}%
\pgfpathlineto{\pgfqpoint{3.070531in}{2.239720in}}%
\pgfpathlineto{\pgfqpoint{3.071471in}{2.238594in}}%
\pgfpathlineto{\pgfqpoint{3.073665in}{2.235968in}}%
\pgfpathlineto{\pgfqpoint{3.074606in}{2.234843in}}%
\pgfpathlineto{\pgfqpoint{3.076800in}{2.232217in}}%
\pgfpathlineto{\pgfqpoint{3.077741in}{2.231091in}}%
\pgfpathlineto{\pgfqpoint{3.079935in}{2.228465in}}%
\pgfpathlineto{\pgfqpoint{3.080875in}{2.227340in}}%
\pgfpathlineto{\pgfqpoint{3.083070in}{2.224714in}}%
\pgfpathlineto{\pgfqpoint{3.084010in}{2.223588in}}%
\pgfpathlineto{\pgfqpoint{3.086204in}{2.220962in}}%
\pgfpathlineto{\pgfqpoint{3.087145in}{2.219837in}}%
\pgfpathlineto{\pgfqpoint{3.089339in}{2.217210in}}%
\pgfpathlineto{\pgfqpoint{3.090280in}{2.216085in}}%
\pgfpathlineto{\pgfqpoint{3.092474in}{2.213459in}}%
\pgfpathlineto{\pgfqpoint{3.093414in}{2.212334in}}%
\pgfpathlineto{\pgfqpoint{3.093414in}{2.208582in}}%
\pgfpathlineto{\pgfqpoint{3.092474in}{2.207457in}}%
\pgfpathlineto{\pgfqpoint{3.090280in}{2.204830in}}%
\pgfpathlineto{\pgfqpoint{3.090280in}{2.201079in}}%
\pgfpathlineto{\pgfqpoint{3.090280in}{2.197327in}}%
\pgfpathlineto{\pgfqpoint{3.089339in}{2.196202in}}%
\pgfpathlineto{\pgfqpoint{3.087145in}{2.193576in}}%
\pgfpathlineto{\pgfqpoint{3.087145in}{2.189824in}}%
\pgfpathlineto{\pgfqpoint{3.087145in}{2.186073in}}%
\pgfpathlineto{\pgfqpoint{3.086204in}{2.184947in}}%
\pgfpathlineto{\pgfqpoint{3.084010in}{2.182321in}}%
\pgfpathlineto{\pgfqpoint{3.084010in}{2.178570in}}%
\pgfpathlineto{\pgfqpoint{3.084010in}{2.174818in}}%
\pgfpathlineto{\pgfqpoint{3.083070in}{2.173693in}}%
\pgfpathlineto{\pgfqpoint{3.080875in}{2.171067in}}%
\pgfpathlineto{\pgfqpoint{3.080875in}{2.167315in}}%
\pgfpathlineto{\pgfqpoint{3.080875in}{2.163564in}}%
\pgfpathlineto{\pgfqpoint{3.079935in}{2.162438in}}%
\pgfpathlineto{\pgfqpoint{3.077741in}{2.159812in}}%
\pgfpathlineto{\pgfqpoint{3.077741in}{2.156060in}}%
\pgfpathlineto{\pgfqpoint{3.077741in}{2.152309in}}%
\pgfpathlineto{\pgfqpoint{3.077741in}{2.148557in}}%
\pgfpathlineto{\pgfqpoint{3.076800in}{2.147432in}}%
\pgfpathlineto{\pgfqpoint{3.074606in}{2.144806in}}%
\pgfpathlineto{\pgfqpoint{3.074606in}{2.141054in}}%
\pgfpathlineto{\pgfqpoint{3.074606in}{2.137303in}}%
\pgfpathlineto{\pgfqpoint{3.073665in}{2.136177in}}%
\pgfpathlineto{\pgfqpoint{3.071471in}{2.133551in}}%
\pgfpathlineto{\pgfqpoint{3.071471in}{2.129800in}}%
\pgfpathlineto{\pgfqpoint{3.071471in}{2.126048in}}%
\pgfpathlineto{\pgfqpoint{3.070531in}{2.124923in}}%
\pgfpathlineto{\pgfqpoint{3.068336in}{2.122297in}}%
\pgfpathlineto{\pgfqpoint{3.068336in}{2.118545in}}%
\pgfpathlineto{\pgfqpoint{3.068336in}{2.114794in}}%
\pgfpathlineto{\pgfqpoint{3.067396in}{2.113668in}}%
\pgfpathlineto{\pgfqpoint{3.065202in}{2.111042in}}%
\pgfpathlineto{\pgfqpoint{3.065202in}{2.107291in}}%
\pgfpathlineto{\pgfqpoint{3.065202in}{2.103539in}}%
\pgfpathlineto{\pgfqpoint{3.064261in}{2.102414in}}%
\pgfpathlineto{\pgfqpoint{3.062067in}{2.099787in}}%
\pgfpathlineto{\pgfqpoint{3.062067in}{2.096036in}}%
\pgfpathlineto{\pgfqpoint{3.062067in}{2.092284in}}%
\pgfpathlineto{\pgfqpoint{3.061126in}{2.091159in}}%
\pgfpathlineto{\pgfqpoint{3.058932in}{2.088533in}}%
\pgfpathlineto{\pgfqpoint{3.058932in}{2.084781in}}%
\pgfpathlineto{\pgfqpoint{3.058932in}{2.081030in}}%
\pgfpathlineto{\pgfqpoint{3.057992in}{2.079904in}}%
\pgfpathlineto{\pgfqpoint{3.055797in}{2.077278in}}%
\pgfpathlineto{\pgfqpoint{3.055797in}{2.073527in}}%
\pgfpathlineto{\pgfqpoint{3.055797in}{2.069775in}}%
\pgfpathlineto{\pgfqpoint{3.054857in}{2.068650in}}%
\pgfpathlineto{\pgfqpoint{3.052663in}{2.066024in}}%
\pgfpathlineto{\pgfqpoint{3.052663in}{2.062272in}}%
\pgfpathlineto{\pgfqpoint{3.052663in}{2.058521in}}%
\pgfpathlineto{\pgfqpoint{3.051722in}{2.057395in}}%
\pgfpathlineto{\pgfqpoint{3.049528in}{2.054769in}}%
\pgfpathlineto{\pgfqpoint{3.049528in}{2.051018in}}%
\pgfpathlineto{\pgfqpoint{3.049528in}{2.047266in}}%
\pgfpathlineto{\pgfqpoint{3.049528in}{2.043514in}}%
\pgfpathlineto{\pgfqpoint{3.048587in}{2.042389in}}%
\pgfpathlineto{\pgfqpoint{3.046393in}{2.039763in}}%
\pgfpathlineto{\pgfqpoint{3.046393in}{2.036011in}}%
\pgfpathlineto{\pgfqpoint{3.046393in}{2.032260in}}%
\pgfpathlineto{\pgfqpoint{3.045453in}{2.031134in}}%
\pgfpathlineto{\pgfqpoint{3.043258in}{2.028508in}}%
\pgfpathlineto{\pgfqpoint{3.043258in}{2.024757in}}%
\pgfpathlineto{\pgfqpoint{3.043258in}{2.021005in}}%
\pgfpathlineto{\pgfqpoint{3.042318in}{2.019880in}}%
\pgfpathlineto{\pgfqpoint{3.040124in}{2.017254in}}%
\pgfpathlineto{\pgfqpoint{3.040124in}{2.013502in}}%
\pgfpathlineto{\pgfqpoint{3.040124in}{2.009751in}}%
\pgfpathlineto{\pgfqpoint{3.039183in}{2.008625in}}%
\pgfpathlineto{\pgfqpoint{3.036989in}{2.005999in}}%
\pgfpathlineto{\pgfqpoint{3.036989in}{2.002248in}}%
\pgfpathlineto{\pgfqpoint{3.036989in}{1.998496in}}%
\pgfpathlineto{\pgfqpoint{3.036048in}{1.997371in}}%
\pgfpathlineto{\pgfqpoint{3.033854in}{1.994745in}}%
\pgfpathlineto{\pgfqpoint{3.033854in}{1.990993in}}%
\pgfpathlineto{\pgfqpoint{3.033854in}{1.987241in}}%
\pgfpathlineto{\pgfqpoint{3.032914in}{1.986116in}}%
\pgfpathlineto{\pgfqpoint{3.030719in}{1.983490in}}%
\pgfpathlineto{\pgfqpoint{3.030719in}{1.979738in}}%
\pgfpathlineto{\pgfqpoint{3.030719in}{1.975987in}}%
\pgfpathlineto{\pgfqpoint{3.029779in}{1.974861in}}%
\pgfpathlineto{\pgfqpoint{3.027585in}{1.972235in}}%
\pgfpathlineto{\pgfqpoint{3.027585in}{1.968484in}}%
\pgfpathlineto{\pgfqpoint{3.027585in}{1.964732in}}%
\pgfpathlineto{\pgfqpoint{3.026644in}{1.963607in}}%
\pgfpathlineto{\pgfqpoint{3.024450in}{1.960981in}}%
\pgfpathlineto{\pgfqpoint{3.024450in}{1.957229in}}%
\pgfpathlineto{\pgfqpoint{3.024450in}{1.953478in}}%
\pgfpathlineto{\pgfqpoint{3.024450in}{1.949726in}}%
\pgfpathlineto{\pgfqpoint{3.023510in}{1.948601in}}%
\pgfpathlineto{\pgfqpoint{3.021315in}{1.945975in}}%
\pgfpathlineto{\pgfqpoint{3.021315in}{1.942223in}}%
\pgfpathlineto{\pgfqpoint{3.021315in}{1.938471in}}%
\pgfpathlineto{\pgfqpoint{3.020375in}{1.937346in}}%
\pgfpathlineto{\pgfqpoint{3.018180in}{1.934720in}}%
\pgfpathlineto{\pgfqpoint{3.018180in}{1.930968in}}%
\pgfpathlineto{\pgfqpoint{3.018180in}{1.927217in}}%
\pgfpathlineto{\pgfqpoint{3.017240in}{1.926091in}}%
\pgfpathlineto{\pgfqpoint{3.015046in}{1.923465in}}%
\pgfpathlineto{\pgfqpoint{3.015046in}{1.919714in}}%
\pgfpathlineto{\pgfqpoint{3.015046in}{1.915962in}}%
\pgfpathlineto{\pgfqpoint{3.014105in}{1.914837in}}%
\pgfpathlineto{\pgfqpoint{3.011911in}{1.912211in}}%
\pgfpathlineto{\pgfqpoint{3.011911in}{1.908459in}}%
\pgfpathlineto{\pgfqpoint{3.011911in}{1.904708in}}%
\pgfpathlineto{\pgfqpoint{3.010971in}{1.903582in}}%
\pgfpathlineto{\pgfqpoint{3.008776in}{1.900956in}}%
\pgfpathlineto{\pgfqpoint{3.008776in}{1.897205in}}%
\pgfpathlineto{\pgfqpoint{3.008776in}{1.893453in}}%
\pgfpathlineto{\pgfqpoint{3.007836in}{1.892328in}}%
\pgfpathlineto{\pgfqpoint{3.005641in}{1.889702in}}%
\pgfpathlineto{\pgfqpoint{3.005641in}{1.885950in}}%
\pgfpathlineto{\pgfqpoint{3.005641in}{1.882198in}}%
\pgfpathlineto{\pgfqpoint{3.004701in}{1.881073in}}%
\pgfpathlineto{\pgfqpoint{3.002507in}{1.878447in}}%
\pgfpathlineto{\pgfqpoint{3.002507in}{1.874695in}}%
\pgfpathlineto{\pgfqpoint{3.002507in}{1.870944in}}%
\pgfpathlineto{\pgfqpoint{3.001566in}{1.869818in}}%
\pgfpathlineto{\pgfqpoint{2.999372in}{1.867192in}}%
\pgfpathlineto{\pgfqpoint{2.999372in}{1.863441in}}%
\pgfpathlineto{\pgfqpoint{2.999372in}{1.859689in}}%
\pgfpathlineto{\pgfqpoint{2.998432in}{1.858564in}}%
\pgfpathlineto{\pgfqpoint{2.996237in}{1.855938in}}%
\pgfpathlineto{\pgfqpoint{2.996237in}{1.852186in}}%
\pgfpathlineto{\pgfqpoint{2.996237in}{1.848435in}}%
\pgfpathlineto{\pgfqpoint{2.996237in}{1.844683in}}%
\pgfpathlineto{\pgfqpoint{2.995297in}{1.843558in}}%
\pgfpathlineto{\pgfqpoint{2.993103in}{1.840932in}}%
\pgfpathlineto{\pgfqpoint{2.993103in}{1.837180in}}%
\pgfpathlineto{\pgfqpoint{2.993103in}{1.833429in}}%
\pgfpathlineto{\pgfqpoint{2.992162in}{1.832303in}}%
\pgfpathlineto{\pgfqpoint{2.989968in}{1.829677in}}%
\pgfpathlineto{\pgfqpoint{2.989968in}{1.825925in}}%
\pgfpathlineto{\pgfqpoint{2.989968in}{1.822174in}}%
\pgfpathlineto{\pgfqpoint{2.989027in}{1.821048in}}%
\pgfpathlineto{\pgfqpoint{2.986833in}{1.818422in}}%
\pgfpathlineto{\pgfqpoint{2.986833in}{1.814671in}}%
\pgfpathlineto{\pgfqpoint{2.986833in}{1.810919in}}%
\pgfpathlineto{\pgfqpoint{2.985893in}{1.809794in}}%
\pgfpathlineto{\pgfqpoint{2.983698in}{1.807168in}}%
\pgfpathlineto{\pgfqpoint{2.983698in}{1.803416in}}%
\pgfpathlineto{\pgfqpoint{2.983698in}{1.799665in}}%
\pgfpathlineto{\pgfqpoint{2.982758in}{1.798539in}}%
\pgfpathlineto{\pgfqpoint{2.980564in}{1.795913in}}%
\pgfpathlineto{\pgfqpoint{2.980564in}{1.792162in}}%
\pgfpathlineto{\pgfqpoint{2.980564in}{1.788410in}}%
\pgfpathlineto{\pgfqpoint{2.979623in}{1.787285in}}%
\pgfpathlineto{\pgfqpoint{2.977429in}{1.784659in}}%
\pgfpathlineto{\pgfqpoint{2.977429in}{1.780907in}}%
\pgfpathlineto{\pgfqpoint{2.977429in}{1.777155in}}%
\pgfpathlineto{\pgfqpoint{2.976488in}{1.776030in}}%
\pgfpathlineto{\pgfqpoint{2.974294in}{1.773404in}}%
\pgfpathlineto{\pgfqpoint{2.974294in}{1.769652in}}%
\pgfpathlineto{\pgfqpoint{2.974294in}{1.765901in}}%
\pgfpathlineto{\pgfqpoint{2.973354in}{1.764775in}}%
\pgfpathlineto{\pgfqpoint{2.971159in}{1.762149in}}%
\pgfpathlineto{\pgfqpoint{2.971159in}{1.758398in}}%
\pgfpathlineto{\pgfqpoint{2.971159in}{1.754646in}}%
\pgfpathlineto{\pgfqpoint{2.971159in}{1.750895in}}%
\pgfpathlineto{\pgfqpoint{2.970219in}{1.749769in}}%
\pgfpathlineto{\pgfqpoint{2.968025in}{1.747143in}}%
\pgfpathlineto{\pgfqpoint{2.968025in}{1.743392in}}%
\pgfpathlineto{\pgfqpoint{2.968025in}{1.739640in}}%
\pgfpathlineto{\pgfqpoint{2.967084in}{1.738515in}}%
\pgfpathlineto{\pgfqpoint{2.964890in}{1.735889in}}%
\pgfpathlineto{\pgfqpoint{2.964890in}{1.732137in}}%
\pgfpathlineto{\pgfqpoint{2.964890in}{1.728386in}}%
\pgfpathlineto{\pgfqpoint{2.963949in}{1.727260in}}%
\pgfpathlineto{\pgfqpoint{2.961755in}{1.724634in}}%
\pgfpathlineto{\pgfqpoint{2.961755in}{1.720882in}}%
\pgfpathlineto{\pgfqpoint{2.961755in}{1.717131in}}%
\pgfpathlineto{\pgfqpoint{2.960815in}{1.716005in}}%
\pgfpathlineto{\pgfqpoint{2.958620in}{1.713379in}}%
\pgfpathlineto{\pgfqpoint{2.958620in}{1.709628in}}%
\pgfpathlineto{\pgfqpoint{2.958620in}{1.705876in}}%
\pgfpathlineto{\pgfqpoint{2.957680in}{1.704751in}}%
\pgfpathlineto{\pgfqpoint{2.955486in}{1.702125in}}%
\pgfpathlineto{\pgfqpoint{2.955486in}{1.698373in}}%
\pgfpathlineto{\pgfqpoint{2.955486in}{1.694622in}}%
\pgfpathlineto{\pgfqpoint{2.954545in}{1.693496in}}%
\pgfpathlineto{\pgfqpoint{2.952351in}{1.690870in}}%
\pgfpathlineto{\pgfqpoint{2.952351in}{1.687119in}}%
\pgfpathlineto{\pgfqpoint{2.952351in}{1.683367in}}%
\pgfpathlineto{\pgfqpoint{2.951410in}{1.682242in}}%
\pgfpathlineto{\pgfqpoint{2.949216in}{1.679616in}}%
\pgfpathlineto{\pgfqpoint{2.949216in}{1.675864in}}%
\pgfpathlineto{\pgfqpoint{2.949216in}{1.672113in}}%
\pgfpathlineto{\pgfqpoint{2.948276in}{1.670987in}}%
\pgfpathlineto{\pgfqpoint{2.946081in}{1.668361in}}%
\pgfpathlineto{\pgfqpoint{2.946081in}{1.664609in}}%
\pgfpathlineto{\pgfqpoint{2.946081in}{1.660858in}}%
\pgfpathlineto{\pgfqpoint{2.945141in}{1.659732in}}%
\pgfpathlineto{\pgfqpoint{2.942947in}{1.657106in}}%
\pgfpathlineto{\pgfqpoint{2.942947in}{1.653355in}}%
\pgfpathlineto{\pgfqpoint{2.942947in}{1.649603in}}%
\pgfpathlineto{\pgfqpoint{2.942947in}{1.645852in}}%
\pgfpathlineto{\pgfqpoint{2.942006in}{1.644726in}}%
\pgfpathlineto{\pgfqpoint{2.939812in}{1.642100in}}%
\pgfpathlineto{\pgfqpoint{2.939812in}{1.638349in}}%
\pgfpathlineto{\pgfqpoint{2.939812in}{1.634597in}}%
\pgfpathlineto{\pgfqpoint{2.938871in}{1.633472in}}%
\pgfpathlineto{\pgfqpoint{2.936677in}{1.630846in}}%
\pgfpathlineto{\pgfqpoint{2.936677in}{1.627094in}}%
\pgfpathlineto{\pgfqpoint{2.936677in}{1.623343in}}%
\pgfpathlineto{\pgfqpoint{2.935737in}{1.622217in}}%
\pgfpathlineto{\pgfqpoint{2.933542in}{1.619591in}}%
\pgfpathlineto{\pgfqpoint{2.933542in}{1.615840in}}%
\pgfpathlineto{\pgfqpoint{2.933542in}{1.612088in}}%
\pgfpathlineto{\pgfqpoint{2.932602in}{1.610963in}}%
\pgfpathlineto{\pgfqpoint{2.930408in}{1.608336in}}%
\pgfpathlineto{\pgfqpoint{2.930408in}{1.604585in}}%
\pgfpathlineto{\pgfqpoint{2.930408in}{1.600833in}}%
\pgfpathlineto{\pgfqpoint{2.929467in}{1.599708in}}%
\pgfpathlineto{\pgfqpoint{2.927273in}{1.597082in}}%
\pgfpathlineto{\pgfqpoint{2.927273in}{1.593330in}}%
\pgfpathlineto{\pgfqpoint{2.927273in}{1.589579in}}%
\pgfpathlineto{\pgfqpoint{2.926332in}{1.588453in}}%
\pgfpathlineto{\pgfqpoint{2.924138in}{1.585827in}}%
\pgfpathlineto{\pgfqpoint{2.924138in}{1.582076in}}%
\pgfpathlineto{\pgfqpoint{2.924138in}{1.578324in}}%
\pgfpathlineto{\pgfqpoint{2.923198in}{1.577199in}}%
\pgfpathlineto{\pgfqpoint{2.921003in}{1.574573in}}%
\pgfpathlineto{\pgfqpoint{2.921003in}{1.570821in}}%
\pgfpathlineto{\pgfqpoint{2.921003in}{1.567070in}}%
\pgfpathlineto{\pgfqpoint{2.920063in}{1.565944in}}%
\pgfpathlineto{\pgfqpoint{2.917869in}{1.563318in}}%
\pgfpathlineto{\pgfqpoint{2.917869in}{1.559566in}}%
\pgfpathlineto{\pgfqpoint{2.917869in}{1.555815in}}%
\pgfpathlineto{\pgfqpoint{2.917869in}{1.552063in}}%
\pgfpathlineto{\pgfqpoint{2.916928in}{1.550938in}}%
\pgfpathlineto{\pgfqpoint{2.914734in}{1.548312in}}%
\pgfpathlineto{\pgfqpoint{2.914734in}{1.544560in}}%
\pgfpathlineto{\pgfqpoint{2.914734in}{1.540809in}}%
\pgfpathlineto{\pgfqpoint{2.913794in}{1.539683in}}%
\pgfpathlineto{\pgfqpoint{2.911599in}{1.537057in}}%
\pgfpathlineto{\pgfqpoint{2.911599in}{1.533306in}}%
\pgfpathlineto{\pgfqpoint{2.911599in}{1.529554in}}%
\pgfpathlineto{\pgfqpoint{2.910659in}{1.528429in}}%
\pgfpathlineto{\pgfqpoint{2.908464in}{1.525803in}}%
\pgfpathlineto{\pgfqpoint{2.908464in}{1.522051in}}%
\pgfpathlineto{\pgfqpoint{2.908464in}{1.518300in}}%
\pgfpathlineto{\pgfqpoint{2.907524in}{1.517174in}}%
\pgfpathlineto{\pgfqpoint{2.905330in}{1.514548in}}%
\pgfpathlineto{\pgfqpoint{2.905330in}{1.510797in}}%
\pgfpathlineto{\pgfqpoint{2.905330in}{1.507045in}}%
\pgfpathlineto{\pgfqpoint{2.904389in}{1.505920in}}%
\pgfpathlineto{\pgfqpoint{2.902195in}{1.503293in}}%
\pgfpathlineto{\pgfqpoint{2.902195in}{1.499542in}}%
\pgfpathlineto{\pgfqpoint{2.902195in}{1.495790in}}%
\pgfpathlineto{\pgfqpoint{2.901255in}{1.494665in}}%
\pgfpathlineto{\pgfqpoint{2.899060in}{1.492039in}}%
\pgfpathlineto{\pgfqpoint{2.899060in}{1.488287in}}%
\pgfpathlineto{\pgfqpoint{2.899060in}{1.484536in}}%
\pgfpathlineto{\pgfqpoint{2.898120in}{1.483410in}}%
\pgfpathlineto{\pgfqpoint{2.895925in}{1.480784in}}%
\pgfpathlineto{\pgfqpoint{2.895925in}{1.477033in}}%
\pgfpathlineto{\pgfqpoint{2.895925in}{1.473281in}}%
\pgfpathlineto{\pgfqpoint{2.894985in}{1.472156in}}%
\pgfpathlineto{\pgfqpoint{2.892791in}{1.469530in}}%
\pgfpathlineto{\pgfqpoint{2.892791in}{1.465778in}}%
\pgfpathlineto{\pgfqpoint{2.892791in}{1.462027in}}%
\pgfpathlineto{\pgfqpoint{2.891850in}{1.460901in}}%
\pgfpathlineto{\pgfqpoint{2.889656in}{1.458275in}}%
\pgfpathlineto{\pgfqpoint{2.889656in}{1.454524in}}%
\pgfpathlineto{\pgfqpoint{2.889656in}{1.450772in}}%
\pgfpathlineto{\pgfqpoint{2.889656in}{1.447020in}}%
\pgfpathlineto{\pgfqpoint{2.888716in}{1.445895in}}%
\pgfpathlineto{\pgfqpoint{2.886521in}{1.443269in}}%
\pgfpathlineto{\pgfqpoint{2.886521in}{1.439517in}}%
\pgfpathlineto{\pgfqpoint{2.886521in}{1.435766in}}%
\pgfpathlineto{\pgfqpoint{2.885581in}{1.434640in}}%
\pgfpathlineto{\pgfqpoint{2.883387in}{1.432014in}}%
\pgfpathlineto{\pgfqpoint{2.883387in}{1.428263in}}%
\pgfpathlineto{\pgfqpoint{2.883387in}{1.424511in}}%
\pgfpathlineto{\pgfqpoint{2.882446in}{1.423386in}}%
\pgfpathlineto{\pgfqpoint{2.880252in}{1.420760in}}%
\pgfpathlineto{\pgfqpoint{2.880252in}{1.417008in}}%
\pgfpathlineto{\pgfqpoint{2.880252in}{1.413257in}}%
\pgfpathlineto{\pgfqpoint{2.879311in}{1.412131in}}%
\pgfpathlineto{\pgfqpoint{2.877117in}{1.409505in}}%
\pgfpathlineto{\pgfqpoint{2.877117in}{1.405754in}}%
\pgfpathlineto{\pgfqpoint{2.877117in}{1.402002in}}%
\pgfpathlineto{\pgfqpoint{2.876177in}{1.400877in}}%
\pgfpathlineto{\pgfqpoint{2.873982in}{1.398251in}}%
\pgfpathlineto{\pgfqpoint{2.873982in}{1.394499in}}%
\pgfpathlineto{\pgfqpoint{2.873982in}{1.390747in}}%
\pgfpathlineto{\pgfqpoint{2.873042in}{1.389622in}}%
\pgfpathlineto{\pgfqpoint{2.870848in}{1.386996in}}%
\pgfpathlineto{\pgfqpoint{2.870848in}{1.383244in}}%
\pgfpathlineto{\pgfqpoint{2.870848in}{1.379493in}}%
\pgfpathlineto{\pgfqpoint{2.869907in}{1.378367in}}%
\pgfpathlineto{\pgfqpoint{2.867713in}{1.375741in}}%
\pgfpathlineto{\pgfqpoint{2.867713in}{1.371990in}}%
\pgfpathlineto{\pgfqpoint{2.867713in}{1.368238in}}%
\pgfpathlineto{\pgfqpoint{2.866772in}{1.367113in}}%
\pgfpathlineto{\pgfqpoint{2.864578in}{1.364487in}}%
\pgfpathlineto{\pgfqpoint{2.864578in}{1.360735in}}%
\pgfpathlineto{\pgfqpoint{2.864578in}{1.356984in}}%
\pgfpathlineto{\pgfqpoint{2.864578in}{1.353232in}}%
\pgfpathlineto{\pgfqpoint{2.863638in}{1.352107in}}%
\pgfpathlineto{\pgfqpoint{2.861443in}{1.349481in}}%
\pgfpathlineto{\pgfqpoint{2.861443in}{1.345729in}}%
\pgfpathlineto{\pgfqpoint{2.861443in}{1.341977in}}%
\pgfpathlineto{\pgfqpoint{2.860503in}{1.340852in}}%
\pgfpathlineto{\pgfqpoint{2.858309in}{1.338226in}}%
\pgfpathlineto{\pgfqpoint{2.858309in}{1.334474in}}%
\pgfpathlineto{\pgfqpoint{2.858309in}{1.330723in}}%
\pgfpathlineto{\pgfqpoint{2.857368in}{1.329597in}}%
\pgfpathlineto{\pgfqpoint{2.855174in}{1.326971in}}%
\pgfpathlineto{\pgfqpoint{2.855174in}{1.323220in}}%
\pgfpathlineto{\pgfqpoint{2.855174in}{1.319468in}}%
\pgfpathlineto{\pgfqpoint{2.854233in}{1.318343in}}%
\pgfpathlineto{\pgfqpoint{2.852039in}{1.315717in}}%
\pgfpathlineto{\pgfqpoint{2.852039in}{1.311965in}}%
\pgfpathlineto{\pgfqpoint{2.852039in}{1.308214in}}%
\pgfpathlineto{\pgfqpoint{2.851099in}{1.307088in}}%
\pgfpathlineto{\pgfqpoint{2.848904in}{1.304462in}}%
\pgfpathlineto{\pgfqpoint{2.848904in}{1.300711in}}%
\pgfpathlineto{\pgfqpoint{2.848904in}{1.296959in}}%
\pgfpathlineto{\pgfqpoint{2.847964in}{1.295834in}}%
\pgfpathlineto{\pgfqpoint{2.845770in}{1.293208in}}%
\pgfpathlineto{\pgfqpoint{2.845770in}{1.289456in}}%
\pgfpathlineto{\pgfqpoint{2.845770in}{1.285704in}}%
\pgfpathlineto{\pgfqpoint{2.844829in}{1.284579in}}%
\pgfpathlineto{\pgfqpoint{2.842635in}{1.281953in}}%
\pgfpathlineto{\pgfqpoint{2.842635in}{1.278201in}}%
\pgfpathlineto{\pgfqpoint{2.842635in}{1.274450in}}%
\pgfpathlineto{\pgfqpoint{2.841694in}{1.273324in}}%
\pgfpathlineto{\pgfqpoint{2.839500in}{1.270698in}}%
\pgfpathlineto{\pgfqpoint{2.839500in}{1.266947in}}%
\pgfpathlineto{\pgfqpoint{2.839500in}{1.263195in}}%
\pgfpathlineto{\pgfqpoint{2.838560in}{1.262070in}}%
\pgfpathlineto{\pgfqpoint{2.836365in}{1.259444in}}%
\pgfpathlineto{\pgfqpoint{2.836365in}{1.255692in}}%
\pgfpathlineto{\pgfqpoint{2.836365in}{1.251941in}}%
\pgfpathlineto{\pgfqpoint{2.836365in}{1.248189in}}%
\pgfpathlineto{\pgfqpoint{2.835425in}{1.247064in}}%
\pgfpathlineto{\pgfqpoint{2.833231in}{1.244438in}}%
\pgfpathlineto{\pgfqpoint{2.833231in}{1.240686in}}%
\pgfpathlineto{\pgfqpoint{2.833231in}{1.236935in}}%
\pgfpathlineto{\pgfqpoint{2.832290in}{1.235809in}}%
\pgfpathlineto{\pgfqpoint{2.830096in}{1.233183in}}%
\pgfpathlineto{\pgfqpoint{2.830096in}{1.229431in}}%
\pgfpathlineto{\pgfqpoint{2.830096in}{1.225680in}}%
\pgfpathlineto{\pgfqpoint{2.829155in}{1.224554in}}%
\pgfpathlineto{\pgfqpoint{2.826961in}{1.221928in}}%
\pgfpathlineto{\pgfqpoint{2.826961in}{1.218177in}}%
\pgfpathlineto{\pgfqpoint{2.826961in}{1.214425in}}%
\pgfpathlineto{\pgfqpoint{2.826021in}{1.213300in}}%
\pgfpathlineto{\pgfqpoint{2.823826in}{1.210674in}}%
\pgfpathlineto{\pgfqpoint{2.823826in}{1.206922in}}%
\pgfpathlineto{\pgfqpoint{2.823826in}{1.203171in}}%
\pgfpathlineto{\pgfqpoint{2.822886in}{1.202045in}}%
\pgfpathlineto{\pgfqpoint{2.820692in}{1.199419in}}%
\pgfpathlineto{\pgfqpoint{2.820692in}{1.195668in}}%
\pgfpathlineto{\pgfqpoint{2.820692in}{1.191916in}}%
\pgfpathlineto{\pgfqpoint{2.819751in}{1.190791in}}%
\pgfpathlineto{\pgfqpoint{2.817557in}{1.188165in}}%
\pgfpathlineto{\pgfqpoint{2.817557in}{1.184413in}}%
\pgfpathlineto{\pgfqpoint{2.817557in}{1.180662in}}%
\pgfpathlineto{\pgfqpoint{2.816617in}{1.179536in}}%
\pgfpathlineto{\pgfqpoint{2.814422in}{1.176910in}}%
\pgfpathlineto{\pgfqpoint{2.814422in}{1.173158in}}%
\pgfpathlineto{\pgfqpoint{2.813482in}{1.172033in}}%
\pgfpathlineto{\pgfqpoint{2.810347in}{1.172033in}}%
\pgfpathlineto{\pgfqpoint{2.807212in}{1.172033in}}%
\pgfpathlineto{\pgfqpoint{2.805018in}{1.169407in}}%
\pgfpathlineto{\pgfqpoint{2.804078in}{1.168281in}}%
\pgfpathlineto{\pgfqpoint{2.800943in}{1.168281in}}%
\pgfpathlineto{\pgfqpoint{2.797808in}{1.168281in}}%
\pgfpathlineto{\pgfqpoint{2.795614in}{1.165655in}}%
\pgfpathlineto{\pgfqpoint{2.794673in}{1.164530in}}%
\pgfpathlineto{\pgfqpoint{2.791539in}{1.164530in}}%
\pgfpathlineto{\pgfqpoint{2.789344in}{1.161904in}}%
\pgfpathlineto{\pgfqpoint{2.788404in}{1.160778in}}%
\pgfpathlineto{\pgfqpoint{2.785269in}{1.160778in}}%
\pgfpathlineto{\pgfqpoint{2.782134in}{1.160778in}}%
\pgfpathlineto{\pgfqpoint{2.779940in}{1.158152in}}%
\pgfpathlineto{\pgfqpoint{2.779000in}{1.157027in}}%
\pgfpathlineto{\pgfqpoint{2.775865in}{1.157027in}}%
\pgfpathlineto{\pgfqpoint{2.772730in}{1.157027in}}%
\pgfpathlineto{\pgfqpoint{2.770536in}{1.154401in}}%
\pgfpathlineto{\pgfqpoint{2.769595in}{1.153275in}}%
\pgfpathlineto{\pgfqpoint{2.766461in}{1.153275in}}%
\pgfpathlineto{\pgfqpoint{2.763326in}{1.153275in}}%
\pgfpathlineto{\pgfqpoint{2.761132in}{1.150649in}}%
\pgfpathlineto{\pgfqpoint{2.760191in}{1.149524in}}%
\pgfpathlineto{\pgfqpoint{2.757056in}{1.149524in}}%
\pgfpathlineto{\pgfqpoint{2.754862in}{1.146898in}}%
\pgfpathlineto{\pgfqpoint{2.753922in}{1.145772in}}%
\pgfpathlineto{\pgfqpoint{2.750787in}{1.145772in}}%
\pgfpathlineto{\pgfqpoint{2.747652in}{1.145772in}}%
\pgfpathlineto{\pgfqpoint{2.745458in}{1.143146in}}%
\pgfpathlineto{\pgfqpoint{2.744517in}{1.142021in}}%
\pgfpathlineto{\pgfqpoint{2.741383in}{1.142021in}}%
\pgfpathlineto{\pgfqpoint{2.738248in}{1.142021in}}%
\pgfpathlineto{\pgfqpoint{2.736054in}{1.139395in}}%
\pgfpathlineto{\pgfqpoint{2.735113in}{1.138269in}}%
\pgfpathlineto{\pgfqpoint{2.731978in}{1.138269in}}%
\pgfpathlineto{\pgfqpoint{2.728844in}{1.138269in}}%
\pgfpathlineto{\pgfqpoint{2.726649in}{1.135643in}}%
\pgfpathlineto{\pgfqpoint{2.725709in}{1.134518in}}%
\pgfpathlineto{\pgfqpoint{2.722574in}{1.134518in}}%
\pgfpathlineto{\pgfqpoint{2.719439in}{1.134518in}}%
\pgfpathlineto{\pgfqpoint{2.717245in}{1.131892in}}%
\pgfpathlineto{\pgfqpoint{2.716305in}{1.130766in}}%
\pgfpathlineto{\pgfqpoint{2.713170in}{1.130766in}}%
\pgfpathlineto{\pgfqpoint{2.710976in}{1.128140in}}%
\pgfpathlineto{\pgfqpoint{2.710035in}{1.127015in}}%
\pgfpathlineto{\pgfqpoint{2.706901in}{1.127015in}}%
\pgfpathlineto{\pgfqpoint{2.703766in}{1.127015in}}%
\pgfpathlineto{\pgfqpoint{2.701571in}{1.124388in}}%
\pgfpathlineto{\pgfqpoint{2.700631in}{1.123263in}}%
\pgfpathlineto{\pgfqpoint{2.697496in}{1.123263in}}%
\pgfpathlineto{\pgfqpoint{2.694362in}{1.123263in}}%
\pgfpathlineto{\pgfqpoint{2.692167in}{1.120637in}}%
\pgfpathlineto{\pgfqpoint{2.691227in}{1.119511in}}%
\pgfpathlineto{\pgfqpoint{2.688092in}{1.119511in}}%
\pgfpathlineto{\pgfqpoint{2.684957in}{1.119511in}}%
\pgfpathlineto{\pgfqpoint{2.682763in}{1.116885in}}%
\pgfpathlineto{\pgfqpoint{2.681823in}{1.115760in}}%
\pgfpathlineto{\pgfqpoint{2.678688in}{1.115760in}}%
\pgfpathlineto{\pgfqpoint{2.676494in}{1.113134in}}%
\pgfpathlineto{\pgfqpoint{2.675553in}{1.112008in}}%
\pgfpathlineto{\pgfqpoint{2.672418in}{1.112008in}}%
\pgfpathlineto{\pgfqpoint{2.669284in}{1.112008in}}%
\pgfpathlineto{\pgfqpoint{2.667089in}{1.109382in}}%
\pgfpathlineto{\pgfqpoint{2.666149in}{1.108257in}}%
\pgfpathlineto{\pgfqpoint{2.663014in}{1.108257in}}%
\pgfpathlineto{\pgfqpoint{2.659879in}{1.108257in}}%
\pgfpathlineto{\pgfqpoint{2.657685in}{1.105631in}}%
\pgfpathlineto{\pgfqpoint{2.656745in}{1.104505in}}%
\pgfpathlineto{\pgfqpoint{2.653610in}{1.104505in}}%
\pgfpathlineto{\pgfqpoint{2.650475in}{1.104505in}}%
\pgfpathlineto{\pgfqpoint{2.648281in}{1.101879in}}%
\pgfpathlineto{\pgfqpoint{2.647340in}{1.100754in}}%
\pgfpathlineto{\pgfqpoint{2.644206in}{1.100754in}}%
\pgfpathlineto{\pgfqpoint{2.642011in}{1.098128in}}%
\pgfpathlineto{\pgfqpoint{2.641071in}{1.097002in}}%
\pgfpathlineto{\pgfqpoint{2.637936in}{1.097002in}}%
\pgfpathlineto{\pgfqpoint{2.634801in}{1.097002in}}%
\pgfpathlineto{\pgfqpoint{2.632607in}{1.094376in}}%
\pgfpathlineto{\pgfqpoint{2.631667in}{1.093251in}}%
\pgfpathlineto{\pgfqpoint{2.628532in}{1.093251in}}%
\pgfpathlineto{\pgfqpoint{2.625397in}{1.093251in}}%
\pgfpathlineto{\pgfqpoint{2.623203in}{1.090625in}}%
\pgfpathlineto{\pgfqpoint{2.622262in}{1.089499in}}%
\pgfpathlineto{\pgfqpoint{2.619128in}{1.089499in}}%
\pgfpathlineto{\pgfqpoint{2.615993in}{1.089499in}}%
\pgfpathlineto{\pgfqpoint{2.613799in}{1.086873in}}%
\pgfpathlineto{\pgfqpoint{2.612858in}{1.085748in}}%
\pgfpathlineto{\pgfqpoint{2.609724in}{1.085748in}}%
\pgfpathlineto{\pgfqpoint{2.606589in}{1.085748in}}%
\pgfpathlineto{\pgfqpoint{2.604394in}{1.083122in}}%
\pgfpathlineto{\pgfqpoint{2.603454in}{1.081996in}}%
\pgfpathlineto{\pgfqpoint{2.600319in}{1.081996in}}%
\pgfpathlineto{\pgfqpoint{2.598125in}{1.079370in}}%
\pgfpathlineto{\pgfqpoint{2.597185in}{1.078245in}}%
\pgfpathlineto{\pgfqpoint{2.594050in}{1.078245in}}%
\pgfpathlineto{\pgfqpoint{2.590915in}{1.078245in}}%
\pgfpathlineto{\pgfqpoint{2.588721in}{1.075619in}}%
\pgfpathlineto{\pgfqpoint{2.587780in}{1.074493in}}%
\pgfpathlineto{\pgfqpoint{2.584646in}{1.074493in}}%
\pgfpathlineto{\pgfqpoint{2.581511in}{1.074493in}}%
\pgfpathlineto{\pgfqpoint{2.579317in}{1.071867in}}%
\pgfpathlineto{\pgfqpoint{2.578376in}{1.070742in}}%
\pgfpathlineto{\pgfqpoint{2.575241in}{1.070742in}}%
\pgfpathlineto{\pgfqpoint{2.572107in}{1.070742in}}%
\pgfpathlineto{\pgfqpoint{2.569912in}{1.068115in}}%
\pgfpathlineto{\pgfqpoint{2.568972in}{1.066990in}}%
\pgfpathlineto{\pgfqpoint{2.565837in}{1.066990in}}%
\pgfpathlineto{\pgfqpoint{2.563643in}{1.064364in}}%
\pgfpathlineto{\pgfqpoint{2.562702in}{1.063238in}}%
\pgfpathlineto{\pgfqpoint{2.559568in}{1.063238in}}%
\pgfpathlineto{\pgfqpoint{2.556433in}{1.063238in}}%
\pgfpathlineto{\pgfqpoint{2.554239in}{1.060612in}}%
\pgfpathlineto{\pgfqpoint{2.553298in}{1.059487in}}%
\pgfpathlineto{\pgfqpoint{2.550163in}{1.059487in}}%
\pgfpathlineto{\pgfqpoint{2.547029in}{1.059487in}}%
\pgfpathlineto{\pgfqpoint{2.544834in}{1.056861in}}%
\pgfpathlineto{\pgfqpoint{2.543894in}{1.055735in}}%
\pgfpathlineto{\pgfqpoint{2.540759in}{1.055735in}}%
\pgfpathlineto{\pgfqpoint{2.537624in}{1.055735in}}%
\pgfpathlineto{\pgfqpoint{2.535430in}{1.053109in}}%
\pgfpathlineto{\pgfqpoint{2.534490in}{1.051984in}}%
\pgfpathlineto{\pgfqpoint{2.531355in}{1.051984in}}%
\pgfpathlineto{\pgfqpoint{2.529161in}{1.049358in}}%
\pgfpathlineto{\pgfqpoint{2.528220in}{1.048232in}}%
\pgfpathlineto{\pgfqpoint{2.525085in}{1.048232in}}%
\pgfpathlineto{\pgfqpoint{2.521951in}{1.048232in}}%
\pgfpathlineto{\pgfqpoint{2.519756in}{1.045606in}}%
\pgfpathlineto{\pgfqpoint{2.518816in}{1.044481in}}%
\pgfpathlineto{\pgfqpoint{2.515681in}{1.044481in}}%
\pgfpathlineto{\pgfqpoint{2.512547in}{1.044481in}}%
\pgfpathlineto{\pgfqpoint{2.510352in}{1.041855in}}%
\pgfpathlineto{\pgfqpoint{2.509412in}{1.040729in}}%
\pgfpathlineto{\pgfqpoint{2.506277in}{1.040729in}}%
\pgfpathlineto{\pgfqpoint{2.503142in}{1.040729in}}%
\pgfpathlineto{\pgfqpoint{2.500948in}{1.038103in}}%
\pgfpathlineto{\pgfqpoint{2.500008in}{1.036978in}}%
\pgfpathlineto{\pgfqpoint{2.496873in}{1.036978in}}%
\pgfpathlineto{\pgfqpoint{2.493738in}{1.036978in}}%
\pgfpathlineto{\pgfqpoint{2.491544in}{1.034352in}}%
\pgfpathlineto{\pgfqpoint{2.490603in}{1.033226in}}%
\pgfpathlineto{\pgfqpoint{2.487469in}{1.033226in}}%
\pgfpathlineto{\pgfqpoint{2.485274in}{1.030600in}}%
\pgfpathlineto{\pgfqpoint{2.484334in}{1.029475in}}%
\pgfpathlineto{\pgfqpoint{2.481199in}{1.029475in}}%
\pgfpathlineto{\pgfqpoint{2.478064in}{1.029475in}}%
\pgfpathlineto{\pgfqpoint{2.475870in}{1.026849in}}%
\pgfpathlineto{\pgfqpoint{2.474930in}{1.025723in}}%
\pgfpathlineto{\pgfqpoint{2.471795in}{1.025723in}}%
\pgfpathlineto{\pgfqpoint{2.468660in}{1.025723in}}%
\pgfpathlineto{\pgfqpoint{2.466466in}{1.023097in}}%
\pgfpathlineto{\pgfqpoint{2.465525in}{1.021972in}}%
\pgfpathlineto{\pgfqpoint{2.462391in}{1.021972in}}%
\pgfpathlineto{\pgfqpoint{2.459256in}{1.021972in}}%
\pgfpathlineto{\pgfqpoint{2.457062in}{1.019346in}}%
\pgfpathlineto{\pgfqpoint{2.456121in}{1.018220in}}%
\pgfpathlineto{\pgfqpoint{2.452986in}{1.018220in}}%
\pgfpathlineto{\pgfqpoint{2.450792in}{1.015594in}}%
\pgfpathlineto{\pgfqpoint{2.449852in}{1.014469in}}%
\pgfpathlineto{\pgfqpoint{2.446717in}{1.014469in}}%
\pgfpathlineto{\pgfqpoint{2.443582in}{1.014469in}}%
\pgfpathlineto{\pgfqpoint{2.441388in}{1.011842in}}%
\pgfpathlineto{\pgfqpoint{2.440447in}{1.010717in}}%
\pgfpathlineto{\pgfqpoint{2.437313in}{1.010717in}}%
\pgfpathlineto{\pgfqpoint{2.434178in}{1.010717in}}%
\pgfpathlineto{\pgfqpoint{2.431984in}{1.008091in}}%
\pgfpathlineto{\pgfqpoint{2.431043in}{1.006965in}}%
\pgfpathlineto{\pgfqpoint{2.427908in}{1.006965in}}%
\pgfpathlineto{\pgfqpoint{2.424774in}{1.006965in}}%
\pgfpathlineto{\pgfqpoint{2.422579in}{1.004339in}}%
\pgfpathlineto{\pgfqpoint{2.421639in}{1.003214in}}%
\pgfpathlineto{\pgfqpoint{2.418504in}{1.003214in}}%
\pgfpathlineto{\pgfqpoint{2.416310in}{1.000588in}}%
\pgfpathlineto{\pgfqpoint{2.415369in}{0.999462in}}%
\pgfpathlineto{\pgfqpoint{2.412235in}{0.999462in}}%
\pgfpathlineto{\pgfqpoint{2.409100in}{0.999462in}}%
\pgfpathlineto{\pgfqpoint{2.406906in}{0.996836in}}%
\pgfpathlineto{\pgfqpoint{2.405965in}{0.995711in}}%
\pgfpathlineto{\pgfqpoint{2.402831in}{0.995711in}}%
\pgfpathlineto{\pgfqpoint{2.399696in}{0.995711in}}%
\pgfpathlineto{\pgfqpoint{2.397501in}{0.993085in}}%
\pgfpathlineto{\pgfqpoint{2.396561in}{0.991959in}}%
\pgfpathlineto{\pgfqpoint{2.393426in}{0.991959in}}%
\pgfpathlineto{\pgfqpoint{2.390292in}{0.991959in}}%
\pgfpathlineto{\pgfqpoint{2.388097in}{0.989333in}}%
\pgfpathlineto{\pgfqpoint{2.387157in}{0.988208in}}%
\pgfpathlineto{\pgfqpoint{2.384022in}{0.988208in}}%
\pgfpathlineto{\pgfqpoint{2.380887in}{0.988208in}}%
\pgfpathlineto{\pgfqpoint{2.378693in}{0.985582in}}%
\pgfpathlineto{\pgfqpoint{2.377753in}{0.984456in}}%
\pgfpathlineto{\pgfqpoint{2.374618in}{0.984456in}}%
\pgfpathlineto{\pgfqpoint{2.372424in}{0.981830in}}%
\pgfpathlineto{\pgfqpoint{2.371483in}{0.980705in}}%
\pgfpathlineto{\pgfqpoint{2.368348in}{0.980705in}}%
\pgfpathlineto{\pgfqpoint{2.365214in}{0.980705in}}%
\pgfpathlineto{\pgfqpoint{2.363019in}{0.978079in}}%
\pgfpathlineto{\pgfqpoint{2.362079in}{0.976953in}}%
\pgfpathlineto{\pgfqpoint{2.358944in}{0.976953in}}%
\pgfpathlineto{\pgfqpoint{2.355809in}{0.976953in}}%
\pgfpathlineto{\pgfqpoint{2.353615in}{0.974327in}}%
\pgfpathlineto{\pgfqpoint{2.352675in}{0.973202in}}%
\pgfpathlineto{\pgfqpoint{2.349540in}{0.973202in}}%
\pgfpathlineto{\pgfqpoint{2.346405in}{0.973202in}}%
\pgfpathlineto{\pgfqpoint{2.344211in}{0.970576in}}%
\pgfpathlineto{\pgfqpoint{2.343270in}{0.969450in}}%
\pgfpathlineto{\pgfqpoint{2.340136in}{0.969450in}}%
\pgfpathlineto{\pgfqpoint{2.337941in}{0.966824in}}%
\pgfpathlineto{\pgfqpoint{2.337001in}{0.965699in}}%
\pgfpathlineto{\pgfqpoint{2.333866in}{0.965699in}}%
\pgfpathlineto{\pgfqpoint{2.330731in}{0.965699in}}%
\pgfpathlineto{\pgfqpoint{2.328537in}{0.963073in}}%
\pgfpathlineto{\pgfqpoint{2.327597in}{0.961947in}}%
\pgfpathlineto{\pgfqpoint{2.324462in}{0.961947in}}%
\pgfpathlineto{\pgfqpoint{2.321327in}{0.961947in}}%
\pgfpathlineto{\pgfqpoint{2.319133in}{0.959321in}}%
\pgfpathlineto{\pgfqpoint{2.318192in}{0.958196in}}%
\pgfpathlineto{\pgfqpoint{2.315058in}{0.958196in}}%
\pgfpathlineto{\pgfqpoint{2.311923in}{0.958196in}}%
\pgfpathlineto{\pgfqpoint{2.309729in}{0.955569in}}%
\pgfpathlineto{\pgfqpoint{2.308788in}{0.954444in}}%
\pgfpathlineto{\pgfqpoint{2.305654in}{0.954444in}}%
\pgfpathlineto{\pgfqpoint{2.303459in}{0.951818in}}%
\pgfpathlineto{\pgfqpoint{2.302519in}{0.950692in}}%
\pgfpathlineto{\pgfqpoint{2.299384in}{0.950692in}}%
\pgfpathlineto{\pgfqpoint{2.296249in}{0.950692in}}%
\pgfpathlineto{\pgfqpoint{2.294055in}{0.948066in}}%
\pgfpathlineto{\pgfqpoint{2.293115in}{0.946941in}}%
\pgfpathlineto{\pgfqpoint{2.289980in}{0.946941in}}%
\pgfpathlineto{\pgfqpoint{2.286845in}{0.946941in}}%
\pgfpathlineto{\pgfqpoint{2.284651in}{0.944315in}}%
\pgfpathlineto{\pgfqpoint{2.283710in}{0.943189in}}%
\pgfpathlineto{\pgfqpoint{2.280576in}{0.943189in}}%
\pgfpathlineto{\pgfqpoint{2.277441in}{0.943189in}}%
\pgfpathlineto{\pgfqpoint{2.275247in}{0.940563in}}%
\pgfpathlineto{\pgfqpoint{2.274306in}{0.939438in}}%
\pgfpathlineto{\pgfqpoint{2.271171in}{0.939438in}}%
\pgfpathlineto{\pgfqpoint{2.268977in}{0.936812in}}%
\pgfpathlineto{\pgfqpoint{2.268037in}{0.935686in}}%
\pgfpathlineto{\pgfqpoint{2.264902in}{0.935686in}}%
\pgfpathlineto{\pgfqpoint{2.261767in}{0.935686in}}%
\pgfpathlineto{\pgfqpoint{2.259573in}{0.933060in}}%
\pgfpathlineto{\pgfqpoint{2.258632in}{0.931935in}}%
\pgfpathlineto{\pgfqpoint{2.255498in}{0.931935in}}%
\pgfpathlineto{\pgfqpoint{2.252363in}{0.931935in}}%
\pgfpathlineto{\pgfqpoint{2.250169in}{0.929309in}}%
\pgfpathlineto{\pgfqpoint{2.249228in}{0.928183in}}%
\pgfpathlineto{\pgfqpoint{2.246093in}{0.928183in}}%
\pgfpathlineto{\pgfqpoint{2.242959in}{0.928183in}}%
\pgfpathlineto{\pgfqpoint{2.240764in}{0.925557in}}%
\pgfpathlineto{\pgfqpoint{2.239824in}{0.924432in}}%
\pgfpathlineto{\pgfqpoint{2.236689in}{0.924432in}}%
\pgfpathlineto{\pgfqpoint{2.233554in}{0.924432in}}%
\pgfpathlineto{\pgfqpoint{2.231360in}{0.921806in}}%
\pgfpathlineto{\pgfqpoint{2.230420in}{0.920680in}}%
\pgfpathlineto{\pgfqpoint{2.227285in}{0.920680in}}%
\pgfpathlineto{\pgfqpoint{2.225091in}{0.918054in}}%
\pgfpathlineto{\pgfqpoint{2.224150in}{0.916929in}}%
\pgfpathlineto{\pgfqpoint{2.221015in}{0.916929in}}%
\pgfpathlineto{\pgfqpoint{2.217881in}{0.916929in}}%
\pgfpathlineto{\pgfqpoint{2.215686in}{0.914303in}}%
\pgfpathlineto{\pgfqpoint{2.214746in}{0.913177in}}%
\pgfpathlineto{\pgfqpoint{2.211611in}{0.913177in}}%
\pgfpathlineto{\pgfqpoint{2.208477in}{0.913177in}}%
\pgfpathlineto{\pgfqpoint{2.206282in}{0.910551in}}%
\pgfpathlineto{\pgfqpoint{2.205342in}{0.909426in}}%
\pgfpathlineto{\pgfqpoint{2.202207in}{0.909426in}}%
\pgfpathlineto{\pgfqpoint{2.199072in}{0.909426in}}%
\pgfpathlineto{\pgfqpoint{2.196878in}{0.906799in}}%
\pgfpathlineto{\pgfqpoint{2.195938in}{0.905674in}}%
\pgfpathlineto{\pgfqpoint{2.192803in}{0.905674in}}%
\pgfpathlineto{\pgfqpoint{2.190608in}{0.903048in}}%
\pgfpathlineto{\pgfqpoint{2.189668in}{0.901922in}}%
\pgfpathlineto{\pgfqpoint{2.186533in}{0.901922in}}%
\pgfpathlineto{\pgfqpoint{2.183399in}{0.901922in}}%
\pgfpathlineto{\pgfqpoint{2.181204in}{0.899296in}}%
\pgfpathlineto{\pgfqpoint{2.180264in}{0.898171in}}%
\pgfpathlineto{\pgfqpoint{2.177129in}{0.898171in}}%
\pgfpathlineto{\pgfqpoint{2.173994in}{0.898171in}}%
\pgfpathclose%
\pgfusepath{fill}%
\end{pgfscope}%
\begin{pgfscope}%
\pgfpathrectangle{\pgfqpoint{0.888750in}{0.419100in}}{\pgfqpoint{2.504659in}{2.933700in}} %
\pgfusepath{clip}%
\pgfsetbuttcap%
\pgfsetroundjoin%
\definecolor{currentfill}{rgb}{0.000000,0.728073,0.000000}%
\pgfsetfillcolor{currentfill}%
\pgfsetfillopacity{0.300000}%
\pgfsetlinewidth{0.000000pt}%
\definecolor{currentstroke}{rgb}{0.000000,0.000000,0.000000}%
\pgfsetstrokecolor{currentstroke}%
\pgfsetdash{}{0pt}%
\pgfpathmoveto{\pgfqpoint{2.173994in}{0.897045in}}%
\pgfpathlineto{\pgfqpoint{2.177129in}{0.897045in}}%
\pgfpathlineto{\pgfqpoint{2.180264in}{0.897045in}}%
\pgfpathlineto{\pgfqpoint{2.182145in}{0.899296in}}%
\pgfpathlineto{\pgfqpoint{2.183399in}{0.900797in}}%
\pgfpathlineto{\pgfqpoint{2.186533in}{0.900797in}}%
\pgfpathlineto{\pgfqpoint{2.189668in}{0.900797in}}%
\pgfpathlineto{\pgfqpoint{2.191549in}{0.903048in}}%
\pgfpathlineto{\pgfqpoint{2.192803in}{0.904549in}}%
\pgfpathlineto{\pgfqpoint{2.195938in}{0.904549in}}%
\pgfpathlineto{\pgfqpoint{2.197818in}{0.906799in}}%
\pgfpathlineto{\pgfqpoint{2.199072in}{0.908300in}}%
\pgfpathlineto{\pgfqpoint{2.202207in}{0.908300in}}%
\pgfpathlineto{\pgfqpoint{2.205342in}{0.908300in}}%
\pgfpathlineto{\pgfqpoint{2.207223in}{0.910551in}}%
\pgfpathlineto{\pgfqpoint{2.208477in}{0.912052in}}%
\pgfpathlineto{\pgfqpoint{2.211611in}{0.912052in}}%
\pgfpathlineto{\pgfqpoint{2.214746in}{0.912052in}}%
\pgfpathlineto{\pgfqpoint{2.216627in}{0.914303in}}%
\pgfpathlineto{\pgfqpoint{2.217881in}{0.915803in}}%
\pgfpathlineto{\pgfqpoint{2.221015in}{0.915803in}}%
\pgfpathlineto{\pgfqpoint{2.224150in}{0.915803in}}%
\pgfpathlineto{\pgfqpoint{2.226031in}{0.918054in}}%
\pgfpathlineto{\pgfqpoint{2.227285in}{0.919555in}}%
\pgfpathlineto{\pgfqpoint{2.230420in}{0.919555in}}%
\pgfpathlineto{\pgfqpoint{2.232301in}{0.921806in}}%
\pgfpathlineto{\pgfqpoint{2.233554in}{0.923306in}}%
\pgfpathlineto{\pgfqpoint{2.236689in}{0.923306in}}%
\pgfpathlineto{\pgfqpoint{2.239824in}{0.923306in}}%
\pgfpathlineto{\pgfqpoint{2.241705in}{0.925557in}}%
\pgfpathlineto{\pgfqpoint{2.242959in}{0.927058in}}%
\pgfpathlineto{\pgfqpoint{2.246093in}{0.927058in}}%
\pgfpathlineto{\pgfqpoint{2.249228in}{0.927058in}}%
\pgfpathlineto{\pgfqpoint{2.251109in}{0.929309in}}%
\pgfpathlineto{\pgfqpoint{2.252363in}{0.930809in}}%
\pgfpathlineto{\pgfqpoint{2.255498in}{0.930809in}}%
\pgfpathlineto{\pgfqpoint{2.258632in}{0.930809in}}%
\pgfpathlineto{\pgfqpoint{2.260513in}{0.933060in}}%
\pgfpathlineto{\pgfqpoint{2.261767in}{0.934561in}}%
\pgfpathlineto{\pgfqpoint{2.264902in}{0.934561in}}%
\pgfpathlineto{\pgfqpoint{2.268037in}{0.934561in}}%
\pgfpathlineto{\pgfqpoint{2.269917in}{0.936812in}}%
\pgfpathlineto{\pgfqpoint{2.271171in}{0.938312in}}%
\pgfpathlineto{\pgfqpoint{2.274306in}{0.938312in}}%
\pgfpathlineto{\pgfqpoint{2.276187in}{0.940563in}}%
\pgfpathlineto{\pgfqpoint{2.277441in}{0.942064in}}%
\pgfpathlineto{\pgfqpoint{2.280576in}{0.942064in}}%
\pgfpathlineto{\pgfqpoint{2.283710in}{0.942064in}}%
\pgfpathlineto{\pgfqpoint{2.285591in}{0.944315in}}%
\pgfpathlineto{\pgfqpoint{2.286845in}{0.945815in}}%
\pgfpathlineto{\pgfqpoint{2.289980in}{0.945815in}}%
\pgfpathlineto{\pgfqpoint{2.293115in}{0.945815in}}%
\pgfpathlineto{\pgfqpoint{2.294995in}{0.948066in}}%
\pgfpathlineto{\pgfqpoint{2.296249in}{0.949567in}}%
\pgfpathlineto{\pgfqpoint{2.299384in}{0.949567in}}%
\pgfpathlineto{\pgfqpoint{2.302519in}{0.949567in}}%
\pgfpathlineto{\pgfqpoint{2.304400in}{0.951818in}}%
\pgfpathlineto{\pgfqpoint{2.305654in}{0.953319in}}%
\pgfpathlineto{\pgfqpoint{2.308788in}{0.953319in}}%
\pgfpathlineto{\pgfqpoint{2.310669in}{0.955569in}}%
\pgfpathlineto{\pgfqpoint{2.311923in}{0.957070in}}%
\pgfpathlineto{\pgfqpoint{2.315058in}{0.957070in}}%
\pgfpathlineto{\pgfqpoint{2.318192in}{0.957070in}}%
\pgfpathlineto{\pgfqpoint{2.320073in}{0.959321in}}%
\pgfpathlineto{\pgfqpoint{2.321327in}{0.960822in}}%
\pgfpathlineto{\pgfqpoint{2.324462in}{0.960822in}}%
\pgfpathlineto{\pgfqpoint{2.327597in}{0.960822in}}%
\pgfpathlineto{\pgfqpoint{2.329478in}{0.963073in}}%
\pgfpathlineto{\pgfqpoint{2.330731in}{0.964573in}}%
\pgfpathlineto{\pgfqpoint{2.333866in}{0.964573in}}%
\pgfpathlineto{\pgfqpoint{2.337001in}{0.964573in}}%
\pgfpathlineto{\pgfqpoint{2.338882in}{0.966824in}}%
\pgfpathlineto{\pgfqpoint{2.340136in}{0.968325in}}%
\pgfpathlineto{\pgfqpoint{2.343270in}{0.968325in}}%
\pgfpathlineto{\pgfqpoint{2.345151in}{0.970576in}}%
\pgfpathlineto{\pgfqpoint{2.346405in}{0.972076in}}%
\pgfpathlineto{\pgfqpoint{2.349540in}{0.972076in}}%
\pgfpathlineto{\pgfqpoint{2.352675in}{0.972076in}}%
\pgfpathlineto{\pgfqpoint{2.354555in}{0.974327in}}%
\pgfpathlineto{\pgfqpoint{2.355809in}{0.975828in}}%
\pgfpathlineto{\pgfqpoint{2.358944in}{0.975828in}}%
\pgfpathlineto{\pgfqpoint{2.362079in}{0.975828in}}%
\pgfpathlineto{\pgfqpoint{2.363960in}{0.978079in}}%
\pgfpathlineto{\pgfqpoint{2.365214in}{0.979579in}}%
\pgfpathlineto{\pgfqpoint{2.368348in}{0.979579in}}%
\pgfpathlineto{\pgfqpoint{2.371483in}{0.979579in}}%
\pgfpathlineto{\pgfqpoint{2.373364in}{0.981830in}}%
\pgfpathlineto{\pgfqpoint{2.374618in}{0.983331in}}%
\pgfpathlineto{\pgfqpoint{2.377753in}{0.983331in}}%
\pgfpathlineto{\pgfqpoint{2.379633in}{0.985582in}}%
\pgfpathlineto{\pgfqpoint{2.380887in}{0.987082in}}%
\pgfpathlineto{\pgfqpoint{2.384022in}{0.987082in}}%
\pgfpathlineto{\pgfqpoint{2.387157in}{0.987082in}}%
\pgfpathlineto{\pgfqpoint{2.389038in}{0.989333in}}%
\pgfpathlineto{\pgfqpoint{2.390292in}{0.990834in}}%
\pgfpathlineto{\pgfqpoint{2.393426in}{0.990834in}}%
\pgfpathlineto{\pgfqpoint{2.396561in}{0.990834in}}%
\pgfpathlineto{\pgfqpoint{2.398442in}{0.993085in}}%
\pgfpathlineto{\pgfqpoint{2.399696in}{0.994585in}}%
\pgfpathlineto{\pgfqpoint{2.402831in}{0.994585in}}%
\pgfpathlineto{\pgfqpoint{2.405965in}{0.994585in}}%
\pgfpathlineto{\pgfqpoint{2.407846in}{0.996836in}}%
\pgfpathlineto{\pgfqpoint{2.409100in}{0.998337in}}%
\pgfpathlineto{\pgfqpoint{2.412235in}{0.998337in}}%
\pgfpathlineto{\pgfqpoint{2.415369in}{0.998337in}}%
\pgfpathlineto{\pgfqpoint{2.417250in}{1.000588in}}%
\pgfpathlineto{\pgfqpoint{2.418504in}{1.002088in}}%
\pgfpathlineto{\pgfqpoint{2.421639in}{1.002088in}}%
\pgfpathlineto{\pgfqpoint{2.423520in}{1.004339in}}%
\pgfpathlineto{\pgfqpoint{2.424774in}{1.005840in}}%
\pgfpathlineto{\pgfqpoint{2.427908in}{1.005840in}}%
\pgfpathlineto{\pgfqpoint{2.431043in}{1.005840in}}%
\pgfpathlineto{\pgfqpoint{2.432924in}{1.008091in}}%
\pgfpathlineto{\pgfqpoint{2.434178in}{1.009592in}}%
\pgfpathlineto{\pgfqpoint{2.437313in}{1.009592in}}%
\pgfpathlineto{\pgfqpoint{2.440447in}{1.009592in}}%
\pgfpathlineto{\pgfqpoint{2.442328in}{1.011842in}}%
\pgfpathlineto{\pgfqpoint{2.443582in}{1.013343in}}%
\pgfpathlineto{\pgfqpoint{2.446717in}{1.013343in}}%
\pgfpathlineto{\pgfqpoint{2.449852in}{1.013343in}}%
\pgfpathlineto{\pgfqpoint{2.451733in}{1.015594in}}%
\pgfpathlineto{\pgfqpoint{2.452986in}{1.017095in}}%
\pgfpathlineto{\pgfqpoint{2.456121in}{1.017095in}}%
\pgfpathlineto{\pgfqpoint{2.458002in}{1.019346in}}%
\pgfpathlineto{\pgfqpoint{2.459256in}{1.020846in}}%
\pgfpathlineto{\pgfqpoint{2.462391in}{1.020846in}}%
\pgfpathlineto{\pgfqpoint{2.465525in}{1.020846in}}%
\pgfpathlineto{\pgfqpoint{2.467406in}{1.023097in}}%
\pgfpathlineto{\pgfqpoint{2.468660in}{1.024598in}}%
\pgfpathlineto{\pgfqpoint{2.471795in}{1.024598in}}%
\pgfpathlineto{\pgfqpoint{2.474930in}{1.024598in}}%
\pgfpathlineto{\pgfqpoint{2.476810in}{1.026849in}}%
\pgfpathlineto{\pgfqpoint{2.478064in}{1.028349in}}%
\pgfpathlineto{\pgfqpoint{2.481199in}{1.028349in}}%
\pgfpathlineto{\pgfqpoint{2.484334in}{1.028349in}}%
\pgfpathlineto{\pgfqpoint{2.486215in}{1.030600in}}%
\pgfpathlineto{\pgfqpoint{2.487469in}{1.032101in}}%
\pgfpathlineto{\pgfqpoint{2.490603in}{1.032101in}}%
\pgfpathlineto{\pgfqpoint{2.492484in}{1.034352in}}%
\pgfpathlineto{\pgfqpoint{2.493738in}{1.035852in}}%
\pgfpathlineto{\pgfqpoint{2.496873in}{1.035852in}}%
\pgfpathlineto{\pgfqpoint{2.500008in}{1.035852in}}%
\pgfpathlineto{\pgfqpoint{2.501888in}{1.038103in}}%
\pgfpathlineto{\pgfqpoint{2.503142in}{1.039604in}}%
\pgfpathlineto{\pgfqpoint{2.506277in}{1.039604in}}%
\pgfpathlineto{\pgfqpoint{2.509412in}{1.039604in}}%
\pgfpathlineto{\pgfqpoint{2.511293in}{1.041855in}}%
\pgfpathlineto{\pgfqpoint{2.512547in}{1.043355in}}%
\pgfpathlineto{\pgfqpoint{2.515681in}{1.043355in}}%
\pgfpathlineto{\pgfqpoint{2.518816in}{1.043355in}}%
\pgfpathlineto{\pgfqpoint{2.520697in}{1.045606in}}%
\pgfpathlineto{\pgfqpoint{2.521951in}{1.047107in}}%
\pgfpathlineto{\pgfqpoint{2.525085in}{1.047107in}}%
\pgfpathlineto{\pgfqpoint{2.528220in}{1.047107in}}%
\pgfpathlineto{\pgfqpoint{2.530101in}{1.049358in}}%
\pgfpathlineto{\pgfqpoint{2.531355in}{1.050858in}}%
\pgfpathlineto{\pgfqpoint{2.534490in}{1.050858in}}%
\pgfpathlineto{\pgfqpoint{2.536371in}{1.053109in}}%
\pgfpathlineto{\pgfqpoint{2.537624in}{1.054610in}}%
\pgfpathlineto{\pgfqpoint{2.540759in}{1.054610in}}%
\pgfpathlineto{\pgfqpoint{2.543894in}{1.054610in}}%
\pgfpathlineto{\pgfqpoint{2.545775in}{1.056861in}}%
\pgfpathlineto{\pgfqpoint{2.547029in}{1.058361in}}%
\pgfpathlineto{\pgfqpoint{2.550163in}{1.058361in}}%
\pgfpathlineto{\pgfqpoint{2.553298in}{1.058361in}}%
\pgfpathlineto{\pgfqpoint{2.555179in}{1.060612in}}%
\pgfpathlineto{\pgfqpoint{2.556433in}{1.062113in}}%
\pgfpathlineto{\pgfqpoint{2.559568in}{1.062113in}}%
\pgfpathlineto{\pgfqpoint{2.562702in}{1.062113in}}%
\pgfpathlineto{\pgfqpoint{2.564583in}{1.064364in}}%
\pgfpathlineto{\pgfqpoint{2.565837in}{1.065865in}}%
\pgfpathlineto{\pgfqpoint{2.568972in}{1.065865in}}%
\pgfpathlineto{\pgfqpoint{2.570853in}{1.068115in}}%
\pgfpathlineto{\pgfqpoint{2.572107in}{1.069616in}}%
\pgfpathlineto{\pgfqpoint{2.575241in}{1.069616in}}%
\pgfpathlineto{\pgfqpoint{2.578376in}{1.069616in}}%
\pgfpathlineto{\pgfqpoint{2.580257in}{1.071867in}}%
\pgfpathlineto{\pgfqpoint{2.581511in}{1.073368in}}%
\pgfpathlineto{\pgfqpoint{2.584646in}{1.073368in}}%
\pgfpathlineto{\pgfqpoint{2.587780in}{1.073368in}}%
\pgfpathlineto{\pgfqpoint{2.589661in}{1.075619in}}%
\pgfpathlineto{\pgfqpoint{2.590915in}{1.077119in}}%
\pgfpathlineto{\pgfqpoint{2.594050in}{1.077119in}}%
\pgfpathlineto{\pgfqpoint{2.597185in}{1.077119in}}%
\pgfpathlineto{\pgfqpoint{2.599065in}{1.079370in}}%
\pgfpathlineto{\pgfqpoint{2.600319in}{1.080871in}}%
\pgfpathlineto{\pgfqpoint{2.603454in}{1.080871in}}%
\pgfpathlineto{\pgfqpoint{2.605335in}{1.083122in}}%
\pgfpathlineto{\pgfqpoint{2.606589in}{1.084622in}}%
\pgfpathlineto{\pgfqpoint{2.609724in}{1.084622in}}%
\pgfpathlineto{\pgfqpoint{2.612858in}{1.084622in}}%
\pgfpathlineto{\pgfqpoint{2.614739in}{1.086873in}}%
\pgfpathlineto{\pgfqpoint{2.615993in}{1.088374in}}%
\pgfpathlineto{\pgfqpoint{2.619128in}{1.088374in}}%
\pgfpathlineto{\pgfqpoint{2.622262in}{1.088374in}}%
\pgfpathlineto{\pgfqpoint{2.624143in}{1.090625in}}%
\pgfpathlineto{\pgfqpoint{2.625397in}{1.092125in}}%
\pgfpathlineto{\pgfqpoint{2.628532in}{1.092125in}}%
\pgfpathlineto{\pgfqpoint{2.631667in}{1.092125in}}%
\pgfpathlineto{\pgfqpoint{2.633548in}{1.094376in}}%
\pgfpathlineto{\pgfqpoint{2.634801in}{1.095877in}}%
\pgfpathlineto{\pgfqpoint{2.637936in}{1.095877in}}%
\pgfpathlineto{\pgfqpoint{2.641071in}{1.095877in}}%
\pgfpathlineto{\pgfqpoint{2.642952in}{1.098128in}}%
\pgfpathlineto{\pgfqpoint{2.644206in}{1.099628in}}%
\pgfpathlineto{\pgfqpoint{2.647340in}{1.099628in}}%
\pgfpathlineto{\pgfqpoint{2.649221in}{1.101879in}}%
\pgfpathlineto{\pgfqpoint{2.650475in}{1.103380in}}%
\pgfpathlineto{\pgfqpoint{2.653610in}{1.103380in}}%
\pgfpathlineto{\pgfqpoint{2.656745in}{1.103380in}}%
\pgfpathlineto{\pgfqpoint{2.658625in}{1.105631in}}%
\pgfpathlineto{\pgfqpoint{2.659879in}{1.107131in}}%
\pgfpathlineto{\pgfqpoint{2.663014in}{1.107131in}}%
\pgfpathlineto{\pgfqpoint{2.666149in}{1.107131in}}%
\pgfpathlineto{\pgfqpoint{2.668030in}{1.109382in}}%
\pgfpathlineto{\pgfqpoint{2.669284in}{1.110883in}}%
\pgfpathlineto{\pgfqpoint{2.672418in}{1.110883in}}%
\pgfpathlineto{\pgfqpoint{2.675553in}{1.110883in}}%
\pgfpathlineto{\pgfqpoint{2.677434in}{1.113134in}}%
\pgfpathlineto{\pgfqpoint{2.678688in}{1.114635in}}%
\pgfpathlineto{\pgfqpoint{2.681823in}{1.114635in}}%
\pgfpathlineto{\pgfqpoint{2.683703in}{1.116885in}}%
\pgfpathlineto{\pgfqpoint{2.684957in}{1.118386in}}%
\pgfpathlineto{\pgfqpoint{2.688092in}{1.118386in}}%
\pgfpathlineto{\pgfqpoint{2.691227in}{1.118386in}}%
\pgfpathlineto{\pgfqpoint{2.693108in}{1.120637in}}%
\pgfpathlineto{\pgfqpoint{2.694362in}{1.122138in}}%
\pgfpathlineto{\pgfqpoint{2.697496in}{1.122138in}}%
\pgfpathlineto{\pgfqpoint{2.700631in}{1.122138in}}%
\pgfpathlineto{\pgfqpoint{2.702512in}{1.124388in}}%
\pgfpathlineto{\pgfqpoint{2.703766in}{1.125889in}}%
\pgfpathlineto{\pgfqpoint{2.706901in}{1.125889in}}%
\pgfpathlineto{\pgfqpoint{2.710035in}{1.125889in}}%
\pgfpathlineto{\pgfqpoint{2.711916in}{1.128140in}}%
\pgfpathlineto{\pgfqpoint{2.713170in}{1.129641in}}%
\pgfpathlineto{\pgfqpoint{2.716305in}{1.129641in}}%
\pgfpathlineto{\pgfqpoint{2.718186in}{1.131892in}}%
\pgfpathlineto{\pgfqpoint{2.719439in}{1.133392in}}%
\pgfpathlineto{\pgfqpoint{2.722574in}{1.133392in}}%
\pgfpathlineto{\pgfqpoint{2.725709in}{1.133392in}}%
\pgfpathlineto{\pgfqpoint{2.727590in}{1.135643in}}%
\pgfpathlineto{\pgfqpoint{2.728844in}{1.137144in}}%
\pgfpathlineto{\pgfqpoint{2.731978in}{1.137144in}}%
\pgfpathlineto{\pgfqpoint{2.735113in}{1.137144in}}%
\pgfpathlineto{\pgfqpoint{2.736994in}{1.139395in}}%
\pgfpathlineto{\pgfqpoint{2.738248in}{1.140895in}}%
\pgfpathlineto{\pgfqpoint{2.741383in}{1.140895in}}%
\pgfpathlineto{\pgfqpoint{2.744517in}{1.140895in}}%
\pgfpathlineto{\pgfqpoint{2.746398in}{1.143146in}}%
\pgfpathlineto{\pgfqpoint{2.747652in}{1.144647in}}%
\pgfpathlineto{\pgfqpoint{2.750787in}{1.144647in}}%
\pgfpathlineto{\pgfqpoint{2.753922in}{1.144647in}}%
\pgfpathlineto{\pgfqpoint{2.755803in}{1.146898in}}%
\pgfpathlineto{\pgfqpoint{2.757056in}{1.148398in}}%
\pgfpathlineto{\pgfqpoint{2.760191in}{1.148398in}}%
\pgfpathlineto{\pgfqpoint{2.762072in}{1.150649in}}%
\pgfpathlineto{\pgfqpoint{2.763326in}{1.152150in}}%
\pgfpathlineto{\pgfqpoint{2.766461in}{1.152150in}}%
\pgfpathlineto{\pgfqpoint{2.769595in}{1.152150in}}%
\pgfpathlineto{\pgfqpoint{2.771476in}{1.154401in}}%
\pgfpathlineto{\pgfqpoint{2.772730in}{1.155901in}}%
\pgfpathlineto{\pgfqpoint{2.775865in}{1.155901in}}%
\pgfpathlineto{\pgfqpoint{2.779000in}{1.155901in}}%
\pgfpathlineto{\pgfqpoint{2.780880in}{1.158152in}}%
\pgfpathlineto{\pgfqpoint{2.782134in}{1.159653in}}%
\pgfpathlineto{\pgfqpoint{2.785269in}{1.159653in}}%
\pgfpathlineto{\pgfqpoint{2.788404in}{1.159653in}}%
\pgfpathlineto{\pgfqpoint{2.790285in}{1.161904in}}%
\pgfpathlineto{\pgfqpoint{2.791539in}{1.163404in}}%
\pgfpathlineto{\pgfqpoint{2.794673in}{1.163404in}}%
\pgfpathlineto{\pgfqpoint{2.796554in}{1.165655in}}%
\pgfpathlineto{\pgfqpoint{2.797808in}{1.167156in}}%
\pgfpathlineto{\pgfqpoint{2.800943in}{1.167156in}}%
\pgfpathlineto{\pgfqpoint{2.804078in}{1.167156in}}%
\pgfpathlineto{\pgfqpoint{2.805958in}{1.169407in}}%
\pgfpathlineto{\pgfqpoint{2.807212in}{1.170908in}}%
\pgfpathlineto{\pgfqpoint{2.810347in}{1.170908in}}%
\pgfpathlineto{\pgfqpoint{2.813482in}{1.170908in}}%
\pgfpathlineto{\pgfqpoint{2.815363in}{1.173158in}}%
\pgfpathlineto{\pgfqpoint{2.815363in}{1.176910in}}%
\pgfpathlineto{\pgfqpoint{2.816617in}{1.178411in}}%
\pgfpathlineto{\pgfqpoint{2.818497in}{1.180662in}}%
\pgfpathlineto{\pgfqpoint{2.818497in}{1.184413in}}%
\pgfpathlineto{\pgfqpoint{2.818497in}{1.188165in}}%
\pgfpathlineto{\pgfqpoint{2.819751in}{1.189665in}}%
\pgfpathlineto{\pgfqpoint{2.821632in}{1.191916in}}%
\pgfpathlineto{\pgfqpoint{2.821632in}{1.195668in}}%
\pgfpathlineto{\pgfqpoint{2.821632in}{1.199419in}}%
\pgfpathlineto{\pgfqpoint{2.822886in}{1.200920in}}%
\pgfpathlineto{\pgfqpoint{2.824767in}{1.203171in}}%
\pgfpathlineto{\pgfqpoint{2.824767in}{1.206922in}}%
\pgfpathlineto{\pgfqpoint{2.824767in}{1.210674in}}%
\pgfpathlineto{\pgfqpoint{2.826021in}{1.212174in}}%
\pgfpathlineto{\pgfqpoint{2.827902in}{1.214425in}}%
\pgfpathlineto{\pgfqpoint{2.827902in}{1.218177in}}%
\pgfpathlineto{\pgfqpoint{2.827902in}{1.221928in}}%
\pgfpathlineto{\pgfqpoint{2.829155in}{1.223429in}}%
\pgfpathlineto{\pgfqpoint{2.831036in}{1.225680in}}%
\pgfpathlineto{\pgfqpoint{2.831036in}{1.229431in}}%
\pgfpathlineto{\pgfqpoint{2.831036in}{1.233183in}}%
\pgfpathlineto{\pgfqpoint{2.832290in}{1.234684in}}%
\pgfpathlineto{\pgfqpoint{2.834171in}{1.236935in}}%
\pgfpathlineto{\pgfqpoint{2.834171in}{1.240686in}}%
\pgfpathlineto{\pgfqpoint{2.834171in}{1.244438in}}%
\pgfpathlineto{\pgfqpoint{2.835425in}{1.245938in}}%
\pgfpathlineto{\pgfqpoint{2.837306in}{1.248189in}}%
\pgfpathlineto{\pgfqpoint{2.837306in}{1.251941in}}%
\pgfpathlineto{\pgfqpoint{2.837306in}{1.255692in}}%
\pgfpathlineto{\pgfqpoint{2.837306in}{1.259444in}}%
\pgfpathlineto{\pgfqpoint{2.838560in}{1.260944in}}%
\pgfpathlineto{\pgfqpoint{2.840441in}{1.263195in}}%
\pgfpathlineto{\pgfqpoint{2.840441in}{1.266947in}}%
\pgfpathlineto{\pgfqpoint{2.840441in}{1.270698in}}%
\pgfpathlineto{\pgfqpoint{2.841694in}{1.272199in}}%
\pgfpathlineto{\pgfqpoint{2.843575in}{1.274450in}}%
\pgfpathlineto{\pgfqpoint{2.843575in}{1.278201in}}%
\pgfpathlineto{\pgfqpoint{2.843575in}{1.281953in}}%
\pgfpathlineto{\pgfqpoint{2.844829in}{1.283454in}}%
\pgfpathlineto{\pgfqpoint{2.846710in}{1.285704in}}%
\pgfpathlineto{\pgfqpoint{2.846710in}{1.289456in}}%
\pgfpathlineto{\pgfqpoint{2.846710in}{1.293208in}}%
\pgfpathlineto{\pgfqpoint{2.847964in}{1.294708in}}%
\pgfpathlineto{\pgfqpoint{2.849845in}{1.296959in}}%
\pgfpathlineto{\pgfqpoint{2.849845in}{1.300711in}}%
\pgfpathlineto{\pgfqpoint{2.849845in}{1.304462in}}%
\pgfpathlineto{\pgfqpoint{2.851099in}{1.305963in}}%
\pgfpathlineto{\pgfqpoint{2.852980in}{1.308214in}}%
\pgfpathlineto{\pgfqpoint{2.852980in}{1.311965in}}%
\pgfpathlineto{\pgfqpoint{2.852980in}{1.315717in}}%
\pgfpathlineto{\pgfqpoint{2.854233in}{1.317217in}}%
\pgfpathlineto{\pgfqpoint{2.856114in}{1.319468in}}%
\pgfpathlineto{\pgfqpoint{2.856114in}{1.323220in}}%
\pgfpathlineto{\pgfqpoint{2.856114in}{1.326971in}}%
\pgfpathlineto{\pgfqpoint{2.857368in}{1.328472in}}%
\pgfpathlineto{\pgfqpoint{2.859249in}{1.330723in}}%
\pgfpathlineto{\pgfqpoint{2.859249in}{1.334474in}}%
\pgfpathlineto{\pgfqpoint{2.859249in}{1.338226in}}%
\pgfpathlineto{\pgfqpoint{2.860503in}{1.339727in}}%
\pgfpathlineto{\pgfqpoint{2.862384in}{1.341977in}}%
\pgfpathlineto{\pgfqpoint{2.862384in}{1.345729in}}%
\pgfpathlineto{\pgfqpoint{2.862384in}{1.349481in}}%
\pgfpathlineto{\pgfqpoint{2.863638in}{1.350981in}}%
\pgfpathlineto{\pgfqpoint{2.865518in}{1.353232in}}%
\pgfpathlineto{\pgfqpoint{2.865518in}{1.356984in}}%
\pgfpathlineto{\pgfqpoint{2.865518in}{1.360735in}}%
\pgfpathlineto{\pgfqpoint{2.865518in}{1.364487in}}%
\pgfpathlineto{\pgfqpoint{2.866772in}{1.365987in}}%
\pgfpathlineto{\pgfqpoint{2.868653in}{1.368238in}}%
\pgfpathlineto{\pgfqpoint{2.868653in}{1.371990in}}%
\pgfpathlineto{\pgfqpoint{2.868653in}{1.375741in}}%
\pgfpathlineto{\pgfqpoint{2.869907in}{1.377242in}}%
\pgfpathlineto{\pgfqpoint{2.871788in}{1.379493in}}%
\pgfpathlineto{\pgfqpoint{2.871788in}{1.383244in}}%
\pgfpathlineto{\pgfqpoint{2.871788in}{1.386996in}}%
\pgfpathlineto{\pgfqpoint{2.873042in}{1.388497in}}%
\pgfpathlineto{\pgfqpoint{2.874923in}{1.390747in}}%
\pgfpathlineto{\pgfqpoint{2.874923in}{1.394499in}}%
\pgfpathlineto{\pgfqpoint{2.874923in}{1.398251in}}%
\pgfpathlineto{\pgfqpoint{2.876177in}{1.399751in}}%
\pgfpathlineto{\pgfqpoint{2.878057in}{1.402002in}}%
\pgfpathlineto{\pgfqpoint{2.878057in}{1.405754in}}%
\pgfpathlineto{\pgfqpoint{2.878057in}{1.409505in}}%
\pgfpathlineto{\pgfqpoint{2.879311in}{1.411006in}}%
\pgfpathlineto{\pgfqpoint{2.881192in}{1.413257in}}%
\pgfpathlineto{\pgfqpoint{2.881192in}{1.417008in}}%
\pgfpathlineto{\pgfqpoint{2.881192in}{1.420760in}}%
\pgfpathlineto{\pgfqpoint{2.882446in}{1.422260in}}%
\pgfpathlineto{\pgfqpoint{2.884327in}{1.424511in}}%
\pgfpathlineto{\pgfqpoint{2.884327in}{1.428263in}}%
\pgfpathlineto{\pgfqpoint{2.884327in}{1.432014in}}%
\pgfpathlineto{\pgfqpoint{2.885581in}{1.433515in}}%
\pgfpathlineto{\pgfqpoint{2.887462in}{1.435766in}}%
\pgfpathlineto{\pgfqpoint{2.887462in}{1.439517in}}%
\pgfpathlineto{\pgfqpoint{2.887462in}{1.443269in}}%
\pgfpathlineto{\pgfqpoint{2.888716in}{1.444770in}}%
\pgfpathlineto{\pgfqpoint{2.890596in}{1.447020in}}%
\pgfpathlineto{\pgfqpoint{2.890596in}{1.450772in}}%
\pgfpathlineto{\pgfqpoint{2.890596in}{1.454524in}}%
\pgfpathlineto{\pgfqpoint{2.890596in}{1.458275in}}%
\pgfpathlineto{\pgfqpoint{2.891850in}{1.459776in}}%
\pgfpathlineto{\pgfqpoint{2.893731in}{1.462027in}}%
\pgfpathlineto{\pgfqpoint{2.893731in}{1.465778in}}%
\pgfpathlineto{\pgfqpoint{2.893731in}{1.469530in}}%
\pgfpathlineto{\pgfqpoint{2.894985in}{1.471030in}}%
\pgfpathlineto{\pgfqpoint{2.896866in}{1.473281in}}%
\pgfpathlineto{\pgfqpoint{2.896866in}{1.477033in}}%
\pgfpathlineto{\pgfqpoint{2.896866in}{1.480784in}}%
\pgfpathlineto{\pgfqpoint{2.898120in}{1.482285in}}%
\pgfpathlineto{\pgfqpoint{2.900001in}{1.484536in}}%
\pgfpathlineto{\pgfqpoint{2.900001in}{1.488287in}}%
\pgfpathlineto{\pgfqpoint{2.900001in}{1.492039in}}%
\pgfpathlineto{\pgfqpoint{2.901255in}{1.493539in}}%
\pgfpathlineto{\pgfqpoint{2.903135in}{1.495790in}}%
\pgfpathlineto{\pgfqpoint{2.903135in}{1.499542in}}%
\pgfpathlineto{\pgfqpoint{2.903135in}{1.503293in}}%
\pgfpathlineto{\pgfqpoint{2.904389in}{1.504794in}}%
\pgfpathlineto{\pgfqpoint{2.906270in}{1.507045in}}%
\pgfpathlineto{\pgfqpoint{2.906270in}{1.510797in}}%
\pgfpathlineto{\pgfqpoint{2.906270in}{1.514548in}}%
\pgfpathlineto{\pgfqpoint{2.907524in}{1.516049in}}%
\pgfpathlineto{\pgfqpoint{2.909405in}{1.518300in}}%
\pgfpathlineto{\pgfqpoint{2.909405in}{1.522051in}}%
\pgfpathlineto{\pgfqpoint{2.909405in}{1.525803in}}%
\pgfpathlineto{\pgfqpoint{2.910659in}{1.527303in}}%
\pgfpathlineto{\pgfqpoint{2.912540in}{1.529554in}}%
\pgfpathlineto{\pgfqpoint{2.912540in}{1.533306in}}%
\pgfpathlineto{\pgfqpoint{2.912540in}{1.537057in}}%
\pgfpathlineto{\pgfqpoint{2.913794in}{1.538558in}}%
\pgfpathlineto{\pgfqpoint{2.915674in}{1.540809in}}%
\pgfpathlineto{\pgfqpoint{2.915674in}{1.544560in}}%
\pgfpathlineto{\pgfqpoint{2.915674in}{1.548312in}}%
\pgfpathlineto{\pgfqpoint{2.916928in}{1.549813in}}%
\pgfpathlineto{\pgfqpoint{2.918809in}{1.552063in}}%
\pgfpathlineto{\pgfqpoint{2.918809in}{1.555815in}}%
\pgfpathlineto{\pgfqpoint{2.918809in}{1.559566in}}%
\pgfpathlineto{\pgfqpoint{2.918809in}{1.563318in}}%
\pgfpathlineto{\pgfqpoint{2.920063in}{1.564819in}}%
\pgfpathlineto{\pgfqpoint{2.921944in}{1.567070in}}%
\pgfpathlineto{\pgfqpoint{2.921944in}{1.570821in}}%
\pgfpathlineto{\pgfqpoint{2.921944in}{1.574573in}}%
\pgfpathlineto{\pgfqpoint{2.923198in}{1.576073in}}%
\pgfpathlineto{\pgfqpoint{2.925079in}{1.578324in}}%
\pgfpathlineto{\pgfqpoint{2.925079in}{1.582076in}}%
\pgfpathlineto{\pgfqpoint{2.925079in}{1.585827in}}%
\pgfpathlineto{\pgfqpoint{2.926332in}{1.587328in}}%
\pgfpathlineto{\pgfqpoint{2.928213in}{1.589579in}}%
\pgfpathlineto{\pgfqpoint{2.928213in}{1.593330in}}%
\pgfpathlineto{\pgfqpoint{2.928213in}{1.597082in}}%
\pgfpathlineto{\pgfqpoint{2.929467in}{1.598582in}}%
\pgfpathlineto{\pgfqpoint{2.931348in}{1.600833in}}%
\pgfpathlineto{\pgfqpoint{2.931348in}{1.604585in}}%
\pgfpathlineto{\pgfqpoint{2.931348in}{1.608336in}}%
\pgfpathlineto{\pgfqpoint{2.932602in}{1.609837in}}%
\pgfpathlineto{\pgfqpoint{2.934483in}{1.612088in}}%
\pgfpathlineto{\pgfqpoint{2.934483in}{1.615840in}}%
\pgfpathlineto{\pgfqpoint{2.934483in}{1.619591in}}%
\pgfpathlineto{\pgfqpoint{2.935737in}{1.621092in}}%
\pgfpathlineto{\pgfqpoint{2.937618in}{1.623343in}}%
\pgfpathlineto{\pgfqpoint{2.937618in}{1.627094in}}%
\pgfpathlineto{\pgfqpoint{2.937618in}{1.630846in}}%
\pgfpathlineto{\pgfqpoint{2.938871in}{1.632346in}}%
\pgfpathlineto{\pgfqpoint{2.940752in}{1.634597in}}%
\pgfpathlineto{\pgfqpoint{2.940752in}{1.638349in}}%
\pgfpathlineto{\pgfqpoint{2.940752in}{1.642100in}}%
\pgfpathlineto{\pgfqpoint{2.942006in}{1.643601in}}%
\pgfpathlineto{\pgfqpoint{2.943887in}{1.645852in}}%
\pgfpathlineto{\pgfqpoint{2.943887in}{1.649603in}}%
\pgfpathlineto{\pgfqpoint{2.943887in}{1.653355in}}%
\pgfpathlineto{\pgfqpoint{2.943887in}{1.657106in}}%
\pgfpathlineto{\pgfqpoint{2.945141in}{1.658607in}}%
\pgfpathlineto{\pgfqpoint{2.947022in}{1.660858in}}%
\pgfpathlineto{\pgfqpoint{2.947022in}{1.664609in}}%
\pgfpathlineto{\pgfqpoint{2.947022in}{1.668361in}}%
\pgfpathlineto{\pgfqpoint{2.948276in}{1.669862in}}%
\pgfpathlineto{\pgfqpoint{2.950157in}{1.672113in}}%
\pgfpathlineto{\pgfqpoint{2.950157in}{1.675864in}}%
\pgfpathlineto{\pgfqpoint{2.950157in}{1.679616in}}%
\pgfpathlineto{\pgfqpoint{2.951410in}{1.681116in}}%
\pgfpathlineto{\pgfqpoint{2.953291in}{1.683367in}}%
\pgfpathlineto{\pgfqpoint{2.953291in}{1.687119in}}%
\pgfpathlineto{\pgfqpoint{2.953291in}{1.690870in}}%
\pgfpathlineto{\pgfqpoint{2.954545in}{1.692371in}}%
\pgfpathlineto{\pgfqpoint{2.956426in}{1.694622in}}%
\pgfpathlineto{\pgfqpoint{2.956426in}{1.698373in}}%
\pgfpathlineto{\pgfqpoint{2.956426in}{1.702125in}}%
\pgfpathlineto{\pgfqpoint{2.957680in}{1.703625in}}%
\pgfpathlineto{\pgfqpoint{2.959561in}{1.705876in}}%
\pgfpathlineto{\pgfqpoint{2.959561in}{1.709628in}}%
\pgfpathlineto{\pgfqpoint{2.959561in}{1.713379in}}%
\pgfpathlineto{\pgfqpoint{2.960815in}{1.714880in}}%
\pgfpathlineto{\pgfqpoint{2.962696in}{1.717131in}}%
\pgfpathlineto{\pgfqpoint{2.962696in}{1.720882in}}%
\pgfpathlineto{\pgfqpoint{2.962696in}{1.724634in}}%
\pgfpathlineto{\pgfqpoint{2.963949in}{1.726135in}}%
\pgfpathlineto{\pgfqpoint{2.965830in}{1.728386in}}%
\pgfpathlineto{\pgfqpoint{2.965830in}{1.732137in}}%
\pgfpathlineto{\pgfqpoint{2.965830in}{1.735889in}}%
\pgfpathlineto{\pgfqpoint{2.967084in}{1.737389in}}%
\pgfpathlineto{\pgfqpoint{2.968965in}{1.739640in}}%
\pgfpathlineto{\pgfqpoint{2.968965in}{1.743392in}}%
\pgfpathlineto{\pgfqpoint{2.968965in}{1.747143in}}%
\pgfpathlineto{\pgfqpoint{2.970219in}{1.748644in}}%
\pgfpathlineto{\pgfqpoint{2.972100in}{1.750895in}}%
\pgfpathlineto{\pgfqpoint{2.972100in}{1.754646in}}%
\pgfpathlineto{\pgfqpoint{2.972100in}{1.758398in}}%
\pgfpathlineto{\pgfqpoint{2.972100in}{1.762149in}}%
\pgfpathlineto{\pgfqpoint{2.973354in}{1.763650in}}%
\pgfpathlineto{\pgfqpoint{2.975234in}{1.765901in}}%
\pgfpathlineto{\pgfqpoint{2.975234in}{1.769652in}}%
\pgfpathlineto{\pgfqpoint{2.975234in}{1.773404in}}%
\pgfpathlineto{\pgfqpoint{2.976488in}{1.774905in}}%
\pgfpathlineto{\pgfqpoint{2.978369in}{1.777155in}}%
\pgfpathlineto{\pgfqpoint{2.978369in}{1.780907in}}%
\pgfpathlineto{\pgfqpoint{2.978369in}{1.784659in}}%
\pgfpathlineto{\pgfqpoint{2.979623in}{1.786159in}}%
\pgfpathlineto{\pgfqpoint{2.981504in}{1.788410in}}%
\pgfpathlineto{\pgfqpoint{2.981504in}{1.792162in}}%
\pgfpathlineto{\pgfqpoint{2.981504in}{1.795913in}}%
\pgfpathlineto{\pgfqpoint{2.982758in}{1.797414in}}%
\pgfpathlineto{\pgfqpoint{2.984639in}{1.799665in}}%
\pgfpathlineto{\pgfqpoint{2.984639in}{1.803416in}}%
\pgfpathlineto{\pgfqpoint{2.984639in}{1.807168in}}%
\pgfpathlineto{\pgfqpoint{2.985893in}{1.808668in}}%
\pgfpathlineto{\pgfqpoint{2.987773in}{1.810919in}}%
\pgfpathlineto{\pgfqpoint{2.987773in}{1.814671in}}%
\pgfpathlineto{\pgfqpoint{2.987773in}{1.818422in}}%
\pgfpathlineto{\pgfqpoint{2.989027in}{1.819923in}}%
\pgfpathlineto{\pgfqpoint{2.990908in}{1.822174in}}%
\pgfpathlineto{\pgfqpoint{2.990908in}{1.825925in}}%
\pgfpathlineto{\pgfqpoint{2.990908in}{1.829677in}}%
\pgfpathlineto{\pgfqpoint{2.992162in}{1.831178in}}%
\pgfpathlineto{\pgfqpoint{2.994043in}{1.833429in}}%
\pgfpathlineto{\pgfqpoint{2.994043in}{1.837180in}}%
\pgfpathlineto{\pgfqpoint{2.994043in}{1.840932in}}%
\pgfpathlineto{\pgfqpoint{2.995297in}{1.842432in}}%
\pgfpathlineto{\pgfqpoint{2.997178in}{1.844683in}}%
\pgfpathlineto{\pgfqpoint{2.997178in}{1.848435in}}%
\pgfpathlineto{\pgfqpoint{2.997178in}{1.852186in}}%
\pgfpathlineto{\pgfqpoint{2.997178in}{1.855938in}}%
\pgfpathlineto{\pgfqpoint{2.998432in}{1.857438in}}%
\pgfpathlineto{\pgfqpoint{3.000312in}{1.859689in}}%
\pgfpathlineto{\pgfqpoint{3.000312in}{1.863441in}}%
\pgfpathlineto{\pgfqpoint{3.000312in}{1.867192in}}%
\pgfpathlineto{\pgfqpoint{3.001566in}{1.868693in}}%
\pgfpathlineto{\pgfqpoint{3.003447in}{1.870944in}}%
\pgfpathlineto{\pgfqpoint{3.003447in}{1.874695in}}%
\pgfpathlineto{\pgfqpoint{3.003447in}{1.878447in}}%
\pgfpathlineto{\pgfqpoint{3.004701in}{1.879948in}}%
\pgfpathlineto{\pgfqpoint{3.006582in}{1.882198in}}%
\pgfpathlineto{\pgfqpoint{3.006582in}{1.885950in}}%
\pgfpathlineto{\pgfqpoint{3.006582in}{1.889702in}}%
\pgfpathlineto{\pgfqpoint{3.007836in}{1.891202in}}%
\pgfpathlineto{\pgfqpoint{3.009717in}{1.893453in}}%
\pgfpathlineto{\pgfqpoint{3.009717in}{1.897205in}}%
\pgfpathlineto{\pgfqpoint{3.009717in}{1.900956in}}%
\pgfpathlineto{\pgfqpoint{3.010971in}{1.902457in}}%
\pgfpathlineto{\pgfqpoint{3.012851in}{1.904708in}}%
\pgfpathlineto{\pgfqpoint{3.012851in}{1.908459in}}%
\pgfpathlineto{\pgfqpoint{3.012851in}{1.912211in}}%
\pgfpathlineto{\pgfqpoint{3.014105in}{1.913711in}}%
\pgfpathlineto{\pgfqpoint{3.015986in}{1.915962in}}%
\pgfpathlineto{\pgfqpoint{3.015986in}{1.919714in}}%
\pgfpathlineto{\pgfqpoint{3.015986in}{1.923465in}}%
\pgfpathlineto{\pgfqpoint{3.017240in}{1.924966in}}%
\pgfpathlineto{\pgfqpoint{3.019121in}{1.927217in}}%
\pgfpathlineto{\pgfqpoint{3.019121in}{1.930968in}}%
\pgfpathlineto{\pgfqpoint{3.019121in}{1.934720in}}%
\pgfpathlineto{\pgfqpoint{3.020375in}{1.936221in}}%
\pgfpathlineto{\pgfqpoint{3.022256in}{1.938471in}}%
\pgfpathlineto{\pgfqpoint{3.022256in}{1.942223in}}%
\pgfpathlineto{\pgfqpoint{3.022256in}{1.945975in}}%
\pgfpathlineto{\pgfqpoint{3.023510in}{1.947475in}}%
\pgfpathlineto{\pgfqpoint{3.025390in}{1.949726in}}%
\pgfpathlineto{\pgfqpoint{3.025390in}{1.953478in}}%
\pgfpathlineto{\pgfqpoint{3.025390in}{1.957229in}}%
\pgfpathlineto{\pgfqpoint{3.025390in}{1.960981in}}%
\pgfpathlineto{\pgfqpoint{3.026644in}{1.962481in}}%
\pgfpathlineto{\pgfqpoint{3.028525in}{1.964732in}}%
\pgfpathlineto{\pgfqpoint{3.028525in}{1.968484in}}%
\pgfpathlineto{\pgfqpoint{3.028525in}{1.972235in}}%
\pgfpathlineto{\pgfqpoint{3.029779in}{1.973736in}}%
\pgfpathlineto{\pgfqpoint{3.031660in}{1.975987in}}%
\pgfpathlineto{\pgfqpoint{3.031660in}{1.979738in}}%
\pgfpathlineto{\pgfqpoint{3.031660in}{1.983490in}}%
\pgfpathlineto{\pgfqpoint{3.032914in}{1.984991in}}%
\pgfpathlineto{\pgfqpoint{3.034795in}{1.987241in}}%
\pgfpathlineto{\pgfqpoint{3.034795in}{1.990993in}}%
\pgfpathlineto{\pgfqpoint{3.034795in}{1.994745in}}%
\pgfpathlineto{\pgfqpoint{3.036048in}{1.996245in}}%
\pgfpathlineto{\pgfqpoint{3.037929in}{1.998496in}}%
\pgfpathlineto{\pgfqpoint{3.037929in}{2.002248in}}%
\pgfpathlineto{\pgfqpoint{3.037929in}{2.005999in}}%
\pgfpathlineto{\pgfqpoint{3.039183in}{2.007500in}}%
\pgfpathlineto{\pgfqpoint{3.041064in}{2.009751in}}%
\pgfpathlineto{\pgfqpoint{3.041064in}{2.013502in}}%
\pgfpathlineto{\pgfqpoint{3.041064in}{2.017254in}}%
\pgfpathlineto{\pgfqpoint{3.042318in}{2.018754in}}%
\pgfpathlineto{\pgfqpoint{3.044199in}{2.021005in}}%
\pgfpathlineto{\pgfqpoint{3.044199in}{2.024757in}}%
\pgfpathlineto{\pgfqpoint{3.044199in}{2.028508in}}%
\pgfpathlineto{\pgfqpoint{3.045453in}{2.030009in}}%
\pgfpathlineto{\pgfqpoint{3.047334in}{2.032260in}}%
\pgfpathlineto{\pgfqpoint{3.047334in}{2.036011in}}%
\pgfpathlineto{\pgfqpoint{3.047334in}{2.039763in}}%
\pgfpathlineto{\pgfqpoint{3.048587in}{2.041264in}}%
\pgfpathlineto{\pgfqpoint{3.050468in}{2.043514in}}%
\pgfpathlineto{\pgfqpoint{3.050468in}{2.047266in}}%
\pgfpathlineto{\pgfqpoint{3.050468in}{2.051018in}}%
\pgfpathlineto{\pgfqpoint{3.050468in}{2.054769in}}%
\pgfpathlineto{\pgfqpoint{3.051722in}{2.056270in}}%
\pgfpathlineto{\pgfqpoint{3.053603in}{2.058521in}}%
\pgfpathlineto{\pgfqpoint{3.053603in}{2.062272in}}%
\pgfpathlineto{\pgfqpoint{3.053603in}{2.066024in}}%
\pgfpathlineto{\pgfqpoint{3.054857in}{2.067524in}}%
\pgfpathlineto{\pgfqpoint{3.056738in}{2.069775in}}%
\pgfpathlineto{\pgfqpoint{3.056738in}{2.073527in}}%
\pgfpathlineto{\pgfqpoint{3.056738in}{2.077278in}}%
\pgfpathlineto{\pgfqpoint{3.057992in}{2.078779in}}%
\pgfpathlineto{\pgfqpoint{3.059873in}{2.081030in}}%
\pgfpathlineto{\pgfqpoint{3.059873in}{2.084781in}}%
\pgfpathlineto{\pgfqpoint{3.059873in}{2.088533in}}%
\pgfpathlineto{\pgfqpoint{3.061126in}{2.090033in}}%
\pgfpathlineto{\pgfqpoint{3.063007in}{2.092284in}}%
\pgfpathlineto{\pgfqpoint{3.063007in}{2.096036in}}%
\pgfpathlineto{\pgfqpoint{3.063007in}{2.099787in}}%
\pgfpathlineto{\pgfqpoint{3.064261in}{2.101288in}}%
\pgfpathlineto{\pgfqpoint{3.066142in}{2.103539in}}%
\pgfpathlineto{\pgfqpoint{3.066142in}{2.107291in}}%
\pgfpathlineto{\pgfqpoint{3.066142in}{2.111042in}}%
\pgfpathlineto{\pgfqpoint{3.067396in}{2.112543in}}%
\pgfpathlineto{\pgfqpoint{3.069277in}{2.114794in}}%
\pgfpathlineto{\pgfqpoint{3.069277in}{2.118545in}}%
\pgfpathlineto{\pgfqpoint{3.069277in}{2.122297in}}%
\pgfpathlineto{\pgfqpoint{3.070531in}{2.123797in}}%
\pgfpathlineto{\pgfqpoint{3.072411in}{2.126048in}}%
\pgfpathlineto{\pgfqpoint{3.072411in}{2.129800in}}%
\pgfpathlineto{\pgfqpoint{3.072411in}{2.133551in}}%
\pgfpathlineto{\pgfqpoint{3.073665in}{2.135052in}}%
\pgfpathlineto{\pgfqpoint{3.075546in}{2.137303in}}%
\pgfpathlineto{\pgfqpoint{3.075546in}{2.141054in}}%
\pgfpathlineto{\pgfqpoint{3.075546in}{2.144806in}}%
\pgfpathlineto{\pgfqpoint{3.076800in}{2.146306in}}%
\pgfpathlineto{\pgfqpoint{3.078681in}{2.148557in}}%
\pgfpathlineto{\pgfqpoint{3.078681in}{2.152309in}}%
\pgfpathlineto{\pgfqpoint{3.078681in}{2.156060in}}%
\pgfpathlineto{\pgfqpoint{3.078681in}{2.159812in}}%
\pgfpathlineto{\pgfqpoint{3.079935in}{2.161313in}}%
\pgfpathlineto{\pgfqpoint{3.081816in}{2.163564in}}%
\pgfpathlineto{\pgfqpoint{3.081816in}{2.167315in}}%
\pgfpathlineto{\pgfqpoint{3.081816in}{2.171067in}}%
\pgfpathlineto{\pgfqpoint{3.083070in}{2.172567in}}%
\pgfpathlineto{\pgfqpoint{3.084950in}{2.174818in}}%
\pgfpathlineto{\pgfqpoint{3.084950in}{2.178570in}}%
\pgfpathlineto{\pgfqpoint{3.084950in}{2.182321in}}%
\pgfpathlineto{\pgfqpoint{3.086204in}{2.183822in}}%
\pgfpathlineto{\pgfqpoint{3.088085in}{2.186073in}}%
\pgfpathlineto{\pgfqpoint{3.088085in}{2.189824in}}%
\pgfpathlineto{\pgfqpoint{3.088085in}{2.193576in}}%
\pgfpathlineto{\pgfqpoint{3.089339in}{2.195076in}}%
\pgfpathlineto{\pgfqpoint{3.091220in}{2.197327in}}%
\pgfpathlineto{\pgfqpoint{3.091220in}{2.201079in}}%
\pgfpathlineto{\pgfqpoint{3.091220in}{2.204830in}}%
\pgfpathlineto{\pgfqpoint{3.092474in}{2.206331in}}%
\pgfpathlineto{\pgfqpoint{3.094355in}{2.208582in}}%
\pgfpathlineto{\pgfqpoint{3.094355in}{2.212334in}}%
\pgfpathlineto{\pgfqpoint{3.092474in}{2.214584in}}%
\pgfpathlineto{\pgfqpoint{3.091220in}{2.216085in}}%
\pgfpathlineto{\pgfqpoint{3.089339in}{2.218336in}}%
\pgfpathlineto{\pgfqpoint{3.088085in}{2.219837in}}%
\pgfpathlineto{\pgfqpoint{3.086204in}{2.222087in}}%
\pgfpathlineto{\pgfqpoint{3.084950in}{2.223588in}}%
\pgfpathlineto{\pgfqpoint{3.083070in}{2.225839in}}%
\pgfpathlineto{\pgfqpoint{3.081816in}{2.227340in}}%
\pgfpathlineto{\pgfqpoint{3.079935in}{2.229591in}}%
\pgfpathlineto{\pgfqpoint{3.078681in}{2.231091in}}%
\pgfpathlineto{\pgfqpoint{3.076800in}{2.233342in}}%
\pgfpathlineto{\pgfqpoint{3.075546in}{2.234843in}}%
\pgfpathlineto{\pgfqpoint{3.073665in}{2.237094in}}%
\pgfpathlineto{\pgfqpoint{3.072411in}{2.238594in}}%
\pgfpathlineto{\pgfqpoint{3.070531in}{2.240845in}}%
\pgfpathlineto{\pgfqpoint{3.069277in}{2.242346in}}%
\pgfpathlineto{\pgfqpoint{3.067396in}{2.244597in}}%
\pgfpathlineto{\pgfqpoint{3.066142in}{2.246097in}}%
\pgfpathlineto{\pgfqpoint{3.064261in}{2.248348in}}%
\pgfpathlineto{\pgfqpoint{3.063007in}{2.249849in}}%
\pgfpathlineto{\pgfqpoint{3.061126in}{2.252100in}}%
\pgfpathlineto{\pgfqpoint{3.059873in}{2.253600in}}%
\pgfpathlineto{\pgfqpoint{3.057992in}{2.255851in}}%
\pgfpathlineto{\pgfqpoint{3.056738in}{2.257352in}}%
\pgfpathlineto{\pgfqpoint{3.054857in}{2.259603in}}%
\pgfpathlineto{\pgfqpoint{3.053603in}{2.261103in}}%
\pgfpathlineto{\pgfqpoint{3.053603in}{2.264855in}}%
\pgfpathlineto{\pgfqpoint{3.051722in}{2.267106in}}%
\pgfpathlineto{\pgfqpoint{3.050468in}{2.268607in}}%
\pgfpathlineto{\pgfqpoint{3.048587in}{2.270857in}}%
\pgfpathlineto{\pgfqpoint{3.047334in}{2.272358in}}%
\pgfpathlineto{\pgfqpoint{3.045453in}{2.274609in}}%
\pgfpathlineto{\pgfqpoint{3.044199in}{2.276110in}}%
\pgfpathlineto{\pgfqpoint{3.042318in}{2.278361in}}%
\pgfpathlineto{\pgfqpoint{3.041064in}{2.279861in}}%
\pgfpathlineto{\pgfqpoint{3.039183in}{2.282112in}}%
\pgfpathlineto{\pgfqpoint{3.037929in}{2.283613in}}%
\pgfpathlineto{\pgfqpoint{3.036048in}{2.285864in}}%
\pgfpathlineto{\pgfqpoint{3.034795in}{2.287364in}}%
\pgfpathlineto{\pgfqpoint{3.032914in}{2.289615in}}%
\pgfpathlineto{\pgfqpoint{3.031660in}{2.291116in}}%
\pgfpathlineto{\pgfqpoint{3.029779in}{2.293367in}}%
\pgfpathlineto{\pgfqpoint{3.028525in}{2.294867in}}%
\pgfpathlineto{\pgfqpoint{3.026644in}{2.297118in}}%
\pgfpathlineto{\pgfqpoint{3.025390in}{2.298619in}}%
\pgfpathlineto{\pgfqpoint{3.023510in}{2.300870in}}%
\pgfpathlineto{\pgfqpoint{3.022256in}{2.302370in}}%
\pgfpathlineto{\pgfqpoint{3.020375in}{2.304621in}}%
\pgfpathlineto{\pgfqpoint{3.019121in}{2.306122in}}%
\pgfpathlineto{\pgfqpoint{3.017240in}{2.308373in}}%
\pgfpathlineto{\pgfqpoint{3.015986in}{2.309873in}}%
\pgfpathlineto{\pgfqpoint{3.014105in}{2.312124in}}%
\pgfpathlineto{\pgfqpoint{3.012851in}{2.313625in}}%
\pgfpathlineto{\pgfqpoint{3.010971in}{2.315876in}}%
\pgfpathlineto{\pgfqpoint{3.009717in}{2.317376in}}%
\pgfpathlineto{\pgfqpoint{3.007836in}{2.319627in}}%
\pgfpathlineto{\pgfqpoint{3.006582in}{2.321128in}}%
\pgfpathlineto{\pgfqpoint{3.004701in}{2.323379in}}%
\pgfpathlineto{\pgfqpoint{3.003447in}{2.324880in}}%
\pgfpathlineto{\pgfqpoint{3.003447in}{2.328631in}}%
\pgfpathlineto{\pgfqpoint{3.001566in}{2.330882in}}%
\pgfpathlineto{\pgfqpoint{3.000312in}{2.332383in}}%
\pgfpathlineto{\pgfqpoint{2.998432in}{2.334634in}}%
\pgfpathlineto{\pgfqpoint{2.997178in}{2.336134in}}%
\pgfpathlineto{\pgfqpoint{2.995297in}{2.338385in}}%
\pgfpathlineto{\pgfqpoint{2.994043in}{2.339886in}}%
\pgfpathlineto{\pgfqpoint{2.992162in}{2.342137in}}%
\pgfpathlineto{\pgfqpoint{2.990908in}{2.343637in}}%
\pgfpathlineto{\pgfqpoint{2.989027in}{2.345888in}}%
\pgfpathlineto{\pgfqpoint{2.987773in}{2.347389in}}%
\pgfpathlineto{\pgfqpoint{2.985893in}{2.349640in}}%
\pgfpathlineto{\pgfqpoint{2.984639in}{2.351140in}}%
\pgfpathlineto{\pgfqpoint{2.982758in}{2.353391in}}%
\pgfpathlineto{\pgfqpoint{2.981504in}{2.354892in}}%
\pgfpathlineto{\pgfqpoint{2.979623in}{2.357143in}}%
\pgfpathlineto{\pgfqpoint{2.978369in}{2.358643in}}%
\pgfpathlineto{\pgfqpoint{2.976488in}{2.360894in}}%
\pgfpathlineto{\pgfqpoint{2.975234in}{2.362395in}}%
\pgfpathlineto{\pgfqpoint{2.973354in}{2.364646in}}%
\pgfpathlineto{\pgfqpoint{2.972100in}{2.366146in}}%
\pgfpathlineto{\pgfqpoint{2.970219in}{2.368397in}}%
\pgfpathlineto{\pgfqpoint{2.968965in}{2.369898in}}%
\pgfpathlineto{\pgfqpoint{2.967084in}{2.372149in}}%
\pgfpathlineto{\pgfqpoint{2.965830in}{2.373649in}}%
\pgfpathlineto{\pgfqpoint{2.963949in}{2.375900in}}%
\pgfpathlineto{\pgfqpoint{2.962696in}{2.377401in}}%
\pgfpathlineto{\pgfqpoint{2.960815in}{2.379652in}}%
\pgfpathlineto{\pgfqpoint{2.959561in}{2.381153in}}%
\pgfpathlineto{\pgfqpoint{2.957680in}{2.383403in}}%
\pgfpathlineto{\pgfqpoint{2.956426in}{2.384904in}}%
\pgfpathlineto{\pgfqpoint{2.956426in}{2.388656in}}%
\pgfpathlineto{\pgfqpoint{2.954545in}{2.390907in}}%
\pgfpathlineto{\pgfqpoint{2.953291in}{2.392407in}}%
\pgfpathlineto{\pgfqpoint{2.951410in}{2.394658in}}%
\pgfpathlineto{\pgfqpoint{2.950157in}{2.396159in}}%
\pgfpathlineto{\pgfqpoint{2.948276in}{2.398410in}}%
\pgfpathlineto{\pgfqpoint{2.947022in}{2.399910in}}%
\pgfpathlineto{\pgfqpoint{2.945141in}{2.402161in}}%
\pgfpathlineto{\pgfqpoint{2.943887in}{2.403662in}}%
\pgfpathlineto{\pgfqpoint{2.942006in}{2.405913in}}%
\pgfpathlineto{\pgfqpoint{2.940752in}{2.407413in}}%
\pgfpathlineto{\pgfqpoint{2.938871in}{2.409664in}}%
\pgfpathlineto{\pgfqpoint{2.937618in}{2.411165in}}%
\pgfpathlineto{\pgfqpoint{2.935737in}{2.413416in}}%
\pgfpathlineto{\pgfqpoint{2.934483in}{2.414916in}}%
\pgfpathlineto{\pgfqpoint{2.932602in}{2.417167in}}%
\pgfpathlineto{\pgfqpoint{2.931348in}{2.418668in}}%
\pgfpathlineto{\pgfqpoint{2.929467in}{2.420919in}}%
\pgfpathlineto{\pgfqpoint{2.928213in}{2.422419in}}%
\pgfpathlineto{\pgfqpoint{2.926332in}{2.424670in}}%
\pgfpathlineto{\pgfqpoint{2.925079in}{2.426171in}}%
\pgfpathlineto{\pgfqpoint{2.923198in}{2.428422in}}%
\pgfpathlineto{\pgfqpoint{2.921944in}{2.429923in}}%
\pgfpathlineto{\pgfqpoint{2.920063in}{2.432173in}}%
\pgfpathlineto{\pgfqpoint{2.918809in}{2.433674in}}%
\pgfpathlineto{\pgfqpoint{2.916928in}{2.435925in}}%
\pgfpathlineto{\pgfqpoint{2.915674in}{2.437426in}}%
\pgfpathlineto{\pgfqpoint{2.913794in}{2.439676in}}%
\pgfpathlineto{\pgfqpoint{2.912540in}{2.441177in}}%
\pgfpathlineto{\pgfqpoint{2.910659in}{2.443428in}}%
\pgfpathlineto{\pgfqpoint{2.909405in}{2.444929in}}%
\pgfpathlineto{\pgfqpoint{2.907524in}{2.447180in}}%
\pgfpathlineto{\pgfqpoint{2.906270in}{2.448680in}}%
\pgfpathlineto{\pgfqpoint{2.906270in}{2.452432in}}%
\pgfpathlineto{\pgfqpoint{2.904389in}{2.454683in}}%
\pgfpathlineto{\pgfqpoint{2.903135in}{2.456183in}}%
\pgfpathlineto{\pgfqpoint{2.901255in}{2.458434in}}%
\pgfpathlineto{\pgfqpoint{2.900001in}{2.459935in}}%
\pgfpathlineto{\pgfqpoint{2.898120in}{2.462186in}}%
\pgfpathlineto{\pgfqpoint{2.896866in}{2.463686in}}%
\pgfpathlineto{\pgfqpoint{2.894985in}{2.465937in}}%
\pgfpathlineto{\pgfqpoint{2.893731in}{2.467438in}}%
\pgfpathlineto{\pgfqpoint{2.891850in}{2.469689in}}%
\pgfpathlineto{\pgfqpoint{2.890596in}{2.471189in}}%
\pgfpathlineto{\pgfqpoint{2.888716in}{2.473440in}}%
\pgfpathlineto{\pgfqpoint{2.887462in}{2.474941in}}%
\pgfpathlineto{\pgfqpoint{2.885581in}{2.477192in}}%
\pgfpathlineto{\pgfqpoint{2.884327in}{2.478692in}}%
\pgfpathlineto{\pgfqpoint{2.882446in}{2.480943in}}%
\pgfpathlineto{\pgfqpoint{2.881192in}{2.482444in}}%
\pgfpathlineto{\pgfqpoint{2.879311in}{2.484695in}}%
\pgfpathlineto{\pgfqpoint{2.878057in}{2.486196in}}%
\pgfpathlineto{\pgfqpoint{2.876177in}{2.488446in}}%
\pgfpathlineto{\pgfqpoint{2.874923in}{2.489947in}}%
\pgfpathlineto{\pgfqpoint{2.873042in}{2.492198in}}%
\pgfpathlineto{\pgfqpoint{2.871788in}{2.493699in}}%
\pgfpathlineto{\pgfqpoint{2.869907in}{2.495950in}}%
\pgfpathlineto{\pgfqpoint{2.868653in}{2.497450in}}%
\pgfpathlineto{\pgfqpoint{2.866772in}{2.499701in}}%
\pgfpathlineto{\pgfqpoint{2.865518in}{2.501202in}}%
\pgfpathlineto{\pgfqpoint{2.863638in}{2.503453in}}%
\pgfpathlineto{\pgfqpoint{2.862384in}{2.504953in}}%
\pgfpathlineto{\pgfqpoint{2.860503in}{2.507204in}}%
\pgfpathlineto{\pgfqpoint{2.859249in}{2.508705in}}%
\pgfpathlineto{\pgfqpoint{2.859249in}{2.512456in}}%
\pgfpathlineto{\pgfqpoint{2.857368in}{2.514707in}}%
\pgfpathlineto{\pgfqpoint{2.856114in}{2.516208in}}%
\pgfpathlineto{\pgfqpoint{2.854233in}{2.518459in}}%
\pgfpathlineto{\pgfqpoint{2.852980in}{2.519959in}}%
\pgfpathlineto{\pgfqpoint{2.851099in}{2.522210in}}%
\pgfpathlineto{\pgfqpoint{2.849845in}{2.523711in}}%
\pgfpathlineto{\pgfqpoint{2.847964in}{2.525962in}}%
\pgfpathlineto{\pgfqpoint{2.846710in}{2.527462in}}%
\pgfpathlineto{\pgfqpoint{2.844829in}{2.529713in}}%
\pgfpathlineto{\pgfqpoint{2.843575in}{2.531214in}}%
\pgfpathlineto{\pgfqpoint{2.841694in}{2.533465in}}%
\pgfpathlineto{\pgfqpoint{2.840441in}{2.534965in}}%
\pgfpathlineto{\pgfqpoint{2.838560in}{2.537216in}}%
\pgfpathlineto{\pgfqpoint{2.837306in}{2.538717in}}%
\pgfpathlineto{\pgfqpoint{2.835425in}{2.540968in}}%
\pgfpathlineto{\pgfqpoint{2.834171in}{2.542469in}}%
\pgfpathlineto{\pgfqpoint{2.832290in}{2.544719in}}%
\pgfpathlineto{\pgfqpoint{2.831036in}{2.546220in}}%
\pgfpathlineto{\pgfqpoint{2.829155in}{2.548471in}}%
\pgfpathlineto{\pgfqpoint{2.827902in}{2.549972in}}%
\pgfpathlineto{\pgfqpoint{2.826021in}{2.552223in}}%
\pgfpathlineto{\pgfqpoint{2.824767in}{2.553723in}}%
\pgfpathlineto{\pgfqpoint{2.822886in}{2.555974in}}%
\pgfpathlineto{\pgfqpoint{2.821632in}{2.557475in}}%
\pgfpathlineto{\pgfqpoint{2.819751in}{2.559726in}}%
\pgfpathlineto{\pgfqpoint{2.818497in}{2.561226in}}%
\pgfpathlineto{\pgfqpoint{2.816617in}{2.563477in}}%
\pgfpathlineto{\pgfqpoint{2.815363in}{2.564978in}}%
\pgfpathlineto{\pgfqpoint{2.813482in}{2.567229in}}%
\pgfpathlineto{\pgfqpoint{2.812228in}{2.568729in}}%
\pgfpathlineto{\pgfqpoint{2.810347in}{2.570980in}}%
\pgfpathlineto{\pgfqpoint{2.809093in}{2.572481in}}%
\pgfpathlineto{\pgfqpoint{2.809093in}{2.576232in}}%
\pgfpathlineto{\pgfqpoint{2.807212in}{2.578483in}}%
\pgfpathlineto{\pgfqpoint{2.805958in}{2.579984in}}%
\pgfpathlineto{\pgfqpoint{2.804078in}{2.582235in}}%
\pgfpathlineto{\pgfqpoint{2.802824in}{2.583735in}}%
\pgfpathlineto{\pgfqpoint{2.800943in}{2.585986in}}%
\pgfpathlineto{\pgfqpoint{2.799689in}{2.587487in}}%
\pgfpathlineto{\pgfqpoint{2.797808in}{2.589738in}}%
\pgfpathlineto{\pgfqpoint{2.796554in}{2.591238in}}%
\pgfpathlineto{\pgfqpoint{2.794673in}{2.593489in}}%
\pgfpathlineto{\pgfqpoint{2.793419in}{2.594990in}}%
\pgfpathlineto{\pgfqpoint{2.791539in}{2.597241in}}%
\pgfpathlineto{\pgfqpoint{2.790285in}{2.598742in}}%
\pgfpathlineto{\pgfqpoint{2.788404in}{2.600992in}}%
\pgfpathlineto{\pgfqpoint{2.787150in}{2.602493in}}%
\pgfpathlineto{\pgfqpoint{2.785269in}{2.604744in}}%
\pgfpathlineto{\pgfqpoint{2.784015in}{2.606245in}}%
\pgfpathlineto{\pgfqpoint{2.782134in}{2.608496in}}%
\pgfpathlineto{\pgfqpoint{2.780880in}{2.609996in}}%
\pgfpathlineto{\pgfqpoint{2.779000in}{2.612247in}}%
\pgfpathlineto{\pgfqpoint{2.777746in}{2.613748in}}%
\pgfpathlineto{\pgfqpoint{2.775865in}{2.615999in}}%
\pgfpathlineto{\pgfqpoint{2.774611in}{2.617499in}}%
\pgfpathlineto{\pgfqpoint{2.772730in}{2.619750in}}%
\pgfpathlineto{\pgfqpoint{2.771476in}{2.621251in}}%
\pgfpathlineto{\pgfqpoint{2.769595in}{2.623502in}}%
\pgfpathlineto{\pgfqpoint{2.768341in}{2.625002in}}%
\pgfpathlineto{\pgfqpoint{2.766461in}{2.627253in}}%
\pgfpathlineto{\pgfqpoint{2.765207in}{2.628754in}}%
\pgfpathlineto{\pgfqpoint{2.763326in}{2.631005in}}%
\pgfpathlineto{\pgfqpoint{2.762072in}{2.632505in}}%
\pgfpathlineto{\pgfqpoint{2.762072in}{2.636257in}}%
\pgfpathlineto{\pgfqpoint{2.760191in}{2.638508in}}%
\pgfpathlineto{\pgfqpoint{2.758937in}{2.640008in}}%
\pgfpathlineto{\pgfqpoint{2.757056in}{2.642259in}}%
\pgfpathlineto{\pgfqpoint{2.755803in}{2.643760in}}%
\pgfpathlineto{\pgfqpoint{2.753922in}{2.646011in}}%
\pgfpathlineto{\pgfqpoint{2.752668in}{2.647512in}}%
\pgfpathlineto{\pgfqpoint{2.750787in}{2.649762in}}%
\pgfpathlineto{\pgfqpoint{2.749533in}{2.651263in}}%
\pgfpathlineto{\pgfqpoint{2.747652in}{2.653514in}}%
\pgfpathlineto{\pgfqpoint{2.746398in}{2.655015in}}%
\pgfpathlineto{\pgfqpoint{2.744517in}{2.657265in}}%
\pgfpathlineto{\pgfqpoint{2.743264in}{2.658766in}}%
\pgfpathlineto{\pgfqpoint{2.741383in}{2.661017in}}%
\pgfpathlineto{\pgfqpoint{2.740129in}{2.662518in}}%
\pgfpathlineto{\pgfqpoint{2.738248in}{2.664769in}}%
\pgfpathlineto{\pgfqpoint{2.736994in}{2.666269in}}%
\pgfpathlineto{\pgfqpoint{2.735113in}{2.668520in}}%
\pgfpathlineto{\pgfqpoint{2.733859in}{2.670021in}}%
\pgfpathlineto{\pgfqpoint{2.731978in}{2.672272in}}%
\pgfpathlineto{\pgfqpoint{2.730725in}{2.673772in}}%
\pgfpathlineto{\pgfqpoint{2.728844in}{2.676023in}}%
\pgfpathlineto{\pgfqpoint{2.727590in}{2.677524in}}%
\pgfpathlineto{\pgfqpoint{2.725709in}{2.679775in}}%
\pgfpathlineto{\pgfqpoint{2.724455in}{2.681275in}}%
\pgfpathlineto{\pgfqpoint{2.722574in}{2.683526in}}%
\pgfpathlineto{\pgfqpoint{2.721320in}{2.685027in}}%
\pgfpathlineto{\pgfqpoint{2.719439in}{2.687278in}}%
\pgfpathlineto{\pgfqpoint{2.718186in}{2.688778in}}%
\pgfpathlineto{\pgfqpoint{2.716305in}{2.691029in}}%
\pgfpathlineto{\pgfqpoint{2.715051in}{2.692530in}}%
\pgfpathlineto{\pgfqpoint{2.713170in}{2.694781in}}%
\pgfpathlineto{\pgfqpoint{2.711916in}{2.696281in}}%
\pgfpathlineto{\pgfqpoint{2.711916in}{2.700033in}}%
\pgfpathlineto{\pgfqpoint{2.710035in}{2.702284in}}%
\pgfpathlineto{\pgfqpoint{2.708781in}{2.703785in}}%
\pgfpathlineto{\pgfqpoint{2.706901in}{2.706035in}}%
\pgfpathlineto{\pgfqpoint{2.705647in}{2.707536in}}%
\pgfpathlineto{\pgfqpoint{2.703766in}{2.709787in}}%
\pgfpathlineto{\pgfqpoint{2.702512in}{2.711288in}}%
\pgfpathlineto{\pgfqpoint{2.700631in}{2.713539in}}%
\pgfpathlineto{\pgfqpoint{2.699377in}{2.715039in}}%
\pgfpathlineto{\pgfqpoint{2.697496in}{2.717290in}}%
\pgfpathlineto{\pgfqpoint{2.696242in}{2.718791in}}%
\pgfpathlineto{\pgfqpoint{2.694362in}{2.721042in}}%
\pgfpathlineto{\pgfqpoint{2.693108in}{2.722542in}}%
\pgfpathlineto{\pgfqpoint{2.691227in}{2.724793in}}%
\pgfpathlineto{\pgfqpoint{2.689973in}{2.726294in}}%
\pgfpathlineto{\pgfqpoint{2.688092in}{2.728545in}}%
\pgfpathlineto{\pgfqpoint{2.686838in}{2.730045in}}%
\pgfpathlineto{\pgfqpoint{2.684957in}{2.732296in}}%
\pgfpathlineto{\pgfqpoint{2.683703in}{2.733797in}}%
\pgfpathlineto{\pgfqpoint{2.681823in}{2.736048in}}%
\pgfpathlineto{\pgfqpoint{2.680569in}{2.737548in}}%
\pgfpathlineto{\pgfqpoint{2.678688in}{2.739799in}}%
\pgfpathlineto{\pgfqpoint{2.677434in}{2.741300in}}%
\pgfpathlineto{\pgfqpoint{2.675553in}{2.743551in}}%
\pgfpathlineto{\pgfqpoint{2.674299in}{2.745051in}}%
\pgfpathlineto{\pgfqpoint{2.672418in}{2.747302in}}%
\pgfpathlineto{\pgfqpoint{2.669284in}{2.747302in}}%
\pgfpathlineto{\pgfqpoint{2.666149in}{2.747302in}}%
\pgfpathlineto{\pgfqpoint{2.663014in}{2.747302in}}%
\pgfpathlineto{\pgfqpoint{2.659879in}{2.747302in}}%
\pgfpathlineto{\pgfqpoint{2.657999in}{2.745051in}}%
\pgfpathlineto{\pgfqpoint{2.656745in}{2.743551in}}%
\pgfpathlineto{\pgfqpoint{2.653610in}{2.743551in}}%
\pgfpathlineto{\pgfqpoint{2.650475in}{2.743551in}}%
\pgfpathlineto{\pgfqpoint{2.647340in}{2.743551in}}%
\pgfpathlineto{\pgfqpoint{2.645460in}{2.741300in}}%
\pgfpathlineto{\pgfqpoint{2.644206in}{2.739799in}}%
\pgfpathlineto{\pgfqpoint{2.641071in}{2.739799in}}%
\pgfpathlineto{\pgfqpoint{2.637936in}{2.739799in}}%
\pgfpathlineto{\pgfqpoint{2.634801in}{2.739799in}}%
\pgfpathlineto{\pgfqpoint{2.632921in}{2.737548in}}%
\pgfpathlineto{\pgfqpoint{2.631667in}{2.736048in}}%
\pgfpathlineto{\pgfqpoint{2.628532in}{2.736048in}}%
\pgfpathlineto{\pgfqpoint{2.625397in}{2.736048in}}%
\pgfpathlineto{\pgfqpoint{2.622262in}{2.736048in}}%
\pgfpathlineto{\pgfqpoint{2.619128in}{2.736048in}}%
\pgfpathlineto{\pgfqpoint{2.617247in}{2.733797in}}%
\pgfpathlineto{\pgfqpoint{2.615993in}{2.732296in}}%
\pgfpathlineto{\pgfqpoint{2.612858in}{2.732296in}}%
\pgfpathlineto{\pgfqpoint{2.609724in}{2.732296in}}%
\pgfpathlineto{\pgfqpoint{2.606589in}{2.732296in}}%
\pgfpathlineto{\pgfqpoint{2.604708in}{2.730045in}}%
\pgfpathlineto{\pgfqpoint{2.603454in}{2.728545in}}%
\pgfpathlineto{\pgfqpoint{2.600319in}{2.728545in}}%
\pgfpathlineto{\pgfqpoint{2.597185in}{2.728545in}}%
\pgfpathlineto{\pgfqpoint{2.594050in}{2.728545in}}%
\pgfpathlineto{\pgfqpoint{2.592169in}{2.726294in}}%
\pgfpathlineto{\pgfqpoint{2.590915in}{2.724793in}}%
\pgfpathlineto{\pgfqpoint{2.587780in}{2.724793in}}%
\pgfpathlineto{\pgfqpoint{2.584646in}{2.724793in}}%
\pgfpathlineto{\pgfqpoint{2.581511in}{2.724793in}}%
\pgfpathlineto{\pgfqpoint{2.579630in}{2.722542in}}%
\pgfpathlineto{\pgfqpoint{2.578376in}{2.721042in}}%
\pgfpathlineto{\pgfqpoint{2.575241in}{2.721042in}}%
\pgfpathlineto{\pgfqpoint{2.572107in}{2.721042in}}%
\pgfpathlineto{\pgfqpoint{2.568972in}{2.721042in}}%
\pgfpathlineto{\pgfqpoint{2.567091in}{2.718791in}}%
\pgfpathlineto{\pgfqpoint{2.565837in}{2.717290in}}%
\pgfpathlineto{\pgfqpoint{2.562702in}{2.717290in}}%
\pgfpathlineto{\pgfqpoint{2.559568in}{2.717290in}}%
\pgfpathlineto{\pgfqpoint{2.556433in}{2.717290in}}%
\pgfpathlineto{\pgfqpoint{2.553298in}{2.717290in}}%
\pgfpathlineto{\pgfqpoint{2.551417in}{2.715039in}}%
\pgfpathlineto{\pgfqpoint{2.550163in}{2.713539in}}%
\pgfpathlineto{\pgfqpoint{2.547029in}{2.713539in}}%
\pgfpathlineto{\pgfqpoint{2.543894in}{2.713539in}}%
\pgfpathlineto{\pgfqpoint{2.540759in}{2.713539in}}%
\pgfpathlineto{\pgfqpoint{2.538878in}{2.711288in}}%
\pgfpathlineto{\pgfqpoint{2.537624in}{2.709787in}}%
\pgfpathlineto{\pgfqpoint{2.534490in}{2.709787in}}%
\pgfpathlineto{\pgfqpoint{2.531355in}{2.709787in}}%
\pgfpathlineto{\pgfqpoint{2.528220in}{2.709787in}}%
\pgfpathlineto{\pgfqpoint{2.526339in}{2.707536in}}%
\pgfpathlineto{\pgfqpoint{2.525085in}{2.706035in}}%
\pgfpathlineto{\pgfqpoint{2.521951in}{2.706035in}}%
\pgfpathlineto{\pgfqpoint{2.518816in}{2.706035in}}%
\pgfpathlineto{\pgfqpoint{2.515681in}{2.706035in}}%
\pgfpathlineto{\pgfqpoint{2.513800in}{2.703785in}}%
\pgfpathlineto{\pgfqpoint{2.512547in}{2.702284in}}%
\pgfpathlineto{\pgfqpoint{2.509412in}{2.702284in}}%
\pgfpathlineto{\pgfqpoint{2.506277in}{2.702284in}}%
\pgfpathlineto{\pgfqpoint{2.503142in}{2.702284in}}%
\pgfpathlineto{\pgfqpoint{2.500008in}{2.702284in}}%
\pgfpathlineto{\pgfqpoint{2.498127in}{2.700033in}}%
\pgfpathlineto{\pgfqpoint{2.496873in}{2.698532in}}%
\pgfpathlineto{\pgfqpoint{2.493738in}{2.698532in}}%
\pgfpathlineto{\pgfqpoint{2.490603in}{2.698532in}}%
\pgfpathlineto{\pgfqpoint{2.487469in}{2.698532in}}%
\pgfpathlineto{\pgfqpoint{2.485588in}{2.696281in}}%
\pgfpathlineto{\pgfqpoint{2.484334in}{2.694781in}}%
\pgfpathlineto{\pgfqpoint{2.481199in}{2.694781in}}%
\pgfpathlineto{\pgfqpoint{2.478064in}{2.694781in}}%
\pgfpathlineto{\pgfqpoint{2.474930in}{2.694781in}}%
\pgfpathlineto{\pgfqpoint{2.473049in}{2.692530in}}%
\pgfpathlineto{\pgfqpoint{2.471795in}{2.691029in}}%
\pgfpathlineto{\pgfqpoint{2.468660in}{2.691029in}}%
\pgfpathlineto{\pgfqpoint{2.465525in}{2.691029in}}%
\pgfpathlineto{\pgfqpoint{2.462391in}{2.691029in}}%
\pgfpathlineto{\pgfqpoint{2.460510in}{2.688778in}}%
\pgfpathlineto{\pgfqpoint{2.459256in}{2.687278in}}%
\pgfpathlineto{\pgfqpoint{2.456121in}{2.687278in}}%
\pgfpathlineto{\pgfqpoint{2.452986in}{2.687278in}}%
\pgfpathlineto{\pgfqpoint{2.449852in}{2.687278in}}%
\pgfpathlineto{\pgfqpoint{2.447971in}{2.685027in}}%
\pgfpathlineto{\pgfqpoint{2.446717in}{2.683526in}}%
\pgfpathlineto{\pgfqpoint{2.443582in}{2.683526in}}%
\pgfpathlineto{\pgfqpoint{2.440447in}{2.683526in}}%
\pgfpathlineto{\pgfqpoint{2.437313in}{2.683526in}}%
\pgfpathlineto{\pgfqpoint{2.434178in}{2.683526in}}%
\pgfpathlineto{\pgfqpoint{2.432297in}{2.681275in}}%
\pgfpathlineto{\pgfqpoint{2.431043in}{2.679775in}}%
\pgfpathlineto{\pgfqpoint{2.427908in}{2.679775in}}%
\pgfpathlineto{\pgfqpoint{2.424774in}{2.679775in}}%
\pgfpathlineto{\pgfqpoint{2.421639in}{2.679775in}}%
\pgfpathlineto{\pgfqpoint{2.419758in}{2.677524in}}%
\pgfpathlineto{\pgfqpoint{2.418504in}{2.676023in}}%
\pgfpathlineto{\pgfqpoint{2.415369in}{2.676023in}}%
\pgfpathlineto{\pgfqpoint{2.412235in}{2.676023in}}%
\pgfpathlineto{\pgfqpoint{2.409100in}{2.676023in}}%
\pgfpathlineto{\pgfqpoint{2.407219in}{2.673772in}}%
\pgfpathlineto{\pgfqpoint{2.405965in}{2.672272in}}%
\pgfpathlineto{\pgfqpoint{2.402831in}{2.672272in}}%
\pgfpathlineto{\pgfqpoint{2.399696in}{2.672272in}}%
\pgfpathlineto{\pgfqpoint{2.396561in}{2.672272in}}%
\pgfpathlineto{\pgfqpoint{2.394680in}{2.670021in}}%
\pgfpathlineto{\pgfqpoint{2.393426in}{2.668520in}}%
\pgfpathlineto{\pgfqpoint{2.390292in}{2.668520in}}%
\pgfpathlineto{\pgfqpoint{2.387157in}{2.668520in}}%
\pgfpathlineto{\pgfqpoint{2.384022in}{2.668520in}}%
\pgfpathlineto{\pgfqpoint{2.380887in}{2.668520in}}%
\pgfpathlineto{\pgfqpoint{2.379006in}{2.666269in}}%
\pgfpathlineto{\pgfqpoint{2.377753in}{2.664769in}}%
\pgfpathlineto{\pgfqpoint{2.374618in}{2.664769in}}%
\pgfpathlineto{\pgfqpoint{2.371483in}{2.664769in}}%
\pgfpathlineto{\pgfqpoint{2.368348in}{2.664769in}}%
\pgfpathlineto{\pgfqpoint{2.366468in}{2.662518in}}%
\pgfpathlineto{\pgfqpoint{2.365214in}{2.661017in}}%
\pgfpathlineto{\pgfqpoint{2.362079in}{2.661017in}}%
\pgfpathlineto{\pgfqpoint{2.358944in}{2.661017in}}%
\pgfpathlineto{\pgfqpoint{2.355809in}{2.661017in}}%
\pgfpathlineto{\pgfqpoint{2.353929in}{2.658766in}}%
\pgfpathlineto{\pgfqpoint{2.352675in}{2.657265in}}%
\pgfpathlineto{\pgfqpoint{2.349540in}{2.657265in}}%
\pgfpathlineto{\pgfqpoint{2.346405in}{2.657265in}}%
\pgfpathlineto{\pgfqpoint{2.343270in}{2.657265in}}%
\pgfpathlineto{\pgfqpoint{2.341390in}{2.655015in}}%
\pgfpathlineto{\pgfqpoint{2.340136in}{2.653514in}}%
\pgfpathlineto{\pgfqpoint{2.337001in}{2.653514in}}%
\pgfpathlineto{\pgfqpoint{2.333866in}{2.653514in}}%
\pgfpathlineto{\pgfqpoint{2.330731in}{2.653514in}}%
\pgfpathlineto{\pgfqpoint{2.327597in}{2.653514in}}%
\pgfpathlineto{\pgfqpoint{2.325716in}{2.651263in}}%
\pgfpathlineto{\pgfqpoint{2.324462in}{2.649762in}}%
\pgfpathlineto{\pgfqpoint{2.321327in}{2.649762in}}%
\pgfpathlineto{\pgfqpoint{2.318192in}{2.649762in}}%
\pgfpathlineto{\pgfqpoint{2.315058in}{2.649762in}}%
\pgfpathlineto{\pgfqpoint{2.313177in}{2.647512in}}%
\pgfpathlineto{\pgfqpoint{2.311923in}{2.646011in}}%
\pgfpathlineto{\pgfqpoint{2.308788in}{2.646011in}}%
\pgfpathlineto{\pgfqpoint{2.305654in}{2.646011in}}%
\pgfpathlineto{\pgfqpoint{2.302519in}{2.646011in}}%
\pgfpathlineto{\pgfqpoint{2.300638in}{2.643760in}}%
\pgfpathlineto{\pgfqpoint{2.299384in}{2.642259in}}%
\pgfpathlineto{\pgfqpoint{2.296249in}{2.642259in}}%
\pgfpathlineto{\pgfqpoint{2.293115in}{2.642259in}}%
\pgfpathlineto{\pgfqpoint{2.289980in}{2.642259in}}%
\pgfpathlineto{\pgfqpoint{2.288099in}{2.640008in}}%
\pgfpathlineto{\pgfqpoint{2.286845in}{2.638508in}}%
\pgfpathlineto{\pgfqpoint{2.283710in}{2.638508in}}%
\pgfpathlineto{\pgfqpoint{2.280576in}{2.638508in}}%
\pgfpathlineto{\pgfqpoint{2.277441in}{2.638508in}}%
\pgfpathlineto{\pgfqpoint{2.275560in}{2.636257in}}%
\pgfpathlineto{\pgfqpoint{2.274306in}{2.634756in}}%
\pgfpathlineto{\pgfqpoint{2.271171in}{2.634756in}}%
\pgfpathlineto{\pgfqpoint{2.268037in}{2.634756in}}%
\pgfpathlineto{\pgfqpoint{2.264902in}{2.634756in}}%
\pgfpathlineto{\pgfqpoint{2.261767in}{2.634756in}}%
\pgfpathlineto{\pgfqpoint{2.259886in}{2.632505in}}%
\pgfpathlineto{\pgfqpoint{2.258632in}{2.631005in}}%
\pgfpathlineto{\pgfqpoint{2.255498in}{2.631005in}}%
\pgfpathlineto{\pgfqpoint{2.252363in}{2.631005in}}%
\pgfpathlineto{\pgfqpoint{2.249228in}{2.631005in}}%
\pgfpathlineto{\pgfqpoint{2.247347in}{2.628754in}}%
\pgfpathlineto{\pgfqpoint{2.246093in}{2.627253in}}%
\pgfpathlineto{\pgfqpoint{2.242959in}{2.627253in}}%
\pgfpathlineto{\pgfqpoint{2.239824in}{2.627253in}}%
\pgfpathlineto{\pgfqpoint{2.236689in}{2.627253in}}%
\pgfpathlineto{\pgfqpoint{2.234808in}{2.625002in}}%
\pgfpathlineto{\pgfqpoint{2.233554in}{2.623502in}}%
\pgfpathlineto{\pgfqpoint{2.230420in}{2.623502in}}%
\pgfpathlineto{\pgfqpoint{2.227285in}{2.623502in}}%
\pgfpathlineto{\pgfqpoint{2.224150in}{2.623502in}}%
\pgfpathlineto{\pgfqpoint{2.222269in}{2.621251in}}%
\pgfpathlineto{\pgfqpoint{2.221015in}{2.619750in}}%
\pgfpathlineto{\pgfqpoint{2.217881in}{2.619750in}}%
\pgfpathlineto{\pgfqpoint{2.214746in}{2.619750in}}%
\pgfpathlineto{\pgfqpoint{2.211611in}{2.619750in}}%
\pgfpathlineto{\pgfqpoint{2.208477in}{2.619750in}}%
\pgfpathlineto{\pgfqpoint{2.206596in}{2.617499in}}%
\pgfpathlineto{\pgfqpoint{2.205342in}{2.615999in}}%
\pgfpathlineto{\pgfqpoint{2.202207in}{2.615999in}}%
\pgfpathlineto{\pgfqpoint{2.199072in}{2.615999in}}%
\pgfpathlineto{\pgfqpoint{2.195938in}{2.615999in}}%
\pgfpathlineto{\pgfqpoint{2.194057in}{2.613748in}}%
\pgfpathlineto{\pgfqpoint{2.192803in}{2.612247in}}%
\pgfpathlineto{\pgfqpoint{2.189668in}{2.612247in}}%
\pgfpathlineto{\pgfqpoint{2.186533in}{2.612247in}}%
\pgfpathlineto{\pgfqpoint{2.183399in}{2.612247in}}%
\pgfpathlineto{\pgfqpoint{2.181518in}{2.609996in}}%
\pgfpathlineto{\pgfqpoint{2.180264in}{2.608496in}}%
\pgfpathlineto{\pgfqpoint{2.177129in}{2.608496in}}%
\pgfpathlineto{\pgfqpoint{2.173994in}{2.608496in}}%
\pgfpathlineto{\pgfqpoint{2.170860in}{2.608496in}}%
\pgfpathlineto{\pgfqpoint{2.168979in}{2.606245in}}%
\pgfpathlineto{\pgfqpoint{2.167725in}{2.604744in}}%
\pgfpathlineto{\pgfqpoint{2.164590in}{2.604744in}}%
\pgfpathlineto{\pgfqpoint{2.161455in}{2.604744in}}%
\pgfpathlineto{\pgfqpoint{2.158321in}{2.604744in}}%
\pgfpathlineto{\pgfqpoint{2.155186in}{2.604744in}}%
\pgfpathlineto{\pgfqpoint{2.153305in}{2.602493in}}%
\pgfpathlineto{\pgfqpoint{2.152051in}{2.600992in}}%
\pgfpathlineto{\pgfqpoint{2.148916in}{2.600992in}}%
\pgfpathlineto{\pgfqpoint{2.145782in}{2.600992in}}%
\pgfpathlineto{\pgfqpoint{2.142647in}{2.600992in}}%
\pgfpathlineto{\pgfqpoint{2.140766in}{2.598742in}}%
\pgfpathlineto{\pgfqpoint{2.139512in}{2.597241in}}%
\pgfpathlineto{\pgfqpoint{2.136377in}{2.597241in}}%
\pgfpathlineto{\pgfqpoint{2.133243in}{2.597241in}}%
\pgfpathlineto{\pgfqpoint{2.130108in}{2.597241in}}%
\pgfpathlineto{\pgfqpoint{2.128227in}{2.594990in}}%
\pgfpathlineto{\pgfqpoint{2.126973in}{2.593489in}}%
\pgfpathlineto{\pgfqpoint{2.123838in}{2.593489in}}%
\pgfpathlineto{\pgfqpoint{2.120704in}{2.593489in}}%
\pgfpathlineto{\pgfqpoint{2.117569in}{2.593489in}}%
\pgfpathlineto{\pgfqpoint{2.115688in}{2.591238in}}%
\pgfpathlineto{\pgfqpoint{2.114434in}{2.589738in}}%
\pgfpathlineto{\pgfqpoint{2.111299in}{2.589738in}}%
\pgfpathlineto{\pgfqpoint{2.108165in}{2.589738in}}%
\pgfpathlineto{\pgfqpoint{2.105030in}{2.589738in}}%
\pgfpathlineto{\pgfqpoint{2.103149in}{2.587487in}}%
\pgfpathlineto{\pgfqpoint{2.101895in}{2.585986in}}%
\pgfpathlineto{\pgfqpoint{2.098761in}{2.585986in}}%
\pgfpathlineto{\pgfqpoint{2.095626in}{2.585986in}}%
\pgfpathlineto{\pgfqpoint{2.092491in}{2.585986in}}%
\pgfpathlineto{\pgfqpoint{2.089356in}{2.585986in}}%
\pgfpathlineto{\pgfqpoint{2.087475in}{2.583735in}}%
\pgfpathlineto{\pgfqpoint{2.086222in}{2.582235in}}%
\pgfpathlineto{\pgfqpoint{2.083087in}{2.582235in}}%
\pgfpathlineto{\pgfqpoint{2.079952in}{2.582235in}}%
\pgfpathlineto{\pgfqpoint{2.076817in}{2.582235in}}%
\pgfpathlineto{\pgfqpoint{2.074936in}{2.579984in}}%
\pgfpathlineto{\pgfqpoint{2.073683in}{2.578483in}}%
\pgfpathlineto{\pgfqpoint{2.070548in}{2.578483in}}%
\pgfpathlineto{\pgfqpoint{2.067413in}{2.578483in}}%
\pgfpathlineto{\pgfqpoint{2.064278in}{2.578483in}}%
\pgfpathlineto{\pgfqpoint{2.062398in}{2.576232in}}%
\pgfpathlineto{\pgfqpoint{2.061144in}{2.574732in}}%
\pgfpathlineto{\pgfqpoint{2.058009in}{2.574732in}}%
\pgfpathlineto{\pgfqpoint{2.054874in}{2.574732in}}%
\pgfpathlineto{\pgfqpoint{2.051739in}{2.574732in}}%
\pgfpathlineto{\pgfqpoint{2.049859in}{2.572481in}}%
\pgfpathlineto{\pgfqpoint{2.048605in}{2.570980in}}%
\pgfpathlineto{\pgfqpoint{2.045470in}{2.570980in}}%
\pgfpathlineto{\pgfqpoint{2.042335in}{2.570980in}}%
\pgfpathlineto{\pgfqpoint{2.039200in}{2.570980in}}%
\pgfpathlineto{\pgfqpoint{2.036066in}{2.570980in}}%
\pgfpathlineto{\pgfqpoint{2.034185in}{2.568729in}}%
\pgfpathlineto{\pgfqpoint{2.032931in}{2.567229in}}%
\pgfpathlineto{\pgfqpoint{2.029796in}{2.567229in}}%
\pgfpathlineto{\pgfqpoint{2.026661in}{2.567229in}}%
\pgfpathlineto{\pgfqpoint{2.023527in}{2.567229in}}%
\pgfpathlineto{\pgfqpoint{2.021646in}{2.564978in}}%
\pgfpathlineto{\pgfqpoint{2.020392in}{2.563477in}}%
\pgfpathlineto{\pgfqpoint{2.017257in}{2.563477in}}%
\pgfpathlineto{\pgfqpoint{2.014122in}{2.563477in}}%
\pgfpathlineto{\pgfqpoint{2.010988in}{2.563477in}}%
\pgfpathlineto{\pgfqpoint{2.009107in}{2.561226in}}%
\pgfpathlineto{\pgfqpoint{2.007853in}{2.559726in}}%
\pgfpathlineto{\pgfqpoint{2.004718in}{2.559726in}}%
\pgfpathlineto{\pgfqpoint{2.001584in}{2.559726in}}%
\pgfpathlineto{\pgfqpoint{1.998449in}{2.559726in}}%
\pgfpathlineto{\pgfqpoint{1.996568in}{2.557475in}}%
\pgfpathlineto{\pgfqpoint{1.995314in}{2.555974in}}%
\pgfpathlineto{\pgfqpoint{1.992179in}{2.555974in}}%
\pgfpathlineto{\pgfqpoint{1.989045in}{2.555974in}}%
\pgfpathlineto{\pgfqpoint{1.985910in}{2.555974in}}%
\pgfpathlineto{\pgfqpoint{1.982775in}{2.555974in}}%
\pgfpathlineto{\pgfqpoint{1.980894in}{2.553723in}}%
\pgfpathlineto{\pgfqpoint{1.979640in}{2.552223in}}%
\pgfpathlineto{\pgfqpoint{1.976506in}{2.552223in}}%
\pgfpathlineto{\pgfqpoint{1.973371in}{2.552223in}}%
\pgfpathlineto{\pgfqpoint{1.970236in}{2.552223in}}%
\pgfpathlineto{\pgfqpoint{1.968355in}{2.549972in}}%
\pgfpathlineto{\pgfqpoint{1.967101in}{2.548471in}}%
\pgfpathlineto{\pgfqpoint{1.963967in}{2.548471in}}%
\pgfpathlineto{\pgfqpoint{1.960832in}{2.548471in}}%
\pgfpathlineto{\pgfqpoint{1.957697in}{2.548471in}}%
\pgfpathlineto{\pgfqpoint{1.955816in}{2.546220in}}%
\pgfpathlineto{\pgfqpoint{1.954562in}{2.544719in}}%
\pgfpathlineto{\pgfqpoint{1.951428in}{2.544719in}}%
\pgfpathlineto{\pgfqpoint{1.948293in}{2.544719in}}%
\pgfpathlineto{\pgfqpoint{1.945158in}{2.544719in}}%
\pgfpathlineto{\pgfqpoint{1.943277in}{2.542469in}}%
\pgfpathlineto{\pgfqpoint{1.942023in}{2.540968in}}%
\pgfpathlineto{\pgfqpoint{1.938889in}{2.540968in}}%
\pgfpathlineto{\pgfqpoint{1.935754in}{2.540968in}}%
\pgfpathlineto{\pgfqpoint{1.932619in}{2.540968in}}%
\pgfpathlineto{\pgfqpoint{1.930738in}{2.538717in}}%
\pgfpathlineto{\pgfqpoint{1.929484in}{2.537216in}}%
\pgfpathlineto{\pgfqpoint{1.926350in}{2.537216in}}%
\pgfpathlineto{\pgfqpoint{1.923215in}{2.537216in}}%
\pgfpathlineto{\pgfqpoint{1.920080in}{2.537216in}}%
\pgfpathlineto{\pgfqpoint{1.916945in}{2.537216in}}%
\pgfpathlineto{\pgfqpoint{1.915065in}{2.534965in}}%
\pgfpathlineto{\pgfqpoint{1.913811in}{2.533465in}}%
\pgfpathlineto{\pgfqpoint{1.910676in}{2.533465in}}%
\pgfpathlineto{\pgfqpoint{1.907541in}{2.533465in}}%
\pgfpathlineto{\pgfqpoint{1.904407in}{2.533465in}}%
\pgfpathlineto{\pgfqpoint{1.902526in}{2.531214in}}%
\pgfpathlineto{\pgfqpoint{1.901272in}{2.529713in}}%
\pgfpathlineto{\pgfqpoint{1.898137in}{2.529713in}}%
\pgfpathlineto{\pgfqpoint{1.895002in}{2.529713in}}%
\pgfpathlineto{\pgfqpoint{1.891868in}{2.529713in}}%
\pgfpathlineto{\pgfqpoint{1.889987in}{2.527462in}}%
\pgfpathlineto{\pgfqpoint{1.888733in}{2.525962in}}%
\pgfpathlineto{\pgfqpoint{1.885598in}{2.525962in}}%
\pgfpathlineto{\pgfqpoint{1.882463in}{2.525962in}}%
\pgfpathlineto{\pgfqpoint{1.879329in}{2.525962in}}%
\pgfpathlineto{\pgfqpoint{1.877448in}{2.523711in}}%
\pgfpathlineto{\pgfqpoint{1.876194in}{2.522210in}}%
\pgfpathlineto{\pgfqpoint{1.873059in}{2.522210in}}%
\pgfpathlineto{\pgfqpoint{1.869924in}{2.522210in}}%
\pgfpathlineto{\pgfqpoint{1.866790in}{2.522210in}}%
\pgfpathlineto{\pgfqpoint{1.863655in}{2.522210in}}%
\pgfpathlineto{\pgfqpoint{1.861774in}{2.519959in}}%
\pgfpathlineto{\pgfqpoint{1.860520in}{2.518459in}}%
\pgfpathlineto{\pgfqpoint{1.857385in}{2.518459in}}%
\pgfpathlineto{\pgfqpoint{1.854251in}{2.518459in}}%
\pgfpathlineto{\pgfqpoint{1.851116in}{2.518459in}}%
\pgfpathlineto{\pgfqpoint{1.849235in}{2.516208in}}%
\pgfpathlineto{\pgfqpoint{1.847981in}{2.514707in}}%
\pgfpathlineto{\pgfqpoint{1.844846in}{2.514707in}}%
\pgfpathlineto{\pgfqpoint{1.841712in}{2.514707in}}%
\pgfpathlineto{\pgfqpoint{1.838577in}{2.514707in}}%
\pgfpathlineto{\pgfqpoint{1.836696in}{2.512456in}}%
\pgfpathlineto{\pgfqpoint{1.835442in}{2.510956in}}%
\pgfpathlineto{\pgfqpoint{1.832307in}{2.510956in}}%
\pgfpathlineto{\pgfqpoint{1.829173in}{2.510956in}}%
\pgfpathlineto{\pgfqpoint{1.826038in}{2.510956in}}%
\pgfpathlineto{\pgfqpoint{1.824157in}{2.508705in}}%
\pgfpathlineto{\pgfqpoint{1.822903in}{2.507204in}}%
\pgfpathlineto{\pgfqpoint{1.819768in}{2.507204in}}%
\pgfpathlineto{\pgfqpoint{1.816634in}{2.507204in}}%
\pgfpathlineto{\pgfqpoint{1.813499in}{2.507204in}}%
\pgfpathlineto{\pgfqpoint{1.811618in}{2.504953in}}%
\pgfpathlineto{\pgfqpoint{1.810364in}{2.503453in}}%
\pgfpathlineto{\pgfqpoint{1.807229in}{2.503453in}}%
\pgfpathlineto{\pgfqpoint{1.804095in}{2.503453in}}%
\pgfpathlineto{\pgfqpoint{1.800960in}{2.503453in}}%
\pgfpathlineto{\pgfqpoint{1.797825in}{2.503453in}}%
\pgfpathlineto{\pgfqpoint{1.795944in}{2.501202in}}%
\pgfpathlineto{\pgfqpoint{1.794691in}{2.499701in}}%
\pgfpathlineto{\pgfqpoint{1.791556in}{2.499701in}}%
\pgfpathlineto{\pgfqpoint{1.788421in}{2.499701in}}%
\pgfpathlineto{\pgfqpoint{1.785286in}{2.499701in}}%
\pgfpathlineto{\pgfqpoint{1.783405in}{2.497450in}}%
\pgfpathlineto{\pgfqpoint{1.782152in}{2.495950in}}%
\pgfpathlineto{\pgfqpoint{1.779017in}{2.495950in}}%
\pgfpathlineto{\pgfqpoint{1.775882in}{2.495950in}}%
\pgfpathlineto{\pgfqpoint{1.772747in}{2.495950in}}%
\pgfpathlineto{\pgfqpoint{1.770866in}{2.493699in}}%
\pgfpathlineto{\pgfqpoint{1.769613in}{2.492198in}}%
\pgfpathlineto{\pgfqpoint{1.766478in}{2.492198in}}%
\pgfpathlineto{\pgfqpoint{1.763343in}{2.492198in}}%
\pgfpathlineto{\pgfqpoint{1.760208in}{2.492198in}}%
\pgfpathlineto{\pgfqpoint{1.758328in}{2.489947in}}%
\pgfpathlineto{\pgfqpoint{1.757074in}{2.488446in}}%
\pgfpathlineto{\pgfqpoint{1.753939in}{2.488446in}}%
\pgfpathlineto{\pgfqpoint{1.750804in}{2.488446in}}%
\pgfpathlineto{\pgfqpoint{1.747669in}{2.488446in}}%
\pgfpathlineto{\pgfqpoint{1.744535in}{2.488446in}}%
\pgfpathlineto{\pgfqpoint{1.742654in}{2.486196in}}%
\pgfpathlineto{\pgfqpoint{1.741400in}{2.484695in}}%
\pgfpathlineto{\pgfqpoint{1.738265in}{2.484695in}}%
\pgfpathlineto{\pgfqpoint{1.735130in}{2.484695in}}%
\pgfpathlineto{\pgfqpoint{1.731996in}{2.484695in}}%
\pgfpathlineto{\pgfqpoint{1.730115in}{2.482444in}}%
\pgfpathlineto{\pgfqpoint{1.728861in}{2.480943in}}%
\pgfpathlineto{\pgfqpoint{1.725726in}{2.480943in}}%
\pgfpathlineto{\pgfqpoint{1.722591in}{2.480943in}}%
\pgfpathlineto{\pgfqpoint{1.719457in}{2.480943in}}%
\pgfpathlineto{\pgfqpoint{1.717576in}{2.478692in}}%
\pgfpathlineto{\pgfqpoint{1.716322in}{2.477192in}}%
\pgfpathlineto{\pgfqpoint{1.713187in}{2.477192in}}%
\pgfpathlineto{\pgfqpoint{1.710052in}{2.477192in}}%
\pgfpathlineto{\pgfqpoint{1.706918in}{2.477192in}}%
\pgfpathlineto{\pgfqpoint{1.705037in}{2.474941in}}%
\pgfpathlineto{\pgfqpoint{1.703783in}{2.473440in}}%
\pgfpathlineto{\pgfqpoint{1.700648in}{2.473440in}}%
\pgfpathlineto{\pgfqpoint{1.697514in}{2.473440in}}%
\pgfpathlineto{\pgfqpoint{1.694379in}{2.473440in}}%
\pgfpathlineto{\pgfqpoint{1.691244in}{2.473440in}}%
\pgfpathlineto{\pgfqpoint{1.689363in}{2.471189in}}%
\pgfpathlineto{\pgfqpoint{1.688109in}{2.469689in}}%
\pgfpathlineto{\pgfqpoint{1.684975in}{2.469689in}}%
\pgfpathlineto{\pgfqpoint{1.681840in}{2.469689in}}%
\pgfpathlineto{\pgfqpoint{1.678705in}{2.469689in}}%
\pgfpathlineto{\pgfqpoint{1.676824in}{2.467438in}}%
\pgfpathlineto{\pgfqpoint{1.675570in}{2.465937in}}%
\pgfpathlineto{\pgfqpoint{1.672436in}{2.465937in}}%
\pgfpathlineto{\pgfqpoint{1.669301in}{2.465937in}}%
\pgfpathlineto{\pgfqpoint{1.666166in}{2.465937in}}%
\pgfpathlineto{\pgfqpoint{1.664285in}{2.463686in}}%
\pgfpathlineto{\pgfqpoint{1.663031in}{2.462186in}}%
\pgfpathlineto{\pgfqpoint{1.659897in}{2.462186in}}%
\pgfpathlineto{\pgfqpoint{1.656762in}{2.462186in}}%
\pgfpathlineto{\pgfqpoint{1.653627in}{2.462186in}}%
\pgfpathlineto{\pgfqpoint{1.651746in}{2.459935in}}%
\pgfpathlineto{\pgfqpoint{1.650492in}{2.458434in}}%
\pgfpathlineto{\pgfqpoint{1.647358in}{2.458434in}}%
\pgfpathlineto{\pgfqpoint{1.644223in}{2.458434in}}%
\pgfpathlineto{\pgfqpoint{1.641088in}{2.458434in}}%
\pgfpathlineto{\pgfqpoint{1.639207in}{2.456183in}}%
\pgfpathlineto{\pgfqpoint{1.637953in}{2.454683in}}%
\pgfpathlineto{\pgfqpoint{1.634819in}{2.454683in}}%
\pgfpathlineto{\pgfqpoint{1.631684in}{2.454683in}}%
\pgfpathlineto{\pgfqpoint{1.628549in}{2.454683in}}%
\pgfpathlineto{\pgfqpoint{1.625414in}{2.454683in}}%
\pgfpathlineto{\pgfqpoint{1.623534in}{2.452432in}}%
\pgfpathlineto{\pgfqpoint{1.622280in}{2.450931in}}%
\pgfpathlineto{\pgfqpoint{1.619145in}{2.450931in}}%
\pgfpathlineto{\pgfqpoint{1.616010in}{2.450931in}}%
\pgfpathlineto{\pgfqpoint{1.612875in}{2.450931in}}%
\pgfpathlineto{\pgfqpoint{1.610995in}{2.448680in}}%
\pgfpathlineto{\pgfqpoint{1.609741in}{2.447180in}}%
\pgfpathlineto{\pgfqpoint{1.606606in}{2.447180in}}%
\pgfpathlineto{\pgfqpoint{1.603471in}{2.447180in}}%
\pgfpathlineto{\pgfqpoint{1.600337in}{2.447180in}}%
\pgfpathlineto{\pgfqpoint{1.598456in}{2.444929in}}%
\pgfpathlineto{\pgfqpoint{1.597202in}{2.443428in}}%
\pgfpathlineto{\pgfqpoint{1.594067in}{2.443428in}}%
\pgfpathlineto{\pgfqpoint{1.590932in}{2.443428in}}%
\pgfpathlineto{\pgfqpoint{1.587798in}{2.443428in}}%
\pgfpathlineto{\pgfqpoint{1.585917in}{2.441177in}}%
\pgfpathlineto{\pgfqpoint{1.584663in}{2.439676in}}%
\pgfpathlineto{\pgfqpoint{1.581528in}{2.439676in}}%
\pgfpathlineto{\pgfqpoint{1.578393in}{2.439676in}}%
\pgfpathlineto{\pgfqpoint{1.575259in}{2.439676in}}%
\pgfpathlineto{\pgfqpoint{1.572124in}{2.439676in}}%
\pgfpathlineto{\pgfqpoint{1.570243in}{2.437426in}}%
\pgfpathlineto{\pgfqpoint{1.568989in}{2.435925in}}%
\pgfpathlineto{\pgfqpoint{1.565854in}{2.435925in}}%
\pgfpathlineto{\pgfqpoint{1.562720in}{2.435925in}}%
\pgfpathlineto{\pgfqpoint{1.559585in}{2.435925in}}%
\pgfpathlineto{\pgfqpoint{1.557704in}{2.433674in}}%
\pgfpathlineto{\pgfqpoint{1.556450in}{2.432173in}}%
\pgfpathlineto{\pgfqpoint{1.553315in}{2.432173in}}%
\pgfpathlineto{\pgfqpoint{1.550181in}{2.432173in}}%
\pgfpathlineto{\pgfqpoint{1.547046in}{2.432173in}}%
\pgfpathlineto{\pgfqpoint{1.545165in}{2.429923in}}%
\pgfpathlineto{\pgfqpoint{1.543911in}{2.428422in}}%
\pgfpathlineto{\pgfqpoint{1.540776in}{2.428422in}}%
\pgfpathlineto{\pgfqpoint{1.537642in}{2.428422in}}%
\pgfpathlineto{\pgfqpoint{1.534507in}{2.428422in}}%
\pgfpathlineto{\pgfqpoint{1.532626in}{2.426171in}}%
\pgfpathlineto{\pgfqpoint{1.531372in}{2.424670in}}%
\pgfpathlineto{\pgfqpoint{1.528237in}{2.424670in}}%
\pgfpathlineto{\pgfqpoint{1.525103in}{2.424670in}}%
\pgfpathlineto{\pgfqpoint{1.521968in}{2.424670in}}%
\pgfpathlineto{\pgfqpoint{1.518833in}{2.424670in}}%
\pgfpathlineto{\pgfqpoint{1.516952in}{2.422419in}}%
\pgfpathlineto{\pgfqpoint{1.515698in}{2.420919in}}%
\pgfpathlineto{\pgfqpoint{1.512564in}{2.420919in}}%
\pgfpathlineto{\pgfqpoint{1.509429in}{2.420919in}}%
\pgfpathlineto{\pgfqpoint{1.506294in}{2.420919in}}%
\pgfpathlineto{\pgfqpoint{1.504413in}{2.418668in}}%
\pgfpathlineto{\pgfqpoint{1.503159in}{2.417167in}}%
\pgfpathlineto{\pgfqpoint{1.500025in}{2.417167in}}%
\pgfpathlineto{\pgfqpoint{1.496890in}{2.417167in}}%
\pgfpathlineto{\pgfqpoint{1.493755in}{2.417167in}}%
\pgfpathlineto{\pgfqpoint{1.491874in}{2.414916in}}%
\pgfpathlineto{\pgfqpoint{1.490621in}{2.413416in}}%
\pgfpathlineto{\pgfqpoint{1.487486in}{2.413416in}}%
\pgfpathlineto{\pgfqpoint{1.484351in}{2.413416in}}%
\pgfpathlineto{\pgfqpoint{1.481216in}{2.413416in}}%
\pgfpathlineto{\pgfqpoint{1.479335in}{2.411165in}}%
\pgfpathlineto{\pgfqpoint{1.478082in}{2.409664in}}%
\pgfpathlineto{\pgfqpoint{1.474947in}{2.409664in}}%
\pgfpathlineto{\pgfqpoint{1.471812in}{2.409664in}}%
\pgfpathlineto{\pgfqpoint{1.468677in}{2.409664in}}%
\pgfpathlineto{\pgfqpoint{1.466796in}{2.407413in}}%
\pgfpathlineto{\pgfqpoint{1.465543in}{2.405913in}}%
\pgfpathlineto{\pgfqpoint{1.462408in}{2.405913in}}%
\pgfpathlineto{\pgfqpoint{1.460527in}{2.403662in}}%
\pgfpathlineto{\pgfqpoint{1.460527in}{2.399910in}}%
\pgfpathlineto{\pgfqpoint{1.459273in}{2.398410in}}%
\pgfpathlineto{\pgfqpoint{1.457392in}{2.396159in}}%
\pgfpathlineto{\pgfqpoint{1.457392in}{2.392407in}}%
\pgfpathlineto{\pgfqpoint{1.457392in}{2.388656in}}%
\pgfpathlineto{\pgfqpoint{1.456138in}{2.387155in}}%
\pgfpathlineto{\pgfqpoint{1.454258in}{2.384904in}}%
\pgfpathlineto{\pgfqpoint{1.454258in}{2.381153in}}%
\pgfpathlineto{\pgfqpoint{1.454258in}{2.377401in}}%
\pgfpathlineto{\pgfqpoint{1.454258in}{2.373649in}}%
\pgfpathlineto{\pgfqpoint{1.453004in}{2.372149in}}%
\pgfpathlineto{\pgfqpoint{1.451123in}{2.369898in}}%
\pgfpathlineto{\pgfqpoint{1.451123in}{2.366146in}}%
\pgfpathlineto{\pgfqpoint{1.451123in}{2.362395in}}%
\pgfpathlineto{\pgfqpoint{1.449869in}{2.360894in}}%
\pgfpathlineto{\pgfqpoint{1.447988in}{2.358643in}}%
\pgfpathlineto{\pgfqpoint{1.447988in}{2.354892in}}%
\pgfpathlineto{\pgfqpoint{1.447988in}{2.351140in}}%
\pgfpathlineto{\pgfqpoint{1.447988in}{2.347389in}}%
\pgfpathlineto{\pgfqpoint{1.446734in}{2.345888in}}%
\pgfpathlineto{\pgfqpoint{1.444853in}{2.343637in}}%
\pgfpathlineto{\pgfqpoint{1.444853in}{2.339886in}}%
\pgfpathlineto{\pgfqpoint{1.444853in}{2.336134in}}%
\pgfpathlineto{\pgfqpoint{1.443599in}{2.334634in}}%
\pgfpathlineto{\pgfqpoint{1.441719in}{2.332383in}}%
\pgfpathlineto{\pgfqpoint{1.441719in}{2.328631in}}%
\pgfpathlineto{\pgfqpoint{1.441719in}{2.324880in}}%
\pgfpathlineto{\pgfqpoint{1.441719in}{2.321128in}}%
\pgfpathlineto{\pgfqpoint{1.440465in}{2.319627in}}%
\pgfpathlineto{\pgfqpoint{1.438584in}{2.317376in}}%
\pgfpathlineto{\pgfqpoint{1.438584in}{2.313625in}}%
\pgfpathlineto{\pgfqpoint{1.438584in}{2.309873in}}%
\pgfpathlineto{\pgfqpoint{1.437330in}{2.308373in}}%
\pgfpathlineto{\pgfqpoint{1.435449in}{2.306122in}}%
\pgfpathlineto{\pgfqpoint{1.435449in}{2.302370in}}%
\pgfpathlineto{\pgfqpoint{1.435449in}{2.298619in}}%
\pgfpathlineto{\pgfqpoint{1.435449in}{2.294867in}}%
\pgfpathlineto{\pgfqpoint{1.434195in}{2.293367in}}%
\pgfpathlineto{\pgfqpoint{1.432314in}{2.291116in}}%
\pgfpathlineto{\pgfqpoint{1.432314in}{2.287364in}}%
\pgfpathlineto{\pgfqpoint{1.432314in}{2.283613in}}%
\pgfpathlineto{\pgfqpoint{1.431060in}{2.282112in}}%
\pgfpathlineto{\pgfqpoint{1.429180in}{2.279861in}}%
\pgfpathlineto{\pgfqpoint{1.429180in}{2.276110in}}%
\pgfpathlineto{\pgfqpoint{1.429180in}{2.272358in}}%
\pgfpathlineto{\pgfqpoint{1.429180in}{2.268607in}}%
\pgfpathlineto{\pgfqpoint{1.427926in}{2.267106in}}%
\pgfpathlineto{\pgfqpoint{1.426045in}{2.264855in}}%
\pgfpathlineto{\pgfqpoint{1.426045in}{2.261103in}}%
\pgfpathlineto{\pgfqpoint{1.426045in}{2.257352in}}%
\pgfpathlineto{\pgfqpoint{1.424791in}{2.255851in}}%
\pgfpathlineto{\pgfqpoint{1.422910in}{2.253600in}}%
\pgfpathlineto{\pgfqpoint{1.422910in}{2.249849in}}%
\pgfpathlineto{\pgfqpoint{1.422910in}{2.246097in}}%
\pgfpathlineto{\pgfqpoint{1.421656in}{2.244597in}}%
\pgfpathlineto{\pgfqpoint{1.419775in}{2.242346in}}%
\pgfpathlineto{\pgfqpoint{1.419775in}{2.238594in}}%
\pgfpathlineto{\pgfqpoint{1.419775in}{2.234843in}}%
\pgfpathlineto{\pgfqpoint{1.419775in}{2.231091in}}%
\pgfpathlineto{\pgfqpoint{1.418521in}{2.229591in}}%
\pgfpathlineto{\pgfqpoint{1.416641in}{2.227340in}}%
\pgfpathlineto{\pgfqpoint{1.416641in}{2.223588in}}%
\pgfpathlineto{\pgfqpoint{1.416641in}{2.219837in}}%
\pgfpathlineto{\pgfqpoint{1.415387in}{2.218336in}}%
\pgfpathlineto{\pgfqpoint{1.413506in}{2.216085in}}%
\pgfpathlineto{\pgfqpoint{1.413506in}{2.212334in}}%
\pgfpathlineto{\pgfqpoint{1.413506in}{2.208582in}}%
\pgfpathlineto{\pgfqpoint{1.413506in}{2.204830in}}%
\pgfpathlineto{\pgfqpoint{1.412252in}{2.203330in}}%
\pgfpathlineto{\pgfqpoint{1.410371in}{2.201079in}}%
\pgfpathlineto{\pgfqpoint{1.410371in}{2.197327in}}%
\pgfpathlineto{\pgfqpoint{1.410371in}{2.193576in}}%
\pgfpathlineto{\pgfqpoint{1.409117in}{2.192075in}}%
\pgfpathlineto{\pgfqpoint{1.407236in}{2.189824in}}%
\pgfpathlineto{\pgfqpoint{1.407236in}{2.186073in}}%
\pgfpathlineto{\pgfqpoint{1.407236in}{2.182321in}}%
\pgfpathlineto{\pgfqpoint{1.407236in}{2.178570in}}%
\pgfpathlineto{\pgfqpoint{1.405982in}{2.177069in}}%
\pgfpathlineto{\pgfqpoint{1.404102in}{2.174818in}}%
\pgfpathlineto{\pgfqpoint{1.404102in}{2.171067in}}%
\pgfpathlineto{\pgfqpoint{1.404102in}{2.167315in}}%
\pgfpathlineto{\pgfqpoint{1.402848in}{2.165814in}}%
\pgfpathlineto{\pgfqpoint{1.400967in}{2.163564in}}%
\pgfpathlineto{\pgfqpoint{1.400967in}{2.159812in}}%
\pgfpathlineto{\pgfqpoint{1.400967in}{2.156060in}}%
\pgfpathlineto{\pgfqpoint{1.400967in}{2.152309in}}%
\pgfpathlineto{\pgfqpoint{1.399713in}{2.150808in}}%
\pgfpathlineto{\pgfqpoint{1.397832in}{2.148557in}}%
\pgfpathlineto{\pgfqpoint{1.397832in}{2.144806in}}%
\pgfpathlineto{\pgfqpoint{1.397832in}{2.141054in}}%
\pgfpathlineto{\pgfqpoint{1.396578in}{2.139554in}}%
\pgfpathlineto{\pgfqpoint{1.394697in}{2.137303in}}%
\pgfpathlineto{\pgfqpoint{1.394697in}{2.133551in}}%
\pgfpathlineto{\pgfqpoint{1.394697in}{2.129800in}}%
\pgfpathlineto{\pgfqpoint{1.394697in}{2.126048in}}%
\pgfpathlineto{\pgfqpoint{1.393444in}{2.124548in}}%
\pgfpathlineto{\pgfqpoint{1.391563in}{2.122297in}}%
\pgfpathlineto{\pgfqpoint{1.391563in}{2.118545in}}%
\pgfpathlineto{\pgfqpoint{1.391563in}{2.114794in}}%
\pgfpathlineto{\pgfqpoint{1.390309in}{2.113293in}}%
\pgfpathlineto{\pgfqpoint{1.388428in}{2.111042in}}%
\pgfpathlineto{\pgfqpoint{1.388428in}{2.107291in}}%
\pgfpathlineto{\pgfqpoint{1.388428in}{2.103539in}}%
\pgfpathlineto{\pgfqpoint{1.388428in}{2.099787in}}%
\pgfpathlineto{\pgfqpoint{1.387174in}{2.098287in}}%
\pgfpathlineto{\pgfqpoint{1.385293in}{2.096036in}}%
\pgfpathlineto{\pgfqpoint{1.385293in}{2.092284in}}%
\pgfpathlineto{\pgfqpoint{1.385293in}{2.088533in}}%
\pgfpathlineto{\pgfqpoint{1.384039in}{2.087032in}}%
\pgfpathlineto{\pgfqpoint{1.382158in}{2.084781in}}%
\pgfpathlineto{\pgfqpoint{1.382158in}{2.081030in}}%
\pgfpathlineto{\pgfqpoint{1.382158in}{2.077278in}}%
\pgfpathlineto{\pgfqpoint{1.382158in}{2.073527in}}%
\pgfpathlineto{\pgfqpoint{1.380905in}{2.072026in}}%
\pgfpathlineto{\pgfqpoint{1.379024in}{2.069775in}}%
\pgfpathlineto{\pgfqpoint{1.379024in}{2.066024in}}%
\pgfpathlineto{\pgfqpoint{1.379024in}{2.062272in}}%
\pgfpathlineto{\pgfqpoint{1.377770in}{2.060772in}}%
\pgfpathlineto{\pgfqpoint{1.375889in}{2.058521in}}%
\pgfpathlineto{\pgfqpoint{1.375889in}{2.054769in}}%
\pgfpathlineto{\pgfqpoint{1.375889in}{2.051018in}}%
\pgfpathlineto{\pgfqpoint{1.375889in}{2.047266in}}%
\pgfpathlineto{\pgfqpoint{1.374635in}{2.045765in}}%
\pgfpathlineto{\pgfqpoint{1.372754in}{2.043514in}}%
\pgfpathlineto{\pgfqpoint{1.372754in}{2.039763in}}%
\pgfpathlineto{\pgfqpoint{1.372754in}{2.036011in}}%
\pgfpathlineto{\pgfqpoint{1.371500in}{2.034511in}}%
\pgfpathlineto{\pgfqpoint{1.369619in}{2.032260in}}%
\pgfpathlineto{\pgfqpoint{1.369619in}{2.028508in}}%
\pgfpathlineto{\pgfqpoint{1.369619in}{2.024757in}}%
\pgfpathlineto{\pgfqpoint{1.368366in}{2.023256in}}%
\pgfpathlineto{\pgfqpoint{1.366485in}{2.021005in}}%
\pgfpathlineto{\pgfqpoint{1.366485in}{2.017254in}}%
\pgfpathlineto{\pgfqpoint{1.366485in}{2.013502in}}%
\pgfpathlineto{\pgfqpoint{1.366485in}{2.009751in}}%
\pgfpathlineto{\pgfqpoint{1.365231in}{2.008250in}}%
\pgfpathlineto{\pgfqpoint{1.363350in}{2.005999in}}%
\pgfpathlineto{\pgfqpoint{1.363350in}{2.002248in}}%
\pgfpathlineto{\pgfqpoint{1.363350in}{1.998496in}}%
\pgfpathlineto{\pgfqpoint{1.362096in}{1.996995in}}%
\pgfpathlineto{\pgfqpoint{1.360215in}{1.994745in}}%
\pgfpathlineto{\pgfqpoint{1.360215in}{1.990993in}}%
\pgfpathlineto{\pgfqpoint{1.360215in}{1.987241in}}%
\pgfpathlineto{\pgfqpoint{1.360215in}{1.983490in}}%
\pgfpathlineto{\pgfqpoint{1.358961in}{1.981989in}}%
\pgfpathlineto{\pgfqpoint{1.357080in}{1.979738in}}%
\pgfpathlineto{\pgfqpoint{1.357080in}{1.975987in}}%
\pgfpathlineto{\pgfqpoint{1.357080in}{1.972235in}}%
\pgfpathlineto{\pgfqpoint{1.355827in}{1.970735in}}%
\pgfpathlineto{\pgfqpoint{1.353946in}{1.968484in}}%
\pgfpathlineto{\pgfqpoint{1.353946in}{1.964732in}}%
\pgfpathlineto{\pgfqpoint{1.353946in}{1.960981in}}%
\pgfpathlineto{\pgfqpoint{1.353946in}{1.957229in}}%
\pgfpathlineto{\pgfqpoint{1.352692in}{1.955729in}}%
\pgfpathlineto{\pgfqpoint{1.350811in}{1.953478in}}%
\pgfpathlineto{\pgfqpoint{1.350811in}{1.949726in}}%
\pgfpathlineto{\pgfqpoint{1.350811in}{1.945975in}}%
\pgfpathlineto{\pgfqpoint{1.349557in}{1.944474in}}%
\pgfpathlineto{\pgfqpoint{1.347676in}{1.942223in}}%
\pgfpathlineto{\pgfqpoint{1.347676in}{1.938471in}}%
\pgfpathlineto{\pgfqpoint{1.347676in}{1.934720in}}%
\pgfpathlineto{\pgfqpoint{1.347676in}{1.930968in}}%
\pgfpathlineto{\pgfqpoint{1.346422in}{1.929468in}}%
\pgfpathlineto{\pgfqpoint{1.344542in}{1.927217in}}%
\pgfpathlineto{\pgfqpoint{1.344542in}{1.923465in}}%
\pgfpathlineto{\pgfqpoint{1.344542in}{1.919714in}}%
\pgfpathlineto{\pgfqpoint{1.343288in}{1.918213in}}%
\pgfpathlineto{\pgfqpoint{1.341407in}{1.915962in}}%
\pgfpathlineto{\pgfqpoint{1.341407in}{1.912211in}}%
\pgfpathlineto{\pgfqpoint{1.341407in}{1.908459in}}%
\pgfpathlineto{\pgfqpoint{1.341407in}{1.904708in}}%
\pgfpathlineto{\pgfqpoint{1.340153in}{1.903207in}}%
\pgfpathlineto{\pgfqpoint{1.338272in}{1.900956in}}%
\pgfpathlineto{\pgfqpoint{1.338272in}{1.897205in}}%
\pgfpathlineto{\pgfqpoint{1.338272in}{1.893453in}}%
\pgfpathlineto{\pgfqpoint{1.337018in}{1.891952in}}%
\pgfpathlineto{\pgfqpoint{1.335137in}{1.889702in}}%
\pgfpathlineto{\pgfqpoint{1.335137in}{1.885950in}}%
\pgfpathlineto{\pgfqpoint{1.335137in}{1.882198in}}%
\pgfpathlineto{\pgfqpoint{1.335137in}{1.878447in}}%
\pgfpathlineto{\pgfqpoint{1.333883in}{1.876946in}}%
\pgfpathlineto{\pgfqpoint{1.332003in}{1.874695in}}%
\pgfpathlineto{\pgfqpoint{1.332003in}{1.870944in}}%
\pgfpathlineto{\pgfqpoint{1.332003in}{1.867192in}}%
\pgfpathlineto{\pgfqpoint{1.330749in}{1.865692in}}%
\pgfpathlineto{\pgfqpoint{1.328868in}{1.863441in}}%
\pgfpathlineto{\pgfqpoint{1.328868in}{1.859689in}}%
\pgfpathlineto{\pgfqpoint{1.328868in}{1.855938in}}%
\pgfpathlineto{\pgfqpoint{1.328868in}{1.852186in}}%
\pgfpathlineto{\pgfqpoint{1.327614in}{1.850686in}}%
\pgfpathlineto{\pgfqpoint{1.325733in}{1.848435in}}%
\pgfpathlineto{\pgfqpoint{1.325733in}{1.844683in}}%
\pgfpathlineto{\pgfqpoint{1.325733in}{1.840932in}}%
\pgfpathlineto{\pgfqpoint{1.324479in}{1.839431in}}%
\pgfpathlineto{\pgfqpoint{1.322598in}{1.837180in}}%
\pgfpathlineto{\pgfqpoint{1.322598in}{1.833429in}}%
\pgfpathlineto{\pgfqpoint{1.322598in}{1.829677in}}%
\pgfpathlineto{\pgfqpoint{1.322598in}{1.825925in}}%
\pgfpathlineto{\pgfqpoint{1.321344in}{1.824425in}}%
\pgfpathlineto{\pgfqpoint{1.319464in}{1.822174in}}%
\pgfpathlineto{\pgfqpoint{1.319464in}{1.818422in}}%
\pgfpathlineto{\pgfqpoint{1.319464in}{1.814671in}}%
\pgfpathlineto{\pgfqpoint{1.318210in}{1.813170in}}%
\pgfpathlineto{\pgfqpoint{1.316329in}{1.810919in}}%
\pgfpathlineto{\pgfqpoint{1.316329in}{1.807168in}}%
\pgfpathlineto{\pgfqpoint{1.316329in}{1.803416in}}%
\pgfpathlineto{\pgfqpoint{1.315075in}{1.801916in}}%
\pgfpathlineto{\pgfqpoint{1.313194in}{1.799665in}}%
\pgfpathlineto{\pgfqpoint{1.313194in}{1.795913in}}%
\pgfpathlineto{\pgfqpoint{1.313194in}{1.792162in}}%
\pgfpathlineto{\pgfqpoint{1.313194in}{1.788410in}}%
\pgfpathlineto{\pgfqpoint{1.311940in}{1.786909in}}%
\pgfpathlineto{\pgfqpoint{1.310059in}{1.784659in}}%
\pgfpathlineto{\pgfqpoint{1.310059in}{1.780907in}}%
\pgfpathlineto{\pgfqpoint{1.310059in}{1.777155in}}%
\pgfpathlineto{\pgfqpoint{1.308805in}{1.775655in}}%
\pgfpathlineto{\pgfqpoint{1.306925in}{1.773404in}}%
\pgfpathlineto{\pgfqpoint{1.306925in}{1.769652in}}%
\pgfpathlineto{\pgfqpoint{1.306925in}{1.765901in}}%
\pgfpathlineto{\pgfqpoint{1.306925in}{1.762149in}}%
\pgfpathlineto{\pgfqpoint{1.305671in}{1.760649in}}%
\pgfpathlineto{\pgfqpoint{1.303790in}{1.758398in}}%
\pgfpathlineto{\pgfqpoint{1.303790in}{1.754646in}}%
\pgfpathlineto{\pgfqpoint{1.303790in}{1.750895in}}%
\pgfpathlineto{\pgfqpoint{1.302536in}{1.749394in}}%
\pgfpathlineto{\pgfqpoint{1.300655in}{1.747143in}}%
\pgfpathlineto{\pgfqpoint{1.300655in}{1.743392in}}%
\pgfpathlineto{\pgfqpoint{1.300655in}{1.739640in}}%
\pgfpathlineto{\pgfqpoint{1.300655in}{1.735889in}}%
\pgfpathlineto{\pgfqpoint{1.299401in}{1.734388in}}%
\pgfpathlineto{\pgfqpoint{1.297520in}{1.732137in}}%
\pgfpathlineto{\pgfqpoint{1.297520in}{1.728386in}}%
\pgfpathlineto{\pgfqpoint{1.297520in}{1.724634in}}%
\pgfpathlineto{\pgfqpoint{1.296266in}{1.723133in}}%
\pgfpathlineto{\pgfqpoint{1.294386in}{1.720882in}}%
\pgfpathlineto{\pgfqpoint{1.294386in}{1.717131in}}%
\pgfpathlineto{\pgfqpoint{1.294386in}{1.713379in}}%
\pgfpathlineto{\pgfqpoint{1.294386in}{1.709628in}}%
\pgfpathlineto{\pgfqpoint{1.293132in}{1.708127in}}%
\pgfpathlineto{\pgfqpoint{1.291251in}{1.705876in}}%
\pgfpathlineto{\pgfqpoint{1.291251in}{1.702125in}}%
\pgfpathlineto{\pgfqpoint{1.293132in}{1.699874in}}%
\pgfpathlineto{\pgfqpoint{1.294386in}{1.698373in}}%
\pgfpathlineto{\pgfqpoint{1.296266in}{1.696122in}}%
\pgfpathlineto{\pgfqpoint{1.297520in}{1.694622in}}%
\pgfpathlineto{\pgfqpoint{1.299401in}{1.692371in}}%
\pgfpathlineto{\pgfqpoint{1.302536in}{1.692371in}}%
\pgfpathlineto{\pgfqpoint{1.303790in}{1.690870in}}%
\pgfpathlineto{\pgfqpoint{1.305671in}{1.688619in}}%
\pgfpathlineto{\pgfqpoint{1.306925in}{1.687119in}}%
\pgfpathlineto{\pgfqpoint{1.308805in}{1.684868in}}%
\pgfpathlineto{\pgfqpoint{1.310059in}{1.683367in}}%
\pgfpathlineto{\pgfqpoint{1.311940in}{1.681116in}}%
\pgfpathlineto{\pgfqpoint{1.315075in}{1.681116in}}%
\pgfpathlineto{\pgfqpoint{1.316329in}{1.679616in}}%
\pgfpathlineto{\pgfqpoint{1.318210in}{1.677365in}}%
\pgfpathlineto{\pgfqpoint{1.319464in}{1.675864in}}%
\pgfpathlineto{\pgfqpoint{1.321344in}{1.673613in}}%
\pgfpathlineto{\pgfqpoint{1.322598in}{1.672113in}}%
\pgfpathlineto{\pgfqpoint{1.324479in}{1.669862in}}%
\pgfpathlineto{\pgfqpoint{1.327614in}{1.669862in}}%
\pgfpathlineto{\pgfqpoint{1.328868in}{1.668361in}}%
\pgfpathlineto{\pgfqpoint{1.330749in}{1.666110in}}%
\pgfpathlineto{\pgfqpoint{1.332003in}{1.664609in}}%
\pgfpathlineto{\pgfqpoint{1.333883in}{1.662359in}}%
\pgfpathlineto{\pgfqpoint{1.335137in}{1.660858in}}%
\pgfpathlineto{\pgfqpoint{1.337018in}{1.658607in}}%
\pgfpathlineto{\pgfqpoint{1.338272in}{1.657106in}}%
\pgfpathlineto{\pgfqpoint{1.340153in}{1.654855in}}%
\pgfpathlineto{\pgfqpoint{1.343288in}{1.654855in}}%
\pgfpathlineto{\pgfqpoint{1.344542in}{1.653355in}}%
\pgfpathlineto{\pgfqpoint{1.346422in}{1.651104in}}%
\pgfpathlineto{\pgfqpoint{1.347676in}{1.649603in}}%
\pgfpathlineto{\pgfqpoint{1.349557in}{1.647352in}}%
\pgfpathlineto{\pgfqpoint{1.350811in}{1.645852in}}%
\pgfpathlineto{\pgfqpoint{1.352692in}{1.643601in}}%
\pgfpathlineto{\pgfqpoint{1.355827in}{1.643601in}}%
\pgfpathlineto{\pgfqpoint{1.357080in}{1.642100in}}%
\pgfpathlineto{\pgfqpoint{1.358961in}{1.639849in}}%
\pgfpathlineto{\pgfqpoint{1.360215in}{1.638349in}}%
\pgfpathlineto{\pgfqpoint{1.362096in}{1.636098in}}%
\pgfpathlineto{\pgfqpoint{1.363350in}{1.634597in}}%
\pgfpathlineto{\pgfqpoint{1.365231in}{1.632346in}}%
\pgfpathlineto{\pgfqpoint{1.368366in}{1.632346in}}%
\pgfpathlineto{\pgfqpoint{1.369619in}{1.630846in}}%
\pgfpathlineto{\pgfqpoint{1.371500in}{1.628595in}}%
\pgfpathlineto{\pgfqpoint{1.372754in}{1.627094in}}%
\pgfpathlineto{\pgfqpoint{1.374635in}{1.624843in}}%
\pgfpathlineto{\pgfqpoint{1.375889in}{1.623343in}}%
\pgfpathlineto{\pgfqpoint{1.377770in}{1.621092in}}%
\pgfpathlineto{\pgfqpoint{1.380905in}{1.621092in}}%
\pgfpathlineto{\pgfqpoint{1.382158in}{1.619591in}}%
\pgfpathlineto{\pgfqpoint{1.384039in}{1.617340in}}%
\pgfpathlineto{\pgfqpoint{1.385293in}{1.615840in}}%
\pgfpathlineto{\pgfqpoint{1.387174in}{1.613589in}}%
\pgfpathlineto{\pgfqpoint{1.388428in}{1.612088in}}%
\pgfpathlineto{\pgfqpoint{1.390309in}{1.609837in}}%
\pgfpathlineto{\pgfqpoint{1.393444in}{1.609837in}}%
\pgfpathlineto{\pgfqpoint{1.394697in}{1.608336in}}%
\pgfpathlineto{\pgfqpoint{1.396578in}{1.606086in}}%
\pgfpathlineto{\pgfqpoint{1.397832in}{1.604585in}}%
\pgfpathlineto{\pgfqpoint{1.399713in}{1.602334in}}%
\pgfpathlineto{\pgfqpoint{1.400967in}{1.600833in}}%
\pgfpathlineto{\pgfqpoint{1.402848in}{1.598582in}}%
\pgfpathlineto{\pgfqpoint{1.405982in}{1.598582in}}%
\pgfpathlineto{\pgfqpoint{1.407236in}{1.597082in}}%
\pgfpathlineto{\pgfqpoint{1.409117in}{1.594831in}}%
\pgfpathlineto{\pgfqpoint{1.410371in}{1.593330in}}%
\pgfpathlineto{\pgfqpoint{1.412252in}{1.591079in}}%
\pgfpathlineto{\pgfqpoint{1.413506in}{1.589579in}}%
\pgfpathlineto{\pgfqpoint{1.415387in}{1.587328in}}%
\pgfpathlineto{\pgfqpoint{1.416641in}{1.585827in}}%
\pgfpathlineto{\pgfqpoint{1.418521in}{1.583576in}}%
\pgfpathlineto{\pgfqpoint{1.421656in}{1.583576in}}%
\pgfpathlineto{\pgfqpoint{1.422910in}{1.582076in}}%
\pgfpathlineto{\pgfqpoint{1.424791in}{1.579825in}}%
\pgfpathlineto{\pgfqpoint{1.426045in}{1.578324in}}%
\pgfpathlineto{\pgfqpoint{1.427926in}{1.576073in}}%
\pgfpathlineto{\pgfqpoint{1.429180in}{1.574573in}}%
\pgfpathlineto{\pgfqpoint{1.431060in}{1.572322in}}%
\pgfpathlineto{\pgfqpoint{1.434195in}{1.572322in}}%
\pgfpathlineto{\pgfqpoint{1.435449in}{1.570821in}}%
\pgfpathlineto{\pgfqpoint{1.437330in}{1.568570in}}%
\pgfpathlineto{\pgfqpoint{1.438584in}{1.567070in}}%
\pgfpathlineto{\pgfqpoint{1.440465in}{1.564819in}}%
\pgfpathlineto{\pgfqpoint{1.441719in}{1.563318in}}%
\pgfpathlineto{\pgfqpoint{1.443599in}{1.561067in}}%
\pgfpathlineto{\pgfqpoint{1.446734in}{1.561067in}}%
\pgfpathlineto{\pgfqpoint{1.447988in}{1.559566in}}%
\pgfpathlineto{\pgfqpoint{1.449869in}{1.557316in}}%
\pgfpathlineto{\pgfqpoint{1.451123in}{1.555815in}}%
\pgfpathlineto{\pgfqpoint{1.453004in}{1.553564in}}%
\pgfpathlineto{\pgfqpoint{1.454258in}{1.552063in}}%
\pgfpathlineto{\pgfqpoint{1.456138in}{1.549813in}}%
\pgfpathlineto{\pgfqpoint{1.459273in}{1.549813in}}%
\pgfpathlineto{\pgfqpoint{1.460527in}{1.548312in}}%
\pgfpathlineto{\pgfqpoint{1.462408in}{1.546061in}}%
\pgfpathlineto{\pgfqpoint{1.463662in}{1.544560in}}%
\pgfpathlineto{\pgfqpoint{1.465543in}{1.542309in}}%
\pgfpathlineto{\pgfqpoint{1.466796in}{1.540809in}}%
\pgfpathlineto{\pgfqpoint{1.468677in}{1.538558in}}%
\pgfpathlineto{\pgfqpoint{1.471812in}{1.538558in}}%
\pgfpathlineto{\pgfqpoint{1.473066in}{1.537057in}}%
\pgfpathlineto{\pgfqpoint{1.474947in}{1.534806in}}%
\pgfpathlineto{\pgfqpoint{1.476201in}{1.533306in}}%
\pgfpathlineto{\pgfqpoint{1.478082in}{1.531055in}}%
\pgfpathlineto{\pgfqpoint{1.479335in}{1.529554in}}%
\pgfpathlineto{\pgfqpoint{1.481216in}{1.527303in}}%
\pgfpathlineto{\pgfqpoint{1.484351in}{1.527303in}}%
\pgfpathlineto{\pgfqpoint{1.485605in}{1.525803in}}%
\pgfpathlineto{\pgfqpoint{1.487486in}{1.523552in}}%
\pgfpathlineto{\pgfqpoint{1.488740in}{1.522051in}}%
\pgfpathlineto{\pgfqpoint{1.490621in}{1.519800in}}%
\pgfpathlineto{\pgfqpoint{1.491874in}{1.518300in}}%
\pgfpathlineto{\pgfqpoint{1.493755in}{1.516049in}}%
\pgfpathlineto{\pgfqpoint{1.495009in}{1.514548in}}%
\pgfpathlineto{\pgfqpoint{1.496890in}{1.512297in}}%
\pgfpathlineto{\pgfqpoint{1.500025in}{1.512297in}}%
\pgfpathlineto{\pgfqpoint{1.501279in}{1.510797in}}%
\pgfpathlineto{\pgfqpoint{1.503159in}{1.508546in}}%
\pgfpathlineto{\pgfqpoint{1.504413in}{1.507045in}}%
\pgfpathlineto{\pgfqpoint{1.506294in}{1.504794in}}%
\pgfpathlineto{\pgfqpoint{1.507548in}{1.503293in}}%
\pgfpathlineto{\pgfqpoint{1.509429in}{1.501043in}}%
\pgfpathlineto{\pgfqpoint{1.512564in}{1.501043in}}%
\pgfpathlineto{\pgfqpoint{1.513818in}{1.499542in}}%
\pgfpathlineto{\pgfqpoint{1.515698in}{1.497291in}}%
\pgfpathlineto{\pgfqpoint{1.516952in}{1.495790in}}%
\pgfpathlineto{\pgfqpoint{1.518833in}{1.493539in}}%
\pgfpathlineto{\pgfqpoint{1.520087in}{1.492039in}}%
\pgfpathlineto{\pgfqpoint{1.521968in}{1.489788in}}%
\pgfpathlineto{\pgfqpoint{1.525103in}{1.489788in}}%
\pgfpathlineto{\pgfqpoint{1.526357in}{1.488287in}}%
\pgfpathlineto{\pgfqpoint{1.528237in}{1.486036in}}%
\pgfpathlineto{\pgfqpoint{1.529491in}{1.484536in}}%
\pgfpathlineto{\pgfqpoint{1.531372in}{1.482285in}}%
\pgfpathlineto{\pgfqpoint{1.532626in}{1.480784in}}%
\pgfpathlineto{\pgfqpoint{1.534507in}{1.478533in}}%
\pgfpathlineto{\pgfqpoint{1.537642in}{1.478533in}}%
\pgfpathlineto{\pgfqpoint{1.538896in}{1.477033in}}%
\pgfpathlineto{\pgfqpoint{1.540776in}{1.474782in}}%
\pgfpathlineto{\pgfqpoint{1.542030in}{1.473281in}}%
\pgfpathlineto{\pgfqpoint{1.543911in}{1.471030in}}%
\pgfpathlineto{\pgfqpoint{1.545165in}{1.469530in}}%
\pgfpathlineto{\pgfqpoint{1.547046in}{1.467279in}}%
\pgfpathlineto{\pgfqpoint{1.550181in}{1.467279in}}%
\pgfpathlineto{\pgfqpoint{1.551435in}{1.465778in}}%
\pgfpathlineto{\pgfqpoint{1.553315in}{1.463527in}}%
\pgfpathlineto{\pgfqpoint{1.554569in}{1.462027in}}%
\pgfpathlineto{\pgfqpoint{1.556450in}{1.459776in}}%
\pgfpathlineto{\pgfqpoint{1.557704in}{1.458275in}}%
\pgfpathlineto{\pgfqpoint{1.559585in}{1.456024in}}%
\pgfpathlineto{\pgfqpoint{1.560839in}{1.454524in}}%
\pgfpathlineto{\pgfqpoint{1.562720in}{1.452273in}}%
\pgfpathlineto{\pgfqpoint{1.565854in}{1.452273in}}%
\pgfpathlineto{\pgfqpoint{1.567108in}{1.450772in}}%
\pgfpathlineto{\pgfqpoint{1.568989in}{1.448521in}}%
\pgfpathlineto{\pgfqpoint{1.570243in}{1.447020in}}%
\pgfpathlineto{\pgfqpoint{1.572124in}{1.444770in}}%
\pgfpathlineto{\pgfqpoint{1.573378in}{1.443269in}}%
\pgfpathlineto{\pgfqpoint{1.575259in}{1.441018in}}%
\pgfpathlineto{\pgfqpoint{1.578393in}{1.441018in}}%
\pgfpathlineto{\pgfqpoint{1.579647in}{1.439517in}}%
\pgfpathlineto{\pgfqpoint{1.581528in}{1.437266in}}%
\pgfpathlineto{\pgfqpoint{1.582782in}{1.435766in}}%
\pgfpathlineto{\pgfqpoint{1.584663in}{1.433515in}}%
\pgfpathlineto{\pgfqpoint{1.585917in}{1.432014in}}%
\pgfpathlineto{\pgfqpoint{1.587798in}{1.429763in}}%
\pgfpathlineto{\pgfqpoint{1.590932in}{1.429763in}}%
\pgfpathlineto{\pgfqpoint{1.592186in}{1.428263in}}%
\pgfpathlineto{\pgfqpoint{1.594067in}{1.426012in}}%
\pgfpathlineto{\pgfqpoint{1.595321in}{1.424511in}}%
\pgfpathlineto{\pgfqpoint{1.597202in}{1.422260in}}%
\pgfpathlineto{\pgfqpoint{1.598456in}{1.420760in}}%
\pgfpathlineto{\pgfqpoint{1.600337in}{1.418509in}}%
\pgfpathlineto{\pgfqpoint{1.603471in}{1.418509in}}%
\pgfpathlineto{\pgfqpoint{1.604725in}{1.417008in}}%
\pgfpathlineto{\pgfqpoint{1.606606in}{1.414757in}}%
\pgfpathlineto{\pgfqpoint{1.607860in}{1.413257in}}%
\pgfpathlineto{\pgfqpoint{1.609741in}{1.411006in}}%
\pgfpathlineto{\pgfqpoint{1.610995in}{1.409505in}}%
\pgfpathlineto{\pgfqpoint{1.612875in}{1.407254in}}%
\pgfpathlineto{\pgfqpoint{1.616010in}{1.407254in}}%
\pgfpathlineto{\pgfqpoint{1.617264in}{1.405754in}}%
\pgfpathlineto{\pgfqpoint{1.619145in}{1.403503in}}%
\pgfpathlineto{\pgfqpoint{1.620399in}{1.402002in}}%
\pgfpathlineto{\pgfqpoint{1.622280in}{1.399751in}}%
\pgfpathlineto{\pgfqpoint{1.623534in}{1.398251in}}%
\pgfpathlineto{\pgfqpoint{1.625414in}{1.396000in}}%
\pgfpathlineto{\pgfqpoint{1.628549in}{1.396000in}}%
\pgfpathlineto{\pgfqpoint{1.629803in}{1.394499in}}%
\pgfpathlineto{\pgfqpoint{1.631684in}{1.392248in}}%
\pgfpathlineto{\pgfqpoint{1.632938in}{1.390747in}}%
\pgfpathlineto{\pgfqpoint{1.634819in}{1.388497in}}%
\pgfpathlineto{\pgfqpoint{1.636073in}{1.386996in}}%
\pgfpathlineto{\pgfqpoint{1.637953in}{1.384745in}}%
\pgfpathlineto{\pgfqpoint{1.639207in}{1.383244in}}%
\pgfpathlineto{\pgfqpoint{1.641088in}{1.380993in}}%
\pgfpathlineto{\pgfqpoint{1.644223in}{1.380993in}}%
\pgfpathlineto{\pgfqpoint{1.645477in}{1.379493in}}%
\pgfpathlineto{\pgfqpoint{1.647358in}{1.377242in}}%
\pgfpathlineto{\pgfqpoint{1.648612in}{1.375741in}}%
\pgfpathlineto{\pgfqpoint{1.650492in}{1.373490in}}%
\pgfpathlineto{\pgfqpoint{1.651746in}{1.371990in}}%
\pgfpathlineto{\pgfqpoint{1.653627in}{1.369739in}}%
\pgfpathlineto{\pgfqpoint{1.656762in}{1.369739in}}%
\pgfpathlineto{\pgfqpoint{1.658016in}{1.368238in}}%
\pgfpathlineto{\pgfqpoint{1.659897in}{1.365987in}}%
\pgfpathlineto{\pgfqpoint{1.661151in}{1.364487in}}%
\pgfpathlineto{\pgfqpoint{1.663031in}{1.362236in}}%
\pgfpathlineto{\pgfqpoint{1.664285in}{1.360735in}}%
\pgfpathlineto{\pgfqpoint{1.666166in}{1.358484in}}%
\pgfpathlineto{\pgfqpoint{1.669301in}{1.358484in}}%
\pgfpathlineto{\pgfqpoint{1.670555in}{1.356984in}}%
\pgfpathlineto{\pgfqpoint{1.672436in}{1.354733in}}%
\pgfpathlineto{\pgfqpoint{1.673689in}{1.353232in}}%
\pgfpathlineto{\pgfqpoint{1.675570in}{1.350981in}}%
\pgfpathlineto{\pgfqpoint{1.676824in}{1.349481in}}%
\pgfpathlineto{\pgfqpoint{1.678705in}{1.347230in}}%
\pgfpathlineto{\pgfqpoint{1.681840in}{1.347230in}}%
\pgfpathlineto{\pgfqpoint{1.683094in}{1.345729in}}%
\pgfpathlineto{\pgfqpoint{1.684975in}{1.343478in}}%
\pgfpathlineto{\pgfqpoint{1.686228in}{1.341977in}}%
\pgfpathlineto{\pgfqpoint{1.688109in}{1.339727in}}%
\pgfpathlineto{\pgfqpoint{1.689363in}{1.338226in}}%
\pgfpathlineto{\pgfqpoint{1.691244in}{1.335975in}}%
\pgfpathlineto{\pgfqpoint{1.694379in}{1.335975in}}%
\pgfpathlineto{\pgfqpoint{1.695633in}{1.334474in}}%
\pgfpathlineto{\pgfqpoint{1.697514in}{1.332224in}}%
\pgfpathlineto{\pgfqpoint{1.698767in}{1.330723in}}%
\pgfpathlineto{\pgfqpoint{1.700648in}{1.328472in}}%
\pgfpathlineto{\pgfqpoint{1.701902in}{1.326971in}}%
\pgfpathlineto{\pgfqpoint{1.703783in}{1.324720in}}%
\pgfpathlineto{\pgfqpoint{1.706918in}{1.324720in}}%
\pgfpathlineto{\pgfqpoint{1.708172in}{1.323220in}}%
\pgfpathlineto{\pgfqpoint{1.710052in}{1.320969in}}%
\pgfpathlineto{\pgfqpoint{1.711306in}{1.319468in}}%
\pgfpathlineto{\pgfqpoint{1.713187in}{1.317217in}}%
\pgfpathlineto{\pgfqpoint{1.714441in}{1.315717in}}%
\pgfpathlineto{\pgfqpoint{1.716322in}{1.313466in}}%
\pgfpathlineto{\pgfqpoint{1.717576in}{1.311965in}}%
\pgfpathlineto{\pgfqpoint{1.719457in}{1.309714in}}%
\pgfpathlineto{\pgfqpoint{1.722591in}{1.309714in}}%
\pgfpathlineto{\pgfqpoint{1.723845in}{1.308214in}}%
\pgfpathlineto{\pgfqpoint{1.725726in}{1.305963in}}%
\pgfpathlineto{\pgfqpoint{1.726980in}{1.304462in}}%
\pgfpathlineto{\pgfqpoint{1.728861in}{1.302211in}}%
\pgfpathlineto{\pgfqpoint{1.730115in}{1.300711in}}%
\pgfpathlineto{\pgfqpoint{1.731996in}{1.298460in}}%
\pgfpathlineto{\pgfqpoint{1.735130in}{1.298460in}}%
\pgfpathlineto{\pgfqpoint{1.736384in}{1.296959in}}%
\pgfpathlineto{\pgfqpoint{1.738265in}{1.294708in}}%
\pgfpathlineto{\pgfqpoint{1.739519in}{1.293208in}}%
\pgfpathlineto{\pgfqpoint{1.741400in}{1.290957in}}%
\pgfpathlineto{\pgfqpoint{1.742654in}{1.289456in}}%
\pgfpathlineto{\pgfqpoint{1.744535in}{1.287205in}}%
\pgfpathlineto{\pgfqpoint{1.747669in}{1.287205in}}%
\pgfpathlineto{\pgfqpoint{1.748923in}{1.285704in}}%
\pgfpathlineto{\pgfqpoint{1.750804in}{1.283454in}}%
\pgfpathlineto{\pgfqpoint{1.752058in}{1.281953in}}%
\pgfpathlineto{\pgfqpoint{1.753939in}{1.279702in}}%
\pgfpathlineto{\pgfqpoint{1.755193in}{1.278201in}}%
\pgfpathlineto{\pgfqpoint{1.757074in}{1.275950in}}%
\pgfpathlineto{\pgfqpoint{1.760208in}{1.275950in}}%
\pgfpathlineto{\pgfqpoint{1.761462in}{1.274450in}}%
\pgfpathlineto{\pgfqpoint{1.763343in}{1.272199in}}%
\pgfpathlineto{\pgfqpoint{1.764597in}{1.270698in}}%
\pgfpathlineto{\pgfqpoint{1.766478in}{1.268447in}}%
\pgfpathlineto{\pgfqpoint{1.767732in}{1.266947in}}%
\pgfpathlineto{\pgfqpoint{1.769613in}{1.264696in}}%
\pgfpathlineto{\pgfqpoint{1.772747in}{1.264696in}}%
\pgfpathlineto{\pgfqpoint{1.774001in}{1.263195in}}%
\pgfpathlineto{\pgfqpoint{1.775882in}{1.260944in}}%
\pgfpathlineto{\pgfqpoint{1.777136in}{1.259444in}}%
\pgfpathlineto{\pgfqpoint{1.779017in}{1.257193in}}%
\pgfpathlineto{\pgfqpoint{1.780271in}{1.255692in}}%
\pgfpathlineto{\pgfqpoint{1.782152in}{1.253441in}}%
\pgfpathlineto{\pgfqpoint{1.785286in}{1.253441in}}%
\pgfpathlineto{\pgfqpoint{1.786540in}{1.251941in}}%
\pgfpathlineto{\pgfqpoint{1.788421in}{1.249690in}}%
\pgfpathlineto{\pgfqpoint{1.789675in}{1.248189in}}%
\pgfpathlineto{\pgfqpoint{1.791556in}{1.245938in}}%
\pgfpathlineto{\pgfqpoint{1.792810in}{1.244438in}}%
\pgfpathlineto{\pgfqpoint{1.794691in}{1.242187in}}%
\pgfpathlineto{\pgfqpoint{1.795944in}{1.240686in}}%
\pgfpathlineto{\pgfqpoint{1.797825in}{1.238435in}}%
\pgfpathlineto{\pgfqpoint{1.800960in}{1.238435in}}%
\pgfpathlineto{\pgfqpoint{1.802214in}{1.236935in}}%
\pgfpathlineto{\pgfqpoint{1.804095in}{1.234684in}}%
\pgfpathlineto{\pgfqpoint{1.805349in}{1.233183in}}%
\pgfpathlineto{\pgfqpoint{1.807229in}{1.230932in}}%
\pgfpathlineto{\pgfqpoint{1.808483in}{1.229431in}}%
\pgfpathlineto{\pgfqpoint{1.810364in}{1.227181in}}%
\pgfpathlineto{\pgfqpoint{1.813499in}{1.227181in}}%
\pgfpathlineto{\pgfqpoint{1.814753in}{1.225680in}}%
\pgfpathlineto{\pgfqpoint{1.816634in}{1.223429in}}%
\pgfpathlineto{\pgfqpoint{1.817888in}{1.221928in}}%
\pgfpathlineto{\pgfqpoint{1.819768in}{1.219677in}}%
\pgfpathlineto{\pgfqpoint{1.821022in}{1.218177in}}%
\pgfpathlineto{\pgfqpoint{1.822903in}{1.215926in}}%
\pgfpathlineto{\pgfqpoint{1.826038in}{1.215926in}}%
\pgfpathlineto{\pgfqpoint{1.827292in}{1.214425in}}%
\pgfpathlineto{\pgfqpoint{1.829173in}{1.212174in}}%
\pgfpathlineto{\pgfqpoint{1.830427in}{1.210674in}}%
\pgfpathlineto{\pgfqpoint{1.832307in}{1.208423in}}%
\pgfpathlineto{\pgfqpoint{1.833561in}{1.206922in}}%
\pgfpathlineto{\pgfqpoint{1.835442in}{1.204671in}}%
\pgfpathlineto{\pgfqpoint{1.838577in}{1.204671in}}%
\pgfpathlineto{\pgfqpoint{1.839831in}{1.203171in}}%
\pgfpathlineto{\pgfqpoint{1.841712in}{1.200920in}}%
\pgfpathlineto{\pgfqpoint{1.842966in}{1.199419in}}%
\pgfpathlineto{\pgfqpoint{1.844846in}{1.197168in}}%
\pgfpathlineto{\pgfqpoint{1.846100in}{1.195668in}}%
\pgfpathlineto{\pgfqpoint{1.847981in}{1.193417in}}%
\pgfpathlineto{\pgfqpoint{1.851116in}{1.193417in}}%
\pgfpathlineto{\pgfqpoint{1.852370in}{1.191916in}}%
\pgfpathlineto{\pgfqpoint{1.854251in}{1.189665in}}%
\pgfpathlineto{\pgfqpoint{1.855505in}{1.188165in}}%
\pgfpathlineto{\pgfqpoint{1.857385in}{1.185914in}}%
\pgfpathlineto{\pgfqpoint{1.858639in}{1.184413in}}%
\pgfpathlineto{\pgfqpoint{1.860520in}{1.182162in}}%
\pgfpathlineto{\pgfqpoint{1.861774in}{1.180662in}}%
\pgfpathlineto{\pgfqpoint{1.863655in}{1.178411in}}%
\pgfpathlineto{\pgfqpoint{1.866790in}{1.178411in}}%
\pgfpathlineto{\pgfqpoint{1.868043in}{1.176910in}}%
\pgfpathlineto{\pgfqpoint{1.869924in}{1.174659in}}%
\pgfpathlineto{\pgfqpoint{1.871178in}{1.173158in}}%
\pgfpathlineto{\pgfqpoint{1.873059in}{1.170908in}}%
\pgfpathlineto{\pgfqpoint{1.874313in}{1.169407in}}%
\pgfpathlineto{\pgfqpoint{1.876194in}{1.167156in}}%
\pgfpathlineto{\pgfqpoint{1.879329in}{1.167156in}}%
\pgfpathlineto{\pgfqpoint{1.880582in}{1.165655in}}%
\pgfpathlineto{\pgfqpoint{1.882463in}{1.163404in}}%
\pgfpathlineto{\pgfqpoint{1.883717in}{1.161904in}}%
\pgfpathlineto{\pgfqpoint{1.885598in}{1.159653in}}%
\pgfpathlineto{\pgfqpoint{1.886852in}{1.158152in}}%
\pgfpathlineto{\pgfqpoint{1.888733in}{1.155901in}}%
\pgfpathlineto{\pgfqpoint{1.891868in}{1.155901in}}%
\pgfpathlineto{\pgfqpoint{1.893121in}{1.154401in}}%
\pgfpathlineto{\pgfqpoint{1.895002in}{1.152150in}}%
\pgfpathlineto{\pgfqpoint{1.896256in}{1.150649in}}%
\pgfpathlineto{\pgfqpoint{1.898137in}{1.148398in}}%
\pgfpathlineto{\pgfqpoint{1.899391in}{1.146898in}}%
\pgfpathlineto{\pgfqpoint{1.901272in}{1.144647in}}%
\pgfpathlineto{\pgfqpoint{1.904407in}{1.144647in}}%
\pgfpathlineto{\pgfqpoint{1.905660in}{1.143146in}}%
\pgfpathlineto{\pgfqpoint{1.907541in}{1.140895in}}%
\pgfpathlineto{\pgfqpoint{1.908795in}{1.139395in}}%
\pgfpathlineto{\pgfqpoint{1.910676in}{1.137144in}}%
\pgfpathlineto{\pgfqpoint{1.911930in}{1.135643in}}%
\pgfpathlineto{\pgfqpoint{1.913811in}{1.133392in}}%
\pgfpathlineto{\pgfqpoint{1.916945in}{1.133392in}}%
\pgfpathlineto{\pgfqpoint{1.918199in}{1.131892in}}%
\pgfpathlineto{\pgfqpoint{1.920080in}{1.129641in}}%
\pgfpathlineto{\pgfqpoint{1.921334in}{1.128140in}}%
\pgfpathlineto{\pgfqpoint{1.923215in}{1.125889in}}%
\pgfpathlineto{\pgfqpoint{1.924469in}{1.124388in}}%
\pgfpathlineto{\pgfqpoint{1.926350in}{1.122138in}}%
\pgfpathlineto{\pgfqpoint{1.929484in}{1.122138in}}%
\pgfpathlineto{\pgfqpoint{1.930738in}{1.120637in}}%
\pgfpathlineto{\pgfqpoint{1.932619in}{1.118386in}}%
\pgfpathlineto{\pgfqpoint{1.933873in}{1.116885in}}%
\pgfpathlineto{\pgfqpoint{1.935754in}{1.114635in}}%
\pgfpathlineto{\pgfqpoint{1.937008in}{1.113134in}}%
\pgfpathlineto{\pgfqpoint{1.938889in}{1.110883in}}%
\pgfpathlineto{\pgfqpoint{1.940143in}{1.109382in}}%
\pgfpathlineto{\pgfqpoint{1.942023in}{1.107131in}}%
\pgfpathlineto{\pgfqpoint{1.945158in}{1.107131in}}%
\pgfpathlineto{\pgfqpoint{1.946412in}{1.105631in}}%
\pgfpathlineto{\pgfqpoint{1.948293in}{1.103380in}}%
\pgfpathlineto{\pgfqpoint{1.949547in}{1.101879in}}%
\pgfpathlineto{\pgfqpoint{1.951428in}{1.099628in}}%
\pgfpathlineto{\pgfqpoint{1.952682in}{1.098128in}}%
\pgfpathlineto{\pgfqpoint{1.954562in}{1.095877in}}%
\pgfpathlineto{\pgfqpoint{1.957697in}{1.095877in}}%
\pgfpathlineto{\pgfqpoint{1.958951in}{1.094376in}}%
\pgfpathlineto{\pgfqpoint{1.960832in}{1.092125in}}%
\pgfpathlineto{\pgfqpoint{1.962086in}{1.090625in}}%
\pgfpathlineto{\pgfqpoint{1.963967in}{1.088374in}}%
\pgfpathlineto{\pgfqpoint{1.965221in}{1.086873in}}%
\pgfpathlineto{\pgfqpoint{1.967101in}{1.084622in}}%
\pgfpathlineto{\pgfqpoint{1.970236in}{1.084622in}}%
\pgfpathlineto{\pgfqpoint{1.971490in}{1.083122in}}%
\pgfpathlineto{\pgfqpoint{1.973371in}{1.080871in}}%
\pgfpathlineto{\pgfqpoint{1.974625in}{1.079370in}}%
\pgfpathlineto{\pgfqpoint{1.976506in}{1.077119in}}%
\pgfpathlineto{\pgfqpoint{1.977759in}{1.075619in}}%
\pgfpathlineto{\pgfqpoint{1.979640in}{1.073368in}}%
\pgfpathlineto{\pgfqpoint{1.982775in}{1.073368in}}%
\pgfpathlineto{\pgfqpoint{1.984029in}{1.071867in}}%
\pgfpathlineto{\pgfqpoint{1.985910in}{1.069616in}}%
\pgfpathlineto{\pgfqpoint{1.987164in}{1.068115in}}%
\pgfpathlineto{\pgfqpoint{1.989045in}{1.065865in}}%
\pgfpathlineto{\pgfqpoint{1.990298in}{1.064364in}}%
\pgfpathlineto{\pgfqpoint{1.992179in}{1.062113in}}%
\pgfpathlineto{\pgfqpoint{1.995314in}{1.062113in}}%
\pgfpathlineto{\pgfqpoint{1.996568in}{1.060612in}}%
\pgfpathlineto{\pgfqpoint{1.998449in}{1.058361in}}%
\pgfpathlineto{\pgfqpoint{1.999703in}{1.056861in}}%
\pgfpathlineto{\pgfqpoint{2.001584in}{1.054610in}}%
\pgfpathlineto{\pgfqpoint{2.002837in}{1.053109in}}%
\pgfpathlineto{\pgfqpoint{2.004718in}{1.050858in}}%
\pgfpathlineto{\pgfqpoint{2.007853in}{1.050858in}}%
\pgfpathlineto{\pgfqpoint{2.009107in}{1.049358in}}%
\pgfpathlineto{\pgfqpoint{2.010988in}{1.047107in}}%
\pgfpathlineto{\pgfqpoint{2.012242in}{1.045606in}}%
\pgfpathlineto{\pgfqpoint{2.014122in}{1.043355in}}%
\pgfpathlineto{\pgfqpoint{2.015376in}{1.041855in}}%
\pgfpathlineto{\pgfqpoint{2.017257in}{1.039604in}}%
\pgfpathlineto{\pgfqpoint{2.018511in}{1.038103in}}%
\pgfpathlineto{\pgfqpoint{2.020392in}{1.035852in}}%
\pgfpathlineto{\pgfqpoint{2.023527in}{1.035852in}}%
\pgfpathlineto{\pgfqpoint{2.024781in}{1.034352in}}%
\pgfpathlineto{\pgfqpoint{2.026661in}{1.032101in}}%
\pgfpathlineto{\pgfqpoint{2.027915in}{1.030600in}}%
\pgfpathlineto{\pgfqpoint{2.029796in}{1.028349in}}%
\pgfpathlineto{\pgfqpoint{2.031050in}{1.026849in}}%
\pgfpathlineto{\pgfqpoint{2.032931in}{1.024598in}}%
\pgfpathlineto{\pgfqpoint{2.036066in}{1.024598in}}%
\pgfpathlineto{\pgfqpoint{2.037320in}{1.023097in}}%
\pgfpathlineto{\pgfqpoint{2.039200in}{1.020846in}}%
\pgfpathlineto{\pgfqpoint{2.040454in}{1.019346in}}%
\pgfpathlineto{\pgfqpoint{2.042335in}{1.017095in}}%
\pgfpathlineto{\pgfqpoint{2.043589in}{1.015594in}}%
\pgfpathlineto{\pgfqpoint{2.045470in}{1.013343in}}%
\pgfpathlineto{\pgfqpoint{2.048605in}{1.013343in}}%
\pgfpathlineto{\pgfqpoint{2.049859in}{1.011842in}}%
\pgfpathlineto{\pgfqpoint{2.051739in}{1.009592in}}%
\pgfpathlineto{\pgfqpoint{2.052993in}{1.008091in}}%
\pgfpathlineto{\pgfqpoint{2.054874in}{1.005840in}}%
\pgfpathlineto{\pgfqpoint{2.056128in}{1.004339in}}%
\pgfpathlineto{\pgfqpoint{2.058009in}{1.002088in}}%
\pgfpathlineto{\pgfqpoint{2.061144in}{1.002088in}}%
\pgfpathlineto{\pgfqpoint{2.062398in}{1.000588in}}%
\pgfpathlineto{\pgfqpoint{2.064278in}{0.998337in}}%
\pgfpathlineto{\pgfqpoint{2.065532in}{0.996836in}}%
\pgfpathlineto{\pgfqpoint{2.067413in}{0.994585in}}%
\pgfpathlineto{\pgfqpoint{2.068667in}{0.993085in}}%
\pgfpathlineto{\pgfqpoint{2.070548in}{0.990834in}}%
\pgfpathlineto{\pgfqpoint{2.073683in}{0.990834in}}%
\pgfpathlineto{\pgfqpoint{2.074936in}{0.989333in}}%
\pgfpathlineto{\pgfqpoint{2.076817in}{0.987082in}}%
\pgfpathlineto{\pgfqpoint{2.078071in}{0.985582in}}%
\pgfpathlineto{\pgfqpoint{2.079952in}{0.983331in}}%
\pgfpathlineto{\pgfqpoint{2.081206in}{0.981830in}}%
\pgfpathlineto{\pgfqpoint{2.083087in}{0.979579in}}%
\pgfpathlineto{\pgfqpoint{2.086222in}{0.979579in}}%
\pgfpathlineto{\pgfqpoint{2.087475in}{0.978079in}}%
\pgfpathlineto{\pgfqpoint{2.089356in}{0.975828in}}%
\pgfpathlineto{\pgfqpoint{2.090610in}{0.974327in}}%
\pgfpathlineto{\pgfqpoint{2.092491in}{0.972076in}}%
\pgfpathlineto{\pgfqpoint{2.093745in}{0.970576in}}%
\pgfpathlineto{\pgfqpoint{2.095626in}{0.968325in}}%
\pgfpathlineto{\pgfqpoint{2.096880in}{0.966824in}}%
\pgfpathlineto{\pgfqpoint{2.098761in}{0.964573in}}%
\pgfpathlineto{\pgfqpoint{2.101895in}{0.964573in}}%
\pgfpathlineto{\pgfqpoint{2.103149in}{0.963073in}}%
\pgfpathlineto{\pgfqpoint{2.105030in}{0.960822in}}%
\pgfpathlineto{\pgfqpoint{2.106284in}{0.959321in}}%
\pgfpathlineto{\pgfqpoint{2.108165in}{0.957070in}}%
\pgfpathlineto{\pgfqpoint{2.109419in}{0.955569in}}%
\pgfpathlineto{\pgfqpoint{2.111299in}{0.953319in}}%
\pgfpathlineto{\pgfqpoint{2.114434in}{0.953319in}}%
\pgfpathlineto{\pgfqpoint{2.115688in}{0.951818in}}%
\pgfpathlineto{\pgfqpoint{2.117569in}{0.949567in}}%
\pgfpathlineto{\pgfqpoint{2.118823in}{0.948066in}}%
\pgfpathlineto{\pgfqpoint{2.120704in}{0.945815in}}%
\pgfpathlineto{\pgfqpoint{2.121958in}{0.944315in}}%
\pgfpathlineto{\pgfqpoint{2.123838in}{0.942064in}}%
\pgfpathlineto{\pgfqpoint{2.126973in}{0.942064in}}%
\pgfpathlineto{\pgfqpoint{2.128227in}{0.940563in}}%
\pgfpathlineto{\pgfqpoint{2.130108in}{0.938312in}}%
\pgfpathlineto{\pgfqpoint{2.131362in}{0.936812in}}%
\pgfpathlineto{\pgfqpoint{2.133243in}{0.934561in}}%
\pgfpathlineto{\pgfqpoint{2.134497in}{0.933060in}}%
\pgfpathlineto{\pgfqpoint{2.136377in}{0.930809in}}%
\pgfpathlineto{\pgfqpoint{2.139512in}{0.930809in}}%
\pgfpathlineto{\pgfqpoint{2.140766in}{0.929309in}}%
\pgfpathlineto{\pgfqpoint{2.142647in}{0.927058in}}%
\pgfpathlineto{\pgfqpoint{2.143901in}{0.925557in}}%
\pgfpathlineto{\pgfqpoint{2.145782in}{0.923306in}}%
\pgfpathlineto{\pgfqpoint{2.147036in}{0.921806in}}%
\pgfpathlineto{\pgfqpoint{2.148916in}{0.919555in}}%
\pgfpathlineto{\pgfqpoint{2.152051in}{0.919555in}}%
\pgfpathlineto{\pgfqpoint{2.153305in}{0.918054in}}%
\pgfpathlineto{\pgfqpoint{2.155186in}{0.915803in}}%
\pgfpathlineto{\pgfqpoint{2.156440in}{0.914303in}}%
\pgfpathlineto{\pgfqpoint{2.158321in}{0.912052in}}%
\pgfpathlineto{\pgfqpoint{2.159575in}{0.910551in}}%
\pgfpathlineto{\pgfqpoint{2.161455in}{0.908300in}}%
\pgfpathlineto{\pgfqpoint{2.162709in}{0.906799in}}%
\pgfpathlineto{\pgfqpoint{2.164590in}{0.904549in}}%
\pgfpathlineto{\pgfqpoint{2.167725in}{0.904549in}}%
\pgfpathlineto{\pgfqpoint{2.168979in}{0.903048in}}%
\pgfpathlineto{\pgfqpoint{2.170860in}{0.900797in}}%
\pgfpathlineto{\pgfqpoint{2.172113in}{0.899296in}}%
\pgfpathclose%
\pgfpathmoveto{\pgfqpoint{2.172584in}{0.899296in}}%
\pgfpathlineto{\pgfqpoint{2.170860in}{0.901360in}}%
\pgfpathlineto{\pgfqpoint{2.169449in}{0.903048in}}%
\pgfpathlineto{\pgfqpoint{2.167725in}{0.905111in}}%
\pgfpathlineto{\pgfqpoint{2.164590in}{0.905111in}}%
\pgfpathlineto{\pgfqpoint{2.163179in}{0.906799in}}%
\pgfpathlineto{\pgfqpoint{2.161455in}{0.908863in}}%
\pgfpathlineto{\pgfqpoint{2.160045in}{0.910551in}}%
\pgfpathlineto{\pgfqpoint{2.158321in}{0.912614in}}%
\pgfpathlineto{\pgfqpoint{2.156910in}{0.914303in}}%
\pgfpathlineto{\pgfqpoint{2.155186in}{0.916366in}}%
\pgfpathlineto{\pgfqpoint{2.153775in}{0.918054in}}%
\pgfpathlineto{\pgfqpoint{2.152051in}{0.920117in}}%
\pgfpathlineto{\pgfqpoint{2.148916in}{0.920117in}}%
\pgfpathlineto{\pgfqpoint{2.147506in}{0.921806in}}%
\pgfpathlineto{\pgfqpoint{2.145782in}{0.923869in}}%
\pgfpathlineto{\pgfqpoint{2.144371in}{0.925557in}}%
\pgfpathlineto{\pgfqpoint{2.142647in}{0.927621in}}%
\pgfpathlineto{\pgfqpoint{2.141236in}{0.929309in}}%
\pgfpathlineto{\pgfqpoint{2.139512in}{0.931372in}}%
\pgfpathlineto{\pgfqpoint{2.136377in}{0.931372in}}%
\pgfpathlineto{\pgfqpoint{2.134967in}{0.933060in}}%
\pgfpathlineto{\pgfqpoint{2.133243in}{0.935124in}}%
\pgfpathlineto{\pgfqpoint{2.131832in}{0.936812in}}%
\pgfpathlineto{\pgfqpoint{2.130108in}{0.938875in}}%
\pgfpathlineto{\pgfqpoint{2.128697in}{0.940563in}}%
\pgfpathlineto{\pgfqpoint{2.126973in}{0.942627in}}%
\pgfpathlineto{\pgfqpoint{2.123838in}{0.942627in}}%
\pgfpathlineto{\pgfqpoint{2.122428in}{0.944315in}}%
\pgfpathlineto{\pgfqpoint{2.120704in}{0.946378in}}%
\pgfpathlineto{\pgfqpoint{2.119293in}{0.948066in}}%
\pgfpathlineto{\pgfqpoint{2.117569in}{0.950130in}}%
\pgfpathlineto{\pgfqpoint{2.116158in}{0.951818in}}%
\pgfpathlineto{\pgfqpoint{2.114434in}{0.953881in}}%
\pgfpathlineto{\pgfqpoint{2.111299in}{0.953881in}}%
\pgfpathlineto{\pgfqpoint{2.109889in}{0.955569in}}%
\pgfpathlineto{\pgfqpoint{2.108165in}{0.957633in}}%
\pgfpathlineto{\pgfqpoint{2.106754in}{0.959321in}}%
\pgfpathlineto{\pgfqpoint{2.105030in}{0.961384in}}%
\pgfpathlineto{\pgfqpoint{2.103619in}{0.963073in}}%
\pgfpathlineto{\pgfqpoint{2.101895in}{0.965136in}}%
\pgfpathlineto{\pgfqpoint{2.098761in}{0.965136in}}%
\pgfpathlineto{\pgfqpoint{2.097350in}{0.966824in}}%
\pgfpathlineto{\pgfqpoint{2.095626in}{0.968887in}}%
\pgfpathlineto{\pgfqpoint{2.094215in}{0.970576in}}%
\pgfpathlineto{\pgfqpoint{2.092491in}{0.972639in}}%
\pgfpathlineto{\pgfqpoint{2.091080in}{0.974327in}}%
\pgfpathlineto{\pgfqpoint{2.089356in}{0.976390in}}%
\pgfpathlineto{\pgfqpoint{2.087946in}{0.978079in}}%
\pgfpathlineto{\pgfqpoint{2.086222in}{0.980142in}}%
\pgfpathlineto{\pgfqpoint{2.083087in}{0.980142in}}%
\pgfpathlineto{\pgfqpoint{2.081676in}{0.981830in}}%
\pgfpathlineto{\pgfqpoint{2.079952in}{0.983894in}}%
\pgfpathlineto{\pgfqpoint{2.078541in}{0.985582in}}%
\pgfpathlineto{\pgfqpoint{2.076817in}{0.987645in}}%
\pgfpathlineto{\pgfqpoint{2.075407in}{0.989333in}}%
\pgfpathlineto{\pgfqpoint{2.073683in}{0.991397in}}%
\pgfpathlineto{\pgfqpoint{2.070548in}{0.991397in}}%
\pgfpathlineto{\pgfqpoint{2.069137in}{0.993085in}}%
\pgfpathlineto{\pgfqpoint{2.067413in}{0.995148in}}%
\pgfpathlineto{\pgfqpoint{2.066002in}{0.996836in}}%
\pgfpathlineto{\pgfqpoint{2.064278in}{0.998900in}}%
\pgfpathlineto{\pgfqpoint{2.062868in}{1.000588in}}%
\pgfpathlineto{\pgfqpoint{2.061144in}{1.002651in}}%
\pgfpathlineto{\pgfqpoint{2.058009in}{1.002651in}}%
\pgfpathlineto{\pgfqpoint{2.056598in}{1.004339in}}%
\pgfpathlineto{\pgfqpoint{2.054874in}{1.006403in}}%
\pgfpathlineto{\pgfqpoint{2.053463in}{1.008091in}}%
\pgfpathlineto{\pgfqpoint{2.051739in}{1.010154in}}%
\pgfpathlineto{\pgfqpoint{2.050329in}{1.011842in}}%
\pgfpathlineto{\pgfqpoint{2.048605in}{1.013906in}}%
\pgfpathlineto{\pgfqpoint{2.045470in}{1.013906in}}%
\pgfpathlineto{\pgfqpoint{2.044059in}{1.015594in}}%
\pgfpathlineto{\pgfqpoint{2.042335in}{1.017657in}}%
\pgfpathlineto{\pgfqpoint{2.040925in}{1.019346in}}%
\pgfpathlineto{\pgfqpoint{2.039200in}{1.021409in}}%
\pgfpathlineto{\pgfqpoint{2.037790in}{1.023097in}}%
\pgfpathlineto{\pgfqpoint{2.036066in}{1.025160in}}%
\pgfpathlineto{\pgfqpoint{2.032931in}{1.025160in}}%
\pgfpathlineto{\pgfqpoint{2.031520in}{1.026849in}}%
\pgfpathlineto{\pgfqpoint{2.029796in}{1.028912in}}%
\pgfpathlineto{\pgfqpoint{2.028386in}{1.030600in}}%
\pgfpathlineto{\pgfqpoint{2.026661in}{1.032663in}}%
\pgfpathlineto{\pgfqpoint{2.025251in}{1.034352in}}%
\pgfpathlineto{\pgfqpoint{2.023527in}{1.036415in}}%
\pgfpathlineto{\pgfqpoint{2.020392in}{1.036415in}}%
\pgfpathlineto{\pgfqpoint{2.018981in}{1.038103in}}%
\pgfpathlineto{\pgfqpoint{2.017257in}{1.040167in}}%
\pgfpathlineto{\pgfqpoint{2.015847in}{1.041855in}}%
\pgfpathlineto{\pgfqpoint{2.014122in}{1.043918in}}%
\pgfpathlineto{\pgfqpoint{2.012712in}{1.045606in}}%
\pgfpathlineto{\pgfqpoint{2.010988in}{1.047670in}}%
\pgfpathlineto{\pgfqpoint{2.009577in}{1.049358in}}%
\pgfpathlineto{\pgfqpoint{2.007853in}{1.051421in}}%
\pgfpathlineto{\pgfqpoint{2.004718in}{1.051421in}}%
\pgfpathlineto{\pgfqpoint{2.003308in}{1.053109in}}%
\pgfpathlineto{\pgfqpoint{2.001584in}{1.055173in}}%
\pgfpathlineto{\pgfqpoint{2.000173in}{1.056861in}}%
\pgfpathlineto{\pgfqpoint{1.998449in}{1.058924in}}%
\pgfpathlineto{\pgfqpoint{1.997038in}{1.060612in}}%
\pgfpathlineto{\pgfqpoint{1.995314in}{1.062676in}}%
\pgfpathlineto{\pgfqpoint{1.992179in}{1.062676in}}%
\pgfpathlineto{\pgfqpoint{1.990769in}{1.064364in}}%
\pgfpathlineto{\pgfqpoint{1.989045in}{1.066427in}}%
\pgfpathlineto{\pgfqpoint{1.987634in}{1.068115in}}%
\pgfpathlineto{\pgfqpoint{1.985910in}{1.070179in}}%
\pgfpathlineto{\pgfqpoint{1.984499in}{1.071867in}}%
\pgfpathlineto{\pgfqpoint{1.982775in}{1.073930in}}%
\pgfpathlineto{\pgfqpoint{1.979640in}{1.073930in}}%
\pgfpathlineto{\pgfqpoint{1.978230in}{1.075619in}}%
\pgfpathlineto{\pgfqpoint{1.976506in}{1.077682in}}%
\pgfpathlineto{\pgfqpoint{1.975095in}{1.079370in}}%
\pgfpathlineto{\pgfqpoint{1.973371in}{1.081433in}}%
\pgfpathlineto{\pgfqpoint{1.971960in}{1.083122in}}%
\pgfpathlineto{\pgfqpoint{1.970236in}{1.085185in}}%
\pgfpathlineto{\pgfqpoint{1.967101in}{1.085185in}}%
\pgfpathlineto{\pgfqpoint{1.965691in}{1.086873in}}%
\pgfpathlineto{\pgfqpoint{1.963967in}{1.088936in}}%
\pgfpathlineto{\pgfqpoint{1.962556in}{1.090625in}}%
\pgfpathlineto{\pgfqpoint{1.960832in}{1.092688in}}%
\pgfpathlineto{\pgfqpoint{1.959421in}{1.094376in}}%
\pgfpathlineto{\pgfqpoint{1.957697in}{1.096440in}}%
\pgfpathlineto{\pgfqpoint{1.954562in}{1.096440in}}%
\pgfpathlineto{\pgfqpoint{1.953152in}{1.098128in}}%
\pgfpathlineto{\pgfqpoint{1.951428in}{1.100191in}}%
\pgfpathlineto{\pgfqpoint{1.950017in}{1.101879in}}%
\pgfpathlineto{\pgfqpoint{1.948293in}{1.103943in}}%
\pgfpathlineto{\pgfqpoint{1.946882in}{1.105631in}}%
\pgfpathlineto{\pgfqpoint{1.945158in}{1.107694in}}%
\pgfpathlineto{\pgfqpoint{1.942023in}{1.107694in}}%
\pgfpathlineto{\pgfqpoint{1.940613in}{1.109382in}}%
\pgfpathlineto{\pgfqpoint{1.938889in}{1.111446in}}%
\pgfpathlineto{\pgfqpoint{1.937478in}{1.113134in}}%
\pgfpathlineto{\pgfqpoint{1.935754in}{1.115197in}}%
\pgfpathlineto{\pgfqpoint{1.934343in}{1.116885in}}%
\pgfpathlineto{\pgfqpoint{1.932619in}{1.118949in}}%
\pgfpathlineto{\pgfqpoint{1.931209in}{1.120637in}}%
\pgfpathlineto{\pgfqpoint{1.929484in}{1.122700in}}%
\pgfpathlineto{\pgfqpoint{1.926350in}{1.122700in}}%
\pgfpathlineto{\pgfqpoint{1.924939in}{1.124388in}}%
\pgfpathlineto{\pgfqpoint{1.923215in}{1.126452in}}%
\pgfpathlineto{\pgfqpoint{1.921804in}{1.128140in}}%
\pgfpathlineto{\pgfqpoint{1.920080in}{1.130203in}}%
\pgfpathlineto{\pgfqpoint{1.918670in}{1.131892in}}%
\pgfpathlineto{\pgfqpoint{1.916945in}{1.133955in}}%
\pgfpathlineto{\pgfqpoint{1.913811in}{1.133955in}}%
\pgfpathlineto{\pgfqpoint{1.912400in}{1.135643in}}%
\pgfpathlineto{\pgfqpoint{1.910676in}{1.137706in}}%
\pgfpathlineto{\pgfqpoint{1.909265in}{1.139395in}}%
\pgfpathlineto{\pgfqpoint{1.907541in}{1.141458in}}%
\pgfpathlineto{\pgfqpoint{1.906131in}{1.143146in}}%
\pgfpathlineto{\pgfqpoint{1.904407in}{1.145210in}}%
\pgfpathlineto{\pgfqpoint{1.901272in}{1.145210in}}%
\pgfpathlineto{\pgfqpoint{1.899861in}{1.146898in}}%
\pgfpathlineto{\pgfqpoint{1.898137in}{1.148961in}}%
\pgfpathlineto{\pgfqpoint{1.896726in}{1.150649in}}%
\pgfpathlineto{\pgfqpoint{1.895002in}{1.152713in}}%
\pgfpathlineto{\pgfqpoint{1.893592in}{1.154401in}}%
\pgfpathlineto{\pgfqpoint{1.891868in}{1.156464in}}%
\pgfpathlineto{\pgfqpoint{1.888733in}{1.156464in}}%
\pgfpathlineto{\pgfqpoint{1.887322in}{1.158152in}}%
\pgfpathlineto{\pgfqpoint{1.885598in}{1.160216in}}%
\pgfpathlineto{\pgfqpoint{1.884187in}{1.161904in}}%
\pgfpathlineto{\pgfqpoint{1.882463in}{1.163967in}}%
\pgfpathlineto{\pgfqpoint{1.881053in}{1.165655in}}%
\pgfpathlineto{\pgfqpoint{1.879329in}{1.167719in}}%
\pgfpathlineto{\pgfqpoint{1.876194in}{1.167719in}}%
\pgfpathlineto{\pgfqpoint{1.874783in}{1.169407in}}%
\pgfpathlineto{\pgfqpoint{1.873059in}{1.171470in}}%
\pgfpathlineto{\pgfqpoint{1.871648in}{1.173158in}}%
\pgfpathlineto{\pgfqpoint{1.869924in}{1.175222in}}%
\pgfpathlineto{\pgfqpoint{1.868514in}{1.176910in}}%
\pgfpathlineto{\pgfqpoint{1.866790in}{1.178973in}}%
\pgfpathlineto{\pgfqpoint{1.863655in}{1.178973in}}%
\pgfpathlineto{\pgfqpoint{1.862244in}{1.180662in}}%
\pgfpathlineto{\pgfqpoint{1.860520in}{1.182725in}}%
\pgfpathlineto{\pgfqpoint{1.859109in}{1.184413in}}%
\pgfpathlineto{\pgfqpoint{1.857385in}{1.186476in}}%
\pgfpathlineto{\pgfqpoint{1.855975in}{1.188165in}}%
\pgfpathlineto{\pgfqpoint{1.854251in}{1.190228in}}%
\pgfpathlineto{\pgfqpoint{1.852840in}{1.191916in}}%
\pgfpathlineto{\pgfqpoint{1.851116in}{1.193979in}}%
\pgfpathlineto{\pgfqpoint{1.847981in}{1.193979in}}%
\pgfpathlineto{\pgfqpoint{1.846571in}{1.195668in}}%
\pgfpathlineto{\pgfqpoint{1.844846in}{1.197731in}}%
\pgfpathlineto{\pgfqpoint{1.843436in}{1.199419in}}%
\pgfpathlineto{\pgfqpoint{1.841712in}{1.201483in}}%
\pgfpathlineto{\pgfqpoint{1.840301in}{1.203171in}}%
\pgfpathlineto{\pgfqpoint{1.838577in}{1.205234in}}%
\pgfpathlineto{\pgfqpoint{1.835442in}{1.205234in}}%
\pgfpathlineto{\pgfqpoint{1.834032in}{1.206922in}}%
\pgfpathlineto{\pgfqpoint{1.832307in}{1.208986in}}%
\pgfpathlineto{\pgfqpoint{1.830897in}{1.210674in}}%
\pgfpathlineto{\pgfqpoint{1.829173in}{1.212737in}}%
\pgfpathlineto{\pgfqpoint{1.827762in}{1.214425in}}%
\pgfpathlineto{\pgfqpoint{1.826038in}{1.216489in}}%
\pgfpathlineto{\pgfqpoint{1.822903in}{1.216489in}}%
\pgfpathlineto{\pgfqpoint{1.821493in}{1.218177in}}%
\pgfpathlineto{\pgfqpoint{1.819768in}{1.220240in}}%
\pgfpathlineto{\pgfqpoint{1.818358in}{1.221928in}}%
\pgfpathlineto{\pgfqpoint{1.816634in}{1.223992in}}%
\pgfpathlineto{\pgfqpoint{1.815223in}{1.225680in}}%
\pgfpathlineto{\pgfqpoint{1.813499in}{1.227743in}}%
\pgfpathlineto{\pgfqpoint{1.810364in}{1.227743in}}%
\pgfpathlineto{\pgfqpoint{1.808954in}{1.229431in}}%
\pgfpathlineto{\pgfqpoint{1.807229in}{1.231495in}}%
\pgfpathlineto{\pgfqpoint{1.805819in}{1.233183in}}%
\pgfpathlineto{\pgfqpoint{1.804095in}{1.235246in}}%
\pgfpathlineto{\pgfqpoint{1.802684in}{1.236935in}}%
\pgfpathlineto{\pgfqpoint{1.800960in}{1.238998in}}%
\pgfpathlineto{\pgfqpoint{1.797825in}{1.238998in}}%
\pgfpathlineto{\pgfqpoint{1.796415in}{1.240686in}}%
\pgfpathlineto{\pgfqpoint{1.794691in}{1.242749in}}%
\pgfpathlineto{\pgfqpoint{1.793280in}{1.244438in}}%
\pgfpathlineto{\pgfqpoint{1.791556in}{1.246501in}}%
\pgfpathlineto{\pgfqpoint{1.790145in}{1.248189in}}%
\pgfpathlineto{\pgfqpoint{1.788421in}{1.250252in}}%
\pgfpathlineto{\pgfqpoint{1.787010in}{1.251941in}}%
\pgfpathlineto{\pgfqpoint{1.785286in}{1.254004in}}%
\pgfpathlineto{\pgfqpoint{1.782152in}{1.254004in}}%
\pgfpathlineto{\pgfqpoint{1.780741in}{1.255692in}}%
\pgfpathlineto{\pgfqpoint{1.779017in}{1.257756in}}%
\pgfpathlineto{\pgfqpoint{1.777606in}{1.259444in}}%
\pgfpathlineto{\pgfqpoint{1.775882in}{1.261507in}}%
\pgfpathlineto{\pgfqpoint{1.774471in}{1.263195in}}%
\pgfpathlineto{\pgfqpoint{1.772747in}{1.265259in}}%
\pgfpathlineto{\pgfqpoint{1.769613in}{1.265259in}}%
\pgfpathlineto{\pgfqpoint{1.768202in}{1.266947in}}%
\pgfpathlineto{\pgfqpoint{1.766478in}{1.269010in}}%
\pgfpathlineto{\pgfqpoint{1.765067in}{1.270698in}}%
\pgfpathlineto{\pgfqpoint{1.763343in}{1.272762in}}%
\pgfpathlineto{\pgfqpoint{1.761932in}{1.274450in}}%
\pgfpathlineto{\pgfqpoint{1.760208in}{1.276513in}}%
\pgfpathlineto{\pgfqpoint{1.757074in}{1.276513in}}%
\pgfpathlineto{\pgfqpoint{1.755663in}{1.278201in}}%
\pgfpathlineto{\pgfqpoint{1.753939in}{1.280265in}}%
\pgfpathlineto{\pgfqpoint{1.752528in}{1.281953in}}%
\pgfpathlineto{\pgfqpoint{1.750804in}{1.284016in}}%
\pgfpathlineto{\pgfqpoint{1.749393in}{1.285704in}}%
\pgfpathlineto{\pgfqpoint{1.747669in}{1.287768in}}%
\pgfpathlineto{\pgfqpoint{1.744535in}{1.287768in}}%
\pgfpathlineto{\pgfqpoint{1.743124in}{1.289456in}}%
\pgfpathlineto{\pgfqpoint{1.741400in}{1.291519in}}%
\pgfpathlineto{\pgfqpoint{1.739989in}{1.293208in}}%
\pgfpathlineto{\pgfqpoint{1.738265in}{1.295271in}}%
\pgfpathlineto{\pgfqpoint{1.736855in}{1.296959in}}%
\pgfpathlineto{\pgfqpoint{1.735130in}{1.299022in}}%
\pgfpathlineto{\pgfqpoint{1.731996in}{1.299022in}}%
\pgfpathlineto{\pgfqpoint{1.730585in}{1.300711in}}%
\pgfpathlineto{\pgfqpoint{1.728861in}{1.302774in}}%
\pgfpathlineto{\pgfqpoint{1.727450in}{1.304462in}}%
\pgfpathlineto{\pgfqpoint{1.725726in}{1.306525in}}%
\pgfpathlineto{\pgfqpoint{1.724316in}{1.308214in}}%
\pgfpathlineto{\pgfqpoint{1.722591in}{1.310277in}}%
\pgfpathlineto{\pgfqpoint{1.719457in}{1.310277in}}%
\pgfpathlineto{\pgfqpoint{1.718046in}{1.311965in}}%
\pgfpathlineto{\pgfqpoint{1.716322in}{1.314029in}}%
\pgfpathlineto{\pgfqpoint{1.714911in}{1.315717in}}%
\pgfpathlineto{\pgfqpoint{1.713187in}{1.317780in}}%
\pgfpathlineto{\pgfqpoint{1.711777in}{1.319468in}}%
\pgfpathlineto{\pgfqpoint{1.710052in}{1.321532in}}%
\pgfpathlineto{\pgfqpoint{1.708642in}{1.323220in}}%
\pgfpathlineto{\pgfqpoint{1.706918in}{1.325283in}}%
\pgfpathlineto{\pgfqpoint{1.703783in}{1.325283in}}%
\pgfpathlineto{\pgfqpoint{1.702372in}{1.326971in}}%
\pgfpathlineto{\pgfqpoint{1.700648in}{1.329035in}}%
\pgfpathlineto{\pgfqpoint{1.699238in}{1.330723in}}%
\pgfpathlineto{\pgfqpoint{1.697514in}{1.332786in}}%
\pgfpathlineto{\pgfqpoint{1.696103in}{1.334474in}}%
\pgfpathlineto{\pgfqpoint{1.694379in}{1.336538in}}%
\pgfpathlineto{\pgfqpoint{1.691244in}{1.336538in}}%
\pgfpathlineto{\pgfqpoint{1.689833in}{1.338226in}}%
\pgfpathlineto{\pgfqpoint{1.688109in}{1.340289in}}%
\pgfpathlineto{\pgfqpoint{1.686699in}{1.341977in}}%
\pgfpathlineto{\pgfqpoint{1.684975in}{1.344041in}}%
\pgfpathlineto{\pgfqpoint{1.683564in}{1.345729in}}%
\pgfpathlineto{\pgfqpoint{1.681840in}{1.347792in}}%
\pgfpathlineto{\pgfqpoint{1.678705in}{1.347792in}}%
\pgfpathlineto{\pgfqpoint{1.677294in}{1.349481in}}%
\pgfpathlineto{\pgfqpoint{1.675570in}{1.351544in}}%
\pgfpathlineto{\pgfqpoint{1.674160in}{1.353232in}}%
\pgfpathlineto{\pgfqpoint{1.672436in}{1.355295in}}%
\pgfpathlineto{\pgfqpoint{1.671025in}{1.356984in}}%
\pgfpathlineto{\pgfqpoint{1.669301in}{1.359047in}}%
\pgfpathlineto{\pgfqpoint{1.666166in}{1.359047in}}%
\pgfpathlineto{\pgfqpoint{1.664755in}{1.360735in}}%
\pgfpathlineto{\pgfqpoint{1.663031in}{1.362799in}}%
\pgfpathlineto{\pgfqpoint{1.661621in}{1.364487in}}%
\pgfpathlineto{\pgfqpoint{1.659897in}{1.366550in}}%
\pgfpathlineto{\pgfqpoint{1.658486in}{1.368238in}}%
\pgfpathlineto{\pgfqpoint{1.656762in}{1.370302in}}%
\pgfpathlineto{\pgfqpoint{1.653627in}{1.370302in}}%
\pgfpathlineto{\pgfqpoint{1.652216in}{1.371990in}}%
\pgfpathlineto{\pgfqpoint{1.650492in}{1.374053in}}%
\pgfpathlineto{\pgfqpoint{1.649082in}{1.375741in}}%
\pgfpathlineto{\pgfqpoint{1.647358in}{1.377805in}}%
\pgfpathlineto{\pgfqpoint{1.645947in}{1.379493in}}%
\pgfpathlineto{\pgfqpoint{1.644223in}{1.381556in}}%
\pgfpathlineto{\pgfqpoint{1.641088in}{1.381556in}}%
\pgfpathlineto{\pgfqpoint{1.639678in}{1.383244in}}%
\pgfpathlineto{\pgfqpoint{1.637953in}{1.385308in}}%
\pgfpathlineto{\pgfqpoint{1.636543in}{1.386996in}}%
\pgfpathlineto{\pgfqpoint{1.634819in}{1.389059in}}%
\pgfpathlineto{\pgfqpoint{1.633408in}{1.390747in}}%
\pgfpathlineto{\pgfqpoint{1.631684in}{1.392811in}}%
\pgfpathlineto{\pgfqpoint{1.630273in}{1.394499in}}%
\pgfpathlineto{\pgfqpoint{1.628549in}{1.396562in}}%
\pgfpathlineto{\pgfqpoint{1.625414in}{1.396562in}}%
\pgfpathlineto{\pgfqpoint{1.624004in}{1.398251in}}%
\pgfpathlineto{\pgfqpoint{1.622280in}{1.400314in}}%
\pgfpathlineto{\pgfqpoint{1.620869in}{1.402002in}}%
\pgfpathlineto{\pgfqpoint{1.619145in}{1.404065in}}%
\pgfpathlineto{\pgfqpoint{1.617734in}{1.405754in}}%
\pgfpathlineto{\pgfqpoint{1.616010in}{1.407817in}}%
\pgfpathlineto{\pgfqpoint{1.612875in}{1.407817in}}%
\pgfpathlineto{\pgfqpoint{1.611465in}{1.409505in}}%
\pgfpathlineto{\pgfqpoint{1.609741in}{1.411568in}}%
\pgfpathlineto{\pgfqpoint{1.608330in}{1.413257in}}%
\pgfpathlineto{\pgfqpoint{1.606606in}{1.415320in}}%
\pgfpathlineto{\pgfqpoint{1.605195in}{1.417008in}}%
\pgfpathlineto{\pgfqpoint{1.603471in}{1.419072in}}%
\pgfpathlineto{\pgfqpoint{1.600337in}{1.419072in}}%
\pgfpathlineto{\pgfqpoint{1.598926in}{1.420760in}}%
\pgfpathlineto{\pgfqpoint{1.597202in}{1.422823in}}%
\pgfpathlineto{\pgfqpoint{1.595791in}{1.424511in}}%
\pgfpathlineto{\pgfqpoint{1.594067in}{1.426575in}}%
\pgfpathlineto{\pgfqpoint{1.592656in}{1.428263in}}%
\pgfpathlineto{\pgfqpoint{1.590932in}{1.430326in}}%
\pgfpathlineto{\pgfqpoint{1.587798in}{1.430326in}}%
\pgfpathlineto{\pgfqpoint{1.586387in}{1.432014in}}%
\pgfpathlineto{\pgfqpoint{1.584663in}{1.434078in}}%
\pgfpathlineto{\pgfqpoint{1.583252in}{1.435766in}}%
\pgfpathlineto{\pgfqpoint{1.581528in}{1.437829in}}%
\pgfpathlineto{\pgfqpoint{1.580117in}{1.439517in}}%
\pgfpathlineto{\pgfqpoint{1.578393in}{1.441581in}}%
\pgfpathlineto{\pgfqpoint{1.575259in}{1.441581in}}%
\pgfpathlineto{\pgfqpoint{1.573848in}{1.443269in}}%
\pgfpathlineto{\pgfqpoint{1.572124in}{1.445332in}}%
\pgfpathlineto{\pgfqpoint{1.570713in}{1.447020in}}%
\pgfpathlineto{\pgfqpoint{1.568989in}{1.449084in}}%
\pgfpathlineto{\pgfqpoint{1.567578in}{1.450772in}}%
\pgfpathlineto{\pgfqpoint{1.565854in}{1.452835in}}%
\pgfpathlineto{\pgfqpoint{1.562720in}{1.452835in}}%
\pgfpathlineto{\pgfqpoint{1.561309in}{1.454524in}}%
\pgfpathlineto{\pgfqpoint{1.559585in}{1.456587in}}%
\pgfpathlineto{\pgfqpoint{1.558174in}{1.458275in}}%
\pgfpathlineto{\pgfqpoint{1.556450in}{1.460338in}}%
\pgfpathlineto{\pgfqpoint{1.555039in}{1.462027in}}%
\pgfpathlineto{\pgfqpoint{1.553315in}{1.464090in}}%
\pgfpathlineto{\pgfqpoint{1.551905in}{1.465778in}}%
\pgfpathlineto{\pgfqpoint{1.550181in}{1.467841in}}%
\pgfpathlineto{\pgfqpoint{1.547046in}{1.467841in}}%
\pgfpathlineto{\pgfqpoint{1.545635in}{1.469530in}}%
\pgfpathlineto{\pgfqpoint{1.543911in}{1.471593in}}%
\pgfpathlineto{\pgfqpoint{1.542501in}{1.473281in}}%
\pgfpathlineto{\pgfqpoint{1.540776in}{1.475345in}}%
\pgfpathlineto{\pgfqpoint{1.539366in}{1.477033in}}%
\pgfpathlineto{\pgfqpoint{1.537642in}{1.479096in}}%
\pgfpathlineto{\pgfqpoint{1.534507in}{1.479096in}}%
\pgfpathlineto{\pgfqpoint{1.533096in}{1.480784in}}%
\pgfpathlineto{\pgfqpoint{1.531372in}{1.482848in}}%
\pgfpathlineto{\pgfqpoint{1.529962in}{1.484536in}}%
\pgfpathlineto{\pgfqpoint{1.528237in}{1.486599in}}%
\pgfpathlineto{\pgfqpoint{1.526827in}{1.488287in}}%
\pgfpathlineto{\pgfqpoint{1.525103in}{1.490351in}}%
\pgfpathlineto{\pgfqpoint{1.521968in}{1.490351in}}%
\pgfpathlineto{\pgfqpoint{1.520557in}{1.492039in}}%
\pgfpathlineto{\pgfqpoint{1.518833in}{1.494102in}}%
\pgfpathlineto{\pgfqpoint{1.517423in}{1.495790in}}%
\pgfpathlineto{\pgfqpoint{1.515698in}{1.497854in}}%
\pgfpathlineto{\pgfqpoint{1.514288in}{1.499542in}}%
\pgfpathlineto{\pgfqpoint{1.512564in}{1.501605in}}%
\pgfpathlineto{\pgfqpoint{1.509429in}{1.501605in}}%
\pgfpathlineto{\pgfqpoint{1.508018in}{1.503293in}}%
\pgfpathlineto{\pgfqpoint{1.506294in}{1.505357in}}%
\pgfpathlineto{\pgfqpoint{1.504884in}{1.507045in}}%
\pgfpathlineto{\pgfqpoint{1.503159in}{1.509108in}}%
\pgfpathlineto{\pgfqpoint{1.501749in}{1.510797in}}%
\pgfpathlineto{\pgfqpoint{1.500025in}{1.512860in}}%
\pgfpathlineto{\pgfqpoint{1.496890in}{1.512860in}}%
\pgfpathlineto{\pgfqpoint{1.495479in}{1.514548in}}%
\pgfpathlineto{\pgfqpoint{1.493755in}{1.516611in}}%
\pgfpathlineto{\pgfqpoint{1.492345in}{1.518300in}}%
\pgfpathlineto{\pgfqpoint{1.490621in}{1.520363in}}%
\pgfpathlineto{\pgfqpoint{1.489210in}{1.522051in}}%
\pgfpathlineto{\pgfqpoint{1.487486in}{1.524114in}}%
\pgfpathlineto{\pgfqpoint{1.486075in}{1.525803in}}%
\pgfpathlineto{\pgfqpoint{1.484351in}{1.527866in}}%
\pgfpathlineto{\pgfqpoint{1.481216in}{1.527866in}}%
\pgfpathlineto{\pgfqpoint{1.479806in}{1.529554in}}%
\pgfpathlineto{\pgfqpoint{1.478082in}{1.531618in}}%
\pgfpathlineto{\pgfqpoint{1.476671in}{1.533306in}}%
\pgfpathlineto{\pgfqpoint{1.474947in}{1.535369in}}%
\pgfpathlineto{\pgfqpoint{1.473536in}{1.537057in}}%
\pgfpathlineto{\pgfqpoint{1.471812in}{1.539121in}}%
\pgfpathlineto{\pgfqpoint{1.468677in}{1.539121in}}%
\pgfpathlineto{\pgfqpoint{1.467267in}{1.540809in}}%
\pgfpathlineto{\pgfqpoint{1.465543in}{1.542872in}}%
\pgfpathlineto{\pgfqpoint{1.464132in}{1.544560in}}%
\pgfpathlineto{\pgfqpoint{1.462408in}{1.546624in}}%
\pgfpathlineto{\pgfqpoint{1.460997in}{1.548312in}}%
\pgfpathlineto{\pgfqpoint{1.459273in}{1.550375in}}%
\pgfpathlineto{\pgfqpoint{1.456138in}{1.550375in}}%
\pgfpathlineto{\pgfqpoint{1.454728in}{1.552063in}}%
\pgfpathlineto{\pgfqpoint{1.453004in}{1.554127in}}%
\pgfpathlineto{\pgfqpoint{1.451593in}{1.555815in}}%
\pgfpathlineto{\pgfqpoint{1.449869in}{1.557878in}}%
\pgfpathlineto{\pgfqpoint{1.448458in}{1.559566in}}%
\pgfpathlineto{\pgfqpoint{1.446734in}{1.561630in}}%
\pgfpathlineto{\pgfqpoint{1.443599in}{1.561630in}}%
\pgfpathlineto{\pgfqpoint{1.442189in}{1.563318in}}%
\pgfpathlineto{\pgfqpoint{1.440465in}{1.565381in}}%
\pgfpathlineto{\pgfqpoint{1.439054in}{1.567070in}}%
\pgfpathlineto{\pgfqpoint{1.437330in}{1.569133in}}%
\pgfpathlineto{\pgfqpoint{1.435919in}{1.570821in}}%
\pgfpathlineto{\pgfqpoint{1.434195in}{1.572884in}}%
\pgfpathlineto{\pgfqpoint{1.431060in}{1.572884in}}%
\pgfpathlineto{\pgfqpoint{1.429650in}{1.574573in}}%
\pgfpathlineto{\pgfqpoint{1.427926in}{1.576636in}}%
\pgfpathlineto{\pgfqpoint{1.426515in}{1.578324in}}%
\pgfpathlineto{\pgfqpoint{1.424791in}{1.580388in}}%
\pgfpathlineto{\pgfqpoint{1.423380in}{1.582076in}}%
\pgfpathlineto{\pgfqpoint{1.421656in}{1.584139in}}%
\pgfpathlineto{\pgfqpoint{1.418521in}{1.584139in}}%
\pgfpathlineto{\pgfqpoint{1.417111in}{1.585827in}}%
\pgfpathlineto{\pgfqpoint{1.415387in}{1.587891in}}%
\pgfpathlineto{\pgfqpoint{1.413976in}{1.589579in}}%
\pgfpathlineto{\pgfqpoint{1.412252in}{1.591642in}}%
\pgfpathlineto{\pgfqpoint{1.410841in}{1.593330in}}%
\pgfpathlineto{\pgfqpoint{1.409117in}{1.595394in}}%
\pgfpathlineto{\pgfqpoint{1.407707in}{1.597082in}}%
\pgfpathlineto{\pgfqpoint{1.405982in}{1.599145in}}%
\pgfpathlineto{\pgfqpoint{1.402848in}{1.599145in}}%
\pgfpathlineto{\pgfqpoint{1.401437in}{1.600833in}}%
\pgfpathlineto{\pgfqpoint{1.399713in}{1.602897in}}%
\pgfpathlineto{\pgfqpoint{1.398302in}{1.604585in}}%
\pgfpathlineto{\pgfqpoint{1.396578in}{1.606648in}}%
\pgfpathlineto{\pgfqpoint{1.395168in}{1.608336in}}%
\pgfpathlineto{\pgfqpoint{1.393444in}{1.610400in}}%
\pgfpathlineto{\pgfqpoint{1.390309in}{1.610400in}}%
\pgfpathlineto{\pgfqpoint{1.388898in}{1.612088in}}%
\pgfpathlineto{\pgfqpoint{1.387174in}{1.614151in}}%
\pgfpathlineto{\pgfqpoint{1.385763in}{1.615840in}}%
\pgfpathlineto{\pgfqpoint{1.384039in}{1.617903in}}%
\pgfpathlineto{\pgfqpoint{1.382629in}{1.619591in}}%
\pgfpathlineto{\pgfqpoint{1.380905in}{1.621654in}}%
\pgfpathlineto{\pgfqpoint{1.377770in}{1.621654in}}%
\pgfpathlineto{\pgfqpoint{1.376359in}{1.623343in}}%
\pgfpathlineto{\pgfqpoint{1.374635in}{1.625406in}}%
\pgfpathlineto{\pgfqpoint{1.373224in}{1.627094in}}%
\pgfpathlineto{\pgfqpoint{1.371500in}{1.629157in}}%
\pgfpathlineto{\pgfqpoint{1.370090in}{1.630846in}}%
\pgfpathlineto{\pgfqpoint{1.368366in}{1.632909in}}%
\pgfpathlineto{\pgfqpoint{1.365231in}{1.632909in}}%
\pgfpathlineto{\pgfqpoint{1.363820in}{1.634597in}}%
\pgfpathlineto{\pgfqpoint{1.362096in}{1.636661in}}%
\pgfpathlineto{\pgfqpoint{1.360685in}{1.638349in}}%
\pgfpathlineto{\pgfqpoint{1.358961in}{1.640412in}}%
\pgfpathlineto{\pgfqpoint{1.357551in}{1.642100in}}%
\pgfpathlineto{\pgfqpoint{1.355827in}{1.644164in}}%
\pgfpathlineto{\pgfqpoint{1.352692in}{1.644164in}}%
\pgfpathlineto{\pgfqpoint{1.351281in}{1.645852in}}%
\pgfpathlineto{\pgfqpoint{1.349557in}{1.647915in}}%
\pgfpathlineto{\pgfqpoint{1.348146in}{1.649603in}}%
\pgfpathlineto{\pgfqpoint{1.346422in}{1.651667in}}%
\pgfpathlineto{\pgfqpoint{1.345012in}{1.653355in}}%
\pgfpathlineto{\pgfqpoint{1.343288in}{1.655418in}}%
\pgfpathlineto{\pgfqpoint{1.340153in}{1.655418in}}%
\pgfpathlineto{\pgfqpoint{1.338742in}{1.657106in}}%
\pgfpathlineto{\pgfqpoint{1.337018in}{1.659170in}}%
\pgfpathlineto{\pgfqpoint{1.335608in}{1.660858in}}%
\pgfpathlineto{\pgfqpoint{1.333883in}{1.662921in}}%
\pgfpathlineto{\pgfqpoint{1.332473in}{1.664609in}}%
\pgfpathlineto{\pgfqpoint{1.330749in}{1.666673in}}%
\pgfpathlineto{\pgfqpoint{1.329338in}{1.668361in}}%
\pgfpathlineto{\pgfqpoint{1.327614in}{1.670424in}}%
\pgfpathlineto{\pgfqpoint{1.324479in}{1.670424in}}%
\pgfpathlineto{\pgfqpoint{1.323069in}{1.672113in}}%
\pgfpathlineto{\pgfqpoint{1.321344in}{1.674176in}}%
\pgfpathlineto{\pgfqpoint{1.319934in}{1.675864in}}%
\pgfpathlineto{\pgfqpoint{1.318210in}{1.677927in}}%
\pgfpathlineto{\pgfqpoint{1.316799in}{1.679616in}}%
\pgfpathlineto{\pgfqpoint{1.315075in}{1.681679in}}%
\pgfpathlineto{\pgfqpoint{1.311940in}{1.681679in}}%
\pgfpathlineto{\pgfqpoint{1.310530in}{1.683367in}}%
\pgfpathlineto{\pgfqpoint{1.308805in}{1.685430in}}%
\pgfpathlineto{\pgfqpoint{1.307395in}{1.687119in}}%
\pgfpathlineto{\pgfqpoint{1.305671in}{1.689182in}}%
\pgfpathlineto{\pgfqpoint{1.304260in}{1.690870in}}%
\pgfpathlineto{\pgfqpoint{1.302536in}{1.692934in}}%
\pgfpathlineto{\pgfqpoint{1.299401in}{1.692934in}}%
\pgfpathlineto{\pgfqpoint{1.297991in}{1.694622in}}%
\pgfpathlineto{\pgfqpoint{1.296266in}{1.696685in}}%
\pgfpathlineto{\pgfqpoint{1.294856in}{1.698373in}}%
\pgfpathlineto{\pgfqpoint{1.293132in}{1.700437in}}%
\pgfpathlineto{\pgfqpoint{1.291721in}{1.702125in}}%
\pgfpathlineto{\pgfqpoint{1.291721in}{1.705876in}}%
\pgfpathlineto{\pgfqpoint{1.293132in}{1.707565in}}%
\pgfpathlineto{\pgfqpoint{1.294856in}{1.709628in}}%
\pgfpathlineto{\pgfqpoint{1.294856in}{1.713379in}}%
\pgfpathlineto{\pgfqpoint{1.294856in}{1.717131in}}%
\pgfpathlineto{\pgfqpoint{1.294856in}{1.720882in}}%
\pgfpathlineto{\pgfqpoint{1.296266in}{1.722571in}}%
\pgfpathlineto{\pgfqpoint{1.297991in}{1.724634in}}%
\pgfpathlineto{\pgfqpoint{1.297991in}{1.728386in}}%
\pgfpathlineto{\pgfqpoint{1.297991in}{1.732137in}}%
\pgfpathlineto{\pgfqpoint{1.299401in}{1.733825in}}%
\pgfpathlineto{\pgfqpoint{1.301125in}{1.735889in}}%
\pgfpathlineto{\pgfqpoint{1.301125in}{1.739640in}}%
\pgfpathlineto{\pgfqpoint{1.301125in}{1.743392in}}%
\pgfpathlineto{\pgfqpoint{1.301125in}{1.747143in}}%
\pgfpathlineto{\pgfqpoint{1.302536in}{1.748831in}}%
\pgfpathlineto{\pgfqpoint{1.304260in}{1.750895in}}%
\pgfpathlineto{\pgfqpoint{1.304260in}{1.754646in}}%
\pgfpathlineto{\pgfqpoint{1.304260in}{1.758398in}}%
\pgfpathlineto{\pgfqpoint{1.305671in}{1.760086in}}%
\pgfpathlineto{\pgfqpoint{1.307395in}{1.762149in}}%
\pgfpathlineto{\pgfqpoint{1.307395in}{1.765901in}}%
\pgfpathlineto{\pgfqpoint{1.307395in}{1.769652in}}%
\pgfpathlineto{\pgfqpoint{1.307395in}{1.773404in}}%
\pgfpathlineto{\pgfqpoint{1.308805in}{1.775092in}}%
\pgfpathlineto{\pgfqpoint{1.310530in}{1.777155in}}%
\pgfpathlineto{\pgfqpoint{1.310530in}{1.780907in}}%
\pgfpathlineto{\pgfqpoint{1.310530in}{1.784659in}}%
\pgfpathlineto{\pgfqpoint{1.311940in}{1.786347in}}%
\pgfpathlineto{\pgfqpoint{1.313664in}{1.788410in}}%
\pgfpathlineto{\pgfqpoint{1.313664in}{1.792162in}}%
\pgfpathlineto{\pgfqpoint{1.313664in}{1.795913in}}%
\pgfpathlineto{\pgfqpoint{1.313664in}{1.799665in}}%
\pgfpathlineto{\pgfqpoint{1.315075in}{1.801353in}}%
\pgfpathlineto{\pgfqpoint{1.316799in}{1.803416in}}%
\pgfpathlineto{\pgfqpoint{1.316799in}{1.807168in}}%
\pgfpathlineto{\pgfqpoint{1.316799in}{1.810919in}}%
\pgfpathlineto{\pgfqpoint{1.318210in}{1.812608in}}%
\pgfpathlineto{\pgfqpoint{1.319934in}{1.814671in}}%
\pgfpathlineto{\pgfqpoint{1.319934in}{1.818422in}}%
\pgfpathlineto{\pgfqpoint{1.319934in}{1.822174in}}%
\pgfpathlineto{\pgfqpoint{1.321344in}{1.823862in}}%
\pgfpathlineto{\pgfqpoint{1.323069in}{1.825925in}}%
\pgfpathlineto{\pgfqpoint{1.323069in}{1.829677in}}%
\pgfpathlineto{\pgfqpoint{1.323069in}{1.833429in}}%
\pgfpathlineto{\pgfqpoint{1.323069in}{1.837180in}}%
\pgfpathlineto{\pgfqpoint{1.324479in}{1.838868in}}%
\pgfpathlineto{\pgfqpoint{1.326203in}{1.840932in}}%
\pgfpathlineto{\pgfqpoint{1.326203in}{1.844683in}}%
\pgfpathlineto{\pgfqpoint{1.326203in}{1.848435in}}%
\pgfpathlineto{\pgfqpoint{1.327614in}{1.850123in}}%
\pgfpathlineto{\pgfqpoint{1.329338in}{1.852186in}}%
\pgfpathlineto{\pgfqpoint{1.329338in}{1.855938in}}%
\pgfpathlineto{\pgfqpoint{1.329338in}{1.859689in}}%
\pgfpathlineto{\pgfqpoint{1.329338in}{1.863441in}}%
\pgfpathlineto{\pgfqpoint{1.330749in}{1.865129in}}%
\pgfpathlineto{\pgfqpoint{1.332473in}{1.867192in}}%
\pgfpathlineto{\pgfqpoint{1.332473in}{1.870944in}}%
\pgfpathlineto{\pgfqpoint{1.332473in}{1.874695in}}%
\pgfpathlineto{\pgfqpoint{1.333883in}{1.876384in}}%
\pgfpathlineto{\pgfqpoint{1.335608in}{1.878447in}}%
\pgfpathlineto{\pgfqpoint{1.335608in}{1.882198in}}%
\pgfpathlineto{\pgfqpoint{1.335608in}{1.885950in}}%
\pgfpathlineto{\pgfqpoint{1.335608in}{1.889702in}}%
\pgfpathlineto{\pgfqpoint{1.337018in}{1.891390in}}%
\pgfpathlineto{\pgfqpoint{1.338742in}{1.893453in}}%
\pgfpathlineto{\pgfqpoint{1.338742in}{1.897205in}}%
\pgfpathlineto{\pgfqpoint{1.338742in}{1.900956in}}%
\pgfpathlineto{\pgfqpoint{1.340153in}{1.902644in}}%
\pgfpathlineto{\pgfqpoint{1.341877in}{1.904708in}}%
\pgfpathlineto{\pgfqpoint{1.341877in}{1.908459in}}%
\pgfpathlineto{\pgfqpoint{1.341877in}{1.912211in}}%
\pgfpathlineto{\pgfqpoint{1.341877in}{1.915962in}}%
\pgfpathlineto{\pgfqpoint{1.343288in}{1.917650in}}%
\pgfpathlineto{\pgfqpoint{1.345012in}{1.919714in}}%
\pgfpathlineto{\pgfqpoint{1.345012in}{1.923465in}}%
\pgfpathlineto{\pgfqpoint{1.345012in}{1.927217in}}%
\pgfpathlineto{\pgfqpoint{1.346422in}{1.928905in}}%
\pgfpathlineto{\pgfqpoint{1.348146in}{1.930968in}}%
\pgfpathlineto{\pgfqpoint{1.348146in}{1.934720in}}%
\pgfpathlineto{\pgfqpoint{1.348146in}{1.938471in}}%
\pgfpathlineto{\pgfqpoint{1.348146in}{1.942223in}}%
\pgfpathlineto{\pgfqpoint{1.349557in}{1.943911in}}%
\pgfpathlineto{\pgfqpoint{1.351281in}{1.945975in}}%
\pgfpathlineto{\pgfqpoint{1.351281in}{1.949726in}}%
\pgfpathlineto{\pgfqpoint{1.351281in}{1.953478in}}%
\pgfpathlineto{\pgfqpoint{1.352692in}{1.955166in}}%
\pgfpathlineto{\pgfqpoint{1.354416in}{1.957229in}}%
\pgfpathlineto{\pgfqpoint{1.354416in}{1.960981in}}%
\pgfpathlineto{\pgfqpoint{1.354416in}{1.964732in}}%
\pgfpathlineto{\pgfqpoint{1.354416in}{1.968484in}}%
\pgfpathlineto{\pgfqpoint{1.355827in}{1.970172in}}%
\pgfpathlineto{\pgfqpoint{1.357551in}{1.972235in}}%
\pgfpathlineto{\pgfqpoint{1.357551in}{1.975987in}}%
\pgfpathlineto{\pgfqpoint{1.357551in}{1.979738in}}%
\pgfpathlineto{\pgfqpoint{1.358961in}{1.981427in}}%
\pgfpathlineto{\pgfqpoint{1.360685in}{1.983490in}}%
\pgfpathlineto{\pgfqpoint{1.360685in}{1.987241in}}%
\pgfpathlineto{\pgfqpoint{1.360685in}{1.990993in}}%
\pgfpathlineto{\pgfqpoint{1.360685in}{1.994745in}}%
\pgfpathlineto{\pgfqpoint{1.362096in}{1.996433in}}%
\pgfpathlineto{\pgfqpoint{1.363820in}{1.998496in}}%
\pgfpathlineto{\pgfqpoint{1.363820in}{2.002248in}}%
\pgfpathlineto{\pgfqpoint{1.363820in}{2.005999in}}%
\pgfpathlineto{\pgfqpoint{1.365231in}{2.007687in}}%
\pgfpathlineto{\pgfqpoint{1.366955in}{2.009751in}}%
\pgfpathlineto{\pgfqpoint{1.366955in}{2.013502in}}%
\pgfpathlineto{\pgfqpoint{1.366955in}{2.017254in}}%
\pgfpathlineto{\pgfqpoint{1.366955in}{2.021005in}}%
\pgfpathlineto{\pgfqpoint{1.368366in}{2.022693in}}%
\pgfpathlineto{\pgfqpoint{1.370090in}{2.024757in}}%
\pgfpathlineto{\pgfqpoint{1.370090in}{2.028508in}}%
\pgfpathlineto{\pgfqpoint{1.370090in}{2.032260in}}%
\pgfpathlineto{\pgfqpoint{1.371500in}{2.033948in}}%
\pgfpathlineto{\pgfqpoint{1.373224in}{2.036011in}}%
\pgfpathlineto{\pgfqpoint{1.373224in}{2.039763in}}%
\pgfpathlineto{\pgfqpoint{1.373224in}{2.043514in}}%
\pgfpathlineto{\pgfqpoint{1.374635in}{2.045203in}}%
\pgfpathlineto{\pgfqpoint{1.376359in}{2.047266in}}%
\pgfpathlineto{\pgfqpoint{1.376359in}{2.051018in}}%
\pgfpathlineto{\pgfqpoint{1.376359in}{2.054769in}}%
\pgfpathlineto{\pgfqpoint{1.376359in}{2.058521in}}%
\pgfpathlineto{\pgfqpoint{1.377770in}{2.060209in}}%
\pgfpathlineto{\pgfqpoint{1.379494in}{2.062272in}}%
\pgfpathlineto{\pgfqpoint{1.379494in}{2.066024in}}%
\pgfpathlineto{\pgfqpoint{1.379494in}{2.069775in}}%
\pgfpathlineto{\pgfqpoint{1.380905in}{2.071463in}}%
\pgfpathlineto{\pgfqpoint{1.382629in}{2.073527in}}%
\pgfpathlineto{\pgfqpoint{1.382629in}{2.077278in}}%
\pgfpathlineto{\pgfqpoint{1.382629in}{2.081030in}}%
\pgfpathlineto{\pgfqpoint{1.382629in}{2.084781in}}%
\pgfpathlineto{\pgfqpoint{1.384039in}{2.086470in}}%
\pgfpathlineto{\pgfqpoint{1.385763in}{2.088533in}}%
\pgfpathlineto{\pgfqpoint{1.385763in}{2.092284in}}%
\pgfpathlineto{\pgfqpoint{1.385763in}{2.096036in}}%
\pgfpathlineto{\pgfqpoint{1.387174in}{2.097724in}}%
\pgfpathlineto{\pgfqpoint{1.388898in}{2.099787in}}%
\pgfpathlineto{\pgfqpoint{1.388898in}{2.103539in}}%
\pgfpathlineto{\pgfqpoint{1.388898in}{2.107291in}}%
\pgfpathlineto{\pgfqpoint{1.388898in}{2.111042in}}%
\pgfpathlineto{\pgfqpoint{1.390309in}{2.112730in}}%
\pgfpathlineto{\pgfqpoint{1.392033in}{2.114794in}}%
\pgfpathlineto{\pgfqpoint{1.392033in}{2.118545in}}%
\pgfpathlineto{\pgfqpoint{1.392033in}{2.122297in}}%
\pgfpathlineto{\pgfqpoint{1.393444in}{2.123985in}}%
\pgfpathlineto{\pgfqpoint{1.395168in}{2.126048in}}%
\pgfpathlineto{\pgfqpoint{1.395168in}{2.129800in}}%
\pgfpathlineto{\pgfqpoint{1.395168in}{2.133551in}}%
\pgfpathlineto{\pgfqpoint{1.395168in}{2.137303in}}%
\pgfpathlineto{\pgfqpoint{1.396578in}{2.138991in}}%
\pgfpathlineto{\pgfqpoint{1.398302in}{2.141054in}}%
\pgfpathlineto{\pgfqpoint{1.398302in}{2.144806in}}%
\pgfpathlineto{\pgfqpoint{1.398302in}{2.148557in}}%
\pgfpathlineto{\pgfqpoint{1.399713in}{2.150246in}}%
\pgfpathlineto{\pgfqpoint{1.401437in}{2.152309in}}%
\pgfpathlineto{\pgfqpoint{1.401437in}{2.156060in}}%
\pgfpathlineto{\pgfqpoint{1.401437in}{2.159812in}}%
\pgfpathlineto{\pgfqpoint{1.401437in}{2.163564in}}%
\pgfpathlineto{\pgfqpoint{1.402848in}{2.165252in}}%
\pgfpathlineto{\pgfqpoint{1.404572in}{2.167315in}}%
\pgfpathlineto{\pgfqpoint{1.404572in}{2.171067in}}%
\pgfpathlineto{\pgfqpoint{1.404572in}{2.174818in}}%
\pgfpathlineto{\pgfqpoint{1.405982in}{2.176506in}}%
\pgfpathlineto{\pgfqpoint{1.407707in}{2.178570in}}%
\pgfpathlineto{\pgfqpoint{1.407707in}{2.182321in}}%
\pgfpathlineto{\pgfqpoint{1.407707in}{2.186073in}}%
\pgfpathlineto{\pgfqpoint{1.407707in}{2.189824in}}%
\pgfpathlineto{\pgfqpoint{1.409117in}{2.191512in}}%
\pgfpathlineto{\pgfqpoint{1.410841in}{2.193576in}}%
\pgfpathlineto{\pgfqpoint{1.410841in}{2.197327in}}%
\pgfpathlineto{\pgfqpoint{1.410841in}{2.201079in}}%
\pgfpathlineto{\pgfqpoint{1.412252in}{2.202767in}}%
\pgfpathlineto{\pgfqpoint{1.413976in}{2.204830in}}%
\pgfpathlineto{\pgfqpoint{1.413976in}{2.208582in}}%
\pgfpathlineto{\pgfqpoint{1.413976in}{2.212334in}}%
\pgfpathlineto{\pgfqpoint{1.413976in}{2.216085in}}%
\pgfpathlineto{\pgfqpoint{1.415387in}{2.217773in}}%
\pgfpathlineto{\pgfqpoint{1.417111in}{2.219837in}}%
\pgfpathlineto{\pgfqpoint{1.417111in}{2.223588in}}%
\pgfpathlineto{\pgfqpoint{1.417111in}{2.227340in}}%
\pgfpathlineto{\pgfqpoint{1.418521in}{2.229028in}}%
\pgfpathlineto{\pgfqpoint{1.420246in}{2.231091in}}%
\pgfpathlineto{\pgfqpoint{1.420246in}{2.234843in}}%
\pgfpathlineto{\pgfqpoint{1.420246in}{2.238594in}}%
\pgfpathlineto{\pgfqpoint{1.420246in}{2.242346in}}%
\pgfpathlineto{\pgfqpoint{1.421656in}{2.244034in}}%
\pgfpathlineto{\pgfqpoint{1.423380in}{2.246097in}}%
\pgfpathlineto{\pgfqpoint{1.423380in}{2.249849in}}%
\pgfpathlineto{\pgfqpoint{1.423380in}{2.253600in}}%
\pgfpathlineto{\pgfqpoint{1.424791in}{2.255289in}}%
\pgfpathlineto{\pgfqpoint{1.426515in}{2.257352in}}%
\pgfpathlineto{\pgfqpoint{1.426515in}{2.261103in}}%
\pgfpathlineto{\pgfqpoint{1.426515in}{2.264855in}}%
\pgfpathlineto{\pgfqpoint{1.427926in}{2.266543in}}%
\pgfpathlineto{\pgfqpoint{1.429650in}{2.268607in}}%
\pgfpathlineto{\pgfqpoint{1.429650in}{2.272358in}}%
\pgfpathlineto{\pgfqpoint{1.429650in}{2.276110in}}%
\pgfpathlineto{\pgfqpoint{1.429650in}{2.279861in}}%
\pgfpathlineto{\pgfqpoint{1.431060in}{2.281549in}}%
\pgfpathlineto{\pgfqpoint{1.432785in}{2.283613in}}%
\pgfpathlineto{\pgfqpoint{1.432785in}{2.287364in}}%
\pgfpathlineto{\pgfqpoint{1.432785in}{2.291116in}}%
\pgfpathlineto{\pgfqpoint{1.434195in}{2.292804in}}%
\pgfpathlineto{\pgfqpoint{1.435919in}{2.294867in}}%
\pgfpathlineto{\pgfqpoint{1.435919in}{2.298619in}}%
\pgfpathlineto{\pgfqpoint{1.435919in}{2.302370in}}%
\pgfpathlineto{\pgfqpoint{1.435919in}{2.306122in}}%
\pgfpathlineto{\pgfqpoint{1.437330in}{2.307810in}}%
\pgfpathlineto{\pgfqpoint{1.439054in}{2.309873in}}%
\pgfpathlineto{\pgfqpoint{1.439054in}{2.313625in}}%
\pgfpathlineto{\pgfqpoint{1.439054in}{2.317376in}}%
\pgfpathlineto{\pgfqpoint{1.440465in}{2.319065in}}%
\pgfpathlineto{\pgfqpoint{1.442189in}{2.321128in}}%
\pgfpathlineto{\pgfqpoint{1.442189in}{2.324880in}}%
\pgfpathlineto{\pgfqpoint{1.442189in}{2.328631in}}%
\pgfpathlineto{\pgfqpoint{1.442189in}{2.332383in}}%
\pgfpathlineto{\pgfqpoint{1.443599in}{2.334071in}}%
\pgfpathlineto{\pgfqpoint{1.445323in}{2.336134in}}%
\pgfpathlineto{\pgfqpoint{1.445323in}{2.339886in}}%
\pgfpathlineto{\pgfqpoint{1.445323in}{2.343637in}}%
\pgfpathlineto{\pgfqpoint{1.446734in}{2.345325in}}%
\pgfpathlineto{\pgfqpoint{1.448458in}{2.347389in}}%
\pgfpathlineto{\pgfqpoint{1.448458in}{2.351140in}}%
\pgfpathlineto{\pgfqpoint{1.448458in}{2.354892in}}%
\pgfpathlineto{\pgfqpoint{1.448458in}{2.358643in}}%
\pgfpathlineto{\pgfqpoint{1.449869in}{2.360332in}}%
\pgfpathlineto{\pgfqpoint{1.451593in}{2.362395in}}%
\pgfpathlineto{\pgfqpoint{1.451593in}{2.366146in}}%
\pgfpathlineto{\pgfqpoint{1.451593in}{2.369898in}}%
\pgfpathlineto{\pgfqpoint{1.453004in}{2.371586in}}%
\pgfpathlineto{\pgfqpoint{1.454728in}{2.373649in}}%
\pgfpathlineto{\pgfqpoint{1.454728in}{2.377401in}}%
\pgfpathlineto{\pgfqpoint{1.454728in}{2.381153in}}%
\pgfpathlineto{\pgfqpoint{1.454728in}{2.384904in}}%
\pgfpathlineto{\pgfqpoint{1.456138in}{2.386592in}}%
\pgfpathlineto{\pgfqpoint{1.457862in}{2.388656in}}%
\pgfpathlineto{\pgfqpoint{1.457862in}{2.392407in}}%
\pgfpathlineto{\pgfqpoint{1.457862in}{2.396159in}}%
\pgfpathlineto{\pgfqpoint{1.459273in}{2.397847in}}%
\pgfpathlineto{\pgfqpoint{1.460997in}{2.399910in}}%
\pgfpathlineto{\pgfqpoint{1.460997in}{2.403662in}}%
\pgfpathlineto{\pgfqpoint{1.462408in}{2.405350in}}%
\pgfpathlineto{\pgfqpoint{1.465543in}{2.405350in}}%
\pgfpathlineto{\pgfqpoint{1.467267in}{2.407413in}}%
\pgfpathlineto{\pgfqpoint{1.468677in}{2.409101in}}%
\pgfpathlineto{\pgfqpoint{1.471812in}{2.409101in}}%
\pgfpathlineto{\pgfqpoint{1.474947in}{2.409101in}}%
\pgfpathlineto{\pgfqpoint{1.478082in}{2.409101in}}%
\pgfpathlineto{\pgfqpoint{1.479806in}{2.411165in}}%
\pgfpathlineto{\pgfqpoint{1.481216in}{2.412853in}}%
\pgfpathlineto{\pgfqpoint{1.484351in}{2.412853in}}%
\pgfpathlineto{\pgfqpoint{1.487486in}{2.412853in}}%
\pgfpathlineto{\pgfqpoint{1.490621in}{2.412853in}}%
\pgfpathlineto{\pgfqpoint{1.492345in}{2.414916in}}%
\pgfpathlineto{\pgfqpoint{1.493755in}{2.416605in}}%
\pgfpathlineto{\pgfqpoint{1.496890in}{2.416605in}}%
\pgfpathlineto{\pgfqpoint{1.500025in}{2.416605in}}%
\pgfpathlineto{\pgfqpoint{1.503159in}{2.416605in}}%
\pgfpathlineto{\pgfqpoint{1.504884in}{2.418668in}}%
\pgfpathlineto{\pgfqpoint{1.506294in}{2.420356in}}%
\pgfpathlineto{\pgfqpoint{1.509429in}{2.420356in}}%
\pgfpathlineto{\pgfqpoint{1.512564in}{2.420356in}}%
\pgfpathlineto{\pgfqpoint{1.515698in}{2.420356in}}%
\pgfpathlineto{\pgfqpoint{1.517423in}{2.422419in}}%
\pgfpathlineto{\pgfqpoint{1.518833in}{2.424108in}}%
\pgfpathlineto{\pgfqpoint{1.521968in}{2.424108in}}%
\pgfpathlineto{\pgfqpoint{1.525103in}{2.424108in}}%
\pgfpathlineto{\pgfqpoint{1.528237in}{2.424108in}}%
\pgfpathlineto{\pgfqpoint{1.531372in}{2.424108in}}%
\pgfpathlineto{\pgfqpoint{1.533096in}{2.426171in}}%
\pgfpathlineto{\pgfqpoint{1.534507in}{2.427859in}}%
\pgfpathlineto{\pgfqpoint{1.537642in}{2.427859in}}%
\pgfpathlineto{\pgfqpoint{1.540776in}{2.427859in}}%
\pgfpathlineto{\pgfqpoint{1.543911in}{2.427859in}}%
\pgfpathlineto{\pgfqpoint{1.545635in}{2.429923in}}%
\pgfpathlineto{\pgfqpoint{1.547046in}{2.431611in}}%
\pgfpathlineto{\pgfqpoint{1.550181in}{2.431611in}}%
\pgfpathlineto{\pgfqpoint{1.553315in}{2.431611in}}%
\pgfpathlineto{\pgfqpoint{1.556450in}{2.431611in}}%
\pgfpathlineto{\pgfqpoint{1.558174in}{2.433674in}}%
\pgfpathlineto{\pgfqpoint{1.559585in}{2.435362in}}%
\pgfpathlineto{\pgfqpoint{1.562720in}{2.435362in}}%
\pgfpathlineto{\pgfqpoint{1.565854in}{2.435362in}}%
\pgfpathlineto{\pgfqpoint{1.568989in}{2.435362in}}%
\pgfpathlineto{\pgfqpoint{1.570713in}{2.437426in}}%
\pgfpathlineto{\pgfqpoint{1.572124in}{2.439114in}}%
\pgfpathlineto{\pgfqpoint{1.575259in}{2.439114in}}%
\pgfpathlineto{\pgfqpoint{1.578393in}{2.439114in}}%
\pgfpathlineto{\pgfqpoint{1.581528in}{2.439114in}}%
\pgfpathlineto{\pgfqpoint{1.584663in}{2.439114in}}%
\pgfpathlineto{\pgfqpoint{1.586387in}{2.441177in}}%
\pgfpathlineto{\pgfqpoint{1.587798in}{2.442865in}}%
\pgfpathlineto{\pgfqpoint{1.590932in}{2.442865in}}%
\pgfpathlineto{\pgfqpoint{1.594067in}{2.442865in}}%
\pgfpathlineto{\pgfqpoint{1.597202in}{2.442865in}}%
\pgfpathlineto{\pgfqpoint{1.598926in}{2.444929in}}%
\pgfpathlineto{\pgfqpoint{1.600337in}{2.446617in}}%
\pgfpathlineto{\pgfqpoint{1.603471in}{2.446617in}}%
\pgfpathlineto{\pgfqpoint{1.606606in}{2.446617in}}%
\pgfpathlineto{\pgfqpoint{1.609741in}{2.446617in}}%
\pgfpathlineto{\pgfqpoint{1.611465in}{2.448680in}}%
\pgfpathlineto{\pgfqpoint{1.612875in}{2.450368in}}%
\pgfpathlineto{\pgfqpoint{1.616010in}{2.450368in}}%
\pgfpathlineto{\pgfqpoint{1.619145in}{2.450368in}}%
\pgfpathlineto{\pgfqpoint{1.622280in}{2.450368in}}%
\pgfpathlineto{\pgfqpoint{1.624004in}{2.452432in}}%
\pgfpathlineto{\pgfqpoint{1.625414in}{2.454120in}}%
\pgfpathlineto{\pgfqpoint{1.628549in}{2.454120in}}%
\pgfpathlineto{\pgfqpoint{1.631684in}{2.454120in}}%
\pgfpathlineto{\pgfqpoint{1.634819in}{2.454120in}}%
\pgfpathlineto{\pgfqpoint{1.637953in}{2.454120in}}%
\pgfpathlineto{\pgfqpoint{1.639678in}{2.456183in}}%
\pgfpathlineto{\pgfqpoint{1.641088in}{2.457871in}}%
\pgfpathlineto{\pgfqpoint{1.644223in}{2.457871in}}%
\pgfpathlineto{\pgfqpoint{1.647358in}{2.457871in}}%
\pgfpathlineto{\pgfqpoint{1.650492in}{2.457871in}}%
\pgfpathlineto{\pgfqpoint{1.652216in}{2.459935in}}%
\pgfpathlineto{\pgfqpoint{1.653627in}{2.461623in}}%
\pgfpathlineto{\pgfqpoint{1.656762in}{2.461623in}}%
\pgfpathlineto{\pgfqpoint{1.659897in}{2.461623in}}%
\pgfpathlineto{\pgfqpoint{1.663031in}{2.461623in}}%
\pgfpathlineto{\pgfqpoint{1.664755in}{2.463686in}}%
\pgfpathlineto{\pgfqpoint{1.666166in}{2.465375in}}%
\pgfpathlineto{\pgfqpoint{1.669301in}{2.465375in}}%
\pgfpathlineto{\pgfqpoint{1.672436in}{2.465375in}}%
\pgfpathlineto{\pgfqpoint{1.675570in}{2.465375in}}%
\pgfpathlineto{\pgfqpoint{1.677294in}{2.467438in}}%
\pgfpathlineto{\pgfqpoint{1.678705in}{2.469126in}}%
\pgfpathlineto{\pgfqpoint{1.681840in}{2.469126in}}%
\pgfpathlineto{\pgfqpoint{1.684975in}{2.469126in}}%
\pgfpathlineto{\pgfqpoint{1.688109in}{2.469126in}}%
\pgfpathlineto{\pgfqpoint{1.689833in}{2.471189in}}%
\pgfpathlineto{\pgfqpoint{1.691244in}{2.472878in}}%
\pgfpathlineto{\pgfqpoint{1.694379in}{2.472878in}}%
\pgfpathlineto{\pgfqpoint{1.697514in}{2.472878in}}%
\pgfpathlineto{\pgfqpoint{1.700648in}{2.472878in}}%
\pgfpathlineto{\pgfqpoint{1.703783in}{2.472878in}}%
\pgfpathlineto{\pgfqpoint{1.705507in}{2.474941in}}%
\pgfpathlineto{\pgfqpoint{1.706918in}{2.476629in}}%
\pgfpathlineto{\pgfqpoint{1.710052in}{2.476629in}}%
\pgfpathlineto{\pgfqpoint{1.713187in}{2.476629in}}%
\pgfpathlineto{\pgfqpoint{1.716322in}{2.476629in}}%
\pgfpathlineto{\pgfqpoint{1.718046in}{2.478692in}}%
\pgfpathlineto{\pgfqpoint{1.719457in}{2.480381in}}%
\pgfpathlineto{\pgfqpoint{1.722591in}{2.480381in}}%
\pgfpathlineto{\pgfqpoint{1.725726in}{2.480381in}}%
\pgfpathlineto{\pgfqpoint{1.728861in}{2.480381in}}%
\pgfpathlineto{\pgfqpoint{1.730585in}{2.482444in}}%
\pgfpathlineto{\pgfqpoint{1.731996in}{2.484132in}}%
\pgfpathlineto{\pgfqpoint{1.735130in}{2.484132in}}%
\pgfpathlineto{\pgfqpoint{1.738265in}{2.484132in}}%
\pgfpathlineto{\pgfqpoint{1.741400in}{2.484132in}}%
\pgfpathlineto{\pgfqpoint{1.743124in}{2.486196in}}%
\pgfpathlineto{\pgfqpoint{1.744535in}{2.487884in}}%
\pgfpathlineto{\pgfqpoint{1.747669in}{2.487884in}}%
\pgfpathlineto{\pgfqpoint{1.750804in}{2.487884in}}%
\pgfpathlineto{\pgfqpoint{1.753939in}{2.487884in}}%
\pgfpathlineto{\pgfqpoint{1.757074in}{2.487884in}}%
\pgfpathlineto{\pgfqpoint{1.758798in}{2.489947in}}%
\pgfpathlineto{\pgfqpoint{1.760208in}{2.491635in}}%
\pgfpathlineto{\pgfqpoint{1.763343in}{2.491635in}}%
\pgfpathlineto{\pgfqpoint{1.766478in}{2.491635in}}%
\pgfpathlineto{\pgfqpoint{1.769613in}{2.491635in}}%
\pgfpathlineto{\pgfqpoint{1.771337in}{2.493699in}}%
\pgfpathlineto{\pgfqpoint{1.772747in}{2.495387in}}%
\pgfpathlineto{\pgfqpoint{1.775882in}{2.495387in}}%
\pgfpathlineto{\pgfqpoint{1.779017in}{2.495387in}}%
\pgfpathlineto{\pgfqpoint{1.782152in}{2.495387in}}%
\pgfpathlineto{\pgfqpoint{1.783876in}{2.497450in}}%
\pgfpathlineto{\pgfqpoint{1.785286in}{2.499138in}}%
\pgfpathlineto{\pgfqpoint{1.788421in}{2.499138in}}%
\pgfpathlineto{\pgfqpoint{1.791556in}{2.499138in}}%
\pgfpathlineto{\pgfqpoint{1.794691in}{2.499138in}}%
\pgfpathlineto{\pgfqpoint{1.796415in}{2.501202in}}%
\pgfpathlineto{\pgfqpoint{1.797825in}{2.502890in}}%
\pgfpathlineto{\pgfqpoint{1.800960in}{2.502890in}}%
\pgfpathlineto{\pgfqpoint{1.804095in}{2.502890in}}%
\pgfpathlineto{\pgfqpoint{1.807229in}{2.502890in}}%
\pgfpathlineto{\pgfqpoint{1.810364in}{2.502890in}}%
\pgfpathlineto{\pgfqpoint{1.812088in}{2.504953in}}%
\pgfpathlineto{\pgfqpoint{1.813499in}{2.506641in}}%
\pgfpathlineto{\pgfqpoint{1.816634in}{2.506641in}}%
\pgfpathlineto{\pgfqpoint{1.819768in}{2.506641in}}%
\pgfpathlineto{\pgfqpoint{1.822903in}{2.506641in}}%
\pgfpathlineto{\pgfqpoint{1.824627in}{2.508705in}}%
\pgfpathlineto{\pgfqpoint{1.826038in}{2.510393in}}%
\pgfpathlineto{\pgfqpoint{1.829173in}{2.510393in}}%
\pgfpathlineto{\pgfqpoint{1.832307in}{2.510393in}}%
\pgfpathlineto{\pgfqpoint{1.835442in}{2.510393in}}%
\pgfpathlineto{\pgfqpoint{1.837166in}{2.512456in}}%
\pgfpathlineto{\pgfqpoint{1.838577in}{2.514144in}}%
\pgfpathlineto{\pgfqpoint{1.841712in}{2.514144in}}%
\pgfpathlineto{\pgfqpoint{1.844846in}{2.514144in}}%
\pgfpathlineto{\pgfqpoint{1.847981in}{2.514144in}}%
\pgfpathlineto{\pgfqpoint{1.849705in}{2.516208in}}%
\pgfpathlineto{\pgfqpoint{1.851116in}{2.517896in}}%
\pgfpathlineto{\pgfqpoint{1.854251in}{2.517896in}}%
\pgfpathlineto{\pgfqpoint{1.857385in}{2.517896in}}%
\pgfpathlineto{\pgfqpoint{1.860520in}{2.517896in}}%
\pgfpathlineto{\pgfqpoint{1.862244in}{2.519959in}}%
\pgfpathlineto{\pgfqpoint{1.863655in}{2.521648in}}%
\pgfpathlineto{\pgfqpoint{1.866790in}{2.521648in}}%
\pgfpathlineto{\pgfqpoint{1.869924in}{2.521648in}}%
\pgfpathlineto{\pgfqpoint{1.873059in}{2.521648in}}%
\pgfpathlineto{\pgfqpoint{1.876194in}{2.521648in}}%
\pgfpathlineto{\pgfqpoint{1.877918in}{2.523711in}}%
\pgfpathlineto{\pgfqpoint{1.879329in}{2.525399in}}%
\pgfpathlineto{\pgfqpoint{1.882463in}{2.525399in}}%
\pgfpathlineto{\pgfqpoint{1.885598in}{2.525399in}}%
\pgfpathlineto{\pgfqpoint{1.888733in}{2.525399in}}%
\pgfpathlineto{\pgfqpoint{1.890457in}{2.527462in}}%
\pgfpathlineto{\pgfqpoint{1.891868in}{2.529151in}}%
\pgfpathlineto{\pgfqpoint{1.895002in}{2.529151in}}%
\pgfpathlineto{\pgfqpoint{1.898137in}{2.529151in}}%
\pgfpathlineto{\pgfqpoint{1.901272in}{2.529151in}}%
\pgfpathlineto{\pgfqpoint{1.902996in}{2.531214in}}%
\pgfpathlineto{\pgfqpoint{1.904407in}{2.532902in}}%
\pgfpathlineto{\pgfqpoint{1.907541in}{2.532902in}}%
\pgfpathlineto{\pgfqpoint{1.910676in}{2.532902in}}%
\pgfpathlineto{\pgfqpoint{1.913811in}{2.532902in}}%
\pgfpathlineto{\pgfqpoint{1.915535in}{2.534965in}}%
\pgfpathlineto{\pgfqpoint{1.916945in}{2.536654in}}%
\pgfpathlineto{\pgfqpoint{1.920080in}{2.536654in}}%
\pgfpathlineto{\pgfqpoint{1.923215in}{2.536654in}}%
\pgfpathlineto{\pgfqpoint{1.926350in}{2.536654in}}%
\pgfpathlineto{\pgfqpoint{1.929484in}{2.536654in}}%
\pgfpathlineto{\pgfqpoint{1.931209in}{2.538717in}}%
\pgfpathlineto{\pgfqpoint{1.932619in}{2.540405in}}%
\pgfpathlineto{\pgfqpoint{1.935754in}{2.540405in}}%
\pgfpathlineto{\pgfqpoint{1.938889in}{2.540405in}}%
\pgfpathlineto{\pgfqpoint{1.942023in}{2.540405in}}%
\pgfpathlineto{\pgfqpoint{1.943748in}{2.542469in}}%
\pgfpathlineto{\pgfqpoint{1.945158in}{2.544157in}}%
\pgfpathlineto{\pgfqpoint{1.948293in}{2.544157in}}%
\pgfpathlineto{\pgfqpoint{1.951428in}{2.544157in}}%
\pgfpathlineto{\pgfqpoint{1.954562in}{2.544157in}}%
\pgfpathlineto{\pgfqpoint{1.956286in}{2.546220in}}%
\pgfpathlineto{\pgfqpoint{1.957697in}{2.547908in}}%
\pgfpathlineto{\pgfqpoint{1.960832in}{2.547908in}}%
\pgfpathlineto{\pgfqpoint{1.963967in}{2.547908in}}%
\pgfpathlineto{\pgfqpoint{1.967101in}{2.547908in}}%
\pgfpathlineto{\pgfqpoint{1.968825in}{2.549972in}}%
\pgfpathlineto{\pgfqpoint{1.970236in}{2.551660in}}%
\pgfpathlineto{\pgfqpoint{1.973371in}{2.551660in}}%
\pgfpathlineto{\pgfqpoint{1.976506in}{2.551660in}}%
\pgfpathlineto{\pgfqpoint{1.979640in}{2.551660in}}%
\pgfpathlineto{\pgfqpoint{1.981364in}{2.553723in}}%
\pgfpathlineto{\pgfqpoint{1.982775in}{2.555411in}}%
\pgfpathlineto{\pgfqpoint{1.985910in}{2.555411in}}%
\pgfpathlineto{\pgfqpoint{1.989045in}{2.555411in}}%
\pgfpathlineto{\pgfqpoint{1.992179in}{2.555411in}}%
\pgfpathlineto{\pgfqpoint{1.995314in}{2.555411in}}%
\pgfpathlineto{\pgfqpoint{1.997038in}{2.557475in}}%
\pgfpathlineto{\pgfqpoint{1.998449in}{2.559163in}}%
\pgfpathlineto{\pgfqpoint{2.001584in}{2.559163in}}%
\pgfpathlineto{\pgfqpoint{2.004718in}{2.559163in}}%
\pgfpathlineto{\pgfqpoint{2.007853in}{2.559163in}}%
\pgfpathlineto{\pgfqpoint{2.009577in}{2.561226in}}%
\pgfpathlineto{\pgfqpoint{2.010988in}{2.562914in}}%
\pgfpathlineto{\pgfqpoint{2.014122in}{2.562914in}}%
\pgfpathlineto{\pgfqpoint{2.017257in}{2.562914in}}%
\pgfpathlineto{\pgfqpoint{2.020392in}{2.562914in}}%
\pgfpathlineto{\pgfqpoint{2.022116in}{2.564978in}}%
\pgfpathlineto{\pgfqpoint{2.023527in}{2.566666in}}%
\pgfpathlineto{\pgfqpoint{2.026661in}{2.566666in}}%
\pgfpathlineto{\pgfqpoint{2.029796in}{2.566666in}}%
\pgfpathlineto{\pgfqpoint{2.032931in}{2.566666in}}%
\pgfpathlineto{\pgfqpoint{2.034655in}{2.568729in}}%
\pgfpathlineto{\pgfqpoint{2.036066in}{2.570417in}}%
\pgfpathlineto{\pgfqpoint{2.039200in}{2.570417in}}%
\pgfpathlineto{\pgfqpoint{2.042335in}{2.570417in}}%
\pgfpathlineto{\pgfqpoint{2.045470in}{2.570417in}}%
\pgfpathlineto{\pgfqpoint{2.048605in}{2.570417in}}%
\pgfpathlineto{\pgfqpoint{2.050329in}{2.572481in}}%
\pgfpathlineto{\pgfqpoint{2.051739in}{2.574169in}}%
\pgfpathlineto{\pgfqpoint{2.054874in}{2.574169in}}%
\pgfpathlineto{\pgfqpoint{2.058009in}{2.574169in}}%
\pgfpathlineto{\pgfqpoint{2.061144in}{2.574169in}}%
\pgfpathlineto{\pgfqpoint{2.062868in}{2.576232in}}%
\pgfpathlineto{\pgfqpoint{2.064278in}{2.577921in}}%
\pgfpathlineto{\pgfqpoint{2.067413in}{2.577921in}}%
\pgfpathlineto{\pgfqpoint{2.070548in}{2.577921in}}%
\pgfpathlineto{\pgfqpoint{2.073683in}{2.577921in}}%
\pgfpathlineto{\pgfqpoint{2.075407in}{2.579984in}}%
\pgfpathlineto{\pgfqpoint{2.076817in}{2.581672in}}%
\pgfpathlineto{\pgfqpoint{2.079952in}{2.581672in}}%
\pgfpathlineto{\pgfqpoint{2.083087in}{2.581672in}}%
\pgfpathlineto{\pgfqpoint{2.086222in}{2.581672in}}%
\pgfpathlineto{\pgfqpoint{2.087946in}{2.583735in}}%
\pgfpathlineto{\pgfqpoint{2.089356in}{2.585424in}}%
\pgfpathlineto{\pgfqpoint{2.092491in}{2.585424in}}%
\pgfpathlineto{\pgfqpoint{2.095626in}{2.585424in}}%
\pgfpathlineto{\pgfqpoint{2.098761in}{2.585424in}}%
\pgfpathlineto{\pgfqpoint{2.101895in}{2.585424in}}%
\pgfpathlineto{\pgfqpoint{2.103619in}{2.587487in}}%
\pgfpathlineto{\pgfqpoint{2.105030in}{2.589175in}}%
\pgfpathlineto{\pgfqpoint{2.108165in}{2.589175in}}%
\pgfpathlineto{\pgfqpoint{2.111299in}{2.589175in}}%
\pgfpathlineto{\pgfqpoint{2.114434in}{2.589175in}}%
\pgfpathlineto{\pgfqpoint{2.116158in}{2.591238in}}%
\pgfpathlineto{\pgfqpoint{2.117569in}{2.592927in}}%
\pgfpathlineto{\pgfqpoint{2.120704in}{2.592927in}}%
\pgfpathlineto{\pgfqpoint{2.123838in}{2.592927in}}%
\pgfpathlineto{\pgfqpoint{2.126973in}{2.592927in}}%
\pgfpathlineto{\pgfqpoint{2.128697in}{2.594990in}}%
\pgfpathlineto{\pgfqpoint{2.130108in}{2.596678in}}%
\pgfpathlineto{\pgfqpoint{2.133243in}{2.596678in}}%
\pgfpathlineto{\pgfqpoint{2.136377in}{2.596678in}}%
\pgfpathlineto{\pgfqpoint{2.139512in}{2.596678in}}%
\pgfpathlineto{\pgfqpoint{2.141236in}{2.598742in}}%
\pgfpathlineto{\pgfqpoint{2.142647in}{2.600430in}}%
\pgfpathlineto{\pgfqpoint{2.145782in}{2.600430in}}%
\pgfpathlineto{\pgfqpoint{2.148916in}{2.600430in}}%
\pgfpathlineto{\pgfqpoint{2.152051in}{2.600430in}}%
\pgfpathlineto{\pgfqpoint{2.153775in}{2.602493in}}%
\pgfpathlineto{\pgfqpoint{2.155186in}{2.604181in}}%
\pgfpathlineto{\pgfqpoint{2.158321in}{2.604181in}}%
\pgfpathlineto{\pgfqpoint{2.161455in}{2.604181in}}%
\pgfpathlineto{\pgfqpoint{2.164590in}{2.604181in}}%
\pgfpathlineto{\pgfqpoint{2.167725in}{2.604181in}}%
\pgfpathlineto{\pgfqpoint{2.169449in}{2.606245in}}%
\pgfpathlineto{\pgfqpoint{2.170860in}{2.607933in}}%
\pgfpathlineto{\pgfqpoint{2.173994in}{2.607933in}}%
\pgfpathlineto{\pgfqpoint{2.177129in}{2.607933in}}%
\pgfpathlineto{\pgfqpoint{2.180264in}{2.607933in}}%
\pgfpathlineto{\pgfqpoint{2.181988in}{2.609996in}}%
\pgfpathlineto{\pgfqpoint{2.183399in}{2.611684in}}%
\pgfpathlineto{\pgfqpoint{2.186533in}{2.611684in}}%
\pgfpathlineto{\pgfqpoint{2.189668in}{2.611684in}}%
\pgfpathlineto{\pgfqpoint{2.192803in}{2.611684in}}%
\pgfpathlineto{\pgfqpoint{2.194527in}{2.613748in}}%
\pgfpathlineto{\pgfqpoint{2.195938in}{2.615436in}}%
\pgfpathlineto{\pgfqpoint{2.199072in}{2.615436in}}%
\pgfpathlineto{\pgfqpoint{2.202207in}{2.615436in}}%
\pgfpathlineto{\pgfqpoint{2.205342in}{2.615436in}}%
\pgfpathlineto{\pgfqpoint{2.207066in}{2.617499in}}%
\pgfpathlineto{\pgfqpoint{2.208477in}{2.619187in}}%
\pgfpathlineto{\pgfqpoint{2.211611in}{2.619187in}}%
\pgfpathlineto{\pgfqpoint{2.214746in}{2.619187in}}%
\pgfpathlineto{\pgfqpoint{2.217881in}{2.619187in}}%
\pgfpathlineto{\pgfqpoint{2.221015in}{2.619187in}}%
\pgfpathlineto{\pgfqpoint{2.222740in}{2.621251in}}%
\pgfpathlineto{\pgfqpoint{2.224150in}{2.622939in}}%
\pgfpathlineto{\pgfqpoint{2.227285in}{2.622939in}}%
\pgfpathlineto{\pgfqpoint{2.230420in}{2.622939in}}%
\pgfpathlineto{\pgfqpoint{2.233554in}{2.622939in}}%
\pgfpathlineto{\pgfqpoint{2.235279in}{2.625002in}}%
\pgfpathlineto{\pgfqpoint{2.236689in}{2.626690in}}%
\pgfpathlineto{\pgfqpoint{2.239824in}{2.626690in}}%
\pgfpathlineto{\pgfqpoint{2.242959in}{2.626690in}}%
\pgfpathlineto{\pgfqpoint{2.246093in}{2.626690in}}%
\pgfpathlineto{\pgfqpoint{2.247818in}{2.628754in}}%
\pgfpathlineto{\pgfqpoint{2.249228in}{2.630442in}}%
\pgfpathlineto{\pgfqpoint{2.252363in}{2.630442in}}%
\pgfpathlineto{\pgfqpoint{2.255498in}{2.630442in}}%
\pgfpathlineto{\pgfqpoint{2.258632in}{2.630442in}}%
\pgfpathlineto{\pgfqpoint{2.260356in}{2.632505in}}%
\pgfpathlineto{\pgfqpoint{2.261767in}{2.634194in}}%
\pgfpathlineto{\pgfqpoint{2.264902in}{2.634194in}}%
\pgfpathlineto{\pgfqpoint{2.268037in}{2.634194in}}%
\pgfpathlineto{\pgfqpoint{2.271171in}{2.634194in}}%
\pgfpathlineto{\pgfqpoint{2.274306in}{2.634194in}}%
\pgfpathlineto{\pgfqpoint{2.276030in}{2.636257in}}%
\pgfpathlineto{\pgfqpoint{2.277441in}{2.637945in}}%
\pgfpathlineto{\pgfqpoint{2.280576in}{2.637945in}}%
\pgfpathlineto{\pgfqpoint{2.283710in}{2.637945in}}%
\pgfpathlineto{\pgfqpoint{2.286845in}{2.637945in}}%
\pgfpathlineto{\pgfqpoint{2.288569in}{2.640008in}}%
\pgfpathlineto{\pgfqpoint{2.289980in}{2.641697in}}%
\pgfpathlineto{\pgfqpoint{2.293115in}{2.641697in}}%
\pgfpathlineto{\pgfqpoint{2.296249in}{2.641697in}}%
\pgfpathlineto{\pgfqpoint{2.299384in}{2.641697in}}%
\pgfpathlineto{\pgfqpoint{2.301108in}{2.643760in}}%
\pgfpathlineto{\pgfqpoint{2.302519in}{2.645448in}}%
\pgfpathlineto{\pgfqpoint{2.305654in}{2.645448in}}%
\pgfpathlineto{\pgfqpoint{2.308788in}{2.645448in}}%
\pgfpathlineto{\pgfqpoint{2.311923in}{2.645448in}}%
\pgfpathlineto{\pgfqpoint{2.313647in}{2.647512in}}%
\pgfpathlineto{\pgfqpoint{2.315058in}{2.649200in}}%
\pgfpathlineto{\pgfqpoint{2.318192in}{2.649200in}}%
\pgfpathlineto{\pgfqpoint{2.321327in}{2.649200in}}%
\pgfpathlineto{\pgfqpoint{2.324462in}{2.649200in}}%
\pgfpathlineto{\pgfqpoint{2.326186in}{2.651263in}}%
\pgfpathlineto{\pgfqpoint{2.327597in}{2.652951in}}%
\pgfpathlineto{\pgfqpoint{2.330731in}{2.652951in}}%
\pgfpathlineto{\pgfqpoint{2.333866in}{2.652951in}}%
\pgfpathlineto{\pgfqpoint{2.337001in}{2.652951in}}%
\pgfpathlineto{\pgfqpoint{2.340136in}{2.652951in}}%
\pgfpathlineto{\pgfqpoint{2.341860in}{2.655015in}}%
\pgfpathlineto{\pgfqpoint{2.343270in}{2.656703in}}%
\pgfpathlineto{\pgfqpoint{2.346405in}{2.656703in}}%
\pgfpathlineto{\pgfqpoint{2.349540in}{2.656703in}}%
\pgfpathlineto{\pgfqpoint{2.352675in}{2.656703in}}%
\pgfpathlineto{\pgfqpoint{2.354399in}{2.658766in}}%
\pgfpathlineto{\pgfqpoint{2.355809in}{2.660454in}}%
\pgfpathlineto{\pgfqpoint{2.358944in}{2.660454in}}%
\pgfpathlineto{\pgfqpoint{2.362079in}{2.660454in}}%
\pgfpathlineto{\pgfqpoint{2.365214in}{2.660454in}}%
\pgfpathlineto{\pgfqpoint{2.366938in}{2.662518in}}%
\pgfpathlineto{\pgfqpoint{2.368348in}{2.664206in}}%
\pgfpathlineto{\pgfqpoint{2.371483in}{2.664206in}}%
\pgfpathlineto{\pgfqpoint{2.374618in}{2.664206in}}%
\pgfpathlineto{\pgfqpoint{2.377753in}{2.664206in}}%
\pgfpathlineto{\pgfqpoint{2.379477in}{2.666269in}}%
\pgfpathlineto{\pgfqpoint{2.380887in}{2.667957in}}%
\pgfpathlineto{\pgfqpoint{2.384022in}{2.667957in}}%
\pgfpathlineto{\pgfqpoint{2.387157in}{2.667957in}}%
\pgfpathlineto{\pgfqpoint{2.390292in}{2.667957in}}%
\pgfpathlineto{\pgfqpoint{2.393426in}{2.667957in}}%
\pgfpathlineto{\pgfqpoint{2.395150in}{2.670021in}}%
\pgfpathlineto{\pgfqpoint{2.396561in}{2.671709in}}%
\pgfpathlineto{\pgfqpoint{2.399696in}{2.671709in}}%
\pgfpathlineto{\pgfqpoint{2.402831in}{2.671709in}}%
\pgfpathlineto{\pgfqpoint{2.405965in}{2.671709in}}%
\pgfpathlineto{\pgfqpoint{2.407689in}{2.673772in}}%
\pgfpathlineto{\pgfqpoint{2.409100in}{2.675460in}}%
\pgfpathlineto{\pgfqpoint{2.412235in}{2.675460in}}%
\pgfpathlineto{\pgfqpoint{2.415369in}{2.675460in}}%
\pgfpathlineto{\pgfqpoint{2.418504in}{2.675460in}}%
\pgfpathlineto{\pgfqpoint{2.420228in}{2.677524in}}%
\pgfpathlineto{\pgfqpoint{2.421639in}{2.679212in}}%
\pgfpathlineto{\pgfqpoint{2.424774in}{2.679212in}}%
\pgfpathlineto{\pgfqpoint{2.427908in}{2.679212in}}%
\pgfpathlineto{\pgfqpoint{2.431043in}{2.679212in}}%
\pgfpathlineto{\pgfqpoint{2.432767in}{2.681275in}}%
\pgfpathlineto{\pgfqpoint{2.434178in}{2.682964in}}%
\pgfpathlineto{\pgfqpoint{2.437313in}{2.682964in}}%
\pgfpathlineto{\pgfqpoint{2.440447in}{2.682964in}}%
\pgfpathlineto{\pgfqpoint{2.443582in}{2.682964in}}%
\pgfpathlineto{\pgfqpoint{2.446717in}{2.682964in}}%
\pgfpathlineto{\pgfqpoint{2.448441in}{2.685027in}}%
\pgfpathlineto{\pgfqpoint{2.449852in}{2.686715in}}%
\pgfpathlineto{\pgfqpoint{2.452986in}{2.686715in}}%
\pgfpathlineto{\pgfqpoint{2.456121in}{2.686715in}}%
\pgfpathlineto{\pgfqpoint{2.459256in}{2.686715in}}%
\pgfpathlineto{\pgfqpoint{2.460980in}{2.688778in}}%
\pgfpathlineto{\pgfqpoint{2.462391in}{2.690467in}}%
\pgfpathlineto{\pgfqpoint{2.465525in}{2.690467in}}%
\pgfpathlineto{\pgfqpoint{2.468660in}{2.690467in}}%
\pgfpathlineto{\pgfqpoint{2.471795in}{2.690467in}}%
\pgfpathlineto{\pgfqpoint{2.473519in}{2.692530in}}%
\pgfpathlineto{\pgfqpoint{2.474930in}{2.694218in}}%
\pgfpathlineto{\pgfqpoint{2.478064in}{2.694218in}}%
\pgfpathlineto{\pgfqpoint{2.481199in}{2.694218in}}%
\pgfpathlineto{\pgfqpoint{2.484334in}{2.694218in}}%
\pgfpathlineto{\pgfqpoint{2.486058in}{2.696281in}}%
\pgfpathlineto{\pgfqpoint{2.487469in}{2.697970in}}%
\pgfpathlineto{\pgfqpoint{2.490603in}{2.697970in}}%
\pgfpathlineto{\pgfqpoint{2.493738in}{2.697970in}}%
\pgfpathlineto{\pgfqpoint{2.496873in}{2.697970in}}%
\pgfpathlineto{\pgfqpoint{2.498597in}{2.700033in}}%
\pgfpathlineto{\pgfqpoint{2.500008in}{2.701721in}}%
\pgfpathlineto{\pgfqpoint{2.503142in}{2.701721in}}%
\pgfpathlineto{\pgfqpoint{2.506277in}{2.701721in}}%
\pgfpathlineto{\pgfqpoint{2.509412in}{2.701721in}}%
\pgfpathlineto{\pgfqpoint{2.512547in}{2.701721in}}%
\pgfpathlineto{\pgfqpoint{2.514271in}{2.703785in}}%
\pgfpathlineto{\pgfqpoint{2.515681in}{2.705473in}}%
\pgfpathlineto{\pgfqpoint{2.518816in}{2.705473in}}%
\pgfpathlineto{\pgfqpoint{2.521951in}{2.705473in}}%
\pgfpathlineto{\pgfqpoint{2.525085in}{2.705473in}}%
\pgfpathlineto{\pgfqpoint{2.526810in}{2.707536in}}%
\pgfpathlineto{\pgfqpoint{2.528220in}{2.709224in}}%
\pgfpathlineto{\pgfqpoint{2.531355in}{2.709224in}}%
\pgfpathlineto{\pgfqpoint{2.534490in}{2.709224in}}%
\pgfpathlineto{\pgfqpoint{2.537624in}{2.709224in}}%
\pgfpathlineto{\pgfqpoint{2.539349in}{2.711288in}}%
\pgfpathlineto{\pgfqpoint{2.540759in}{2.712976in}}%
\pgfpathlineto{\pgfqpoint{2.543894in}{2.712976in}}%
\pgfpathlineto{\pgfqpoint{2.547029in}{2.712976in}}%
\pgfpathlineto{\pgfqpoint{2.550163in}{2.712976in}}%
\pgfpathlineto{\pgfqpoint{2.551888in}{2.715039in}}%
\pgfpathlineto{\pgfqpoint{2.553298in}{2.716727in}}%
\pgfpathlineto{\pgfqpoint{2.556433in}{2.716727in}}%
\pgfpathlineto{\pgfqpoint{2.559568in}{2.716727in}}%
\pgfpathlineto{\pgfqpoint{2.562702in}{2.716727in}}%
\pgfpathlineto{\pgfqpoint{2.565837in}{2.716727in}}%
\pgfpathlineto{\pgfqpoint{2.567561in}{2.718791in}}%
\pgfpathlineto{\pgfqpoint{2.568972in}{2.720479in}}%
\pgfpathlineto{\pgfqpoint{2.572107in}{2.720479in}}%
\pgfpathlineto{\pgfqpoint{2.575241in}{2.720479in}}%
\pgfpathlineto{\pgfqpoint{2.578376in}{2.720479in}}%
\pgfpathlineto{\pgfqpoint{2.580100in}{2.722542in}}%
\pgfpathlineto{\pgfqpoint{2.581511in}{2.724230in}}%
\pgfpathlineto{\pgfqpoint{2.584646in}{2.724230in}}%
\pgfpathlineto{\pgfqpoint{2.587780in}{2.724230in}}%
\pgfpathlineto{\pgfqpoint{2.590915in}{2.724230in}}%
\pgfpathlineto{\pgfqpoint{2.592639in}{2.726294in}}%
\pgfpathlineto{\pgfqpoint{2.594050in}{2.727982in}}%
\pgfpathlineto{\pgfqpoint{2.597185in}{2.727982in}}%
\pgfpathlineto{\pgfqpoint{2.600319in}{2.727982in}}%
\pgfpathlineto{\pgfqpoint{2.603454in}{2.727982in}}%
\pgfpathlineto{\pgfqpoint{2.605178in}{2.730045in}}%
\pgfpathlineto{\pgfqpoint{2.606589in}{2.731733in}}%
\pgfpathlineto{\pgfqpoint{2.609724in}{2.731733in}}%
\pgfpathlineto{\pgfqpoint{2.612858in}{2.731733in}}%
\pgfpathlineto{\pgfqpoint{2.615993in}{2.731733in}}%
\pgfpathlineto{\pgfqpoint{2.617717in}{2.733797in}}%
\pgfpathlineto{\pgfqpoint{2.619128in}{2.735485in}}%
\pgfpathlineto{\pgfqpoint{2.622262in}{2.735485in}}%
\pgfpathlineto{\pgfqpoint{2.625397in}{2.735485in}}%
\pgfpathlineto{\pgfqpoint{2.628532in}{2.735485in}}%
\pgfpathlineto{\pgfqpoint{2.631667in}{2.735485in}}%
\pgfpathlineto{\pgfqpoint{2.633391in}{2.737548in}}%
\pgfpathlineto{\pgfqpoint{2.634801in}{2.739237in}}%
\pgfpathlineto{\pgfqpoint{2.637936in}{2.739237in}}%
\pgfpathlineto{\pgfqpoint{2.641071in}{2.739237in}}%
\pgfpathlineto{\pgfqpoint{2.644206in}{2.739237in}}%
\pgfpathlineto{\pgfqpoint{2.645930in}{2.741300in}}%
\pgfpathlineto{\pgfqpoint{2.647340in}{2.742988in}}%
\pgfpathlineto{\pgfqpoint{2.650475in}{2.742988in}}%
\pgfpathlineto{\pgfqpoint{2.653610in}{2.742988in}}%
\pgfpathlineto{\pgfqpoint{2.656745in}{2.742988in}}%
\pgfpathlineto{\pgfqpoint{2.658469in}{2.745051in}}%
\pgfpathlineto{\pgfqpoint{2.659879in}{2.746740in}}%
\pgfpathlineto{\pgfqpoint{2.663014in}{2.746740in}}%
\pgfpathlineto{\pgfqpoint{2.666149in}{2.746740in}}%
\pgfpathlineto{\pgfqpoint{2.669284in}{2.746740in}}%
\pgfpathlineto{\pgfqpoint{2.672418in}{2.746740in}}%
\pgfpathlineto{\pgfqpoint{2.673829in}{2.745051in}}%
\pgfpathlineto{\pgfqpoint{2.675553in}{2.742988in}}%
\pgfpathlineto{\pgfqpoint{2.676964in}{2.741300in}}%
\pgfpathlineto{\pgfqpoint{2.678688in}{2.739237in}}%
\pgfpathlineto{\pgfqpoint{2.680098in}{2.737548in}}%
\pgfpathlineto{\pgfqpoint{2.681823in}{2.735485in}}%
\pgfpathlineto{\pgfqpoint{2.683233in}{2.733797in}}%
\pgfpathlineto{\pgfqpoint{2.684957in}{2.731733in}}%
\pgfpathlineto{\pgfqpoint{2.686368in}{2.730045in}}%
\pgfpathlineto{\pgfqpoint{2.688092in}{2.727982in}}%
\pgfpathlineto{\pgfqpoint{2.689503in}{2.726294in}}%
\pgfpathlineto{\pgfqpoint{2.691227in}{2.724230in}}%
\pgfpathlineto{\pgfqpoint{2.692637in}{2.722542in}}%
\pgfpathlineto{\pgfqpoint{2.694362in}{2.720479in}}%
\pgfpathlineto{\pgfqpoint{2.695772in}{2.718791in}}%
\pgfpathlineto{\pgfqpoint{2.697496in}{2.716727in}}%
\pgfpathlineto{\pgfqpoint{2.698907in}{2.715039in}}%
\pgfpathlineto{\pgfqpoint{2.700631in}{2.712976in}}%
\pgfpathlineto{\pgfqpoint{2.702042in}{2.711288in}}%
\pgfpathlineto{\pgfqpoint{2.703766in}{2.709224in}}%
\pgfpathlineto{\pgfqpoint{2.705176in}{2.707536in}}%
\pgfpathlineto{\pgfqpoint{2.706901in}{2.705473in}}%
\pgfpathlineto{\pgfqpoint{2.708311in}{2.703785in}}%
\pgfpathlineto{\pgfqpoint{2.710035in}{2.701721in}}%
\pgfpathlineto{\pgfqpoint{2.711446in}{2.700033in}}%
\pgfpathlineto{\pgfqpoint{2.711446in}{2.696281in}}%
\pgfpathlineto{\pgfqpoint{2.713170in}{2.694218in}}%
\pgfpathlineto{\pgfqpoint{2.714581in}{2.692530in}}%
\pgfpathlineto{\pgfqpoint{2.716305in}{2.690467in}}%
\pgfpathlineto{\pgfqpoint{2.717715in}{2.688778in}}%
\pgfpathlineto{\pgfqpoint{2.719439in}{2.686715in}}%
\pgfpathlineto{\pgfqpoint{2.720850in}{2.685027in}}%
\pgfpathlineto{\pgfqpoint{2.722574in}{2.682964in}}%
\pgfpathlineto{\pgfqpoint{2.723985in}{2.681275in}}%
\pgfpathlineto{\pgfqpoint{2.725709in}{2.679212in}}%
\pgfpathlineto{\pgfqpoint{2.727120in}{2.677524in}}%
\pgfpathlineto{\pgfqpoint{2.728844in}{2.675460in}}%
\pgfpathlineto{\pgfqpoint{2.730254in}{2.673772in}}%
\pgfpathlineto{\pgfqpoint{2.731978in}{2.671709in}}%
\pgfpathlineto{\pgfqpoint{2.733389in}{2.670021in}}%
\pgfpathlineto{\pgfqpoint{2.735113in}{2.667957in}}%
\pgfpathlineto{\pgfqpoint{2.736524in}{2.666269in}}%
\pgfpathlineto{\pgfqpoint{2.738248in}{2.664206in}}%
\pgfpathlineto{\pgfqpoint{2.739659in}{2.662518in}}%
\pgfpathlineto{\pgfqpoint{2.741383in}{2.660454in}}%
\pgfpathlineto{\pgfqpoint{2.742793in}{2.658766in}}%
\pgfpathlineto{\pgfqpoint{2.744517in}{2.656703in}}%
\pgfpathlineto{\pgfqpoint{2.745928in}{2.655015in}}%
\pgfpathlineto{\pgfqpoint{2.747652in}{2.652951in}}%
\pgfpathlineto{\pgfqpoint{2.749063in}{2.651263in}}%
\pgfpathlineto{\pgfqpoint{2.750787in}{2.649200in}}%
\pgfpathlineto{\pgfqpoint{2.752198in}{2.647512in}}%
\pgfpathlineto{\pgfqpoint{2.753922in}{2.645448in}}%
\pgfpathlineto{\pgfqpoint{2.755332in}{2.643760in}}%
\pgfpathlineto{\pgfqpoint{2.757056in}{2.641697in}}%
\pgfpathlineto{\pgfqpoint{2.758467in}{2.640008in}}%
\pgfpathlineto{\pgfqpoint{2.760191in}{2.637945in}}%
\pgfpathlineto{\pgfqpoint{2.761602in}{2.636257in}}%
\pgfpathlineto{\pgfqpoint{2.761602in}{2.632505in}}%
\pgfpathlineto{\pgfqpoint{2.763326in}{2.630442in}}%
\pgfpathlineto{\pgfqpoint{2.764737in}{2.628754in}}%
\pgfpathlineto{\pgfqpoint{2.766461in}{2.626690in}}%
\pgfpathlineto{\pgfqpoint{2.767871in}{2.625002in}}%
\pgfpathlineto{\pgfqpoint{2.769595in}{2.622939in}}%
\pgfpathlineto{\pgfqpoint{2.771006in}{2.621251in}}%
\pgfpathlineto{\pgfqpoint{2.772730in}{2.619187in}}%
\pgfpathlineto{\pgfqpoint{2.774141in}{2.617499in}}%
\pgfpathlineto{\pgfqpoint{2.775865in}{2.615436in}}%
\pgfpathlineto{\pgfqpoint{2.777275in}{2.613748in}}%
\pgfpathlineto{\pgfqpoint{2.779000in}{2.611684in}}%
\pgfpathlineto{\pgfqpoint{2.780410in}{2.609996in}}%
\pgfpathlineto{\pgfqpoint{2.782134in}{2.607933in}}%
\pgfpathlineto{\pgfqpoint{2.783545in}{2.606245in}}%
\pgfpathlineto{\pgfqpoint{2.785269in}{2.604181in}}%
\pgfpathlineto{\pgfqpoint{2.786680in}{2.602493in}}%
\pgfpathlineto{\pgfqpoint{2.788404in}{2.600430in}}%
\pgfpathlineto{\pgfqpoint{2.789814in}{2.598742in}}%
\pgfpathlineto{\pgfqpoint{2.791539in}{2.596678in}}%
\pgfpathlineto{\pgfqpoint{2.792949in}{2.594990in}}%
\pgfpathlineto{\pgfqpoint{2.794673in}{2.592927in}}%
\pgfpathlineto{\pgfqpoint{2.796084in}{2.591238in}}%
\pgfpathlineto{\pgfqpoint{2.797808in}{2.589175in}}%
\pgfpathlineto{\pgfqpoint{2.799219in}{2.587487in}}%
\pgfpathlineto{\pgfqpoint{2.800943in}{2.585424in}}%
\pgfpathlineto{\pgfqpoint{2.802353in}{2.583735in}}%
\pgfpathlineto{\pgfqpoint{2.804078in}{2.581672in}}%
\pgfpathlineto{\pgfqpoint{2.805488in}{2.579984in}}%
\pgfpathlineto{\pgfqpoint{2.807212in}{2.577921in}}%
\pgfpathlineto{\pgfqpoint{2.808623in}{2.576232in}}%
\pgfpathlineto{\pgfqpoint{2.808623in}{2.572481in}}%
\pgfpathlineto{\pgfqpoint{2.810347in}{2.570417in}}%
\pgfpathlineto{\pgfqpoint{2.811758in}{2.568729in}}%
\pgfpathlineto{\pgfqpoint{2.813482in}{2.566666in}}%
\pgfpathlineto{\pgfqpoint{2.814892in}{2.564978in}}%
\pgfpathlineto{\pgfqpoint{2.816617in}{2.562914in}}%
\pgfpathlineto{\pgfqpoint{2.818027in}{2.561226in}}%
\pgfpathlineto{\pgfqpoint{2.819751in}{2.559163in}}%
\pgfpathlineto{\pgfqpoint{2.821162in}{2.557475in}}%
\pgfpathlineto{\pgfqpoint{2.822886in}{2.555411in}}%
\pgfpathlineto{\pgfqpoint{2.824297in}{2.553723in}}%
\pgfpathlineto{\pgfqpoint{2.826021in}{2.551660in}}%
\pgfpathlineto{\pgfqpoint{2.827431in}{2.549972in}}%
\pgfpathlineto{\pgfqpoint{2.829155in}{2.547908in}}%
\pgfpathlineto{\pgfqpoint{2.830566in}{2.546220in}}%
\pgfpathlineto{\pgfqpoint{2.832290in}{2.544157in}}%
\pgfpathlineto{\pgfqpoint{2.833701in}{2.542469in}}%
\pgfpathlineto{\pgfqpoint{2.835425in}{2.540405in}}%
\pgfpathlineto{\pgfqpoint{2.836836in}{2.538717in}}%
\pgfpathlineto{\pgfqpoint{2.838560in}{2.536654in}}%
\pgfpathlineto{\pgfqpoint{2.839970in}{2.534965in}}%
\pgfpathlineto{\pgfqpoint{2.841694in}{2.532902in}}%
\pgfpathlineto{\pgfqpoint{2.843105in}{2.531214in}}%
\pgfpathlineto{\pgfqpoint{2.844829in}{2.529151in}}%
\pgfpathlineto{\pgfqpoint{2.846240in}{2.527462in}}%
\pgfpathlineto{\pgfqpoint{2.847964in}{2.525399in}}%
\pgfpathlineto{\pgfqpoint{2.849375in}{2.523711in}}%
\pgfpathlineto{\pgfqpoint{2.851099in}{2.521648in}}%
\pgfpathlineto{\pgfqpoint{2.852509in}{2.519959in}}%
\pgfpathlineto{\pgfqpoint{2.854233in}{2.517896in}}%
\pgfpathlineto{\pgfqpoint{2.855644in}{2.516208in}}%
\pgfpathlineto{\pgfqpoint{2.857368in}{2.514144in}}%
\pgfpathlineto{\pgfqpoint{2.858779in}{2.512456in}}%
\pgfpathlineto{\pgfqpoint{2.858779in}{2.508705in}}%
\pgfpathlineto{\pgfqpoint{2.860503in}{2.506641in}}%
\pgfpathlineto{\pgfqpoint{2.861914in}{2.504953in}}%
\pgfpathlineto{\pgfqpoint{2.863638in}{2.502890in}}%
\pgfpathlineto{\pgfqpoint{2.865048in}{2.501202in}}%
\pgfpathlineto{\pgfqpoint{2.866772in}{2.499138in}}%
\pgfpathlineto{\pgfqpoint{2.868183in}{2.497450in}}%
\pgfpathlineto{\pgfqpoint{2.869907in}{2.495387in}}%
\pgfpathlineto{\pgfqpoint{2.871318in}{2.493699in}}%
\pgfpathlineto{\pgfqpoint{2.873042in}{2.491635in}}%
\pgfpathlineto{\pgfqpoint{2.874453in}{2.489947in}}%
\pgfpathlineto{\pgfqpoint{2.876177in}{2.487884in}}%
\pgfpathlineto{\pgfqpoint{2.877587in}{2.486196in}}%
\pgfpathlineto{\pgfqpoint{2.879311in}{2.484132in}}%
\pgfpathlineto{\pgfqpoint{2.880722in}{2.482444in}}%
\pgfpathlineto{\pgfqpoint{2.882446in}{2.480381in}}%
\pgfpathlineto{\pgfqpoint{2.883857in}{2.478692in}}%
\pgfpathlineto{\pgfqpoint{2.885581in}{2.476629in}}%
\pgfpathlineto{\pgfqpoint{2.886991in}{2.474941in}}%
\pgfpathlineto{\pgfqpoint{2.888716in}{2.472878in}}%
\pgfpathlineto{\pgfqpoint{2.890126in}{2.471189in}}%
\pgfpathlineto{\pgfqpoint{2.891850in}{2.469126in}}%
\pgfpathlineto{\pgfqpoint{2.893261in}{2.467438in}}%
\pgfpathlineto{\pgfqpoint{2.894985in}{2.465375in}}%
\pgfpathlineto{\pgfqpoint{2.896396in}{2.463686in}}%
\pgfpathlineto{\pgfqpoint{2.898120in}{2.461623in}}%
\pgfpathlineto{\pgfqpoint{2.899530in}{2.459935in}}%
\pgfpathlineto{\pgfqpoint{2.901255in}{2.457871in}}%
\pgfpathlineto{\pgfqpoint{2.902665in}{2.456183in}}%
\pgfpathlineto{\pgfqpoint{2.904389in}{2.454120in}}%
\pgfpathlineto{\pgfqpoint{2.905800in}{2.452432in}}%
\pgfpathlineto{\pgfqpoint{2.905800in}{2.448680in}}%
\pgfpathlineto{\pgfqpoint{2.907524in}{2.446617in}}%
\pgfpathlineto{\pgfqpoint{2.908935in}{2.444929in}}%
\pgfpathlineto{\pgfqpoint{2.910659in}{2.442865in}}%
\pgfpathlineto{\pgfqpoint{2.912069in}{2.441177in}}%
\pgfpathlineto{\pgfqpoint{2.913794in}{2.439114in}}%
\pgfpathlineto{\pgfqpoint{2.915204in}{2.437426in}}%
\pgfpathlineto{\pgfqpoint{2.916928in}{2.435362in}}%
\pgfpathlineto{\pgfqpoint{2.918339in}{2.433674in}}%
\pgfpathlineto{\pgfqpoint{2.920063in}{2.431611in}}%
\pgfpathlineto{\pgfqpoint{2.921474in}{2.429923in}}%
\pgfpathlineto{\pgfqpoint{2.923198in}{2.427859in}}%
\pgfpathlineto{\pgfqpoint{2.924608in}{2.426171in}}%
\pgfpathlineto{\pgfqpoint{2.926332in}{2.424108in}}%
\pgfpathlineto{\pgfqpoint{2.927743in}{2.422419in}}%
\pgfpathlineto{\pgfqpoint{2.929467in}{2.420356in}}%
\pgfpathlineto{\pgfqpoint{2.930878in}{2.418668in}}%
\pgfpathlineto{\pgfqpoint{2.932602in}{2.416605in}}%
\pgfpathlineto{\pgfqpoint{2.934013in}{2.414916in}}%
\pgfpathlineto{\pgfqpoint{2.935737in}{2.412853in}}%
\pgfpathlineto{\pgfqpoint{2.937147in}{2.411165in}}%
\pgfpathlineto{\pgfqpoint{2.938871in}{2.409101in}}%
\pgfpathlineto{\pgfqpoint{2.940282in}{2.407413in}}%
\pgfpathlineto{\pgfqpoint{2.942006in}{2.405350in}}%
\pgfpathlineto{\pgfqpoint{2.943417in}{2.403662in}}%
\pgfpathlineto{\pgfqpoint{2.945141in}{2.401598in}}%
\pgfpathlineto{\pgfqpoint{2.946552in}{2.399910in}}%
\pgfpathlineto{\pgfqpoint{2.948276in}{2.397847in}}%
\pgfpathlineto{\pgfqpoint{2.949686in}{2.396159in}}%
\pgfpathlineto{\pgfqpoint{2.951410in}{2.394095in}}%
\pgfpathlineto{\pgfqpoint{2.952821in}{2.392407in}}%
\pgfpathlineto{\pgfqpoint{2.954545in}{2.390344in}}%
\pgfpathlineto{\pgfqpoint{2.955956in}{2.388656in}}%
\pgfpathlineto{\pgfqpoint{2.955956in}{2.384904in}}%
\pgfpathlineto{\pgfqpoint{2.957680in}{2.382841in}}%
\pgfpathlineto{\pgfqpoint{2.959091in}{2.381153in}}%
\pgfpathlineto{\pgfqpoint{2.960815in}{2.379089in}}%
\pgfpathlineto{\pgfqpoint{2.962225in}{2.377401in}}%
\pgfpathlineto{\pgfqpoint{2.963949in}{2.375338in}}%
\pgfpathlineto{\pgfqpoint{2.965360in}{2.373649in}}%
\pgfpathlineto{\pgfqpoint{2.967084in}{2.371586in}}%
\pgfpathlineto{\pgfqpoint{2.968495in}{2.369898in}}%
\pgfpathlineto{\pgfqpoint{2.970219in}{2.367835in}}%
\pgfpathlineto{\pgfqpoint{2.971630in}{2.366146in}}%
\pgfpathlineto{\pgfqpoint{2.973354in}{2.364083in}}%
\pgfpathlineto{\pgfqpoint{2.974764in}{2.362395in}}%
\pgfpathlineto{\pgfqpoint{2.976488in}{2.360332in}}%
\pgfpathlineto{\pgfqpoint{2.977899in}{2.358643in}}%
\pgfpathlineto{\pgfqpoint{2.979623in}{2.356580in}}%
\pgfpathlineto{\pgfqpoint{2.981034in}{2.354892in}}%
\pgfpathlineto{\pgfqpoint{2.982758in}{2.352828in}}%
\pgfpathlineto{\pgfqpoint{2.984168in}{2.351140in}}%
\pgfpathlineto{\pgfqpoint{2.985893in}{2.349077in}}%
\pgfpathlineto{\pgfqpoint{2.987303in}{2.347389in}}%
\pgfpathlineto{\pgfqpoint{2.989027in}{2.345325in}}%
\pgfpathlineto{\pgfqpoint{2.990438in}{2.343637in}}%
\pgfpathlineto{\pgfqpoint{2.992162in}{2.341574in}}%
\pgfpathlineto{\pgfqpoint{2.993573in}{2.339886in}}%
\pgfpathlineto{\pgfqpoint{2.995297in}{2.337822in}}%
\pgfpathlineto{\pgfqpoint{2.996707in}{2.336134in}}%
\pgfpathlineto{\pgfqpoint{2.998432in}{2.334071in}}%
\pgfpathlineto{\pgfqpoint{2.999842in}{2.332383in}}%
\pgfpathlineto{\pgfqpoint{3.001566in}{2.330319in}}%
\pgfpathlineto{\pgfqpoint{3.002977in}{2.328631in}}%
\pgfpathlineto{\pgfqpoint{3.002977in}{2.324880in}}%
\pgfpathlineto{\pgfqpoint{3.004701in}{2.322816in}}%
\pgfpathlineto{\pgfqpoint{3.006112in}{2.321128in}}%
\pgfpathlineto{\pgfqpoint{3.007836in}{2.319065in}}%
\pgfpathlineto{\pgfqpoint{3.009246in}{2.317376in}}%
\pgfpathlineto{\pgfqpoint{3.010971in}{2.315313in}}%
\pgfpathlineto{\pgfqpoint{3.012381in}{2.313625in}}%
\pgfpathlineto{\pgfqpoint{3.014105in}{2.311562in}}%
\pgfpathlineto{\pgfqpoint{3.015516in}{2.309873in}}%
\pgfpathlineto{\pgfqpoint{3.017240in}{2.307810in}}%
\pgfpathlineto{\pgfqpoint{3.018651in}{2.306122in}}%
\pgfpathlineto{\pgfqpoint{3.020375in}{2.304059in}}%
\pgfpathlineto{\pgfqpoint{3.021785in}{2.302370in}}%
\pgfpathlineto{\pgfqpoint{3.023510in}{2.300307in}}%
\pgfpathlineto{\pgfqpoint{3.024920in}{2.298619in}}%
\pgfpathlineto{\pgfqpoint{3.026644in}{2.296555in}}%
\pgfpathlineto{\pgfqpoint{3.028055in}{2.294867in}}%
\pgfpathlineto{\pgfqpoint{3.029779in}{2.292804in}}%
\pgfpathlineto{\pgfqpoint{3.031190in}{2.291116in}}%
\pgfpathlineto{\pgfqpoint{3.032914in}{2.289052in}}%
\pgfpathlineto{\pgfqpoint{3.034324in}{2.287364in}}%
\pgfpathlineto{\pgfqpoint{3.036048in}{2.285301in}}%
\pgfpathlineto{\pgfqpoint{3.037459in}{2.283613in}}%
\pgfpathlineto{\pgfqpoint{3.039183in}{2.281549in}}%
\pgfpathlineto{\pgfqpoint{3.040594in}{2.279861in}}%
\pgfpathlineto{\pgfqpoint{3.042318in}{2.277798in}}%
\pgfpathlineto{\pgfqpoint{3.043729in}{2.276110in}}%
\pgfpathlineto{\pgfqpoint{3.045453in}{2.274046in}}%
\pgfpathlineto{\pgfqpoint{3.046863in}{2.272358in}}%
\pgfpathlineto{\pgfqpoint{3.048587in}{2.270295in}}%
\pgfpathlineto{\pgfqpoint{3.049998in}{2.268607in}}%
\pgfpathlineto{\pgfqpoint{3.051722in}{2.266543in}}%
\pgfpathlineto{\pgfqpoint{3.053133in}{2.264855in}}%
\pgfpathlineto{\pgfqpoint{3.053133in}{2.261103in}}%
\pgfpathlineto{\pgfqpoint{3.054857in}{2.259040in}}%
\pgfpathlineto{\pgfqpoint{3.056268in}{2.257352in}}%
\pgfpathlineto{\pgfqpoint{3.057992in}{2.255289in}}%
\pgfpathlineto{\pgfqpoint{3.059402in}{2.253600in}}%
\pgfpathlineto{\pgfqpoint{3.061126in}{2.251537in}}%
\pgfpathlineto{\pgfqpoint{3.062537in}{2.249849in}}%
\pgfpathlineto{\pgfqpoint{3.064261in}{2.247786in}}%
\pgfpathlineto{\pgfqpoint{3.065672in}{2.246097in}}%
\pgfpathlineto{\pgfqpoint{3.067396in}{2.244034in}}%
\pgfpathlineto{\pgfqpoint{3.068807in}{2.242346in}}%
\pgfpathlineto{\pgfqpoint{3.070531in}{2.240282in}}%
\pgfpathlineto{\pgfqpoint{3.071941in}{2.238594in}}%
\pgfpathlineto{\pgfqpoint{3.073665in}{2.236531in}}%
\pgfpathlineto{\pgfqpoint{3.075076in}{2.234843in}}%
\pgfpathlineto{\pgfqpoint{3.076800in}{2.232779in}}%
\pgfpathlineto{\pgfqpoint{3.078211in}{2.231091in}}%
\pgfpathlineto{\pgfqpoint{3.079935in}{2.229028in}}%
\pgfpathlineto{\pgfqpoint{3.081345in}{2.227340in}}%
\pgfpathlineto{\pgfqpoint{3.083070in}{2.225276in}}%
\pgfpathlineto{\pgfqpoint{3.084480in}{2.223588in}}%
\pgfpathlineto{\pgfqpoint{3.086204in}{2.221525in}}%
\pgfpathlineto{\pgfqpoint{3.087615in}{2.219837in}}%
\pgfpathlineto{\pgfqpoint{3.089339in}{2.217773in}}%
\pgfpathlineto{\pgfqpoint{3.090750in}{2.216085in}}%
\pgfpathlineto{\pgfqpoint{3.092474in}{2.214022in}}%
\pgfpathlineto{\pgfqpoint{3.093884in}{2.212334in}}%
\pgfpathlineto{\pgfqpoint{3.093884in}{2.208582in}}%
\pgfpathlineto{\pgfqpoint{3.092474in}{2.206894in}}%
\pgfpathlineto{\pgfqpoint{3.090750in}{2.204830in}}%
\pgfpathlineto{\pgfqpoint{3.090750in}{2.201079in}}%
\pgfpathlineto{\pgfqpoint{3.090750in}{2.197327in}}%
\pgfpathlineto{\pgfqpoint{3.089339in}{2.195639in}}%
\pgfpathlineto{\pgfqpoint{3.087615in}{2.193576in}}%
\pgfpathlineto{\pgfqpoint{3.087615in}{2.189824in}}%
\pgfpathlineto{\pgfqpoint{3.087615in}{2.186073in}}%
\pgfpathlineto{\pgfqpoint{3.086204in}{2.184385in}}%
\pgfpathlineto{\pgfqpoint{3.084480in}{2.182321in}}%
\pgfpathlineto{\pgfqpoint{3.084480in}{2.178570in}}%
\pgfpathlineto{\pgfqpoint{3.084480in}{2.174818in}}%
\pgfpathlineto{\pgfqpoint{3.083070in}{2.173130in}}%
\pgfpathlineto{\pgfqpoint{3.081345in}{2.171067in}}%
\pgfpathlineto{\pgfqpoint{3.081345in}{2.167315in}}%
\pgfpathlineto{\pgfqpoint{3.081345in}{2.163564in}}%
\pgfpathlineto{\pgfqpoint{3.079935in}{2.161875in}}%
\pgfpathlineto{\pgfqpoint{3.078211in}{2.159812in}}%
\pgfpathlineto{\pgfqpoint{3.078211in}{2.156060in}}%
\pgfpathlineto{\pgfqpoint{3.078211in}{2.152309in}}%
\pgfpathlineto{\pgfqpoint{3.078211in}{2.148557in}}%
\pgfpathlineto{\pgfqpoint{3.076800in}{2.146869in}}%
\pgfpathlineto{\pgfqpoint{3.075076in}{2.144806in}}%
\pgfpathlineto{\pgfqpoint{3.075076in}{2.141054in}}%
\pgfpathlineto{\pgfqpoint{3.075076in}{2.137303in}}%
\pgfpathlineto{\pgfqpoint{3.073665in}{2.135615in}}%
\pgfpathlineto{\pgfqpoint{3.071941in}{2.133551in}}%
\pgfpathlineto{\pgfqpoint{3.071941in}{2.129800in}}%
\pgfpathlineto{\pgfqpoint{3.071941in}{2.126048in}}%
\pgfpathlineto{\pgfqpoint{3.070531in}{2.124360in}}%
\pgfpathlineto{\pgfqpoint{3.068807in}{2.122297in}}%
\pgfpathlineto{\pgfqpoint{3.068807in}{2.118545in}}%
\pgfpathlineto{\pgfqpoint{3.068807in}{2.114794in}}%
\pgfpathlineto{\pgfqpoint{3.067396in}{2.113105in}}%
\pgfpathlineto{\pgfqpoint{3.065672in}{2.111042in}}%
\pgfpathlineto{\pgfqpoint{3.065672in}{2.107291in}}%
\pgfpathlineto{\pgfqpoint{3.065672in}{2.103539in}}%
\pgfpathlineto{\pgfqpoint{3.064261in}{2.101851in}}%
\pgfpathlineto{\pgfqpoint{3.062537in}{2.099787in}}%
\pgfpathlineto{\pgfqpoint{3.062537in}{2.096036in}}%
\pgfpathlineto{\pgfqpoint{3.062537in}{2.092284in}}%
\pgfpathlineto{\pgfqpoint{3.061126in}{2.090596in}}%
\pgfpathlineto{\pgfqpoint{3.059402in}{2.088533in}}%
\pgfpathlineto{\pgfqpoint{3.059402in}{2.084781in}}%
\pgfpathlineto{\pgfqpoint{3.059402in}{2.081030in}}%
\pgfpathlineto{\pgfqpoint{3.057992in}{2.079342in}}%
\pgfpathlineto{\pgfqpoint{3.056268in}{2.077278in}}%
\pgfpathlineto{\pgfqpoint{3.056268in}{2.073527in}}%
\pgfpathlineto{\pgfqpoint{3.056268in}{2.069775in}}%
\pgfpathlineto{\pgfqpoint{3.054857in}{2.068087in}}%
\pgfpathlineto{\pgfqpoint{3.053133in}{2.066024in}}%
\pgfpathlineto{\pgfqpoint{3.053133in}{2.062272in}}%
\pgfpathlineto{\pgfqpoint{3.053133in}{2.058521in}}%
\pgfpathlineto{\pgfqpoint{3.051722in}{2.056832in}}%
\pgfpathlineto{\pgfqpoint{3.049998in}{2.054769in}}%
\pgfpathlineto{\pgfqpoint{3.049998in}{2.051018in}}%
\pgfpathlineto{\pgfqpoint{3.049998in}{2.047266in}}%
\pgfpathlineto{\pgfqpoint{3.049998in}{2.043514in}}%
\pgfpathlineto{\pgfqpoint{3.048587in}{2.041826in}}%
\pgfpathlineto{\pgfqpoint{3.046863in}{2.039763in}}%
\pgfpathlineto{\pgfqpoint{3.046863in}{2.036011in}}%
\pgfpathlineto{\pgfqpoint{3.046863in}{2.032260in}}%
\pgfpathlineto{\pgfqpoint{3.045453in}{2.030572in}}%
\pgfpathlineto{\pgfqpoint{3.043729in}{2.028508in}}%
\pgfpathlineto{\pgfqpoint{3.043729in}{2.024757in}}%
\pgfpathlineto{\pgfqpoint{3.043729in}{2.021005in}}%
\pgfpathlineto{\pgfqpoint{3.042318in}{2.019317in}}%
\pgfpathlineto{\pgfqpoint{3.040594in}{2.017254in}}%
\pgfpathlineto{\pgfqpoint{3.040594in}{2.013502in}}%
\pgfpathlineto{\pgfqpoint{3.040594in}{2.009751in}}%
\pgfpathlineto{\pgfqpoint{3.039183in}{2.008062in}}%
\pgfpathlineto{\pgfqpoint{3.037459in}{2.005999in}}%
\pgfpathlineto{\pgfqpoint{3.037459in}{2.002248in}}%
\pgfpathlineto{\pgfqpoint{3.037459in}{1.998496in}}%
\pgfpathlineto{\pgfqpoint{3.036048in}{1.996808in}}%
\pgfpathlineto{\pgfqpoint{3.034324in}{1.994745in}}%
\pgfpathlineto{\pgfqpoint{3.034324in}{1.990993in}}%
\pgfpathlineto{\pgfqpoint{3.034324in}{1.987241in}}%
\pgfpathlineto{\pgfqpoint{3.032914in}{1.985553in}}%
\pgfpathlineto{\pgfqpoint{3.031190in}{1.983490in}}%
\pgfpathlineto{\pgfqpoint{3.031190in}{1.979738in}}%
\pgfpathlineto{\pgfqpoint{3.031190in}{1.975987in}}%
\pgfpathlineto{\pgfqpoint{3.029779in}{1.974299in}}%
\pgfpathlineto{\pgfqpoint{3.028055in}{1.972235in}}%
\pgfpathlineto{\pgfqpoint{3.028055in}{1.968484in}}%
\pgfpathlineto{\pgfqpoint{3.028055in}{1.964732in}}%
\pgfpathlineto{\pgfqpoint{3.026644in}{1.963044in}}%
\pgfpathlineto{\pgfqpoint{3.024920in}{1.960981in}}%
\pgfpathlineto{\pgfqpoint{3.024920in}{1.957229in}}%
\pgfpathlineto{\pgfqpoint{3.024920in}{1.953478in}}%
\pgfpathlineto{\pgfqpoint{3.024920in}{1.949726in}}%
\pgfpathlineto{\pgfqpoint{3.023510in}{1.948038in}}%
\pgfpathlineto{\pgfqpoint{3.021785in}{1.945975in}}%
\pgfpathlineto{\pgfqpoint{3.021785in}{1.942223in}}%
\pgfpathlineto{\pgfqpoint{3.021785in}{1.938471in}}%
\pgfpathlineto{\pgfqpoint{3.020375in}{1.936783in}}%
\pgfpathlineto{\pgfqpoint{3.018651in}{1.934720in}}%
\pgfpathlineto{\pgfqpoint{3.018651in}{1.930968in}}%
\pgfpathlineto{\pgfqpoint{3.018651in}{1.927217in}}%
\pgfpathlineto{\pgfqpoint{3.017240in}{1.925529in}}%
\pgfpathlineto{\pgfqpoint{3.015516in}{1.923465in}}%
\pgfpathlineto{\pgfqpoint{3.015516in}{1.919714in}}%
\pgfpathlineto{\pgfqpoint{3.015516in}{1.915962in}}%
\pgfpathlineto{\pgfqpoint{3.014105in}{1.914274in}}%
\pgfpathlineto{\pgfqpoint{3.012381in}{1.912211in}}%
\pgfpathlineto{\pgfqpoint{3.012381in}{1.908459in}}%
\pgfpathlineto{\pgfqpoint{3.012381in}{1.904708in}}%
\pgfpathlineto{\pgfqpoint{3.010971in}{1.903019in}}%
\pgfpathlineto{\pgfqpoint{3.009246in}{1.900956in}}%
\pgfpathlineto{\pgfqpoint{3.009246in}{1.897205in}}%
\pgfpathlineto{\pgfqpoint{3.009246in}{1.893453in}}%
\pgfpathlineto{\pgfqpoint{3.007836in}{1.891765in}}%
\pgfpathlineto{\pgfqpoint{3.006112in}{1.889702in}}%
\pgfpathlineto{\pgfqpoint{3.006112in}{1.885950in}}%
\pgfpathlineto{\pgfqpoint{3.006112in}{1.882198in}}%
\pgfpathlineto{\pgfqpoint{3.004701in}{1.880510in}}%
\pgfpathlineto{\pgfqpoint{3.002977in}{1.878447in}}%
\pgfpathlineto{\pgfqpoint{3.002977in}{1.874695in}}%
\pgfpathlineto{\pgfqpoint{3.002977in}{1.870944in}}%
\pgfpathlineto{\pgfqpoint{3.001566in}{1.869256in}}%
\pgfpathlineto{\pgfqpoint{2.999842in}{1.867192in}}%
\pgfpathlineto{\pgfqpoint{2.999842in}{1.863441in}}%
\pgfpathlineto{\pgfqpoint{2.999842in}{1.859689in}}%
\pgfpathlineto{\pgfqpoint{2.998432in}{1.858001in}}%
\pgfpathlineto{\pgfqpoint{2.996707in}{1.855938in}}%
\pgfpathlineto{\pgfqpoint{2.996707in}{1.852186in}}%
\pgfpathlineto{\pgfqpoint{2.996707in}{1.848435in}}%
\pgfpathlineto{\pgfqpoint{2.996707in}{1.844683in}}%
\pgfpathlineto{\pgfqpoint{2.995297in}{1.842995in}}%
\pgfpathlineto{\pgfqpoint{2.993573in}{1.840932in}}%
\pgfpathlineto{\pgfqpoint{2.993573in}{1.837180in}}%
\pgfpathlineto{\pgfqpoint{2.993573in}{1.833429in}}%
\pgfpathlineto{\pgfqpoint{2.992162in}{1.831740in}}%
\pgfpathlineto{\pgfqpoint{2.990438in}{1.829677in}}%
\pgfpathlineto{\pgfqpoint{2.990438in}{1.825925in}}%
\pgfpathlineto{\pgfqpoint{2.990438in}{1.822174in}}%
\pgfpathlineto{\pgfqpoint{2.989027in}{1.820486in}}%
\pgfpathlineto{\pgfqpoint{2.987303in}{1.818422in}}%
\pgfpathlineto{\pgfqpoint{2.987303in}{1.814671in}}%
\pgfpathlineto{\pgfqpoint{2.987303in}{1.810919in}}%
\pgfpathlineto{\pgfqpoint{2.985893in}{1.809231in}}%
\pgfpathlineto{\pgfqpoint{2.984168in}{1.807168in}}%
\pgfpathlineto{\pgfqpoint{2.984168in}{1.803416in}}%
\pgfpathlineto{\pgfqpoint{2.984168in}{1.799665in}}%
\pgfpathlineto{\pgfqpoint{2.982758in}{1.797977in}}%
\pgfpathlineto{\pgfqpoint{2.981034in}{1.795913in}}%
\pgfpathlineto{\pgfqpoint{2.981034in}{1.792162in}}%
\pgfpathlineto{\pgfqpoint{2.981034in}{1.788410in}}%
\pgfpathlineto{\pgfqpoint{2.979623in}{1.786722in}}%
\pgfpathlineto{\pgfqpoint{2.977899in}{1.784659in}}%
\pgfpathlineto{\pgfqpoint{2.977899in}{1.780907in}}%
\pgfpathlineto{\pgfqpoint{2.977899in}{1.777155in}}%
\pgfpathlineto{\pgfqpoint{2.976488in}{1.775467in}}%
\pgfpathlineto{\pgfqpoint{2.974764in}{1.773404in}}%
\pgfpathlineto{\pgfqpoint{2.974764in}{1.769652in}}%
\pgfpathlineto{\pgfqpoint{2.974764in}{1.765901in}}%
\pgfpathlineto{\pgfqpoint{2.973354in}{1.764213in}}%
\pgfpathlineto{\pgfqpoint{2.971630in}{1.762149in}}%
\pgfpathlineto{\pgfqpoint{2.971630in}{1.758398in}}%
\pgfpathlineto{\pgfqpoint{2.971630in}{1.754646in}}%
\pgfpathlineto{\pgfqpoint{2.971630in}{1.750895in}}%
\pgfpathlineto{\pgfqpoint{2.970219in}{1.749207in}}%
\pgfpathlineto{\pgfqpoint{2.968495in}{1.747143in}}%
\pgfpathlineto{\pgfqpoint{2.968495in}{1.743392in}}%
\pgfpathlineto{\pgfqpoint{2.968495in}{1.739640in}}%
\pgfpathlineto{\pgfqpoint{2.967084in}{1.737952in}}%
\pgfpathlineto{\pgfqpoint{2.965360in}{1.735889in}}%
\pgfpathlineto{\pgfqpoint{2.965360in}{1.732137in}}%
\pgfpathlineto{\pgfqpoint{2.965360in}{1.728386in}}%
\pgfpathlineto{\pgfqpoint{2.963949in}{1.726697in}}%
\pgfpathlineto{\pgfqpoint{2.962225in}{1.724634in}}%
\pgfpathlineto{\pgfqpoint{2.962225in}{1.720882in}}%
\pgfpathlineto{\pgfqpoint{2.962225in}{1.717131in}}%
\pgfpathlineto{\pgfqpoint{2.960815in}{1.715443in}}%
\pgfpathlineto{\pgfqpoint{2.959091in}{1.713379in}}%
\pgfpathlineto{\pgfqpoint{2.959091in}{1.709628in}}%
\pgfpathlineto{\pgfqpoint{2.959091in}{1.705876in}}%
\pgfpathlineto{\pgfqpoint{2.957680in}{1.704188in}}%
\pgfpathlineto{\pgfqpoint{2.955956in}{1.702125in}}%
\pgfpathlineto{\pgfqpoint{2.955956in}{1.698373in}}%
\pgfpathlineto{\pgfqpoint{2.955956in}{1.694622in}}%
\pgfpathlineto{\pgfqpoint{2.954545in}{1.692934in}}%
\pgfpathlineto{\pgfqpoint{2.952821in}{1.690870in}}%
\pgfpathlineto{\pgfqpoint{2.952821in}{1.687119in}}%
\pgfpathlineto{\pgfqpoint{2.952821in}{1.683367in}}%
\pgfpathlineto{\pgfqpoint{2.951410in}{1.681679in}}%
\pgfpathlineto{\pgfqpoint{2.949686in}{1.679616in}}%
\pgfpathlineto{\pgfqpoint{2.949686in}{1.675864in}}%
\pgfpathlineto{\pgfqpoint{2.949686in}{1.672113in}}%
\pgfpathlineto{\pgfqpoint{2.948276in}{1.670424in}}%
\pgfpathlineto{\pgfqpoint{2.946552in}{1.668361in}}%
\pgfpathlineto{\pgfqpoint{2.946552in}{1.664609in}}%
\pgfpathlineto{\pgfqpoint{2.946552in}{1.660858in}}%
\pgfpathlineto{\pgfqpoint{2.945141in}{1.659170in}}%
\pgfpathlineto{\pgfqpoint{2.943417in}{1.657106in}}%
\pgfpathlineto{\pgfqpoint{2.943417in}{1.653355in}}%
\pgfpathlineto{\pgfqpoint{2.943417in}{1.649603in}}%
\pgfpathlineto{\pgfqpoint{2.943417in}{1.645852in}}%
\pgfpathlineto{\pgfqpoint{2.942006in}{1.644164in}}%
\pgfpathlineto{\pgfqpoint{2.940282in}{1.642100in}}%
\pgfpathlineto{\pgfqpoint{2.940282in}{1.638349in}}%
\pgfpathlineto{\pgfqpoint{2.940282in}{1.634597in}}%
\pgfpathlineto{\pgfqpoint{2.938871in}{1.632909in}}%
\pgfpathlineto{\pgfqpoint{2.937147in}{1.630846in}}%
\pgfpathlineto{\pgfqpoint{2.937147in}{1.627094in}}%
\pgfpathlineto{\pgfqpoint{2.937147in}{1.623343in}}%
\pgfpathlineto{\pgfqpoint{2.935737in}{1.621654in}}%
\pgfpathlineto{\pgfqpoint{2.934013in}{1.619591in}}%
\pgfpathlineto{\pgfqpoint{2.934013in}{1.615840in}}%
\pgfpathlineto{\pgfqpoint{2.934013in}{1.612088in}}%
\pgfpathlineto{\pgfqpoint{2.932602in}{1.610400in}}%
\pgfpathlineto{\pgfqpoint{2.930878in}{1.608336in}}%
\pgfpathlineto{\pgfqpoint{2.930878in}{1.604585in}}%
\pgfpathlineto{\pgfqpoint{2.930878in}{1.600833in}}%
\pgfpathlineto{\pgfqpoint{2.929467in}{1.599145in}}%
\pgfpathlineto{\pgfqpoint{2.927743in}{1.597082in}}%
\pgfpathlineto{\pgfqpoint{2.927743in}{1.593330in}}%
\pgfpathlineto{\pgfqpoint{2.927743in}{1.589579in}}%
\pgfpathlineto{\pgfqpoint{2.926332in}{1.587891in}}%
\pgfpathlineto{\pgfqpoint{2.924608in}{1.585827in}}%
\pgfpathlineto{\pgfqpoint{2.924608in}{1.582076in}}%
\pgfpathlineto{\pgfqpoint{2.924608in}{1.578324in}}%
\pgfpathlineto{\pgfqpoint{2.923198in}{1.576636in}}%
\pgfpathlineto{\pgfqpoint{2.921474in}{1.574573in}}%
\pgfpathlineto{\pgfqpoint{2.921474in}{1.570821in}}%
\pgfpathlineto{\pgfqpoint{2.921474in}{1.567070in}}%
\pgfpathlineto{\pgfqpoint{2.920063in}{1.565381in}}%
\pgfpathlineto{\pgfqpoint{2.918339in}{1.563318in}}%
\pgfpathlineto{\pgfqpoint{2.918339in}{1.559566in}}%
\pgfpathlineto{\pgfqpoint{2.918339in}{1.555815in}}%
\pgfpathlineto{\pgfqpoint{2.918339in}{1.552063in}}%
\pgfpathlineto{\pgfqpoint{2.916928in}{1.550375in}}%
\pgfpathlineto{\pgfqpoint{2.915204in}{1.548312in}}%
\pgfpathlineto{\pgfqpoint{2.915204in}{1.544560in}}%
\pgfpathlineto{\pgfqpoint{2.915204in}{1.540809in}}%
\pgfpathlineto{\pgfqpoint{2.913794in}{1.539121in}}%
\pgfpathlineto{\pgfqpoint{2.912069in}{1.537057in}}%
\pgfpathlineto{\pgfqpoint{2.912069in}{1.533306in}}%
\pgfpathlineto{\pgfqpoint{2.912069in}{1.529554in}}%
\pgfpathlineto{\pgfqpoint{2.910659in}{1.527866in}}%
\pgfpathlineto{\pgfqpoint{2.908935in}{1.525803in}}%
\pgfpathlineto{\pgfqpoint{2.908935in}{1.522051in}}%
\pgfpathlineto{\pgfqpoint{2.908935in}{1.518300in}}%
\pgfpathlineto{\pgfqpoint{2.907524in}{1.516611in}}%
\pgfpathlineto{\pgfqpoint{2.905800in}{1.514548in}}%
\pgfpathlineto{\pgfqpoint{2.905800in}{1.510797in}}%
\pgfpathlineto{\pgfqpoint{2.905800in}{1.507045in}}%
\pgfpathlineto{\pgfqpoint{2.904389in}{1.505357in}}%
\pgfpathlineto{\pgfqpoint{2.902665in}{1.503293in}}%
\pgfpathlineto{\pgfqpoint{2.902665in}{1.499542in}}%
\pgfpathlineto{\pgfqpoint{2.902665in}{1.495790in}}%
\pgfpathlineto{\pgfqpoint{2.901255in}{1.494102in}}%
\pgfpathlineto{\pgfqpoint{2.899530in}{1.492039in}}%
\pgfpathlineto{\pgfqpoint{2.899530in}{1.488287in}}%
\pgfpathlineto{\pgfqpoint{2.899530in}{1.484536in}}%
\pgfpathlineto{\pgfqpoint{2.898120in}{1.482848in}}%
\pgfpathlineto{\pgfqpoint{2.896396in}{1.480784in}}%
\pgfpathlineto{\pgfqpoint{2.896396in}{1.477033in}}%
\pgfpathlineto{\pgfqpoint{2.896396in}{1.473281in}}%
\pgfpathlineto{\pgfqpoint{2.894985in}{1.471593in}}%
\pgfpathlineto{\pgfqpoint{2.893261in}{1.469530in}}%
\pgfpathlineto{\pgfqpoint{2.893261in}{1.465778in}}%
\pgfpathlineto{\pgfqpoint{2.893261in}{1.462027in}}%
\pgfpathlineto{\pgfqpoint{2.891850in}{1.460338in}}%
\pgfpathlineto{\pgfqpoint{2.890126in}{1.458275in}}%
\pgfpathlineto{\pgfqpoint{2.890126in}{1.454524in}}%
\pgfpathlineto{\pgfqpoint{2.890126in}{1.450772in}}%
\pgfpathlineto{\pgfqpoint{2.890126in}{1.447020in}}%
\pgfpathlineto{\pgfqpoint{2.888716in}{1.445332in}}%
\pgfpathlineto{\pgfqpoint{2.886991in}{1.443269in}}%
\pgfpathlineto{\pgfqpoint{2.886991in}{1.439517in}}%
\pgfpathlineto{\pgfqpoint{2.886991in}{1.435766in}}%
\pgfpathlineto{\pgfqpoint{2.885581in}{1.434078in}}%
\pgfpathlineto{\pgfqpoint{2.883857in}{1.432014in}}%
\pgfpathlineto{\pgfqpoint{2.883857in}{1.428263in}}%
\pgfpathlineto{\pgfqpoint{2.883857in}{1.424511in}}%
\pgfpathlineto{\pgfqpoint{2.882446in}{1.422823in}}%
\pgfpathlineto{\pgfqpoint{2.880722in}{1.420760in}}%
\pgfpathlineto{\pgfqpoint{2.880722in}{1.417008in}}%
\pgfpathlineto{\pgfqpoint{2.880722in}{1.413257in}}%
\pgfpathlineto{\pgfqpoint{2.879311in}{1.411568in}}%
\pgfpathlineto{\pgfqpoint{2.877587in}{1.409505in}}%
\pgfpathlineto{\pgfqpoint{2.877587in}{1.405754in}}%
\pgfpathlineto{\pgfqpoint{2.877587in}{1.402002in}}%
\pgfpathlineto{\pgfqpoint{2.876177in}{1.400314in}}%
\pgfpathlineto{\pgfqpoint{2.874453in}{1.398251in}}%
\pgfpathlineto{\pgfqpoint{2.874453in}{1.394499in}}%
\pgfpathlineto{\pgfqpoint{2.874453in}{1.390747in}}%
\pgfpathlineto{\pgfqpoint{2.873042in}{1.389059in}}%
\pgfpathlineto{\pgfqpoint{2.871318in}{1.386996in}}%
\pgfpathlineto{\pgfqpoint{2.871318in}{1.383244in}}%
\pgfpathlineto{\pgfqpoint{2.871318in}{1.379493in}}%
\pgfpathlineto{\pgfqpoint{2.869907in}{1.377805in}}%
\pgfpathlineto{\pgfqpoint{2.868183in}{1.375741in}}%
\pgfpathlineto{\pgfqpoint{2.868183in}{1.371990in}}%
\pgfpathlineto{\pgfqpoint{2.868183in}{1.368238in}}%
\pgfpathlineto{\pgfqpoint{2.866772in}{1.366550in}}%
\pgfpathlineto{\pgfqpoint{2.865048in}{1.364487in}}%
\pgfpathlineto{\pgfqpoint{2.865048in}{1.360735in}}%
\pgfpathlineto{\pgfqpoint{2.865048in}{1.356984in}}%
\pgfpathlineto{\pgfqpoint{2.865048in}{1.353232in}}%
\pgfpathlineto{\pgfqpoint{2.863638in}{1.351544in}}%
\pgfpathlineto{\pgfqpoint{2.861914in}{1.349481in}}%
\pgfpathlineto{\pgfqpoint{2.861914in}{1.345729in}}%
\pgfpathlineto{\pgfqpoint{2.861914in}{1.341977in}}%
\pgfpathlineto{\pgfqpoint{2.860503in}{1.340289in}}%
\pgfpathlineto{\pgfqpoint{2.858779in}{1.338226in}}%
\pgfpathlineto{\pgfqpoint{2.858779in}{1.334474in}}%
\pgfpathlineto{\pgfqpoint{2.858779in}{1.330723in}}%
\pgfpathlineto{\pgfqpoint{2.857368in}{1.329035in}}%
\pgfpathlineto{\pgfqpoint{2.855644in}{1.326971in}}%
\pgfpathlineto{\pgfqpoint{2.855644in}{1.323220in}}%
\pgfpathlineto{\pgfqpoint{2.855644in}{1.319468in}}%
\pgfpathlineto{\pgfqpoint{2.854233in}{1.317780in}}%
\pgfpathlineto{\pgfqpoint{2.852509in}{1.315717in}}%
\pgfpathlineto{\pgfqpoint{2.852509in}{1.311965in}}%
\pgfpathlineto{\pgfqpoint{2.852509in}{1.308214in}}%
\pgfpathlineto{\pgfqpoint{2.851099in}{1.306525in}}%
\pgfpathlineto{\pgfqpoint{2.849375in}{1.304462in}}%
\pgfpathlineto{\pgfqpoint{2.849375in}{1.300711in}}%
\pgfpathlineto{\pgfqpoint{2.849375in}{1.296959in}}%
\pgfpathlineto{\pgfqpoint{2.847964in}{1.295271in}}%
\pgfpathlineto{\pgfqpoint{2.846240in}{1.293208in}}%
\pgfpathlineto{\pgfqpoint{2.846240in}{1.289456in}}%
\pgfpathlineto{\pgfqpoint{2.846240in}{1.285704in}}%
\pgfpathlineto{\pgfqpoint{2.844829in}{1.284016in}}%
\pgfpathlineto{\pgfqpoint{2.843105in}{1.281953in}}%
\pgfpathlineto{\pgfqpoint{2.843105in}{1.278201in}}%
\pgfpathlineto{\pgfqpoint{2.843105in}{1.274450in}}%
\pgfpathlineto{\pgfqpoint{2.841694in}{1.272762in}}%
\pgfpathlineto{\pgfqpoint{2.839970in}{1.270698in}}%
\pgfpathlineto{\pgfqpoint{2.839970in}{1.266947in}}%
\pgfpathlineto{\pgfqpoint{2.839970in}{1.263195in}}%
\pgfpathlineto{\pgfqpoint{2.838560in}{1.261507in}}%
\pgfpathlineto{\pgfqpoint{2.836836in}{1.259444in}}%
\pgfpathlineto{\pgfqpoint{2.836836in}{1.255692in}}%
\pgfpathlineto{\pgfqpoint{2.836836in}{1.251941in}}%
\pgfpathlineto{\pgfqpoint{2.836836in}{1.248189in}}%
\pgfpathlineto{\pgfqpoint{2.835425in}{1.246501in}}%
\pgfpathlineto{\pgfqpoint{2.833701in}{1.244438in}}%
\pgfpathlineto{\pgfqpoint{2.833701in}{1.240686in}}%
\pgfpathlineto{\pgfqpoint{2.833701in}{1.236935in}}%
\pgfpathlineto{\pgfqpoint{2.832290in}{1.235246in}}%
\pgfpathlineto{\pgfqpoint{2.830566in}{1.233183in}}%
\pgfpathlineto{\pgfqpoint{2.830566in}{1.229431in}}%
\pgfpathlineto{\pgfqpoint{2.830566in}{1.225680in}}%
\pgfpathlineto{\pgfqpoint{2.829155in}{1.223992in}}%
\pgfpathlineto{\pgfqpoint{2.827431in}{1.221928in}}%
\pgfpathlineto{\pgfqpoint{2.827431in}{1.218177in}}%
\pgfpathlineto{\pgfqpoint{2.827431in}{1.214425in}}%
\pgfpathlineto{\pgfqpoint{2.826021in}{1.212737in}}%
\pgfpathlineto{\pgfqpoint{2.824297in}{1.210674in}}%
\pgfpathlineto{\pgfqpoint{2.824297in}{1.206922in}}%
\pgfpathlineto{\pgfqpoint{2.824297in}{1.203171in}}%
\pgfpathlineto{\pgfqpoint{2.822886in}{1.201483in}}%
\pgfpathlineto{\pgfqpoint{2.821162in}{1.199419in}}%
\pgfpathlineto{\pgfqpoint{2.821162in}{1.195668in}}%
\pgfpathlineto{\pgfqpoint{2.821162in}{1.191916in}}%
\pgfpathlineto{\pgfqpoint{2.819751in}{1.190228in}}%
\pgfpathlineto{\pgfqpoint{2.818027in}{1.188165in}}%
\pgfpathlineto{\pgfqpoint{2.818027in}{1.184413in}}%
\pgfpathlineto{\pgfqpoint{2.818027in}{1.180662in}}%
\pgfpathlineto{\pgfqpoint{2.816617in}{1.178973in}}%
\pgfpathlineto{\pgfqpoint{2.814892in}{1.176910in}}%
\pgfpathlineto{\pgfqpoint{2.814892in}{1.173158in}}%
\pgfpathlineto{\pgfqpoint{2.813482in}{1.171470in}}%
\pgfpathlineto{\pgfqpoint{2.810347in}{1.171470in}}%
\pgfpathlineto{\pgfqpoint{2.807212in}{1.171470in}}%
\pgfpathlineto{\pgfqpoint{2.805488in}{1.169407in}}%
\pgfpathlineto{\pgfqpoint{2.804078in}{1.167719in}}%
\pgfpathlineto{\pgfqpoint{2.800943in}{1.167719in}}%
\pgfpathlineto{\pgfqpoint{2.797808in}{1.167719in}}%
\pgfpathlineto{\pgfqpoint{2.796084in}{1.165655in}}%
\pgfpathlineto{\pgfqpoint{2.794673in}{1.163967in}}%
\pgfpathlineto{\pgfqpoint{2.791539in}{1.163967in}}%
\pgfpathlineto{\pgfqpoint{2.789814in}{1.161904in}}%
\pgfpathlineto{\pgfqpoint{2.788404in}{1.160216in}}%
\pgfpathlineto{\pgfqpoint{2.785269in}{1.160216in}}%
\pgfpathlineto{\pgfqpoint{2.782134in}{1.160216in}}%
\pgfpathlineto{\pgfqpoint{2.780410in}{1.158152in}}%
\pgfpathlineto{\pgfqpoint{2.779000in}{1.156464in}}%
\pgfpathlineto{\pgfqpoint{2.775865in}{1.156464in}}%
\pgfpathlineto{\pgfqpoint{2.772730in}{1.156464in}}%
\pgfpathlineto{\pgfqpoint{2.771006in}{1.154401in}}%
\pgfpathlineto{\pgfqpoint{2.769595in}{1.152713in}}%
\pgfpathlineto{\pgfqpoint{2.766461in}{1.152713in}}%
\pgfpathlineto{\pgfqpoint{2.763326in}{1.152713in}}%
\pgfpathlineto{\pgfqpoint{2.761602in}{1.150649in}}%
\pgfpathlineto{\pgfqpoint{2.760191in}{1.148961in}}%
\pgfpathlineto{\pgfqpoint{2.757056in}{1.148961in}}%
\pgfpathlineto{\pgfqpoint{2.755332in}{1.146898in}}%
\pgfpathlineto{\pgfqpoint{2.753922in}{1.145210in}}%
\pgfpathlineto{\pgfqpoint{2.750787in}{1.145210in}}%
\pgfpathlineto{\pgfqpoint{2.747652in}{1.145210in}}%
\pgfpathlineto{\pgfqpoint{2.745928in}{1.143146in}}%
\pgfpathlineto{\pgfqpoint{2.744517in}{1.141458in}}%
\pgfpathlineto{\pgfqpoint{2.741383in}{1.141458in}}%
\pgfpathlineto{\pgfqpoint{2.738248in}{1.141458in}}%
\pgfpathlineto{\pgfqpoint{2.736524in}{1.139395in}}%
\pgfpathlineto{\pgfqpoint{2.735113in}{1.137706in}}%
\pgfpathlineto{\pgfqpoint{2.731978in}{1.137706in}}%
\pgfpathlineto{\pgfqpoint{2.728844in}{1.137706in}}%
\pgfpathlineto{\pgfqpoint{2.727120in}{1.135643in}}%
\pgfpathlineto{\pgfqpoint{2.725709in}{1.133955in}}%
\pgfpathlineto{\pgfqpoint{2.722574in}{1.133955in}}%
\pgfpathlineto{\pgfqpoint{2.719439in}{1.133955in}}%
\pgfpathlineto{\pgfqpoint{2.717715in}{1.131892in}}%
\pgfpathlineto{\pgfqpoint{2.716305in}{1.130203in}}%
\pgfpathlineto{\pgfqpoint{2.713170in}{1.130203in}}%
\pgfpathlineto{\pgfqpoint{2.711446in}{1.128140in}}%
\pgfpathlineto{\pgfqpoint{2.710035in}{1.126452in}}%
\pgfpathlineto{\pgfqpoint{2.706901in}{1.126452in}}%
\pgfpathlineto{\pgfqpoint{2.703766in}{1.126452in}}%
\pgfpathlineto{\pgfqpoint{2.702042in}{1.124388in}}%
\pgfpathlineto{\pgfqpoint{2.700631in}{1.122700in}}%
\pgfpathlineto{\pgfqpoint{2.697496in}{1.122700in}}%
\pgfpathlineto{\pgfqpoint{2.694362in}{1.122700in}}%
\pgfpathlineto{\pgfqpoint{2.692637in}{1.120637in}}%
\pgfpathlineto{\pgfqpoint{2.691227in}{1.118949in}}%
\pgfpathlineto{\pgfqpoint{2.688092in}{1.118949in}}%
\pgfpathlineto{\pgfqpoint{2.684957in}{1.118949in}}%
\pgfpathlineto{\pgfqpoint{2.683233in}{1.116885in}}%
\pgfpathlineto{\pgfqpoint{2.681823in}{1.115197in}}%
\pgfpathlineto{\pgfqpoint{2.678688in}{1.115197in}}%
\pgfpathlineto{\pgfqpoint{2.676964in}{1.113134in}}%
\pgfpathlineto{\pgfqpoint{2.675553in}{1.111446in}}%
\pgfpathlineto{\pgfqpoint{2.672418in}{1.111446in}}%
\pgfpathlineto{\pgfqpoint{2.669284in}{1.111446in}}%
\pgfpathlineto{\pgfqpoint{2.667560in}{1.109382in}}%
\pgfpathlineto{\pgfqpoint{2.666149in}{1.107694in}}%
\pgfpathlineto{\pgfqpoint{2.663014in}{1.107694in}}%
\pgfpathlineto{\pgfqpoint{2.659879in}{1.107694in}}%
\pgfpathlineto{\pgfqpoint{2.658155in}{1.105631in}}%
\pgfpathlineto{\pgfqpoint{2.656745in}{1.103943in}}%
\pgfpathlineto{\pgfqpoint{2.653610in}{1.103943in}}%
\pgfpathlineto{\pgfqpoint{2.650475in}{1.103943in}}%
\pgfpathlineto{\pgfqpoint{2.648751in}{1.101879in}}%
\pgfpathlineto{\pgfqpoint{2.647340in}{1.100191in}}%
\pgfpathlineto{\pgfqpoint{2.644206in}{1.100191in}}%
\pgfpathlineto{\pgfqpoint{2.642482in}{1.098128in}}%
\pgfpathlineto{\pgfqpoint{2.641071in}{1.096440in}}%
\pgfpathlineto{\pgfqpoint{2.637936in}{1.096440in}}%
\pgfpathlineto{\pgfqpoint{2.634801in}{1.096440in}}%
\pgfpathlineto{\pgfqpoint{2.633077in}{1.094376in}}%
\pgfpathlineto{\pgfqpoint{2.631667in}{1.092688in}}%
\pgfpathlineto{\pgfqpoint{2.628532in}{1.092688in}}%
\pgfpathlineto{\pgfqpoint{2.625397in}{1.092688in}}%
\pgfpathlineto{\pgfqpoint{2.623673in}{1.090625in}}%
\pgfpathlineto{\pgfqpoint{2.622262in}{1.088936in}}%
\pgfpathlineto{\pgfqpoint{2.619128in}{1.088936in}}%
\pgfpathlineto{\pgfqpoint{2.615993in}{1.088936in}}%
\pgfpathlineto{\pgfqpoint{2.614269in}{1.086873in}}%
\pgfpathlineto{\pgfqpoint{2.612858in}{1.085185in}}%
\pgfpathlineto{\pgfqpoint{2.609724in}{1.085185in}}%
\pgfpathlineto{\pgfqpoint{2.606589in}{1.085185in}}%
\pgfpathlineto{\pgfqpoint{2.604865in}{1.083122in}}%
\pgfpathlineto{\pgfqpoint{2.603454in}{1.081433in}}%
\pgfpathlineto{\pgfqpoint{2.600319in}{1.081433in}}%
\pgfpathlineto{\pgfqpoint{2.598595in}{1.079370in}}%
\pgfpathlineto{\pgfqpoint{2.597185in}{1.077682in}}%
\pgfpathlineto{\pgfqpoint{2.594050in}{1.077682in}}%
\pgfpathlineto{\pgfqpoint{2.590915in}{1.077682in}}%
\pgfpathlineto{\pgfqpoint{2.589191in}{1.075619in}}%
\pgfpathlineto{\pgfqpoint{2.587780in}{1.073930in}}%
\pgfpathlineto{\pgfqpoint{2.584646in}{1.073930in}}%
\pgfpathlineto{\pgfqpoint{2.581511in}{1.073930in}}%
\pgfpathlineto{\pgfqpoint{2.579787in}{1.071867in}}%
\pgfpathlineto{\pgfqpoint{2.578376in}{1.070179in}}%
\pgfpathlineto{\pgfqpoint{2.575241in}{1.070179in}}%
\pgfpathlineto{\pgfqpoint{2.572107in}{1.070179in}}%
\pgfpathlineto{\pgfqpoint{2.570383in}{1.068115in}}%
\pgfpathlineto{\pgfqpoint{2.568972in}{1.066427in}}%
\pgfpathlineto{\pgfqpoint{2.565837in}{1.066427in}}%
\pgfpathlineto{\pgfqpoint{2.564113in}{1.064364in}}%
\pgfpathlineto{\pgfqpoint{2.562702in}{1.062676in}}%
\pgfpathlineto{\pgfqpoint{2.559568in}{1.062676in}}%
\pgfpathlineto{\pgfqpoint{2.556433in}{1.062676in}}%
\pgfpathlineto{\pgfqpoint{2.554709in}{1.060612in}}%
\pgfpathlineto{\pgfqpoint{2.553298in}{1.058924in}}%
\pgfpathlineto{\pgfqpoint{2.550163in}{1.058924in}}%
\pgfpathlineto{\pgfqpoint{2.547029in}{1.058924in}}%
\pgfpathlineto{\pgfqpoint{2.545305in}{1.056861in}}%
\pgfpathlineto{\pgfqpoint{2.543894in}{1.055173in}}%
\pgfpathlineto{\pgfqpoint{2.540759in}{1.055173in}}%
\pgfpathlineto{\pgfqpoint{2.537624in}{1.055173in}}%
\pgfpathlineto{\pgfqpoint{2.535900in}{1.053109in}}%
\pgfpathlineto{\pgfqpoint{2.534490in}{1.051421in}}%
\pgfpathlineto{\pgfqpoint{2.531355in}{1.051421in}}%
\pgfpathlineto{\pgfqpoint{2.529631in}{1.049358in}}%
\pgfpathlineto{\pgfqpoint{2.528220in}{1.047670in}}%
\pgfpathlineto{\pgfqpoint{2.525085in}{1.047670in}}%
\pgfpathlineto{\pgfqpoint{2.521951in}{1.047670in}}%
\pgfpathlineto{\pgfqpoint{2.520227in}{1.045606in}}%
\pgfpathlineto{\pgfqpoint{2.518816in}{1.043918in}}%
\pgfpathlineto{\pgfqpoint{2.515681in}{1.043918in}}%
\pgfpathlineto{\pgfqpoint{2.512547in}{1.043918in}}%
\pgfpathlineto{\pgfqpoint{2.510822in}{1.041855in}}%
\pgfpathlineto{\pgfqpoint{2.509412in}{1.040167in}}%
\pgfpathlineto{\pgfqpoint{2.506277in}{1.040167in}}%
\pgfpathlineto{\pgfqpoint{2.503142in}{1.040167in}}%
\pgfpathlineto{\pgfqpoint{2.501418in}{1.038103in}}%
\pgfpathlineto{\pgfqpoint{2.500008in}{1.036415in}}%
\pgfpathlineto{\pgfqpoint{2.496873in}{1.036415in}}%
\pgfpathlineto{\pgfqpoint{2.493738in}{1.036415in}}%
\pgfpathlineto{\pgfqpoint{2.492014in}{1.034352in}}%
\pgfpathlineto{\pgfqpoint{2.490603in}{1.032663in}}%
\pgfpathlineto{\pgfqpoint{2.487469in}{1.032663in}}%
\pgfpathlineto{\pgfqpoint{2.485744in}{1.030600in}}%
\pgfpathlineto{\pgfqpoint{2.484334in}{1.028912in}}%
\pgfpathlineto{\pgfqpoint{2.481199in}{1.028912in}}%
\pgfpathlineto{\pgfqpoint{2.478064in}{1.028912in}}%
\pgfpathlineto{\pgfqpoint{2.476340in}{1.026849in}}%
\pgfpathlineto{\pgfqpoint{2.474930in}{1.025160in}}%
\pgfpathlineto{\pgfqpoint{2.471795in}{1.025160in}}%
\pgfpathlineto{\pgfqpoint{2.468660in}{1.025160in}}%
\pgfpathlineto{\pgfqpoint{2.466936in}{1.023097in}}%
\pgfpathlineto{\pgfqpoint{2.465525in}{1.021409in}}%
\pgfpathlineto{\pgfqpoint{2.462391in}{1.021409in}}%
\pgfpathlineto{\pgfqpoint{2.459256in}{1.021409in}}%
\pgfpathlineto{\pgfqpoint{2.457532in}{1.019346in}}%
\pgfpathlineto{\pgfqpoint{2.456121in}{1.017657in}}%
\pgfpathlineto{\pgfqpoint{2.452986in}{1.017657in}}%
\pgfpathlineto{\pgfqpoint{2.451262in}{1.015594in}}%
\pgfpathlineto{\pgfqpoint{2.449852in}{1.013906in}}%
\pgfpathlineto{\pgfqpoint{2.446717in}{1.013906in}}%
\pgfpathlineto{\pgfqpoint{2.443582in}{1.013906in}}%
\pgfpathlineto{\pgfqpoint{2.441858in}{1.011842in}}%
\pgfpathlineto{\pgfqpoint{2.440447in}{1.010154in}}%
\pgfpathlineto{\pgfqpoint{2.437313in}{1.010154in}}%
\pgfpathlineto{\pgfqpoint{2.434178in}{1.010154in}}%
\pgfpathlineto{\pgfqpoint{2.432454in}{1.008091in}}%
\pgfpathlineto{\pgfqpoint{2.431043in}{1.006403in}}%
\pgfpathlineto{\pgfqpoint{2.427908in}{1.006403in}}%
\pgfpathlineto{\pgfqpoint{2.424774in}{1.006403in}}%
\pgfpathlineto{\pgfqpoint{2.423050in}{1.004339in}}%
\pgfpathlineto{\pgfqpoint{2.421639in}{1.002651in}}%
\pgfpathlineto{\pgfqpoint{2.418504in}{1.002651in}}%
\pgfpathlineto{\pgfqpoint{2.416780in}{1.000588in}}%
\pgfpathlineto{\pgfqpoint{2.415369in}{0.998900in}}%
\pgfpathlineto{\pgfqpoint{2.412235in}{0.998900in}}%
\pgfpathlineto{\pgfqpoint{2.409100in}{0.998900in}}%
\pgfpathlineto{\pgfqpoint{2.407376in}{0.996836in}}%
\pgfpathlineto{\pgfqpoint{2.405965in}{0.995148in}}%
\pgfpathlineto{\pgfqpoint{2.402831in}{0.995148in}}%
\pgfpathlineto{\pgfqpoint{2.399696in}{0.995148in}}%
\pgfpathlineto{\pgfqpoint{2.397972in}{0.993085in}}%
\pgfpathlineto{\pgfqpoint{2.396561in}{0.991397in}}%
\pgfpathlineto{\pgfqpoint{2.393426in}{0.991397in}}%
\pgfpathlineto{\pgfqpoint{2.390292in}{0.991397in}}%
\pgfpathlineto{\pgfqpoint{2.388567in}{0.989333in}}%
\pgfpathlineto{\pgfqpoint{2.387157in}{0.987645in}}%
\pgfpathlineto{\pgfqpoint{2.384022in}{0.987645in}}%
\pgfpathlineto{\pgfqpoint{2.380887in}{0.987645in}}%
\pgfpathlineto{\pgfqpoint{2.379163in}{0.985582in}}%
\pgfpathlineto{\pgfqpoint{2.377753in}{0.983894in}}%
\pgfpathlineto{\pgfqpoint{2.374618in}{0.983894in}}%
\pgfpathlineto{\pgfqpoint{2.372894in}{0.981830in}}%
\pgfpathlineto{\pgfqpoint{2.371483in}{0.980142in}}%
\pgfpathlineto{\pgfqpoint{2.368348in}{0.980142in}}%
\pgfpathlineto{\pgfqpoint{2.365214in}{0.980142in}}%
\pgfpathlineto{\pgfqpoint{2.363490in}{0.978079in}}%
\pgfpathlineto{\pgfqpoint{2.362079in}{0.976390in}}%
\pgfpathlineto{\pgfqpoint{2.358944in}{0.976390in}}%
\pgfpathlineto{\pgfqpoint{2.355809in}{0.976390in}}%
\pgfpathlineto{\pgfqpoint{2.354085in}{0.974327in}}%
\pgfpathlineto{\pgfqpoint{2.352675in}{0.972639in}}%
\pgfpathlineto{\pgfqpoint{2.349540in}{0.972639in}}%
\pgfpathlineto{\pgfqpoint{2.346405in}{0.972639in}}%
\pgfpathlineto{\pgfqpoint{2.344681in}{0.970576in}}%
\pgfpathlineto{\pgfqpoint{2.343270in}{0.968887in}}%
\pgfpathlineto{\pgfqpoint{2.340136in}{0.968887in}}%
\pgfpathlineto{\pgfqpoint{2.338412in}{0.966824in}}%
\pgfpathlineto{\pgfqpoint{2.337001in}{0.965136in}}%
\pgfpathlineto{\pgfqpoint{2.333866in}{0.965136in}}%
\pgfpathlineto{\pgfqpoint{2.330731in}{0.965136in}}%
\pgfpathlineto{\pgfqpoint{2.329007in}{0.963073in}}%
\pgfpathlineto{\pgfqpoint{2.327597in}{0.961384in}}%
\pgfpathlineto{\pgfqpoint{2.324462in}{0.961384in}}%
\pgfpathlineto{\pgfqpoint{2.321327in}{0.961384in}}%
\pgfpathlineto{\pgfqpoint{2.319603in}{0.959321in}}%
\pgfpathlineto{\pgfqpoint{2.318192in}{0.957633in}}%
\pgfpathlineto{\pgfqpoint{2.315058in}{0.957633in}}%
\pgfpathlineto{\pgfqpoint{2.311923in}{0.957633in}}%
\pgfpathlineto{\pgfqpoint{2.310199in}{0.955569in}}%
\pgfpathlineto{\pgfqpoint{2.308788in}{0.953881in}}%
\pgfpathlineto{\pgfqpoint{2.305654in}{0.953881in}}%
\pgfpathlineto{\pgfqpoint{2.303929in}{0.951818in}}%
\pgfpathlineto{\pgfqpoint{2.302519in}{0.950130in}}%
\pgfpathlineto{\pgfqpoint{2.299384in}{0.950130in}}%
\pgfpathlineto{\pgfqpoint{2.296249in}{0.950130in}}%
\pgfpathlineto{\pgfqpoint{2.294525in}{0.948066in}}%
\pgfpathlineto{\pgfqpoint{2.293115in}{0.946378in}}%
\pgfpathlineto{\pgfqpoint{2.289980in}{0.946378in}}%
\pgfpathlineto{\pgfqpoint{2.286845in}{0.946378in}}%
\pgfpathlineto{\pgfqpoint{2.285121in}{0.944315in}}%
\pgfpathlineto{\pgfqpoint{2.283710in}{0.942627in}}%
\pgfpathlineto{\pgfqpoint{2.280576in}{0.942627in}}%
\pgfpathlineto{\pgfqpoint{2.277441in}{0.942627in}}%
\pgfpathlineto{\pgfqpoint{2.275717in}{0.940563in}}%
\pgfpathlineto{\pgfqpoint{2.274306in}{0.938875in}}%
\pgfpathlineto{\pgfqpoint{2.271171in}{0.938875in}}%
\pgfpathlineto{\pgfqpoint{2.269447in}{0.936812in}}%
\pgfpathlineto{\pgfqpoint{2.268037in}{0.935124in}}%
\pgfpathlineto{\pgfqpoint{2.264902in}{0.935124in}}%
\pgfpathlineto{\pgfqpoint{2.261767in}{0.935124in}}%
\pgfpathlineto{\pgfqpoint{2.260043in}{0.933060in}}%
\pgfpathlineto{\pgfqpoint{2.258632in}{0.931372in}}%
\pgfpathlineto{\pgfqpoint{2.255498in}{0.931372in}}%
\pgfpathlineto{\pgfqpoint{2.252363in}{0.931372in}}%
\pgfpathlineto{\pgfqpoint{2.250639in}{0.929309in}}%
\pgfpathlineto{\pgfqpoint{2.249228in}{0.927621in}}%
\pgfpathlineto{\pgfqpoint{2.246093in}{0.927621in}}%
\pgfpathlineto{\pgfqpoint{2.242959in}{0.927621in}}%
\pgfpathlineto{\pgfqpoint{2.241235in}{0.925557in}}%
\pgfpathlineto{\pgfqpoint{2.239824in}{0.923869in}}%
\pgfpathlineto{\pgfqpoint{2.236689in}{0.923869in}}%
\pgfpathlineto{\pgfqpoint{2.233554in}{0.923869in}}%
\pgfpathlineto{\pgfqpoint{2.231830in}{0.921806in}}%
\pgfpathlineto{\pgfqpoint{2.230420in}{0.920117in}}%
\pgfpathlineto{\pgfqpoint{2.227285in}{0.920117in}}%
\pgfpathlineto{\pgfqpoint{2.225561in}{0.918054in}}%
\pgfpathlineto{\pgfqpoint{2.224150in}{0.916366in}}%
\pgfpathlineto{\pgfqpoint{2.221015in}{0.916366in}}%
\pgfpathlineto{\pgfqpoint{2.217881in}{0.916366in}}%
\pgfpathlineto{\pgfqpoint{2.216157in}{0.914303in}}%
\pgfpathlineto{\pgfqpoint{2.214746in}{0.912614in}}%
\pgfpathlineto{\pgfqpoint{2.211611in}{0.912614in}}%
\pgfpathlineto{\pgfqpoint{2.208477in}{0.912614in}}%
\pgfpathlineto{\pgfqpoint{2.206752in}{0.910551in}}%
\pgfpathlineto{\pgfqpoint{2.205342in}{0.908863in}}%
\pgfpathlineto{\pgfqpoint{2.202207in}{0.908863in}}%
\pgfpathlineto{\pgfqpoint{2.199072in}{0.908863in}}%
\pgfpathlineto{\pgfqpoint{2.197348in}{0.906799in}}%
\pgfpathlineto{\pgfqpoint{2.195938in}{0.905111in}}%
\pgfpathlineto{\pgfqpoint{2.192803in}{0.905111in}}%
\pgfpathlineto{\pgfqpoint{2.191079in}{0.903048in}}%
\pgfpathlineto{\pgfqpoint{2.189668in}{0.901360in}}%
\pgfpathlineto{\pgfqpoint{2.186533in}{0.901360in}}%
\pgfpathlineto{\pgfqpoint{2.183399in}{0.901360in}}%
\pgfpathlineto{\pgfqpoint{2.181674in}{0.899296in}}%
\pgfpathlineto{\pgfqpoint{2.180264in}{0.897608in}}%
\pgfpathlineto{\pgfqpoint{2.177129in}{0.897608in}}%
\pgfpathlineto{\pgfqpoint{2.173994in}{0.897608in}}%
\pgfpathclose%
\pgfusepath{fill}%
\end{pgfscope}%
\begin{pgfscope}%
\pgfpathrectangle{\pgfqpoint{0.888750in}{0.419100in}}{\pgfqpoint{2.504659in}{2.933700in}} %
\pgfusepath{clip}%
\pgfsetbuttcap%
\pgfsetroundjoin%
\definecolor{currentfill}{rgb}{0.000000,0.666700,0.648390}%
\pgfsetfillcolor{currentfill}%
\pgfsetfillopacity{0.300000}%
\pgfsetlinewidth{0.000000pt}%
\definecolor{currentstroke}{rgb}{0.000000,0.000000,0.000000}%
\pgfsetstrokecolor{currentstroke}%
\pgfsetdash{}{0pt}%
\pgfpathmoveto{\pgfqpoint{2.173994in}{0.896483in}}%
\pgfpathlineto{\pgfqpoint{2.177129in}{0.896483in}}%
\pgfpathlineto{\pgfqpoint{2.180264in}{0.896483in}}%
\pgfpathlineto{\pgfqpoint{2.182615in}{0.899296in}}%
\pgfpathlineto{\pgfqpoint{2.183399in}{0.900234in}}%
\pgfpathlineto{\pgfqpoint{2.186533in}{0.900234in}}%
\pgfpathlineto{\pgfqpoint{2.189668in}{0.900234in}}%
\pgfpathlineto{\pgfqpoint{2.192019in}{0.903048in}}%
\pgfpathlineto{\pgfqpoint{2.192803in}{0.903986in}}%
\pgfpathlineto{\pgfqpoint{2.195938in}{0.903986in}}%
\pgfpathlineto{\pgfqpoint{2.198289in}{0.906799in}}%
\pgfpathlineto{\pgfqpoint{2.199072in}{0.907737in}}%
\pgfpathlineto{\pgfqpoint{2.202207in}{0.907737in}}%
\pgfpathlineto{\pgfqpoint{2.205342in}{0.907737in}}%
\pgfpathlineto{\pgfqpoint{2.207693in}{0.910551in}}%
\pgfpathlineto{\pgfqpoint{2.208477in}{0.911489in}}%
\pgfpathlineto{\pgfqpoint{2.211611in}{0.911489in}}%
\pgfpathlineto{\pgfqpoint{2.214746in}{0.911489in}}%
\pgfpathlineto{\pgfqpoint{2.217097in}{0.914303in}}%
\pgfpathlineto{\pgfqpoint{2.217881in}{0.915240in}}%
\pgfpathlineto{\pgfqpoint{2.221015in}{0.915240in}}%
\pgfpathlineto{\pgfqpoint{2.224150in}{0.915240in}}%
\pgfpathlineto{\pgfqpoint{2.226501in}{0.918054in}}%
\pgfpathlineto{\pgfqpoint{2.227285in}{0.918992in}}%
\pgfpathlineto{\pgfqpoint{2.230420in}{0.918992in}}%
\pgfpathlineto{\pgfqpoint{2.232771in}{0.921806in}}%
\pgfpathlineto{\pgfqpoint{2.233554in}{0.922744in}}%
\pgfpathlineto{\pgfqpoint{2.236689in}{0.922744in}}%
\pgfpathlineto{\pgfqpoint{2.239824in}{0.922744in}}%
\pgfpathlineto{\pgfqpoint{2.242175in}{0.925557in}}%
\pgfpathlineto{\pgfqpoint{2.242959in}{0.926495in}}%
\pgfpathlineto{\pgfqpoint{2.246093in}{0.926495in}}%
\pgfpathlineto{\pgfqpoint{2.249228in}{0.926495in}}%
\pgfpathlineto{\pgfqpoint{2.251579in}{0.929309in}}%
\pgfpathlineto{\pgfqpoint{2.252363in}{0.930247in}}%
\pgfpathlineto{\pgfqpoint{2.255498in}{0.930247in}}%
\pgfpathlineto{\pgfqpoint{2.258632in}{0.930247in}}%
\pgfpathlineto{\pgfqpoint{2.260983in}{0.933060in}}%
\pgfpathlineto{\pgfqpoint{2.261767in}{0.933998in}}%
\pgfpathlineto{\pgfqpoint{2.264902in}{0.933998in}}%
\pgfpathlineto{\pgfqpoint{2.268037in}{0.933998in}}%
\pgfpathlineto{\pgfqpoint{2.270388in}{0.936812in}}%
\pgfpathlineto{\pgfqpoint{2.271171in}{0.937750in}}%
\pgfpathlineto{\pgfqpoint{2.274306in}{0.937750in}}%
\pgfpathlineto{\pgfqpoint{2.276657in}{0.940563in}}%
\pgfpathlineto{\pgfqpoint{2.277441in}{0.941501in}}%
\pgfpathlineto{\pgfqpoint{2.280576in}{0.941501in}}%
\pgfpathlineto{\pgfqpoint{2.283710in}{0.941501in}}%
\pgfpathlineto{\pgfqpoint{2.286061in}{0.944315in}}%
\pgfpathlineto{\pgfqpoint{2.286845in}{0.945253in}}%
\pgfpathlineto{\pgfqpoint{2.289980in}{0.945253in}}%
\pgfpathlineto{\pgfqpoint{2.293115in}{0.945253in}}%
\pgfpathlineto{\pgfqpoint{2.295466in}{0.948066in}}%
\pgfpathlineto{\pgfqpoint{2.296249in}{0.949004in}}%
\pgfpathlineto{\pgfqpoint{2.299384in}{0.949004in}}%
\pgfpathlineto{\pgfqpoint{2.302519in}{0.949004in}}%
\pgfpathlineto{\pgfqpoint{2.304870in}{0.951818in}}%
\pgfpathlineto{\pgfqpoint{2.305654in}{0.952756in}}%
\pgfpathlineto{\pgfqpoint{2.308788in}{0.952756in}}%
\pgfpathlineto{\pgfqpoint{2.311139in}{0.955569in}}%
\pgfpathlineto{\pgfqpoint{2.311923in}{0.956507in}}%
\pgfpathlineto{\pgfqpoint{2.315058in}{0.956507in}}%
\pgfpathlineto{\pgfqpoint{2.318192in}{0.956507in}}%
\pgfpathlineto{\pgfqpoint{2.320544in}{0.959321in}}%
\pgfpathlineto{\pgfqpoint{2.321327in}{0.960259in}}%
\pgfpathlineto{\pgfqpoint{2.324462in}{0.960259in}}%
\pgfpathlineto{\pgfqpoint{2.327597in}{0.960259in}}%
\pgfpathlineto{\pgfqpoint{2.329948in}{0.963073in}}%
\pgfpathlineto{\pgfqpoint{2.330731in}{0.964010in}}%
\pgfpathlineto{\pgfqpoint{2.333866in}{0.964010in}}%
\pgfpathlineto{\pgfqpoint{2.337001in}{0.964010in}}%
\pgfpathlineto{\pgfqpoint{2.339352in}{0.966824in}}%
\pgfpathlineto{\pgfqpoint{2.340136in}{0.967762in}}%
\pgfpathlineto{\pgfqpoint{2.343270in}{0.967762in}}%
\pgfpathlineto{\pgfqpoint{2.345621in}{0.970576in}}%
\pgfpathlineto{\pgfqpoint{2.346405in}{0.971513in}}%
\pgfpathlineto{\pgfqpoint{2.349540in}{0.971513in}}%
\pgfpathlineto{\pgfqpoint{2.352675in}{0.971513in}}%
\pgfpathlineto{\pgfqpoint{2.355026in}{0.974327in}}%
\pgfpathlineto{\pgfqpoint{2.355809in}{0.975265in}}%
\pgfpathlineto{\pgfqpoint{2.358944in}{0.975265in}}%
\pgfpathlineto{\pgfqpoint{2.362079in}{0.975265in}}%
\pgfpathlineto{\pgfqpoint{2.364430in}{0.978079in}}%
\pgfpathlineto{\pgfqpoint{2.365214in}{0.979017in}}%
\pgfpathlineto{\pgfqpoint{2.368348in}{0.979017in}}%
\pgfpathlineto{\pgfqpoint{2.371483in}{0.979017in}}%
\pgfpathlineto{\pgfqpoint{2.373834in}{0.981830in}}%
\pgfpathlineto{\pgfqpoint{2.374618in}{0.982768in}}%
\pgfpathlineto{\pgfqpoint{2.377753in}{0.982768in}}%
\pgfpathlineto{\pgfqpoint{2.380104in}{0.985582in}}%
\pgfpathlineto{\pgfqpoint{2.380887in}{0.986520in}}%
\pgfpathlineto{\pgfqpoint{2.384022in}{0.986520in}}%
\pgfpathlineto{\pgfqpoint{2.387157in}{0.986520in}}%
\pgfpathlineto{\pgfqpoint{2.389508in}{0.989333in}}%
\pgfpathlineto{\pgfqpoint{2.390292in}{0.990271in}}%
\pgfpathlineto{\pgfqpoint{2.393426in}{0.990271in}}%
\pgfpathlineto{\pgfqpoint{2.396561in}{0.990271in}}%
\pgfpathlineto{\pgfqpoint{2.398912in}{0.993085in}}%
\pgfpathlineto{\pgfqpoint{2.399696in}{0.994023in}}%
\pgfpathlineto{\pgfqpoint{2.402831in}{0.994023in}}%
\pgfpathlineto{\pgfqpoint{2.405965in}{0.994023in}}%
\pgfpathlineto{\pgfqpoint{2.408316in}{0.996836in}}%
\pgfpathlineto{\pgfqpoint{2.409100in}{0.997774in}}%
\pgfpathlineto{\pgfqpoint{2.412235in}{0.997774in}}%
\pgfpathlineto{\pgfqpoint{2.415369in}{0.997774in}}%
\pgfpathlineto{\pgfqpoint{2.417721in}{1.000588in}}%
\pgfpathlineto{\pgfqpoint{2.418504in}{1.001526in}}%
\pgfpathlineto{\pgfqpoint{2.421639in}{1.001526in}}%
\pgfpathlineto{\pgfqpoint{2.423990in}{1.004339in}}%
\pgfpathlineto{\pgfqpoint{2.424774in}{1.005277in}}%
\pgfpathlineto{\pgfqpoint{2.427908in}{1.005277in}}%
\pgfpathlineto{\pgfqpoint{2.431043in}{1.005277in}}%
\pgfpathlineto{\pgfqpoint{2.433394in}{1.008091in}}%
\pgfpathlineto{\pgfqpoint{2.434178in}{1.009029in}}%
\pgfpathlineto{\pgfqpoint{2.437313in}{1.009029in}}%
\pgfpathlineto{\pgfqpoint{2.440447in}{1.009029in}}%
\pgfpathlineto{\pgfqpoint{2.442798in}{1.011842in}}%
\pgfpathlineto{\pgfqpoint{2.443582in}{1.012780in}}%
\pgfpathlineto{\pgfqpoint{2.446717in}{1.012780in}}%
\pgfpathlineto{\pgfqpoint{2.449852in}{1.012780in}}%
\pgfpathlineto{\pgfqpoint{2.452203in}{1.015594in}}%
\pgfpathlineto{\pgfqpoint{2.452986in}{1.016532in}}%
\pgfpathlineto{\pgfqpoint{2.456121in}{1.016532in}}%
\pgfpathlineto{\pgfqpoint{2.458472in}{1.019346in}}%
\pgfpathlineto{\pgfqpoint{2.459256in}{1.020283in}}%
\pgfpathlineto{\pgfqpoint{2.462391in}{1.020283in}}%
\pgfpathlineto{\pgfqpoint{2.465525in}{1.020283in}}%
\pgfpathlineto{\pgfqpoint{2.467876in}{1.023097in}}%
\pgfpathlineto{\pgfqpoint{2.468660in}{1.024035in}}%
\pgfpathlineto{\pgfqpoint{2.471795in}{1.024035in}}%
\pgfpathlineto{\pgfqpoint{2.474930in}{1.024035in}}%
\pgfpathlineto{\pgfqpoint{2.477281in}{1.026849in}}%
\pgfpathlineto{\pgfqpoint{2.478064in}{1.027786in}}%
\pgfpathlineto{\pgfqpoint{2.481199in}{1.027786in}}%
\pgfpathlineto{\pgfqpoint{2.484334in}{1.027786in}}%
\pgfpathlineto{\pgfqpoint{2.486685in}{1.030600in}}%
\pgfpathlineto{\pgfqpoint{2.487469in}{1.031538in}}%
\pgfpathlineto{\pgfqpoint{2.490603in}{1.031538in}}%
\pgfpathlineto{\pgfqpoint{2.492954in}{1.034352in}}%
\pgfpathlineto{\pgfqpoint{2.493738in}{1.035290in}}%
\pgfpathlineto{\pgfqpoint{2.496873in}{1.035290in}}%
\pgfpathlineto{\pgfqpoint{2.500008in}{1.035290in}}%
\pgfpathlineto{\pgfqpoint{2.502359in}{1.038103in}}%
\pgfpathlineto{\pgfqpoint{2.503142in}{1.039041in}}%
\pgfpathlineto{\pgfqpoint{2.506277in}{1.039041in}}%
\pgfpathlineto{\pgfqpoint{2.509412in}{1.039041in}}%
\pgfpathlineto{\pgfqpoint{2.511763in}{1.041855in}}%
\pgfpathlineto{\pgfqpoint{2.512547in}{1.042793in}}%
\pgfpathlineto{\pgfqpoint{2.515681in}{1.042793in}}%
\pgfpathlineto{\pgfqpoint{2.518816in}{1.042793in}}%
\pgfpathlineto{\pgfqpoint{2.521167in}{1.045606in}}%
\pgfpathlineto{\pgfqpoint{2.521951in}{1.046544in}}%
\pgfpathlineto{\pgfqpoint{2.525085in}{1.046544in}}%
\pgfpathlineto{\pgfqpoint{2.528220in}{1.046544in}}%
\pgfpathlineto{\pgfqpoint{2.530571in}{1.049358in}}%
\pgfpathlineto{\pgfqpoint{2.531355in}{1.050296in}}%
\pgfpathlineto{\pgfqpoint{2.534490in}{1.050296in}}%
\pgfpathlineto{\pgfqpoint{2.536841in}{1.053109in}}%
\pgfpathlineto{\pgfqpoint{2.537624in}{1.054047in}}%
\pgfpathlineto{\pgfqpoint{2.540759in}{1.054047in}}%
\pgfpathlineto{\pgfqpoint{2.543894in}{1.054047in}}%
\pgfpathlineto{\pgfqpoint{2.546245in}{1.056861in}}%
\pgfpathlineto{\pgfqpoint{2.547029in}{1.057799in}}%
\pgfpathlineto{\pgfqpoint{2.550163in}{1.057799in}}%
\pgfpathlineto{\pgfqpoint{2.553298in}{1.057799in}}%
\pgfpathlineto{\pgfqpoint{2.555649in}{1.060612in}}%
\pgfpathlineto{\pgfqpoint{2.556433in}{1.061550in}}%
\pgfpathlineto{\pgfqpoint{2.559568in}{1.061550in}}%
\pgfpathlineto{\pgfqpoint{2.562702in}{1.061550in}}%
\pgfpathlineto{\pgfqpoint{2.565053in}{1.064364in}}%
\pgfpathlineto{\pgfqpoint{2.565837in}{1.065302in}}%
\pgfpathlineto{\pgfqpoint{2.568972in}{1.065302in}}%
\pgfpathlineto{\pgfqpoint{2.571323in}{1.068115in}}%
\pgfpathlineto{\pgfqpoint{2.572107in}{1.069053in}}%
\pgfpathlineto{\pgfqpoint{2.575241in}{1.069053in}}%
\pgfpathlineto{\pgfqpoint{2.578376in}{1.069053in}}%
\pgfpathlineto{\pgfqpoint{2.580727in}{1.071867in}}%
\pgfpathlineto{\pgfqpoint{2.581511in}{1.072805in}}%
\pgfpathlineto{\pgfqpoint{2.584646in}{1.072805in}}%
\pgfpathlineto{\pgfqpoint{2.587780in}{1.072805in}}%
\pgfpathlineto{\pgfqpoint{2.590131in}{1.075619in}}%
\pgfpathlineto{\pgfqpoint{2.590915in}{1.076556in}}%
\pgfpathlineto{\pgfqpoint{2.594050in}{1.076556in}}%
\pgfpathlineto{\pgfqpoint{2.597185in}{1.076556in}}%
\pgfpathlineto{\pgfqpoint{2.599536in}{1.079370in}}%
\pgfpathlineto{\pgfqpoint{2.600319in}{1.080308in}}%
\pgfpathlineto{\pgfqpoint{2.603454in}{1.080308in}}%
\pgfpathlineto{\pgfqpoint{2.605805in}{1.083122in}}%
\pgfpathlineto{\pgfqpoint{2.606589in}{1.084059in}}%
\pgfpathlineto{\pgfqpoint{2.609724in}{1.084059in}}%
\pgfpathlineto{\pgfqpoint{2.612858in}{1.084059in}}%
\pgfpathlineto{\pgfqpoint{2.615209in}{1.086873in}}%
\pgfpathlineto{\pgfqpoint{2.615993in}{1.087811in}}%
\pgfpathlineto{\pgfqpoint{2.619128in}{1.087811in}}%
\pgfpathlineto{\pgfqpoint{2.622262in}{1.087811in}}%
\pgfpathlineto{\pgfqpoint{2.624614in}{1.090625in}}%
\pgfpathlineto{\pgfqpoint{2.625397in}{1.091563in}}%
\pgfpathlineto{\pgfqpoint{2.628532in}{1.091563in}}%
\pgfpathlineto{\pgfqpoint{2.631667in}{1.091563in}}%
\pgfpathlineto{\pgfqpoint{2.634018in}{1.094376in}}%
\pgfpathlineto{\pgfqpoint{2.634801in}{1.095314in}}%
\pgfpathlineto{\pgfqpoint{2.637936in}{1.095314in}}%
\pgfpathlineto{\pgfqpoint{2.641071in}{1.095314in}}%
\pgfpathlineto{\pgfqpoint{2.643422in}{1.098128in}}%
\pgfpathlineto{\pgfqpoint{2.644206in}{1.099066in}}%
\pgfpathlineto{\pgfqpoint{2.647340in}{1.099066in}}%
\pgfpathlineto{\pgfqpoint{2.649691in}{1.101879in}}%
\pgfpathlineto{\pgfqpoint{2.650475in}{1.102817in}}%
\pgfpathlineto{\pgfqpoint{2.653610in}{1.102817in}}%
\pgfpathlineto{\pgfqpoint{2.656745in}{1.102817in}}%
\pgfpathlineto{\pgfqpoint{2.659096in}{1.105631in}}%
\pgfpathlineto{\pgfqpoint{2.659879in}{1.106569in}}%
\pgfpathlineto{\pgfqpoint{2.663014in}{1.106569in}}%
\pgfpathlineto{\pgfqpoint{2.666149in}{1.106569in}}%
\pgfpathlineto{\pgfqpoint{2.668500in}{1.109382in}}%
\pgfpathlineto{\pgfqpoint{2.669284in}{1.110320in}}%
\pgfpathlineto{\pgfqpoint{2.672418in}{1.110320in}}%
\pgfpathlineto{\pgfqpoint{2.675553in}{1.110320in}}%
\pgfpathlineto{\pgfqpoint{2.677904in}{1.113134in}}%
\pgfpathlineto{\pgfqpoint{2.678688in}{1.114072in}}%
\pgfpathlineto{\pgfqpoint{2.681823in}{1.114072in}}%
\pgfpathlineto{\pgfqpoint{2.684174in}{1.116885in}}%
\pgfpathlineto{\pgfqpoint{2.684957in}{1.117823in}}%
\pgfpathlineto{\pgfqpoint{2.688092in}{1.117823in}}%
\pgfpathlineto{\pgfqpoint{2.691227in}{1.117823in}}%
\pgfpathlineto{\pgfqpoint{2.693578in}{1.120637in}}%
\pgfpathlineto{\pgfqpoint{2.694362in}{1.121575in}}%
\pgfpathlineto{\pgfqpoint{2.697496in}{1.121575in}}%
\pgfpathlineto{\pgfqpoint{2.700631in}{1.121575in}}%
\pgfpathlineto{\pgfqpoint{2.702982in}{1.124388in}}%
\pgfpathlineto{\pgfqpoint{2.703766in}{1.125326in}}%
\pgfpathlineto{\pgfqpoint{2.706901in}{1.125326in}}%
\pgfpathlineto{\pgfqpoint{2.710035in}{1.125326in}}%
\pgfpathlineto{\pgfqpoint{2.712386in}{1.128140in}}%
\pgfpathlineto{\pgfqpoint{2.713170in}{1.129078in}}%
\pgfpathlineto{\pgfqpoint{2.716305in}{1.129078in}}%
\pgfpathlineto{\pgfqpoint{2.718656in}{1.131892in}}%
\pgfpathlineto{\pgfqpoint{2.719439in}{1.132829in}}%
\pgfpathlineto{\pgfqpoint{2.722574in}{1.132829in}}%
\pgfpathlineto{\pgfqpoint{2.725709in}{1.132829in}}%
\pgfpathlineto{\pgfqpoint{2.728060in}{1.135643in}}%
\pgfpathlineto{\pgfqpoint{2.728844in}{1.136581in}}%
\pgfpathlineto{\pgfqpoint{2.731978in}{1.136581in}}%
\pgfpathlineto{\pgfqpoint{2.735113in}{1.136581in}}%
\pgfpathlineto{\pgfqpoint{2.737464in}{1.139395in}}%
\pgfpathlineto{\pgfqpoint{2.738248in}{1.140333in}}%
\pgfpathlineto{\pgfqpoint{2.741383in}{1.140333in}}%
\pgfpathlineto{\pgfqpoint{2.744517in}{1.140333in}}%
\pgfpathlineto{\pgfqpoint{2.746868in}{1.143146in}}%
\pgfpathlineto{\pgfqpoint{2.747652in}{1.144084in}}%
\pgfpathlineto{\pgfqpoint{2.750787in}{1.144084in}}%
\pgfpathlineto{\pgfqpoint{2.753922in}{1.144084in}}%
\pgfpathlineto{\pgfqpoint{2.756273in}{1.146898in}}%
\pgfpathlineto{\pgfqpoint{2.757056in}{1.147836in}}%
\pgfpathlineto{\pgfqpoint{2.760191in}{1.147836in}}%
\pgfpathlineto{\pgfqpoint{2.762542in}{1.150649in}}%
\pgfpathlineto{\pgfqpoint{2.763326in}{1.151587in}}%
\pgfpathlineto{\pgfqpoint{2.766461in}{1.151587in}}%
\pgfpathlineto{\pgfqpoint{2.769595in}{1.151587in}}%
\pgfpathlineto{\pgfqpoint{2.771946in}{1.154401in}}%
\pgfpathlineto{\pgfqpoint{2.772730in}{1.155339in}}%
\pgfpathlineto{\pgfqpoint{2.775865in}{1.155339in}}%
\pgfpathlineto{\pgfqpoint{2.779000in}{1.155339in}}%
\pgfpathlineto{\pgfqpoint{2.781351in}{1.158152in}}%
\pgfpathlineto{\pgfqpoint{2.782134in}{1.159090in}}%
\pgfpathlineto{\pgfqpoint{2.785269in}{1.159090in}}%
\pgfpathlineto{\pgfqpoint{2.788404in}{1.159090in}}%
\pgfpathlineto{\pgfqpoint{2.790755in}{1.161904in}}%
\pgfpathlineto{\pgfqpoint{2.791539in}{1.162842in}}%
\pgfpathlineto{\pgfqpoint{2.794673in}{1.162842in}}%
\pgfpathlineto{\pgfqpoint{2.797024in}{1.165655in}}%
\pgfpathlineto{\pgfqpoint{2.797808in}{1.166593in}}%
\pgfpathlineto{\pgfqpoint{2.800943in}{1.166593in}}%
\pgfpathlineto{\pgfqpoint{2.804078in}{1.166593in}}%
\pgfpathlineto{\pgfqpoint{2.806429in}{1.169407in}}%
\pgfpathlineto{\pgfqpoint{2.807212in}{1.170345in}}%
\pgfpathlineto{\pgfqpoint{2.810347in}{1.170345in}}%
\pgfpathlineto{\pgfqpoint{2.813482in}{1.170345in}}%
\pgfpathlineto{\pgfqpoint{2.815833in}{1.173158in}}%
\pgfpathlineto{\pgfqpoint{2.815833in}{1.176910in}}%
\pgfpathlineto{\pgfqpoint{2.816617in}{1.177848in}}%
\pgfpathlineto{\pgfqpoint{2.818968in}{1.180662in}}%
\pgfpathlineto{\pgfqpoint{2.818968in}{1.184413in}}%
\pgfpathlineto{\pgfqpoint{2.818968in}{1.188165in}}%
\pgfpathlineto{\pgfqpoint{2.819751in}{1.189102in}}%
\pgfpathlineto{\pgfqpoint{2.822102in}{1.191916in}}%
\pgfpathlineto{\pgfqpoint{2.822102in}{1.195668in}}%
\pgfpathlineto{\pgfqpoint{2.822102in}{1.199419in}}%
\pgfpathlineto{\pgfqpoint{2.822886in}{1.200357in}}%
\pgfpathlineto{\pgfqpoint{2.825237in}{1.203171in}}%
\pgfpathlineto{\pgfqpoint{2.825237in}{1.206922in}}%
\pgfpathlineto{\pgfqpoint{2.825237in}{1.210674in}}%
\pgfpathlineto{\pgfqpoint{2.826021in}{1.211612in}}%
\pgfpathlineto{\pgfqpoint{2.828372in}{1.214425in}}%
\pgfpathlineto{\pgfqpoint{2.828372in}{1.218177in}}%
\pgfpathlineto{\pgfqpoint{2.828372in}{1.221928in}}%
\pgfpathlineto{\pgfqpoint{2.829155in}{1.222866in}}%
\pgfpathlineto{\pgfqpoint{2.831507in}{1.225680in}}%
\pgfpathlineto{\pgfqpoint{2.831507in}{1.229431in}}%
\pgfpathlineto{\pgfqpoint{2.831507in}{1.233183in}}%
\pgfpathlineto{\pgfqpoint{2.832290in}{1.234121in}}%
\pgfpathlineto{\pgfqpoint{2.834641in}{1.236935in}}%
\pgfpathlineto{\pgfqpoint{2.834641in}{1.240686in}}%
\pgfpathlineto{\pgfqpoint{2.834641in}{1.244438in}}%
\pgfpathlineto{\pgfqpoint{2.835425in}{1.245375in}}%
\pgfpathlineto{\pgfqpoint{2.837776in}{1.248189in}}%
\pgfpathlineto{\pgfqpoint{2.837776in}{1.251941in}}%
\pgfpathlineto{\pgfqpoint{2.837776in}{1.255692in}}%
\pgfpathlineto{\pgfqpoint{2.837776in}{1.259444in}}%
\pgfpathlineto{\pgfqpoint{2.838560in}{1.260382in}}%
\pgfpathlineto{\pgfqpoint{2.840911in}{1.263195in}}%
\pgfpathlineto{\pgfqpoint{2.840911in}{1.266947in}}%
\pgfpathlineto{\pgfqpoint{2.840911in}{1.270698in}}%
\pgfpathlineto{\pgfqpoint{2.841694in}{1.271636in}}%
\pgfpathlineto{\pgfqpoint{2.844046in}{1.274450in}}%
\pgfpathlineto{\pgfqpoint{2.844046in}{1.278201in}}%
\pgfpathlineto{\pgfqpoint{2.844046in}{1.281953in}}%
\pgfpathlineto{\pgfqpoint{2.844829in}{1.282891in}}%
\pgfpathlineto{\pgfqpoint{2.847180in}{1.285704in}}%
\pgfpathlineto{\pgfqpoint{2.847180in}{1.289456in}}%
\pgfpathlineto{\pgfqpoint{2.847180in}{1.293208in}}%
\pgfpathlineto{\pgfqpoint{2.847964in}{1.294145in}}%
\pgfpathlineto{\pgfqpoint{2.850315in}{1.296959in}}%
\pgfpathlineto{\pgfqpoint{2.850315in}{1.300711in}}%
\pgfpathlineto{\pgfqpoint{2.850315in}{1.304462in}}%
\pgfpathlineto{\pgfqpoint{2.851099in}{1.305400in}}%
\pgfpathlineto{\pgfqpoint{2.853450in}{1.308214in}}%
\pgfpathlineto{\pgfqpoint{2.853450in}{1.311965in}}%
\pgfpathlineto{\pgfqpoint{2.853450in}{1.315717in}}%
\pgfpathlineto{\pgfqpoint{2.854233in}{1.316655in}}%
\pgfpathlineto{\pgfqpoint{2.856584in}{1.319468in}}%
\pgfpathlineto{\pgfqpoint{2.856584in}{1.323220in}}%
\pgfpathlineto{\pgfqpoint{2.856584in}{1.326971in}}%
\pgfpathlineto{\pgfqpoint{2.857368in}{1.327909in}}%
\pgfpathlineto{\pgfqpoint{2.859719in}{1.330723in}}%
\pgfpathlineto{\pgfqpoint{2.859719in}{1.334474in}}%
\pgfpathlineto{\pgfqpoint{2.859719in}{1.338226in}}%
\pgfpathlineto{\pgfqpoint{2.860503in}{1.339164in}}%
\pgfpathlineto{\pgfqpoint{2.862854in}{1.341977in}}%
\pgfpathlineto{\pgfqpoint{2.862854in}{1.345729in}}%
\pgfpathlineto{\pgfqpoint{2.862854in}{1.349481in}}%
\pgfpathlineto{\pgfqpoint{2.863638in}{1.350418in}}%
\pgfpathlineto{\pgfqpoint{2.865989in}{1.353232in}}%
\pgfpathlineto{\pgfqpoint{2.865989in}{1.356984in}}%
\pgfpathlineto{\pgfqpoint{2.865989in}{1.360735in}}%
\pgfpathlineto{\pgfqpoint{2.865989in}{1.364487in}}%
\pgfpathlineto{\pgfqpoint{2.866772in}{1.365425in}}%
\pgfpathlineto{\pgfqpoint{2.869123in}{1.368238in}}%
\pgfpathlineto{\pgfqpoint{2.869123in}{1.371990in}}%
\pgfpathlineto{\pgfqpoint{2.869123in}{1.375741in}}%
\pgfpathlineto{\pgfqpoint{2.869907in}{1.376679in}}%
\pgfpathlineto{\pgfqpoint{2.872258in}{1.379493in}}%
\pgfpathlineto{\pgfqpoint{2.872258in}{1.383244in}}%
\pgfpathlineto{\pgfqpoint{2.872258in}{1.386996in}}%
\pgfpathlineto{\pgfqpoint{2.873042in}{1.387934in}}%
\pgfpathlineto{\pgfqpoint{2.875393in}{1.390747in}}%
\pgfpathlineto{\pgfqpoint{2.875393in}{1.394499in}}%
\pgfpathlineto{\pgfqpoint{2.875393in}{1.398251in}}%
\pgfpathlineto{\pgfqpoint{2.876177in}{1.399188in}}%
\pgfpathlineto{\pgfqpoint{2.878528in}{1.402002in}}%
\pgfpathlineto{\pgfqpoint{2.878528in}{1.405754in}}%
\pgfpathlineto{\pgfqpoint{2.878528in}{1.409505in}}%
\pgfpathlineto{\pgfqpoint{2.879311in}{1.410443in}}%
\pgfpathlineto{\pgfqpoint{2.881662in}{1.413257in}}%
\pgfpathlineto{\pgfqpoint{2.881662in}{1.417008in}}%
\pgfpathlineto{\pgfqpoint{2.881662in}{1.420760in}}%
\pgfpathlineto{\pgfqpoint{2.882446in}{1.421698in}}%
\pgfpathlineto{\pgfqpoint{2.884797in}{1.424511in}}%
\pgfpathlineto{\pgfqpoint{2.884797in}{1.428263in}}%
\pgfpathlineto{\pgfqpoint{2.884797in}{1.432014in}}%
\pgfpathlineto{\pgfqpoint{2.885581in}{1.432952in}}%
\pgfpathlineto{\pgfqpoint{2.887932in}{1.435766in}}%
\pgfpathlineto{\pgfqpoint{2.887932in}{1.439517in}}%
\pgfpathlineto{\pgfqpoint{2.887932in}{1.443269in}}%
\pgfpathlineto{\pgfqpoint{2.888716in}{1.444207in}}%
\pgfpathlineto{\pgfqpoint{2.891067in}{1.447020in}}%
\pgfpathlineto{\pgfqpoint{2.891067in}{1.450772in}}%
\pgfpathlineto{\pgfqpoint{2.891067in}{1.454524in}}%
\pgfpathlineto{\pgfqpoint{2.891067in}{1.458275in}}%
\pgfpathlineto{\pgfqpoint{2.891850in}{1.459213in}}%
\pgfpathlineto{\pgfqpoint{2.894201in}{1.462027in}}%
\pgfpathlineto{\pgfqpoint{2.894201in}{1.465778in}}%
\pgfpathlineto{\pgfqpoint{2.894201in}{1.469530in}}%
\pgfpathlineto{\pgfqpoint{2.894985in}{1.470468in}}%
\pgfpathlineto{\pgfqpoint{2.897336in}{1.473281in}}%
\pgfpathlineto{\pgfqpoint{2.897336in}{1.477033in}}%
\pgfpathlineto{\pgfqpoint{2.897336in}{1.480784in}}%
\pgfpathlineto{\pgfqpoint{2.898120in}{1.481722in}}%
\pgfpathlineto{\pgfqpoint{2.900471in}{1.484536in}}%
\pgfpathlineto{\pgfqpoint{2.900471in}{1.488287in}}%
\pgfpathlineto{\pgfqpoint{2.900471in}{1.492039in}}%
\pgfpathlineto{\pgfqpoint{2.901255in}{1.492977in}}%
\pgfpathlineto{\pgfqpoint{2.903606in}{1.495790in}}%
\pgfpathlineto{\pgfqpoint{2.903606in}{1.499542in}}%
\pgfpathlineto{\pgfqpoint{2.903606in}{1.503293in}}%
\pgfpathlineto{\pgfqpoint{2.904389in}{1.504231in}}%
\pgfpathlineto{\pgfqpoint{2.906740in}{1.507045in}}%
\pgfpathlineto{\pgfqpoint{2.906740in}{1.510797in}}%
\pgfpathlineto{\pgfqpoint{2.906740in}{1.514548in}}%
\pgfpathlineto{\pgfqpoint{2.907524in}{1.515486in}}%
\pgfpathlineto{\pgfqpoint{2.909875in}{1.518300in}}%
\pgfpathlineto{\pgfqpoint{2.909875in}{1.522051in}}%
\pgfpathlineto{\pgfqpoint{2.909875in}{1.525803in}}%
\pgfpathlineto{\pgfqpoint{2.910659in}{1.526741in}}%
\pgfpathlineto{\pgfqpoint{2.913010in}{1.529554in}}%
\pgfpathlineto{\pgfqpoint{2.913010in}{1.533306in}}%
\pgfpathlineto{\pgfqpoint{2.913010in}{1.537057in}}%
\pgfpathlineto{\pgfqpoint{2.913794in}{1.537995in}}%
\pgfpathlineto{\pgfqpoint{2.916145in}{1.540809in}}%
\pgfpathlineto{\pgfqpoint{2.916145in}{1.544560in}}%
\pgfpathlineto{\pgfqpoint{2.916145in}{1.548312in}}%
\pgfpathlineto{\pgfqpoint{2.916928in}{1.549250in}}%
\pgfpathlineto{\pgfqpoint{2.919279in}{1.552063in}}%
\pgfpathlineto{\pgfqpoint{2.919279in}{1.555815in}}%
\pgfpathlineto{\pgfqpoint{2.919279in}{1.559566in}}%
\pgfpathlineto{\pgfqpoint{2.919279in}{1.563318in}}%
\pgfpathlineto{\pgfqpoint{2.920063in}{1.564256in}}%
\pgfpathlineto{\pgfqpoint{2.922414in}{1.567070in}}%
\pgfpathlineto{\pgfqpoint{2.922414in}{1.570821in}}%
\pgfpathlineto{\pgfqpoint{2.922414in}{1.574573in}}%
\pgfpathlineto{\pgfqpoint{2.923198in}{1.575511in}}%
\pgfpathlineto{\pgfqpoint{2.925549in}{1.578324in}}%
\pgfpathlineto{\pgfqpoint{2.925549in}{1.582076in}}%
\pgfpathlineto{\pgfqpoint{2.925549in}{1.585827in}}%
\pgfpathlineto{\pgfqpoint{2.926332in}{1.586765in}}%
\pgfpathlineto{\pgfqpoint{2.928684in}{1.589579in}}%
\pgfpathlineto{\pgfqpoint{2.928684in}{1.593330in}}%
\pgfpathlineto{\pgfqpoint{2.928684in}{1.597082in}}%
\pgfpathlineto{\pgfqpoint{2.929467in}{1.598020in}}%
\pgfpathlineto{\pgfqpoint{2.931818in}{1.600833in}}%
\pgfpathlineto{\pgfqpoint{2.931818in}{1.604585in}}%
\pgfpathlineto{\pgfqpoint{2.931818in}{1.608336in}}%
\pgfpathlineto{\pgfqpoint{2.932602in}{1.609274in}}%
\pgfpathlineto{\pgfqpoint{2.934953in}{1.612088in}}%
\pgfpathlineto{\pgfqpoint{2.934953in}{1.615840in}}%
\pgfpathlineto{\pgfqpoint{2.934953in}{1.619591in}}%
\pgfpathlineto{\pgfqpoint{2.935737in}{1.620529in}}%
\pgfpathlineto{\pgfqpoint{2.938088in}{1.623343in}}%
\pgfpathlineto{\pgfqpoint{2.938088in}{1.627094in}}%
\pgfpathlineto{\pgfqpoint{2.938088in}{1.630846in}}%
\pgfpathlineto{\pgfqpoint{2.938871in}{1.631784in}}%
\pgfpathlineto{\pgfqpoint{2.941223in}{1.634597in}}%
\pgfpathlineto{\pgfqpoint{2.941223in}{1.638349in}}%
\pgfpathlineto{\pgfqpoint{2.941223in}{1.642100in}}%
\pgfpathlineto{\pgfqpoint{2.942006in}{1.643038in}}%
\pgfpathlineto{\pgfqpoint{2.944357in}{1.645852in}}%
\pgfpathlineto{\pgfqpoint{2.944357in}{1.649603in}}%
\pgfpathlineto{\pgfqpoint{2.944357in}{1.653355in}}%
\pgfpathlineto{\pgfqpoint{2.944357in}{1.657106in}}%
\pgfpathlineto{\pgfqpoint{2.945141in}{1.658044in}}%
\pgfpathlineto{\pgfqpoint{2.947492in}{1.660858in}}%
\pgfpathlineto{\pgfqpoint{2.947492in}{1.664609in}}%
\pgfpathlineto{\pgfqpoint{2.947492in}{1.668361in}}%
\pgfpathlineto{\pgfqpoint{2.948276in}{1.669299in}}%
\pgfpathlineto{\pgfqpoint{2.950627in}{1.672113in}}%
\pgfpathlineto{\pgfqpoint{2.950627in}{1.675864in}}%
\pgfpathlineto{\pgfqpoint{2.950627in}{1.679616in}}%
\pgfpathlineto{\pgfqpoint{2.951410in}{1.680553in}}%
\pgfpathlineto{\pgfqpoint{2.953761in}{1.683367in}}%
\pgfpathlineto{\pgfqpoint{2.953761in}{1.687119in}}%
\pgfpathlineto{\pgfqpoint{2.953761in}{1.690870in}}%
\pgfpathlineto{\pgfqpoint{2.954545in}{1.691808in}}%
\pgfpathlineto{\pgfqpoint{2.956896in}{1.694622in}}%
\pgfpathlineto{\pgfqpoint{2.956896in}{1.698373in}}%
\pgfpathlineto{\pgfqpoint{2.956896in}{1.702125in}}%
\pgfpathlineto{\pgfqpoint{2.957680in}{1.703063in}}%
\pgfpathlineto{\pgfqpoint{2.960031in}{1.705876in}}%
\pgfpathlineto{\pgfqpoint{2.960031in}{1.709628in}}%
\pgfpathlineto{\pgfqpoint{2.960031in}{1.713379in}}%
\pgfpathlineto{\pgfqpoint{2.960815in}{1.714317in}}%
\pgfpathlineto{\pgfqpoint{2.963166in}{1.717131in}}%
\pgfpathlineto{\pgfqpoint{2.963166in}{1.720882in}}%
\pgfpathlineto{\pgfqpoint{2.963166in}{1.724634in}}%
\pgfpathlineto{\pgfqpoint{2.963949in}{1.725572in}}%
\pgfpathlineto{\pgfqpoint{2.966300in}{1.728386in}}%
\pgfpathlineto{\pgfqpoint{2.966300in}{1.732137in}}%
\pgfpathlineto{\pgfqpoint{2.966300in}{1.735889in}}%
\pgfpathlineto{\pgfqpoint{2.967084in}{1.736827in}}%
\pgfpathlineto{\pgfqpoint{2.969435in}{1.739640in}}%
\pgfpathlineto{\pgfqpoint{2.969435in}{1.743392in}}%
\pgfpathlineto{\pgfqpoint{2.969435in}{1.747143in}}%
\pgfpathlineto{\pgfqpoint{2.970219in}{1.748081in}}%
\pgfpathlineto{\pgfqpoint{2.972570in}{1.750895in}}%
\pgfpathlineto{\pgfqpoint{2.972570in}{1.754646in}}%
\pgfpathlineto{\pgfqpoint{2.972570in}{1.758398in}}%
\pgfpathlineto{\pgfqpoint{2.972570in}{1.762149in}}%
\pgfpathlineto{\pgfqpoint{2.973354in}{1.763087in}}%
\pgfpathlineto{\pgfqpoint{2.975705in}{1.765901in}}%
\pgfpathlineto{\pgfqpoint{2.975705in}{1.769652in}}%
\pgfpathlineto{\pgfqpoint{2.975705in}{1.773404in}}%
\pgfpathlineto{\pgfqpoint{2.976488in}{1.774342in}}%
\pgfpathlineto{\pgfqpoint{2.978839in}{1.777155in}}%
\pgfpathlineto{\pgfqpoint{2.978839in}{1.780907in}}%
\pgfpathlineto{\pgfqpoint{2.978839in}{1.784659in}}%
\pgfpathlineto{\pgfqpoint{2.979623in}{1.785596in}}%
\pgfpathlineto{\pgfqpoint{2.981974in}{1.788410in}}%
\pgfpathlineto{\pgfqpoint{2.981974in}{1.792162in}}%
\pgfpathlineto{\pgfqpoint{2.981974in}{1.795913in}}%
\pgfpathlineto{\pgfqpoint{2.982758in}{1.796851in}}%
\pgfpathlineto{\pgfqpoint{2.985109in}{1.799665in}}%
\pgfpathlineto{\pgfqpoint{2.985109in}{1.803416in}}%
\pgfpathlineto{\pgfqpoint{2.985109in}{1.807168in}}%
\pgfpathlineto{\pgfqpoint{2.985893in}{1.808106in}}%
\pgfpathlineto{\pgfqpoint{2.988244in}{1.810919in}}%
\pgfpathlineto{\pgfqpoint{2.988244in}{1.814671in}}%
\pgfpathlineto{\pgfqpoint{2.988244in}{1.818422in}}%
\pgfpathlineto{\pgfqpoint{2.989027in}{1.819360in}}%
\pgfpathlineto{\pgfqpoint{2.991378in}{1.822174in}}%
\pgfpathlineto{\pgfqpoint{2.991378in}{1.825925in}}%
\pgfpathlineto{\pgfqpoint{2.991378in}{1.829677in}}%
\pgfpathlineto{\pgfqpoint{2.992162in}{1.830615in}}%
\pgfpathlineto{\pgfqpoint{2.994513in}{1.833429in}}%
\pgfpathlineto{\pgfqpoint{2.994513in}{1.837180in}}%
\pgfpathlineto{\pgfqpoint{2.994513in}{1.840932in}}%
\pgfpathlineto{\pgfqpoint{2.995297in}{1.841869in}}%
\pgfpathlineto{\pgfqpoint{2.997648in}{1.844683in}}%
\pgfpathlineto{\pgfqpoint{2.997648in}{1.848435in}}%
\pgfpathlineto{\pgfqpoint{2.997648in}{1.852186in}}%
\pgfpathlineto{\pgfqpoint{2.997648in}{1.855938in}}%
\pgfpathlineto{\pgfqpoint{2.998432in}{1.856876in}}%
\pgfpathlineto{\pgfqpoint{3.000783in}{1.859689in}}%
\pgfpathlineto{\pgfqpoint{3.000783in}{1.863441in}}%
\pgfpathlineto{\pgfqpoint{3.000783in}{1.867192in}}%
\pgfpathlineto{\pgfqpoint{3.001566in}{1.868130in}}%
\pgfpathlineto{\pgfqpoint{3.003917in}{1.870944in}}%
\pgfpathlineto{\pgfqpoint{3.003917in}{1.874695in}}%
\pgfpathlineto{\pgfqpoint{3.003917in}{1.878447in}}%
\pgfpathlineto{\pgfqpoint{3.004701in}{1.879385in}}%
\pgfpathlineto{\pgfqpoint{3.007052in}{1.882198in}}%
\pgfpathlineto{\pgfqpoint{3.007052in}{1.885950in}}%
\pgfpathlineto{\pgfqpoint{3.007052in}{1.889702in}}%
\pgfpathlineto{\pgfqpoint{3.007836in}{1.890639in}}%
\pgfpathlineto{\pgfqpoint{3.010187in}{1.893453in}}%
\pgfpathlineto{\pgfqpoint{3.010187in}{1.897205in}}%
\pgfpathlineto{\pgfqpoint{3.010187in}{1.900956in}}%
\pgfpathlineto{\pgfqpoint{3.010971in}{1.901894in}}%
\pgfpathlineto{\pgfqpoint{3.013322in}{1.904708in}}%
\pgfpathlineto{\pgfqpoint{3.013322in}{1.908459in}}%
\pgfpathlineto{\pgfqpoint{3.013322in}{1.912211in}}%
\pgfpathlineto{\pgfqpoint{3.014105in}{1.913149in}}%
\pgfpathlineto{\pgfqpoint{3.016456in}{1.915962in}}%
\pgfpathlineto{\pgfqpoint{3.016456in}{1.919714in}}%
\pgfpathlineto{\pgfqpoint{3.016456in}{1.923465in}}%
\pgfpathlineto{\pgfqpoint{3.017240in}{1.924403in}}%
\pgfpathlineto{\pgfqpoint{3.019591in}{1.927217in}}%
\pgfpathlineto{\pgfqpoint{3.019591in}{1.930968in}}%
\pgfpathlineto{\pgfqpoint{3.019591in}{1.934720in}}%
\pgfpathlineto{\pgfqpoint{3.020375in}{1.935658in}}%
\pgfpathlineto{\pgfqpoint{3.022726in}{1.938471in}}%
\pgfpathlineto{\pgfqpoint{3.022726in}{1.942223in}}%
\pgfpathlineto{\pgfqpoint{3.022726in}{1.945975in}}%
\pgfpathlineto{\pgfqpoint{3.023510in}{1.946912in}}%
\pgfpathlineto{\pgfqpoint{3.025861in}{1.949726in}}%
\pgfpathlineto{\pgfqpoint{3.025861in}{1.953478in}}%
\pgfpathlineto{\pgfqpoint{3.025861in}{1.957229in}}%
\pgfpathlineto{\pgfqpoint{3.025861in}{1.960981in}}%
\pgfpathlineto{\pgfqpoint{3.026644in}{1.961919in}}%
\pgfpathlineto{\pgfqpoint{3.028995in}{1.964732in}}%
\pgfpathlineto{\pgfqpoint{3.028995in}{1.968484in}}%
\pgfpathlineto{\pgfqpoint{3.028995in}{1.972235in}}%
\pgfpathlineto{\pgfqpoint{3.029779in}{1.973173in}}%
\pgfpathlineto{\pgfqpoint{3.032130in}{1.975987in}}%
\pgfpathlineto{\pgfqpoint{3.032130in}{1.979738in}}%
\pgfpathlineto{\pgfqpoint{3.032130in}{1.983490in}}%
\pgfpathlineto{\pgfqpoint{3.032914in}{1.984428in}}%
\pgfpathlineto{\pgfqpoint{3.035265in}{1.987241in}}%
\pgfpathlineto{\pgfqpoint{3.035265in}{1.990993in}}%
\pgfpathlineto{\pgfqpoint{3.035265in}{1.994745in}}%
\pgfpathlineto{\pgfqpoint{3.036048in}{1.995682in}}%
\pgfpathlineto{\pgfqpoint{3.038400in}{1.998496in}}%
\pgfpathlineto{\pgfqpoint{3.038400in}{2.002248in}}%
\pgfpathlineto{\pgfqpoint{3.038400in}{2.005999in}}%
\pgfpathlineto{\pgfqpoint{3.039183in}{2.006937in}}%
\pgfpathlineto{\pgfqpoint{3.041534in}{2.009751in}}%
\pgfpathlineto{\pgfqpoint{3.041534in}{2.013502in}}%
\pgfpathlineto{\pgfqpoint{3.041534in}{2.017254in}}%
\pgfpathlineto{\pgfqpoint{3.042318in}{2.018192in}}%
\pgfpathlineto{\pgfqpoint{3.044669in}{2.021005in}}%
\pgfpathlineto{\pgfqpoint{3.044669in}{2.024757in}}%
\pgfpathlineto{\pgfqpoint{3.044669in}{2.028508in}}%
\pgfpathlineto{\pgfqpoint{3.045453in}{2.029446in}}%
\pgfpathlineto{\pgfqpoint{3.047804in}{2.032260in}}%
\pgfpathlineto{\pgfqpoint{3.047804in}{2.036011in}}%
\pgfpathlineto{\pgfqpoint{3.047804in}{2.039763in}}%
\pgfpathlineto{\pgfqpoint{3.048587in}{2.040701in}}%
\pgfpathlineto{\pgfqpoint{3.050938in}{2.043514in}}%
\pgfpathlineto{\pgfqpoint{3.050938in}{2.047266in}}%
\pgfpathlineto{\pgfqpoint{3.050938in}{2.051018in}}%
\pgfpathlineto{\pgfqpoint{3.050938in}{2.054769in}}%
\pgfpathlineto{\pgfqpoint{3.051722in}{2.055707in}}%
\pgfpathlineto{\pgfqpoint{3.054073in}{2.058521in}}%
\pgfpathlineto{\pgfqpoint{3.054073in}{2.062272in}}%
\pgfpathlineto{\pgfqpoint{3.054073in}{2.066024in}}%
\pgfpathlineto{\pgfqpoint{3.054857in}{2.066962in}}%
\pgfpathlineto{\pgfqpoint{3.057208in}{2.069775in}}%
\pgfpathlineto{\pgfqpoint{3.057208in}{2.073527in}}%
\pgfpathlineto{\pgfqpoint{3.057208in}{2.077278in}}%
\pgfpathlineto{\pgfqpoint{3.057992in}{2.078216in}}%
\pgfpathlineto{\pgfqpoint{3.060343in}{2.081030in}}%
\pgfpathlineto{\pgfqpoint{3.060343in}{2.084781in}}%
\pgfpathlineto{\pgfqpoint{3.060343in}{2.088533in}}%
\pgfpathlineto{\pgfqpoint{3.061126in}{2.089471in}}%
\pgfpathlineto{\pgfqpoint{3.063477in}{2.092284in}}%
\pgfpathlineto{\pgfqpoint{3.063477in}{2.096036in}}%
\pgfpathlineto{\pgfqpoint{3.063477in}{2.099787in}}%
\pgfpathlineto{\pgfqpoint{3.064261in}{2.100725in}}%
\pgfpathlineto{\pgfqpoint{3.066612in}{2.103539in}}%
\pgfpathlineto{\pgfqpoint{3.066612in}{2.107291in}}%
\pgfpathlineto{\pgfqpoint{3.066612in}{2.111042in}}%
\pgfpathlineto{\pgfqpoint{3.067396in}{2.111980in}}%
\pgfpathlineto{\pgfqpoint{3.069747in}{2.114794in}}%
\pgfpathlineto{\pgfqpoint{3.069747in}{2.118545in}}%
\pgfpathlineto{\pgfqpoint{3.069747in}{2.122297in}}%
\pgfpathlineto{\pgfqpoint{3.070531in}{2.123235in}}%
\pgfpathlineto{\pgfqpoint{3.072882in}{2.126048in}}%
\pgfpathlineto{\pgfqpoint{3.072882in}{2.129800in}}%
\pgfpathlineto{\pgfqpoint{3.072882in}{2.133551in}}%
\pgfpathlineto{\pgfqpoint{3.073665in}{2.134489in}}%
\pgfpathlineto{\pgfqpoint{3.076016in}{2.137303in}}%
\pgfpathlineto{\pgfqpoint{3.076016in}{2.141054in}}%
\pgfpathlineto{\pgfqpoint{3.076016in}{2.144806in}}%
\pgfpathlineto{\pgfqpoint{3.076800in}{2.145744in}}%
\pgfpathlineto{\pgfqpoint{3.079151in}{2.148557in}}%
\pgfpathlineto{\pgfqpoint{3.079151in}{2.152309in}}%
\pgfpathlineto{\pgfqpoint{3.079151in}{2.156060in}}%
\pgfpathlineto{\pgfqpoint{3.079151in}{2.159812in}}%
\pgfpathlineto{\pgfqpoint{3.079935in}{2.160750in}}%
\pgfpathlineto{\pgfqpoint{3.082286in}{2.163564in}}%
\pgfpathlineto{\pgfqpoint{3.082286in}{2.167315in}}%
\pgfpathlineto{\pgfqpoint{3.082286in}{2.171067in}}%
\pgfpathlineto{\pgfqpoint{3.083070in}{2.172005in}}%
\pgfpathlineto{\pgfqpoint{3.085421in}{2.174818in}}%
\pgfpathlineto{\pgfqpoint{3.085421in}{2.178570in}}%
\pgfpathlineto{\pgfqpoint{3.085421in}{2.182321in}}%
\pgfpathlineto{\pgfqpoint{3.086204in}{2.183259in}}%
\pgfpathlineto{\pgfqpoint{3.088555in}{2.186073in}}%
\pgfpathlineto{\pgfqpoint{3.088555in}{2.189824in}}%
\pgfpathlineto{\pgfqpoint{3.088555in}{2.193576in}}%
\pgfpathlineto{\pgfqpoint{3.089339in}{2.194514in}}%
\pgfpathlineto{\pgfqpoint{3.091690in}{2.197327in}}%
\pgfpathlineto{\pgfqpoint{3.091690in}{2.201079in}}%
\pgfpathlineto{\pgfqpoint{3.091690in}{2.204830in}}%
\pgfpathlineto{\pgfqpoint{3.092474in}{2.205768in}}%
\pgfpathlineto{\pgfqpoint{3.094825in}{2.208582in}}%
\pgfpathlineto{\pgfqpoint{3.094825in}{2.212334in}}%
\pgfpathlineto{\pgfqpoint{3.092474in}{2.215147in}}%
\pgfpathlineto{\pgfqpoint{3.091690in}{2.216085in}}%
\pgfpathlineto{\pgfqpoint{3.089339in}{2.218899in}}%
\pgfpathlineto{\pgfqpoint{3.088555in}{2.219837in}}%
\pgfpathlineto{\pgfqpoint{3.086204in}{2.222650in}}%
\pgfpathlineto{\pgfqpoint{3.085421in}{2.223588in}}%
\pgfpathlineto{\pgfqpoint{3.083070in}{2.226402in}}%
\pgfpathlineto{\pgfqpoint{3.082286in}{2.227340in}}%
\pgfpathlineto{\pgfqpoint{3.079935in}{2.230153in}}%
\pgfpathlineto{\pgfqpoint{3.079151in}{2.231091in}}%
\pgfpathlineto{\pgfqpoint{3.076800in}{2.233905in}}%
\pgfpathlineto{\pgfqpoint{3.076016in}{2.234843in}}%
\pgfpathlineto{\pgfqpoint{3.073665in}{2.237656in}}%
\pgfpathlineto{\pgfqpoint{3.072882in}{2.238594in}}%
\pgfpathlineto{\pgfqpoint{3.070531in}{2.241408in}}%
\pgfpathlineto{\pgfqpoint{3.069747in}{2.242346in}}%
\pgfpathlineto{\pgfqpoint{3.067396in}{2.245159in}}%
\pgfpathlineto{\pgfqpoint{3.066612in}{2.246097in}}%
\pgfpathlineto{\pgfqpoint{3.064261in}{2.248911in}}%
\pgfpathlineto{\pgfqpoint{3.063477in}{2.249849in}}%
\pgfpathlineto{\pgfqpoint{3.061126in}{2.252663in}}%
\pgfpathlineto{\pgfqpoint{3.060343in}{2.253600in}}%
\pgfpathlineto{\pgfqpoint{3.057992in}{2.256414in}}%
\pgfpathlineto{\pgfqpoint{3.057208in}{2.257352in}}%
\pgfpathlineto{\pgfqpoint{3.054857in}{2.260166in}}%
\pgfpathlineto{\pgfqpoint{3.054073in}{2.261103in}}%
\pgfpathlineto{\pgfqpoint{3.054073in}{2.264855in}}%
\pgfpathlineto{\pgfqpoint{3.051722in}{2.267669in}}%
\pgfpathlineto{\pgfqpoint{3.050938in}{2.268607in}}%
\pgfpathlineto{\pgfqpoint{3.048587in}{2.271420in}}%
\pgfpathlineto{\pgfqpoint{3.047804in}{2.272358in}}%
\pgfpathlineto{\pgfqpoint{3.045453in}{2.275172in}}%
\pgfpathlineto{\pgfqpoint{3.044669in}{2.276110in}}%
\pgfpathlineto{\pgfqpoint{3.042318in}{2.278923in}}%
\pgfpathlineto{\pgfqpoint{3.041534in}{2.279861in}}%
\pgfpathlineto{\pgfqpoint{3.039183in}{2.282675in}}%
\pgfpathlineto{\pgfqpoint{3.038400in}{2.283613in}}%
\pgfpathlineto{\pgfqpoint{3.036048in}{2.286426in}}%
\pgfpathlineto{\pgfqpoint{3.035265in}{2.287364in}}%
\pgfpathlineto{\pgfqpoint{3.032914in}{2.290178in}}%
\pgfpathlineto{\pgfqpoint{3.032130in}{2.291116in}}%
\pgfpathlineto{\pgfqpoint{3.029779in}{2.293929in}}%
\pgfpathlineto{\pgfqpoint{3.028995in}{2.294867in}}%
\pgfpathlineto{\pgfqpoint{3.026644in}{2.297681in}}%
\pgfpathlineto{\pgfqpoint{3.025861in}{2.298619in}}%
\pgfpathlineto{\pgfqpoint{3.023510in}{2.301432in}}%
\pgfpathlineto{\pgfqpoint{3.022726in}{2.302370in}}%
\pgfpathlineto{\pgfqpoint{3.020375in}{2.305184in}}%
\pgfpathlineto{\pgfqpoint{3.019591in}{2.306122in}}%
\pgfpathlineto{\pgfqpoint{3.017240in}{2.308936in}}%
\pgfpathlineto{\pgfqpoint{3.016456in}{2.309873in}}%
\pgfpathlineto{\pgfqpoint{3.014105in}{2.312687in}}%
\pgfpathlineto{\pgfqpoint{3.013322in}{2.313625in}}%
\pgfpathlineto{\pgfqpoint{3.010971in}{2.316439in}}%
\pgfpathlineto{\pgfqpoint{3.010187in}{2.317376in}}%
\pgfpathlineto{\pgfqpoint{3.007836in}{2.320190in}}%
\pgfpathlineto{\pgfqpoint{3.007052in}{2.321128in}}%
\pgfpathlineto{\pgfqpoint{3.004701in}{2.323942in}}%
\pgfpathlineto{\pgfqpoint{3.003917in}{2.324880in}}%
\pgfpathlineto{\pgfqpoint{3.003917in}{2.328631in}}%
\pgfpathlineto{\pgfqpoint{3.001566in}{2.331445in}}%
\pgfpathlineto{\pgfqpoint{3.000783in}{2.332383in}}%
\pgfpathlineto{\pgfqpoint{2.998432in}{2.335196in}}%
\pgfpathlineto{\pgfqpoint{2.997648in}{2.336134in}}%
\pgfpathlineto{\pgfqpoint{2.995297in}{2.338948in}}%
\pgfpathlineto{\pgfqpoint{2.994513in}{2.339886in}}%
\pgfpathlineto{\pgfqpoint{2.992162in}{2.342699in}}%
\pgfpathlineto{\pgfqpoint{2.991378in}{2.343637in}}%
\pgfpathlineto{\pgfqpoint{2.989027in}{2.346451in}}%
\pgfpathlineto{\pgfqpoint{2.988244in}{2.347389in}}%
\pgfpathlineto{\pgfqpoint{2.985893in}{2.350202in}}%
\pgfpathlineto{\pgfqpoint{2.985109in}{2.351140in}}%
\pgfpathlineto{\pgfqpoint{2.982758in}{2.353954in}}%
\pgfpathlineto{\pgfqpoint{2.981974in}{2.354892in}}%
\pgfpathlineto{\pgfqpoint{2.979623in}{2.357705in}}%
\pgfpathlineto{\pgfqpoint{2.978839in}{2.358643in}}%
\pgfpathlineto{\pgfqpoint{2.976488in}{2.361457in}}%
\pgfpathlineto{\pgfqpoint{2.975705in}{2.362395in}}%
\pgfpathlineto{\pgfqpoint{2.973354in}{2.365209in}}%
\pgfpathlineto{\pgfqpoint{2.972570in}{2.366146in}}%
\pgfpathlineto{\pgfqpoint{2.970219in}{2.368960in}}%
\pgfpathlineto{\pgfqpoint{2.969435in}{2.369898in}}%
\pgfpathlineto{\pgfqpoint{2.967084in}{2.372712in}}%
\pgfpathlineto{\pgfqpoint{2.966300in}{2.373649in}}%
\pgfpathlineto{\pgfqpoint{2.963949in}{2.376463in}}%
\pgfpathlineto{\pgfqpoint{2.963166in}{2.377401in}}%
\pgfpathlineto{\pgfqpoint{2.960815in}{2.380215in}}%
\pgfpathlineto{\pgfqpoint{2.960031in}{2.381153in}}%
\pgfpathlineto{\pgfqpoint{2.957680in}{2.383966in}}%
\pgfpathlineto{\pgfqpoint{2.956896in}{2.384904in}}%
\pgfpathlineto{\pgfqpoint{2.956896in}{2.388656in}}%
\pgfpathlineto{\pgfqpoint{2.954545in}{2.391469in}}%
\pgfpathlineto{\pgfqpoint{2.953761in}{2.392407in}}%
\pgfpathlineto{\pgfqpoint{2.951410in}{2.395221in}}%
\pgfpathlineto{\pgfqpoint{2.950627in}{2.396159in}}%
\pgfpathlineto{\pgfqpoint{2.948276in}{2.398972in}}%
\pgfpathlineto{\pgfqpoint{2.947492in}{2.399910in}}%
\pgfpathlineto{\pgfqpoint{2.945141in}{2.402724in}}%
\pgfpathlineto{\pgfqpoint{2.944357in}{2.403662in}}%
\pgfpathlineto{\pgfqpoint{2.942006in}{2.406475in}}%
\pgfpathlineto{\pgfqpoint{2.941223in}{2.407413in}}%
\pgfpathlineto{\pgfqpoint{2.938871in}{2.410227in}}%
\pgfpathlineto{\pgfqpoint{2.938088in}{2.411165in}}%
\pgfpathlineto{\pgfqpoint{2.935737in}{2.413978in}}%
\pgfpathlineto{\pgfqpoint{2.934953in}{2.414916in}}%
\pgfpathlineto{\pgfqpoint{2.932602in}{2.417730in}}%
\pgfpathlineto{\pgfqpoint{2.931818in}{2.418668in}}%
\pgfpathlineto{\pgfqpoint{2.929467in}{2.421482in}}%
\pgfpathlineto{\pgfqpoint{2.928684in}{2.422419in}}%
\pgfpathlineto{\pgfqpoint{2.926332in}{2.425233in}}%
\pgfpathlineto{\pgfqpoint{2.925549in}{2.426171in}}%
\pgfpathlineto{\pgfqpoint{2.923198in}{2.428985in}}%
\pgfpathlineto{\pgfqpoint{2.922414in}{2.429923in}}%
\pgfpathlineto{\pgfqpoint{2.920063in}{2.432736in}}%
\pgfpathlineto{\pgfqpoint{2.919279in}{2.433674in}}%
\pgfpathlineto{\pgfqpoint{2.916928in}{2.436488in}}%
\pgfpathlineto{\pgfqpoint{2.916145in}{2.437426in}}%
\pgfpathlineto{\pgfqpoint{2.913794in}{2.440239in}}%
\pgfpathlineto{\pgfqpoint{2.913010in}{2.441177in}}%
\pgfpathlineto{\pgfqpoint{2.910659in}{2.443991in}}%
\pgfpathlineto{\pgfqpoint{2.909875in}{2.444929in}}%
\pgfpathlineto{\pgfqpoint{2.907524in}{2.447742in}}%
\pgfpathlineto{\pgfqpoint{2.906740in}{2.448680in}}%
\pgfpathlineto{\pgfqpoint{2.906740in}{2.452432in}}%
\pgfpathlineto{\pgfqpoint{2.904389in}{2.455245in}}%
\pgfpathlineto{\pgfqpoint{2.903606in}{2.456183in}}%
\pgfpathlineto{\pgfqpoint{2.901255in}{2.458997in}}%
\pgfpathlineto{\pgfqpoint{2.900471in}{2.459935in}}%
\pgfpathlineto{\pgfqpoint{2.898120in}{2.462748in}}%
\pgfpathlineto{\pgfqpoint{2.897336in}{2.463686in}}%
\pgfpathlineto{\pgfqpoint{2.894985in}{2.466500in}}%
\pgfpathlineto{\pgfqpoint{2.894201in}{2.467438in}}%
\pgfpathlineto{\pgfqpoint{2.891850in}{2.470252in}}%
\pgfpathlineto{\pgfqpoint{2.891067in}{2.471189in}}%
\pgfpathlineto{\pgfqpoint{2.888716in}{2.474003in}}%
\pgfpathlineto{\pgfqpoint{2.887932in}{2.474941in}}%
\pgfpathlineto{\pgfqpoint{2.885581in}{2.477755in}}%
\pgfpathlineto{\pgfqpoint{2.884797in}{2.478692in}}%
\pgfpathlineto{\pgfqpoint{2.882446in}{2.481506in}}%
\pgfpathlineto{\pgfqpoint{2.881662in}{2.482444in}}%
\pgfpathlineto{\pgfqpoint{2.879311in}{2.485258in}}%
\pgfpathlineto{\pgfqpoint{2.878528in}{2.486196in}}%
\pgfpathlineto{\pgfqpoint{2.876177in}{2.489009in}}%
\pgfpathlineto{\pgfqpoint{2.875393in}{2.489947in}}%
\pgfpathlineto{\pgfqpoint{2.873042in}{2.492761in}}%
\pgfpathlineto{\pgfqpoint{2.872258in}{2.493699in}}%
\pgfpathlineto{\pgfqpoint{2.869907in}{2.496512in}}%
\pgfpathlineto{\pgfqpoint{2.869123in}{2.497450in}}%
\pgfpathlineto{\pgfqpoint{2.866772in}{2.500264in}}%
\pgfpathlineto{\pgfqpoint{2.865989in}{2.501202in}}%
\pgfpathlineto{\pgfqpoint{2.863638in}{2.504015in}}%
\pgfpathlineto{\pgfqpoint{2.862854in}{2.504953in}}%
\pgfpathlineto{\pgfqpoint{2.860503in}{2.507767in}}%
\pgfpathlineto{\pgfqpoint{2.859719in}{2.508705in}}%
\pgfpathlineto{\pgfqpoint{2.859719in}{2.512456in}}%
\pgfpathlineto{\pgfqpoint{2.857368in}{2.515270in}}%
\pgfpathlineto{\pgfqpoint{2.856584in}{2.516208in}}%
\pgfpathlineto{\pgfqpoint{2.854233in}{2.519021in}}%
\pgfpathlineto{\pgfqpoint{2.853450in}{2.519959in}}%
\pgfpathlineto{\pgfqpoint{2.851099in}{2.522773in}}%
\pgfpathlineto{\pgfqpoint{2.850315in}{2.523711in}}%
\pgfpathlineto{\pgfqpoint{2.847964in}{2.526525in}}%
\pgfpathlineto{\pgfqpoint{2.847180in}{2.527462in}}%
\pgfpathlineto{\pgfqpoint{2.844829in}{2.530276in}}%
\pgfpathlineto{\pgfqpoint{2.844046in}{2.531214in}}%
\pgfpathlineto{\pgfqpoint{2.841694in}{2.534028in}}%
\pgfpathlineto{\pgfqpoint{2.840911in}{2.534965in}}%
\pgfpathlineto{\pgfqpoint{2.838560in}{2.537779in}}%
\pgfpathlineto{\pgfqpoint{2.837776in}{2.538717in}}%
\pgfpathlineto{\pgfqpoint{2.835425in}{2.541531in}}%
\pgfpathlineto{\pgfqpoint{2.834641in}{2.542469in}}%
\pgfpathlineto{\pgfqpoint{2.832290in}{2.545282in}}%
\pgfpathlineto{\pgfqpoint{2.831507in}{2.546220in}}%
\pgfpathlineto{\pgfqpoint{2.829155in}{2.549034in}}%
\pgfpathlineto{\pgfqpoint{2.828372in}{2.549972in}}%
\pgfpathlineto{\pgfqpoint{2.826021in}{2.552785in}}%
\pgfpathlineto{\pgfqpoint{2.825237in}{2.553723in}}%
\pgfpathlineto{\pgfqpoint{2.822886in}{2.556537in}}%
\pgfpathlineto{\pgfqpoint{2.822102in}{2.557475in}}%
\pgfpathlineto{\pgfqpoint{2.819751in}{2.560288in}}%
\pgfpathlineto{\pgfqpoint{2.818968in}{2.561226in}}%
\pgfpathlineto{\pgfqpoint{2.816617in}{2.564040in}}%
\pgfpathlineto{\pgfqpoint{2.815833in}{2.564978in}}%
\pgfpathlineto{\pgfqpoint{2.813482in}{2.567791in}}%
\pgfpathlineto{\pgfqpoint{2.812698in}{2.568729in}}%
\pgfpathlineto{\pgfqpoint{2.810347in}{2.571543in}}%
\pgfpathlineto{\pgfqpoint{2.809563in}{2.572481in}}%
\pgfpathlineto{\pgfqpoint{2.809563in}{2.576232in}}%
\pgfpathlineto{\pgfqpoint{2.807212in}{2.579046in}}%
\pgfpathlineto{\pgfqpoint{2.806429in}{2.579984in}}%
\pgfpathlineto{\pgfqpoint{2.804078in}{2.582798in}}%
\pgfpathlineto{\pgfqpoint{2.803294in}{2.583735in}}%
\pgfpathlineto{\pgfqpoint{2.800943in}{2.586549in}}%
\pgfpathlineto{\pgfqpoint{2.800159in}{2.587487in}}%
\pgfpathlineto{\pgfqpoint{2.797808in}{2.590301in}}%
\pgfpathlineto{\pgfqpoint{2.797024in}{2.591238in}}%
\pgfpathlineto{\pgfqpoint{2.794673in}{2.594052in}}%
\pgfpathlineto{\pgfqpoint{2.793890in}{2.594990in}}%
\pgfpathlineto{\pgfqpoint{2.791539in}{2.597804in}}%
\pgfpathlineto{\pgfqpoint{2.790755in}{2.598742in}}%
\pgfpathlineto{\pgfqpoint{2.788404in}{2.601555in}}%
\pgfpathlineto{\pgfqpoint{2.787620in}{2.602493in}}%
\pgfpathlineto{\pgfqpoint{2.785269in}{2.605307in}}%
\pgfpathlineto{\pgfqpoint{2.784485in}{2.606245in}}%
\pgfpathlineto{\pgfqpoint{2.782134in}{2.609058in}}%
\pgfpathlineto{\pgfqpoint{2.781351in}{2.609996in}}%
\pgfpathlineto{\pgfqpoint{2.779000in}{2.612810in}}%
\pgfpathlineto{\pgfqpoint{2.778216in}{2.613748in}}%
\pgfpathlineto{\pgfqpoint{2.775865in}{2.616561in}}%
\pgfpathlineto{\pgfqpoint{2.775081in}{2.617499in}}%
\pgfpathlineto{\pgfqpoint{2.772730in}{2.620313in}}%
\pgfpathlineto{\pgfqpoint{2.771946in}{2.621251in}}%
\pgfpathlineto{\pgfqpoint{2.769595in}{2.624064in}}%
\pgfpathlineto{\pgfqpoint{2.768812in}{2.625002in}}%
\pgfpathlineto{\pgfqpoint{2.766461in}{2.627816in}}%
\pgfpathlineto{\pgfqpoint{2.765677in}{2.628754in}}%
\pgfpathlineto{\pgfqpoint{2.763326in}{2.631567in}}%
\pgfpathlineto{\pgfqpoint{2.762542in}{2.632505in}}%
\pgfpathlineto{\pgfqpoint{2.762542in}{2.636257in}}%
\pgfpathlineto{\pgfqpoint{2.760191in}{2.639071in}}%
\pgfpathlineto{\pgfqpoint{2.759407in}{2.640008in}}%
\pgfpathlineto{\pgfqpoint{2.757056in}{2.642822in}}%
\pgfpathlineto{\pgfqpoint{2.756273in}{2.643760in}}%
\pgfpathlineto{\pgfqpoint{2.753922in}{2.646574in}}%
\pgfpathlineto{\pgfqpoint{2.753138in}{2.647512in}}%
\pgfpathlineto{\pgfqpoint{2.750787in}{2.650325in}}%
\pgfpathlineto{\pgfqpoint{2.750003in}{2.651263in}}%
\pgfpathlineto{\pgfqpoint{2.747652in}{2.654077in}}%
\pgfpathlineto{\pgfqpoint{2.746868in}{2.655015in}}%
\pgfpathlineto{\pgfqpoint{2.744517in}{2.657828in}}%
\pgfpathlineto{\pgfqpoint{2.743734in}{2.658766in}}%
\pgfpathlineto{\pgfqpoint{2.741383in}{2.661580in}}%
\pgfpathlineto{\pgfqpoint{2.740599in}{2.662518in}}%
\pgfpathlineto{\pgfqpoint{2.738248in}{2.665331in}}%
\pgfpathlineto{\pgfqpoint{2.737464in}{2.666269in}}%
\pgfpathlineto{\pgfqpoint{2.735113in}{2.669083in}}%
\pgfpathlineto{\pgfqpoint{2.734330in}{2.670021in}}%
\pgfpathlineto{\pgfqpoint{2.731978in}{2.672834in}}%
\pgfpathlineto{\pgfqpoint{2.731195in}{2.673772in}}%
\pgfpathlineto{\pgfqpoint{2.728844in}{2.676586in}}%
\pgfpathlineto{\pgfqpoint{2.728060in}{2.677524in}}%
\pgfpathlineto{\pgfqpoint{2.725709in}{2.680337in}}%
\pgfpathlineto{\pgfqpoint{2.724925in}{2.681275in}}%
\pgfpathlineto{\pgfqpoint{2.722574in}{2.684089in}}%
\pgfpathlineto{\pgfqpoint{2.721791in}{2.685027in}}%
\pgfpathlineto{\pgfqpoint{2.719439in}{2.687841in}}%
\pgfpathlineto{\pgfqpoint{2.718656in}{2.688778in}}%
\pgfpathlineto{\pgfqpoint{2.716305in}{2.691592in}}%
\pgfpathlineto{\pgfqpoint{2.715521in}{2.692530in}}%
\pgfpathlineto{\pgfqpoint{2.713170in}{2.695344in}}%
\pgfpathlineto{\pgfqpoint{2.712386in}{2.696281in}}%
\pgfpathlineto{\pgfqpoint{2.712386in}{2.700033in}}%
\pgfpathlineto{\pgfqpoint{2.710035in}{2.702847in}}%
\pgfpathlineto{\pgfqpoint{2.709252in}{2.703785in}}%
\pgfpathlineto{\pgfqpoint{2.706901in}{2.706598in}}%
\pgfpathlineto{\pgfqpoint{2.706117in}{2.707536in}}%
\pgfpathlineto{\pgfqpoint{2.703766in}{2.710350in}}%
\pgfpathlineto{\pgfqpoint{2.702982in}{2.711288in}}%
\pgfpathlineto{\pgfqpoint{2.700631in}{2.714101in}}%
\pgfpathlineto{\pgfqpoint{2.699847in}{2.715039in}}%
\pgfpathlineto{\pgfqpoint{2.697496in}{2.717853in}}%
\pgfpathlineto{\pgfqpoint{2.696713in}{2.718791in}}%
\pgfpathlineto{\pgfqpoint{2.694362in}{2.721604in}}%
\pgfpathlineto{\pgfqpoint{2.693578in}{2.722542in}}%
\pgfpathlineto{\pgfqpoint{2.691227in}{2.725356in}}%
\pgfpathlineto{\pgfqpoint{2.690443in}{2.726294in}}%
\pgfpathlineto{\pgfqpoint{2.688092in}{2.729107in}}%
\pgfpathlineto{\pgfqpoint{2.687308in}{2.730045in}}%
\pgfpathlineto{\pgfqpoint{2.684957in}{2.732859in}}%
\pgfpathlineto{\pgfqpoint{2.684174in}{2.733797in}}%
\pgfpathlineto{\pgfqpoint{2.681823in}{2.736610in}}%
\pgfpathlineto{\pgfqpoint{2.681039in}{2.737548in}}%
\pgfpathlineto{\pgfqpoint{2.678688in}{2.740362in}}%
\pgfpathlineto{\pgfqpoint{2.677904in}{2.741300in}}%
\pgfpathlineto{\pgfqpoint{2.675553in}{2.744114in}}%
\pgfpathlineto{\pgfqpoint{2.674769in}{2.745051in}}%
\pgfpathlineto{\pgfqpoint{2.672418in}{2.747865in}}%
\pgfpathlineto{\pgfqpoint{2.669284in}{2.747865in}}%
\pgfpathlineto{\pgfqpoint{2.666149in}{2.747865in}}%
\pgfpathlineto{\pgfqpoint{2.663014in}{2.747865in}}%
\pgfpathlineto{\pgfqpoint{2.659879in}{2.747865in}}%
\pgfpathlineto{\pgfqpoint{2.657528in}{2.745051in}}%
\pgfpathlineto{\pgfqpoint{2.656745in}{2.744114in}}%
\pgfpathlineto{\pgfqpoint{2.653610in}{2.744114in}}%
\pgfpathlineto{\pgfqpoint{2.650475in}{2.744114in}}%
\pgfpathlineto{\pgfqpoint{2.647340in}{2.744114in}}%
\pgfpathlineto{\pgfqpoint{2.644989in}{2.741300in}}%
\pgfpathlineto{\pgfqpoint{2.644206in}{2.740362in}}%
\pgfpathlineto{\pgfqpoint{2.641071in}{2.740362in}}%
\pgfpathlineto{\pgfqpoint{2.637936in}{2.740362in}}%
\pgfpathlineto{\pgfqpoint{2.634801in}{2.740362in}}%
\pgfpathlineto{\pgfqpoint{2.632450in}{2.737548in}}%
\pgfpathlineto{\pgfqpoint{2.631667in}{2.736610in}}%
\pgfpathlineto{\pgfqpoint{2.628532in}{2.736610in}}%
\pgfpathlineto{\pgfqpoint{2.625397in}{2.736610in}}%
\pgfpathlineto{\pgfqpoint{2.622262in}{2.736610in}}%
\pgfpathlineto{\pgfqpoint{2.619128in}{2.736610in}}%
\pgfpathlineto{\pgfqpoint{2.616777in}{2.733797in}}%
\pgfpathlineto{\pgfqpoint{2.615993in}{2.732859in}}%
\pgfpathlineto{\pgfqpoint{2.612858in}{2.732859in}}%
\pgfpathlineto{\pgfqpoint{2.609724in}{2.732859in}}%
\pgfpathlineto{\pgfqpoint{2.606589in}{2.732859in}}%
\pgfpathlineto{\pgfqpoint{2.604238in}{2.730045in}}%
\pgfpathlineto{\pgfqpoint{2.603454in}{2.729107in}}%
\pgfpathlineto{\pgfqpoint{2.600319in}{2.729107in}}%
\pgfpathlineto{\pgfqpoint{2.597185in}{2.729107in}}%
\pgfpathlineto{\pgfqpoint{2.594050in}{2.729107in}}%
\pgfpathlineto{\pgfqpoint{2.591699in}{2.726294in}}%
\pgfpathlineto{\pgfqpoint{2.590915in}{2.725356in}}%
\pgfpathlineto{\pgfqpoint{2.587780in}{2.725356in}}%
\pgfpathlineto{\pgfqpoint{2.584646in}{2.725356in}}%
\pgfpathlineto{\pgfqpoint{2.581511in}{2.725356in}}%
\pgfpathlineto{\pgfqpoint{2.579160in}{2.722542in}}%
\pgfpathlineto{\pgfqpoint{2.578376in}{2.721604in}}%
\pgfpathlineto{\pgfqpoint{2.575241in}{2.721604in}}%
\pgfpathlineto{\pgfqpoint{2.572107in}{2.721604in}}%
\pgfpathlineto{\pgfqpoint{2.568972in}{2.721604in}}%
\pgfpathlineto{\pgfqpoint{2.566621in}{2.718791in}}%
\pgfpathlineto{\pgfqpoint{2.565837in}{2.717853in}}%
\pgfpathlineto{\pgfqpoint{2.562702in}{2.717853in}}%
\pgfpathlineto{\pgfqpoint{2.559568in}{2.717853in}}%
\pgfpathlineto{\pgfqpoint{2.556433in}{2.717853in}}%
\pgfpathlineto{\pgfqpoint{2.553298in}{2.717853in}}%
\pgfpathlineto{\pgfqpoint{2.550947in}{2.715039in}}%
\pgfpathlineto{\pgfqpoint{2.550163in}{2.714101in}}%
\pgfpathlineto{\pgfqpoint{2.547029in}{2.714101in}}%
\pgfpathlineto{\pgfqpoint{2.543894in}{2.714101in}}%
\pgfpathlineto{\pgfqpoint{2.540759in}{2.714101in}}%
\pgfpathlineto{\pgfqpoint{2.538408in}{2.711288in}}%
\pgfpathlineto{\pgfqpoint{2.537624in}{2.710350in}}%
\pgfpathlineto{\pgfqpoint{2.534490in}{2.710350in}}%
\pgfpathlineto{\pgfqpoint{2.531355in}{2.710350in}}%
\pgfpathlineto{\pgfqpoint{2.528220in}{2.710350in}}%
\pgfpathlineto{\pgfqpoint{2.525869in}{2.707536in}}%
\pgfpathlineto{\pgfqpoint{2.525085in}{2.706598in}}%
\pgfpathlineto{\pgfqpoint{2.521951in}{2.706598in}}%
\pgfpathlineto{\pgfqpoint{2.518816in}{2.706598in}}%
\pgfpathlineto{\pgfqpoint{2.515681in}{2.706598in}}%
\pgfpathlineto{\pgfqpoint{2.513330in}{2.703785in}}%
\pgfpathlineto{\pgfqpoint{2.512547in}{2.702847in}}%
\pgfpathlineto{\pgfqpoint{2.509412in}{2.702847in}}%
\pgfpathlineto{\pgfqpoint{2.506277in}{2.702847in}}%
\pgfpathlineto{\pgfqpoint{2.503142in}{2.702847in}}%
\pgfpathlineto{\pgfqpoint{2.500008in}{2.702847in}}%
\pgfpathlineto{\pgfqpoint{2.497656in}{2.700033in}}%
\pgfpathlineto{\pgfqpoint{2.496873in}{2.699095in}}%
\pgfpathlineto{\pgfqpoint{2.493738in}{2.699095in}}%
\pgfpathlineto{\pgfqpoint{2.490603in}{2.699095in}}%
\pgfpathlineto{\pgfqpoint{2.487469in}{2.699095in}}%
\pgfpathlineto{\pgfqpoint{2.485118in}{2.696281in}}%
\pgfpathlineto{\pgfqpoint{2.484334in}{2.695344in}}%
\pgfpathlineto{\pgfqpoint{2.481199in}{2.695344in}}%
\pgfpathlineto{\pgfqpoint{2.478064in}{2.695344in}}%
\pgfpathlineto{\pgfqpoint{2.474930in}{2.695344in}}%
\pgfpathlineto{\pgfqpoint{2.472579in}{2.692530in}}%
\pgfpathlineto{\pgfqpoint{2.471795in}{2.691592in}}%
\pgfpathlineto{\pgfqpoint{2.468660in}{2.691592in}}%
\pgfpathlineto{\pgfqpoint{2.465525in}{2.691592in}}%
\pgfpathlineto{\pgfqpoint{2.462391in}{2.691592in}}%
\pgfpathlineto{\pgfqpoint{2.460040in}{2.688778in}}%
\pgfpathlineto{\pgfqpoint{2.459256in}{2.687841in}}%
\pgfpathlineto{\pgfqpoint{2.456121in}{2.687841in}}%
\pgfpathlineto{\pgfqpoint{2.452986in}{2.687841in}}%
\pgfpathlineto{\pgfqpoint{2.449852in}{2.687841in}}%
\pgfpathlineto{\pgfqpoint{2.447501in}{2.685027in}}%
\pgfpathlineto{\pgfqpoint{2.446717in}{2.684089in}}%
\pgfpathlineto{\pgfqpoint{2.443582in}{2.684089in}}%
\pgfpathlineto{\pgfqpoint{2.440447in}{2.684089in}}%
\pgfpathlineto{\pgfqpoint{2.437313in}{2.684089in}}%
\pgfpathlineto{\pgfqpoint{2.434178in}{2.684089in}}%
\pgfpathlineto{\pgfqpoint{2.431827in}{2.681275in}}%
\pgfpathlineto{\pgfqpoint{2.431043in}{2.680337in}}%
\pgfpathlineto{\pgfqpoint{2.427908in}{2.680337in}}%
\pgfpathlineto{\pgfqpoint{2.424774in}{2.680337in}}%
\pgfpathlineto{\pgfqpoint{2.421639in}{2.680337in}}%
\pgfpathlineto{\pgfqpoint{2.419288in}{2.677524in}}%
\pgfpathlineto{\pgfqpoint{2.418504in}{2.676586in}}%
\pgfpathlineto{\pgfqpoint{2.415369in}{2.676586in}}%
\pgfpathlineto{\pgfqpoint{2.412235in}{2.676586in}}%
\pgfpathlineto{\pgfqpoint{2.409100in}{2.676586in}}%
\pgfpathlineto{\pgfqpoint{2.406749in}{2.673772in}}%
\pgfpathlineto{\pgfqpoint{2.405965in}{2.672834in}}%
\pgfpathlineto{\pgfqpoint{2.402831in}{2.672834in}}%
\pgfpathlineto{\pgfqpoint{2.399696in}{2.672834in}}%
\pgfpathlineto{\pgfqpoint{2.396561in}{2.672834in}}%
\pgfpathlineto{\pgfqpoint{2.394210in}{2.670021in}}%
\pgfpathlineto{\pgfqpoint{2.393426in}{2.669083in}}%
\pgfpathlineto{\pgfqpoint{2.390292in}{2.669083in}}%
\pgfpathlineto{\pgfqpoint{2.387157in}{2.669083in}}%
\pgfpathlineto{\pgfqpoint{2.384022in}{2.669083in}}%
\pgfpathlineto{\pgfqpoint{2.380887in}{2.669083in}}%
\pgfpathlineto{\pgfqpoint{2.378536in}{2.666269in}}%
\pgfpathlineto{\pgfqpoint{2.377753in}{2.665331in}}%
\pgfpathlineto{\pgfqpoint{2.374618in}{2.665331in}}%
\pgfpathlineto{\pgfqpoint{2.371483in}{2.665331in}}%
\pgfpathlineto{\pgfqpoint{2.368348in}{2.665331in}}%
\pgfpathlineto{\pgfqpoint{2.365997in}{2.662518in}}%
\pgfpathlineto{\pgfqpoint{2.365214in}{2.661580in}}%
\pgfpathlineto{\pgfqpoint{2.362079in}{2.661580in}}%
\pgfpathlineto{\pgfqpoint{2.358944in}{2.661580in}}%
\pgfpathlineto{\pgfqpoint{2.355809in}{2.661580in}}%
\pgfpathlineto{\pgfqpoint{2.353458in}{2.658766in}}%
\pgfpathlineto{\pgfqpoint{2.352675in}{2.657828in}}%
\pgfpathlineto{\pgfqpoint{2.349540in}{2.657828in}}%
\pgfpathlineto{\pgfqpoint{2.346405in}{2.657828in}}%
\pgfpathlineto{\pgfqpoint{2.343270in}{2.657828in}}%
\pgfpathlineto{\pgfqpoint{2.340919in}{2.655015in}}%
\pgfpathlineto{\pgfqpoint{2.340136in}{2.654077in}}%
\pgfpathlineto{\pgfqpoint{2.337001in}{2.654077in}}%
\pgfpathlineto{\pgfqpoint{2.333866in}{2.654077in}}%
\pgfpathlineto{\pgfqpoint{2.330731in}{2.654077in}}%
\pgfpathlineto{\pgfqpoint{2.327597in}{2.654077in}}%
\pgfpathlineto{\pgfqpoint{2.325246in}{2.651263in}}%
\pgfpathlineto{\pgfqpoint{2.324462in}{2.650325in}}%
\pgfpathlineto{\pgfqpoint{2.321327in}{2.650325in}}%
\pgfpathlineto{\pgfqpoint{2.318192in}{2.650325in}}%
\pgfpathlineto{\pgfqpoint{2.315058in}{2.650325in}}%
\pgfpathlineto{\pgfqpoint{2.312707in}{2.647512in}}%
\pgfpathlineto{\pgfqpoint{2.311923in}{2.646574in}}%
\pgfpathlineto{\pgfqpoint{2.308788in}{2.646574in}}%
\pgfpathlineto{\pgfqpoint{2.305654in}{2.646574in}}%
\pgfpathlineto{\pgfqpoint{2.302519in}{2.646574in}}%
\pgfpathlineto{\pgfqpoint{2.300168in}{2.643760in}}%
\pgfpathlineto{\pgfqpoint{2.299384in}{2.642822in}}%
\pgfpathlineto{\pgfqpoint{2.296249in}{2.642822in}}%
\pgfpathlineto{\pgfqpoint{2.293115in}{2.642822in}}%
\pgfpathlineto{\pgfqpoint{2.289980in}{2.642822in}}%
\pgfpathlineto{\pgfqpoint{2.287629in}{2.640008in}}%
\pgfpathlineto{\pgfqpoint{2.286845in}{2.639071in}}%
\pgfpathlineto{\pgfqpoint{2.283710in}{2.639071in}}%
\pgfpathlineto{\pgfqpoint{2.280576in}{2.639071in}}%
\pgfpathlineto{\pgfqpoint{2.277441in}{2.639071in}}%
\pgfpathlineto{\pgfqpoint{2.275090in}{2.636257in}}%
\pgfpathlineto{\pgfqpoint{2.274306in}{2.635319in}}%
\pgfpathlineto{\pgfqpoint{2.271171in}{2.635319in}}%
\pgfpathlineto{\pgfqpoint{2.268037in}{2.635319in}}%
\pgfpathlineto{\pgfqpoint{2.264902in}{2.635319in}}%
\pgfpathlineto{\pgfqpoint{2.261767in}{2.635319in}}%
\pgfpathlineto{\pgfqpoint{2.259416in}{2.632505in}}%
\pgfpathlineto{\pgfqpoint{2.258632in}{2.631567in}}%
\pgfpathlineto{\pgfqpoint{2.255498in}{2.631567in}}%
\pgfpathlineto{\pgfqpoint{2.252363in}{2.631567in}}%
\pgfpathlineto{\pgfqpoint{2.249228in}{2.631567in}}%
\pgfpathlineto{\pgfqpoint{2.246877in}{2.628754in}}%
\pgfpathlineto{\pgfqpoint{2.246093in}{2.627816in}}%
\pgfpathlineto{\pgfqpoint{2.242959in}{2.627816in}}%
\pgfpathlineto{\pgfqpoint{2.239824in}{2.627816in}}%
\pgfpathlineto{\pgfqpoint{2.236689in}{2.627816in}}%
\pgfpathlineto{\pgfqpoint{2.234338in}{2.625002in}}%
\pgfpathlineto{\pgfqpoint{2.233554in}{2.624064in}}%
\pgfpathlineto{\pgfqpoint{2.230420in}{2.624064in}}%
\pgfpathlineto{\pgfqpoint{2.227285in}{2.624064in}}%
\pgfpathlineto{\pgfqpoint{2.224150in}{2.624064in}}%
\pgfpathlineto{\pgfqpoint{2.221799in}{2.621251in}}%
\pgfpathlineto{\pgfqpoint{2.221015in}{2.620313in}}%
\pgfpathlineto{\pgfqpoint{2.217881in}{2.620313in}}%
\pgfpathlineto{\pgfqpoint{2.214746in}{2.620313in}}%
\pgfpathlineto{\pgfqpoint{2.211611in}{2.620313in}}%
\pgfpathlineto{\pgfqpoint{2.208477in}{2.620313in}}%
\pgfpathlineto{\pgfqpoint{2.206125in}{2.617499in}}%
\pgfpathlineto{\pgfqpoint{2.205342in}{2.616561in}}%
\pgfpathlineto{\pgfqpoint{2.202207in}{2.616561in}}%
\pgfpathlineto{\pgfqpoint{2.199072in}{2.616561in}}%
\pgfpathlineto{\pgfqpoint{2.195938in}{2.616561in}}%
\pgfpathlineto{\pgfqpoint{2.193586in}{2.613748in}}%
\pgfpathlineto{\pgfqpoint{2.192803in}{2.612810in}}%
\pgfpathlineto{\pgfqpoint{2.189668in}{2.612810in}}%
\pgfpathlineto{\pgfqpoint{2.186533in}{2.612810in}}%
\pgfpathlineto{\pgfqpoint{2.183399in}{2.612810in}}%
\pgfpathlineto{\pgfqpoint{2.181048in}{2.609996in}}%
\pgfpathlineto{\pgfqpoint{2.180264in}{2.609058in}}%
\pgfpathlineto{\pgfqpoint{2.177129in}{2.609058in}}%
\pgfpathlineto{\pgfqpoint{2.173994in}{2.609058in}}%
\pgfpathlineto{\pgfqpoint{2.170860in}{2.609058in}}%
\pgfpathlineto{\pgfqpoint{2.168509in}{2.606245in}}%
\pgfpathlineto{\pgfqpoint{2.167725in}{2.605307in}}%
\pgfpathlineto{\pgfqpoint{2.164590in}{2.605307in}}%
\pgfpathlineto{\pgfqpoint{2.161455in}{2.605307in}}%
\pgfpathlineto{\pgfqpoint{2.158321in}{2.605307in}}%
\pgfpathlineto{\pgfqpoint{2.155186in}{2.605307in}}%
\pgfpathlineto{\pgfqpoint{2.152835in}{2.602493in}}%
\pgfpathlineto{\pgfqpoint{2.152051in}{2.601555in}}%
\pgfpathlineto{\pgfqpoint{2.148916in}{2.601555in}}%
\pgfpathlineto{\pgfqpoint{2.145782in}{2.601555in}}%
\pgfpathlineto{\pgfqpoint{2.142647in}{2.601555in}}%
\pgfpathlineto{\pgfqpoint{2.140296in}{2.598742in}}%
\pgfpathlineto{\pgfqpoint{2.139512in}{2.597804in}}%
\pgfpathlineto{\pgfqpoint{2.136377in}{2.597804in}}%
\pgfpathlineto{\pgfqpoint{2.133243in}{2.597804in}}%
\pgfpathlineto{\pgfqpoint{2.130108in}{2.597804in}}%
\pgfpathlineto{\pgfqpoint{2.127757in}{2.594990in}}%
\pgfpathlineto{\pgfqpoint{2.126973in}{2.594052in}}%
\pgfpathlineto{\pgfqpoint{2.123838in}{2.594052in}}%
\pgfpathlineto{\pgfqpoint{2.120704in}{2.594052in}}%
\pgfpathlineto{\pgfqpoint{2.117569in}{2.594052in}}%
\pgfpathlineto{\pgfqpoint{2.115218in}{2.591238in}}%
\pgfpathlineto{\pgfqpoint{2.114434in}{2.590301in}}%
\pgfpathlineto{\pgfqpoint{2.111299in}{2.590301in}}%
\pgfpathlineto{\pgfqpoint{2.108165in}{2.590301in}}%
\pgfpathlineto{\pgfqpoint{2.105030in}{2.590301in}}%
\pgfpathlineto{\pgfqpoint{2.102679in}{2.587487in}}%
\pgfpathlineto{\pgfqpoint{2.101895in}{2.586549in}}%
\pgfpathlineto{\pgfqpoint{2.098761in}{2.586549in}}%
\pgfpathlineto{\pgfqpoint{2.095626in}{2.586549in}}%
\pgfpathlineto{\pgfqpoint{2.092491in}{2.586549in}}%
\pgfpathlineto{\pgfqpoint{2.089356in}{2.586549in}}%
\pgfpathlineto{\pgfqpoint{2.087005in}{2.583735in}}%
\pgfpathlineto{\pgfqpoint{2.086222in}{2.582798in}}%
\pgfpathlineto{\pgfqpoint{2.083087in}{2.582798in}}%
\pgfpathlineto{\pgfqpoint{2.079952in}{2.582798in}}%
\pgfpathlineto{\pgfqpoint{2.076817in}{2.582798in}}%
\pgfpathlineto{\pgfqpoint{2.074466in}{2.579984in}}%
\pgfpathlineto{\pgfqpoint{2.073683in}{2.579046in}}%
\pgfpathlineto{\pgfqpoint{2.070548in}{2.579046in}}%
\pgfpathlineto{\pgfqpoint{2.067413in}{2.579046in}}%
\pgfpathlineto{\pgfqpoint{2.064278in}{2.579046in}}%
\pgfpathlineto{\pgfqpoint{2.061927in}{2.576232in}}%
\pgfpathlineto{\pgfqpoint{2.061144in}{2.575294in}}%
\pgfpathlineto{\pgfqpoint{2.058009in}{2.575294in}}%
\pgfpathlineto{\pgfqpoint{2.054874in}{2.575294in}}%
\pgfpathlineto{\pgfqpoint{2.051739in}{2.575294in}}%
\pgfpathlineto{\pgfqpoint{2.049388in}{2.572481in}}%
\pgfpathlineto{\pgfqpoint{2.048605in}{2.571543in}}%
\pgfpathlineto{\pgfqpoint{2.045470in}{2.571543in}}%
\pgfpathlineto{\pgfqpoint{2.042335in}{2.571543in}}%
\pgfpathlineto{\pgfqpoint{2.039200in}{2.571543in}}%
\pgfpathlineto{\pgfqpoint{2.036066in}{2.571543in}}%
\pgfpathlineto{\pgfqpoint{2.033715in}{2.568729in}}%
\pgfpathlineto{\pgfqpoint{2.032931in}{2.567791in}}%
\pgfpathlineto{\pgfqpoint{2.029796in}{2.567791in}}%
\pgfpathlineto{\pgfqpoint{2.026661in}{2.567791in}}%
\pgfpathlineto{\pgfqpoint{2.023527in}{2.567791in}}%
\pgfpathlineto{\pgfqpoint{2.021176in}{2.564978in}}%
\pgfpathlineto{\pgfqpoint{2.020392in}{2.564040in}}%
\pgfpathlineto{\pgfqpoint{2.017257in}{2.564040in}}%
\pgfpathlineto{\pgfqpoint{2.014122in}{2.564040in}}%
\pgfpathlineto{\pgfqpoint{2.010988in}{2.564040in}}%
\pgfpathlineto{\pgfqpoint{2.008637in}{2.561226in}}%
\pgfpathlineto{\pgfqpoint{2.007853in}{2.560288in}}%
\pgfpathlineto{\pgfqpoint{2.004718in}{2.560288in}}%
\pgfpathlineto{\pgfqpoint{2.001584in}{2.560288in}}%
\pgfpathlineto{\pgfqpoint{1.998449in}{2.560288in}}%
\pgfpathlineto{\pgfqpoint{1.996098in}{2.557475in}}%
\pgfpathlineto{\pgfqpoint{1.995314in}{2.556537in}}%
\pgfpathlineto{\pgfqpoint{1.992179in}{2.556537in}}%
\pgfpathlineto{\pgfqpoint{1.989045in}{2.556537in}}%
\pgfpathlineto{\pgfqpoint{1.985910in}{2.556537in}}%
\pgfpathlineto{\pgfqpoint{1.982775in}{2.556537in}}%
\pgfpathlineto{\pgfqpoint{1.980424in}{2.553723in}}%
\pgfpathlineto{\pgfqpoint{1.979640in}{2.552785in}}%
\pgfpathlineto{\pgfqpoint{1.976506in}{2.552785in}}%
\pgfpathlineto{\pgfqpoint{1.973371in}{2.552785in}}%
\pgfpathlineto{\pgfqpoint{1.970236in}{2.552785in}}%
\pgfpathlineto{\pgfqpoint{1.967885in}{2.549972in}}%
\pgfpathlineto{\pgfqpoint{1.967101in}{2.549034in}}%
\pgfpathlineto{\pgfqpoint{1.963967in}{2.549034in}}%
\pgfpathlineto{\pgfqpoint{1.960832in}{2.549034in}}%
\pgfpathlineto{\pgfqpoint{1.957697in}{2.549034in}}%
\pgfpathlineto{\pgfqpoint{1.955346in}{2.546220in}}%
\pgfpathlineto{\pgfqpoint{1.954562in}{2.545282in}}%
\pgfpathlineto{\pgfqpoint{1.951428in}{2.545282in}}%
\pgfpathlineto{\pgfqpoint{1.948293in}{2.545282in}}%
\pgfpathlineto{\pgfqpoint{1.945158in}{2.545282in}}%
\pgfpathlineto{\pgfqpoint{1.942807in}{2.542469in}}%
\pgfpathlineto{\pgfqpoint{1.942023in}{2.541531in}}%
\pgfpathlineto{\pgfqpoint{1.938889in}{2.541531in}}%
\pgfpathlineto{\pgfqpoint{1.935754in}{2.541531in}}%
\pgfpathlineto{\pgfqpoint{1.932619in}{2.541531in}}%
\pgfpathlineto{\pgfqpoint{1.930268in}{2.538717in}}%
\pgfpathlineto{\pgfqpoint{1.929484in}{2.537779in}}%
\pgfpathlineto{\pgfqpoint{1.926350in}{2.537779in}}%
\pgfpathlineto{\pgfqpoint{1.923215in}{2.537779in}}%
\pgfpathlineto{\pgfqpoint{1.920080in}{2.537779in}}%
\pgfpathlineto{\pgfqpoint{1.916945in}{2.537779in}}%
\pgfpathlineto{\pgfqpoint{1.914594in}{2.534965in}}%
\pgfpathlineto{\pgfqpoint{1.913811in}{2.534028in}}%
\pgfpathlineto{\pgfqpoint{1.910676in}{2.534028in}}%
\pgfpathlineto{\pgfqpoint{1.907541in}{2.534028in}}%
\pgfpathlineto{\pgfqpoint{1.904407in}{2.534028in}}%
\pgfpathlineto{\pgfqpoint{1.902055in}{2.531214in}}%
\pgfpathlineto{\pgfqpoint{1.901272in}{2.530276in}}%
\pgfpathlineto{\pgfqpoint{1.898137in}{2.530276in}}%
\pgfpathlineto{\pgfqpoint{1.895002in}{2.530276in}}%
\pgfpathlineto{\pgfqpoint{1.891868in}{2.530276in}}%
\pgfpathlineto{\pgfqpoint{1.889516in}{2.527462in}}%
\pgfpathlineto{\pgfqpoint{1.888733in}{2.526525in}}%
\pgfpathlineto{\pgfqpoint{1.885598in}{2.526525in}}%
\pgfpathlineto{\pgfqpoint{1.882463in}{2.526525in}}%
\pgfpathlineto{\pgfqpoint{1.879329in}{2.526525in}}%
\pgfpathlineto{\pgfqpoint{1.876978in}{2.523711in}}%
\pgfpathlineto{\pgfqpoint{1.876194in}{2.522773in}}%
\pgfpathlineto{\pgfqpoint{1.873059in}{2.522773in}}%
\pgfpathlineto{\pgfqpoint{1.869924in}{2.522773in}}%
\pgfpathlineto{\pgfqpoint{1.866790in}{2.522773in}}%
\pgfpathlineto{\pgfqpoint{1.863655in}{2.522773in}}%
\pgfpathlineto{\pgfqpoint{1.861304in}{2.519959in}}%
\pgfpathlineto{\pgfqpoint{1.860520in}{2.519021in}}%
\pgfpathlineto{\pgfqpoint{1.857385in}{2.519021in}}%
\pgfpathlineto{\pgfqpoint{1.854251in}{2.519021in}}%
\pgfpathlineto{\pgfqpoint{1.851116in}{2.519021in}}%
\pgfpathlineto{\pgfqpoint{1.848765in}{2.516208in}}%
\pgfpathlineto{\pgfqpoint{1.847981in}{2.515270in}}%
\pgfpathlineto{\pgfqpoint{1.844846in}{2.515270in}}%
\pgfpathlineto{\pgfqpoint{1.841712in}{2.515270in}}%
\pgfpathlineto{\pgfqpoint{1.838577in}{2.515270in}}%
\pgfpathlineto{\pgfqpoint{1.836226in}{2.512456in}}%
\pgfpathlineto{\pgfqpoint{1.835442in}{2.511518in}}%
\pgfpathlineto{\pgfqpoint{1.832307in}{2.511518in}}%
\pgfpathlineto{\pgfqpoint{1.829173in}{2.511518in}}%
\pgfpathlineto{\pgfqpoint{1.826038in}{2.511518in}}%
\pgfpathlineto{\pgfqpoint{1.823687in}{2.508705in}}%
\pgfpathlineto{\pgfqpoint{1.822903in}{2.507767in}}%
\pgfpathlineto{\pgfqpoint{1.819768in}{2.507767in}}%
\pgfpathlineto{\pgfqpoint{1.816634in}{2.507767in}}%
\pgfpathlineto{\pgfqpoint{1.813499in}{2.507767in}}%
\pgfpathlineto{\pgfqpoint{1.811148in}{2.504953in}}%
\pgfpathlineto{\pgfqpoint{1.810364in}{2.504015in}}%
\pgfpathlineto{\pgfqpoint{1.807229in}{2.504015in}}%
\pgfpathlineto{\pgfqpoint{1.804095in}{2.504015in}}%
\pgfpathlineto{\pgfqpoint{1.800960in}{2.504015in}}%
\pgfpathlineto{\pgfqpoint{1.797825in}{2.504015in}}%
\pgfpathlineto{\pgfqpoint{1.795474in}{2.501202in}}%
\pgfpathlineto{\pgfqpoint{1.794691in}{2.500264in}}%
\pgfpathlineto{\pgfqpoint{1.791556in}{2.500264in}}%
\pgfpathlineto{\pgfqpoint{1.788421in}{2.500264in}}%
\pgfpathlineto{\pgfqpoint{1.785286in}{2.500264in}}%
\pgfpathlineto{\pgfqpoint{1.782935in}{2.497450in}}%
\pgfpathlineto{\pgfqpoint{1.782152in}{2.496512in}}%
\pgfpathlineto{\pgfqpoint{1.779017in}{2.496512in}}%
\pgfpathlineto{\pgfqpoint{1.775882in}{2.496512in}}%
\pgfpathlineto{\pgfqpoint{1.772747in}{2.496512in}}%
\pgfpathlineto{\pgfqpoint{1.770396in}{2.493699in}}%
\pgfpathlineto{\pgfqpoint{1.769613in}{2.492761in}}%
\pgfpathlineto{\pgfqpoint{1.766478in}{2.492761in}}%
\pgfpathlineto{\pgfqpoint{1.763343in}{2.492761in}}%
\pgfpathlineto{\pgfqpoint{1.760208in}{2.492761in}}%
\pgfpathlineto{\pgfqpoint{1.757857in}{2.489947in}}%
\pgfpathlineto{\pgfqpoint{1.757074in}{2.489009in}}%
\pgfpathlineto{\pgfqpoint{1.753939in}{2.489009in}}%
\pgfpathlineto{\pgfqpoint{1.750804in}{2.489009in}}%
\pgfpathlineto{\pgfqpoint{1.747669in}{2.489009in}}%
\pgfpathlineto{\pgfqpoint{1.744535in}{2.489009in}}%
\pgfpathlineto{\pgfqpoint{1.742184in}{2.486196in}}%
\pgfpathlineto{\pgfqpoint{1.741400in}{2.485258in}}%
\pgfpathlineto{\pgfqpoint{1.738265in}{2.485258in}}%
\pgfpathlineto{\pgfqpoint{1.735130in}{2.485258in}}%
\pgfpathlineto{\pgfqpoint{1.731996in}{2.485258in}}%
\pgfpathlineto{\pgfqpoint{1.729645in}{2.482444in}}%
\pgfpathlineto{\pgfqpoint{1.728861in}{2.481506in}}%
\pgfpathlineto{\pgfqpoint{1.725726in}{2.481506in}}%
\pgfpathlineto{\pgfqpoint{1.722591in}{2.481506in}}%
\pgfpathlineto{\pgfqpoint{1.719457in}{2.481506in}}%
\pgfpathlineto{\pgfqpoint{1.717106in}{2.478692in}}%
\pgfpathlineto{\pgfqpoint{1.716322in}{2.477755in}}%
\pgfpathlineto{\pgfqpoint{1.713187in}{2.477755in}}%
\pgfpathlineto{\pgfqpoint{1.710052in}{2.477755in}}%
\pgfpathlineto{\pgfqpoint{1.706918in}{2.477755in}}%
\pgfpathlineto{\pgfqpoint{1.704567in}{2.474941in}}%
\pgfpathlineto{\pgfqpoint{1.703783in}{2.474003in}}%
\pgfpathlineto{\pgfqpoint{1.700648in}{2.474003in}}%
\pgfpathlineto{\pgfqpoint{1.697514in}{2.474003in}}%
\pgfpathlineto{\pgfqpoint{1.694379in}{2.474003in}}%
\pgfpathlineto{\pgfqpoint{1.691244in}{2.474003in}}%
\pgfpathlineto{\pgfqpoint{1.688893in}{2.471189in}}%
\pgfpathlineto{\pgfqpoint{1.688109in}{2.470252in}}%
\pgfpathlineto{\pgfqpoint{1.684975in}{2.470252in}}%
\pgfpathlineto{\pgfqpoint{1.681840in}{2.470252in}}%
\pgfpathlineto{\pgfqpoint{1.678705in}{2.470252in}}%
\pgfpathlineto{\pgfqpoint{1.676354in}{2.467438in}}%
\pgfpathlineto{\pgfqpoint{1.675570in}{2.466500in}}%
\pgfpathlineto{\pgfqpoint{1.672436in}{2.466500in}}%
\pgfpathlineto{\pgfqpoint{1.669301in}{2.466500in}}%
\pgfpathlineto{\pgfqpoint{1.666166in}{2.466500in}}%
\pgfpathlineto{\pgfqpoint{1.663815in}{2.463686in}}%
\pgfpathlineto{\pgfqpoint{1.663031in}{2.462748in}}%
\pgfpathlineto{\pgfqpoint{1.659897in}{2.462748in}}%
\pgfpathlineto{\pgfqpoint{1.656762in}{2.462748in}}%
\pgfpathlineto{\pgfqpoint{1.653627in}{2.462748in}}%
\pgfpathlineto{\pgfqpoint{1.651276in}{2.459935in}}%
\pgfpathlineto{\pgfqpoint{1.650492in}{2.458997in}}%
\pgfpathlineto{\pgfqpoint{1.647358in}{2.458997in}}%
\pgfpathlineto{\pgfqpoint{1.644223in}{2.458997in}}%
\pgfpathlineto{\pgfqpoint{1.641088in}{2.458997in}}%
\pgfpathlineto{\pgfqpoint{1.638737in}{2.456183in}}%
\pgfpathlineto{\pgfqpoint{1.637953in}{2.455245in}}%
\pgfpathlineto{\pgfqpoint{1.634819in}{2.455245in}}%
\pgfpathlineto{\pgfqpoint{1.631684in}{2.455245in}}%
\pgfpathlineto{\pgfqpoint{1.628549in}{2.455245in}}%
\pgfpathlineto{\pgfqpoint{1.625414in}{2.455245in}}%
\pgfpathlineto{\pgfqpoint{1.623063in}{2.452432in}}%
\pgfpathlineto{\pgfqpoint{1.622280in}{2.451494in}}%
\pgfpathlineto{\pgfqpoint{1.619145in}{2.451494in}}%
\pgfpathlineto{\pgfqpoint{1.616010in}{2.451494in}}%
\pgfpathlineto{\pgfqpoint{1.612875in}{2.451494in}}%
\pgfpathlineto{\pgfqpoint{1.610524in}{2.448680in}}%
\pgfpathlineto{\pgfqpoint{1.609741in}{2.447742in}}%
\pgfpathlineto{\pgfqpoint{1.606606in}{2.447742in}}%
\pgfpathlineto{\pgfqpoint{1.603471in}{2.447742in}}%
\pgfpathlineto{\pgfqpoint{1.600337in}{2.447742in}}%
\pgfpathlineto{\pgfqpoint{1.597985in}{2.444929in}}%
\pgfpathlineto{\pgfqpoint{1.597202in}{2.443991in}}%
\pgfpathlineto{\pgfqpoint{1.594067in}{2.443991in}}%
\pgfpathlineto{\pgfqpoint{1.590932in}{2.443991in}}%
\pgfpathlineto{\pgfqpoint{1.587798in}{2.443991in}}%
\pgfpathlineto{\pgfqpoint{1.585446in}{2.441177in}}%
\pgfpathlineto{\pgfqpoint{1.584663in}{2.440239in}}%
\pgfpathlineto{\pgfqpoint{1.581528in}{2.440239in}}%
\pgfpathlineto{\pgfqpoint{1.578393in}{2.440239in}}%
\pgfpathlineto{\pgfqpoint{1.575259in}{2.440239in}}%
\pgfpathlineto{\pgfqpoint{1.572124in}{2.440239in}}%
\pgfpathlineto{\pgfqpoint{1.569773in}{2.437426in}}%
\pgfpathlineto{\pgfqpoint{1.568989in}{2.436488in}}%
\pgfpathlineto{\pgfqpoint{1.565854in}{2.436488in}}%
\pgfpathlineto{\pgfqpoint{1.562720in}{2.436488in}}%
\pgfpathlineto{\pgfqpoint{1.559585in}{2.436488in}}%
\pgfpathlineto{\pgfqpoint{1.557234in}{2.433674in}}%
\pgfpathlineto{\pgfqpoint{1.556450in}{2.432736in}}%
\pgfpathlineto{\pgfqpoint{1.553315in}{2.432736in}}%
\pgfpathlineto{\pgfqpoint{1.550181in}{2.432736in}}%
\pgfpathlineto{\pgfqpoint{1.547046in}{2.432736in}}%
\pgfpathlineto{\pgfqpoint{1.544695in}{2.429923in}}%
\pgfpathlineto{\pgfqpoint{1.543911in}{2.428985in}}%
\pgfpathlineto{\pgfqpoint{1.540776in}{2.428985in}}%
\pgfpathlineto{\pgfqpoint{1.537642in}{2.428985in}}%
\pgfpathlineto{\pgfqpoint{1.534507in}{2.428985in}}%
\pgfpathlineto{\pgfqpoint{1.532156in}{2.426171in}}%
\pgfpathlineto{\pgfqpoint{1.531372in}{2.425233in}}%
\pgfpathlineto{\pgfqpoint{1.528237in}{2.425233in}}%
\pgfpathlineto{\pgfqpoint{1.525103in}{2.425233in}}%
\pgfpathlineto{\pgfqpoint{1.521968in}{2.425233in}}%
\pgfpathlineto{\pgfqpoint{1.518833in}{2.425233in}}%
\pgfpathlineto{\pgfqpoint{1.516482in}{2.422419in}}%
\pgfpathlineto{\pgfqpoint{1.515698in}{2.421482in}}%
\pgfpathlineto{\pgfqpoint{1.512564in}{2.421482in}}%
\pgfpathlineto{\pgfqpoint{1.509429in}{2.421482in}}%
\pgfpathlineto{\pgfqpoint{1.506294in}{2.421482in}}%
\pgfpathlineto{\pgfqpoint{1.503943in}{2.418668in}}%
\pgfpathlineto{\pgfqpoint{1.503159in}{2.417730in}}%
\pgfpathlineto{\pgfqpoint{1.500025in}{2.417730in}}%
\pgfpathlineto{\pgfqpoint{1.496890in}{2.417730in}}%
\pgfpathlineto{\pgfqpoint{1.493755in}{2.417730in}}%
\pgfpathlineto{\pgfqpoint{1.491404in}{2.414916in}}%
\pgfpathlineto{\pgfqpoint{1.490621in}{2.413978in}}%
\pgfpathlineto{\pgfqpoint{1.487486in}{2.413978in}}%
\pgfpathlineto{\pgfqpoint{1.484351in}{2.413978in}}%
\pgfpathlineto{\pgfqpoint{1.481216in}{2.413978in}}%
\pgfpathlineto{\pgfqpoint{1.478865in}{2.411165in}}%
\pgfpathlineto{\pgfqpoint{1.478082in}{2.410227in}}%
\pgfpathlineto{\pgfqpoint{1.474947in}{2.410227in}}%
\pgfpathlineto{\pgfqpoint{1.471812in}{2.410227in}}%
\pgfpathlineto{\pgfqpoint{1.468677in}{2.410227in}}%
\pgfpathlineto{\pgfqpoint{1.466326in}{2.407413in}}%
\pgfpathlineto{\pgfqpoint{1.465543in}{2.406475in}}%
\pgfpathlineto{\pgfqpoint{1.462408in}{2.406475in}}%
\pgfpathlineto{\pgfqpoint{1.460057in}{2.403662in}}%
\pgfpathlineto{\pgfqpoint{1.460057in}{2.399910in}}%
\pgfpathlineto{\pgfqpoint{1.459273in}{2.398972in}}%
\pgfpathlineto{\pgfqpoint{1.456922in}{2.396159in}}%
\pgfpathlineto{\pgfqpoint{1.456922in}{2.392407in}}%
\pgfpathlineto{\pgfqpoint{1.456922in}{2.388656in}}%
\pgfpathlineto{\pgfqpoint{1.456138in}{2.387718in}}%
\pgfpathlineto{\pgfqpoint{1.453787in}{2.384904in}}%
\pgfpathlineto{\pgfqpoint{1.453787in}{2.381153in}}%
\pgfpathlineto{\pgfqpoint{1.453787in}{2.377401in}}%
\pgfpathlineto{\pgfqpoint{1.453787in}{2.373649in}}%
\pgfpathlineto{\pgfqpoint{1.453004in}{2.372712in}}%
\pgfpathlineto{\pgfqpoint{1.450653in}{2.369898in}}%
\pgfpathlineto{\pgfqpoint{1.450653in}{2.366146in}}%
\pgfpathlineto{\pgfqpoint{1.450653in}{2.362395in}}%
\pgfpathlineto{\pgfqpoint{1.449869in}{2.361457in}}%
\pgfpathlineto{\pgfqpoint{1.447518in}{2.358643in}}%
\pgfpathlineto{\pgfqpoint{1.447518in}{2.354892in}}%
\pgfpathlineto{\pgfqpoint{1.447518in}{2.351140in}}%
\pgfpathlineto{\pgfqpoint{1.447518in}{2.347389in}}%
\pgfpathlineto{\pgfqpoint{1.446734in}{2.346451in}}%
\pgfpathlineto{\pgfqpoint{1.444383in}{2.343637in}}%
\pgfpathlineto{\pgfqpoint{1.444383in}{2.339886in}}%
\pgfpathlineto{\pgfqpoint{1.444383in}{2.336134in}}%
\pgfpathlineto{\pgfqpoint{1.443599in}{2.335196in}}%
\pgfpathlineto{\pgfqpoint{1.441248in}{2.332383in}}%
\pgfpathlineto{\pgfqpoint{1.441248in}{2.328631in}}%
\pgfpathlineto{\pgfqpoint{1.441248in}{2.324880in}}%
\pgfpathlineto{\pgfqpoint{1.441248in}{2.321128in}}%
\pgfpathlineto{\pgfqpoint{1.440465in}{2.320190in}}%
\pgfpathlineto{\pgfqpoint{1.438114in}{2.317376in}}%
\pgfpathlineto{\pgfqpoint{1.438114in}{2.313625in}}%
\pgfpathlineto{\pgfqpoint{1.438114in}{2.309873in}}%
\pgfpathlineto{\pgfqpoint{1.437330in}{2.308936in}}%
\pgfpathlineto{\pgfqpoint{1.434979in}{2.306122in}}%
\pgfpathlineto{\pgfqpoint{1.434979in}{2.302370in}}%
\pgfpathlineto{\pgfqpoint{1.434979in}{2.298619in}}%
\pgfpathlineto{\pgfqpoint{1.434979in}{2.294867in}}%
\pgfpathlineto{\pgfqpoint{1.434195in}{2.293929in}}%
\pgfpathlineto{\pgfqpoint{1.431844in}{2.291116in}}%
\pgfpathlineto{\pgfqpoint{1.431844in}{2.287364in}}%
\pgfpathlineto{\pgfqpoint{1.431844in}{2.283613in}}%
\pgfpathlineto{\pgfqpoint{1.431060in}{2.282675in}}%
\pgfpathlineto{\pgfqpoint{1.428709in}{2.279861in}}%
\pgfpathlineto{\pgfqpoint{1.428709in}{2.276110in}}%
\pgfpathlineto{\pgfqpoint{1.428709in}{2.272358in}}%
\pgfpathlineto{\pgfqpoint{1.428709in}{2.268607in}}%
\pgfpathlineto{\pgfqpoint{1.427926in}{2.267669in}}%
\pgfpathlineto{\pgfqpoint{1.425575in}{2.264855in}}%
\pgfpathlineto{\pgfqpoint{1.425575in}{2.261103in}}%
\pgfpathlineto{\pgfqpoint{1.425575in}{2.257352in}}%
\pgfpathlineto{\pgfqpoint{1.424791in}{2.256414in}}%
\pgfpathlineto{\pgfqpoint{1.422440in}{2.253600in}}%
\pgfpathlineto{\pgfqpoint{1.422440in}{2.249849in}}%
\pgfpathlineto{\pgfqpoint{1.422440in}{2.246097in}}%
\pgfpathlineto{\pgfqpoint{1.421656in}{2.245159in}}%
\pgfpathlineto{\pgfqpoint{1.419305in}{2.242346in}}%
\pgfpathlineto{\pgfqpoint{1.419305in}{2.238594in}}%
\pgfpathlineto{\pgfqpoint{1.419305in}{2.234843in}}%
\pgfpathlineto{\pgfqpoint{1.419305in}{2.231091in}}%
\pgfpathlineto{\pgfqpoint{1.418521in}{2.230153in}}%
\pgfpathlineto{\pgfqpoint{1.416170in}{2.227340in}}%
\pgfpathlineto{\pgfqpoint{1.416170in}{2.223588in}}%
\pgfpathlineto{\pgfqpoint{1.416170in}{2.219837in}}%
\pgfpathlineto{\pgfqpoint{1.415387in}{2.218899in}}%
\pgfpathlineto{\pgfqpoint{1.413036in}{2.216085in}}%
\pgfpathlineto{\pgfqpoint{1.413036in}{2.212334in}}%
\pgfpathlineto{\pgfqpoint{1.413036in}{2.208582in}}%
\pgfpathlineto{\pgfqpoint{1.413036in}{2.204830in}}%
\pgfpathlineto{\pgfqpoint{1.412252in}{2.203893in}}%
\pgfpathlineto{\pgfqpoint{1.409901in}{2.201079in}}%
\pgfpathlineto{\pgfqpoint{1.409901in}{2.197327in}}%
\pgfpathlineto{\pgfqpoint{1.409901in}{2.193576in}}%
\pgfpathlineto{\pgfqpoint{1.409117in}{2.192638in}}%
\pgfpathlineto{\pgfqpoint{1.406766in}{2.189824in}}%
\pgfpathlineto{\pgfqpoint{1.406766in}{2.186073in}}%
\pgfpathlineto{\pgfqpoint{1.406766in}{2.182321in}}%
\pgfpathlineto{\pgfqpoint{1.406766in}{2.178570in}}%
\pgfpathlineto{\pgfqpoint{1.405982in}{2.177632in}}%
\pgfpathlineto{\pgfqpoint{1.403631in}{2.174818in}}%
\pgfpathlineto{\pgfqpoint{1.403631in}{2.171067in}}%
\pgfpathlineto{\pgfqpoint{1.403631in}{2.167315in}}%
\pgfpathlineto{\pgfqpoint{1.402848in}{2.166377in}}%
\pgfpathlineto{\pgfqpoint{1.400497in}{2.163564in}}%
\pgfpathlineto{\pgfqpoint{1.400497in}{2.159812in}}%
\pgfpathlineto{\pgfqpoint{1.400497in}{2.156060in}}%
\pgfpathlineto{\pgfqpoint{1.400497in}{2.152309in}}%
\pgfpathlineto{\pgfqpoint{1.399713in}{2.151371in}}%
\pgfpathlineto{\pgfqpoint{1.397362in}{2.148557in}}%
\pgfpathlineto{\pgfqpoint{1.397362in}{2.144806in}}%
\pgfpathlineto{\pgfqpoint{1.397362in}{2.141054in}}%
\pgfpathlineto{\pgfqpoint{1.396578in}{2.140116in}}%
\pgfpathlineto{\pgfqpoint{1.394227in}{2.137303in}}%
\pgfpathlineto{\pgfqpoint{1.394227in}{2.133551in}}%
\pgfpathlineto{\pgfqpoint{1.394227in}{2.129800in}}%
\pgfpathlineto{\pgfqpoint{1.394227in}{2.126048in}}%
\pgfpathlineto{\pgfqpoint{1.393444in}{2.125110in}}%
\pgfpathlineto{\pgfqpoint{1.391092in}{2.122297in}}%
\pgfpathlineto{\pgfqpoint{1.391092in}{2.118545in}}%
\pgfpathlineto{\pgfqpoint{1.391092in}{2.114794in}}%
\pgfpathlineto{\pgfqpoint{1.390309in}{2.113856in}}%
\pgfpathlineto{\pgfqpoint{1.387958in}{2.111042in}}%
\pgfpathlineto{\pgfqpoint{1.387958in}{2.107291in}}%
\pgfpathlineto{\pgfqpoint{1.387958in}{2.103539in}}%
\pgfpathlineto{\pgfqpoint{1.387958in}{2.099787in}}%
\pgfpathlineto{\pgfqpoint{1.387174in}{2.098850in}}%
\pgfpathlineto{\pgfqpoint{1.384823in}{2.096036in}}%
\pgfpathlineto{\pgfqpoint{1.384823in}{2.092284in}}%
\pgfpathlineto{\pgfqpoint{1.384823in}{2.088533in}}%
\pgfpathlineto{\pgfqpoint{1.384039in}{2.087595in}}%
\pgfpathlineto{\pgfqpoint{1.381688in}{2.084781in}}%
\pgfpathlineto{\pgfqpoint{1.381688in}{2.081030in}}%
\pgfpathlineto{\pgfqpoint{1.381688in}{2.077278in}}%
\pgfpathlineto{\pgfqpoint{1.381688in}{2.073527in}}%
\pgfpathlineto{\pgfqpoint{1.380905in}{2.072589in}}%
\pgfpathlineto{\pgfqpoint{1.378553in}{2.069775in}}%
\pgfpathlineto{\pgfqpoint{1.378553in}{2.066024in}}%
\pgfpathlineto{\pgfqpoint{1.378553in}{2.062272in}}%
\pgfpathlineto{\pgfqpoint{1.377770in}{2.061334in}}%
\pgfpathlineto{\pgfqpoint{1.375419in}{2.058521in}}%
\pgfpathlineto{\pgfqpoint{1.375419in}{2.054769in}}%
\pgfpathlineto{\pgfqpoint{1.375419in}{2.051018in}}%
\pgfpathlineto{\pgfqpoint{1.375419in}{2.047266in}}%
\pgfpathlineto{\pgfqpoint{1.374635in}{2.046328in}}%
\pgfpathlineto{\pgfqpoint{1.372284in}{2.043514in}}%
\pgfpathlineto{\pgfqpoint{1.372284in}{2.039763in}}%
\pgfpathlineto{\pgfqpoint{1.372284in}{2.036011in}}%
\pgfpathlineto{\pgfqpoint{1.371500in}{2.035073in}}%
\pgfpathlineto{\pgfqpoint{1.369149in}{2.032260in}}%
\pgfpathlineto{\pgfqpoint{1.369149in}{2.028508in}}%
\pgfpathlineto{\pgfqpoint{1.369149in}{2.024757in}}%
\pgfpathlineto{\pgfqpoint{1.368366in}{2.023819in}}%
\pgfpathlineto{\pgfqpoint{1.366015in}{2.021005in}}%
\pgfpathlineto{\pgfqpoint{1.366015in}{2.017254in}}%
\pgfpathlineto{\pgfqpoint{1.366015in}{2.013502in}}%
\pgfpathlineto{\pgfqpoint{1.366015in}{2.009751in}}%
\pgfpathlineto{\pgfqpoint{1.365231in}{2.008813in}}%
\pgfpathlineto{\pgfqpoint{1.362880in}{2.005999in}}%
\pgfpathlineto{\pgfqpoint{1.362880in}{2.002248in}}%
\pgfpathlineto{\pgfqpoint{1.362880in}{1.998496in}}%
\pgfpathlineto{\pgfqpoint{1.362096in}{1.997558in}}%
\pgfpathlineto{\pgfqpoint{1.359745in}{1.994745in}}%
\pgfpathlineto{\pgfqpoint{1.359745in}{1.990993in}}%
\pgfpathlineto{\pgfqpoint{1.359745in}{1.987241in}}%
\pgfpathlineto{\pgfqpoint{1.359745in}{1.983490in}}%
\pgfpathlineto{\pgfqpoint{1.358961in}{1.982552in}}%
\pgfpathlineto{\pgfqpoint{1.356610in}{1.979738in}}%
\pgfpathlineto{\pgfqpoint{1.356610in}{1.975987in}}%
\pgfpathlineto{\pgfqpoint{1.356610in}{1.972235in}}%
\pgfpathlineto{\pgfqpoint{1.355827in}{1.971297in}}%
\pgfpathlineto{\pgfqpoint{1.353476in}{1.968484in}}%
\pgfpathlineto{\pgfqpoint{1.353476in}{1.964732in}}%
\pgfpathlineto{\pgfqpoint{1.353476in}{1.960981in}}%
\pgfpathlineto{\pgfqpoint{1.353476in}{1.957229in}}%
\pgfpathlineto{\pgfqpoint{1.352692in}{1.956291in}}%
\pgfpathlineto{\pgfqpoint{1.350341in}{1.953478in}}%
\pgfpathlineto{\pgfqpoint{1.350341in}{1.949726in}}%
\pgfpathlineto{\pgfqpoint{1.350341in}{1.945975in}}%
\pgfpathlineto{\pgfqpoint{1.349557in}{1.945037in}}%
\pgfpathlineto{\pgfqpoint{1.347206in}{1.942223in}}%
\pgfpathlineto{\pgfqpoint{1.347206in}{1.938471in}}%
\pgfpathlineto{\pgfqpoint{1.347206in}{1.934720in}}%
\pgfpathlineto{\pgfqpoint{1.347206in}{1.930968in}}%
\pgfpathlineto{\pgfqpoint{1.346422in}{1.930031in}}%
\pgfpathlineto{\pgfqpoint{1.344071in}{1.927217in}}%
\pgfpathlineto{\pgfqpoint{1.344071in}{1.923465in}}%
\pgfpathlineto{\pgfqpoint{1.344071in}{1.919714in}}%
\pgfpathlineto{\pgfqpoint{1.343288in}{1.918776in}}%
\pgfpathlineto{\pgfqpoint{1.340937in}{1.915962in}}%
\pgfpathlineto{\pgfqpoint{1.340937in}{1.912211in}}%
\pgfpathlineto{\pgfqpoint{1.340937in}{1.908459in}}%
\pgfpathlineto{\pgfqpoint{1.340937in}{1.904708in}}%
\pgfpathlineto{\pgfqpoint{1.340153in}{1.903770in}}%
\pgfpathlineto{\pgfqpoint{1.337802in}{1.900956in}}%
\pgfpathlineto{\pgfqpoint{1.337802in}{1.897205in}}%
\pgfpathlineto{\pgfqpoint{1.337802in}{1.893453in}}%
\pgfpathlineto{\pgfqpoint{1.337018in}{1.892515in}}%
\pgfpathlineto{\pgfqpoint{1.334667in}{1.889702in}}%
\pgfpathlineto{\pgfqpoint{1.334667in}{1.885950in}}%
\pgfpathlineto{\pgfqpoint{1.334667in}{1.882198in}}%
\pgfpathlineto{\pgfqpoint{1.334667in}{1.878447in}}%
\pgfpathlineto{\pgfqpoint{1.333883in}{1.877509in}}%
\pgfpathlineto{\pgfqpoint{1.331532in}{1.874695in}}%
\pgfpathlineto{\pgfqpoint{1.331532in}{1.870944in}}%
\pgfpathlineto{\pgfqpoint{1.331532in}{1.867192in}}%
\pgfpathlineto{\pgfqpoint{1.330749in}{1.866254in}}%
\pgfpathlineto{\pgfqpoint{1.328398in}{1.863441in}}%
\pgfpathlineto{\pgfqpoint{1.328398in}{1.859689in}}%
\pgfpathlineto{\pgfqpoint{1.328398in}{1.855938in}}%
\pgfpathlineto{\pgfqpoint{1.328398in}{1.852186in}}%
\pgfpathlineto{\pgfqpoint{1.327614in}{1.851248in}}%
\pgfpathlineto{\pgfqpoint{1.325263in}{1.848435in}}%
\pgfpathlineto{\pgfqpoint{1.325263in}{1.844683in}}%
\pgfpathlineto{\pgfqpoint{1.325263in}{1.840932in}}%
\pgfpathlineto{\pgfqpoint{1.324479in}{1.839994in}}%
\pgfpathlineto{\pgfqpoint{1.322128in}{1.837180in}}%
\pgfpathlineto{\pgfqpoint{1.322128in}{1.833429in}}%
\pgfpathlineto{\pgfqpoint{1.322128in}{1.829677in}}%
\pgfpathlineto{\pgfqpoint{1.322128in}{1.825925in}}%
\pgfpathlineto{\pgfqpoint{1.321344in}{1.824988in}}%
\pgfpathlineto{\pgfqpoint{1.318993in}{1.822174in}}%
\pgfpathlineto{\pgfqpoint{1.318993in}{1.818422in}}%
\pgfpathlineto{\pgfqpoint{1.318993in}{1.814671in}}%
\pgfpathlineto{\pgfqpoint{1.318210in}{1.813733in}}%
\pgfpathlineto{\pgfqpoint{1.315859in}{1.810919in}}%
\pgfpathlineto{\pgfqpoint{1.315859in}{1.807168in}}%
\pgfpathlineto{\pgfqpoint{1.315859in}{1.803416in}}%
\pgfpathlineto{\pgfqpoint{1.315075in}{1.802478in}}%
\pgfpathlineto{\pgfqpoint{1.312724in}{1.799665in}}%
\pgfpathlineto{\pgfqpoint{1.312724in}{1.795913in}}%
\pgfpathlineto{\pgfqpoint{1.312724in}{1.792162in}}%
\pgfpathlineto{\pgfqpoint{1.312724in}{1.788410in}}%
\pgfpathlineto{\pgfqpoint{1.311940in}{1.787472in}}%
\pgfpathlineto{\pgfqpoint{1.309589in}{1.784659in}}%
\pgfpathlineto{\pgfqpoint{1.309589in}{1.780907in}}%
\pgfpathlineto{\pgfqpoint{1.309589in}{1.777155in}}%
\pgfpathlineto{\pgfqpoint{1.308805in}{1.776218in}}%
\pgfpathlineto{\pgfqpoint{1.306454in}{1.773404in}}%
\pgfpathlineto{\pgfqpoint{1.306454in}{1.769652in}}%
\pgfpathlineto{\pgfqpoint{1.306454in}{1.765901in}}%
\pgfpathlineto{\pgfqpoint{1.306454in}{1.762149in}}%
\pgfpathlineto{\pgfqpoint{1.305671in}{1.761211in}}%
\pgfpathlineto{\pgfqpoint{1.303320in}{1.758398in}}%
\pgfpathlineto{\pgfqpoint{1.303320in}{1.754646in}}%
\pgfpathlineto{\pgfqpoint{1.303320in}{1.750895in}}%
\pgfpathlineto{\pgfqpoint{1.302536in}{1.749957in}}%
\pgfpathlineto{\pgfqpoint{1.300185in}{1.747143in}}%
\pgfpathlineto{\pgfqpoint{1.300185in}{1.743392in}}%
\pgfpathlineto{\pgfqpoint{1.300185in}{1.739640in}}%
\pgfpathlineto{\pgfqpoint{1.300185in}{1.735889in}}%
\pgfpathlineto{\pgfqpoint{1.299401in}{1.734951in}}%
\pgfpathlineto{\pgfqpoint{1.297050in}{1.732137in}}%
\pgfpathlineto{\pgfqpoint{1.297050in}{1.728386in}}%
\pgfpathlineto{\pgfqpoint{1.297050in}{1.724634in}}%
\pgfpathlineto{\pgfqpoint{1.296266in}{1.723696in}}%
\pgfpathlineto{\pgfqpoint{1.293915in}{1.720882in}}%
\pgfpathlineto{\pgfqpoint{1.293915in}{1.717131in}}%
\pgfpathlineto{\pgfqpoint{1.293915in}{1.713379in}}%
\pgfpathlineto{\pgfqpoint{1.293915in}{1.709628in}}%
\pgfpathlineto{\pgfqpoint{1.293132in}{1.708690in}}%
\pgfpathlineto{\pgfqpoint{1.290781in}{1.705876in}}%
\pgfpathlineto{\pgfqpoint{1.290781in}{1.702125in}}%
\pgfpathlineto{\pgfqpoint{1.293132in}{1.699311in}}%
\pgfpathlineto{\pgfqpoint{1.293915in}{1.698373in}}%
\pgfpathlineto{\pgfqpoint{1.296266in}{1.695560in}}%
\pgfpathlineto{\pgfqpoint{1.297050in}{1.694622in}}%
\pgfpathlineto{\pgfqpoint{1.299401in}{1.691808in}}%
\pgfpathlineto{\pgfqpoint{1.302536in}{1.691808in}}%
\pgfpathlineto{\pgfqpoint{1.303320in}{1.690870in}}%
\pgfpathlineto{\pgfqpoint{1.305671in}{1.688057in}}%
\pgfpathlineto{\pgfqpoint{1.306454in}{1.687119in}}%
\pgfpathlineto{\pgfqpoint{1.308805in}{1.684305in}}%
\pgfpathlineto{\pgfqpoint{1.309589in}{1.683367in}}%
\pgfpathlineto{\pgfqpoint{1.311940in}{1.680553in}}%
\pgfpathlineto{\pgfqpoint{1.315075in}{1.680553in}}%
\pgfpathlineto{\pgfqpoint{1.315859in}{1.679616in}}%
\pgfpathlineto{\pgfqpoint{1.318210in}{1.676802in}}%
\pgfpathlineto{\pgfqpoint{1.318993in}{1.675864in}}%
\pgfpathlineto{\pgfqpoint{1.321344in}{1.673050in}}%
\pgfpathlineto{\pgfqpoint{1.322128in}{1.672113in}}%
\pgfpathlineto{\pgfqpoint{1.324479in}{1.669299in}}%
\pgfpathlineto{\pgfqpoint{1.327614in}{1.669299in}}%
\pgfpathlineto{\pgfqpoint{1.328398in}{1.668361in}}%
\pgfpathlineto{\pgfqpoint{1.330749in}{1.665547in}}%
\pgfpathlineto{\pgfqpoint{1.331532in}{1.664609in}}%
\pgfpathlineto{\pgfqpoint{1.333883in}{1.661796in}}%
\pgfpathlineto{\pgfqpoint{1.334667in}{1.660858in}}%
\pgfpathlineto{\pgfqpoint{1.337018in}{1.658044in}}%
\pgfpathlineto{\pgfqpoint{1.337802in}{1.657106in}}%
\pgfpathlineto{\pgfqpoint{1.340153in}{1.654293in}}%
\pgfpathlineto{\pgfqpoint{1.343288in}{1.654293in}}%
\pgfpathlineto{\pgfqpoint{1.344071in}{1.653355in}}%
\pgfpathlineto{\pgfqpoint{1.346422in}{1.650541in}}%
\pgfpathlineto{\pgfqpoint{1.347206in}{1.649603in}}%
\pgfpathlineto{\pgfqpoint{1.349557in}{1.646790in}}%
\pgfpathlineto{\pgfqpoint{1.350341in}{1.645852in}}%
\pgfpathlineto{\pgfqpoint{1.352692in}{1.643038in}}%
\pgfpathlineto{\pgfqpoint{1.355827in}{1.643038in}}%
\pgfpathlineto{\pgfqpoint{1.356610in}{1.642100in}}%
\pgfpathlineto{\pgfqpoint{1.358961in}{1.639287in}}%
\pgfpathlineto{\pgfqpoint{1.359745in}{1.638349in}}%
\pgfpathlineto{\pgfqpoint{1.362096in}{1.635535in}}%
\pgfpathlineto{\pgfqpoint{1.362880in}{1.634597in}}%
\pgfpathlineto{\pgfqpoint{1.365231in}{1.631784in}}%
\pgfpathlineto{\pgfqpoint{1.368366in}{1.631784in}}%
\pgfpathlineto{\pgfqpoint{1.369149in}{1.630846in}}%
\pgfpathlineto{\pgfqpoint{1.371500in}{1.628032in}}%
\pgfpathlineto{\pgfqpoint{1.372284in}{1.627094in}}%
\pgfpathlineto{\pgfqpoint{1.374635in}{1.624280in}}%
\pgfpathlineto{\pgfqpoint{1.375419in}{1.623343in}}%
\pgfpathlineto{\pgfqpoint{1.377770in}{1.620529in}}%
\pgfpathlineto{\pgfqpoint{1.380905in}{1.620529in}}%
\pgfpathlineto{\pgfqpoint{1.381688in}{1.619591in}}%
\pgfpathlineto{\pgfqpoint{1.384039in}{1.616777in}}%
\pgfpathlineto{\pgfqpoint{1.384823in}{1.615840in}}%
\pgfpathlineto{\pgfqpoint{1.387174in}{1.613026in}}%
\pgfpathlineto{\pgfqpoint{1.387958in}{1.612088in}}%
\pgfpathlineto{\pgfqpoint{1.390309in}{1.609274in}}%
\pgfpathlineto{\pgfqpoint{1.393444in}{1.609274in}}%
\pgfpathlineto{\pgfqpoint{1.394227in}{1.608336in}}%
\pgfpathlineto{\pgfqpoint{1.396578in}{1.605523in}}%
\pgfpathlineto{\pgfqpoint{1.397362in}{1.604585in}}%
\pgfpathlineto{\pgfqpoint{1.399713in}{1.601771in}}%
\pgfpathlineto{\pgfqpoint{1.400497in}{1.600833in}}%
\pgfpathlineto{\pgfqpoint{1.402848in}{1.598020in}}%
\pgfpathlineto{\pgfqpoint{1.405982in}{1.598020in}}%
\pgfpathlineto{\pgfqpoint{1.406766in}{1.597082in}}%
\pgfpathlineto{\pgfqpoint{1.409117in}{1.594268in}}%
\pgfpathlineto{\pgfqpoint{1.409901in}{1.593330in}}%
\pgfpathlineto{\pgfqpoint{1.412252in}{1.590517in}}%
\pgfpathlineto{\pgfqpoint{1.413036in}{1.589579in}}%
\pgfpathlineto{\pgfqpoint{1.415387in}{1.586765in}}%
\pgfpathlineto{\pgfqpoint{1.416170in}{1.585827in}}%
\pgfpathlineto{\pgfqpoint{1.418521in}{1.583014in}}%
\pgfpathlineto{\pgfqpoint{1.421656in}{1.583014in}}%
\pgfpathlineto{\pgfqpoint{1.422440in}{1.582076in}}%
\pgfpathlineto{\pgfqpoint{1.424791in}{1.579262in}}%
\pgfpathlineto{\pgfqpoint{1.425575in}{1.578324in}}%
\pgfpathlineto{\pgfqpoint{1.427926in}{1.575511in}}%
\pgfpathlineto{\pgfqpoint{1.428709in}{1.574573in}}%
\pgfpathlineto{\pgfqpoint{1.431060in}{1.571759in}}%
\pgfpathlineto{\pgfqpoint{1.434195in}{1.571759in}}%
\pgfpathlineto{\pgfqpoint{1.434979in}{1.570821in}}%
\pgfpathlineto{\pgfqpoint{1.437330in}{1.568007in}}%
\pgfpathlineto{\pgfqpoint{1.438114in}{1.567070in}}%
\pgfpathlineto{\pgfqpoint{1.440465in}{1.564256in}}%
\pgfpathlineto{\pgfqpoint{1.441248in}{1.563318in}}%
\pgfpathlineto{\pgfqpoint{1.443599in}{1.560504in}}%
\pgfpathlineto{\pgfqpoint{1.446734in}{1.560504in}}%
\pgfpathlineto{\pgfqpoint{1.447518in}{1.559566in}}%
\pgfpathlineto{\pgfqpoint{1.449869in}{1.556753in}}%
\pgfpathlineto{\pgfqpoint{1.450653in}{1.555815in}}%
\pgfpathlineto{\pgfqpoint{1.453004in}{1.553001in}}%
\pgfpathlineto{\pgfqpoint{1.453787in}{1.552063in}}%
\pgfpathlineto{\pgfqpoint{1.456138in}{1.549250in}}%
\pgfpathlineto{\pgfqpoint{1.459273in}{1.549250in}}%
\pgfpathlineto{\pgfqpoint{1.460057in}{1.548312in}}%
\pgfpathlineto{\pgfqpoint{1.462408in}{1.545498in}}%
\pgfpathlineto{\pgfqpoint{1.463192in}{1.544560in}}%
\pgfpathlineto{\pgfqpoint{1.465543in}{1.541747in}}%
\pgfpathlineto{\pgfqpoint{1.466326in}{1.540809in}}%
\pgfpathlineto{\pgfqpoint{1.468677in}{1.537995in}}%
\pgfpathlineto{\pgfqpoint{1.471812in}{1.537995in}}%
\pgfpathlineto{\pgfqpoint{1.472596in}{1.537057in}}%
\pgfpathlineto{\pgfqpoint{1.474947in}{1.534244in}}%
\pgfpathlineto{\pgfqpoint{1.475730in}{1.533306in}}%
\pgfpathlineto{\pgfqpoint{1.478082in}{1.530492in}}%
\pgfpathlineto{\pgfqpoint{1.478865in}{1.529554in}}%
\pgfpathlineto{\pgfqpoint{1.481216in}{1.526741in}}%
\pgfpathlineto{\pgfqpoint{1.484351in}{1.526741in}}%
\pgfpathlineto{\pgfqpoint{1.485135in}{1.525803in}}%
\pgfpathlineto{\pgfqpoint{1.487486in}{1.522989in}}%
\pgfpathlineto{\pgfqpoint{1.488269in}{1.522051in}}%
\pgfpathlineto{\pgfqpoint{1.490621in}{1.519238in}}%
\pgfpathlineto{\pgfqpoint{1.491404in}{1.518300in}}%
\pgfpathlineto{\pgfqpoint{1.493755in}{1.515486in}}%
\pgfpathlineto{\pgfqpoint{1.494539in}{1.514548in}}%
\pgfpathlineto{\pgfqpoint{1.496890in}{1.511734in}}%
\pgfpathlineto{\pgfqpoint{1.500025in}{1.511734in}}%
\pgfpathlineto{\pgfqpoint{1.500808in}{1.510797in}}%
\pgfpathlineto{\pgfqpoint{1.503159in}{1.507983in}}%
\pgfpathlineto{\pgfqpoint{1.503943in}{1.507045in}}%
\pgfpathlineto{\pgfqpoint{1.506294in}{1.504231in}}%
\pgfpathlineto{\pgfqpoint{1.507078in}{1.503293in}}%
\pgfpathlineto{\pgfqpoint{1.509429in}{1.500480in}}%
\pgfpathlineto{\pgfqpoint{1.512564in}{1.500480in}}%
\pgfpathlineto{\pgfqpoint{1.513347in}{1.499542in}}%
\pgfpathlineto{\pgfqpoint{1.515698in}{1.496728in}}%
\pgfpathlineto{\pgfqpoint{1.516482in}{1.495790in}}%
\pgfpathlineto{\pgfqpoint{1.518833in}{1.492977in}}%
\pgfpathlineto{\pgfqpoint{1.519617in}{1.492039in}}%
\pgfpathlineto{\pgfqpoint{1.521968in}{1.489225in}}%
\pgfpathlineto{\pgfqpoint{1.525103in}{1.489225in}}%
\pgfpathlineto{\pgfqpoint{1.525886in}{1.488287in}}%
\pgfpathlineto{\pgfqpoint{1.528237in}{1.485474in}}%
\pgfpathlineto{\pgfqpoint{1.529021in}{1.484536in}}%
\pgfpathlineto{\pgfqpoint{1.531372in}{1.481722in}}%
\pgfpathlineto{\pgfqpoint{1.532156in}{1.480784in}}%
\pgfpathlineto{\pgfqpoint{1.534507in}{1.477971in}}%
\pgfpathlineto{\pgfqpoint{1.537642in}{1.477971in}}%
\pgfpathlineto{\pgfqpoint{1.538425in}{1.477033in}}%
\pgfpathlineto{\pgfqpoint{1.540776in}{1.474219in}}%
\pgfpathlineto{\pgfqpoint{1.541560in}{1.473281in}}%
\pgfpathlineto{\pgfqpoint{1.543911in}{1.470468in}}%
\pgfpathlineto{\pgfqpoint{1.544695in}{1.469530in}}%
\pgfpathlineto{\pgfqpoint{1.547046in}{1.466716in}}%
\pgfpathlineto{\pgfqpoint{1.550181in}{1.466716in}}%
\pgfpathlineto{\pgfqpoint{1.550964in}{1.465778in}}%
\pgfpathlineto{\pgfqpoint{1.553315in}{1.462964in}}%
\pgfpathlineto{\pgfqpoint{1.554099in}{1.462027in}}%
\pgfpathlineto{\pgfqpoint{1.556450in}{1.459213in}}%
\pgfpathlineto{\pgfqpoint{1.557234in}{1.458275in}}%
\pgfpathlineto{\pgfqpoint{1.559585in}{1.455461in}}%
\pgfpathlineto{\pgfqpoint{1.560369in}{1.454524in}}%
\pgfpathlineto{\pgfqpoint{1.562720in}{1.451710in}}%
\pgfpathlineto{\pgfqpoint{1.565854in}{1.451710in}}%
\pgfpathlineto{\pgfqpoint{1.566638in}{1.450772in}}%
\pgfpathlineto{\pgfqpoint{1.568989in}{1.447958in}}%
\pgfpathlineto{\pgfqpoint{1.569773in}{1.447020in}}%
\pgfpathlineto{\pgfqpoint{1.572124in}{1.444207in}}%
\pgfpathlineto{\pgfqpoint{1.572908in}{1.443269in}}%
\pgfpathlineto{\pgfqpoint{1.575259in}{1.440455in}}%
\pgfpathlineto{\pgfqpoint{1.578393in}{1.440455in}}%
\pgfpathlineto{\pgfqpoint{1.579177in}{1.439517in}}%
\pgfpathlineto{\pgfqpoint{1.581528in}{1.436704in}}%
\pgfpathlineto{\pgfqpoint{1.582312in}{1.435766in}}%
\pgfpathlineto{\pgfqpoint{1.584663in}{1.432952in}}%
\pgfpathlineto{\pgfqpoint{1.585446in}{1.432014in}}%
\pgfpathlineto{\pgfqpoint{1.587798in}{1.429201in}}%
\pgfpathlineto{\pgfqpoint{1.590932in}{1.429201in}}%
\pgfpathlineto{\pgfqpoint{1.591716in}{1.428263in}}%
\pgfpathlineto{\pgfqpoint{1.594067in}{1.425449in}}%
\pgfpathlineto{\pgfqpoint{1.594851in}{1.424511in}}%
\pgfpathlineto{\pgfqpoint{1.597202in}{1.421698in}}%
\pgfpathlineto{\pgfqpoint{1.597985in}{1.420760in}}%
\pgfpathlineto{\pgfqpoint{1.600337in}{1.417946in}}%
\pgfpathlineto{\pgfqpoint{1.603471in}{1.417946in}}%
\pgfpathlineto{\pgfqpoint{1.604255in}{1.417008in}}%
\pgfpathlineto{\pgfqpoint{1.606606in}{1.414195in}}%
\pgfpathlineto{\pgfqpoint{1.607390in}{1.413257in}}%
\pgfpathlineto{\pgfqpoint{1.609741in}{1.410443in}}%
\pgfpathlineto{\pgfqpoint{1.610524in}{1.409505in}}%
\pgfpathlineto{\pgfqpoint{1.612875in}{1.406691in}}%
\pgfpathlineto{\pgfqpoint{1.616010in}{1.406691in}}%
\pgfpathlineto{\pgfqpoint{1.616794in}{1.405754in}}%
\pgfpathlineto{\pgfqpoint{1.619145in}{1.402940in}}%
\pgfpathlineto{\pgfqpoint{1.619929in}{1.402002in}}%
\pgfpathlineto{\pgfqpoint{1.622280in}{1.399188in}}%
\pgfpathlineto{\pgfqpoint{1.623063in}{1.398251in}}%
\pgfpathlineto{\pgfqpoint{1.625414in}{1.395437in}}%
\pgfpathlineto{\pgfqpoint{1.628549in}{1.395437in}}%
\pgfpathlineto{\pgfqpoint{1.629333in}{1.394499in}}%
\pgfpathlineto{\pgfqpoint{1.631684in}{1.391685in}}%
\pgfpathlineto{\pgfqpoint{1.632468in}{1.390747in}}%
\pgfpathlineto{\pgfqpoint{1.634819in}{1.387934in}}%
\pgfpathlineto{\pgfqpoint{1.635602in}{1.386996in}}%
\pgfpathlineto{\pgfqpoint{1.637953in}{1.384182in}}%
\pgfpathlineto{\pgfqpoint{1.638737in}{1.383244in}}%
\pgfpathlineto{\pgfqpoint{1.641088in}{1.380431in}}%
\pgfpathlineto{\pgfqpoint{1.644223in}{1.380431in}}%
\pgfpathlineto{\pgfqpoint{1.645007in}{1.379493in}}%
\pgfpathlineto{\pgfqpoint{1.647358in}{1.376679in}}%
\pgfpathlineto{\pgfqpoint{1.648141in}{1.375741in}}%
\pgfpathlineto{\pgfqpoint{1.650492in}{1.372928in}}%
\pgfpathlineto{\pgfqpoint{1.651276in}{1.371990in}}%
\pgfpathlineto{\pgfqpoint{1.653627in}{1.369176in}}%
\pgfpathlineto{\pgfqpoint{1.656762in}{1.369176in}}%
\pgfpathlineto{\pgfqpoint{1.657546in}{1.368238in}}%
\pgfpathlineto{\pgfqpoint{1.659897in}{1.365425in}}%
\pgfpathlineto{\pgfqpoint{1.660680in}{1.364487in}}%
\pgfpathlineto{\pgfqpoint{1.663031in}{1.361673in}}%
\pgfpathlineto{\pgfqpoint{1.663815in}{1.360735in}}%
\pgfpathlineto{\pgfqpoint{1.666166in}{1.357922in}}%
\pgfpathlineto{\pgfqpoint{1.669301in}{1.357922in}}%
\pgfpathlineto{\pgfqpoint{1.670085in}{1.356984in}}%
\pgfpathlineto{\pgfqpoint{1.672436in}{1.354170in}}%
\pgfpathlineto{\pgfqpoint{1.673219in}{1.353232in}}%
\pgfpathlineto{\pgfqpoint{1.675570in}{1.350418in}}%
\pgfpathlineto{\pgfqpoint{1.676354in}{1.349481in}}%
\pgfpathlineto{\pgfqpoint{1.678705in}{1.346667in}}%
\pgfpathlineto{\pgfqpoint{1.681840in}{1.346667in}}%
\pgfpathlineto{\pgfqpoint{1.682623in}{1.345729in}}%
\pgfpathlineto{\pgfqpoint{1.684975in}{1.342915in}}%
\pgfpathlineto{\pgfqpoint{1.685758in}{1.341977in}}%
\pgfpathlineto{\pgfqpoint{1.688109in}{1.339164in}}%
\pgfpathlineto{\pgfqpoint{1.688893in}{1.338226in}}%
\pgfpathlineto{\pgfqpoint{1.691244in}{1.335412in}}%
\pgfpathlineto{\pgfqpoint{1.694379in}{1.335412in}}%
\pgfpathlineto{\pgfqpoint{1.695162in}{1.334474in}}%
\pgfpathlineto{\pgfqpoint{1.697514in}{1.331661in}}%
\pgfpathlineto{\pgfqpoint{1.698297in}{1.330723in}}%
\pgfpathlineto{\pgfqpoint{1.700648in}{1.327909in}}%
\pgfpathlineto{\pgfqpoint{1.701432in}{1.326971in}}%
\pgfpathlineto{\pgfqpoint{1.703783in}{1.324158in}}%
\pgfpathlineto{\pgfqpoint{1.706918in}{1.324158in}}%
\pgfpathlineto{\pgfqpoint{1.707701in}{1.323220in}}%
\pgfpathlineto{\pgfqpoint{1.710052in}{1.320406in}}%
\pgfpathlineto{\pgfqpoint{1.710836in}{1.319468in}}%
\pgfpathlineto{\pgfqpoint{1.713187in}{1.316655in}}%
\pgfpathlineto{\pgfqpoint{1.713971in}{1.315717in}}%
\pgfpathlineto{\pgfqpoint{1.716322in}{1.312903in}}%
\pgfpathlineto{\pgfqpoint{1.717106in}{1.311965in}}%
\pgfpathlineto{\pgfqpoint{1.719457in}{1.309152in}}%
\pgfpathlineto{\pgfqpoint{1.722591in}{1.309152in}}%
\pgfpathlineto{\pgfqpoint{1.723375in}{1.308214in}}%
\pgfpathlineto{\pgfqpoint{1.725726in}{1.305400in}}%
\pgfpathlineto{\pgfqpoint{1.726510in}{1.304462in}}%
\pgfpathlineto{\pgfqpoint{1.728861in}{1.301648in}}%
\pgfpathlineto{\pgfqpoint{1.729645in}{1.300711in}}%
\pgfpathlineto{\pgfqpoint{1.731996in}{1.297897in}}%
\pgfpathlineto{\pgfqpoint{1.735130in}{1.297897in}}%
\pgfpathlineto{\pgfqpoint{1.735914in}{1.296959in}}%
\pgfpathlineto{\pgfqpoint{1.738265in}{1.294145in}}%
\pgfpathlineto{\pgfqpoint{1.739049in}{1.293208in}}%
\pgfpathlineto{\pgfqpoint{1.741400in}{1.290394in}}%
\pgfpathlineto{\pgfqpoint{1.742184in}{1.289456in}}%
\pgfpathlineto{\pgfqpoint{1.744535in}{1.286642in}}%
\pgfpathlineto{\pgfqpoint{1.747669in}{1.286642in}}%
\pgfpathlineto{\pgfqpoint{1.748453in}{1.285704in}}%
\pgfpathlineto{\pgfqpoint{1.750804in}{1.282891in}}%
\pgfpathlineto{\pgfqpoint{1.751588in}{1.281953in}}%
\pgfpathlineto{\pgfqpoint{1.753939in}{1.279139in}}%
\pgfpathlineto{\pgfqpoint{1.754723in}{1.278201in}}%
\pgfpathlineto{\pgfqpoint{1.757074in}{1.275388in}}%
\pgfpathlineto{\pgfqpoint{1.760208in}{1.275388in}}%
\pgfpathlineto{\pgfqpoint{1.760992in}{1.274450in}}%
\pgfpathlineto{\pgfqpoint{1.763343in}{1.271636in}}%
\pgfpathlineto{\pgfqpoint{1.764127in}{1.270698in}}%
\pgfpathlineto{\pgfqpoint{1.766478in}{1.267885in}}%
\pgfpathlineto{\pgfqpoint{1.767262in}{1.266947in}}%
\pgfpathlineto{\pgfqpoint{1.769613in}{1.264133in}}%
\pgfpathlineto{\pgfqpoint{1.772747in}{1.264133in}}%
\pgfpathlineto{\pgfqpoint{1.773531in}{1.263195in}}%
\pgfpathlineto{\pgfqpoint{1.775882in}{1.260382in}}%
\pgfpathlineto{\pgfqpoint{1.776666in}{1.259444in}}%
\pgfpathlineto{\pgfqpoint{1.779017in}{1.256630in}}%
\pgfpathlineto{\pgfqpoint{1.779800in}{1.255692in}}%
\pgfpathlineto{\pgfqpoint{1.782152in}{1.252879in}}%
\pgfpathlineto{\pgfqpoint{1.785286in}{1.252879in}}%
\pgfpathlineto{\pgfqpoint{1.786070in}{1.251941in}}%
\pgfpathlineto{\pgfqpoint{1.788421in}{1.249127in}}%
\pgfpathlineto{\pgfqpoint{1.789205in}{1.248189in}}%
\pgfpathlineto{\pgfqpoint{1.791556in}{1.245375in}}%
\pgfpathlineto{\pgfqpoint{1.792339in}{1.244438in}}%
\pgfpathlineto{\pgfqpoint{1.794691in}{1.241624in}}%
\pgfpathlineto{\pgfqpoint{1.795474in}{1.240686in}}%
\pgfpathlineto{\pgfqpoint{1.797825in}{1.237872in}}%
\pgfpathlineto{\pgfqpoint{1.800960in}{1.237872in}}%
\pgfpathlineto{\pgfqpoint{1.801744in}{1.236935in}}%
\pgfpathlineto{\pgfqpoint{1.804095in}{1.234121in}}%
\pgfpathlineto{\pgfqpoint{1.804878in}{1.233183in}}%
\pgfpathlineto{\pgfqpoint{1.807229in}{1.230369in}}%
\pgfpathlineto{\pgfqpoint{1.808013in}{1.229431in}}%
\pgfpathlineto{\pgfqpoint{1.810364in}{1.226618in}}%
\pgfpathlineto{\pgfqpoint{1.813499in}{1.226618in}}%
\pgfpathlineto{\pgfqpoint{1.814283in}{1.225680in}}%
\pgfpathlineto{\pgfqpoint{1.816634in}{1.222866in}}%
\pgfpathlineto{\pgfqpoint{1.817417in}{1.221928in}}%
\pgfpathlineto{\pgfqpoint{1.819768in}{1.219115in}}%
\pgfpathlineto{\pgfqpoint{1.820552in}{1.218177in}}%
\pgfpathlineto{\pgfqpoint{1.822903in}{1.215363in}}%
\pgfpathlineto{\pgfqpoint{1.826038in}{1.215363in}}%
\pgfpathlineto{\pgfqpoint{1.826822in}{1.214425in}}%
\pgfpathlineto{\pgfqpoint{1.829173in}{1.211612in}}%
\pgfpathlineto{\pgfqpoint{1.829956in}{1.210674in}}%
\pgfpathlineto{\pgfqpoint{1.832307in}{1.207860in}}%
\pgfpathlineto{\pgfqpoint{1.833091in}{1.206922in}}%
\pgfpathlineto{\pgfqpoint{1.835442in}{1.204109in}}%
\pgfpathlineto{\pgfqpoint{1.838577in}{1.204109in}}%
\pgfpathlineto{\pgfqpoint{1.839361in}{1.203171in}}%
\pgfpathlineto{\pgfqpoint{1.841712in}{1.200357in}}%
\pgfpathlineto{\pgfqpoint{1.842495in}{1.199419in}}%
\pgfpathlineto{\pgfqpoint{1.844846in}{1.196606in}}%
\pgfpathlineto{\pgfqpoint{1.845630in}{1.195668in}}%
\pgfpathlineto{\pgfqpoint{1.847981in}{1.192854in}}%
\pgfpathlineto{\pgfqpoint{1.851116in}{1.192854in}}%
\pgfpathlineto{\pgfqpoint{1.851900in}{1.191916in}}%
\pgfpathlineto{\pgfqpoint{1.854251in}{1.189102in}}%
\pgfpathlineto{\pgfqpoint{1.855034in}{1.188165in}}%
\pgfpathlineto{\pgfqpoint{1.857385in}{1.185351in}}%
\pgfpathlineto{\pgfqpoint{1.858169in}{1.184413in}}%
\pgfpathlineto{\pgfqpoint{1.860520in}{1.181599in}}%
\pgfpathlineto{\pgfqpoint{1.861304in}{1.180662in}}%
\pgfpathlineto{\pgfqpoint{1.863655in}{1.177848in}}%
\pgfpathlineto{\pgfqpoint{1.866790in}{1.177848in}}%
\pgfpathlineto{\pgfqpoint{1.867573in}{1.176910in}}%
\pgfpathlineto{\pgfqpoint{1.869924in}{1.174096in}}%
\pgfpathlineto{\pgfqpoint{1.870708in}{1.173158in}}%
\pgfpathlineto{\pgfqpoint{1.873059in}{1.170345in}}%
\pgfpathlineto{\pgfqpoint{1.873843in}{1.169407in}}%
\pgfpathlineto{\pgfqpoint{1.876194in}{1.166593in}}%
\pgfpathlineto{\pgfqpoint{1.879329in}{1.166593in}}%
\pgfpathlineto{\pgfqpoint{1.880112in}{1.165655in}}%
\pgfpathlineto{\pgfqpoint{1.882463in}{1.162842in}}%
\pgfpathlineto{\pgfqpoint{1.883247in}{1.161904in}}%
\pgfpathlineto{\pgfqpoint{1.885598in}{1.159090in}}%
\pgfpathlineto{\pgfqpoint{1.886382in}{1.158152in}}%
\pgfpathlineto{\pgfqpoint{1.888733in}{1.155339in}}%
\pgfpathlineto{\pgfqpoint{1.891868in}{1.155339in}}%
\pgfpathlineto{\pgfqpoint{1.892651in}{1.154401in}}%
\pgfpathlineto{\pgfqpoint{1.895002in}{1.151587in}}%
\pgfpathlineto{\pgfqpoint{1.895786in}{1.150649in}}%
\pgfpathlineto{\pgfqpoint{1.898137in}{1.147836in}}%
\pgfpathlineto{\pgfqpoint{1.898921in}{1.146898in}}%
\pgfpathlineto{\pgfqpoint{1.901272in}{1.144084in}}%
\pgfpathlineto{\pgfqpoint{1.904407in}{1.144084in}}%
\pgfpathlineto{\pgfqpoint{1.905190in}{1.143146in}}%
\pgfpathlineto{\pgfqpoint{1.907541in}{1.140333in}}%
\pgfpathlineto{\pgfqpoint{1.908325in}{1.139395in}}%
\pgfpathlineto{\pgfqpoint{1.910676in}{1.136581in}}%
\pgfpathlineto{\pgfqpoint{1.911460in}{1.135643in}}%
\pgfpathlineto{\pgfqpoint{1.913811in}{1.132829in}}%
\pgfpathlineto{\pgfqpoint{1.916945in}{1.132829in}}%
\pgfpathlineto{\pgfqpoint{1.917729in}{1.131892in}}%
\pgfpathlineto{\pgfqpoint{1.920080in}{1.129078in}}%
\pgfpathlineto{\pgfqpoint{1.920864in}{1.128140in}}%
\pgfpathlineto{\pgfqpoint{1.923215in}{1.125326in}}%
\pgfpathlineto{\pgfqpoint{1.923999in}{1.124388in}}%
\pgfpathlineto{\pgfqpoint{1.926350in}{1.121575in}}%
\pgfpathlineto{\pgfqpoint{1.929484in}{1.121575in}}%
\pgfpathlineto{\pgfqpoint{1.930268in}{1.120637in}}%
\pgfpathlineto{\pgfqpoint{1.932619in}{1.117823in}}%
\pgfpathlineto{\pgfqpoint{1.933403in}{1.116885in}}%
\pgfpathlineto{\pgfqpoint{1.935754in}{1.114072in}}%
\pgfpathlineto{\pgfqpoint{1.936538in}{1.113134in}}%
\pgfpathlineto{\pgfqpoint{1.938889in}{1.110320in}}%
\pgfpathlineto{\pgfqpoint{1.939672in}{1.109382in}}%
\pgfpathlineto{\pgfqpoint{1.942023in}{1.106569in}}%
\pgfpathlineto{\pgfqpoint{1.945158in}{1.106569in}}%
\pgfpathlineto{\pgfqpoint{1.945942in}{1.105631in}}%
\pgfpathlineto{\pgfqpoint{1.948293in}{1.102817in}}%
\pgfpathlineto{\pgfqpoint{1.949077in}{1.101879in}}%
\pgfpathlineto{\pgfqpoint{1.951428in}{1.099066in}}%
\pgfpathlineto{\pgfqpoint{1.952211in}{1.098128in}}%
\pgfpathlineto{\pgfqpoint{1.954562in}{1.095314in}}%
\pgfpathlineto{\pgfqpoint{1.957697in}{1.095314in}}%
\pgfpathlineto{\pgfqpoint{1.958481in}{1.094376in}}%
\pgfpathlineto{\pgfqpoint{1.960832in}{1.091563in}}%
\pgfpathlineto{\pgfqpoint{1.961616in}{1.090625in}}%
\pgfpathlineto{\pgfqpoint{1.963967in}{1.087811in}}%
\pgfpathlineto{\pgfqpoint{1.964750in}{1.086873in}}%
\pgfpathlineto{\pgfqpoint{1.967101in}{1.084059in}}%
\pgfpathlineto{\pgfqpoint{1.970236in}{1.084059in}}%
\pgfpathlineto{\pgfqpoint{1.971020in}{1.083122in}}%
\pgfpathlineto{\pgfqpoint{1.973371in}{1.080308in}}%
\pgfpathlineto{\pgfqpoint{1.974155in}{1.079370in}}%
\pgfpathlineto{\pgfqpoint{1.976506in}{1.076556in}}%
\pgfpathlineto{\pgfqpoint{1.977289in}{1.075619in}}%
\pgfpathlineto{\pgfqpoint{1.979640in}{1.072805in}}%
\pgfpathlineto{\pgfqpoint{1.982775in}{1.072805in}}%
\pgfpathlineto{\pgfqpoint{1.983559in}{1.071867in}}%
\pgfpathlineto{\pgfqpoint{1.985910in}{1.069053in}}%
\pgfpathlineto{\pgfqpoint{1.986693in}{1.068115in}}%
\pgfpathlineto{\pgfqpoint{1.989045in}{1.065302in}}%
\pgfpathlineto{\pgfqpoint{1.989828in}{1.064364in}}%
\pgfpathlineto{\pgfqpoint{1.992179in}{1.061550in}}%
\pgfpathlineto{\pgfqpoint{1.995314in}{1.061550in}}%
\pgfpathlineto{\pgfqpoint{1.996098in}{1.060612in}}%
\pgfpathlineto{\pgfqpoint{1.998449in}{1.057799in}}%
\pgfpathlineto{\pgfqpoint{1.999232in}{1.056861in}}%
\pgfpathlineto{\pgfqpoint{2.001584in}{1.054047in}}%
\pgfpathlineto{\pgfqpoint{2.002367in}{1.053109in}}%
\pgfpathlineto{\pgfqpoint{2.004718in}{1.050296in}}%
\pgfpathlineto{\pgfqpoint{2.007853in}{1.050296in}}%
\pgfpathlineto{\pgfqpoint{2.008637in}{1.049358in}}%
\pgfpathlineto{\pgfqpoint{2.010988in}{1.046544in}}%
\pgfpathlineto{\pgfqpoint{2.011771in}{1.045606in}}%
\pgfpathlineto{\pgfqpoint{2.014122in}{1.042793in}}%
\pgfpathlineto{\pgfqpoint{2.014906in}{1.041855in}}%
\pgfpathlineto{\pgfqpoint{2.017257in}{1.039041in}}%
\pgfpathlineto{\pgfqpoint{2.018041in}{1.038103in}}%
\pgfpathlineto{\pgfqpoint{2.020392in}{1.035290in}}%
\pgfpathlineto{\pgfqpoint{2.023527in}{1.035290in}}%
\pgfpathlineto{\pgfqpoint{2.024310in}{1.034352in}}%
\pgfpathlineto{\pgfqpoint{2.026661in}{1.031538in}}%
\pgfpathlineto{\pgfqpoint{2.027445in}{1.030600in}}%
\pgfpathlineto{\pgfqpoint{2.029796in}{1.027786in}}%
\pgfpathlineto{\pgfqpoint{2.030580in}{1.026849in}}%
\pgfpathlineto{\pgfqpoint{2.032931in}{1.024035in}}%
\pgfpathlineto{\pgfqpoint{2.036066in}{1.024035in}}%
\pgfpathlineto{\pgfqpoint{2.036849in}{1.023097in}}%
\pgfpathlineto{\pgfqpoint{2.039200in}{1.020283in}}%
\pgfpathlineto{\pgfqpoint{2.039984in}{1.019346in}}%
\pgfpathlineto{\pgfqpoint{2.042335in}{1.016532in}}%
\pgfpathlineto{\pgfqpoint{2.043119in}{1.015594in}}%
\pgfpathlineto{\pgfqpoint{2.045470in}{1.012780in}}%
\pgfpathlineto{\pgfqpoint{2.048605in}{1.012780in}}%
\pgfpathlineto{\pgfqpoint{2.049388in}{1.011842in}}%
\pgfpathlineto{\pgfqpoint{2.051739in}{1.009029in}}%
\pgfpathlineto{\pgfqpoint{2.052523in}{1.008091in}}%
\pgfpathlineto{\pgfqpoint{2.054874in}{1.005277in}}%
\pgfpathlineto{\pgfqpoint{2.055658in}{1.004339in}}%
\pgfpathlineto{\pgfqpoint{2.058009in}{1.001526in}}%
\pgfpathlineto{\pgfqpoint{2.061144in}{1.001526in}}%
\pgfpathlineto{\pgfqpoint{2.061927in}{1.000588in}}%
\pgfpathlineto{\pgfqpoint{2.064278in}{0.997774in}}%
\pgfpathlineto{\pgfqpoint{2.065062in}{0.996836in}}%
\pgfpathlineto{\pgfqpoint{2.067413in}{0.994023in}}%
\pgfpathlineto{\pgfqpoint{2.068197in}{0.993085in}}%
\pgfpathlineto{\pgfqpoint{2.070548in}{0.990271in}}%
\pgfpathlineto{\pgfqpoint{2.073683in}{0.990271in}}%
\pgfpathlineto{\pgfqpoint{2.074466in}{0.989333in}}%
\pgfpathlineto{\pgfqpoint{2.076817in}{0.986520in}}%
\pgfpathlineto{\pgfqpoint{2.077601in}{0.985582in}}%
\pgfpathlineto{\pgfqpoint{2.079952in}{0.982768in}}%
\pgfpathlineto{\pgfqpoint{2.080736in}{0.981830in}}%
\pgfpathlineto{\pgfqpoint{2.083087in}{0.979017in}}%
\pgfpathlineto{\pgfqpoint{2.086222in}{0.979017in}}%
\pgfpathlineto{\pgfqpoint{2.087005in}{0.978079in}}%
\pgfpathlineto{\pgfqpoint{2.089356in}{0.975265in}}%
\pgfpathlineto{\pgfqpoint{2.090140in}{0.974327in}}%
\pgfpathlineto{\pgfqpoint{2.092491in}{0.971513in}}%
\pgfpathlineto{\pgfqpoint{2.093275in}{0.970576in}}%
\pgfpathlineto{\pgfqpoint{2.095626in}{0.967762in}}%
\pgfpathlineto{\pgfqpoint{2.096409in}{0.966824in}}%
\pgfpathlineto{\pgfqpoint{2.098761in}{0.964010in}}%
\pgfpathlineto{\pgfqpoint{2.101895in}{0.964010in}}%
\pgfpathlineto{\pgfqpoint{2.102679in}{0.963073in}}%
\pgfpathlineto{\pgfqpoint{2.105030in}{0.960259in}}%
\pgfpathlineto{\pgfqpoint{2.105814in}{0.959321in}}%
\pgfpathlineto{\pgfqpoint{2.108165in}{0.956507in}}%
\pgfpathlineto{\pgfqpoint{2.108948in}{0.955569in}}%
\pgfpathlineto{\pgfqpoint{2.111299in}{0.952756in}}%
\pgfpathlineto{\pgfqpoint{2.114434in}{0.952756in}}%
\pgfpathlineto{\pgfqpoint{2.115218in}{0.951818in}}%
\pgfpathlineto{\pgfqpoint{2.117569in}{0.949004in}}%
\pgfpathlineto{\pgfqpoint{2.118353in}{0.948066in}}%
\pgfpathlineto{\pgfqpoint{2.120704in}{0.945253in}}%
\pgfpathlineto{\pgfqpoint{2.121487in}{0.944315in}}%
\pgfpathlineto{\pgfqpoint{2.123838in}{0.941501in}}%
\pgfpathlineto{\pgfqpoint{2.126973in}{0.941501in}}%
\pgfpathlineto{\pgfqpoint{2.127757in}{0.940563in}}%
\pgfpathlineto{\pgfqpoint{2.130108in}{0.937750in}}%
\pgfpathlineto{\pgfqpoint{2.130892in}{0.936812in}}%
\pgfpathlineto{\pgfqpoint{2.133243in}{0.933998in}}%
\pgfpathlineto{\pgfqpoint{2.134026in}{0.933060in}}%
\pgfpathlineto{\pgfqpoint{2.136377in}{0.930247in}}%
\pgfpathlineto{\pgfqpoint{2.139512in}{0.930247in}}%
\pgfpathlineto{\pgfqpoint{2.140296in}{0.929309in}}%
\pgfpathlineto{\pgfqpoint{2.142647in}{0.926495in}}%
\pgfpathlineto{\pgfqpoint{2.143431in}{0.925557in}}%
\pgfpathlineto{\pgfqpoint{2.145782in}{0.922744in}}%
\pgfpathlineto{\pgfqpoint{2.146565in}{0.921806in}}%
\pgfpathlineto{\pgfqpoint{2.148916in}{0.918992in}}%
\pgfpathlineto{\pgfqpoint{2.152051in}{0.918992in}}%
\pgfpathlineto{\pgfqpoint{2.152835in}{0.918054in}}%
\pgfpathlineto{\pgfqpoint{2.155186in}{0.915240in}}%
\pgfpathlineto{\pgfqpoint{2.155970in}{0.914303in}}%
\pgfpathlineto{\pgfqpoint{2.158321in}{0.911489in}}%
\pgfpathlineto{\pgfqpoint{2.159104in}{0.910551in}}%
\pgfpathlineto{\pgfqpoint{2.161455in}{0.907737in}}%
\pgfpathlineto{\pgfqpoint{2.162239in}{0.906799in}}%
\pgfpathlineto{\pgfqpoint{2.164590in}{0.903986in}}%
\pgfpathlineto{\pgfqpoint{2.167725in}{0.903986in}}%
\pgfpathlineto{\pgfqpoint{2.168509in}{0.903048in}}%
\pgfpathlineto{\pgfqpoint{2.170860in}{0.900234in}}%
\pgfpathlineto{\pgfqpoint{2.171643in}{0.899296in}}%
\pgfpathclose%
\pgfpathmoveto{\pgfqpoint{2.172113in}{0.899296in}}%
\pgfpathlineto{\pgfqpoint{2.170860in}{0.900797in}}%
\pgfpathlineto{\pgfqpoint{2.168979in}{0.903048in}}%
\pgfpathlineto{\pgfqpoint{2.167725in}{0.904549in}}%
\pgfpathlineto{\pgfqpoint{2.164590in}{0.904549in}}%
\pgfpathlineto{\pgfqpoint{2.162709in}{0.906799in}}%
\pgfpathlineto{\pgfqpoint{2.161455in}{0.908300in}}%
\pgfpathlineto{\pgfqpoint{2.159575in}{0.910551in}}%
\pgfpathlineto{\pgfqpoint{2.158321in}{0.912052in}}%
\pgfpathlineto{\pgfqpoint{2.156440in}{0.914303in}}%
\pgfpathlineto{\pgfqpoint{2.155186in}{0.915803in}}%
\pgfpathlineto{\pgfqpoint{2.153305in}{0.918054in}}%
\pgfpathlineto{\pgfqpoint{2.152051in}{0.919555in}}%
\pgfpathlineto{\pgfqpoint{2.148916in}{0.919555in}}%
\pgfpathlineto{\pgfqpoint{2.147036in}{0.921806in}}%
\pgfpathlineto{\pgfqpoint{2.145782in}{0.923306in}}%
\pgfpathlineto{\pgfqpoint{2.143901in}{0.925557in}}%
\pgfpathlineto{\pgfqpoint{2.142647in}{0.927058in}}%
\pgfpathlineto{\pgfqpoint{2.140766in}{0.929309in}}%
\pgfpathlineto{\pgfqpoint{2.139512in}{0.930809in}}%
\pgfpathlineto{\pgfqpoint{2.136377in}{0.930809in}}%
\pgfpathlineto{\pgfqpoint{2.134497in}{0.933060in}}%
\pgfpathlineto{\pgfqpoint{2.133243in}{0.934561in}}%
\pgfpathlineto{\pgfqpoint{2.131362in}{0.936812in}}%
\pgfpathlineto{\pgfqpoint{2.130108in}{0.938312in}}%
\pgfpathlineto{\pgfqpoint{2.128227in}{0.940563in}}%
\pgfpathlineto{\pgfqpoint{2.126973in}{0.942064in}}%
\pgfpathlineto{\pgfqpoint{2.123838in}{0.942064in}}%
\pgfpathlineto{\pgfqpoint{2.121958in}{0.944315in}}%
\pgfpathlineto{\pgfqpoint{2.120704in}{0.945815in}}%
\pgfpathlineto{\pgfqpoint{2.118823in}{0.948066in}}%
\pgfpathlineto{\pgfqpoint{2.117569in}{0.949567in}}%
\pgfpathlineto{\pgfqpoint{2.115688in}{0.951818in}}%
\pgfpathlineto{\pgfqpoint{2.114434in}{0.953319in}}%
\pgfpathlineto{\pgfqpoint{2.111299in}{0.953319in}}%
\pgfpathlineto{\pgfqpoint{2.109419in}{0.955569in}}%
\pgfpathlineto{\pgfqpoint{2.108165in}{0.957070in}}%
\pgfpathlineto{\pgfqpoint{2.106284in}{0.959321in}}%
\pgfpathlineto{\pgfqpoint{2.105030in}{0.960822in}}%
\pgfpathlineto{\pgfqpoint{2.103149in}{0.963073in}}%
\pgfpathlineto{\pgfqpoint{2.101895in}{0.964573in}}%
\pgfpathlineto{\pgfqpoint{2.098761in}{0.964573in}}%
\pgfpathlineto{\pgfqpoint{2.096880in}{0.966824in}}%
\pgfpathlineto{\pgfqpoint{2.095626in}{0.968325in}}%
\pgfpathlineto{\pgfqpoint{2.093745in}{0.970576in}}%
\pgfpathlineto{\pgfqpoint{2.092491in}{0.972076in}}%
\pgfpathlineto{\pgfqpoint{2.090610in}{0.974327in}}%
\pgfpathlineto{\pgfqpoint{2.089356in}{0.975828in}}%
\pgfpathlineto{\pgfqpoint{2.087475in}{0.978079in}}%
\pgfpathlineto{\pgfqpoint{2.086222in}{0.979579in}}%
\pgfpathlineto{\pgfqpoint{2.083087in}{0.979579in}}%
\pgfpathlineto{\pgfqpoint{2.081206in}{0.981830in}}%
\pgfpathlineto{\pgfqpoint{2.079952in}{0.983331in}}%
\pgfpathlineto{\pgfqpoint{2.078071in}{0.985582in}}%
\pgfpathlineto{\pgfqpoint{2.076817in}{0.987082in}}%
\pgfpathlineto{\pgfqpoint{2.074936in}{0.989333in}}%
\pgfpathlineto{\pgfqpoint{2.073683in}{0.990834in}}%
\pgfpathlineto{\pgfqpoint{2.070548in}{0.990834in}}%
\pgfpathlineto{\pgfqpoint{2.068667in}{0.993085in}}%
\pgfpathlineto{\pgfqpoint{2.067413in}{0.994585in}}%
\pgfpathlineto{\pgfqpoint{2.065532in}{0.996836in}}%
\pgfpathlineto{\pgfqpoint{2.064278in}{0.998337in}}%
\pgfpathlineto{\pgfqpoint{2.062398in}{1.000588in}}%
\pgfpathlineto{\pgfqpoint{2.061144in}{1.002088in}}%
\pgfpathlineto{\pgfqpoint{2.058009in}{1.002088in}}%
\pgfpathlineto{\pgfqpoint{2.056128in}{1.004339in}}%
\pgfpathlineto{\pgfqpoint{2.054874in}{1.005840in}}%
\pgfpathlineto{\pgfqpoint{2.052993in}{1.008091in}}%
\pgfpathlineto{\pgfqpoint{2.051739in}{1.009592in}}%
\pgfpathlineto{\pgfqpoint{2.049859in}{1.011842in}}%
\pgfpathlineto{\pgfqpoint{2.048605in}{1.013343in}}%
\pgfpathlineto{\pgfqpoint{2.045470in}{1.013343in}}%
\pgfpathlineto{\pgfqpoint{2.043589in}{1.015594in}}%
\pgfpathlineto{\pgfqpoint{2.042335in}{1.017095in}}%
\pgfpathlineto{\pgfqpoint{2.040454in}{1.019346in}}%
\pgfpathlineto{\pgfqpoint{2.039200in}{1.020846in}}%
\pgfpathlineto{\pgfqpoint{2.037320in}{1.023097in}}%
\pgfpathlineto{\pgfqpoint{2.036066in}{1.024598in}}%
\pgfpathlineto{\pgfqpoint{2.032931in}{1.024598in}}%
\pgfpathlineto{\pgfqpoint{2.031050in}{1.026849in}}%
\pgfpathlineto{\pgfqpoint{2.029796in}{1.028349in}}%
\pgfpathlineto{\pgfqpoint{2.027915in}{1.030600in}}%
\pgfpathlineto{\pgfqpoint{2.026661in}{1.032101in}}%
\pgfpathlineto{\pgfqpoint{2.024781in}{1.034352in}}%
\pgfpathlineto{\pgfqpoint{2.023527in}{1.035852in}}%
\pgfpathlineto{\pgfqpoint{2.020392in}{1.035852in}}%
\pgfpathlineto{\pgfqpoint{2.018511in}{1.038103in}}%
\pgfpathlineto{\pgfqpoint{2.017257in}{1.039604in}}%
\pgfpathlineto{\pgfqpoint{2.015376in}{1.041855in}}%
\pgfpathlineto{\pgfqpoint{2.014122in}{1.043355in}}%
\pgfpathlineto{\pgfqpoint{2.012242in}{1.045606in}}%
\pgfpathlineto{\pgfqpoint{2.010988in}{1.047107in}}%
\pgfpathlineto{\pgfqpoint{2.009107in}{1.049358in}}%
\pgfpathlineto{\pgfqpoint{2.007853in}{1.050858in}}%
\pgfpathlineto{\pgfqpoint{2.004718in}{1.050858in}}%
\pgfpathlineto{\pgfqpoint{2.002837in}{1.053109in}}%
\pgfpathlineto{\pgfqpoint{2.001584in}{1.054610in}}%
\pgfpathlineto{\pgfqpoint{1.999703in}{1.056861in}}%
\pgfpathlineto{\pgfqpoint{1.998449in}{1.058361in}}%
\pgfpathlineto{\pgfqpoint{1.996568in}{1.060612in}}%
\pgfpathlineto{\pgfqpoint{1.995314in}{1.062113in}}%
\pgfpathlineto{\pgfqpoint{1.992179in}{1.062113in}}%
\pgfpathlineto{\pgfqpoint{1.990298in}{1.064364in}}%
\pgfpathlineto{\pgfqpoint{1.989045in}{1.065865in}}%
\pgfpathlineto{\pgfqpoint{1.987164in}{1.068115in}}%
\pgfpathlineto{\pgfqpoint{1.985910in}{1.069616in}}%
\pgfpathlineto{\pgfqpoint{1.984029in}{1.071867in}}%
\pgfpathlineto{\pgfqpoint{1.982775in}{1.073368in}}%
\pgfpathlineto{\pgfqpoint{1.979640in}{1.073368in}}%
\pgfpathlineto{\pgfqpoint{1.977759in}{1.075619in}}%
\pgfpathlineto{\pgfqpoint{1.976506in}{1.077119in}}%
\pgfpathlineto{\pgfqpoint{1.974625in}{1.079370in}}%
\pgfpathlineto{\pgfqpoint{1.973371in}{1.080871in}}%
\pgfpathlineto{\pgfqpoint{1.971490in}{1.083122in}}%
\pgfpathlineto{\pgfqpoint{1.970236in}{1.084622in}}%
\pgfpathlineto{\pgfqpoint{1.967101in}{1.084622in}}%
\pgfpathlineto{\pgfqpoint{1.965221in}{1.086873in}}%
\pgfpathlineto{\pgfqpoint{1.963967in}{1.088374in}}%
\pgfpathlineto{\pgfqpoint{1.962086in}{1.090625in}}%
\pgfpathlineto{\pgfqpoint{1.960832in}{1.092125in}}%
\pgfpathlineto{\pgfqpoint{1.958951in}{1.094376in}}%
\pgfpathlineto{\pgfqpoint{1.957697in}{1.095877in}}%
\pgfpathlineto{\pgfqpoint{1.954562in}{1.095877in}}%
\pgfpathlineto{\pgfqpoint{1.952682in}{1.098128in}}%
\pgfpathlineto{\pgfqpoint{1.951428in}{1.099628in}}%
\pgfpathlineto{\pgfqpoint{1.949547in}{1.101879in}}%
\pgfpathlineto{\pgfqpoint{1.948293in}{1.103380in}}%
\pgfpathlineto{\pgfqpoint{1.946412in}{1.105631in}}%
\pgfpathlineto{\pgfqpoint{1.945158in}{1.107131in}}%
\pgfpathlineto{\pgfqpoint{1.942023in}{1.107131in}}%
\pgfpathlineto{\pgfqpoint{1.940143in}{1.109382in}}%
\pgfpathlineto{\pgfqpoint{1.938889in}{1.110883in}}%
\pgfpathlineto{\pgfqpoint{1.937008in}{1.113134in}}%
\pgfpathlineto{\pgfqpoint{1.935754in}{1.114635in}}%
\pgfpathlineto{\pgfqpoint{1.933873in}{1.116885in}}%
\pgfpathlineto{\pgfqpoint{1.932619in}{1.118386in}}%
\pgfpathlineto{\pgfqpoint{1.930738in}{1.120637in}}%
\pgfpathlineto{\pgfqpoint{1.929484in}{1.122138in}}%
\pgfpathlineto{\pgfqpoint{1.926350in}{1.122138in}}%
\pgfpathlineto{\pgfqpoint{1.924469in}{1.124388in}}%
\pgfpathlineto{\pgfqpoint{1.923215in}{1.125889in}}%
\pgfpathlineto{\pgfqpoint{1.921334in}{1.128140in}}%
\pgfpathlineto{\pgfqpoint{1.920080in}{1.129641in}}%
\pgfpathlineto{\pgfqpoint{1.918199in}{1.131892in}}%
\pgfpathlineto{\pgfqpoint{1.916945in}{1.133392in}}%
\pgfpathlineto{\pgfqpoint{1.913811in}{1.133392in}}%
\pgfpathlineto{\pgfqpoint{1.911930in}{1.135643in}}%
\pgfpathlineto{\pgfqpoint{1.910676in}{1.137144in}}%
\pgfpathlineto{\pgfqpoint{1.908795in}{1.139395in}}%
\pgfpathlineto{\pgfqpoint{1.907541in}{1.140895in}}%
\pgfpathlineto{\pgfqpoint{1.905660in}{1.143146in}}%
\pgfpathlineto{\pgfqpoint{1.904407in}{1.144647in}}%
\pgfpathlineto{\pgfqpoint{1.901272in}{1.144647in}}%
\pgfpathlineto{\pgfqpoint{1.899391in}{1.146898in}}%
\pgfpathlineto{\pgfqpoint{1.898137in}{1.148398in}}%
\pgfpathlineto{\pgfqpoint{1.896256in}{1.150649in}}%
\pgfpathlineto{\pgfqpoint{1.895002in}{1.152150in}}%
\pgfpathlineto{\pgfqpoint{1.893121in}{1.154401in}}%
\pgfpathlineto{\pgfqpoint{1.891868in}{1.155901in}}%
\pgfpathlineto{\pgfqpoint{1.888733in}{1.155901in}}%
\pgfpathlineto{\pgfqpoint{1.886852in}{1.158152in}}%
\pgfpathlineto{\pgfqpoint{1.885598in}{1.159653in}}%
\pgfpathlineto{\pgfqpoint{1.883717in}{1.161904in}}%
\pgfpathlineto{\pgfqpoint{1.882463in}{1.163404in}}%
\pgfpathlineto{\pgfqpoint{1.880582in}{1.165655in}}%
\pgfpathlineto{\pgfqpoint{1.879329in}{1.167156in}}%
\pgfpathlineto{\pgfqpoint{1.876194in}{1.167156in}}%
\pgfpathlineto{\pgfqpoint{1.874313in}{1.169407in}}%
\pgfpathlineto{\pgfqpoint{1.873059in}{1.170908in}}%
\pgfpathlineto{\pgfqpoint{1.871178in}{1.173158in}}%
\pgfpathlineto{\pgfqpoint{1.869924in}{1.174659in}}%
\pgfpathlineto{\pgfqpoint{1.868043in}{1.176910in}}%
\pgfpathlineto{\pgfqpoint{1.866790in}{1.178411in}}%
\pgfpathlineto{\pgfqpoint{1.863655in}{1.178411in}}%
\pgfpathlineto{\pgfqpoint{1.861774in}{1.180662in}}%
\pgfpathlineto{\pgfqpoint{1.860520in}{1.182162in}}%
\pgfpathlineto{\pgfqpoint{1.858639in}{1.184413in}}%
\pgfpathlineto{\pgfqpoint{1.857385in}{1.185914in}}%
\pgfpathlineto{\pgfqpoint{1.855505in}{1.188165in}}%
\pgfpathlineto{\pgfqpoint{1.854251in}{1.189665in}}%
\pgfpathlineto{\pgfqpoint{1.852370in}{1.191916in}}%
\pgfpathlineto{\pgfqpoint{1.851116in}{1.193417in}}%
\pgfpathlineto{\pgfqpoint{1.847981in}{1.193417in}}%
\pgfpathlineto{\pgfqpoint{1.846100in}{1.195668in}}%
\pgfpathlineto{\pgfqpoint{1.844846in}{1.197168in}}%
\pgfpathlineto{\pgfqpoint{1.842966in}{1.199419in}}%
\pgfpathlineto{\pgfqpoint{1.841712in}{1.200920in}}%
\pgfpathlineto{\pgfqpoint{1.839831in}{1.203171in}}%
\pgfpathlineto{\pgfqpoint{1.838577in}{1.204671in}}%
\pgfpathlineto{\pgfqpoint{1.835442in}{1.204671in}}%
\pgfpathlineto{\pgfqpoint{1.833561in}{1.206922in}}%
\pgfpathlineto{\pgfqpoint{1.832307in}{1.208423in}}%
\pgfpathlineto{\pgfqpoint{1.830427in}{1.210674in}}%
\pgfpathlineto{\pgfqpoint{1.829173in}{1.212174in}}%
\pgfpathlineto{\pgfqpoint{1.827292in}{1.214425in}}%
\pgfpathlineto{\pgfqpoint{1.826038in}{1.215926in}}%
\pgfpathlineto{\pgfqpoint{1.822903in}{1.215926in}}%
\pgfpathlineto{\pgfqpoint{1.821022in}{1.218177in}}%
\pgfpathlineto{\pgfqpoint{1.819768in}{1.219677in}}%
\pgfpathlineto{\pgfqpoint{1.817888in}{1.221928in}}%
\pgfpathlineto{\pgfqpoint{1.816634in}{1.223429in}}%
\pgfpathlineto{\pgfqpoint{1.814753in}{1.225680in}}%
\pgfpathlineto{\pgfqpoint{1.813499in}{1.227181in}}%
\pgfpathlineto{\pgfqpoint{1.810364in}{1.227181in}}%
\pgfpathlineto{\pgfqpoint{1.808483in}{1.229431in}}%
\pgfpathlineto{\pgfqpoint{1.807229in}{1.230932in}}%
\pgfpathlineto{\pgfqpoint{1.805349in}{1.233183in}}%
\pgfpathlineto{\pgfqpoint{1.804095in}{1.234684in}}%
\pgfpathlineto{\pgfqpoint{1.802214in}{1.236935in}}%
\pgfpathlineto{\pgfqpoint{1.800960in}{1.238435in}}%
\pgfpathlineto{\pgfqpoint{1.797825in}{1.238435in}}%
\pgfpathlineto{\pgfqpoint{1.795944in}{1.240686in}}%
\pgfpathlineto{\pgfqpoint{1.794691in}{1.242187in}}%
\pgfpathlineto{\pgfqpoint{1.792810in}{1.244438in}}%
\pgfpathlineto{\pgfqpoint{1.791556in}{1.245938in}}%
\pgfpathlineto{\pgfqpoint{1.789675in}{1.248189in}}%
\pgfpathlineto{\pgfqpoint{1.788421in}{1.249690in}}%
\pgfpathlineto{\pgfqpoint{1.786540in}{1.251941in}}%
\pgfpathlineto{\pgfqpoint{1.785286in}{1.253441in}}%
\pgfpathlineto{\pgfqpoint{1.782152in}{1.253441in}}%
\pgfpathlineto{\pgfqpoint{1.780271in}{1.255692in}}%
\pgfpathlineto{\pgfqpoint{1.779017in}{1.257193in}}%
\pgfpathlineto{\pgfqpoint{1.777136in}{1.259444in}}%
\pgfpathlineto{\pgfqpoint{1.775882in}{1.260944in}}%
\pgfpathlineto{\pgfqpoint{1.774001in}{1.263195in}}%
\pgfpathlineto{\pgfqpoint{1.772747in}{1.264696in}}%
\pgfpathlineto{\pgfqpoint{1.769613in}{1.264696in}}%
\pgfpathlineto{\pgfqpoint{1.767732in}{1.266947in}}%
\pgfpathlineto{\pgfqpoint{1.766478in}{1.268447in}}%
\pgfpathlineto{\pgfqpoint{1.764597in}{1.270698in}}%
\pgfpathlineto{\pgfqpoint{1.763343in}{1.272199in}}%
\pgfpathlineto{\pgfqpoint{1.761462in}{1.274450in}}%
\pgfpathlineto{\pgfqpoint{1.760208in}{1.275950in}}%
\pgfpathlineto{\pgfqpoint{1.757074in}{1.275950in}}%
\pgfpathlineto{\pgfqpoint{1.755193in}{1.278201in}}%
\pgfpathlineto{\pgfqpoint{1.753939in}{1.279702in}}%
\pgfpathlineto{\pgfqpoint{1.752058in}{1.281953in}}%
\pgfpathlineto{\pgfqpoint{1.750804in}{1.283454in}}%
\pgfpathlineto{\pgfqpoint{1.748923in}{1.285704in}}%
\pgfpathlineto{\pgfqpoint{1.747669in}{1.287205in}}%
\pgfpathlineto{\pgfqpoint{1.744535in}{1.287205in}}%
\pgfpathlineto{\pgfqpoint{1.742654in}{1.289456in}}%
\pgfpathlineto{\pgfqpoint{1.741400in}{1.290957in}}%
\pgfpathlineto{\pgfqpoint{1.739519in}{1.293208in}}%
\pgfpathlineto{\pgfqpoint{1.738265in}{1.294708in}}%
\pgfpathlineto{\pgfqpoint{1.736384in}{1.296959in}}%
\pgfpathlineto{\pgfqpoint{1.735130in}{1.298460in}}%
\pgfpathlineto{\pgfqpoint{1.731996in}{1.298460in}}%
\pgfpathlineto{\pgfqpoint{1.730115in}{1.300711in}}%
\pgfpathlineto{\pgfqpoint{1.728861in}{1.302211in}}%
\pgfpathlineto{\pgfqpoint{1.726980in}{1.304462in}}%
\pgfpathlineto{\pgfqpoint{1.725726in}{1.305963in}}%
\pgfpathlineto{\pgfqpoint{1.723845in}{1.308214in}}%
\pgfpathlineto{\pgfqpoint{1.722591in}{1.309714in}}%
\pgfpathlineto{\pgfqpoint{1.719457in}{1.309714in}}%
\pgfpathlineto{\pgfqpoint{1.717576in}{1.311965in}}%
\pgfpathlineto{\pgfqpoint{1.716322in}{1.313466in}}%
\pgfpathlineto{\pgfqpoint{1.714441in}{1.315717in}}%
\pgfpathlineto{\pgfqpoint{1.713187in}{1.317217in}}%
\pgfpathlineto{\pgfqpoint{1.711306in}{1.319468in}}%
\pgfpathlineto{\pgfqpoint{1.710052in}{1.320969in}}%
\pgfpathlineto{\pgfqpoint{1.708172in}{1.323220in}}%
\pgfpathlineto{\pgfqpoint{1.706918in}{1.324720in}}%
\pgfpathlineto{\pgfqpoint{1.703783in}{1.324720in}}%
\pgfpathlineto{\pgfqpoint{1.701902in}{1.326971in}}%
\pgfpathlineto{\pgfqpoint{1.700648in}{1.328472in}}%
\pgfpathlineto{\pgfqpoint{1.698767in}{1.330723in}}%
\pgfpathlineto{\pgfqpoint{1.697514in}{1.332224in}}%
\pgfpathlineto{\pgfqpoint{1.695633in}{1.334474in}}%
\pgfpathlineto{\pgfqpoint{1.694379in}{1.335975in}}%
\pgfpathlineto{\pgfqpoint{1.691244in}{1.335975in}}%
\pgfpathlineto{\pgfqpoint{1.689363in}{1.338226in}}%
\pgfpathlineto{\pgfqpoint{1.688109in}{1.339727in}}%
\pgfpathlineto{\pgfqpoint{1.686228in}{1.341977in}}%
\pgfpathlineto{\pgfqpoint{1.684975in}{1.343478in}}%
\pgfpathlineto{\pgfqpoint{1.683094in}{1.345729in}}%
\pgfpathlineto{\pgfqpoint{1.681840in}{1.347230in}}%
\pgfpathlineto{\pgfqpoint{1.678705in}{1.347230in}}%
\pgfpathlineto{\pgfqpoint{1.676824in}{1.349481in}}%
\pgfpathlineto{\pgfqpoint{1.675570in}{1.350981in}}%
\pgfpathlineto{\pgfqpoint{1.673689in}{1.353232in}}%
\pgfpathlineto{\pgfqpoint{1.672436in}{1.354733in}}%
\pgfpathlineto{\pgfqpoint{1.670555in}{1.356984in}}%
\pgfpathlineto{\pgfqpoint{1.669301in}{1.358484in}}%
\pgfpathlineto{\pgfqpoint{1.666166in}{1.358484in}}%
\pgfpathlineto{\pgfqpoint{1.664285in}{1.360735in}}%
\pgfpathlineto{\pgfqpoint{1.663031in}{1.362236in}}%
\pgfpathlineto{\pgfqpoint{1.661151in}{1.364487in}}%
\pgfpathlineto{\pgfqpoint{1.659897in}{1.365987in}}%
\pgfpathlineto{\pgfqpoint{1.658016in}{1.368238in}}%
\pgfpathlineto{\pgfqpoint{1.656762in}{1.369739in}}%
\pgfpathlineto{\pgfqpoint{1.653627in}{1.369739in}}%
\pgfpathlineto{\pgfqpoint{1.651746in}{1.371990in}}%
\pgfpathlineto{\pgfqpoint{1.650492in}{1.373490in}}%
\pgfpathlineto{\pgfqpoint{1.648612in}{1.375741in}}%
\pgfpathlineto{\pgfqpoint{1.647358in}{1.377242in}}%
\pgfpathlineto{\pgfqpoint{1.645477in}{1.379493in}}%
\pgfpathlineto{\pgfqpoint{1.644223in}{1.380993in}}%
\pgfpathlineto{\pgfqpoint{1.641088in}{1.380993in}}%
\pgfpathlineto{\pgfqpoint{1.639207in}{1.383244in}}%
\pgfpathlineto{\pgfqpoint{1.637953in}{1.384745in}}%
\pgfpathlineto{\pgfqpoint{1.636073in}{1.386996in}}%
\pgfpathlineto{\pgfqpoint{1.634819in}{1.388497in}}%
\pgfpathlineto{\pgfqpoint{1.632938in}{1.390747in}}%
\pgfpathlineto{\pgfqpoint{1.631684in}{1.392248in}}%
\pgfpathlineto{\pgfqpoint{1.629803in}{1.394499in}}%
\pgfpathlineto{\pgfqpoint{1.628549in}{1.396000in}}%
\pgfpathlineto{\pgfqpoint{1.625414in}{1.396000in}}%
\pgfpathlineto{\pgfqpoint{1.623534in}{1.398251in}}%
\pgfpathlineto{\pgfqpoint{1.622280in}{1.399751in}}%
\pgfpathlineto{\pgfqpoint{1.620399in}{1.402002in}}%
\pgfpathlineto{\pgfqpoint{1.619145in}{1.403503in}}%
\pgfpathlineto{\pgfqpoint{1.617264in}{1.405754in}}%
\pgfpathlineto{\pgfqpoint{1.616010in}{1.407254in}}%
\pgfpathlineto{\pgfqpoint{1.612875in}{1.407254in}}%
\pgfpathlineto{\pgfqpoint{1.610995in}{1.409505in}}%
\pgfpathlineto{\pgfqpoint{1.609741in}{1.411006in}}%
\pgfpathlineto{\pgfqpoint{1.607860in}{1.413257in}}%
\pgfpathlineto{\pgfqpoint{1.606606in}{1.414757in}}%
\pgfpathlineto{\pgfqpoint{1.604725in}{1.417008in}}%
\pgfpathlineto{\pgfqpoint{1.603471in}{1.418509in}}%
\pgfpathlineto{\pgfqpoint{1.600337in}{1.418509in}}%
\pgfpathlineto{\pgfqpoint{1.598456in}{1.420760in}}%
\pgfpathlineto{\pgfqpoint{1.597202in}{1.422260in}}%
\pgfpathlineto{\pgfqpoint{1.595321in}{1.424511in}}%
\pgfpathlineto{\pgfqpoint{1.594067in}{1.426012in}}%
\pgfpathlineto{\pgfqpoint{1.592186in}{1.428263in}}%
\pgfpathlineto{\pgfqpoint{1.590932in}{1.429763in}}%
\pgfpathlineto{\pgfqpoint{1.587798in}{1.429763in}}%
\pgfpathlineto{\pgfqpoint{1.585917in}{1.432014in}}%
\pgfpathlineto{\pgfqpoint{1.584663in}{1.433515in}}%
\pgfpathlineto{\pgfqpoint{1.582782in}{1.435766in}}%
\pgfpathlineto{\pgfqpoint{1.581528in}{1.437266in}}%
\pgfpathlineto{\pgfqpoint{1.579647in}{1.439517in}}%
\pgfpathlineto{\pgfqpoint{1.578393in}{1.441018in}}%
\pgfpathlineto{\pgfqpoint{1.575259in}{1.441018in}}%
\pgfpathlineto{\pgfqpoint{1.573378in}{1.443269in}}%
\pgfpathlineto{\pgfqpoint{1.572124in}{1.444770in}}%
\pgfpathlineto{\pgfqpoint{1.570243in}{1.447020in}}%
\pgfpathlineto{\pgfqpoint{1.568989in}{1.448521in}}%
\pgfpathlineto{\pgfqpoint{1.567108in}{1.450772in}}%
\pgfpathlineto{\pgfqpoint{1.565854in}{1.452273in}}%
\pgfpathlineto{\pgfqpoint{1.562720in}{1.452273in}}%
\pgfpathlineto{\pgfqpoint{1.560839in}{1.454524in}}%
\pgfpathlineto{\pgfqpoint{1.559585in}{1.456024in}}%
\pgfpathlineto{\pgfqpoint{1.557704in}{1.458275in}}%
\pgfpathlineto{\pgfqpoint{1.556450in}{1.459776in}}%
\pgfpathlineto{\pgfqpoint{1.554569in}{1.462027in}}%
\pgfpathlineto{\pgfqpoint{1.553315in}{1.463527in}}%
\pgfpathlineto{\pgfqpoint{1.551435in}{1.465778in}}%
\pgfpathlineto{\pgfqpoint{1.550181in}{1.467279in}}%
\pgfpathlineto{\pgfqpoint{1.547046in}{1.467279in}}%
\pgfpathlineto{\pgfqpoint{1.545165in}{1.469530in}}%
\pgfpathlineto{\pgfqpoint{1.543911in}{1.471030in}}%
\pgfpathlineto{\pgfqpoint{1.542030in}{1.473281in}}%
\pgfpathlineto{\pgfqpoint{1.540776in}{1.474782in}}%
\pgfpathlineto{\pgfqpoint{1.538896in}{1.477033in}}%
\pgfpathlineto{\pgfqpoint{1.537642in}{1.478533in}}%
\pgfpathlineto{\pgfqpoint{1.534507in}{1.478533in}}%
\pgfpathlineto{\pgfqpoint{1.532626in}{1.480784in}}%
\pgfpathlineto{\pgfqpoint{1.531372in}{1.482285in}}%
\pgfpathlineto{\pgfqpoint{1.529491in}{1.484536in}}%
\pgfpathlineto{\pgfqpoint{1.528237in}{1.486036in}}%
\pgfpathlineto{\pgfqpoint{1.526357in}{1.488287in}}%
\pgfpathlineto{\pgfqpoint{1.525103in}{1.489788in}}%
\pgfpathlineto{\pgfqpoint{1.521968in}{1.489788in}}%
\pgfpathlineto{\pgfqpoint{1.520087in}{1.492039in}}%
\pgfpathlineto{\pgfqpoint{1.518833in}{1.493539in}}%
\pgfpathlineto{\pgfqpoint{1.516952in}{1.495790in}}%
\pgfpathlineto{\pgfqpoint{1.515698in}{1.497291in}}%
\pgfpathlineto{\pgfqpoint{1.513818in}{1.499542in}}%
\pgfpathlineto{\pgfqpoint{1.512564in}{1.501043in}}%
\pgfpathlineto{\pgfqpoint{1.509429in}{1.501043in}}%
\pgfpathlineto{\pgfqpoint{1.507548in}{1.503293in}}%
\pgfpathlineto{\pgfqpoint{1.506294in}{1.504794in}}%
\pgfpathlineto{\pgfqpoint{1.504413in}{1.507045in}}%
\pgfpathlineto{\pgfqpoint{1.503159in}{1.508546in}}%
\pgfpathlineto{\pgfqpoint{1.501279in}{1.510797in}}%
\pgfpathlineto{\pgfqpoint{1.500025in}{1.512297in}}%
\pgfpathlineto{\pgfqpoint{1.496890in}{1.512297in}}%
\pgfpathlineto{\pgfqpoint{1.495009in}{1.514548in}}%
\pgfpathlineto{\pgfqpoint{1.493755in}{1.516049in}}%
\pgfpathlineto{\pgfqpoint{1.491874in}{1.518300in}}%
\pgfpathlineto{\pgfqpoint{1.490621in}{1.519800in}}%
\pgfpathlineto{\pgfqpoint{1.488740in}{1.522051in}}%
\pgfpathlineto{\pgfqpoint{1.487486in}{1.523552in}}%
\pgfpathlineto{\pgfqpoint{1.485605in}{1.525803in}}%
\pgfpathlineto{\pgfqpoint{1.484351in}{1.527303in}}%
\pgfpathlineto{\pgfqpoint{1.481216in}{1.527303in}}%
\pgfpathlineto{\pgfqpoint{1.479335in}{1.529554in}}%
\pgfpathlineto{\pgfqpoint{1.478082in}{1.531055in}}%
\pgfpathlineto{\pgfqpoint{1.476201in}{1.533306in}}%
\pgfpathlineto{\pgfqpoint{1.474947in}{1.534806in}}%
\pgfpathlineto{\pgfqpoint{1.473066in}{1.537057in}}%
\pgfpathlineto{\pgfqpoint{1.471812in}{1.538558in}}%
\pgfpathlineto{\pgfqpoint{1.468677in}{1.538558in}}%
\pgfpathlineto{\pgfqpoint{1.466796in}{1.540809in}}%
\pgfpathlineto{\pgfqpoint{1.465543in}{1.542309in}}%
\pgfpathlineto{\pgfqpoint{1.463662in}{1.544560in}}%
\pgfpathlineto{\pgfqpoint{1.462408in}{1.546061in}}%
\pgfpathlineto{\pgfqpoint{1.460527in}{1.548312in}}%
\pgfpathlineto{\pgfqpoint{1.459273in}{1.549813in}}%
\pgfpathlineto{\pgfqpoint{1.456138in}{1.549813in}}%
\pgfpathlineto{\pgfqpoint{1.454258in}{1.552063in}}%
\pgfpathlineto{\pgfqpoint{1.453004in}{1.553564in}}%
\pgfpathlineto{\pgfqpoint{1.451123in}{1.555815in}}%
\pgfpathlineto{\pgfqpoint{1.449869in}{1.557316in}}%
\pgfpathlineto{\pgfqpoint{1.447988in}{1.559566in}}%
\pgfpathlineto{\pgfqpoint{1.446734in}{1.561067in}}%
\pgfpathlineto{\pgfqpoint{1.443599in}{1.561067in}}%
\pgfpathlineto{\pgfqpoint{1.441719in}{1.563318in}}%
\pgfpathlineto{\pgfqpoint{1.440465in}{1.564819in}}%
\pgfpathlineto{\pgfqpoint{1.438584in}{1.567070in}}%
\pgfpathlineto{\pgfqpoint{1.437330in}{1.568570in}}%
\pgfpathlineto{\pgfqpoint{1.435449in}{1.570821in}}%
\pgfpathlineto{\pgfqpoint{1.434195in}{1.572322in}}%
\pgfpathlineto{\pgfqpoint{1.431060in}{1.572322in}}%
\pgfpathlineto{\pgfqpoint{1.429180in}{1.574573in}}%
\pgfpathlineto{\pgfqpoint{1.427926in}{1.576073in}}%
\pgfpathlineto{\pgfqpoint{1.426045in}{1.578324in}}%
\pgfpathlineto{\pgfqpoint{1.424791in}{1.579825in}}%
\pgfpathlineto{\pgfqpoint{1.422910in}{1.582076in}}%
\pgfpathlineto{\pgfqpoint{1.421656in}{1.583576in}}%
\pgfpathlineto{\pgfqpoint{1.418521in}{1.583576in}}%
\pgfpathlineto{\pgfqpoint{1.416641in}{1.585827in}}%
\pgfpathlineto{\pgfqpoint{1.415387in}{1.587328in}}%
\pgfpathlineto{\pgfqpoint{1.413506in}{1.589579in}}%
\pgfpathlineto{\pgfqpoint{1.412252in}{1.591079in}}%
\pgfpathlineto{\pgfqpoint{1.410371in}{1.593330in}}%
\pgfpathlineto{\pgfqpoint{1.409117in}{1.594831in}}%
\pgfpathlineto{\pgfqpoint{1.407236in}{1.597082in}}%
\pgfpathlineto{\pgfqpoint{1.405982in}{1.598582in}}%
\pgfpathlineto{\pgfqpoint{1.402848in}{1.598582in}}%
\pgfpathlineto{\pgfqpoint{1.400967in}{1.600833in}}%
\pgfpathlineto{\pgfqpoint{1.399713in}{1.602334in}}%
\pgfpathlineto{\pgfqpoint{1.397832in}{1.604585in}}%
\pgfpathlineto{\pgfqpoint{1.396578in}{1.606086in}}%
\pgfpathlineto{\pgfqpoint{1.394697in}{1.608336in}}%
\pgfpathlineto{\pgfqpoint{1.393444in}{1.609837in}}%
\pgfpathlineto{\pgfqpoint{1.390309in}{1.609837in}}%
\pgfpathlineto{\pgfqpoint{1.388428in}{1.612088in}}%
\pgfpathlineto{\pgfqpoint{1.387174in}{1.613589in}}%
\pgfpathlineto{\pgfqpoint{1.385293in}{1.615840in}}%
\pgfpathlineto{\pgfqpoint{1.384039in}{1.617340in}}%
\pgfpathlineto{\pgfqpoint{1.382158in}{1.619591in}}%
\pgfpathlineto{\pgfqpoint{1.380905in}{1.621092in}}%
\pgfpathlineto{\pgfqpoint{1.377770in}{1.621092in}}%
\pgfpathlineto{\pgfqpoint{1.375889in}{1.623343in}}%
\pgfpathlineto{\pgfqpoint{1.374635in}{1.624843in}}%
\pgfpathlineto{\pgfqpoint{1.372754in}{1.627094in}}%
\pgfpathlineto{\pgfqpoint{1.371500in}{1.628595in}}%
\pgfpathlineto{\pgfqpoint{1.369619in}{1.630846in}}%
\pgfpathlineto{\pgfqpoint{1.368366in}{1.632346in}}%
\pgfpathlineto{\pgfqpoint{1.365231in}{1.632346in}}%
\pgfpathlineto{\pgfqpoint{1.363350in}{1.634597in}}%
\pgfpathlineto{\pgfqpoint{1.362096in}{1.636098in}}%
\pgfpathlineto{\pgfqpoint{1.360215in}{1.638349in}}%
\pgfpathlineto{\pgfqpoint{1.358961in}{1.639849in}}%
\pgfpathlineto{\pgfqpoint{1.357080in}{1.642100in}}%
\pgfpathlineto{\pgfqpoint{1.355827in}{1.643601in}}%
\pgfpathlineto{\pgfqpoint{1.352692in}{1.643601in}}%
\pgfpathlineto{\pgfqpoint{1.350811in}{1.645852in}}%
\pgfpathlineto{\pgfqpoint{1.349557in}{1.647352in}}%
\pgfpathlineto{\pgfqpoint{1.347676in}{1.649603in}}%
\pgfpathlineto{\pgfqpoint{1.346422in}{1.651104in}}%
\pgfpathlineto{\pgfqpoint{1.344542in}{1.653355in}}%
\pgfpathlineto{\pgfqpoint{1.343288in}{1.654855in}}%
\pgfpathlineto{\pgfqpoint{1.340153in}{1.654855in}}%
\pgfpathlineto{\pgfqpoint{1.338272in}{1.657106in}}%
\pgfpathlineto{\pgfqpoint{1.337018in}{1.658607in}}%
\pgfpathlineto{\pgfqpoint{1.335137in}{1.660858in}}%
\pgfpathlineto{\pgfqpoint{1.333883in}{1.662359in}}%
\pgfpathlineto{\pgfqpoint{1.332003in}{1.664609in}}%
\pgfpathlineto{\pgfqpoint{1.330749in}{1.666110in}}%
\pgfpathlineto{\pgfqpoint{1.328868in}{1.668361in}}%
\pgfpathlineto{\pgfqpoint{1.327614in}{1.669862in}}%
\pgfpathlineto{\pgfqpoint{1.324479in}{1.669862in}}%
\pgfpathlineto{\pgfqpoint{1.322598in}{1.672113in}}%
\pgfpathlineto{\pgfqpoint{1.321344in}{1.673613in}}%
\pgfpathlineto{\pgfqpoint{1.319464in}{1.675864in}}%
\pgfpathlineto{\pgfqpoint{1.318210in}{1.677365in}}%
\pgfpathlineto{\pgfqpoint{1.316329in}{1.679616in}}%
\pgfpathlineto{\pgfqpoint{1.315075in}{1.681116in}}%
\pgfpathlineto{\pgfqpoint{1.311940in}{1.681116in}}%
\pgfpathlineto{\pgfqpoint{1.310059in}{1.683367in}}%
\pgfpathlineto{\pgfqpoint{1.308805in}{1.684868in}}%
\pgfpathlineto{\pgfqpoint{1.306925in}{1.687119in}}%
\pgfpathlineto{\pgfqpoint{1.305671in}{1.688619in}}%
\pgfpathlineto{\pgfqpoint{1.303790in}{1.690870in}}%
\pgfpathlineto{\pgfqpoint{1.302536in}{1.692371in}}%
\pgfpathlineto{\pgfqpoint{1.299401in}{1.692371in}}%
\pgfpathlineto{\pgfqpoint{1.297520in}{1.694622in}}%
\pgfpathlineto{\pgfqpoint{1.296266in}{1.696122in}}%
\pgfpathlineto{\pgfqpoint{1.294386in}{1.698373in}}%
\pgfpathlineto{\pgfqpoint{1.293132in}{1.699874in}}%
\pgfpathlineto{\pgfqpoint{1.291251in}{1.702125in}}%
\pgfpathlineto{\pgfqpoint{1.291251in}{1.705876in}}%
\pgfpathlineto{\pgfqpoint{1.293132in}{1.708127in}}%
\pgfpathlineto{\pgfqpoint{1.294386in}{1.709628in}}%
\pgfpathlineto{\pgfqpoint{1.294386in}{1.713379in}}%
\pgfpathlineto{\pgfqpoint{1.294386in}{1.717131in}}%
\pgfpathlineto{\pgfqpoint{1.294386in}{1.720882in}}%
\pgfpathlineto{\pgfqpoint{1.296266in}{1.723133in}}%
\pgfpathlineto{\pgfqpoint{1.297520in}{1.724634in}}%
\pgfpathlineto{\pgfqpoint{1.297520in}{1.728386in}}%
\pgfpathlineto{\pgfqpoint{1.297520in}{1.732137in}}%
\pgfpathlineto{\pgfqpoint{1.299401in}{1.734388in}}%
\pgfpathlineto{\pgfqpoint{1.300655in}{1.735889in}}%
\pgfpathlineto{\pgfqpoint{1.300655in}{1.739640in}}%
\pgfpathlineto{\pgfqpoint{1.300655in}{1.743392in}}%
\pgfpathlineto{\pgfqpoint{1.300655in}{1.747143in}}%
\pgfpathlineto{\pgfqpoint{1.302536in}{1.749394in}}%
\pgfpathlineto{\pgfqpoint{1.303790in}{1.750895in}}%
\pgfpathlineto{\pgfqpoint{1.303790in}{1.754646in}}%
\pgfpathlineto{\pgfqpoint{1.303790in}{1.758398in}}%
\pgfpathlineto{\pgfqpoint{1.305671in}{1.760649in}}%
\pgfpathlineto{\pgfqpoint{1.306925in}{1.762149in}}%
\pgfpathlineto{\pgfqpoint{1.306925in}{1.765901in}}%
\pgfpathlineto{\pgfqpoint{1.306925in}{1.769652in}}%
\pgfpathlineto{\pgfqpoint{1.306925in}{1.773404in}}%
\pgfpathlineto{\pgfqpoint{1.308805in}{1.775655in}}%
\pgfpathlineto{\pgfqpoint{1.310059in}{1.777155in}}%
\pgfpathlineto{\pgfqpoint{1.310059in}{1.780907in}}%
\pgfpathlineto{\pgfqpoint{1.310059in}{1.784659in}}%
\pgfpathlineto{\pgfqpoint{1.311940in}{1.786909in}}%
\pgfpathlineto{\pgfqpoint{1.313194in}{1.788410in}}%
\pgfpathlineto{\pgfqpoint{1.313194in}{1.792162in}}%
\pgfpathlineto{\pgfqpoint{1.313194in}{1.795913in}}%
\pgfpathlineto{\pgfqpoint{1.313194in}{1.799665in}}%
\pgfpathlineto{\pgfqpoint{1.315075in}{1.801916in}}%
\pgfpathlineto{\pgfqpoint{1.316329in}{1.803416in}}%
\pgfpathlineto{\pgfqpoint{1.316329in}{1.807168in}}%
\pgfpathlineto{\pgfqpoint{1.316329in}{1.810919in}}%
\pgfpathlineto{\pgfqpoint{1.318210in}{1.813170in}}%
\pgfpathlineto{\pgfqpoint{1.319464in}{1.814671in}}%
\pgfpathlineto{\pgfqpoint{1.319464in}{1.818422in}}%
\pgfpathlineto{\pgfqpoint{1.319464in}{1.822174in}}%
\pgfpathlineto{\pgfqpoint{1.321344in}{1.824425in}}%
\pgfpathlineto{\pgfqpoint{1.322598in}{1.825925in}}%
\pgfpathlineto{\pgfqpoint{1.322598in}{1.829677in}}%
\pgfpathlineto{\pgfqpoint{1.322598in}{1.833429in}}%
\pgfpathlineto{\pgfqpoint{1.322598in}{1.837180in}}%
\pgfpathlineto{\pgfqpoint{1.324479in}{1.839431in}}%
\pgfpathlineto{\pgfqpoint{1.325733in}{1.840932in}}%
\pgfpathlineto{\pgfqpoint{1.325733in}{1.844683in}}%
\pgfpathlineto{\pgfqpoint{1.325733in}{1.848435in}}%
\pgfpathlineto{\pgfqpoint{1.327614in}{1.850686in}}%
\pgfpathlineto{\pgfqpoint{1.328868in}{1.852186in}}%
\pgfpathlineto{\pgfqpoint{1.328868in}{1.855938in}}%
\pgfpathlineto{\pgfqpoint{1.328868in}{1.859689in}}%
\pgfpathlineto{\pgfqpoint{1.328868in}{1.863441in}}%
\pgfpathlineto{\pgfqpoint{1.330749in}{1.865692in}}%
\pgfpathlineto{\pgfqpoint{1.332003in}{1.867192in}}%
\pgfpathlineto{\pgfqpoint{1.332003in}{1.870944in}}%
\pgfpathlineto{\pgfqpoint{1.332003in}{1.874695in}}%
\pgfpathlineto{\pgfqpoint{1.333883in}{1.876946in}}%
\pgfpathlineto{\pgfqpoint{1.335137in}{1.878447in}}%
\pgfpathlineto{\pgfqpoint{1.335137in}{1.882198in}}%
\pgfpathlineto{\pgfqpoint{1.335137in}{1.885950in}}%
\pgfpathlineto{\pgfqpoint{1.335137in}{1.889702in}}%
\pgfpathlineto{\pgfqpoint{1.337018in}{1.891952in}}%
\pgfpathlineto{\pgfqpoint{1.338272in}{1.893453in}}%
\pgfpathlineto{\pgfqpoint{1.338272in}{1.897205in}}%
\pgfpathlineto{\pgfqpoint{1.338272in}{1.900956in}}%
\pgfpathlineto{\pgfqpoint{1.340153in}{1.903207in}}%
\pgfpathlineto{\pgfqpoint{1.341407in}{1.904708in}}%
\pgfpathlineto{\pgfqpoint{1.341407in}{1.908459in}}%
\pgfpathlineto{\pgfqpoint{1.341407in}{1.912211in}}%
\pgfpathlineto{\pgfqpoint{1.341407in}{1.915962in}}%
\pgfpathlineto{\pgfqpoint{1.343288in}{1.918213in}}%
\pgfpathlineto{\pgfqpoint{1.344542in}{1.919714in}}%
\pgfpathlineto{\pgfqpoint{1.344542in}{1.923465in}}%
\pgfpathlineto{\pgfqpoint{1.344542in}{1.927217in}}%
\pgfpathlineto{\pgfqpoint{1.346422in}{1.929468in}}%
\pgfpathlineto{\pgfqpoint{1.347676in}{1.930968in}}%
\pgfpathlineto{\pgfqpoint{1.347676in}{1.934720in}}%
\pgfpathlineto{\pgfqpoint{1.347676in}{1.938471in}}%
\pgfpathlineto{\pgfqpoint{1.347676in}{1.942223in}}%
\pgfpathlineto{\pgfqpoint{1.349557in}{1.944474in}}%
\pgfpathlineto{\pgfqpoint{1.350811in}{1.945975in}}%
\pgfpathlineto{\pgfqpoint{1.350811in}{1.949726in}}%
\pgfpathlineto{\pgfqpoint{1.350811in}{1.953478in}}%
\pgfpathlineto{\pgfqpoint{1.352692in}{1.955729in}}%
\pgfpathlineto{\pgfqpoint{1.353946in}{1.957229in}}%
\pgfpathlineto{\pgfqpoint{1.353946in}{1.960981in}}%
\pgfpathlineto{\pgfqpoint{1.353946in}{1.964732in}}%
\pgfpathlineto{\pgfqpoint{1.353946in}{1.968484in}}%
\pgfpathlineto{\pgfqpoint{1.355827in}{1.970735in}}%
\pgfpathlineto{\pgfqpoint{1.357080in}{1.972235in}}%
\pgfpathlineto{\pgfqpoint{1.357080in}{1.975987in}}%
\pgfpathlineto{\pgfqpoint{1.357080in}{1.979738in}}%
\pgfpathlineto{\pgfqpoint{1.358961in}{1.981989in}}%
\pgfpathlineto{\pgfqpoint{1.360215in}{1.983490in}}%
\pgfpathlineto{\pgfqpoint{1.360215in}{1.987241in}}%
\pgfpathlineto{\pgfqpoint{1.360215in}{1.990993in}}%
\pgfpathlineto{\pgfqpoint{1.360215in}{1.994745in}}%
\pgfpathlineto{\pgfqpoint{1.362096in}{1.996995in}}%
\pgfpathlineto{\pgfqpoint{1.363350in}{1.998496in}}%
\pgfpathlineto{\pgfqpoint{1.363350in}{2.002248in}}%
\pgfpathlineto{\pgfqpoint{1.363350in}{2.005999in}}%
\pgfpathlineto{\pgfqpoint{1.365231in}{2.008250in}}%
\pgfpathlineto{\pgfqpoint{1.366485in}{2.009751in}}%
\pgfpathlineto{\pgfqpoint{1.366485in}{2.013502in}}%
\pgfpathlineto{\pgfqpoint{1.366485in}{2.017254in}}%
\pgfpathlineto{\pgfqpoint{1.366485in}{2.021005in}}%
\pgfpathlineto{\pgfqpoint{1.368366in}{2.023256in}}%
\pgfpathlineto{\pgfqpoint{1.369619in}{2.024757in}}%
\pgfpathlineto{\pgfqpoint{1.369619in}{2.028508in}}%
\pgfpathlineto{\pgfqpoint{1.369619in}{2.032260in}}%
\pgfpathlineto{\pgfqpoint{1.371500in}{2.034511in}}%
\pgfpathlineto{\pgfqpoint{1.372754in}{2.036011in}}%
\pgfpathlineto{\pgfqpoint{1.372754in}{2.039763in}}%
\pgfpathlineto{\pgfqpoint{1.372754in}{2.043514in}}%
\pgfpathlineto{\pgfqpoint{1.374635in}{2.045765in}}%
\pgfpathlineto{\pgfqpoint{1.375889in}{2.047266in}}%
\pgfpathlineto{\pgfqpoint{1.375889in}{2.051018in}}%
\pgfpathlineto{\pgfqpoint{1.375889in}{2.054769in}}%
\pgfpathlineto{\pgfqpoint{1.375889in}{2.058521in}}%
\pgfpathlineto{\pgfqpoint{1.377770in}{2.060772in}}%
\pgfpathlineto{\pgfqpoint{1.379024in}{2.062272in}}%
\pgfpathlineto{\pgfqpoint{1.379024in}{2.066024in}}%
\pgfpathlineto{\pgfqpoint{1.379024in}{2.069775in}}%
\pgfpathlineto{\pgfqpoint{1.380905in}{2.072026in}}%
\pgfpathlineto{\pgfqpoint{1.382158in}{2.073527in}}%
\pgfpathlineto{\pgfqpoint{1.382158in}{2.077278in}}%
\pgfpathlineto{\pgfqpoint{1.382158in}{2.081030in}}%
\pgfpathlineto{\pgfqpoint{1.382158in}{2.084781in}}%
\pgfpathlineto{\pgfqpoint{1.384039in}{2.087032in}}%
\pgfpathlineto{\pgfqpoint{1.385293in}{2.088533in}}%
\pgfpathlineto{\pgfqpoint{1.385293in}{2.092284in}}%
\pgfpathlineto{\pgfqpoint{1.385293in}{2.096036in}}%
\pgfpathlineto{\pgfqpoint{1.387174in}{2.098287in}}%
\pgfpathlineto{\pgfqpoint{1.388428in}{2.099787in}}%
\pgfpathlineto{\pgfqpoint{1.388428in}{2.103539in}}%
\pgfpathlineto{\pgfqpoint{1.388428in}{2.107291in}}%
\pgfpathlineto{\pgfqpoint{1.388428in}{2.111042in}}%
\pgfpathlineto{\pgfqpoint{1.390309in}{2.113293in}}%
\pgfpathlineto{\pgfqpoint{1.391563in}{2.114794in}}%
\pgfpathlineto{\pgfqpoint{1.391563in}{2.118545in}}%
\pgfpathlineto{\pgfqpoint{1.391563in}{2.122297in}}%
\pgfpathlineto{\pgfqpoint{1.393444in}{2.124548in}}%
\pgfpathlineto{\pgfqpoint{1.394697in}{2.126048in}}%
\pgfpathlineto{\pgfqpoint{1.394697in}{2.129800in}}%
\pgfpathlineto{\pgfqpoint{1.394697in}{2.133551in}}%
\pgfpathlineto{\pgfqpoint{1.394697in}{2.137303in}}%
\pgfpathlineto{\pgfqpoint{1.396578in}{2.139554in}}%
\pgfpathlineto{\pgfqpoint{1.397832in}{2.141054in}}%
\pgfpathlineto{\pgfqpoint{1.397832in}{2.144806in}}%
\pgfpathlineto{\pgfqpoint{1.397832in}{2.148557in}}%
\pgfpathlineto{\pgfqpoint{1.399713in}{2.150808in}}%
\pgfpathlineto{\pgfqpoint{1.400967in}{2.152309in}}%
\pgfpathlineto{\pgfqpoint{1.400967in}{2.156060in}}%
\pgfpathlineto{\pgfqpoint{1.400967in}{2.159812in}}%
\pgfpathlineto{\pgfqpoint{1.400967in}{2.163564in}}%
\pgfpathlineto{\pgfqpoint{1.402848in}{2.165814in}}%
\pgfpathlineto{\pgfqpoint{1.404102in}{2.167315in}}%
\pgfpathlineto{\pgfqpoint{1.404102in}{2.171067in}}%
\pgfpathlineto{\pgfqpoint{1.404102in}{2.174818in}}%
\pgfpathlineto{\pgfqpoint{1.405982in}{2.177069in}}%
\pgfpathlineto{\pgfqpoint{1.407236in}{2.178570in}}%
\pgfpathlineto{\pgfqpoint{1.407236in}{2.182321in}}%
\pgfpathlineto{\pgfqpoint{1.407236in}{2.186073in}}%
\pgfpathlineto{\pgfqpoint{1.407236in}{2.189824in}}%
\pgfpathlineto{\pgfqpoint{1.409117in}{2.192075in}}%
\pgfpathlineto{\pgfqpoint{1.410371in}{2.193576in}}%
\pgfpathlineto{\pgfqpoint{1.410371in}{2.197327in}}%
\pgfpathlineto{\pgfqpoint{1.410371in}{2.201079in}}%
\pgfpathlineto{\pgfqpoint{1.412252in}{2.203330in}}%
\pgfpathlineto{\pgfqpoint{1.413506in}{2.204830in}}%
\pgfpathlineto{\pgfqpoint{1.413506in}{2.208582in}}%
\pgfpathlineto{\pgfqpoint{1.413506in}{2.212334in}}%
\pgfpathlineto{\pgfqpoint{1.413506in}{2.216085in}}%
\pgfpathlineto{\pgfqpoint{1.415387in}{2.218336in}}%
\pgfpathlineto{\pgfqpoint{1.416641in}{2.219837in}}%
\pgfpathlineto{\pgfqpoint{1.416641in}{2.223588in}}%
\pgfpathlineto{\pgfqpoint{1.416641in}{2.227340in}}%
\pgfpathlineto{\pgfqpoint{1.418521in}{2.229591in}}%
\pgfpathlineto{\pgfqpoint{1.419775in}{2.231091in}}%
\pgfpathlineto{\pgfqpoint{1.419775in}{2.234843in}}%
\pgfpathlineto{\pgfqpoint{1.419775in}{2.238594in}}%
\pgfpathlineto{\pgfqpoint{1.419775in}{2.242346in}}%
\pgfpathlineto{\pgfqpoint{1.421656in}{2.244597in}}%
\pgfpathlineto{\pgfqpoint{1.422910in}{2.246097in}}%
\pgfpathlineto{\pgfqpoint{1.422910in}{2.249849in}}%
\pgfpathlineto{\pgfqpoint{1.422910in}{2.253600in}}%
\pgfpathlineto{\pgfqpoint{1.424791in}{2.255851in}}%
\pgfpathlineto{\pgfqpoint{1.426045in}{2.257352in}}%
\pgfpathlineto{\pgfqpoint{1.426045in}{2.261103in}}%
\pgfpathlineto{\pgfqpoint{1.426045in}{2.264855in}}%
\pgfpathlineto{\pgfqpoint{1.427926in}{2.267106in}}%
\pgfpathlineto{\pgfqpoint{1.429180in}{2.268607in}}%
\pgfpathlineto{\pgfqpoint{1.429180in}{2.272358in}}%
\pgfpathlineto{\pgfqpoint{1.429180in}{2.276110in}}%
\pgfpathlineto{\pgfqpoint{1.429180in}{2.279861in}}%
\pgfpathlineto{\pgfqpoint{1.431060in}{2.282112in}}%
\pgfpathlineto{\pgfqpoint{1.432314in}{2.283613in}}%
\pgfpathlineto{\pgfqpoint{1.432314in}{2.287364in}}%
\pgfpathlineto{\pgfqpoint{1.432314in}{2.291116in}}%
\pgfpathlineto{\pgfqpoint{1.434195in}{2.293367in}}%
\pgfpathlineto{\pgfqpoint{1.435449in}{2.294867in}}%
\pgfpathlineto{\pgfqpoint{1.435449in}{2.298619in}}%
\pgfpathlineto{\pgfqpoint{1.435449in}{2.302370in}}%
\pgfpathlineto{\pgfqpoint{1.435449in}{2.306122in}}%
\pgfpathlineto{\pgfqpoint{1.437330in}{2.308373in}}%
\pgfpathlineto{\pgfqpoint{1.438584in}{2.309873in}}%
\pgfpathlineto{\pgfqpoint{1.438584in}{2.313625in}}%
\pgfpathlineto{\pgfqpoint{1.438584in}{2.317376in}}%
\pgfpathlineto{\pgfqpoint{1.440465in}{2.319627in}}%
\pgfpathlineto{\pgfqpoint{1.441719in}{2.321128in}}%
\pgfpathlineto{\pgfqpoint{1.441719in}{2.324880in}}%
\pgfpathlineto{\pgfqpoint{1.441719in}{2.328631in}}%
\pgfpathlineto{\pgfqpoint{1.441719in}{2.332383in}}%
\pgfpathlineto{\pgfqpoint{1.443599in}{2.334634in}}%
\pgfpathlineto{\pgfqpoint{1.444853in}{2.336134in}}%
\pgfpathlineto{\pgfqpoint{1.444853in}{2.339886in}}%
\pgfpathlineto{\pgfqpoint{1.444853in}{2.343637in}}%
\pgfpathlineto{\pgfqpoint{1.446734in}{2.345888in}}%
\pgfpathlineto{\pgfqpoint{1.447988in}{2.347389in}}%
\pgfpathlineto{\pgfqpoint{1.447988in}{2.351140in}}%
\pgfpathlineto{\pgfqpoint{1.447988in}{2.354892in}}%
\pgfpathlineto{\pgfqpoint{1.447988in}{2.358643in}}%
\pgfpathlineto{\pgfqpoint{1.449869in}{2.360894in}}%
\pgfpathlineto{\pgfqpoint{1.451123in}{2.362395in}}%
\pgfpathlineto{\pgfqpoint{1.451123in}{2.366146in}}%
\pgfpathlineto{\pgfqpoint{1.451123in}{2.369898in}}%
\pgfpathlineto{\pgfqpoint{1.453004in}{2.372149in}}%
\pgfpathlineto{\pgfqpoint{1.454258in}{2.373649in}}%
\pgfpathlineto{\pgfqpoint{1.454258in}{2.377401in}}%
\pgfpathlineto{\pgfqpoint{1.454258in}{2.381153in}}%
\pgfpathlineto{\pgfqpoint{1.454258in}{2.384904in}}%
\pgfpathlineto{\pgfqpoint{1.456138in}{2.387155in}}%
\pgfpathlineto{\pgfqpoint{1.457392in}{2.388656in}}%
\pgfpathlineto{\pgfqpoint{1.457392in}{2.392407in}}%
\pgfpathlineto{\pgfqpoint{1.457392in}{2.396159in}}%
\pgfpathlineto{\pgfqpoint{1.459273in}{2.398410in}}%
\pgfpathlineto{\pgfqpoint{1.460527in}{2.399910in}}%
\pgfpathlineto{\pgfqpoint{1.460527in}{2.403662in}}%
\pgfpathlineto{\pgfqpoint{1.462408in}{2.405913in}}%
\pgfpathlineto{\pgfqpoint{1.465543in}{2.405913in}}%
\pgfpathlineto{\pgfqpoint{1.466796in}{2.407413in}}%
\pgfpathlineto{\pgfqpoint{1.468677in}{2.409664in}}%
\pgfpathlineto{\pgfqpoint{1.471812in}{2.409664in}}%
\pgfpathlineto{\pgfqpoint{1.474947in}{2.409664in}}%
\pgfpathlineto{\pgfqpoint{1.478082in}{2.409664in}}%
\pgfpathlineto{\pgfqpoint{1.479335in}{2.411165in}}%
\pgfpathlineto{\pgfqpoint{1.481216in}{2.413416in}}%
\pgfpathlineto{\pgfqpoint{1.484351in}{2.413416in}}%
\pgfpathlineto{\pgfqpoint{1.487486in}{2.413416in}}%
\pgfpathlineto{\pgfqpoint{1.490621in}{2.413416in}}%
\pgfpathlineto{\pgfqpoint{1.491874in}{2.414916in}}%
\pgfpathlineto{\pgfqpoint{1.493755in}{2.417167in}}%
\pgfpathlineto{\pgfqpoint{1.496890in}{2.417167in}}%
\pgfpathlineto{\pgfqpoint{1.500025in}{2.417167in}}%
\pgfpathlineto{\pgfqpoint{1.503159in}{2.417167in}}%
\pgfpathlineto{\pgfqpoint{1.504413in}{2.418668in}}%
\pgfpathlineto{\pgfqpoint{1.506294in}{2.420919in}}%
\pgfpathlineto{\pgfqpoint{1.509429in}{2.420919in}}%
\pgfpathlineto{\pgfqpoint{1.512564in}{2.420919in}}%
\pgfpathlineto{\pgfqpoint{1.515698in}{2.420919in}}%
\pgfpathlineto{\pgfqpoint{1.516952in}{2.422419in}}%
\pgfpathlineto{\pgfqpoint{1.518833in}{2.424670in}}%
\pgfpathlineto{\pgfqpoint{1.521968in}{2.424670in}}%
\pgfpathlineto{\pgfqpoint{1.525103in}{2.424670in}}%
\pgfpathlineto{\pgfqpoint{1.528237in}{2.424670in}}%
\pgfpathlineto{\pgfqpoint{1.531372in}{2.424670in}}%
\pgfpathlineto{\pgfqpoint{1.532626in}{2.426171in}}%
\pgfpathlineto{\pgfqpoint{1.534507in}{2.428422in}}%
\pgfpathlineto{\pgfqpoint{1.537642in}{2.428422in}}%
\pgfpathlineto{\pgfqpoint{1.540776in}{2.428422in}}%
\pgfpathlineto{\pgfqpoint{1.543911in}{2.428422in}}%
\pgfpathlineto{\pgfqpoint{1.545165in}{2.429923in}}%
\pgfpathlineto{\pgfqpoint{1.547046in}{2.432173in}}%
\pgfpathlineto{\pgfqpoint{1.550181in}{2.432173in}}%
\pgfpathlineto{\pgfqpoint{1.553315in}{2.432173in}}%
\pgfpathlineto{\pgfqpoint{1.556450in}{2.432173in}}%
\pgfpathlineto{\pgfqpoint{1.557704in}{2.433674in}}%
\pgfpathlineto{\pgfqpoint{1.559585in}{2.435925in}}%
\pgfpathlineto{\pgfqpoint{1.562720in}{2.435925in}}%
\pgfpathlineto{\pgfqpoint{1.565854in}{2.435925in}}%
\pgfpathlineto{\pgfqpoint{1.568989in}{2.435925in}}%
\pgfpathlineto{\pgfqpoint{1.570243in}{2.437426in}}%
\pgfpathlineto{\pgfqpoint{1.572124in}{2.439676in}}%
\pgfpathlineto{\pgfqpoint{1.575259in}{2.439676in}}%
\pgfpathlineto{\pgfqpoint{1.578393in}{2.439676in}}%
\pgfpathlineto{\pgfqpoint{1.581528in}{2.439676in}}%
\pgfpathlineto{\pgfqpoint{1.584663in}{2.439676in}}%
\pgfpathlineto{\pgfqpoint{1.585917in}{2.441177in}}%
\pgfpathlineto{\pgfqpoint{1.587798in}{2.443428in}}%
\pgfpathlineto{\pgfqpoint{1.590932in}{2.443428in}}%
\pgfpathlineto{\pgfqpoint{1.594067in}{2.443428in}}%
\pgfpathlineto{\pgfqpoint{1.597202in}{2.443428in}}%
\pgfpathlineto{\pgfqpoint{1.598456in}{2.444929in}}%
\pgfpathlineto{\pgfqpoint{1.600337in}{2.447180in}}%
\pgfpathlineto{\pgfqpoint{1.603471in}{2.447180in}}%
\pgfpathlineto{\pgfqpoint{1.606606in}{2.447180in}}%
\pgfpathlineto{\pgfqpoint{1.609741in}{2.447180in}}%
\pgfpathlineto{\pgfqpoint{1.610995in}{2.448680in}}%
\pgfpathlineto{\pgfqpoint{1.612875in}{2.450931in}}%
\pgfpathlineto{\pgfqpoint{1.616010in}{2.450931in}}%
\pgfpathlineto{\pgfqpoint{1.619145in}{2.450931in}}%
\pgfpathlineto{\pgfqpoint{1.622280in}{2.450931in}}%
\pgfpathlineto{\pgfqpoint{1.623534in}{2.452432in}}%
\pgfpathlineto{\pgfqpoint{1.625414in}{2.454683in}}%
\pgfpathlineto{\pgfqpoint{1.628549in}{2.454683in}}%
\pgfpathlineto{\pgfqpoint{1.631684in}{2.454683in}}%
\pgfpathlineto{\pgfqpoint{1.634819in}{2.454683in}}%
\pgfpathlineto{\pgfqpoint{1.637953in}{2.454683in}}%
\pgfpathlineto{\pgfqpoint{1.639207in}{2.456183in}}%
\pgfpathlineto{\pgfqpoint{1.641088in}{2.458434in}}%
\pgfpathlineto{\pgfqpoint{1.644223in}{2.458434in}}%
\pgfpathlineto{\pgfqpoint{1.647358in}{2.458434in}}%
\pgfpathlineto{\pgfqpoint{1.650492in}{2.458434in}}%
\pgfpathlineto{\pgfqpoint{1.651746in}{2.459935in}}%
\pgfpathlineto{\pgfqpoint{1.653627in}{2.462186in}}%
\pgfpathlineto{\pgfqpoint{1.656762in}{2.462186in}}%
\pgfpathlineto{\pgfqpoint{1.659897in}{2.462186in}}%
\pgfpathlineto{\pgfqpoint{1.663031in}{2.462186in}}%
\pgfpathlineto{\pgfqpoint{1.664285in}{2.463686in}}%
\pgfpathlineto{\pgfqpoint{1.666166in}{2.465937in}}%
\pgfpathlineto{\pgfqpoint{1.669301in}{2.465937in}}%
\pgfpathlineto{\pgfqpoint{1.672436in}{2.465937in}}%
\pgfpathlineto{\pgfqpoint{1.675570in}{2.465937in}}%
\pgfpathlineto{\pgfqpoint{1.676824in}{2.467438in}}%
\pgfpathlineto{\pgfqpoint{1.678705in}{2.469689in}}%
\pgfpathlineto{\pgfqpoint{1.681840in}{2.469689in}}%
\pgfpathlineto{\pgfqpoint{1.684975in}{2.469689in}}%
\pgfpathlineto{\pgfqpoint{1.688109in}{2.469689in}}%
\pgfpathlineto{\pgfqpoint{1.689363in}{2.471189in}}%
\pgfpathlineto{\pgfqpoint{1.691244in}{2.473440in}}%
\pgfpathlineto{\pgfqpoint{1.694379in}{2.473440in}}%
\pgfpathlineto{\pgfqpoint{1.697514in}{2.473440in}}%
\pgfpathlineto{\pgfqpoint{1.700648in}{2.473440in}}%
\pgfpathlineto{\pgfqpoint{1.703783in}{2.473440in}}%
\pgfpathlineto{\pgfqpoint{1.705037in}{2.474941in}}%
\pgfpathlineto{\pgfqpoint{1.706918in}{2.477192in}}%
\pgfpathlineto{\pgfqpoint{1.710052in}{2.477192in}}%
\pgfpathlineto{\pgfqpoint{1.713187in}{2.477192in}}%
\pgfpathlineto{\pgfqpoint{1.716322in}{2.477192in}}%
\pgfpathlineto{\pgfqpoint{1.717576in}{2.478692in}}%
\pgfpathlineto{\pgfqpoint{1.719457in}{2.480943in}}%
\pgfpathlineto{\pgfqpoint{1.722591in}{2.480943in}}%
\pgfpathlineto{\pgfqpoint{1.725726in}{2.480943in}}%
\pgfpathlineto{\pgfqpoint{1.728861in}{2.480943in}}%
\pgfpathlineto{\pgfqpoint{1.730115in}{2.482444in}}%
\pgfpathlineto{\pgfqpoint{1.731996in}{2.484695in}}%
\pgfpathlineto{\pgfqpoint{1.735130in}{2.484695in}}%
\pgfpathlineto{\pgfqpoint{1.738265in}{2.484695in}}%
\pgfpathlineto{\pgfqpoint{1.741400in}{2.484695in}}%
\pgfpathlineto{\pgfqpoint{1.742654in}{2.486196in}}%
\pgfpathlineto{\pgfqpoint{1.744535in}{2.488446in}}%
\pgfpathlineto{\pgfqpoint{1.747669in}{2.488446in}}%
\pgfpathlineto{\pgfqpoint{1.750804in}{2.488446in}}%
\pgfpathlineto{\pgfqpoint{1.753939in}{2.488446in}}%
\pgfpathlineto{\pgfqpoint{1.757074in}{2.488446in}}%
\pgfpathlineto{\pgfqpoint{1.758328in}{2.489947in}}%
\pgfpathlineto{\pgfqpoint{1.760208in}{2.492198in}}%
\pgfpathlineto{\pgfqpoint{1.763343in}{2.492198in}}%
\pgfpathlineto{\pgfqpoint{1.766478in}{2.492198in}}%
\pgfpathlineto{\pgfqpoint{1.769613in}{2.492198in}}%
\pgfpathlineto{\pgfqpoint{1.770866in}{2.493699in}}%
\pgfpathlineto{\pgfqpoint{1.772747in}{2.495950in}}%
\pgfpathlineto{\pgfqpoint{1.775882in}{2.495950in}}%
\pgfpathlineto{\pgfqpoint{1.779017in}{2.495950in}}%
\pgfpathlineto{\pgfqpoint{1.782152in}{2.495950in}}%
\pgfpathlineto{\pgfqpoint{1.783405in}{2.497450in}}%
\pgfpathlineto{\pgfqpoint{1.785286in}{2.499701in}}%
\pgfpathlineto{\pgfqpoint{1.788421in}{2.499701in}}%
\pgfpathlineto{\pgfqpoint{1.791556in}{2.499701in}}%
\pgfpathlineto{\pgfqpoint{1.794691in}{2.499701in}}%
\pgfpathlineto{\pgfqpoint{1.795944in}{2.501202in}}%
\pgfpathlineto{\pgfqpoint{1.797825in}{2.503453in}}%
\pgfpathlineto{\pgfqpoint{1.800960in}{2.503453in}}%
\pgfpathlineto{\pgfqpoint{1.804095in}{2.503453in}}%
\pgfpathlineto{\pgfqpoint{1.807229in}{2.503453in}}%
\pgfpathlineto{\pgfqpoint{1.810364in}{2.503453in}}%
\pgfpathlineto{\pgfqpoint{1.811618in}{2.504953in}}%
\pgfpathlineto{\pgfqpoint{1.813499in}{2.507204in}}%
\pgfpathlineto{\pgfqpoint{1.816634in}{2.507204in}}%
\pgfpathlineto{\pgfqpoint{1.819768in}{2.507204in}}%
\pgfpathlineto{\pgfqpoint{1.822903in}{2.507204in}}%
\pgfpathlineto{\pgfqpoint{1.824157in}{2.508705in}}%
\pgfpathlineto{\pgfqpoint{1.826038in}{2.510956in}}%
\pgfpathlineto{\pgfqpoint{1.829173in}{2.510956in}}%
\pgfpathlineto{\pgfqpoint{1.832307in}{2.510956in}}%
\pgfpathlineto{\pgfqpoint{1.835442in}{2.510956in}}%
\pgfpathlineto{\pgfqpoint{1.836696in}{2.512456in}}%
\pgfpathlineto{\pgfqpoint{1.838577in}{2.514707in}}%
\pgfpathlineto{\pgfqpoint{1.841712in}{2.514707in}}%
\pgfpathlineto{\pgfqpoint{1.844846in}{2.514707in}}%
\pgfpathlineto{\pgfqpoint{1.847981in}{2.514707in}}%
\pgfpathlineto{\pgfqpoint{1.849235in}{2.516208in}}%
\pgfpathlineto{\pgfqpoint{1.851116in}{2.518459in}}%
\pgfpathlineto{\pgfqpoint{1.854251in}{2.518459in}}%
\pgfpathlineto{\pgfqpoint{1.857385in}{2.518459in}}%
\pgfpathlineto{\pgfqpoint{1.860520in}{2.518459in}}%
\pgfpathlineto{\pgfqpoint{1.861774in}{2.519959in}}%
\pgfpathlineto{\pgfqpoint{1.863655in}{2.522210in}}%
\pgfpathlineto{\pgfqpoint{1.866790in}{2.522210in}}%
\pgfpathlineto{\pgfqpoint{1.869924in}{2.522210in}}%
\pgfpathlineto{\pgfqpoint{1.873059in}{2.522210in}}%
\pgfpathlineto{\pgfqpoint{1.876194in}{2.522210in}}%
\pgfpathlineto{\pgfqpoint{1.877448in}{2.523711in}}%
\pgfpathlineto{\pgfqpoint{1.879329in}{2.525962in}}%
\pgfpathlineto{\pgfqpoint{1.882463in}{2.525962in}}%
\pgfpathlineto{\pgfqpoint{1.885598in}{2.525962in}}%
\pgfpathlineto{\pgfqpoint{1.888733in}{2.525962in}}%
\pgfpathlineto{\pgfqpoint{1.889987in}{2.527462in}}%
\pgfpathlineto{\pgfqpoint{1.891868in}{2.529713in}}%
\pgfpathlineto{\pgfqpoint{1.895002in}{2.529713in}}%
\pgfpathlineto{\pgfqpoint{1.898137in}{2.529713in}}%
\pgfpathlineto{\pgfqpoint{1.901272in}{2.529713in}}%
\pgfpathlineto{\pgfqpoint{1.902526in}{2.531214in}}%
\pgfpathlineto{\pgfqpoint{1.904407in}{2.533465in}}%
\pgfpathlineto{\pgfqpoint{1.907541in}{2.533465in}}%
\pgfpathlineto{\pgfqpoint{1.910676in}{2.533465in}}%
\pgfpathlineto{\pgfqpoint{1.913811in}{2.533465in}}%
\pgfpathlineto{\pgfqpoint{1.915065in}{2.534965in}}%
\pgfpathlineto{\pgfqpoint{1.916945in}{2.537216in}}%
\pgfpathlineto{\pgfqpoint{1.920080in}{2.537216in}}%
\pgfpathlineto{\pgfqpoint{1.923215in}{2.537216in}}%
\pgfpathlineto{\pgfqpoint{1.926350in}{2.537216in}}%
\pgfpathlineto{\pgfqpoint{1.929484in}{2.537216in}}%
\pgfpathlineto{\pgfqpoint{1.930738in}{2.538717in}}%
\pgfpathlineto{\pgfqpoint{1.932619in}{2.540968in}}%
\pgfpathlineto{\pgfqpoint{1.935754in}{2.540968in}}%
\pgfpathlineto{\pgfqpoint{1.938889in}{2.540968in}}%
\pgfpathlineto{\pgfqpoint{1.942023in}{2.540968in}}%
\pgfpathlineto{\pgfqpoint{1.943277in}{2.542469in}}%
\pgfpathlineto{\pgfqpoint{1.945158in}{2.544719in}}%
\pgfpathlineto{\pgfqpoint{1.948293in}{2.544719in}}%
\pgfpathlineto{\pgfqpoint{1.951428in}{2.544719in}}%
\pgfpathlineto{\pgfqpoint{1.954562in}{2.544719in}}%
\pgfpathlineto{\pgfqpoint{1.955816in}{2.546220in}}%
\pgfpathlineto{\pgfqpoint{1.957697in}{2.548471in}}%
\pgfpathlineto{\pgfqpoint{1.960832in}{2.548471in}}%
\pgfpathlineto{\pgfqpoint{1.963967in}{2.548471in}}%
\pgfpathlineto{\pgfqpoint{1.967101in}{2.548471in}}%
\pgfpathlineto{\pgfqpoint{1.968355in}{2.549972in}}%
\pgfpathlineto{\pgfqpoint{1.970236in}{2.552223in}}%
\pgfpathlineto{\pgfqpoint{1.973371in}{2.552223in}}%
\pgfpathlineto{\pgfqpoint{1.976506in}{2.552223in}}%
\pgfpathlineto{\pgfqpoint{1.979640in}{2.552223in}}%
\pgfpathlineto{\pgfqpoint{1.980894in}{2.553723in}}%
\pgfpathlineto{\pgfqpoint{1.982775in}{2.555974in}}%
\pgfpathlineto{\pgfqpoint{1.985910in}{2.555974in}}%
\pgfpathlineto{\pgfqpoint{1.989045in}{2.555974in}}%
\pgfpathlineto{\pgfqpoint{1.992179in}{2.555974in}}%
\pgfpathlineto{\pgfqpoint{1.995314in}{2.555974in}}%
\pgfpathlineto{\pgfqpoint{1.996568in}{2.557475in}}%
\pgfpathlineto{\pgfqpoint{1.998449in}{2.559726in}}%
\pgfpathlineto{\pgfqpoint{2.001584in}{2.559726in}}%
\pgfpathlineto{\pgfqpoint{2.004718in}{2.559726in}}%
\pgfpathlineto{\pgfqpoint{2.007853in}{2.559726in}}%
\pgfpathlineto{\pgfqpoint{2.009107in}{2.561226in}}%
\pgfpathlineto{\pgfqpoint{2.010988in}{2.563477in}}%
\pgfpathlineto{\pgfqpoint{2.014122in}{2.563477in}}%
\pgfpathlineto{\pgfqpoint{2.017257in}{2.563477in}}%
\pgfpathlineto{\pgfqpoint{2.020392in}{2.563477in}}%
\pgfpathlineto{\pgfqpoint{2.021646in}{2.564978in}}%
\pgfpathlineto{\pgfqpoint{2.023527in}{2.567229in}}%
\pgfpathlineto{\pgfqpoint{2.026661in}{2.567229in}}%
\pgfpathlineto{\pgfqpoint{2.029796in}{2.567229in}}%
\pgfpathlineto{\pgfqpoint{2.032931in}{2.567229in}}%
\pgfpathlineto{\pgfqpoint{2.034185in}{2.568729in}}%
\pgfpathlineto{\pgfqpoint{2.036066in}{2.570980in}}%
\pgfpathlineto{\pgfqpoint{2.039200in}{2.570980in}}%
\pgfpathlineto{\pgfqpoint{2.042335in}{2.570980in}}%
\pgfpathlineto{\pgfqpoint{2.045470in}{2.570980in}}%
\pgfpathlineto{\pgfqpoint{2.048605in}{2.570980in}}%
\pgfpathlineto{\pgfqpoint{2.049859in}{2.572481in}}%
\pgfpathlineto{\pgfqpoint{2.051739in}{2.574732in}}%
\pgfpathlineto{\pgfqpoint{2.054874in}{2.574732in}}%
\pgfpathlineto{\pgfqpoint{2.058009in}{2.574732in}}%
\pgfpathlineto{\pgfqpoint{2.061144in}{2.574732in}}%
\pgfpathlineto{\pgfqpoint{2.062398in}{2.576232in}}%
\pgfpathlineto{\pgfqpoint{2.064278in}{2.578483in}}%
\pgfpathlineto{\pgfqpoint{2.067413in}{2.578483in}}%
\pgfpathlineto{\pgfqpoint{2.070548in}{2.578483in}}%
\pgfpathlineto{\pgfqpoint{2.073683in}{2.578483in}}%
\pgfpathlineto{\pgfqpoint{2.074936in}{2.579984in}}%
\pgfpathlineto{\pgfqpoint{2.076817in}{2.582235in}}%
\pgfpathlineto{\pgfqpoint{2.079952in}{2.582235in}}%
\pgfpathlineto{\pgfqpoint{2.083087in}{2.582235in}}%
\pgfpathlineto{\pgfqpoint{2.086222in}{2.582235in}}%
\pgfpathlineto{\pgfqpoint{2.087475in}{2.583735in}}%
\pgfpathlineto{\pgfqpoint{2.089356in}{2.585986in}}%
\pgfpathlineto{\pgfqpoint{2.092491in}{2.585986in}}%
\pgfpathlineto{\pgfqpoint{2.095626in}{2.585986in}}%
\pgfpathlineto{\pgfqpoint{2.098761in}{2.585986in}}%
\pgfpathlineto{\pgfqpoint{2.101895in}{2.585986in}}%
\pgfpathlineto{\pgfqpoint{2.103149in}{2.587487in}}%
\pgfpathlineto{\pgfqpoint{2.105030in}{2.589738in}}%
\pgfpathlineto{\pgfqpoint{2.108165in}{2.589738in}}%
\pgfpathlineto{\pgfqpoint{2.111299in}{2.589738in}}%
\pgfpathlineto{\pgfqpoint{2.114434in}{2.589738in}}%
\pgfpathlineto{\pgfqpoint{2.115688in}{2.591238in}}%
\pgfpathlineto{\pgfqpoint{2.117569in}{2.593489in}}%
\pgfpathlineto{\pgfqpoint{2.120704in}{2.593489in}}%
\pgfpathlineto{\pgfqpoint{2.123838in}{2.593489in}}%
\pgfpathlineto{\pgfqpoint{2.126973in}{2.593489in}}%
\pgfpathlineto{\pgfqpoint{2.128227in}{2.594990in}}%
\pgfpathlineto{\pgfqpoint{2.130108in}{2.597241in}}%
\pgfpathlineto{\pgfqpoint{2.133243in}{2.597241in}}%
\pgfpathlineto{\pgfqpoint{2.136377in}{2.597241in}}%
\pgfpathlineto{\pgfqpoint{2.139512in}{2.597241in}}%
\pgfpathlineto{\pgfqpoint{2.140766in}{2.598742in}}%
\pgfpathlineto{\pgfqpoint{2.142647in}{2.600992in}}%
\pgfpathlineto{\pgfqpoint{2.145782in}{2.600992in}}%
\pgfpathlineto{\pgfqpoint{2.148916in}{2.600992in}}%
\pgfpathlineto{\pgfqpoint{2.152051in}{2.600992in}}%
\pgfpathlineto{\pgfqpoint{2.153305in}{2.602493in}}%
\pgfpathlineto{\pgfqpoint{2.155186in}{2.604744in}}%
\pgfpathlineto{\pgfqpoint{2.158321in}{2.604744in}}%
\pgfpathlineto{\pgfqpoint{2.161455in}{2.604744in}}%
\pgfpathlineto{\pgfqpoint{2.164590in}{2.604744in}}%
\pgfpathlineto{\pgfqpoint{2.167725in}{2.604744in}}%
\pgfpathlineto{\pgfqpoint{2.168979in}{2.606245in}}%
\pgfpathlineto{\pgfqpoint{2.170860in}{2.608496in}}%
\pgfpathlineto{\pgfqpoint{2.173994in}{2.608496in}}%
\pgfpathlineto{\pgfqpoint{2.177129in}{2.608496in}}%
\pgfpathlineto{\pgfqpoint{2.180264in}{2.608496in}}%
\pgfpathlineto{\pgfqpoint{2.181518in}{2.609996in}}%
\pgfpathlineto{\pgfqpoint{2.183399in}{2.612247in}}%
\pgfpathlineto{\pgfqpoint{2.186533in}{2.612247in}}%
\pgfpathlineto{\pgfqpoint{2.189668in}{2.612247in}}%
\pgfpathlineto{\pgfqpoint{2.192803in}{2.612247in}}%
\pgfpathlineto{\pgfqpoint{2.194057in}{2.613748in}}%
\pgfpathlineto{\pgfqpoint{2.195938in}{2.615999in}}%
\pgfpathlineto{\pgfqpoint{2.199072in}{2.615999in}}%
\pgfpathlineto{\pgfqpoint{2.202207in}{2.615999in}}%
\pgfpathlineto{\pgfqpoint{2.205342in}{2.615999in}}%
\pgfpathlineto{\pgfqpoint{2.206596in}{2.617499in}}%
\pgfpathlineto{\pgfqpoint{2.208477in}{2.619750in}}%
\pgfpathlineto{\pgfqpoint{2.211611in}{2.619750in}}%
\pgfpathlineto{\pgfqpoint{2.214746in}{2.619750in}}%
\pgfpathlineto{\pgfqpoint{2.217881in}{2.619750in}}%
\pgfpathlineto{\pgfqpoint{2.221015in}{2.619750in}}%
\pgfpathlineto{\pgfqpoint{2.222269in}{2.621251in}}%
\pgfpathlineto{\pgfqpoint{2.224150in}{2.623502in}}%
\pgfpathlineto{\pgfqpoint{2.227285in}{2.623502in}}%
\pgfpathlineto{\pgfqpoint{2.230420in}{2.623502in}}%
\pgfpathlineto{\pgfqpoint{2.233554in}{2.623502in}}%
\pgfpathlineto{\pgfqpoint{2.234808in}{2.625002in}}%
\pgfpathlineto{\pgfqpoint{2.236689in}{2.627253in}}%
\pgfpathlineto{\pgfqpoint{2.239824in}{2.627253in}}%
\pgfpathlineto{\pgfqpoint{2.242959in}{2.627253in}}%
\pgfpathlineto{\pgfqpoint{2.246093in}{2.627253in}}%
\pgfpathlineto{\pgfqpoint{2.247347in}{2.628754in}}%
\pgfpathlineto{\pgfqpoint{2.249228in}{2.631005in}}%
\pgfpathlineto{\pgfqpoint{2.252363in}{2.631005in}}%
\pgfpathlineto{\pgfqpoint{2.255498in}{2.631005in}}%
\pgfpathlineto{\pgfqpoint{2.258632in}{2.631005in}}%
\pgfpathlineto{\pgfqpoint{2.259886in}{2.632505in}}%
\pgfpathlineto{\pgfqpoint{2.261767in}{2.634756in}}%
\pgfpathlineto{\pgfqpoint{2.264902in}{2.634756in}}%
\pgfpathlineto{\pgfqpoint{2.268037in}{2.634756in}}%
\pgfpathlineto{\pgfqpoint{2.271171in}{2.634756in}}%
\pgfpathlineto{\pgfqpoint{2.274306in}{2.634756in}}%
\pgfpathlineto{\pgfqpoint{2.275560in}{2.636257in}}%
\pgfpathlineto{\pgfqpoint{2.277441in}{2.638508in}}%
\pgfpathlineto{\pgfqpoint{2.280576in}{2.638508in}}%
\pgfpathlineto{\pgfqpoint{2.283710in}{2.638508in}}%
\pgfpathlineto{\pgfqpoint{2.286845in}{2.638508in}}%
\pgfpathlineto{\pgfqpoint{2.288099in}{2.640008in}}%
\pgfpathlineto{\pgfqpoint{2.289980in}{2.642259in}}%
\pgfpathlineto{\pgfqpoint{2.293115in}{2.642259in}}%
\pgfpathlineto{\pgfqpoint{2.296249in}{2.642259in}}%
\pgfpathlineto{\pgfqpoint{2.299384in}{2.642259in}}%
\pgfpathlineto{\pgfqpoint{2.300638in}{2.643760in}}%
\pgfpathlineto{\pgfqpoint{2.302519in}{2.646011in}}%
\pgfpathlineto{\pgfqpoint{2.305654in}{2.646011in}}%
\pgfpathlineto{\pgfqpoint{2.308788in}{2.646011in}}%
\pgfpathlineto{\pgfqpoint{2.311923in}{2.646011in}}%
\pgfpathlineto{\pgfqpoint{2.313177in}{2.647512in}}%
\pgfpathlineto{\pgfqpoint{2.315058in}{2.649762in}}%
\pgfpathlineto{\pgfqpoint{2.318192in}{2.649762in}}%
\pgfpathlineto{\pgfqpoint{2.321327in}{2.649762in}}%
\pgfpathlineto{\pgfqpoint{2.324462in}{2.649762in}}%
\pgfpathlineto{\pgfqpoint{2.325716in}{2.651263in}}%
\pgfpathlineto{\pgfqpoint{2.327597in}{2.653514in}}%
\pgfpathlineto{\pgfqpoint{2.330731in}{2.653514in}}%
\pgfpathlineto{\pgfqpoint{2.333866in}{2.653514in}}%
\pgfpathlineto{\pgfqpoint{2.337001in}{2.653514in}}%
\pgfpathlineto{\pgfqpoint{2.340136in}{2.653514in}}%
\pgfpathlineto{\pgfqpoint{2.341390in}{2.655015in}}%
\pgfpathlineto{\pgfqpoint{2.343270in}{2.657265in}}%
\pgfpathlineto{\pgfqpoint{2.346405in}{2.657265in}}%
\pgfpathlineto{\pgfqpoint{2.349540in}{2.657265in}}%
\pgfpathlineto{\pgfqpoint{2.352675in}{2.657265in}}%
\pgfpathlineto{\pgfqpoint{2.353929in}{2.658766in}}%
\pgfpathlineto{\pgfqpoint{2.355809in}{2.661017in}}%
\pgfpathlineto{\pgfqpoint{2.358944in}{2.661017in}}%
\pgfpathlineto{\pgfqpoint{2.362079in}{2.661017in}}%
\pgfpathlineto{\pgfqpoint{2.365214in}{2.661017in}}%
\pgfpathlineto{\pgfqpoint{2.366468in}{2.662518in}}%
\pgfpathlineto{\pgfqpoint{2.368348in}{2.664769in}}%
\pgfpathlineto{\pgfqpoint{2.371483in}{2.664769in}}%
\pgfpathlineto{\pgfqpoint{2.374618in}{2.664769in}}%
\pgfpathlineto{\pgfqpoint{2.377753in}{2.664769in}}%
\pgfpathlineto{\pgfqpoint{2.379006in}{2.666269in}}%
\pgfpathlineto{\pgfqpoint{2.380887in}{2.668520in}}%
\pgfpathlineto{\pgfqpoint{2.384022in}{2.668520in}}%
\pgfpathlineto{\pgfqpoint{2.387157in}{2.668520in}}%
\pgfpathlineto{\pgfqpoint{2.390292in}{2.668520in}}%
\pgfpathlineto{\pgfqpoint{2.393426in}{2.668520in}}%
\pgfpathlineto{\pgfqpoint{2.394680in}{2.670021in}}%
\pgfpathlineto{\pgfqpoint{2.396561in}{2.672272in}}%
\pgfpathlineto{\pgfqpoint{2.399696in}{2.672272in}}%
\pgfpathlineto{\pgfqpoint{2.402831in}{2.672272in}}%
\pgfpathlineto{\pgfqpoint{2.405965in}{2.672272in}}%
\pgfpathlineto{\pgfqpoint{2.407219in}{2.673772in}}%
\pgfpathlineto{\pgfqpoint{2.409100in}{2.676023in}}%
\pgfpathlineto{\pgfqpoint{2.412235in}{2.676023in}}%
\pgfpathlineto{\pgfqpoint{2.415369in}{2.676023in}}%
\pgfpathlineto{\pgfqpoint{2.418504in}{2.676023in}}%
\pgfpathlineto{\pgfqpoint{2.419758in}{2.677524in}}%
\pgfpathlineto{\pgfqpoint{2.421639in}{2.679775in}}%
\pgfpathlineto{\pgfqpoint{2.424774in}{2.679775in}}%
\pgfpathlineto{\pgfqpoint{2.427908in}{2.679775in}}%
\pgfpathlineto{\pgfqpoint{2.431043in}{2.679775in}}%
\pgfpathlineto{\pgfqpoint{2.432297in}{2.681275in}}%
\pgfpathlineto{\pgfqpoint{2.434178in}{2.683526in}}%
\pgfpathlineto{\pgfqpoint{2.437313in}{2.683526in}}%
\pgfpathlineto{\pgfqpoint{2.440447in}{2.683526in}}%
\pgfpathlineto{\pgfqpoint{2.443582in}{2.683526in}}%
\pgfpathlineto{\pgfqpoint{2.446717in}{2.683526in}}%
\pgfpathlineto{\pgfqpoint{2.447971in}{2.685027in}}%
\pgfpathlineto{\pgfqpoint{2.449852in}{2.687278in}}%
\pgfpathlineto{\pgfqpoint{2.452986in}{2.687278in}}%
\pgfpathlineto{\pgfqpoint{2.456121in}{2.687278in}}%
\pgfpathlineto{\pgfqpoint{2.459256in}{2.687278in}}%
\pgfpathlineto{\pgfqpoint{2.460510in}{2.688778in}}%
\pgfpathlineto{\pgfqpoint{2.462391in}{2.691029in}}%
\pgfpathlineto{\pgfqpoint{2.465525in}{2.691029in}}%
\pgfpathlineto{\pgfqpoint{2.468660in}{2.691029in}}%
\pgfpathlineto{\pgfqpoint{2.471795in}{2.691029in}}%
\pgfpathlineto{\pgfqpoint{2.473049in}{2.692530in}}%
\pgfpathlineto{\pgfqpoint{2.474930in}{2.694781in}}%
\pgfpathlineto{\pgfqpoint{2.478064in}{2.694781in}}%
\pgfpathlineto{\pgfqpoint{2.481199in}{2.694781in}}%
\pgfpathlineto{\pgfqpoint{2.484334in}{2.694781in}}%
\pgfpathlineto{\pgfqpoint{2.485588in}{2.696281in}}%
\pgfpathlineto{\pgfqpoint{2.487469in}{2.698532in}}%
\pgfpathlineto{\pgfqpoint{2.490603in}{2.698532in}}%
\pgfpathlineto{\pgfqpoint{2.493738in}{2.698532in}}%
\pgfpathlineto{\pgfqpoint{2.496873in}{2.698532in}}%
\pgfpathlineto{\pgfqpoint{2.498127in}{2.700033in}}%
\pgfpathlineto{\pgfqpoint{2.500008in}{2.702284in}}%
\pgfpathlineto{\pgfqpoint{2.503142in}{2.702284in}}%
\pgfpathlineto{\pgfqpoint{2.506277in}{2.702284in}}%
\pgfpathlineto{\pgfqpoint{2.509412in}{2.702284in}}%
\pgfpathlineto{\pgfqpoint{2.512547in}{2.702284in}}%
\pgfpathlineto{\pgfqpoint{2.513800in}{2.703785in}}%
\pgfpathlineto{\pgfqpoint{2.515681in}{2.706035in}}%
\pgfpathlineto{\pgfqpoint{2.518816in}{2.706035in}}%
\pgfpathlineto{\pgfqpoint{2.521951in}{2.706035in}}%
\pgfpathlineto{\pgfqpoint{2.525085in}{2.706035in}}%
\pgfpathlineto{\pgfqpoint{2.526339in}{2.707536in}}%
\pgfpathlineto{\pgfqpoint{2.528220in}{2.709787in}}%
\pgfpathlineto{\pgfqpoint{2.531355in}{2.709787in}}%
\pgfpathlineto{\pgfqpoint{2.534490in}{2.709787in}}%
\pgfpathlineto{\pgfqpoint{2.537624in}{2.709787in}}%
\pgfpathlineto{\pgfqpoint{2.538878in}{2.711288in}}%
\pgfpathlineto{\pgfqpoint{2.540759in}{2.713539in}}%
\pgfpathlineto{\pgfqpoint{2.543894in}{2.713539in}}%
\pgfpathlineto{\pgfqpoint{2.547029in}{2.713539in}}%
\pgfpathlineto{\pgfqpoint{2.550163in}{2.713539in}}%
\pgfpathlineto{\pgfqpoint{2.551417in}{2.715039in}}%
\pgfpathlineto{\pgfqpoint{2.553298in}{2.717290in}}%
\pgfpathlineto{\pgfqpoint{2.556433in}{2.717290in}}%
\pgfpathlineto{\pgfqpoint{2.559568in}{2.717290in}}%
\pgfpathlineto{\pgfqpoint{2.562702in}{2.717290in}}%
\pgfpathlineto{\pgfqpoint{2.565837in}{2.717290in}}%
\pgfpathlineto{\pgfqpoint{2.567091in}{2.718791in}}%
\pgfpathlineto{\pgfqpoint{2.568972in}{2.721042in}}%
\pgfpathlineto{\pgfqpoint{2.572107in}{2.721042in}}%
\pgfpathlineto{\pgfqpoint{2.575241in}{2.721042in}}%
\pgfpathlineto{\pgfqpoint{2.578376in}{2.721042in}}%
\pgfpathlineto{\pgfqpoint{2.579630in}{2.722542in}}%
\pgfpathlineto{\pgfqpoint{2.581511in}{2.724793in}}%
\pgfpathlineto{\pgfqpoint{2.584646in}{2.724793in}}%
\pgfpathlineto{\pgfqpoint{2.587780in}{2.724793in}}%
\pgfpathlineto{\pgfqpoint{2.590915in}{2.724793in}}%
\pgfpathlineto{\pgfqpoint{2.592169in}{2.726294in}}%
\pgfpathlineto{\pgfqpoint{2.594050in}{2.728545in}}%
\pgfpathlineto{\pgfqpoint{2.597185in}{2.728545in}}%
\pgfpathlineto{\pgfqpoint{2.600319in}{2.728545in}}%
\pgfpathlineto{\pgfqpoint{2.603454in}{2.728545in}}%
\pgfpathlineto{\pgfqpoint{2.604708in}{2.730045in}}%
\pgfpathlineto{\pgfqpoint{2.606589in}{2.732296in}}%
\pgfpathlineto{\pgfqpoint{2.609724in}{2.732296in}}%
\pgfpathlineto{\pgfqpoint{2.612858in}{2.732296in}}%
\pgfpathlineto{\pgfqpoint{2.615993in}{2.732296in}}%
\pgfpathlineto{\pgfqpoint{2.617247in}{2.733797in}}%
\pgfpathlineto{\pgfqpoint{2.619128in}{2.736048in}}%
\pgfpathlineto{\pgfqpoint{2.622262in}{2.736048in}}%
\pgfpathlineto{\pgfqpoint{2.625397in}{2.736048in}}%
\pgfpathlineto{\pgfqpoint{2.628532in}{2.736048in}}%
\pgfpathlineto{\pgfqpoint{2.631667in}{2.736048in}}%
\pgfpathlineto{\pgfqpoint{2.632921in}{2.737548in}}%
\pgfpathlineto{\pgfqpoint{2.634801in}{2.739799in}}%
\pgfpathlineto{\pgfqpoint{2.637936in}{2.739799in}}%
\pgfpathlineto{\pgfqpoint{2.641071in}{2.739799in}}%
\pgfpathlineto{\pgfqpoint{2.644206in}{2.739799in}}%
\pgfpathlineto{\pgfqpoint{2.645460in}{2.741300in}}%
\pgfpathlineto{\pgfqpoint{2.647340in}{2.743551in}}%
\pgfpathlineto{\pgfqpoint{2.650475in}{2.743551in}}%
\pgfpathlineto{\pgfqpoint{2.653610in}{2.743551in}}%
\pgfpathlineto{\pgfqpoint{2.656745in}{2.743551in}}%
\pgfpathlineto{\pgfqpoint{2.657999in}{2.745051in}}%
\pgfpathlineto{\pgfqpoint{2.659879in}{2.747302in}}%
\pgfpathlineto{\pgfqpoint{2.663014in}{2.747302in}}%
\pgfpathlineto{\pgfqpoint{2.666149in}{2.747302in}}%
\pgfpathlineto{\pgfqpoint{2.669284in}{2.747302in}}%
\pgfpathlineto{\pgfqpoint{2.672418in}{2.747302in}}%
\pgfpathlineto{\pgfqpoint{2.674299in}{2.745051in}}%
\pgfpathlineto{\pgfqpoint{2.675553in}{2.743551in}}%
\pgfpathlineto{\pgfqpoint{2.677434in}{2.741300in}}%
\pgfpathlineto{\pgfqpoint{2.678688in}{2.739799in}}%
\pgfpathlineto{\pgfqpoint{2.680569in}{2.737548in}}%
\pgfpathlineto{\pgfqpoint{2.681823in}{2.736048in}}%
\pgfpathlineto{\pgfqpoint{2.683703in}{2.733797in}}%
\pgfpathlineto{\pgfqpoint{2.684957in}{2.732296in}}%
\pgfpathlineto{\pgfqpoint{2.686838in}{2.730045in}}%
\pgfpathlineto{\pgfqpoint{2.688092in}{2.728545in}}%
\pgfpathlineto{\pgfqpoint{2.689973in}{2.726294in}}%
\pgfpathlineto{\pgfqpoint{2.691227in}{2.724793in}}%
\pgfpathlineto{\pgfqpoint{2.693108in}{2.722542in}}%
\pgfpathlineto{\pgfqpoint{2.694362in}{2.721042in}}%
\pgfpathlineto{\pgfqpoint{2.696242in}{2.718791in}}%
\pgfpathlineto{\pgfqpoint{2.697496in}{2.717290in}}%
\pgfpathlineto{\pgfqpoint{2.699377in}{2.715039in}}%
\pgfpathlineto{\pgfqpoint{2.700631in}{2.713539in}}%
\pgfpathlineto{\pgfqpoint{2.702512in}{2.711288in}}%
\pgfpathlineto{\pgfqpoint{2.703766in}{2.709787in}}%
\pgfpathlineto{\pgfqpoint{2.705647in}{2.707536in}}%
\pgfpathlineto{\pgfqpoint{2.706901in}{2.706035in}}%
\pgfpathlineto{\pgfqpoint{2.708781in}{2.703785in}}%
\pgfpathlineto{\pgfqpoint{2.710035in}{2.702284in}}%
\pgfpathlineto{\pgfqpoint{2.711916in}{2.700033in}}%
\pgfpathlineto{\pgfqpoint{2.711916in}{2.696281in}}%
\pgfpathlineto{\pgfqpoint{2.713170in}{2.694781in}}%
\pgfpathlineto{\pgfqpoint{2.715051in}{2.692530in}}%
\pgfpathlineto{\pgfqpoint{2.716305in}{2.691029in}}%
\pgfpathlineto{\pgfqpoint{2.718186in}{2.688778in}}%
\pgfpathlineto{\pgfqpoint{2.719439in}{2.687278in}}%
\pgfpathlineto{\pgfqpoint{2.721320in}{2.685027in}}%
\pgfpathlineto{\pgfqpoint{2.722574in}{2.683526in}}%
\pgfpathlineto{\pgfqpoint{2.724455in}{2.681275in}}%
\pgfpathlineto{\pgfqpoint{2.725709in}{2.679775in}}%
\pgfpathlineto{\pgfqpoint{2.727590in}{2.677524in}}%
\pgfpathlineto{\pgfqpoint{2.728844in}{2.676023in}}%
\pgfpathlineto{\pgfqpoint{2.730725in}{2.673772in}}%
\pgfpathlineto{\pgfqpoint{2.731978in}{2.672272in}}%
\pgfpathlineto{\pgfqpoint{2.733859in}{2.670021in}}%
\pgfpathlineto{\pgfqpoint{2.735113in}{2.668520in}}%
\pgfpathlineto{\pgfqpoint{2.736994in}{2.666269in}}%
\pgfpathlineto{\pgfqpoint{2.738248in}{2.664769in}}%
\pgfpathlineto{\pgfqpoint{2.740129in}{2.662518in}}%
\pgfpathlineto{\pgfqpoint{2.741383in}{2.661017in}}%
\pgfpathlineto{\pgfqpoint{2.743264in}{2.658766in}}%
\pgfpathlineto{\pgfqpoint{2.744517in}{2.657265in}}%
\pgfpathlineto{\pgfqpoint{2.746398in}{2.655015in}}%
\pgfpathlineto{\pgfqpoint{2.747652in}{2.653514in}}%
\pgfpathlineto{\pgfqpoint{2.749533in}{2.651263in}}%
\pgfpathlineto{\pgfqpoint{2.750787in}{2.649762in}}%
\pgfpathlineto{\pgfqpoint{2.752668in}{2.647512in}}%
\pgfpathlineto{\pgfqpoint{2.753922in}{2.646011in}}%
\pgfpathlineto{\pgfqpoint{2.755803in}{2.643760in}}%
\pgfpathlineto{\pgfqpoint{2.757056in}{2.642259in}}%
\pgfpathlineto{\pgfqpoint{2.758937in}{2.640008in}}%
\pgfpathlineto{\pgfqpoint{2.760191in}{2.638508in}}%
\pgfpathlineto{\pgfqpoint{2.762072in}{2.636257in}}%
\pgfpathlineto{\pgfqpoint{2.762072in}{2.632505in}}%
\pgfpathlineto{\pgfqpoint{2.763326in}{2.631005in}}%
\pgfpathlineto{\pgfqpoint{2.765207in}{2.628754in}}%
\pgfpathlineto{\pgfqpoint{2.766461in}{2.627253in}}%
\pgfpathlineto{\pgfqpoint{2.768341in}{2.625002in}}%
\pgfpathlineto{\pgfqpoint{2.769595in}{2.623502in}}%
\pgfpathlineto{\pgfqpoint{2.771476in}{2.621251in}}%
\pgfpathlineto{\pgfqpoint{2.772730in}{2.619750in}}%
\pgfpathlineto{\pgfqpoint{2.774611in}{2.617499in}}%
\pgfpathlineto{\pgfqpoint{2.775865in}{2.615999in}}%
\pgfpathlineto{\pgfqpoint{2.777746in}{2.613748in}}%
\pgfpathlineto{\pgfqpoint{2.779000in}{2.612247in}}%
\pgfpathlineto{\pgfqpoint{2.780880in}{2.609996in}}%
\pgfpathlineto{\pgfqpoint{2.782134in}{2.608496in}}%
\pgfpathlineto{\pgfqpoint{2.784015in}{2.606245in}}%
\pgfpathlineto{\pgfqpoint{2.785269in}{2.604744in}}%
\pgfpathlineto{\pgfqpoint{2.787150in}{2.602493in}}%
\pgfpathlineto{\pgfqpoint{2.788404in}{2.600992in}}%
\pgfpathlineto{\pgfqpoint{2.790285in}{2.598742in}}%
\pgfpathlineto{\pgfqpoint{2.791539in}{2.597241in}}%
\pgfpathlineto{\pgfqpoint{2.793419in}{2.594990in}}%
\pgfpathlineto{\pgfqpoint{2.794673in}{2.593489in}}%
\pgfpathlineto{\pgfqpoint{2.796554in}{2.591238in}}%
\pgfpathlineto{\pgfqpoint{2.797808in}{2.589738in}}%
\pgfpathlineto{\pgfqpoint{2.799689in}{2.587487in}}%
\pgfpathlineto{\pgfqpoint{2.800943in}{2.585986in}}%
\pgfpathlineto{\pgfqpoint{2.802824in}{2.583735in}}%
\pgfpathlineto{\pgfqpoint{2.804078in}{2.582235in}}%
\pgfpathlineto{\pgfqpoint{2.805958in}{2.579984in}}%
\pgfpathlineto{\pgfqpoint{2.807212in}{2.578483in}}%
\pgfpathlineto{\pgfqpoint{2.809093in}{2.576232in}}%
\pgfpathlineto{\pgfqpoint{2.809093in}{2.572481in}}%
\pgfpathlineto{\pgfqpoint{2.810347in}{2.570980in}}%
\pgfpathlineto{\pgfqpoint{2.812228in}{2.568729in}}%
\pgfpathlineto{\pgfqpoint{2.813482in}{2.567229in}}%
\pgfpathlineto{\pgfqpoint{2.815363in}{2.564978in}}%
\pgfpathlineto{\pgfqpoint{2.816617in}{2.563477in}}%
\pgfpathlineto{\pgfqpoint{2.818497in}{2.561226in}}%
\pgfpathlineto{\pgfqpoint{2.819751in}{2.559726in}}%
\pgfpathlineto{\pgfqpoint{2.821632in}{2.557475in}}%
\pgfpathlineto{\pgfqpoint{2.822886in}{2.555974in}}%
\pgfpathlineto{\pgfqpoint{2.824767in}{2.553723in}}%
\pgfpathlineto{\pgfqpoint{2.826021in}{2.552223in}}%
\pgfpathlineto{\pgfqpoint{2.827902in}{2.549972in}}%
\pgfpathlineto{\pgfqpoint{2.829155in}{2.548471in}}%
\pgfpathlineto{\pgfqpoint{2.831036in}{2.546220in}}%
\pgfpathlineto{\pgfqpoint{2.832290in}{2.544719in}}%
\pgfpathlineto{\pgfqpoint{2.834171in}{2.542469in}}%
\pgfpathlineto{\pgfqpoint{2.835425in}{2.540968in}}%
\pgfpathlineto{\pgfqpoint{2.837306in}{2.538717in}}%
\pgfpathlineto{\pgfqpoint{2.838560in}{2.537216in}}%
\pgfpathlineto{\pgfqpoint{2.840441in}{2.534965in}}%
\pgfpathlineto{\pgfqpoint{2.841694in}{2.533465in}}%
\pgfpathlineto{\pgfqpoint{2.843575in}{2.531214in}}%
\pgfpathlineto{\pgfqpoint{2.844829in}{2.529713in}}%
\pgfpathlineto{\pgfqpoint{2.846710in}{2.527462in}}%
\pgfpathlineto{\pgfqpoint{2.847964in}{2.525962in}}%
\pgfpathlineto{\pgfqpoint{2.849845in}{2.523711in}}%
\pgfpathlineto{\pgfqpoint{2.851099in}{2.522210in}}%
\pgfpathlineto{\pgfqpoint{2.852980in}{2.519959in}}%
\pgfpathlineto{\pgfqpoint{2.854233in}{2.518459in}}%
\pgfpathlineto{\pgfqpoint{2.856114in}{2.516208in}}%
\pgfpathlineto{\pgfqpoint{2.857368in}{2.514707in}}%
\pgfpathlineto{\pgfqpoint{2.859249in}{2.512456in}}%
\pgfpathlineto{\pgfqpoint{2.859249in}{2.508705in}}%
\pgfpathlineto{\pgfqpoint{2.860503in}{2.507204in}}%
\pgfpathlineto{\pgfqpoint{2.862384in}{2.504953in}}%
\pgfpathlineto{\pgfqpoint{2.863638in}{2.503453in}}%
\pgfpathlineto{\pgfqpoint{2.865518in}{2.501202in}}%
\pgfpathlineto{\pgfqpoint{2.866772in}{2.499701in}}%
\pgfpathlineto{\pgfqpoint{2.868653in}{2.497450in}}%
\pgfpathlineto{\pgfqpoint{2.869907in}{2.495950in}}%
\pgfpathlineto{\pgfqpoint{2.871788in}{2.493699in}}%
\pgfpathlineto{\pgfqpoint{2.873042in}{2.492198in}}%
\pgfpathlineto{\pgfqpoint{2.874923in}{2.489947in}}%
\pgfpathlineto{\pgfqpoint{2.876177in}{2.488446in}}%
\pgfpathlineto{\pgfqpoint{2.878057in}{2.486196in}}%
\pgfpathlineto{\pgfqpoint{2.879311in}{2.484695in}}%
\pgfpathlineto{\pgfqpoint{2.881192in}{2.482444in}}%
\pgfpathlineto{\pgfqpoint{2.882446in}{2.480943in}}%
\pgfpathlineto{\pgfqpoint{2.884327in}{2.478692in}}%
\pgfpathlineto{\pgfqpoint{2.885581in}{2.477192in}}%
\pgfpathlineto{\pgfqpoint{2.887462in}{2.474941in}}%
\pgfpathlineto{\pgfqpoint{2.888716in}{2.473440in}}%
\pgfpathlineto{\pgfqpoint{2.890596in}{2.471189in}}%
\pgfpathlineto{\pgfqpoint{2.891850in}{2.469689in}}%
\pgfpathlineto{\pgfqpoint{2.893731in}{2.467438in}}%
\pgfpathlineto{\pgfqpoint{2.894985in}{2.465937in}}%
\pgfpathlineto{\pgfqpoint{2.896866in}{2.463686in}}%
\pgfpathlineto{\pgfqpoint{2.898120in}{2.462186in}}%
\pgfpathlineto{\pgfqpoint{2.900001in}{2.459935in}}%
\pgfpathlineto{\pgfqpoint{2.901255in}{2.458434in}}%
\pgfpathlineto{\pgfqpoint{2.903135in}{2.456183in}}%
\pgfpathlineto{\pgfqpoint{2.904389in}{2.454683in}}%
\pgfpathlineto{\pgfqpoint{2.906270in}{2.452432in}}%
\pgfpathlineto{\pgfqpoint{2.906270in}{2.448680in}}%
\pgfpathlineto{\pgfqpoint{2.907524in}{2.447180in}}%
\pgfpathlineto{\pgfqpoint{2.909405in}{2.444929in}}%
\pgfpathlineto{\pgfqpoint{2.910659in}{2.443428in}}%
\pgfpathlineto{\pgfqpoint{2.912540in}{2.441177in}}%
\pgfpathlineto{\pgfqpoint{2.913794in}{2.439676in}}%
\pgfpathlineto{\pgfqpoint{2.915674in}{2.437426in}}%
\pgfpathlineto{\pgfqpoint{2.916928in}{2.435925in}}%
\pgfpathlineto{\pgfqpoint{2.918809in}{2.433674in}}%
\pgfpathlineto{\pgfqpoint{2.920063in}{2.432173in}}%
\pgfpathlineto{\pgfqpoint{2.921944in}{2.429923in}}%
\pgfpathlineto{\pgfqpoint{2.923198in}{2.428422in}}%
\pgfpathlineto{\pgfqpoint{2.925079in}{2.426171in}}%
\pgfpathlineto{\pgfqpoint{2.926332in}{2.424670in}}%
\pgfpathlineto{\pgfqpoint{2.928213in}{2.422419in}}%
\pgfpathlineto{\pgfqpoint{2.929467in}{2.420919in}}%
\pgfpathlineto{\pgfqpoint{2.931348in}{2.418668in}}%
\pgfpathlineto{\pgfqpoint{2.932602in}{2.417167in}}%
\pgfpathlineto{\pgfqpoint{2.934483in}{2.414916in}}%
\pgfpathlineto{\pgfqpoint{2.935737in}{2.413416in}}%
\pgfpathlineto{\pgfqpoint{2.937618in}{2.411165in}}%
\pgfpathlineto{\pgfqpoint{2.938871in}{2.409664in}}%
\pgfpathlineto{\pgfqpoint{2.940752in}{2.407413in}}%
\pgfpathlineto{\pgfqpoint{2.942006in}{2.405913in}}%
\pgfpathlineto{\pgfqpoint{2.943887in}{2.403662in}}%
\pgfpathlineto{\pgfqpoint{2.945141in}{2.402161in}}%
\pgfpathlineto{\pgfqpoint{2.947022in}{2.399910in}}%
\pgfpathlineto{\pgfqpoint{2.948276in}{2.398410in}}%
\pgfpathlineto{\pgfqpoint{2.950157in}{2.396159in}}%
\pgfpathlineto{\pgfqpoint{2.951410in}{2.394658in}}%
\pgfpathlineto{\pgfqpoint{2.953291in}{2.392407in}}%
\pgfpathlineto{\pgfqpoint{2.954545in}{2.390907in}}%
\pgfpathlineto{\pgfqpoint{2.956426in}{2.388656in}}%
\pgfpathlineto{\pgfqpoint{2.956426in}{2.384904in}}%
\pgfpathlineto{\pgfqpoint{2.957680in}{2.383403in}}%
\pgfpathlineto{\pgfqpoint{2.959561in}{2.381153in}}%
\pgfpathlineto{\pgfqpoint{2.960815in}{2.379652in}}%
\pgfpathlineto{\pgfqpoint{2.962696in}{2.377401in}}%
\pgfpathlineto{\pgfqpoint{2.963949in}{2.375900in}}%
\pgfpathlineto{\pgfqpoint{2.965830in}{2.373649in}}%
\pgfpathlineto{\pgfqpoint{2.967084in}{2.372149in}}%
\pgfpathlineto{\pgfqpoint{2.968965in}{2.369898in}}%
\pgfpathlineto{\pgfqpoint{2.970219in}{2.368397in}}%
\pgfpathlineto{\pgfqpoint{2.972100in}{2.366146in}}%
\pgfpathlineto{\pgfqpoint{2.973354in}{2.364646in}}%
\pgfpathlineto{\pgfqpoint{2.975234in}{2.362395in}}%
\pgfpathlineto{\pgfqpoint{2.976488in}{2.360894in}}%
\pgfpathlineto{\pgfqpoint{2.978369in}{2.358643in}}%
\pgfpathlineto{\pgfqpoint{2.979623in}{2.357143in}}%
\pgfpathlineto{\pgfqpoint{2.981504in}{2.354892in}}%
\pgfpathlineto{\pgfqpoint{2.982758in}{2.353391in}}%
\pgfpathlineto{\pgfqpoint{2.984639in}{2.351140in}}%
\pgfpathlineto{\pgfqpoint{2.985893in}{2.349640in}}%
\pgfpathlineto{\pgfqpoint{2.987773in}{2.347389in}}%
\pgfpathlineto{\pgfqpoint{2.989027in}{2.345888in}}%
\pgfpathlineto{\pgfqpoint{2.990908in}{2.343637in}}%
\pgfpathlineto{\pgfqpoint{2.992162in}{2.342137in}}%
\pgfpathlineto{\pgfqpoint{2.994043in}{2.339886in}}%
\pgfpathlineto{\pgfqpoint{2.995297in}{2.338385in}}%
\pgfpathlineto{\pgfqpoint{2.997178in}{2.336134in}}%
\pgfpathlineto{\pgfqpoint{2.998432in}{2.334634in}}%
\pgfpathlineto{\pgfqpoint{3.000312in}{2.332383in}}%
\pgfpathlineto{\pgfqpoint{3.001566in}{2.330882in}}%
\pgfpathlineto{\pgfqpoint{3.003447in}{2.328631in}}%
\pgfpathlineto{\pgfqpoint{3.003447in}{2.324880in}}%
\pgfpathlineto{\pgfqpoint{3.004701in}{2.323379in}}%
\pgfpathlineto{\pgfqpoint{3.006582in}{2.321128in}}%
\pgfpathlineto{\pgfqpoint{3.007836in}{2.319627in}}%
\pgfpathlineto{\pgfqpoint{3.009717in}{2.317376in}}%
\pgfpathlineto{\pgfqpoint{3.010971in}{2.315876in}}%
\pgfpathlineto{\pgfqpoint{3.012851in}{2.313625in}}%
\pgfpathlineto{\pgfqpoint{3.014105in}{2.312124in}}%
\pgfpathlineto{\pgfqpoint{3.015986in}{2.309873in}}%
\pgfpathlineto{\pgfqpoint{3.017240in}{2.308373in}}%
\pgfpathlineto{\pgfqpoint{3.019121in}{2.306122in}}%
\pgfpathlineto{\pgfqpoint{3.020375in}{2.304621in}}%
\pgfpathlineto{\pgfqpoint{3.022256in}{2.302370in}}%
\pgfpathlineto{\pgfqpoint{3.023510in}{2.300870in}}%
\pgfpathlineto{\pgfqpoint{3.025390in}{2.298619in}}%
\pgfpathlineto{\pgfqpoint{3.026644in}{2.297118in}}%
\pgfpathlineto{\pgfqpoint{3.028525in}{2.294867in}}%
\pgfpathlineto{\pgfqpoint{3.029779in}{2.293367in}}%
\pgfpathlineto{\pgfqpoint{3.031660in}{2.291116in}}%
\pgfpathlineto{\pgfqpoint{3.032914in}{2.289615in}}%
\pgfpathlineto{\pgfqpoint{3.034795in}{2.287364in}}%
\pgfpathlineto{\pgfqpoint{3.036048in}{2.285864in}}%
\pgfpathlineto{\pgfqpoint{3.037929in}{2.283613in}}%
\pgfpathlineto{\pgfqpoint{3.039183in}{2.282112in}}%
\pgfpathlineto{\pgfqpoint{3.041064in}{2.279861in}}%
\pgfpathlineto{\pgfqpoint{3.042318in}{2.278361in}}%
\pgfpathlineto{\pgfqpoint{3.044199in}{2.276110in}}%
\pgfpathlineto{\pgfqpoint{3.045453in}{2.274609in}}%
\pgfpathlineto{\pgfqpoint{3.047334in}{2.272358in}}%
\pgfpathlineto{\pgfqpoint{3.048587in}{2.270857in}}%
\pgfpathlineto{\pgfqpoint{3.050468in}{2.268607in}}%
\pgfpathlineto{\pgfqpoint{3.051722in}{2.267106in}}%
\pgfpathlineto{\pgfqpoint{3.053603in}{2.264855in}}%
\pgfpathlineto{\pgfqpoint{3.053603in}{2.261103in}}%
\pgfpathlineto{\pgfqpoint{3.054857in}{2.259603in}}%
\pgfpathlineto{\pgfqpoint{3.056738in}{2.257352in}}%
\pgfpathlineto{\pgfqpoint{3.057992in}{2.255851in}}%
\pgfpathlineto{\pgfqpoint{3.059873in}{2.253600in}}%
\pgfpathlineto{\pgfqpoint{3.061126in}{2.252100in}}%
\pgfpathlineto{\pgfqpoint{3.063007in}{2.249849in}}%
\pgfpathlineto{\pgfqpoint{3.064261in}{2.248348in}}%
\pgfpathlineto{\pgfqpoint{3.066142in}{2.246097in}}%
\pgfpathlineto{\pgfqpoint{3.067396in}{2.244597in}}%
\pgfpathlineto{\pgfqpoint{3.069277in}{2.242346in}}%
\pgfpathlineto{\pgfqpoint{3.070531in}{2.240845in}}%
\pgfpathlineto{\pgfqpoint{3.072411in}{2.238594in}}%
\pgfpathlineto{\pgfqpoint{3.073665in}{2.237094in}}%
\pgfpathlineto{\pgfqpoint{3.075546in}{2.234843in}}%
\pgfpathlineto{\pgfqpoint{3.076800in}{2.233342in}}%
\pgfpathlineto{\pgfqpoint{3.078681in}{2.231091in}}%
\pgfpathlineto{\pgfqpoint{3.079935in}{2.229591in}}%
\pgfpathlineto{\pgfqpoint{3.081816in}{2.227340in}}%
\pgfpathlineto{\pgfqpoint{3.083070in}{2.225839in}}%
\pgfpathlineto{\pgfqpoint{3.084950in}{2.223588in}}%
\pgfpathlineto{\pgfqpoint{3.086204in}{2.222087in}}%
\pgfpathlineto{\pgfqpoint{3.088085in}{2.219837in}}%
\pgfpathlineto{\pgfqpoint{3.089339in}{2.218336in}}%
\pgfpathlineto{\pgfqpoint{3.091220in}{2.216085in}}%
\pgfpathlineto{\pgfqpoint{3.092474in}{2.214584in}}%
\pgfpathlineto{\pgfqpoint{3.094355in}{2.212334in}}%
\pgfpathlineto{\pgfqpoint{3.094355in}{2.208582in}}%
\pgfpathlineto{\pgfqpoint{3.092474in}{2.206331in}}%
\pgfpathlineto{\pgfqpoint{3.091220in}{2.204830in}}%
\pgfpathlineto{\pgfqpoint{3.091220in}{2.201079in}}%
\pgfpathlineto{\pgfqpoint{3.091220in}{2.197327in}}%
\pgfpathlineto{\pgfqpoint{3.089339in}{2.195076in}}%
\pgfpathlineto{\pgfqpoint{3.088085in}{2.193576in}}%
\pgfpathlineto{\pgfqpoint{3.088085in}{2.189824in}}%
\pgfpathlineto{\pgfqpoint{3.088085in}{2.186073in}}%
\pgfpathlineto{\pgfqpoint{3.086204in}{2.183822in}}%
\pgfpathlineto{\pgfqpoint{3.084950in}{2.182321in}}%
\pgfpathlineto{\pgfqpoint{3.084950in}{2.178570in}}%
\pgfpathlineto{\pgfqpoint{3.084950in}{2.174818in}}%
\pgfpathlineto{\pgfqpoint{3.083070in}{2.172567in}}%
\pgfpathlineto{\pgfqpoint{3.081816in}{2.171067in}}%
\pgfpathlineto{\pgfqpoint{3.081816in}{2.167315in}}%
\pgfpathlineto{\pgfqpoint{3.081816in}{2.163564in}}%
\pgfpathlineto{\pgfqpoint{3.079935in}{2.161313in}}%
\pgfpathlineto{\pgfqpoint{3.078681in}{2.159812in}}%
\pgfpathlineto{\pgfqpoint{3.078681in}{2.156060in}}%
\pgfpathlineto{\pgfqpoint{3.078681in}{2.152309in}}%
\pgfpathlineto{\pgfqpoint{3.078681in}{2.148557in}}%
\pgfpathlineto{\pgfqpoint{3.076800in}{2.146306in}}%
\pgfpathlineto{\pgfqpoint{3.075546in}{2.144806in}}%
\pgfpathlineto{\pgfqpoint{3.075546in}{2.141054in}}%
\pgfpathlineto{\pgfqpoint{3.075546in}{2.137303in}}%
\pgfpathlineto{\pgfqpoint{3.073665in}{2.135052in}}%
\pgfpathlineto{\pgfqpoint{3.072411in}{2.133551in}}%
\pgfpathlineto{\pgfqpoint{3.072411in}{2.129800in}}%
\pgfpathlineto{\pgfqpoint{3.072411in}{2.126048in}}%
\pgfpathlineto{\pgfqpoint{3.070531in}{2.123797in}}%
\pgfpathlineto{\pgfqpoint{3.069277in}{2.122297in}}%
\pgfpathlineto{\pgfqpoint{3.069277in}{2.118545in}}%
\pgfpathlineto{\pgfqpoint{3.069277in}{2.114794in}}%
\pgfpathlineto{\pgfqpoint{3.067396in}{2.112543in}}%
\pgfpathlineto{\pgfqpoint{3.066142in}{2.111042in}}%
\pgfpathlineto{\pgfqpoint{3.066142in}{2.107291in}}%
\pgfpathlineto{\pgfqpoint{3.066142in}{2.103539in}}%
\pgfpathlineto{\pgfqpoint{3.064261in}{2.101288in}}%
\pgfpathlineto{\pgfqpoint{3.063007in}{2.099787in}}%
\pgfpathlineto{\pgfqpoint{3.063007in}{2.096036in}}%
\pgfpathlineto{\pgfqpoint{3.063007in}{2.092284in}}%
\pgfpathlineto{\pgfqpoint{3.061126in}{2.090033in}}%
\pgfpathlineto{\pgfqpoint{3.059873in}{2.088533in}}%
\pgfpathlineto{\pgfqpoint{3.059873in}{2.084781in}}%
\pgfpathlineto{\pgfqpoint{3.059873in}{2.081030in}}%
\pgfpathlineto{\pgfqpoint{3.057992in}{2.078779in}}%
\pgfpathlineto{\pgfqpoint{3.056738in}{2.077278in}}%
\pgfpathlineto{\pgfqpoint{3.056738in}{2.073527in}}%
\pgfpathlineto{\pgfqpoint{3.056738in}{2.069775in}}%
\pgfpathlineto{\pgfqpoint{3.054857in}{2.067524in}}%
\pgfpathlineto{\pgfqpoint{3.053603in}{2.066024in}}%
\pgfpathlineto{\pgfqpoint{3.053603in}{2.062272in}}%
\pgfpathlineto{\pgfqpoint{3.053603in}{2.058521in}}%
\pgfpathlineto{\pgfqpoint{3.051722in}{2.056270in}}%
\pgfpathlineto{\pgfqpoint{3.050468in}{2.054769in}}%
\pgfpathlineto{\pgfqpoint{3.050468in}{2.051018in}}%
\pgfpathlineto{\pgfqpoint{3.050468in}{2.047266in}}%
\pgfpathlineto{\pgfqpoint{3.050468in}{2.043514in}}%
\pgfpathlineto{\pgfqpoint{3.048587in}{2.041264in}}%
\pgfpathlineto{\pgfqpoint{3.047334in}{2.039763in}}%
\pgfpathlineto{\pgfqpoint{3.047334in}{2.036011in}}%
\pgfpathlineto{\pgfqpoint{3.047334in}{2.032260in}}%
\pgfpathlineto{\pgfqpoint{3.045453in}{2.030009in}}%
\pgfpathlineto{\pgfqpoint{3.044199in}{2.028508in}}%
\pgfpathlineto{\pgfqpoint{3.044199in}{2.024757in}}%
\pgfpathlineto{\pgfqpoint{3.044199in}{2.021005in}}%
\pgfpathlineto{\pgfqpoint{3.042318in}{2.018754in}}%
\pgfpathlineto{\pgfqpoint{3.041064in}{2.017254in}}%
\pgfpathlineto{\pgfqpoint{3.041064in}{2.013502in}}%
\pgfpathlineto{\pgfqpoint{3.041064in}{2.009751in}}%
\pgfpathlineto{\pgfqpoint{3.039183in}{2.007500in}}%
\pgfpathlineto{\pgfqpoint{3.037929in}{2.005999in}}%
\pgfpathlineto{\pgfqpoint{3.037929in}{2.002248in}}%
\pgfpathlineto{\pgfqpoint{3.037929in}{1.998496in}}%
\pgfpathlineto{\pgfqpoint{3.036048in}{1.996245in}}%
\pgfpathlineto{\pgfqpoint{3.034795in}{1.994745in}}%
\pgfpathlineto{\pgfqpoint{3.034795in}{1.990993in}}%
\pgfpathlineto{\pgfqpoint{3.034795in}{1.987241in}}%
\pgfpathlineto{\pgfqpoint{3.032914in}{1.984991in}}%
\pgfpathlineto{\pgfqpoint{3.031660in}{1.983490in}}%
\pgfpathlineto{\pgfqpoint{3.031660in}{1.979738in}}%
\pgfpathlineto{\pgfqpoint{3.031660in}{1.975987in}}%
\pgfpathlineto{\pgfqpoint{3.029779in}{1.973736in}}%
\pgfpathlineto{\pgfqpoint{3.028525in}{1.972235in}}%
\pgfpathlineto{\pgfqpoint{3.028525in}{1.968484in}}%
\pgfpathlineto{\pgfqpoint{3.028525in}{1.964732in}}%
\pgfpathlineto{\pgfqpoint{3.026644in}{1.962481in}}%
\pgfpathlineto{\pgfqpoint{3.025390in}{1.960981in}}%
\pgfpathlineto{\pgfqpoint{3.025390in}{1.957229in}}%
\pgfpathlineto{\pgfqpoint{3.025390in}{1.953478in}}%
\pgfpathlineto{\pgfqpoint{3.025390in}{1.949726in}}%
\pgfpathlineto{\pgfqpoint{3.023510in}{1.947475in}}%
\pgfpathlineto{\pgfqpoint{3.022256in}{1.945975in}}%
\pgfpathlineto{\pgfqpoint{3.022256in}{1.942223in}}%
\pgfpathlineto{\pgfqpoint{3.022256in}{1.938471in}}%
\pgfpathlineto{\pgfqpoint{3.020375in}{1.936221in}}%
\pgfpathlineto{\pgfqpoint{3.019121in}{1.934720in}}%
\pgfpathlineto{\pgfqpoint{3.019121in}{1.930968in}}%
\pgfpathlineto{\pgfqpoint{3.019121in}{1.927217in}}%
\pgfpathlineto{\pgfqpoint{3.017240in}{1.924966in}}%
\pgfpathlineto{\pgfqpoint{3.015986in}{1.923465in}}%
\pgfpathlineto{\pgfqpoint{3.015986in}{1.919714in}}%
\pgfpathlineto{\pgfqpoint{3.015986in}{1.915962in}}%
\pgfpathlineto{\pgfqpoint{3.014105in}{1.913711in}}%
\pgfpathlineto{\pgfqpoint{3.012851in}{1.912211in}}%
\pgfpathlineto{\pgfqpoint{3.012851in}{1.908459in}}%
\pgfpathlineto{\pgfqpoint{3.012851in}{1.904708in}}%
\pgfpathlineto{\pgfqpoint{3.010971in}{1.902457in}}%
\pgfpathlineto{\pgfqpoint{3.009717in}{1.900956in}}%
\pgfpathlineto{\pgfqpoint{3.009717in}{1.897205in}}%
\pgfpathlineto{\pgfqpoint{3.009717in}{1.893453in}}%
\pgfpathlineto{\pgfqpoint{3.007836in}{1.891202in}}%
\pgfpathlineto{\pgfqpoint{3.006582in}{1.889702in}}%
\pgfpathlineto{\pgfqpoint{3.006582in}{1.885950in}}%
\pgfpathlineto{\pgfqpoint{3.006582in}{1.882198in}}%
\pgfpathlineto{\pgfqpoint{3.004701in}{1.879948in}}%
\pgfpathlineto{\pgfqpoint{3.003447in}{1.878447in}}%
\pgfpathlineto{\pgfqpoint{3.003447in}{1.874695in}}%
\pgfpathlineto{\pgfqpoint{3.003447in}{1.870944in}}%
\pgfpathlineto{\pgfqpoint{3.001566in}{1.868693in}}%
\pgfpathlineto{\pgfqpoint{3.000312in}{1.867192in}}%
\pgfpathlineto{\pgfqpoint{3.000312in}{1.863441in}}%
\pgfpathlineto{\pgfqpoint{3.000312in}{1.859689in}}%
\pgfpathlineto{\pgfqpoint{2.998432in}{1.857438in}}%
\pgfpathlineto{\pgfqpoint{2.997178in}{1.855938in}}%
\pgfpathlineto{\pgfqpoint{2.997178in}{1.852186in}}%
\pgfpathlineto{\pgfqpoint{2.997178in}{1.848435in}}%
\pgfpathlineto{\pgfqpoint{2.997178in}{1.844683in}}%
\pgfpathlineto{\pgfqpoint{2.995297in}{1.842432in}}%
\pgfpathlineto{\pgfqpoint{2.994043in}{1.840932in}}%
\pgfpathlineto{\pgfqpoint{2.994043in}{1.837180in}}%
\pgfpathlineto{\pgfqpoint{2.994043in}{1.833429in}}%
\pgfpathlineto{\pgfqpoint{2.992162in}{1.831178in}}%
\pgfpathlineto{\pgfqpoint{2.990908in}{1.829677in}}%
\pgfpathlineto{\pgfqpoint{2.990908in}{1.825925in}}%
\pgfpathlineto{\pgfqpoint{2.990908in}{1.822174in}}%
\pgfpathlineto{\pgfqpoint{2.989027in}{1.819923in}}%
\pgfpathlineto{\pgfqpoint{2.987773in}{1.818422in}}%
\pgfpathlineto{\pgfqpoint{2.987773in}{1.814671in}}%
\pgfpathlineto{\pgfqpoint{2.987773in}{1.810919in}}%
\pgfpathlineto{\pgfqpoint{2.985893in}{1.808668in}}%
\pgfpathlineto{\pgfqpoint{2.984639in}{1.807168in}}%
\pgfpathlineto{\pgfqpoint{2.984639in}{1.803416in}}%
\pgfpathlineto{\pgfqpoint{2.984639in}{1.799665in}}%
\pgfpathlineto{\pgfqpoint{2.982758in}{1.797414in}}%
\pgfpathlineto{\pgfqpoint{2.981504in}{1.795913in}}%
\pgfpathlineto{\pgfqpoint{2.981504in}{1.792162in}}%
\pgfpathlineto{\pgfqpoint{2.981504in}{1.788410in}}%
\pgfpathlineto{\pgfqpoint{2.979623in}{1.786159in}}%
\pgfpathlineto{\pgfqpoint{2.978369in}{1.784659in}}%
\pgfpathlineto{\pgfqpoint{2.978369in}{1.780907in}}%
\pgfpathlineto{\pgfqpoint{2.978369in}{1.777155in}}%
\pgfpathlineto{\pgfqpoint{2.976488in}{1.774905in}}%
\pgfpathlineto{\pgfqpoint{2.975234in}{1.773404in}}%
\pgfpathlineto{\pgfqpoint{2.975234in}{1.769652in}}%
\pgfpathlineto{\pgfqpoint{2.975234in}{1.765901in}}%
\pgfpathlineto{\pgfqpoint{2.973354in}{1.763650in}}%
\pgfpathlineto{\pgfqpoint{2.972100in}{1.762149in}}%
\pgfpathlineto{\pgfqpoint{2.972100in}{1.758398in}}%
\pgfpathlineto{\pgfqpoint{2.972100in}{1.754646in}}%
\pgfpathlineto{\pgfqpoint{2.972100in}{1.750895in}}%
\pgfpathlineto{\pgfqpoint{2.970219in}{1.748644in}}%
\pgfpathlineto{\pgfqpoint{2.968965in}{1.747143in}}%
\pgfpathlineto{\pgfqpoint{2.968965in}{1.743392in}}%
\pgfpathlineto{\pgfqpoint{2.968965in}{1.739640in}}%
\pgfpathlineto{\pgfqpoint{2.967084in}{1.737389in}}%
\pgfpathlineto{\pgfqpoint{2.965830in}{1.735889in}}%
\pgfpathlineto{\pgfqpoint{2.965830in}{1.732137in}}%
\pgfpathlineto{\pgfqpoint{2.965830in}{1.728386in}}%
\pgfpathlineto{\pgfqpoint{2.963949in}{1.726135in}}%
\pgfpathlineto{\pgfqpoint{2.962696in}{1.724634in}}%
\pgfpathlineto{\pgfqpoint{2.962696in}{1.720882in}}%
\pgfpathlineto{\pgfqpoint{2.962696in}{1.717131in}}%
\pgfpathlineto{\pgfqpoint{2.960815in}{1.714880in}}%
\pgfpathlineto{\pgfqpoint{2.959561in}{1.713379in}}%
\pgfpathlineto{\pgfqpoint{2.959561in}{1.709628in}}%
\pgfpathlineto{\pgfqpoint{2.959561in}{1.705876in}}%
\pgfpathlineto{\pgfqpoint{2.957680in}{1.703625in}}%
\pgfpathlineto{\pgfqpoint{2.956426in}{1.702125in}}%
\pgfpathlineto{\pgfqpoint{2.956426in}{1.698373in}}%
\pgfpathlineto{\pgfqpoint{2.956426in}{1.694622in}}%
\pgfpathlineto{\pgfqpoint{2.954545in}{1.692371in}}%
\pgfpathlineto{\pgfqpoint{2.953291in}{1.690870in}}%
\pgfpathlineto{\pgfqpoint{2.953291in}{1.687119in}}%
\pgfpathlineto{\pgfqpoint{2.953291in}{1.683367in}}%
\pgfpathlineto{\pgfqpoint{2.951410in}{1.681116in}}%
\pgfpathlineto{\pgfqpoint{2.950157in}{1.679616in}}%
\pgfpathlineto{\pgfqpoint{2.950157in}{1.675864in}}%
\pgfpathlineto{\pgfqpoint{2.950157in}{1.672113in}}%
\pgfpathlineto{\pgfqpoint{2.948276in}{1.669862in}}%
\pgfpathlineto{\pgfqpoint{2.947022in}{1.668361in}}%
\pgfpathlineto{\pgfqpoint{2.947022in}{1.664609in}}%
\pgfpathlineto{\pgfqpoint{2.947022in}{1.660858in}}%
\pgfpathlineto{\pgfqpoint{2.945141in}{1.658607in}}%
\pgfpathlineto{\pgfqpoint{2.943887in}{1.657106in}}%
\pgfpathlineto{\pgfqpoint{2.943887in}{1.653355in}}%
\pgfpathlineto{\pgfqpoint{2.943887in}{1.649603in}}%
\pgfpathlineto{\pgfqpoint{2.943887in}{1.645852in}}%
\pgfpathlineto{\pgfqpoint{2.942006in}{1.643601in}}%
\pgfpathlineto{\pgfqpoint{2.940752in}{1.642100in}}%
\pgfpathlineto{\pgfqpoint{2.940752in}{1.638349in}}%
\pgfpathlineto{\pgfqpoint{2.940752in}{1.634597in}}%
\pgfpathlineto{\pgfqpoint{2.938871in}{1.632346in}}%
\pgfpathlineto{\pgfqpoint{2.937618in}{1.630846in}}%
\pgfpathlineto{\pgfqpoint{2.937618in}{1.627094in}}%
\pgfpathlineto{\pgfqpoint{2.937618in}{1.623343in}}%
\pgfpathlineto{\pgfqpoint{2.935737in}{1.621092in}}%
\pgfpathlineto{\pgfqpoint{2.934483in}{1.619591in}}%
\pgfpathlineto{\pgfqpoint{2.934483in}{1.615840in}}%
\pgfpathlineto{\pgfqpoint{2.934483in}{1.612088in}}%
\pgfpathlineto{\pgfqpoint{2.932602in}{1.609837in}}%
\pgfpathlineto{\pgfqpoint{2.931348in}{1.608336in}}%
\pgfpathlineto{\pgfqpoint{2.931348in}{1.604585in}}%
\pgfpathlineto{\pgfqpoint{2.931348in}{1.600833in}}%
\pgfpathlineto{\pgfqpoint{2.929467in}{1.598582in}}%
\pgfpathlineto{\pgfqpoint{2.928213in}{1.597082in}}%
\pgfpathlineto{\pgfqpoint{2.928213in}{1.593330in}}%
\pgfpathlineto{\pgfqpoint{2.928213in}{1.589579in}}%
\pgfpathlineto{\pgfqpoint{2.926332in}{1.587328in}}%
\pgfpathlineto{\pgfqpoint{2.925079in}{1.585827in}}%
\pgfpathlineto{\pgfqpoint{2.925079in}{1.582076in}}%
\pgfpathlineto{\pgfqpoint{2.925079in}{1.578324in}}%
\pgfpathlineto{\pgfqpoint{2.923198in}{1.576073in}}%
\pgfpathlineto{\pgfqpoint{2.921944in}{1.574573in}}%
\pgfpathlineto{\pgfqpoint{2.921944in}{1.570821in}}%
\pgfpathlineto{\pgfqpoint{2.921944in}{1.567070in}}%
\pgfpathlineto{\pgfqpoint{2.920063in}{1.564819in}}%
\pgfpathlineto{\pgfqpoint{2.918809in}{1.563318in}}%
\pgfpathlineto{\pgfqpoint{2.918809in}{1.559566in}}%
\pgfpathlineto{\pgfqpoint{2.918809in}{1.555815in}}%
\pgfpathlineto{\pgfqpoint{2.918809in}{1.552063in}}%
\pgfpathlineto{\pgfqpoint{2.916928in}{1.549813in}}%
\pgfpathlineto{\pgfqpoint{2.915674in}{1.548312in}}%
\pgfpathlineto{\pgfqpoint{2.915674in}{1.544560in}}%
\pgfpathlineto{\pgfqpoint{2.915674in}{1.540809in}}%
\pgfpathlineto{\pgfqpoint{2.913794in}{1.538558in}}%
\pgfpathlineto{\pgfqpoint{2.912540in}{1.537057in}}%
\pgfpathlineto{\pgfqpoint{2.912540in}{1.533306in}}%
\pgfpathlineto{\pgfqpoint{2.912540in}{1.529554in}}%
\pgfpathlineto{\pgfqpoint{2.910659in}{1.527303in}}%
\pgfpathlineto{\pgfqpoint{2.909405in}{1.525803in}}%
\pgfpathlineto{\pgfqpoint{2.909405in}{1.522051in}}%
\pgfpathlineto{\pgfqpoint{2.909405in}{1.518300in}}%
\pgfpathlineto{\pgfqpoint{2.907524in}{1.516049in}}%
\pgfpathlineto{\pgfqpoint{2.906270in}{1.514548in}}%
\pgfpathlineto{\pgfqpoint{2.906270in}{1.510797in}}%
\pgfpathlineto{\pgfqpoint{2.906270in}{1.507045in}}%
\pgfpathlineto{\pgfqpoint{2.904389in}{1.504794in}}%
\pgfpathlineto{\pgfqpoint{2.903135in}{1.503293in}}%
\pgfpathlineto{\pgfqpoint{2.903135in}{1.499542in}}%
\pgfpathlineto{\pgfqpoint{2.903135in}{1.495790in}}%
\pgfpathlineto{\pgfqpoint{2.901255in}{1.493539in}}%
\pgfpathlineto{\pgfqpoint{2.900001in}{1.492039in}}%
\pgfpathlineto{\pgfqpoint{2.900001in}{1.488287in}}%
\pgfpathlineto{\pgfqpoint{2.900001in}{1.484536in}}%
\pgfpathlineto{\pgfqpoint{2.898120in}{1.482285in}}%
\pgfpathlineto{\pgfqpoint{2.896866in}{1.480784in}}%
\pgfpathlineto{\pgfqpoint{2.896866in}{1.477033in}}%
\pgfpathlineto{\pgfqpoint{2.896866in}{1.473281in}}%
\pgfpathlineto{\pgfqpoint{2.894985in}{1.471030in}}%
\pgfpathlineto{\pgfqpoint{2.893731in}{1.469530in}}%
\pgfpathlineto{\pgfqpoint{2.893731in}{1.465778in}}%
\pgfpathlineto{\pgfqpoint{2.893731in}{1.462027in}}%
\pgfpathlineto{\pgfqpoint{2.891850in}{1.459776in}}%
\pgfpathlineto{\pgfqpoint{2.890596in}{1.458275in}}%
\pgfpathlineto{\pgfqpoint{2.890596in}{1.454524in}}%
\pgfpathlineto{\pgfqpoint{2.890596in}{1.450772in}}%
\pgfpathlineto{\pgfqpoint{2.890596in}{1.447020in}}%
\pgfpathlineto{\pgfqpoint{2.888716in}{1.444770in}}%
\pgfpathlineto{\pgfqpoint{2.887462in}{1.443269in}}%
\pgfpathlineto{\pgfqpoint{2.887462in}{1.439517in}}%
\pgfpathlineto{\pgfqpoint{2.887462in}{1.435766in}}%
\pgfpathlineto{\pgfqpoint{2.885581in}{1.433515in}}%
\pgfpathlineto{\pgfqpoint{2.884327in}{1.432014in}}%
\pgfpathlineto{\pgfqpoint{2.884327in}{1.428263in}}%
\pgfpathlineto{\pgfqpoint{2.884327in}{1.424511in}}%
\pgfpathlineto{\pgfqpoint{2.882446in}{1.422260in}}%
\pgfpathlineto{\pgfqpoint{2.881192in}{1.420760in}}%
\pgfpathlineto{\pgfqpoint{2.881192in}{1.417008in}}%
\pgfpathlineto{\pgfqpoint{2.881192in}{1.413257in}}%
\pgfpathlineto{\pgfqpoint{2.879311in}{1.411006in}}%
\pgfpathlineto{\pgfqpoint{2.878057in}{1.409505in}}%
\pgfpathlineto{\pgfqpoint{2.878057in}{1.405754in}}%
\pgfpathlineto{\pgfqpoint{2.878057in}{1.402002in}}%
\pgfpathlineto{\pgfqpoint{2.876177in}{1.399751in}}%
\pgfpathlineto{\pgfqpoint{2.874923in}{1.398251in}}%
\pgfpathlineto{\pgfqpoint{2.874923in}{1.394499in}}%
\pgfpathlineto{\pgfqpoint{2.874923in}{1.390747in}}%
\pgfpathlineto{\pgfqpoint{2.873042in}{1.388497in}}%
\pgfpathlineto{\pgfqpoint{2.871788in}{1.386996in}}%
\pgfpathlineto{\pgfqpoint{2.871788in}{1.383244in}}%
\pgfpathlineto{\pgfqpoint{2.871788in}{1.379493in}}%
\pgfpathlineto{\pgfqpoint{2.869907in}{1.377242in}}%
\pgfpathlineto{\pgfqpoint{2.868653in}{1.375741in}}%
\pgfpathlineto{\pgfqpoint{2.868653in}{1.371990in}}%
\pgfpathlineto{\pgfqpoint{2.868653in}{1.368238in}}%
\pgfpathlineto{\pgfqpoint{2.866772in}{1.365987in}}%
\pgfpathlineto{\pgfqpoint{2.865518in}{1.364487in}}%
\pgfpathlineto{\pgfqpoint{2.865518in}{1.360735in}}%
\pgfpathlineto{\pgfqpoint{2.865518in}{1.356984in}}%
\pgfpathlineto{\pgfqpoint{2.865518in}{1.353232in}}%
\pgfpathlineto{\pgfqpoint{2.863638in}{1.350981in}}%
\pgfpathlineto{\pgfqpoint{2.862384in}{1.349481in}}%
\pgfpathlineto{\pgfqpoint{2.862384in}{1.345729in}}%
\pgfpathlineto{\pgfqpoint{2.862384in}{1.341977in}}%
\pgfpathlineto{\pgfqpoint{2.860503in}{1.339727in}}%
\pgfpathlineto{\pgfqpoint{2.859249in}{1.338226in}}%
\pgfpathlineto{\pgfqpoint{2.859249in}{1.334474in}}%
\pgfpathlineto{\pgfqpoint{2.859249in}{1.330723in}}%
\pgfpathlineto{\pgfqpoint{2.857368in}{1.328472in}}%
\pgfpathlineto{\pgfqpoint{2.856114in}{1.326971in}}%
\pgfpathlineto{\pgfqpoint{2.856114in}{1.323220in}}%
\pgfpathlineto{\pgfqpoint{2.856114in}{1.319468in}}%
\pgfpathlineto{\pgfqpoint{2.854233in}{1.317217in}}%
\pgfpathlineto{\pgfqpoint{2.852980in}{1.315717in}}%
\pgfpathlineto{\pgfqpoint{2.852980in}{1.311965in}}%
\pgfpathlineto{\pgfqpoint{2.852980in}{1.308214in}}%
\pgfpathlineto{\pgfqpoint{2.851099in}{1.305963in}}%
\pgfpathlineto{\pgfqpoint{2.849845in}{1.304462in}}%
\pgfpathlineto{\pgfqpoint{2.849845in}{1.300711in}}%
\pgfpathlineto{\pgfqpoint{2.849845in}{1.296959in}}%
\pgfpathlineto{\pgfqpoint{2.847964in}{1.294708in}}%
\pgfpathlineto{\pgfqpoint{2.846710in}{1.293208in}}%
\pgfpathlineto{\pgfqpoint{2.846710in}{1.289456in}}%
\pgfpathlineto{\pgfqpoint{2.846710in}{1.285704in}}%
\pgfpathlineto{\pgfqpoint{2.844829in}{1.283454in}}%
\pgfpathlineto{\pgfqpoint{2.843575in}{1.281953in}}%
\pgfpathlineto{\pgfqpoint{2.843575in}{1.278201in}}%
\pgfpathlineto{\pgfqpoint{2.843575in}{1.274450in}}%
\pgfpathlineto{\pgfqpoint{2.841694in}{1.272199in}}%
\pgfpathlineto{\pgfqpoint{2.840441in}{1.270698in}}%
\pgfpathlineto{\pgfqpoint{2.840441in}{1.266947in}}%
\pgfpathlineto{\pgfqpoint{2.840441in}{1.263195in}}%
\pgfpathlineto{\pgfqpoint{2.838560in}{1.260944in}}%
\pgfpathlineto{\pgfqpoint{2.837306in}{1.259444in}}%
\pgfpathlineto{\pgfqpoint{2.837306in}{1.255692in}}%
\pgfpathlineto{\pgfqpoint{2.837306in}{1.251941in}}%
\pgfpathlineto{\pgfqpoint{2.837306in}{1.248189in}}%
\pgfpathlineto{\pgfqpoint{2.835425in}{1.245938in}}%
\pgfpathlineto{\pgfqpoint{2.834171in}{1.244438in}}%
\pgfpathlineto{\pgfqpoint{2.834171in}{1.240686in}}%
\pgfpathlineto{\pgfqpoint{2.834171in}{1.236935in}}%
\pgfpathlineto{\pgfqpoint{2.832290in}{1.234684in}}%
\pgfpathlineto{\pgfqpoint{2.831036in}{1.233183in}}%
\pgfpathlineto{\pgfqpoint{2.831036in}{1.229431in}}%
\pgfpathlineto{\pgfqpoint{2.831036in}{1.225680in}}%
\pgfpathlineto{\pgfqpoint{2.829155in}{1.223429in}}%
\pgfpathlineto{\pgfqpoint{2.827902in}{1.221928in}}%
\pgfpathlineto{\pgfqpoint{2.827902in}{1.218177in}}%
\pgfpathlineto{\pgfqpoint{2.827902in}{1.214425in}}%
\pgfpathlineto{\pgfqpoint{2.826021in}{1.212174in}}%
\pgfpathlineto{\pgfqpoint{2.824767in}{1.210674in}}%
\pgfpathlineto{\pgfqpoint{2.824767in}{1.206922in}}%
\pgfpathlineto{\pgfqpoint{2.824767in}{1.203171in}}%
\pgfpathlineto{\pgfqpoint{2.822886in}{1.200920in}}%
\pgfpathlineto{\pgfqpoint{2.821632in}{1.199419in}}%
\pgfpathlineto{\pgfqpoint{2.821632in}{1.195668in}}%
\pgfpathlineto{\pgfqpoint{2.821632in}{1.191916in}}%
\pgfpathlineto{\pgfqpoint{2.819751in}{1.189665in}}%
\pgfpathlineto{\pgfqpoint{2.818497in}{1.188165in}}%
\pgfpathlineto{\pgfqpoint{2.818497in}{1.184413in}}%
\pgfpathlineto{\pgfqpoint{2.818497in}{1.180662in}}%
\pgfpathlineto{\pgfqpoint{2.816617in}{1.178411in}}%
\pgfpathlineto{\pgfqpoint{2.815363in}{1.176910in}}%
\pgfpathlineto{\pgfqpoint{2.815363in}{1.173158in}}%
\pgfpathlineto{\pgfqpoint{2.813482in}{1.170908in}}%
\pgfpathlineto{\pgfqpoint{2.810347in}{1.170908in}}%
\pgfpathlineto{\pgfqpoint{2.807212in}{1.170908in}}%
\pgfpathlineto{\pgfqpoint{2.805958in}{1.169407in}}%
\pgfpathlineto{\pgfqpoint{2.804078in}{1.167156in}}%
\pgfpathlineto{\pgfqpoint{2.800943in}{1.167156in}}%
\pgfpathlineto{\pgfqpoint{2.797808in}{1.167156in}}%
\pgfpathlineto{\pgfqpoint{2.796554in}{1.165655in}}%
\pgfpathlineto{\pgfqpoint{2.794673in}{1.163404in}}%
\pgfpathlineto{\pgfqpoint{2.791539in}{1.163404in}}%
\pgfpathlineto{\pgfqpoint{2.790285in}{1.161904in}}%
\pgfpathlineto{\pgfqpoint{2.788404in}{1.159653in}}%
\pgfpathlineto{\pgfqpoint{2.785269in}{1.159653in}}%
\pgfpathlineto{\pgfqpoint{2.782134in}{1.159653in}}%
\pgfpathlineto{\pgfqpoint{2.780880in}{1.158152in}}%
\pgfpathlineto{\pgfqpoint{2.779000in}{1.155901in}}%
\pgfpathlineto{\pgfqpoint{2.775865in}{1.155901in}}%
\pgfpathlineto{\pgfqpoint{2.772730in}{1.155901in}}%
\pgfpathlineto{\pgfqpoint{2.771476in}{1.154401in}}%
\pgfpathlineto{\pgfqpoint{2.769595in}{1.152150in}}%
\pgfpathlineto{\pgfqpoint{2.766461in}{1.152150in}}%
\pgfpathlineto{\pgfqpoint{2.763326in}{1.152150in}}%
\pgfpathlineto{\pgfqpoint{2.762072in}{1.150649in}}%
\pgfpathlineto{\pgfqpoint{2.760191in}{1.148398in}}%
\pgfpathlineto{\pgfqpoint{2.757056in}{1.148398in}}%
\pgfpathlineto{\pgfqpoint{2.755803in}{1.146898in}}%
\pgfpathlineto{\pgfqpoint{2.753922in}{1.144647in}}%
\pgfpathlineto{\pgfqpoint{2.750787in}{1.144647in}}%
\pgfpathlineto{\pgfqpoint{2.747652in}{1.144647in}}%
\pgfpathlineto{\pgfqpoint{2.746398in}{1.143146in}}%
\pgfpathlineto{\pgfqpoint{2.744517in}{1.140895in}}%
\pgfpathlineto{\pgfqpoint{2.741383in}{1.140895in}}%
\pgfpathlineto{\pgfqpoint{2.738248in}{1.140895in}}%
\pgfpathlineto{\pgfqpoint{2.736994in}{1.139395in}}%
\pgfpathlineto{\pgfqpoint{2.735113in}{1.137144in}}%
\pgfpathlineto{\pgfqpoint{2.731978in}{1.137144in}}%
\pgfpathlineto{\pgfqpoint{2.728844in}{1.137144in}}%
\pgfpathlineto{\pgfqpoint{2.727590in}{1.135643in}}%
\pgfpathlineto{\pgfqpoint{2.725709in}{1.133392in}}%
\pgfpathlineto{\pgfqpoint{2.722574in}{1.133392in}}%
\pgfpathlineto{\pgfqpoint{2.719439in}{1.133392in}}%
\pgfpathlineto{\pgfqpoint{2.718186in}{1.131892in}}%
\pgfpathlineto{\pgfqpoint{2.716305in}{1.129641in}}%
\pgfpathlineto{\pgfqpoint{2.713170in}{1.129641in}}%
\pgfpathlineto{\pgfqpoint{2.711916in}{1.128140in}}%
\pgfpathlineto{\pgfqpoint{2.710035in}{1.125889in}}%
\pgfpathlineto{\pgfqpoint{2.706901in}{1.125889in}}%
\pgfpathlineto{\pgfqpoint{2.703766in}{1.125889in}}%
\pgfpathlineto{\pgfqpoint{2.702512in}{1.124388in}}%
\pgfpathlineto{\pgfqpoint{2.700631in}{1.122138in}}%
\pgfpathlineto{\pgfqpoint{2.697496in}{1.122138in}}%
\pgfpathlineto{\pgfqpoint{2.694362in}{1.122138in}}%
\pgfpathlineto{\pgfqpoint{2.693108in}{1.120637in}}%
\pgfpathlineto{\pgfqpoint{2.691227in}{1.118386in}}%
\pgfpathlineto{\pgfqpoint{2.688092in}{1.118386in}}%
\pgfpathlineto{\pgfqpoint{2.684957in}{1.118386in}}%
\pgfpathlineto{\pgfqpoint{2.683703in}{1.116885in}}%
\pgfpathlineto{\pgfqpoint{2.681823in}{1.114635in}}%
\pgfpathlineto{\pgfqpoint{2.678688in}{1.114635in}}%
\pgfpathlineto{\pgfqpoint{2.677434in}{1.113134in}}%
\pgfpathlineto{\pgfqpoint{2.675553in}{1.110883in}}%
\pgfpathlineto{\pgfqpoint{2.672418in}{1.110883in}}%
\pgfpathlineto{\pgfqpoint{2.669284in}{1.110883in}}%
\pgfpathlineto{\pgfqpoint{2.668030in}{1.109382in}}%
\pgfpathlineto{\pgfqpoint{2.666149in}{1.107131in}}%
\pgfpathlineto{\pgfqpoint{2.663014in}{1.107131in}}%
\pgfpathlineto{\pgfqpoint{2.659879in}{1.107131in}}%
\pgfpathlineto{\pgfqpoint{2.658625in}{1.105631in}}%
\pgfpathlineto{\pgfqpoint{2.656745in}{1.103380in}}%
\pgfpathlineto{\pgfqpoint{2.653610in}{1.103380in}}%
\pgfpathlineto{\pgfqpoint{2.650475in}{1.103380in}}%
\pgfpathlineto{\pgfqpoint{2.649221in}{1.101879in}}%
\pgfpathlineto{\pgfqpoint{2.647340in}{1.099628in}}%
\pgfpathlineto{\pgfqpoint{2.644206in}{1.099628in}}%
\pgfpathlineto{\pgfqpoint{2.642952in}{1.098128in}}%
\pgfpathlineto{\pgfqpoint{2.641071in}{1.095877in}}%
\pgfpathlineto{\pgfqpoint{2.637936in}{1.095877in}}%
\pgfpathlineto{\pgfqpoint{2.634801in}{1.095877in}}%
\pgfpathlineto{\pgfqpoint{2.633548in}{1.094376in}}%
\pgfpathlineto{\pgfqpoint{2.631667in}{1.092125in}}%
\pgfpathlineto{\pgfqpoint{2.628532in}{1.092125in}}%
\pgfpathlineto{\pgfqpoint{2.625397in}{1.092125in}}%
\pgfpathlineto{\pgfqpoint{2.624143in}{1.090625in}}%
\pgfpathlineto{\pgfqpoint{2.622262in}{1.088374in}}%
\pgfpathlineto{\pgfqpoint{2.619128in}{1.088374in}}%
\pgfpathlineto{\pgfqpoint{2.615993in}{1.088374in}}%
\pgfpathlineto{\pgfqpoint{2.614739in}{1.086873in}}%
\pgfpathlineto{\pgfqpoint{2.612858in}{1.084622in}}%
\pgfpathlineto{\pgfqpoint{2.609724in}{1.084622in}}%
\pgfpathlineto{\pgfqpoint{2.606589in}{1.084622in}}%
\pgfpathlineto{\pgfqpoint{2.605335in}{1.083122in}}%
\pgfpathlineto{\pgfqpoint{2.603454in}{1.080871in}}%
\pgfpathlineto{\pgfqpoint{2.600319in}{1.080871in}}%
\pgfpathlineto{\pgfqpoint{2.599065in}{1.079370in}}%
\pgfpathlineto{\pgfqpoint{2.597185in}{1.077119in}}%
\pgfpathlineto{\pgfqpoint{2.594050in}{1.077119in}}%
\pgfpathlineto{\pgfqpoint{2.590915in}{1.077119in}}%
\pgfpathlineto{\pgfqpoint{2.589661in}{1.075619in}}%
\pgfpathlineto{\pgfqpoint{2.587780in}{1.073368in}}%
\pgfpathlineto{\pgfqpoint{2.584646in}{1.073368in}}%
\pgfpathlineto{\pgfqpoint{2.581511in}{1.073368in}}%
\pgfpathlineto{\pgfqpoint{2.580257in}{1.071867in}}%
\pgfpathlineto{\pgfqpoint{2.578376in}{1.069616in}}%
\pgfpathlineto{\pgfqpoint{2.575241in}{1.069616in}}%
\pgfpathlineto{\pgfqpoint{2.572107in}{1.069616in}}%
\pgfpathlineto{\pgfqpoint{2.570853in}{1.068115in}}%
\pgfpathlineto{\pgfqpoint{2.568972in}{1.065865in}}%
\pgfpathlineto{\pgfqpoint{2.565837in}{1.065865in}}%
\pgfpathlineto{\pgfqpoint{2.564583in}{1.064364in}}%
\pgfpathlineto{\pgfqpoint{2.562702in}{1.062113in}}%
\pgfpathlineto{\pgfqpoint{2.559568in}{1.062113in}}%
\pgfpathlineto{\pgfqpoint{2.556433in}{1.062113in}}%
\pgfpathlineto{\pgfqpoint{2.555179in}{1.060612in}}%
\pgfpathlineto{\pgfqpoint{2.553298in}{1.058361in}}%
\pgfpathlineto{\pgfqpoint{2.550163in}{1.058361in}}%
\pgfpathlineto{\pgfqpoint{2.547029in}{1.058361in}}%
\pgfpathlineto{\pgfqpoint{2.545775in}{1.056861in}}%
\pgfpathlineto{\pgfqpoint{2.543894in}{1.054610in}}%
\pgfpathlineto{\pgfqpoint{2.540759in}{1.054610in}}%
\pgfpathlineto{\pgfqpoint{2.537624in}{1.054610in}}%
\pgfpathlineto{\pgfqpoint{2.536371in}{1.053109in}}%
\pgfpathlineto{\pgfqpoint{2.534490in}{1.050858in}}%
\pgfpathlineto{\pgfqpoint{2.531355in}{1.050858in}}%
\pgfpathlineto{\pgfqpoint{2.530101in}{1.049358in}}%
\pgfpathlineto{\pgfqpoint{2.528220in}{1.047107in}}%
\pgfpathlineto{\pgfqpoint{2.525085in}{1.047107in}}%
\pgfpathlineto{\pgfqpoint{2.521951in}{1.047107in}}%
\pgfpathlineto{\pgfqpoint{2.520697in}{1.045606in}}%
\pgfpathlineto{\pgfqpoint{2.518816in}{1.043355in}}%
\pgfpathlineto{\pgfqpoint{2.515681in}{1.043355in}}%
\pgfpathlineto{\pgfqpoint{2.512547in}{1.043355in}}%
\pgfpathlineto{\pgfqpoint{2.511293in}{1.041855in}}%
\pgfpathlineto{\pgfqpoint{2.509412in}{1.039604in}}%
\pgfpathlineto{\pgfqpoint{2.506277in}{1.039604in}}%
\pgfpathlineto{\pgfqpoint{2.503142in}{1.039604in}}%
\pgfpathlineto{\pgfqpoint{2.501888in}{1.038103in}}%
\pgfpathlineto{\pgfqpoint{2.500008in}{1.035852in}}%
\pgfpathlineto{\pgfqpoint{2.496873in}{1.035852in}}%
\pgfpathlineto{\pgfqpoint{2.493738in}{1.035852in}}%
\pgfpathlineto{\pgfqpoint{2.492484in}{1.034352in}}%
\pgfpathlineto{\pgfqpoint{2.490603in}{1.032101in}}%
\pgfpathlineto{\pgfqpoint{2.487469in}{1.032101in}}%
\pgfpathlineto{\pgfqpoint{2.486215in}{1.030600in}}%
\pgfpathlineto{\pgfqpoint{2.484334in}{1.028349in}}%
\pgfpathlineto{\pgfqpoint{2.481199in}{1.028349in}}%
\pgfpathlineto{\pgfqpoint{2.478064in}{1.028349in}}%
\pgfpathlineto{\pgfqpoint{2.476810in}{1.026849in}}%
\pgfpathlineto{\pgfqpoint{2.474930in}{1.024598in}}%
\pgfpathlineto{\pgfqpoint{2.471795in}{1.024598in}}%
\pgfpathlineto{\pgfqpoint{2.468660in}{1.024598in}}%
\pgfpathlineto{\pgfqpoint{2.467406in}{1.023097in}}%
\pgfpathlineto{\pgfqpoint{2.465525in}{1.020846in}}%
\pgfpathlineto{\pgfqpoint{2.462391in}{1.020846in}}%
\pgfpathlineto{\pgfqpoint{2.459256in}{1.020846in}}%
\pgfpathlineto{\pgfqpoint{2.458002in}{1.019346in}}%
\pgfpathlineto{\pgfqpoint{2.456121in}{1.017095in}}%
\pgfpathlineto{\pgfqpoint{2.452986in}{1.017095in}}%
\pgfpathlineto{\pgfqpoint{2.451733in}{1.015594in}}%
\pgfpathlineto{\pgfqpoint{2.449852in}{1.013343in}}%
\pgfpathlineto{\pgfqpoint{2.446717in}{1.013343in}}%
\pgfpathlineto{\pgfqpoint{2.443582in}{1.013343in}}%
\pgfpathlineto{\pgfqpoint{2.442328in}{1.011842in}}%
\pgfpathlineto{\pgfqpoint{2.440447in}{1.009592in}}%
\pgfpathlineto{\pgfqpoint{2.437313in}{1.009592in}}%
\pgfpathlineto{\pgfqpoint{2.434178in}{1.009592in}}%
\pgfpathlineto{\pgfqpoint{2.432924in}{1.008091in}}%
\pgfpathlineto{\pgfqpoint{2.431043in}{1.005840in}}%
\pgfpathlineto{\pgfqpoint{2.427908in}{1.005840in}}%
\pgfpathlineto{\pgfqpoint{2.424774in}{1.005840in}}%
\pgfpathlineto{\pgfqpoint{2.423520in}{1.004339in}}%
\pgfpathlineto{\pgfqpoint{2.421639in}{1.002088in}}%
\pgfpathlineto{\pgfqpoint{2.418504in}{1.002088in}}%
\pgfpathlineto{\pgfqpoint{2.417250in}{1.000588in}}%
\pgfpathlineto{\pgfqpoint{2.415369in}{0.998337in}}%
\pgfpathlineto{\pgfqpoint{2.412235in}{0.998337in}}%
\pgfpathlineto{\pgfqpoint{2.409100in}{0.998337in}}%
\pgfpathlineto{\pgfqpoint{2.407846in}{0.996836in}}%
\pgfpathlineto{\pgfqpoint{2.405965in}{0.994585in}}%
\pgfpathlineto{\pgfqpoint{2.402831in}{0.994585in}}%
\pgfpathlineto{\pgfqpoint{2.399696in}{0.994585in}}%
\pgfpathlineto{\pgfqpoint{2.398442in}{0.993085in}}%
\pgfpathlineto{\pgfqpoint{2.396561in}{0.990834in}}%
\pgfpathlineto{\pgfqpoint{2.393426in}{0.990834in}}%
\pgfpathlineto{\pgfqpoint{2.390292in}{0.990834in}}%
\pgfpathlineto{\pgfqpoint{2.389038in}{0.989333in}}%
\pgfpathlineto{\pgfqpoint{2.387157in}{0.987082in}}%
\pgfpathlineto{\pgfqpoint{2.384022in}{0.987082in}}%
\pgfpathlineto{\pgfqpoint{2.380887in}{0.987082in}}%
\pgfpathlineto{\pgfqpoint{2.379633in}{0.985582in}}%
\pgfpathlineto{\pgfqpoint{2.377753in}{0.983331in}}%
\pgfpathlineto{\pgfqpoint{2.374618in}{0.983331in}}%
\pgfpathlineto{\pgfqpoint{2.373364in}{0.981830in}}%
\pgfpathlineto{\pgfqpoint{2.371483in}{0.979579in}}%
\pgfpathlineto{\pgfqpoint{2.368348in}{0.979579in}}%
\pgfpathlineto{\pgfqpoint{2.365214in}{0.979579in}}%
\pgfpathlineto{\pgfqpoint{2.363960in}{0.978079in}}%
\pgfpathlineto{\pgfqpoint{2.362079in}{0.975828in}}%
\pgfpathlineto{\pgfqpoint{2.358944in}{0.975828in}}%
\pgfpathlineto{\pgfqpoint{2.355809in}{0.975828in}}%
\pgfpathlineto{\pgfqpoint{2.354555in}{0.974327in}}%
\pgfpathlineto{\pgfqpoint{2.352675in}{0.972076in}}%
\pgfpathlineto{\pgfqpoint{2.349540in}{0.972076in}}%
\pgfpathlineto{\pgfqpoint{2.346405in}{0.972076in}}%
\pgfpathlineto{\pgfqpoint{2.345151in}{0.970576in}}%
\pgfpathlineto{\pgfqpoint{2.343270in}{0.968325in}}%
\pgfpathlineto{\pgfqpoint{2.340136in}{0.968325in}}%
\pgfpathlineto{\pgfqpoint{2.338882in}{0.966824in}}%
\pgfpathlineto{\pgfqpoint{2.337001in}{0.964573in}}%
\pgfpathlineto{\pgfqpoint{2.333866in}{0.964573in}}%
\pgfpathlineto{\pgfqpoint{2.330731in}{0.964573in}}%
\pgfpathlineto{\pgfqpoint{2.329478in}{0.963073in}}%
\pgfpathlineto{\pgfqpoint{2.327597in}{0.960822in}}%
\pgfpathlineto{\pgfqpoint{2.324462in}{0.960822in}}%
\pgfpathlineto{\pgfqpoint{2.321327in}{0.960822in}}%
\pgfpathlineto{\pgfqpoint{2.320073in}{0.959321in}}%
\pgfpathlineto{\pgfqpoint{2.318192in}{0.957070in}}%
\pgfpathlineto{\pgfqpoint{2.315058in}{0.957070in}}%
\pgfpathlineto{\pgfqpoint{2.311923in}{0.957070in}}%
\pgfpathlineto{\pgfqpoint{2.310669in}{0.955569in}}%
\pgfpathlineto{\pgfqpoint{2.308788in}{0.953319in}}%
\pgfpathlineto{\pgfqpoint{2.305654in}{0.953319in}}%
\pgfpathlineto{\pgfqpoint{2.304400in}{0.951818in}}%
\pgfpathlineto{\pgfqpoint{2.302519in}{0.949567in}}%
\pgfpathlineto{\pgfqpoint{2.299384in}{0.949567in}}%
\pgfpathlineto{\pgfqpoint{2.296249in}{0.949567in}}%
\pgfpathlineto{\pgfqpoint{2.294995in}{0.948066in}}%
\pgfpathlineto{\pgfqpoint{2.293115in}{0.945815in}}%
\pgfpathlineto{\pgfqpoint{2.289980in}{0.945815in}}%
\pgfpathlineto{\pgfqpoint{2.286845in}{0.945815in}}%
\pgfpathlineto{\pgfqpoint{2.285591in}{0.944315in}}%
\pgfpathlineto{\pgfqpoint{2.283710in}{0.942064in}}%
\pgfpathlineto{\pgfqpoint{2.280576in}{0.942064in}}%
\pgfpathlineto{\pgfqpoint{2.277441in}{0.942064in}}%
\pgfpathlineto{\pgfqpoint{2.276187in}{0.940563in}}%
\pgfpathlineto{\pgfqpoint{2.274306in}{0.938312in}}%
\pgfpathlineto{\pgfqpoint{2.271171in}{0.938312in}}%
\pgfpathlineto{\pgfqpoint{2.269917in}{0.936812in}}%
\pgfpathlineto{\pgfqpoint{2.268037in}{0.934561in}}%
\pgfpathlineto{\pgfqpoint{2.264902in}{0.934561in}}%
\pgfpathlineto{\pgfqpoint{2.261767in}{0.934561in}}%
\pgfpathlineto{\pgfqpoint{2.260513in}{0.933060in}}%
\pgfpathlineto{\pgfqpoint{2.258632in}{0.930809in}}%
\pgfpathlineto{\pgfqpoint{2.255498in}{0.930809in}}%
\pgfpathlineto{\pgfqpoint{2.252363in}{0.930809in}}%
\pgfpathlineto{\pgfqpoint{2.251109in}{0.929309in}}%
\pgfpathlineto{\pgfqpoint{2.249228in}{0.927058in}}%
\pgfpathlineto{\pgfqpoint{2.246093in}{0.927058in}}%
\pgfpathlineto{\pgfqpoint{2.242959in}{0.927058in}}%
\pgfpathlineto{\pgfqpoint{2.241705in}{0.925557in}}%
\pgfpathlineto{\pgfqpoint{2.239824in}{0.923306in}}%
\pgfpathlineto{\pgfqpoint{2.236689in}{0.923306in}}%
\pgfpathlineto{\pgfqpoint{2.233554in}{0.923306in}}%
\pgfpathlineto{\pgfqpoint{2.232301in}{0.921806in}}%
\pgfpathlineto{\pgfqpoint{2.230420in}{0.919555in}}%
\pgfpathlineto{\pgfqpoint{2.227285in}{0.919555in}}%
\pgfpathlineto{\pgfqpoint{2.226031in}{0.918054in}}%
\pgfpathlineto{\pgfqpoint{2.224150in}{0.915803in}}%
\pgfpathlineto{\pgfqpoint{2.221015in}{0.915803in}}%
\pgfpathlineto{\pgfqpoint{2.217881in}{0.915803in}}%
\pgfpathlineto{\pgfqpoint{2.216627in}{0.914303in}}%
\pgfpathlineto{\pgfqpoint{2.214746in}{0.912052in}}%
\pgfpathlineto{\pgfqpoint{2.211611in}{0.912052in}}%
\pgfpathlineto{\pgfqpoint{2.208477in}{0.912052in}}%
\pgfpathlineto{\pgfqpoint{2.207223in}{0.910551in}}%
\pgfpathlineto{\pgfqpoint{2.205342in}{0.908300in}}%
\pgfpathlineto{\pgfqpoint{2.202207in}{0.908300in}}%
\pgfpathlineto{\pgfqpoint{2.199072in}{0.908300in}}%
\pgfpathlineto{\pgfqpoint{2.197818in}{0.906799in}}%
\pgfpathlineto{\pgfqpoint{2.195938in}{0.904549in}}%
\pgfpathlineto{\pgfqpoint{2.192803in}{0.904549in}}%
\pgfpathlineto{\pgfqpoint{2.191549in}{0.903048in}}%
\pgfpathlineto{\pgfqpoint{2.189668in}{0.900797in}}%
\pgfpathlineto{\pgfqpoint{2.186533in}{0.900797in}}%
\pgfpathlineto{\pgfqpoint{2.183399in}{0.900797in}}%
\pgfpathlineto{\pgfqpoint{2.182145in}{0.899296in}}%
\pgfpathlineto{\pgfqpoint{2.180264in}{0.897045in}}%
\pgfpathlineto{\pgfqpoint{2.177129in}{0.897045in}}%
\pgfpathlineto{\pgfqpoint{2.173994in}{0.897045in}}%
\pgfpathclose%
\pgfusepath{fill}%
\end{pgfscope}%
\begin{pgfscope}%
\pgfpathrectangle{\pgfqpoint{0.888750in}{0.419100in}}{\pgfqpoint{2.504659in}{2.933700in}} %
\pgfusepath{clip}%
\pgfsetbuttcap%
\pgfsetroundjoin%
\definecolor{currentfill}{rgb}{0.000000,0.109812,0.866700}%
\pgfsetfillcolor{currentfill}%
\pgfsetfillopacity{0.300000}%
\pgfsetlinewidth{0.000000pt}%
\definecolor{currentstroke}{rgb}{0.000000,0.000000,0.000000}%
\pgfsetstrokecolor{currentstroke}%
\pgfsetdash{}{0pt}%
\pgfpathmoveto{\pgfqpoint{2.173994in}{0.895920in}}%
\pgfpathlineto{\pgfqpoint{2.177129in}{0.895920in}}%
\pgfpathlineto{\pgfqpoint{2.180264in}{0.895920in}}%
\pgfpathlineto{\pgfqpoint{2.183085in}{0.899296in}}%
\pgfpathlineto{\pgfqpoint{2.183399in}{0.899672in}}%
\pgfpathlineto{\pgfqpoint{2.186533in}{0.899672in}}%
\pgfpathlineto{\pgfqpoint{2.189668in}{0.899672in}}%
\pgfpathlineto{\pgfqpoint{2.192489in}{0.903048in}}%
\pgfpathlineto{\pgfqpoint{2.192803in}{0.903423in}}%
\pgfpathlineto{\pgfqpoint{2.195938in}{0.903423in}}%
\pgfpathlineto{\pgfqpoint{2.198759in}{0.906799in}}%
\pgfpathlineto{\pgfqpoint{2.199072in}{0.907175in}}%
\pgfpathlineto{\pgfqpoint{2.202207in}{0.907175in}}%
\pgfpathlineto{\pgfqpoint{2.205342in}{0.907175in}}%
\pgfpathlineto{\pgfqpoint{2.208163in}{0.910551in}}%
\pgfpathlineto{\pgfqpoint{2.208477in}{0.910926in}}%
\pgfpathlineto{\pgfqpoint{2.211611in}{0.910926in}}%
\pgfpathlineto{\pgfqpoint{2.214746in}{0.910926in}}%
\pgfpathlineto{\pgfqpoint{2.217567in}{0.914303in}}%
\pgfpathlineto{\pgfqpoint{2.217881in}{0.914678in}}%
\pgfpathlineto{\pgfqpoint{2.221015in}{0.914678in}}%
\pgfpathlineto{\pgfqpoint{2.224150in}{0.914678in}}%
\pgfpathlineto{\pgfqpoint{2.226971in}{0.918054in}}%
\pgfpathlineto{\pgfqpoint{2.227285in}{0.918429in}}%
\pgfpathlineto{\pgfqpoint{2.230420in}{0.918429in}}%
\pgfpathlineto{\pgfqpoint{2.233241in}{0.921806in}}%
\pgfpathlineto{\pgfqpoint{2.233554in}{0.922181in}}%
\pgfpathlineto{\pgfqpoint{2.236689in}{0.922181in}}%
\pgfpathlineto{\pgfqpoint{2.239824in}{0.922181in}}%
\pgfpathlineto{\pgfqpoint{2.242645in}{0.925557in}}%
\pgfpathlineto{\pgfqpoint{2.242959in}{0.925932in}}%
\pgfpathlineto{\pgfqpoint{2.246093in}{0.925932in}}%
\pgfpathlineto{\pgfqpoint{2.249228in}{0.925932in}}%
\pgfpathlineto{\pgfqpoint{2.252049in}{0.929309in}}%
\pgfpathlineto{\pgfqpoint{2.252363in}{0.929684in}}%
\pgfpathlineto{\pgfqpoint{2.255498in}{0.929684in}}%
\pgfpathlineto{\pgfqpoint{2.258632in}{0.929684in}}%
\pgfpathlineto{\pgfqpoint{2.261454in}{0.933060in}}%
\pgfpathlineto{\pgfqpoint{2.261767in}{0.933435in}}%
\pgfpathlineto{\pgfqpoint{2.264902in}{0.933435in}}%
\pgfpathlineto{\pgfqpoint{2.268037in}{0.933435in}}%
\pgfpathlineto{\pgfqpoint{2.270858in}{0.936812in}}%
\pgfpathlineto{\pgfqpoint{2.271171in}{0.937187in}}%
\pgfpathlineto{\pgfqpoint{2.274306in}{0.937187in}}%
\pgfpathlineto{\pgfqpoint{2.277127in}{0.940563in}}%
\pgfpathlineto{\pgfqpoint{2.277441in}{0.940938in}}%
\pgfpathlineto{\pgfqpoint{2.280576in}{0.940938in}}%
\pgfpathlineto{\pgfqpoint{2.283710in}{0.940938in}}%
\pgfpathlineto{\pgfqpoint{2.286532in}{0.944315in}}%
\pgfpathlineto{\pgfqpoint{2.286845in}{0.944690in}}%
\pgfpathlineto{\pgfqpoint{2.289980in}{0.944690in}}%
\pgfpathlineto{\pgfqpoint{2.293115in}{0.944690in}}%
\pgfpathlineto{\pgfqpoint{2.295936in}{0.948066in}}%
\pgfpathlineto{\pgfqpoint{2.296249in}{0.948442in}}%
\pgfpathlineto{\pgfqpoint{2.299384in}{0.948442in}}%
\pgfpathlineto{\pgfqpoint{2.302519in}{0.948442in}}%
\pgfpathlineto{\pgfqpoint{2.305340in}{0.951818in}}%
\pgfpathlineto{\pgfqpoint{2.305654in}{0.952193in}}%
\pgfpathlineto{\pgfqpoint{2.308788in}{0.952193in}}%
\pgfpathlineto{\pgfqpoint{2.311610in}{0.955569in}}%
\pgfpathlineto{\pgfqpoint{2.311923in}{0.955945in}}%
\pgfpathlineto{\pgfqpoint{2.315058in}{0.955945in}}%
\pgfpathlineto{\pgfqpoint{2.318192in}{0.955945in}}%
\pgfpathlineto{\pgfqpoint{2.321014in}{0.959321in}}%
\pgfpathlineto{\pgfqpoint{2.321327in}{0.959696in}}%
\pgfpathlineto{\pgfqpoint{2.324462in}{0.959696in}}%
\pgfpathlineto{\pgfqpoint{2.327597in}{0.959696in}}%
\pgfpathlineto{\pgfqpoint{2.330418in}{0.963073in}}%
\pgfpathlineto{\pgfqpoint{2.330731in}{0.963448in}}%
\pgfpathlineto{\pgfqpoint{2.333866in}{0.963448in}}%
\pgfpathlineto{\pgfqpoint{2.337001in}{0.963448in}}%
\pgfpathlineto{\pgfqpoint{2.339822in}{0.966824in}}%
\pgfpathlineto{\pgfqpoint{2.340136in}{0.967199in}}%
\pgfpathlineto{\pgfqpoint{2.343270in}{0.967199in}}%
\pgfpathlineto{\pgfqpoint{2.346092in}{0.970576in}}%
\pgfpathlineto{\pgfqpoint{2.346405in}{0.970951in}}%
\pgfpathlineto{\pgfqpoint{2.349540in}{0.970951in}}%
\pgfpathlineto{\pgfqpoint{2.352675in}{0.970951in}}%
\pgfpathlineto{\pgfqpoint{2.355496in}{0.974327in}}%
\pgfpathlineto{\pgfqpoint{2.355809in}{0.974702in}}%
\pgfpathlineto{\pgfqpoint{2.358944in}{0.974702in}}%
\pgfpathlineto{\pgfqpoint{2.362079in}{0.974702in}}%
\pgfpathlineto{\pgfqpoint{2.364900in}{0.978079in}}%
\pgfpathlineto{\pgfqpoint{2.365214in}{0.978454in}}%
\pgfpathlineto{\pgfqpoint{2.368348in}{0.978454in}}%
\pgfpathlineto{\pgfqpoint{2.371483in}{0.978454in}}%
\pgfpathlineto{\pgfqpoint{2.374304in}{0.981830in}}%
\pgfpathlineto{\pgfqpoint{2.374618in}{0.982205in}}%
\pgfpathlineto{\pgfqpoint{2.377753in}{0.982205in}}%
\pgfpathlineto{\pgfqpoint{2.380574in}{0.985582in}}%
\pgfpathlineto{\pgfqpoint{2.380887in}{0.985957in}}%
\pgfpathlineto{\pgfqpoint{2.384022in}{0.985957in}}%
\pgfpathlineto{\pgfqpoint{2.387157in}{0.985957in}}%
\pgfpathlineto{\pgfqpoint{2.389978in}{0.989333in}}%
\pgfpathlineto{\pgfqpoint{2.390292in}{0.989708in}}%
\pgfpathlineto{\pgfqpoint{2.393426in}{0.989708in}}%
\pgfpathlineto{\pgfqpoint{2.396561in}{0.989708in}}%
\pgfpathlineto{\pgfqpoint{2.399382in}{0.993085in}}%
\pgfpathlineto{\pgfqpoint{2.399696in}{0.993460in}}%
\pgfpathlineto{\pgfqpoint{2.402831in}{0.993460in}}%
\pgfpathlineto{\pgfqpoint{2.405965in}{0.993460in}}%
\pgfpathlineto{\pgfqpoint{2.408787in}{0.996836in}}%
\pgfpathlineto{\pgfqpoint{2.409100in}{0.997211in}}%
\pgfpathlineto{\pgfqpoint{2.412235in}{0.997211in}}%
\pgfpathlineto{\pgfqpoint{2.415369in}{0.997211in}}%
\pgfpathlineto{\pgfqpoint{2.418191in}{1.000588in}}%
\pgfpathlineto{\pgfqpoint{2.418504in}{1.000963in}}%
\pgfpathlineto{\pgfqpoint{2.421639in}{1.000963in}}%
\pgfpathlineto{\pgfqpoint{2.424460in}{1.004339in}}%
\pgfpathlineto{\pgfqpoint{2.424774in}{1.004715in}}%
\pgfpathlineto{\pgfqpoint{2.427908in}{1.004715in}}%
\pgfpathlineto{\pgfqpoint{2.431043in}{1.004715in}}%
\pgfpathlineto{\pgfqpoint{2.433864in}{1.008091in}}%
\pgfpathlineto{\pgfqpoint{2.434178in}{1.008466in}}%
\pgfpathlineto{\pgfqpoint{2.437313in}{1.008466in}}%
\pgfpathlineto{\pgfqpoint{2.440447in}{1.008466in}}%
\pgfpathlineto{\pgfqpoint{2.443269in}{1.011842in}}%
\pgfpathlineto{\pgfqpoint{2.443582in}{1.012218in}}%
\pgfpathlineto{\pgfqpoint{2.446717in}{1.012218in}}%
\pgfpathlineto{\pgfqpoint{2.449852in}{1.012218in}}%
\pgfpathlineto{\pgfqpoint{2.452673in}{1.015594in}}%
\pgfpathlineto{\pgfqpoint{2.452986in}{1.015969in}}%
\pgfpathlineto{\pgfqpoint{2.456121in}{1.015969in}}%
\pgfpathlineto{\pgfqpoint{2.458942in}{1.019346in}}%
\pgfpathlineto{\pgfqpoint{2.459256in}{1.019721in}}%
\pgfpathlineto{\pgfqpoint{2.462391in}{1.019721in}}%
\pgfpathlineto{\pgfqpoint{2.465525in}{1.019721in}}%
\pgfpathlineto{\pgfqpoint{2.468347in}{1.023097in}}%
\pgfpathlineto{\pgfqpoint{2.468660in}{1.023472in}}%
\pgfpathlineto{\pgfqpoint{2.471795in}{1.023472in}}%
\pgfpathlineto{\pgfqpoint{2.474930in}{1.023472in}}%
\pgfpathlineto{\pgfqpoint{2.477751in}{1.026849in}}%
\pgfpathlineto{\pgfqpoint{2.478064in}{1.027224in}}%
\pgfpathlineto{\pgfqpoint{2.481199in}{1.027224in}}%
\pgfpathlineto{\pgfqpoint{2.484334in}{1.027224in}}%
\pgfpathlineto{\pgfqpoint{2.487155in}{1.030600in}}%
\pgfpathlineto{\pgfqpoint{2.487469in}{1.030975in}}%
\pgfpathlineto{\pgfqpoint{2.490603in}{1.030975in}}%
\pgfpathlineto{\pgfqpoint{2.493425in}{1.034352in}}%
\pgfpathlineto{\pgfqpoint{2.493738in}{1.034727in}}%
\pgfpathlineto{\pgfqpoint{2.496873in}{1.034727in}}%
\pgfpathlineto{\pgfqpoint{2.500008in}{1.034727in}}%
\pgfpathlineto{\pgfqpoint{2.502829in}{1.038103in}}%
\pgfpathlineto{\pgfqpoint{2.503142in}{1.038478in}}%
\pgfpathlineto{\pgfqpoint{2.506277in}{1.038478in}}%
\pgfpathlineto{\pgfqpoint{2.509412in}{1.038478in}}%
\pgfpathlineto{\pgfqpoint{2.512233in}{1.041855in}}%
\pgfpathlineto{\pgfqpoint{2.512547in}{1.042230in}}%
\pgfpathlineto{\pgfqpoint{2.515681in}{1.042230in}}%
\pgfpathlineto{\pgfqpoint{2.518816in}{1.042230in}}%
\pgfpathlineto{\pgfqpoint{2.521637in}{1.045606in}}%
\pgfpathlineto{\pgfqpoint{2.521951in}{1.045981in}}%
\pgfpathlineto{\pgfqpoint{2.525085in}{1.045981in}}%
\pgfpathlineto{\pgfqpoint{2.528220in}{1.045981in}}%
\pgfpathlineto{\pgfqpoint{2.531041in}{1.049358in}}%
\pgfpathlineto{\pgfqpoint{2.531355in}{1.049733in}}%
\pgfpathlineto{\pgfqpoint{2.534490in}{1.049733in}}%
\pgfpathlineto{\pgfqpoint{2.537311in}{1.053109in}}%
\pgfpathlineto{\pgfqpoint{2.537624in}{1.053484in}}%
\pgfpathlineto{\pgfqpoint{2.540759in}{1.053484in}}%
\pgfpathlineto{\pgfqpoint{2.543894in}{1.053484in}}%
\pgfpathlineto{\pgfqpoint{2.546715in}{1.056861in}}%
\pgfpathlineto{\pgfqpoint{2.547029in}{1.057236in}}%
\pgfpathlineto{\pgfqpoint{2.550163in}{1.057236in}}%
\pgfpathlineto{\pgfqpoint{2.553298in}{1.057236in}}%
\pgfpathlineto{\pgfqpoint{2.556119in}{1.060612in}}%
\pgfpathlineto{\pgfqpoint{2.556433in}{1.060988in}}%
\pgfpathlineto{\pgfqpoint{2.559568in}{1.060988in}}%
\pgfpathlineto{\pgfqpoint{2.562702in}{1.060988in}}%
\pgfpathlineto{\pgfqpoint{2.565524in}{1.064364in}}%
\pgfpathlineto{\pgfqpoint{2.565837in}{1.064739in}}%
\pgfpathlineto{\pgfqpoint{2.568972in}{1.064739in}}%
\pgfpathlineto{\pgfqpoint{2.571793in}{1.068115in}}%
\pgfpathlineto{\pgfqpoint{2.572107in}{1.068491in}}%
\pgfpathlineto{\pgfqpoint{2.575241in}{1.068491in}}%
\pgfpathlineto{\pgfqpoint{2.578376in}{1.068491in}}%
\pgfpathlineto{\pgfqpoint{2.581197in}{1.071867in}}%
\pgfpathlineto{\pgfqpoint{2.581511in}{1.072242in}}%
\pgfpathlineto{\pgfqpoint{2.584646in}{1.072242in}}%
\pgfpathlineto{\pgfqpoint{2.587780in}{1.072242in}}%
\pgfpathlineto{\pgfqpoint{2.590602in}{1.075619in}}%
\pgfpathlineto{\pgfqpoint{2.590915in}{1.075994in}}%
\pgfpathlineto{\pgfqpoint{2.594050in}{1.075994in}}%
\pgfpathlineto{\pgfqpoint{2.597185in}{1.075994in}}%
\pgfpathlineto{\pgfqpoint{2.600006in}{1.079370in}}%
\pgfpathlineto{\pgfqpoint{2.600319in}{1.079745in}}%
\pgfpathlineto{\pgfqpoint{2.603454in}{1.079745in}}%
\pgfpathlineto{\pgfqpoint{2.606275in}{1.083122in}}%
\pgfpathlineto{\pgfqpoint{2.606589in}{1.083497in}}%
\pgfpathlineto{\pgfqpoint{2.609724in}{1.083497in}}%
\pgfpathlineto{\pgfqpoint{2.612858in}{1.083497in}}%
\pgfpathlineto{\pgfqpoint{2.615680in}{1.086873in}}%
\pgfpathlineto{\pgfqpoint{2.615993in}{1.087248in}}%
\pgfpathlineto{\pgfqpoint{2.619128in}{1.087248in}}%
\pgfpathlineto{\pgfqpoint{2.622262in}{1.087248in}}%
\pgfpathlineto{\pgfqpoint{2.625084in}{1.090625in}}%
\pgfpathlineto{\pgfqpoint{2.625397in}{1.091000in}}%
\pgfpathlineto{\pgfqpoint{2.628532in}{1.091000in}}%
\pgfpathlineto{\pgfqpoint{2.631667in}{1.091000in}}%
\pgfpathlineto{\pgfqpoint{2.634488in}{1.094376in}}%
\pgfpathlineto{\pgfqpoint{2.634801in}{1.094751in}}%
\pgfpathlineto{\pgfqpoint{2.637936in}{1.094751in}}%
\pgfpathlineto{\pgfqpoint{2.641071in}{1.094751in}}%
\pgfpathlineto{\pgfqpoint{2.643892in}{1.098128in}}%
\pgfpathlineto{\pgfqpoint{2.644206in}{1.098503in}}%
\pgfpathlineto{\pgfqpoint{2.647340in}{1.098503in}}%
\pgfpathlineto{\pgfqpoint{2.650162in}{1.101879in}}%
\pgfpathlineto{\pgfqpoint{2.650475in}{1.102254in}}%
\pgfpathlineto{\pgfqpoint{2.653610in}{1.102254in}}%
\pgfpathlineto{\pgfqpoint{2.656745in}{1.102254in}}%
\pgfpathlineto{\pgfqpoint{2.659566in}{1.105631in}}%
\pgfpathlineto{\pgfqpoint{2.659879in}{1.106006in}}%
\pgfpathlineto{\pgfqpoint{2.663014in}{1.106006in}}%
\pgfpathlineto{\pgfqpoint{2.666149in}{1.106006in}}%
\pgfpathlineto{\pgfqpoint{2.668970in}{1.109382in}}%
\pgfpathlineto{\pgfqpoint{2.669284in}{1.109758in}}%
\pgfpathlineto{\pgfqpoint{2.672418in}{1.109758in}}%
\pgfpathlineto{\pgfqpoint{2.675553in}{1.109758in}}%
\pgfpathlineto{\pgfqpoint{2.678374in}{1.113134in}}%
\pgfpathlineto{\pgfqpoint{2.678688in}{1.113509in}}%
\pgfpathlineto{\pgfqpoint{2.681823in}{1.113509in}}%
\pgfpathlineto{\pgfqpoint{2.684644in}{1.116885in}}%
\pgfpathlineto{\pgfqpoint{2.684957in}{1.117261in}}%
\pgfpathlineto{\pgfqpoint{2.688092in}{1.117261in}}%
\pgfpathlineto{\pgfqpoint{2.691227in}{1.117261in}}%
\pgfpathlineto{\pgfqpoint{2.694048in}{1.120637in}}%
\pgfpathlineto{\pgfqpoint{2.694362in}{1.121012in}}%
\pgfpathlineto{\pgfqpoint{2.697496in}{1.121012in}}%
\pgfpathlineto{\pgfqpoint{2.700631in}{1.121012in}}%
\pgfpathlineto{\pgfqpoint{2.703452in}{1.124388in}}%
\pgfpathlineto{\pgfqpoint{2.703766in}{1.124764in}}%
\pgfpathlineto{\pgfqpoint{2.706901in}{1.124764in}}%
\pgfpathlineto{\pgfqpoint{2.710035in}{1.124764in}}%
\pgfpathlineto{\pgfqpoint{2.712857in}{1.128140in}}%
\pgfpathlineto{\pgfqpoint{2.713170in}{1.128515in}}%
\pgfpathlineto{\pgfqpoint{2.716305in}{1.128515in}}%
\pgfpathlineto{\pgfqpoint{2.719126in}{1.131892in}}%
\pgfpathlineto{\pgfqpoint{2.719439in}{1.132267in}}%
\pgfpathlineto{\pgfqpoint{2.722574in}{1.132267in}}%
\pgfpathlineto{\pgfqpoint{2.725709in}{1.132267in}}%
\pgfpathlineto{\pgfqpoint{2.728530in}{1.135643in}}%
\pgfpathlineto{\pgfqpoint{2.728844in}{1.136018in}}%
\pgfpathlineto{\pgfqpoint{2.731978in}{1.136018in}}%
\pgfpathlineto{\pgfqpoint{2.735113in}{1.136018in}}%
\pgfpathlineto{\pgfqpoint{2.737934in}{1.139395in}}%
\pgfpathlineto{\pgfqpoint{2.738248in}{1.139770in}}%
\pgfpathlineto{\pgfqpoint{2.741383in}{1.139770in}}%
\pgfpathlineto{\pgfqpoint{2.744517in}{1.139770in}}%
\pgfpathlineto{\pgfqpoint{2.747339in}{1.143146in}}%
\pgfpathlineto{\pgfqpoint{2.747652in}{1.143521in}}%
\pgfpathlineto{\pgfqpoint{2.750787in}{1.143521in}}%
\pgfpathlineto{\pgfqpoint{2.753922in}{1.143521in}}%
\pgfpathlineto{\pgfqpoint{2.756743in}{1.146898in}}%
\pgfpathlineto{\pgfqpoint{2.757056in}{1.147273in}}%
\pgfpathlineto{\pgfqpoint{2.760191in}{1.147273in}}%
\pgfpathlineto{\pgfqpoint{2.763012in}{1.150649in}}%
\pgfpathlineto{\pgfqpoint{2.763326in}{1.151024in}}%
\pgfpathlineto{\pgfqpoint{2.766461in}{1.151024in}}%
\pgfpathlineto{\pgfqpoint{2.769595in}{1.151024in}}%
\pgfpathlineto{\pgfqpoint{2.772417in}{1.154401in}}%
\pgfpathlineto{\pgfqpoint{2.772730in}{1.154776in}}%
\pgfpathlineto{\pgfqpoint{2.775865in}{1.154776in}}%
\pgfpathlineto{\pgfqpoint{2.779000in}{1.154776in}}%
\pgfpathlineto{\pgfqpoint{2.781821in}{1.158152in}}%
\pgfpathlineto{\pgfqpoint{2.782134in}{1.158527in}}%
\pgfpathlineto{\pgfqpoint{2.785269in}{1.158527in}}%
\pgfpathlineto{\pgfqpoint{2.788404in}{1.158527in}}%
\pgfpathlineto{\pgfqpoint{2.791225in}{1.161904in}}%
\pgfpathlineto{\pgfqpoint{2.791539in}{1.162279in}}%
\pgfpathlineto{\pgfqpoint{2.794673in}{1.162279in}}%
\pgfpathlineto{\pgfqpoint{2.797495in}{1.165655in}}%
\pgfpathlineto{\pgfqpoint{2.797808in}{1.166031in}}%
\pgfpathlineto{\pgfqpoint{2.800943in}{1.166031in}}%
\pgfpathlineto{\pgfqpoint{2.804078in}{1.166031in}}%
\pgfpathlineto{\pgfqpoint{2.806899in}{1.169407in}}%
\pgfpathlineto{\pgfqpoint{2.807212in}{1.169782in}}%
\pgfpathlineto{\pgfqpoint{2.810347in}{1.169782in}}%
\pgfpathlineto{\pgfqpoint{2.813482in}{1.169782in}}%
\pgfpathlineto{\pgfqpoint{2.816303in}{1.173158in}}%
\pgfpathlineto{\pgfqpoint{2.816303in}{1.176910in}}%
\pgfpathlineto{\pgfqpoint{2.816617in}{1.177285in}}%
\pgfpathlineto{\pgfqpoint{2.819438in}{1.180662in}}%
\pgfpathlineto{\pgfqpoint{2.819438in}{1.184413in}}%
\pgfpathlineto{\pgfqpoint{2.819438in}{1.188165in}}%
\pgfpathlineto{\pgfqpoint{2.819751in}{1.188540in}}%
\pgfpathlineto{\pgfqpoint{2.822573in}{1.191916in}}%
\pgfpathlineto{\pgfqpoint{2.822573in}{1.195668in}}%
\pgfpathlineto{\pgfqpoint{2.822573in}{1.199419in}}%
\pgfpathlineto{\pgfqpoint{2.822886in}{1.199794in}}%
\pgfpathlineto{\pgfqpoint{2.825707in}{1.203171in}}%
\pgfpathlineto{\pgfqpoint{2.825707in}{1.206922in}}%
\pgfpathlineto{\pgfqpoint{2.825707in}{1.210674in}}%
\pgfpathlineto{\pgfqpoint{2.826021in}{1.211049in}}%
\pgfpathlineto{\pgfqpoint{2.828842in}{1.214425in}}%
\pgfpathlineto{\pgfqpoint{2.828842in}{1.218177in}}%
\pgfpathlineto{\pgfqpoint{2.828842in}{1.221928in}}%
\pgfpathlineto{\pgfqpoint{2.829155in}{1.222304in}}%
\pgfpathlineto{\pgfqpoint{2.831977in}{1.225680in}}%
\pgfpathlineto{\pgfqpoint{2.831977in}{1.229431in}}%
\pgfpathlineto{\pgfqpoint{2.831977in}{1.233183in}}%
\pgfpathlineto{\pgfqpoint{2.832290in}{1.233558in}}%
\pgfpathlineto{\pgfqpoint{2.835111in}{1.236935in}}%
\pgfpathlineto{\pgfqpoint{2.835111in}{1.240686in}}%
\pgfpathlineto{\pgfqpoint{2.835111in}{1.244438in}}%
\pgfpathlineto{\pgfqpoint{2.835425in}{1.244813in}}%
\pgfpathlineto{\pgfqpoint{2.838246in}{1.248189in}}%
\pgfpathlineto{\pgfqpoint{2.838246in}{1.251941in}}%
\pgfpathlineto{\pgfqpoint{2.838246in}{1.255692in}}%
\pgfpathlineto{\pgfqpoint{2.838246in}{1.259444in}}%
\pgfpathlineto{\pgfqpoint{2.838560in}{1.259819in}}%
\pgfpathlineto{\pgfqpoint{2.841381in}{1.263195in}}%
\pgfpathlineto{\pgfqpoint{2.841381in}{1.266947in}}%
\pgfpathlineto{\pgfqpoint{2.841381in}{1.270698in}}%
\pgfpathlineto{\pgfqpoint{2.841694in}{1.271073in}}%
\pgfpathlineto{\pgfqpoint{2.844516in}{1.274450in}}%
\pgfpathlineto{\pgfqpoint{2.844516in}{1.278201in}}%
\pgfpathlineto{\pgfqpoint{2.844516in}{1.281953in}}%
\pgfpathlineto{\pgfqpoint{2.844829in}{1.282328in}}%
\pgfpathlineto{\pgfqpoint{2.847650in}{1.285704in}}%
\pgfpathlineto{\pgfqpoint{2.847650in}{1.289456in}}%
\pgfpathlineto{\pgfqpoint{2.847650in}{1.293208in}}%
\pgfpathlineto{\pgfqpoint{2.847964in}{1.293583in}}%
\pgfpathlineto{\pgfqpoint{2.850785in}{1.296959in}}%
\pgfpathlineto{\pgfqpoint{2.850785in}{1.300711in}}%
\pgfpathlineto{\pgfqpoint{2.850785in}{1.304462in}}%
\pgfpathlineto{\pgfqpoint{2.851099in}{1.304837in}}%
\pgfpathlineto{\pgfqpoint{2.853920in}{1.308214in}}%
\pgfpathlineto{\pgfqpoint{2.853920in}{1.311965in}}%
\pgfpathlineto{\pgfqpoint{2.853920in}{1.315717in}}%
\pgfpathlineto{\pgfqpoint{2.854233in}{1.316092in}}%
\pgfpathlineto{\pgfqpoint{2.857055in}{1.319468in}}%
\pgfpathlineto{\pgfqpoint{2.857055in}{1.323220in}}%
\pgfpathlineto{\pgfqpoint{2.857055in}{1.326971in}}%
\pgfpathlineto{\pgfqpoint{2.857368in}{1.327347in}}%
\pgfpathlineto{\pgfqpoint{2.860189in}{1.330723in}}%
\pgfpathlineto{\pgfqpoint{2.860189in}{1.334474in}}%
\pgfpathlineto{\pgfqpoint{2.860189in}{1.338226in}}%
\pgfpathlineto{\pgfqpoint{2.860503in}{1.338601in}}%
\pgfpathlineto{\pgfqpoint{2.863324in}{1.341977in}}%
\pgfpathlineto{\pgfqpoint{2.863324in}{1.345729in}}%
\pgfpathlineto{\pgfqpoint{2.863324in}{1.349481in}}%
\pgfpathlineto{\pgfqpoint{2.863638in}{1.349856in}}%
\pgfpathlineto{\pgfqpoint{2.866459in}{1.353232in}}%
\pgfpathlineto{\pgfqpoint{2.866459in}{1.356984in}}%
\pgfpathlineto{\pgfqpoint{2.866459in}{1.360735in}}%
\pgfpathlineto{\pgfqpoint{2.866459in}{1.364487in}}%
\pgfpathlineto{\pgfqpoint{2.866772in}{1.364862in}}%
\pgfpathlineto{\pgfqpoint{2.869594in}{1.368238in}}%
\pgfpathlineto{\pgfqpoint{2.869594in}{1.371990in}}%
\pgfpathlineto{\pgfqpoint{2.869594in}{1.375741in}}%
\pgfpathlineto{\pgfqpoint{2.869907in}{1.376116in}}%
\pgfpathlineto{\pgfqpoint{2.872728in}{1.379493in}}%
\pgfpathlineto{\pgfqpoint{2.872728in}{1.383244in}}%
\pgfpathlineto{\pgfqpoint{2.872728in}{1.386996in}}%
\pgfpathlineto{\pgfqpoint{2.873042in}{1.387371in}}%
\pgfpathlineto{\pgfqpoint{2.875863in}{1.390747in}}%
\pgfpathlineto{\pgfqpoint{2.875863in}{1.394499in}}%
\pgfpathlineto{\pgfqpoint{2.875863in}{1.398251in}}%
\pgfpathlineto{\pgfqpoint{2.876177in}{1.398626in}}%
\pgfpathlineto{\pgfqpoint{2.878998in}{1.402002in}}%
\pgfpathlineto{\pgfqpoint{2.878998in}{1.405754in}}%
\pgfpathlineto{\pgfqpoint{2.878998in}{1.409505in}}%
\pgfpathlineto{\pgfqpoint{2.879311in}{1.409880in}}%
\pgfpathlineto{\pgfqpoint{2.882133in}{1.413257in}}%
\pgfpathlineto{\pgfqpoint{2.882133in}{1.417008in}}%
\pgfpathlineto{\pgfqpoint{2.882133in}{1.420760in}}%
\pgfpathlineto{\pgfqpoint{2.882446in}{1.421135in}}%
\pgfpathlineto{\pgfqpoint{2.885267in}{1.424511in}}%
\pgfpathlineto{\pgfqpoint{2.885267in}{1.428263in}}%
\pgfpathlineto{\pgfqpoint{2.885267in}{1.432014in}}%
\pgfpathlineto{\pgfqpoint{2.885581in}{1.432389in}}%
\pgfpathlineto{\pgfqpoint{2.888402in}{1.435766in}}%
\pgfpathlineto{\pgfqpoint{2.888402in}{1.439517in}}%
\pgfpathlineto{\pgfqpoint{2.888402in}{1.443269in}}%
\pgfpathlineto{\pgfqpoint{2.888716in}{1.443644in}}%
\pgfpathlineto{\pgfqpoint{2.891537in}{1.447020in}}%
\pgfpathlineto{\pgfqpoint{2.891537in}{1.450772in}}%
\pgfpathlineto{\pgfqpoint{2.891537in}{1.454524in}}%
\pgfpathlineto{\pgfqpoint{2.891537in}{1.458275in}}%
\pgfpathlineto{\pgfqpoint{2.891850in}{1.458650in}}%
\pgfpathlineto{\pgfqpoint{2.894672in}{1.462027in}}%
\pgfpathlineto{\pgfqpoint{2.894672in}{1.465778in}}%
\pgfpathlineto{\pgfqpoint{2.894672in}{1.469530in}}%
\pgfpathlineto{\pgfqpoint{2.894985in}{1.469905in}}%
\pgfpathlineto{\pgfqpoint{2.897806in}{1.473281in}}%
\pgfpathlineto{\pgfqpoint{2.897806in}{1.477033in}}%
\pgfpathlineto{\pgfqpoint{2.897806in}{1.480784in}}%
\pgfpathlineto{\pgfqpoint{2.898120in}{1.481159in}}%
\pgfpathlineto{\pgfqpoint{2.900941in}{1.484536in}}%
\pgfpathlineto{\pgfqpoint{2.900941in}{1.488287in}}%
\pgfpathlineto{\pgfqpoint{2.900941in}{1.492039in}}%
\pgfpathlineto{\pgfqpoint{2.901255in}{1.492414in}}%
\pgfpathlineto{\pgfqpoint{2.904076in}{1.495790in}}%
\pgfpathlineto{\pgfqpoint{2.904076in}{1.499542in}}%
\pgfpathlineto{\pgfqpoint{2.904076in}{1.503293in}}%
\pgfpathlineto{\pgfqpoint{2.904389in}{1.503669in}}%
\pgfpathlineto{\pgfqpoint{2.907211in}{1.507045in}}%
\pgfpathlineto{\pgfqpoint{2.907211in}{1.510797in}}%
\pgfpathlineto{\pgfqpoint{2.907211in}{1.514548in}}%
\pgfpathlineto{\pgfqpoint{2.907524in}{1.514923in}}%
\pgfpathlineto{\pgfqpoint{2.910345in}{1.518300in}}%
\pgfpathlineto{\pgfqpoint{2.910345in}{1.522051in}}%
\pgfpathlineto{\pgfqpoint{2.910345in}{1.525803in}}%
\pgfpathlineto{\pgfqpoint{2.910659in}{1.526178in}}%
\pgfpathlineto{\pgfqpoint{2.913480in}{1.529554in}}%
\pgfpathlineto{\pgfqpoint{2.913480in}{1.533306in}}%
\pgfpathlineto{\pgfqpoint{2.913480in}{1.537057in}}%
\pgfpathlineto{\pgfqpoint{2.913794in}{1.537432in}}%
\pgfpathlineto{\pgfqpoint{2.916615in}{1.540809in}}%
\pgfpathlineto{\pgfqpoint{2.916615in}{1.544560in}}%
\pgfpathlineto{\pgfqpoint{2.916615in}{1.548312in}}%
\pgfpathlineto{\pgfqpoint{2.916928in}{1.548687in}}%
\pgfpathlineto{\pgfqpoint{2.919750in}{1.552063in}}%
\pgfpathlineto{\pgfqpoint{2.919750in}{1.555815in}}%
\pgfpathlineto{\pgfqpoint{2.919750in}{1.559566in}}%
\pgfpathlineto{\pgfqpoint{2.919750in}{1.563318in}}%
\pgfpathlineto{\pgfqpoint{2.920063in}{1.563693in}}%
\pgfpathlineto{\pgfqpoint{2.922884in}{1.567070in}}%
\pgfpathlineto{\pgfqpoint{2.922884in}{1.570821in}}%
\pgfpathlineto{\pgfqpoint{2.922884in}{1.574573in}}%
\pgfpathlineto{\pgfqpoint{2.923198in}{1.574948in}}%
\pgfpathlineto{\pgfqpoint{2.926019in}{1.578324in}}%
\pgfpathlineto{\pgfqpoint{2.926019in}{1.582076in}}%
\pgfpathlineto{\pgfqpoint{2.926019in}{1.585827in}}%
\pgfpathlineto{\pgfqpoint{2.926332in}{1.586202in}}%
\pgfpathlineto{\pgfqpoint{2.929154in}{1.589579in}}%
\pgfpathlineto{\pgfqpoint{2.929154in}{1.593330in}}%
\pgfpathlineto{\pgfqpoint{2.929154in}{1.597082in}}%
\pgfpathlineto{\pgfqpoint{2.929467in}{1.597457in}}%
\pgfpathlineto{\pgfqpoint{2.932288in}{1.600833in}}%
\pgfpathlineto{\pgfqpoint{2.932288in}{1.604585in}}%
\pgfpathlineto{\pgfqpoint{2.932288in}{1.608336in}}%
\pgfpathlineto{\pgfqpoint{2.932602in}{1.608712in}}%
\pgfpathlineto{\pgfqpoint{2.935423in}{1.612088in}}%
\pgfpathlineto{\pgfqpoint{2.935423in}{1.615840in}}%
\pgfpathlineto{\pgfqpoint{2.935423in}{1.619591in}}%
\pgfpathlineto{\pgfqpoint{2.935737in}{1.619966in}}%
\pgfpathlineto{\pgfqpoint{2.938558in}{1.623343in}}%
\pgfpathlineto{\pgfqpoint{2.938558in}{1.627094in}}%
\pgfpathlineto{\pgfqpoint{2.938558in}{1.630846in}}%
\pgfpathlineto{\pgfqpoint{2.938871in}{1.631221in}}%
\pgfpathlineto{\pgfqpoint{2.941693in}{1.634597in}}%
\pgfpathlineto{\pgfqpoint{2.941693in}{1.638349in}}%
\pgfpathlineto{\pgfqpoint{2.941693in}{1.642100in}}%
\pgfpathlineto{\pgfqpoint{2.942006in}{1.642475in}}%
\pgfpathlineto{\pgfqpoint{2.944827in}{1.645852in}}%
\pgfpathlineto{\pgfqpoint{2.944827in}{1.649603in}}%
\pgfpathlineto{\pgfqpoint{2.944827in}{1.653355in}}%
\pgfpathlineto{\pgfqpoint{2.944827in}{1.657106in}}%
\pgfpathlineto{\pgfqpoint{2.945141in}{1.657482in}}%
\pgfpathlineto{\pgfqpoint{2.947962in}{1.660858in}}%
\pgfpathlineto{\pgfqpoint{2.947962in}{1.664609in}}%
\pgfpathlineto{\pgfqpoint{2.947962in}{1.668361in}}%
\pgfpathlineto{\pgfqpoint{2.948276in}{1.668736in}}%
\pgfpathlineto{\pgfqpoint{2.951097in}{1.672113in}}%
\pgfpathlineto{\pgfqpoint{2.951097in}{1.675864in}}%
\pgfpathlineto{\pgfqpoint{2.951097in}{1.679616in}}%
\pgfpathlineto{\pgfqpoint{2.951410in}{1.679991in}}%
\pgfpathlineto{\pgfqpoint{2.954232in}{1.683367in}}%
\pgfpathlineto{\pgfqpoint{2.954232in}{1.687119in}}%
\pgfpathlineto{\pgfqpoint{2.954232in}{1.690870in}}%
\pgfpathlineto{\pgfqpoint{2.954545in}{1.691245in}}%
\pgfpathlineto{\pgfqpoint{2.957366in}{1.694622in}}%
\pgfpathlineto{\pgfqpoint{2.957366in}{1.698373in}}%
\pgfpathlineto{\pgfqpoint{2.957366in}{1.702125in}}%
\pgfpathlineto{\pgfqpoint{2.957680in}{1.702500in}}%
\pgfpathlineto{\pgfqpoint{2.960501in}{1.705876in}}%
\pgfpathlineto{\pgfqpoint{2.960501in}{1.709628in}}%
\pgfpathlineto{\pgfqpoint{2.960501in}{1.713379in}}%
\pgfpathlineto{\pgfqpoint{2.960815in}{1.713755in}}%
\pgfpathlineto{\pgfqpoint{2.963636in}{1.717131in}}%
\pgfpathlineto{\pgfqpoint{2.963636in}{1.720882in}}%
\pgfpathlineto{\pgfqpoint{2.963636in}{1.724634in}}%
\pgfpathlineto{\pgfqpoint{2.963949in}{1.725009in}}%
\pgfpathlineto{\pgfqpoint{2.966771in}{1.728386in}}%
\pgfpathlineto{\pgfqpoint{2.966771in}{1.732137in}}%
\pgfpathlineto{\pgfqpoint{2.966771in}{1.735889in}}%
\pgfpathlineto{\pgfqpoint{2.967084in}{1.736264in}}%
\pgfpathlineto{\pgfqpoint{2.969905in}{1.739640in}}%
\pgfpathlineto{\pgfqpoint{2.969905in}{1.743392in}}%
\pgfpathlineto{\pgfqpoint{2.969905in}{1.747143in}}%
\pgfpathlineto{\pgfqpoint{2.970219in}{1.747518in}}%
\pgfpathlineto{\pgfqpoint{2.973040in}{1.750895in}}%
\pgfpathlineto{\pgfqpoint{2.973040in}{1.754646in}}%
\pgfpathlineto{\pgfqpoint{2.973040in}{1.758398in}}%
\pgfpathlineto{\pgfqpoint{2.973040in}{1.762149in}}%
\pgfpathlineto{\pgfqpoint{2.973354in}{1.762525in}}%
\pgfpathlineto{\pgfqpoint{2.976175in}{1.765901in}}%
\pgfpathlineto{\pgfqpoint{2.976175in}{1.769652in}}%
\pgfpathlineto{\pgfqpoint{2.976175in}{1.773404in}}%
\pgfpathlineto{\pgfqpoint{2.976488in}{1.773779in}}%
\pgfpathlineto{\pgfqpoint{2.979310in}{1.777155in}}%
\pgfpathlineto{\pgfqpoint{2.979310in}{1.780907in}}%
\pgfpathlineto{\pgfqpoint{2.979310in}{1.784659in}}%
\pgfpathlineto{\pgfqpoint{2.979623in}{1.785034in}}%
\pgfpathlineto{\pgfqpoint{2.982444in}{1.788410in}}%
\pgfpathlineto{\pgfqpoint{2.982444in}{1.792162in}}%
\pgfpathlineto{\pgfqpoint{2.982444in}{1.795913in}}%
\pgfpathlineto{\pgfqpoint{2.982758in}{1.796288in}}%
\pgfpathlineto{\pgfqpoint{2.985579in}{1.799665in}}%
\pgfpathlineto{\pgfqpoint{2.985579in}{1.803416in}}%
\pgfpathlineto{\pgfqpoint{2.985579in}{1.807168in}}%
\pgfpathlineto{\pgfqpoint{2.985893in}{1.807543in}}%
\pgfpathlineto{\pgfqpoint{2.988714in}{1.810919in}}%
\pgfpathlineto{\pgfqpoint{2.988714in}{1.814671in}}%
\pgfpathlineto{\pgfqpoint{2.988714in}{1.818422in}}%
\pgfpathlineto{\pgfqpoint{2.989027in}{1.818798in}}%
\pgfpathlineto{\pgfqpoint{2.991849in}{1.822174in}}%
\pgfpathlineto{\pgfqpoint{2.991849in}{1.825925in}}%
\pgfpathlineto{\pgfqpoint{2.991849in}{1.829677in}}%
\pgfpathlineto{\pgfqpoint{2.992162in}{1.830052in}}%
\pgfpathlineto{\pgfqpoint{2.994983in}{1.833429in}}%
\pgfpathlineto{\pgfqpoint{2.994983in}{1.837180in}}%
\pgfpathlineto{\pgfqpoint{2.994983in}{1.840932in}}%
\pgfpathlineto{\pgfqpoint{2.995297in}{1.841307in}}%
\pgfpathlineto{\pgfqpoint{2.998118in}{1.844683in}}%
\pgfpathlineto{\pgfqpoint{2.998118in}{1.848435in}}%
\pgfpathlineto{\pgfqpoint{2.998118in}{1.852186in}}%
\pgfpathlineto{\pgfqpoint{2.998118in}{1.855938in}}%
\pgfpathlineto{\pgfqpoint{2.998432in}{1.856313in}}%
\pgfpathlineto{\pgfqpoint{3.001253in}{1.859689in}}%
\pgfpathlineto{\pgfqpoint{3.001253in}{1.863441in}}%
\pgfpathlineto{\pgfqpoint{3.001253in}{1.867192in}}%
\pgfpathlineto{\pgfqpoint{3.001566in}{1.867567in}}%
\pgfpathlineto{\pgfqpoint{3.004388in}{1.870944in}}%
\pgfpathlineto{\pgfqpoint{3.004388in}{1.874695in}}%
\pgfpathlineto{\pgfqpoint{3.004388in}{1.878447in}}%
\pgfpathlineto{\pgfqpoint{3.004701in}{1.878822in}}%
\pgfpathlineto{\pgfqpoint{3.007522in}{1.882198in}}%
\pgfpathlineto{\pgfqpoint{3.007522in}{1.885950in}}%
\pgfpathlineto{\pgfqpoint{3.007522in}{1.889702in}}%
\pgfpathlineto{\pgfqpoint{3.007836in}{1.890077in}}%
\pgfpathlineto{\pgfqpoint{3.010657in}{1.893453in}}%
\pgfpathlineto{\pgfqpoint{3.010657in}{1.897205in}}%
\pgfpathlineto{\pgfqpoint{3.010657in}{1.900956in}}%
\pgfpathlineto{\pgfqpoint{3.010971in}{1.901331in}}%
\pgfpathlineto{\pgfqpoint{3.013792in}{1.904708in}}%
\pgfpathlineto{\pgfqpoint{3.013792in}{1.908459in}}%
\pgfpathlineto{\pgfqpoint{3.013792in}{1.912211in}}%
\pgfpathlineto{\pgfqpoint{3.014105in}{1.912586in}}%
\pgfpathlineto{\pgfqpoint{3.016927in}{1.915962in}}%
\pgfpathlineto{\pgfqpoint{3.016927in}{1.919714in}}%
\pgfpathlineto{\pgfqpoint{3.016927in}{1.923465in}}%
\pgfpathlineto{\pgfqpoint{3.017240in}{1.923840in}}%
\pgfpathlineto{\pgfqpoint{3.020061in}{1.927217in}}%
\pgfpathlineto{\pgfqpoint{3.020061in}{1.930968in}}%
\pgfpathlineto{\pgfqpoint{3.020061in}{1.934720in}}%
\pgfpathlineto{\pgfqpoint{3.020375in}{1.935095in}}%
\pgfpathlineto{\pgfqpoint{3.023196in}{1.938471in}}%
\pgfpathlineto{\pgfqpoint{3.023196in}{1.942223in}}%
\pgfpathlineto{\pgfqpoint{3.023196in}{1.945975in}}%
\pgfpathlineto{\pgfqpoint{3.023510in}{1.946350in}}%
\pgfpathlineto{\pgfqpoint{3.026331in}{1.949726in}}%
\pgfpathlineto{\pgfqpoint{3.026331in}{1.953478in}}%
\pgfpathlineto{\pgfqpoint{3.026331in}{1.957229in}}%
\pgfpathlineto{\pgfqpoint{3.026331in}{1.960981in}}%
\pgfpathlineto{\pgfqpoint{3.026644in}{1.961356in}}%
\pgfpathlineto{\pgfqpoint{3.029466in}{1.964732in}}%
\pgfpathlineto{\pgfqpoint{3.029466in}{1.968484in}}%
\pgfpathlineto{\pgfqpoint{3.029466in}{1.972235in}}%
\pgfpathlineto{\pgfqpoint{3.029779in}{1.972610in}}%
\pgfpathlineto{\pgfqpoint{3.032600in}{1.975987in}}%
\pgfpathlineto{\pgfqpoint{3.032600in}{1.979738in}}%
\pgfpathlineto{\pgfqpoint{3.032600in}{1.983490in}}%
\pgfpathlineto{\pgfqpoint{3.032914in}{1.983865in}}%
\pgfpathlineto{\pgfqpoint{3.035735in}{1.987241in}}%
\pgfpathlineto{\pgfqpoint{3.035735in}{1.990993in}}%
\pgfpathlineto{\pgfqpoint{3.035735in}{1.994745in}}%
\pgfpathlineto{\pgfqpoint{3.036048in}{1.995120in}}%
\pgfpathlineto{\pgfqpoint{3.038870in}{1.998496in}}%
\pgfpathlineto{\pgfqpoint{3.038870in}{2.002248in}}%
\pgfpathlineto{\pgfqpoint{3.038870in}{2.005999in}}%
\pgfpathlineto{\pgfqpoint{3.039183in}{2.006374in}}%
\pgfpathlineto{\pgfqpoint{3.042004in}{2.009751in}}%
\pgfpathlineto{\pgfqpoint{3.042004in}{2.013502in}}%
\pgfpathlineto{\pgfqpoint{3.042004in}{2.017254in}}%
\pgfpathlineto{\pgfqpoint{3.042318in}{2.017629in}}%
\pgfpathlineto{\pgfqpoint{3.045139in}{2.021005in}}%
\pgfpathlineto{\pgfqpoint{3.045139in}{2.024757in}}%
\pgfpathlineto{\pgfqpoint{3.045139in}{2.028508in}}%
\pgfpathlineto{\pgfqpoint{3.045453in}{2.028883in}}%
\pgfpathlineto{\pgfqpoint{3.048274in}{2.032260in}}%
\pgfpathlineto{\pgfqpoint{3.048274in}{2.036011in}}%
\pgfpathlineto{\pgfqpoint{3.048274in}{2.039763in}}%
\pgfpathlineto{\pgfqpoint{3.048587in}{2.040138in}}%
\pgfpathlineto{\pgfqpoint{3.051409in}{2.043514in}}%
\pgfpathlineto{\pgfqpoint{3.051409in}{2.047266in}}%
\pgfpathlineto{\pgfqpoint{3.051409in}{2.051018in}}%
\pgfpathlineto{\pgfqpoint{3.051409in}{2.054769in}}%
\pgfpathlineto{\pgfqpoint{3.051722in}{2.055144in}}%
\pgfpathlineto{\pgfqpoint{3.054543in}{2.058521in}}%
\pgfpathlineto{\pgfqpoint{3.054543in}{2.062272in}}%
\pgfpathlineto{\pgfqpoint{3.054543in}{2.066024in}}%
\pgfpathlineto{\pgfqpoint{3.054857in}{2.066399in}}%
\pgfpathlineto{\pgfqpoint{3.057678in}{2.069775in}}%
\pgfpathlineto{\pgfqpoint{3.057678in}{2.073527in}}%
\pgfpathlineto{\pgfqpoint{3.057678in}{2.077278in}}%
\pgfpathlineto{\pgfqpoint{3.057992in}{2.077653in}}%
\pgfpathlineto{\pgfqpoint{3.060813in}{2.081030in}}%
\pgfpathlineto{\pgfqpoint{3.060813in}{2.084781in}}%
\pgfpathlineto{\pgfqpoint{3.060813in}{2.088533in}}%
\pgfpathlineto{\pgfqpoint{3.061126in}{2.088908in}}%
\pgfpathlineto{\pgfqpoint{3.063948in}{2.092284in}}%
\pgfpathlineto{\pgfqpoint{3.063948in}{2.096036in}}%
\pgfpathlineto{\pgfqpoint{3.063948in}{2.099787in}}%
\pgfpathlineto{\pgfqpoint{3.064261in}{2.100163in}}%
\pgfpathlineto{\pgfqpoint{3.067082in}{2.103539in}}%
\pgfpathlineto{\pgfqpoint{3.067082in}{2.107291in}}%
\pgfpathlineto{\pgfqpoint{3.067082in}{2.111042in}}%
\pgfpathlineto{\pgfqpoint{3.067396in}{2.111417in}}%
\pgfpathlineto{\pgfqpoint{3.070217in}{2.114794in}}%
\pgfpathlineto{\pgfqpoint{3.070217in}{2.118545in}}%
\pgfpathlineto{\pgfqpoint{3.070217in}{2.122297in}}%
\pgfpathlineto{\pgfqpoint{3.070531in}{2.122672in}}%
\pgfpathlineto{\pgfqpoint{3.073352in}{2.126048in}}%
\pgfpathlineto{\pgfqpoint{3.073352in}{2.129800in}}%
\pgfpathlineto{\pgfqpoint{3.073352in}{2.133551in}}%
\pgfpathlineto{\pgfqpoint{3.073665in}{2.133926in}}%
\pgfpathlineto{\pgfqpoint{3.076487in}{2.137303in}}%
\pgfpathlineto{\pgfqpoint{3.076487in}{2.141054in}}%
\pgfpathlineto{\pgfqpoint{3.076487in}{2.144806in}}%
\pgfpathlineto{\pgfqpoint{3.076800in}{2.145181in}}%
\pgfpathlineto{\pgfqpoint{3.079621in}{2.148557in}}%
\pgfpathlineto{\pgfqpoint{3.079621in}{2.152309in}}%
\pgfpathlineto{\pgfqpoint{3.079621in}{2.156060in}}%
\pgfpathlineto{\pgfqpoint{3.079621in}{2.159812in}}%
\pgfpathlineto{\pgfqpoint{3.079935in}{2.160187in}}%
\pgfpathlineto{\pgfqpoint{3.082756in}{2.163564in}}%
\pgfpathlineto{\pgfqpoint{3.082756in}{2.167315in}}%
\pgfpathlineto{\pgfqpoint{3.082756in}{2.171067in}}%
\pgfpathlineto{\pgfqpoint{3.083070in}{2.171442in}}%
\pgfpathlineto{\pgfqpoint{3.085891in}{2.174818in}}%
\pgfpathlineto{\pgfqpoint{3.085891in}{2.178570in}}%
\pgfpathlineto{\pgfqpoint{3.085891in}{2.182321in}}%
\pgfpathlineto{\pgfqpoint{3.086204in}{2.182696in}}%
\pgfpathlineto{\pgfqpoint{3.089026in}{2.186073in}}%
\pgfpathlineto{\pgfqpoint{3.089026in}{2.189824in}}%
\pgfpathlineto{\pgfqpoint{3.089026in}{2.193576in}}%
\pgfpathlineto{\pgfqpoint{3.089339in}{2.193951in}}%
\pgfpathlineto{\pgfqpoint{3.092160in}{2.197327in}}%
\pgfpathlineto{\pgfqpoint{3.092160in}{2.201079in}}%
\pgfpathlineto{\pgfqpoint{3.092160in}{2.204830in}}%
\pgfpathlineto{\pgfqpoint{3.092474in}{2.205206in}}%
\pgfpathlineto{\pgfqpoint{3.095295in}{2.208582in}}%
\pgfpathlineto{\pgfqpoint{3.095295in}{2.212334in}}%
\pgfpathlineto{\pgfqpoint{3.092474in}{2.215710in}}%
\pgfpathlineto{\pgfqpoint{3.092160in}{2.216085in}}%
\pgfpathlineto{\pgfqpoint{3.089339in}{2.219461in}}%
\pgfpathlineto{\pgfqpoint{3.089026in}{2.219837in}}%
\pgfpathlineto{\pgfqpoint{3.086204in}{2.223213in}}%
\pgfpathlineto{\pgfqpoint{3.085891in}{2.223588in}}%
\pgfpathlineto{\pgfqpoint{3.083070in}{2.226964in}}%
\pgfpathlineto{\pgfqpoint{3.082756in}{2.227340in}}%
\pgfpathlineto{\pgfqpoint{3.079935in}{2.230716in}}%
\pgfpathlineto{\pgfqpoint{3.079621in}{2.231091in}}%
\pgfpathlineto{\pgfqpoint{3.076800in}{2.234468in}}%
\pgfpathlineto{\pgfqpoint{3.076487in}{2.234843in}}%
\pgfpathlineto{\pgfqpoint{3.073665in}{2.238219in}}%
\pgfpathlineto{\pgfqpoint{3.073352in}{2.238594in}}%
\pgfpathlineto{\pgfqpoint{3.070531in}{2.241971in}}%
\pgfpathlineto{\pgfqpoint{3.070217in}{2.242346in}}%
\pgfpathlineto{\pgfqpoint{3.067396in}{2.245722in}}%
\pgfpathlineto{\pgfqpoint{3.067082in}{2.246097in}}%
\pgfpathlineto{\pgfqpoint{3.064261in}{2.249474in}}%
\pgfpathlineto{\pgfqpoint{3.063948in}{2.249849in}}%
\pgfpathlineto{\pgfqpoint{3.061126in}{2.253225in}}%
\pgfpathlineto{\pgfqpoint{3.060813in}{2.253600in}}%
\pgfpathlineto{\pgfqpoint{3.057992in}{2.256977in}}%
\pgfpathlineto{\pgfqpoint{3.057678in}{2.257352in}}%
\pgfpathlineto{\pgfqpoint{3.054857in}{2.260728in}}%
\pgfpathlineto{\pgfqpoint{3.054543in}{2.261103in}}%
\pgfpathlineto{\pgfqpoint{3.054543in}{2.264855in}}%
\pgfpathlineto{\pgfqpoint{3.051722in}{2.268231in}}%
\pgfpathlineto{\pgfqpoint{3.051409in}{2.268607in}}%
\pgfpathlineto{\pgfqpoint{3.048587in}{2.271983in}}%
\pgfpathlineto{\pgfqpoint{3.048274in}{2.272358in}}%
\pgfpathlineto{\pgfqpoint{3.045453in}{2.275734in}}%
\pgfpathlineto{\pgfqpoint{3.045139in}{2.276110in}}%
\pgfpathlineto{\pgfqpoint{3.042318in}{2.279486in}}%
\pgfpathlineto{\pgfqpoint{3.042004in}{2.279861in}}%
\pgfpathlineto{\pgfqpoint{3.039183in}{2.283238in}}%
\pgfpathlineto{\pgfqpoint{3.038870in}{2.283613in}}%
\pgfpathlineto{\pgfqpoint{3.036048in}{2.286989in}}%
\pgfpathlineto{\pgfqpoint{3.035735in}{2.287364in}}%
\pgfpathlineto{\pgfqpoint{3.032914in}{2.290741in}}%
\pgfpathlineto{\pgfqpoint{3.032600in}{2.291116in}}%
\pgfpathlineto{\pgfqpoint{3.029779in}{2.294492in}}%
\pgfpathlineto{\pgfqpoint{3.029466in}{2.294867in}}%
\pgfpathlineto{\pgfqpoint{3.026644in}{2.298244in}}%
\pgfpathlineto{\pgfqpoint{3.026331in}{2.298619in}}%
\pgfpathlineto{\pgfqpoint{3.023510in}{2.301995in}}%
\pgfpathlineto{\pgfqpoint{3.023196in}{2.302370in}}%
\pgfpathlineto{\pgfqpoint{3.020375in}{2.305747in}}%
\pgfpathlineto{\pgfqpoint{3.020061in}{2.306122in}}%
\pgfpathlineto{\pgfqpoint{3.017240in}{2.309498in}}%
\pgfpathlineto{\pgfqpoint{3.016927in}{2.309873in}}%
\pgfpathlineto{\pgfqpoint{3.014105in}{2.313250in}}%
\pgfpathlineto{\pgfqpoint{3.013792in}{2.313625in}}%
\pgfpathlineto{\pgfqpoint{3.010971in}{2.317001in}}%
\pgfpathlineto{\pgfqpoint{3.010657in}{2.317376in}}%
\pgfpathlineto{\pgfqpoint{3.007836in}{2.320753in}}%
\pgfpathlineto{\pgfqpoint{3.007522in}{2.321128in}}%
\pgfpathlineto{\pgfqpoint{3.004701in}{2.324504in}}%
\pgfpathlineto{\pgfqpoint{3.004388in}{2.324880in}}%
\pgfpathlineto{\pgfqpoint{3.004388in}{2.328631in}}%
\pgfpathlineto{\pgfqpoint{3.001566in}{2.332007in}}%
\pgfpathlineto{\pgfqpoint{3.001253in}{2.332383in}}%
\pgfpathlineto{\pgfqpoint{2.998432in}{2.335759in}}%
\pgfpathlineto{\pgfqpoint{2.998118in}{2.336134in}}%
\pgfpathlineto{\pgfqpoint{2.995297in}{2.339511in}}%
\pgfpathlineto{\pgfqpoint{2.994983in}{2.339886in}}%
\pgfpathlineto{\pgfqpoint{2.992162in}{2.343262in}}%
\pgfpathlineto{\pgfqpoint{2.991849in}{2.343637in}}%
\pgfpathlineto{\pgfqpoint{2.989027in}{2.347014in}}%
\pgfpathlineto{\pgfqpoint{2.988714in}{2.347389in}}%
\pgfpathlineto{\pgfqpoint{2.985893in}{2.350765in}}%
\pgfpathlineto{\pgfqpoint{2.985579in}{2.351140in}}%
\pgfpathlineto{\pgfqpoint{2.982758in}{2.354517in}}%
\pgfpathlineto{\pgfqpoint{2.982444in}{2.354892in}}%
\pgfpathlineto{\pgfqpoint{2.979623in}{2.358268in}}%
\pgfpathlineto{\pgfqpoint{2.979310in}{2.358643in}}%
\pgfpathlineto{\pgfqpoint{2.976488in}{2.362020in}}%
\pgfpathlineto{\pgfqpoint{2.976175in}{2.362395in}}%
\pgfpathlineto{\pgfqpoint{2.973354in}{2.365771in}}%
\pgfpathlineto{\pgfqpoint{2.973040in}{2.366146in}}%
\pgfpathlineto{\pgfqpoint{2.970219in}{2.369523in}}%
\pgfpathlineto{\pgfqpoint{2.969905in}{2.369898in}}%
\pgfpathlineto{\pgfqpoint{2.967084in}{2.373274in}}%
\pgfpathlineto{\pgfqpoint{2.966771in}{2.373649in}}%
\pgfpathlineto{\pgfqpoint{2.963949in}{2.377026in}}%
\pgfpathlineto{\pgfqpoint{2.963636in}{2.377401in}}%
\pgfpathlineto{\pgfqpoint{2.960815in}{2.380777in}}%
\pgfpathlineto{\pgfqpoint{2.960501in}{2.381153in}}%
\pgfpathlineto{\pgfqpoint{2.957680in}{2.384529in}}%
\pgfpathlineto{\pgfqpoint{2.957366in}{2.384904in}}%
\pgfpathlineto{\pgfqpoint{2.957366in}{2.388656in}}%
\pgfpathlineto{\pgfqpoint{2.954545in}{2.392032in}}%
\pgfpathlineto{\pgfqpoint{2.954232in}{2.392407in}}%
\pgfpathlineto{\pgfqpoint{2.951410in}{2.395784in}}%
\pgfpathlineto{\pgfqpoint{2.951097in}{2.396159in}}%
\pgfpathlineto{\pgfqpoint{2.948276in}{2.399535in}}%
\pgfpathlineto{\pgfqpoint{2.947962in}{2.399910in}}%
\pgfpathlineto{\pgfqpoint{2.945141in}{2.403287in}}%
\pgfpathlineto{\pgfqpoint{2.944827in}{2.403662in}}%
\pgfpathlineto{\pgfqpoint{2.942006in}{2.407038in}}%
\pgfpathlineto{\pgfqpoint{2.941693in}{2.407413in}}%
\pgfpathlineto{\pgfqpoint{2.938871in}{2.410790in}}%
\pgfpathlineto{\pgfqpoint{2.938558in}{2.411165in}}%
\pgfpathlineto{\pgfqpoint{2.935737in}{2.414541in}}%
\pgfpathlineto{\pgfqpoint{2.935423in}{2.414916in}}%
\pgfpathlineto{\pgfqpoint{2.932602in}{2.418293in}}%
\pgfpathlineto{\pgfqpoint{2.932288in}{2.418668in}}%
\pgfpathlineto{\pgfqpoint{2.929467in}{2.422044in}}%
\pgfpathlineto{\pgfqpoint{2.929154in}{2.422419in}}%
\pgfpathlineto{\pgfqpoint{2.926332in}{2.425796in}}%
\pgfpathlineto{\pgfqpoint{2.926019in}{2.426171in}}%
\pgfpathlineto{\pgfqpoint{2.923198in}{2.429547in}}%
\pgfpathlineto{\pgfqpoint{2.922884in}{2.429923in}}%
\pgfpathlineto{\pgfqpoint{2.920063in}{2.433299in}}%
\pgfpathlineto{\pgfqpoint{2.919750in}{2.433674in}}%
\pgfpathlineto{\pgfqpoint{2.916928in}{2.437050in}}%
\pgfpathlineto{\pgfqpoint{2.916615in}{2.437426in}}%
\pgfpathlineto{\pgfqpoint{2.913794in}{2.440802in}}%
\pgfpathlineto{\pgfqpoint{2.913480in}{2.441177in}}%
\pgfpathlineto{\pgfqpoint{2.910659in}{2.444553in}}%
\pgfpathlineto{\pgfqpoint{2.910345in}{2.444929in}}%
\pgfpathlineto{\pgfqpoint{2.907524in}{2.448305in}}%
\pgfpathlineto{\pgfqpoint{2.907211in}{2.448680in}}%
\pgfpathlineto{\pgfqpoint{2.907211in}{2.452432in}}%
\pgfpathlineto{\pgfqpoint{2.904389in}{2.455808in}}%
\pgfpathlineto{\pgfqpoint{2.904076in}{2.456183in}}%
\pgfpathlineto{\pgfqpoint{2.901255in}{2.459560in}}%
\pgfpathlineto{\pgfqpoint{2.900941in}{2.459935in}}%
\pgfpathlineto{\pgfqpoint{2.898120in}{2.463311in}}%
\pgfpathlineto{\pgfqpoint{2.897806in}{2.463686in}}%
\pgfpathlineto{\pgfqpoint{2.894985in}{2.467063in}}%
\pgfpathlineto{\pgfqpoint{2.894672in}{2.467438in}}%
\pgfpathlineto{\pgfqpoint{2.891850in}{2.470814in}}%
\pgfpathlineto{\pgfqpoint{2.891537in}{2.471189in}}%
\pgfpathlineto{\pgfqpoint{2.888716in}{2.474566in}}%
\pgfpathlineto{\pgfqpoint{2.888402in}{2.474941in}}%
\pgfpathlineto{\pgfqpoint{2.885581in}{2.478317in}}%
\pgfpathlineto{\pgfqpoint{2.885267in}{2.478692in}}%
\pgfpathlineto{\pgfqpoint{2.882446in}{2.482069in}}%
\pgfpathlineto{\pgfqpoint{2.882133in}{2.482444in}}%
\pgfpathlineto{\pgfqpoint{2.879311in}{2.485820in}}%
\pgfpathlineto{\pgfqpoint{2.878998in}{2.486196in}}%
\pgfpathlineto{\pgfqpoint{2.876177in}{2.489572in}}%
\pgfpathlineto{\pgfqpoint{2.875863in}{2.489947in}}%
\pgfpathlineto{\pgfqpoint{2.873042in}{2.493323in}}%
\pgfpathlineto{\pgfqpoint{2.872728in}{2.493699in}}%
\pgfpathlineto{\pgfqpoint{2.869907in}{2.497075in}}%
\pgfpathlineto{\pgfqpoint{2.869594in}{2.497450in}}%
\pgfpathlineto{\pgfqpoint{2.866772in}{2.500827in}}%
\pgfpathlineto{\pgfqpoint{2.866459in}{2.501202in}}%
\pgfpathlineto{\pgfqpoint{2.863638in}{2.504578in}}%
\pgfpathlineto{\pgfqpoint{2.863324in}{2.504953in}}%
\pgfpathlineto{\pgfqpoint{2.860503in}{2.508330in}}%
\pgfpathlineto{\pgfqpoint{2.860189in}{2.508705in}}%
\pgfpathlineto{\pgfqpoint{2.860189in}{2.512456in}}%
\pgfpathlineto{\pgfqpoint{2.857368in}{2.515833in}}%
\pgfpathlineto{\pgfqpoint{2.857055in}{2.516208in}}%
\pgfpathlineto{\pgfqpoint{2.854233in}{2.519584in}}%
\pgfpathlineto{\pgfqpoint{2.853920in}{2.519959in}}%
\pgfpathlineto{\pgfqpoint{2.851099in}{2.523336in}}%
\pgfpathlineto{\pgfqpoint{2.850785in}{2.523711in}}%
\pgfpathlineto{\pgfqpoint{2.847964in}{2.527087in}}%
\pgfpathlineto{\pgfqpoint{2.847650in}{2.527462in}}%
\pgfpathlineto{\pgfqpoint{2.844829in}{2.530839in}}%
\pgfpathlineto{\pgfqpoint{2.844516in}{2.531214in}}%
\pgfpathlineto{\pgfqpoint{2.841694in}{2.534590in}}%
\pgfpathlineto{\pgfqpoint{2.841381in}{2.534965in}}%
\pgfpathlineto{\pgfqpoint{2.838560in}{2.538342in}}%
\pgfpathlineto{\pgfqpoint{2.838246in}{2.538717in}}%
\pgfpathlineto{\pgfqpoint{2.835425in}{2.542093in}}%
\pgfpathlineto{\pgfqpoint{2.835111in}{2.542469in}}%
\pgfpathlineto{\pgfqpoint{2.832290in}{2.545845in}}%
\pgfpathlineto{\pgfqpoint{2.831977in}{2.546220in}}%
\pgfpathlineto{\pgfqpoint{2.829155in}{2.549596in}}%
\pgfpathlineto{\pgfqpoint{2.828842in}{2.549972in}}%
\pgfpathlineto{\pgfqpoint{2.826021in}{2.553348in}}%
\pgfpathlineto{\pgfqpoint{2.825707in}{2.553723in}}%
\pgfpathlineto{\pgfqpoint{2.822886in}{2.557100in}}%
\pgfpathlineto{\pgfqpoint{2.822573in}{2.557475in}}%
\pgfpathlineto{\pgfqpoint{2.819751in}{2.560851in}}%
\pgfpathlineto{\pgfqpoint{2.819438in}{2.561226in}}%
\pgfpathlineto{\pgfqpoint{2.816617in}{2.564603in}}%
\pgfpathlineto{\pgfqpoint{2.816303in}{2.564978in}}%
\pgfpathlineto{\pgfqpoint{2.813482in}{2.568354in}}%
\pgfpathlineto{\pgfqpoint{2.813168in}{2.568729in}}%
\pgfpathlineto{\pgfqpoint{2.810347in}{2.572106in}}%
\pgfpathlineto{\pgfqpoint{2.810034in}{2.572481in}}%
\pgfpathlineto{\pgfqpoint{2.810034in}{2.576232in}}%
\pgfpathlineto{\pgfqpoint{2.807212in}{2.579609in}}%
\pgfpathlineto{\pgfqpoint{2.806899in}{2.579984in}}%
\pgfpathlineto{\pgfqpoint{2.804078in}{2.583360in}}%
\pgfpathlineto{\pgfqpoint{2.803764in}{2.583735in}}%
\pgfpathlineto{\pgfqpoint{2.800943in}{2.587112in}}%
\pgfpathlineto{\pgfqpoint{2.800629in}{2.587487in}}%
\pgfpathlineto{\pgfqpoint{2.797808in}{2.590863in}}%
\pgfpathlineto{\pgfqpoint{2.797495in}{2.591238in}}%
\pgfpathlineto{\pgfqpoint{2.794673in}{2.594615in}}%
\pgfpathlineto{\pgfqpoint{2.794360in}{2.594990in}}%
\pgfpathlineto{\pgfqpoint{2.791539in}{2.598366in}}%
\pgfpathlineto{\pgfqpoint{2.791225in}{2.598742in}}%
\pgfpathlineto{\pgfqpoint{2.788404in}{2.602118in}}%
\pgfpathlineto{\pgfqpoint{2.788090in}{2.602493in}}%
\pgfpathlineto{\pgfqpoint{2.785269in}{2.605869in}}%
\pgfpathlineto{\pgfqpoint{2.784956in}{2.606245in}}%
\pgfpathlineto{\pgfqpoint{2.782134in}{2.609621in}}%
\pgfpathlineto{\pgfqpoint{2.781821in}{2.609996in}}%
\pgfpathlineto{\pgfqpoint{2.779000in}{2.613373in}}%
\pgfpathlineto{\pgfqpoint{2.778686in}{2.613748in}}%
\pgfpathlineto{\pgfqpoint{2.775865in}{2.617124in}}%
\pgfpathlineto{\pgfqpoint{2.775551in}{2.617499in}}%
\pgfpathlineto{\pgfqpoint{2.772730in}{2.620876in}}%
\pgfpathlineto{\pgfqpoint{2.772417in}{2.621251in}}%
\pgfpathlineto{\pgfqpoint{2.769595in}{2.624627in}}%
\pgfpathlineto{\pgfqpoint{2.769282in}{2.625002in}}%
\pgfpathlineto{\pgfqpoint{2.766461in}{2.628379in}}%
\pgfpathlineto{\pgfqpoint{2.766147in}{2.628754in}}%
\pgfpathlineto{\pgfqpoint{2.763326in}{2.632130in}}%
\pgfpathlineto{\pgfqpoint{2.763012in}{2.632505in}}%
\pgfpathlineto{\pgfqpoint{2.763012in}{2.636257in}}%
\pgfpathlineto{\pgfqpoint{2.760191in}{2.639633in}}%
\pgfpathlineto{\pgfqpoint{2.759878in}{2.640008in}}%
\pgfpathlineto{\pgfqpoint{2.757056in}{2.643385in}}%
\pgfpathlineto{\pgfqpoint{2.756743in}{2.643760in}}%
\pgfpathlineto{\pgfqpoint{2.753922in}{2.647136in}}%
\pgfpathlineto{\pgfqpoint{2.753608in}{2.647512in}}%
\pgfpathlineto{\pgfqpoint{2.750787in}{2.650888in}}%
\pgfpathlineto{\pgfqpoint{2.750473in}{2.651263in}}%
\pgfpathlineto{\pgfqpoint{2.747652in}{2.654639in}}%
\pgfpathlineto{\pgfqpoint{2.747339in}{2.655015in}}%
\pgfpathlineto{\pgfqpoint{2.744517in}{2.658391in}}%
\pgfpathlineto{\pgfqpoint{2.744204in}{2.658766in}}%
\pgfpathlineto{\pgfqpoint{2.741383in}{2.662142in}}%
\pgfpathlineto{\pgfqpoint{2.741069in}{2.662518in}}%
\pgfpathlineto{\pgfqpoint{2.738248in}{2.665894in}}%
\pgfpathlineto{\pgfqpoint{2.737934in}{2.666269in}}%
\pgfpathlineto{\pgfqpoint{2.735113in}{2.669646in}}%
\pgfpathlineto{\pgfqpoint{2.734800in}{2.670021in}}%
\pgfpathlineto{\pgfqpoint{2.731978in}{2.673397in}}%
\pgfpathlineto{\pgfqpoint{2.731665in}{2.673772in}}%
\pgfpathlineto{\pgfqpoint{2.728844in}{2.677149in}}%
\pgfpathlineto{\pgfqpoint{2.728530in}{2.677524in}}%
\pgfpathlineto{\pgfqpoint{2.725709in}{2.680900in}}%
\pgfpathlineto{\pgfqpoint{2.725396in}{2.681275in}}%
\pgfpathlineto{\pgfqpoint{2.722574in}{2.684652in}}%
\pgfpathlineto{\pgfqpoint{2.722261in}{2.685027in}}%
\pgfpathlineto{\pgfqpoint{2.719439in}{2.688403in}}%
\pgfpathlineto{\pgfqpoint{2.719126in}{2.688778in}}%
\pgfpathlineto{\pgfqpoint{2.716305in}{2.692155in}}%
\pgfpathlineto{\pgfqpoint{2.715991in}{2.692530in}}%
\pgfpathlineto{\pgfqpoint{2.713170in}{2.695906in}}%
\pgfpathlineto{\pgfqpoint{2.712857in}{2.696281in}}%
\pgfpathlineto{\pgfqpoint{2.712857in}{2.700033in}}%
\pgfpathlineto{\pgfqpoint{2.710035in}{2.703409in}}%
\pgfpathlineto{\pgfqpoint{2.709722in}{2.703785in}}%
\pgfpathlineto{\pgfqpoint{2.706901in}{2.707161in}}%
\pgfpathlineto{\pgfqpoint{2.706587in}{2.707536in}}%
\pgfpathlineto{\pgfqpoint{2.703766in}{2.710912in}}%
\pgfpathlineto{\pgfqpoint{2.703452in}{2.711288in}}%
\pgfpathlineto{\pgfqpoint{2.700631in}{2.714664in}}%
\pgfpathlineto{\pgfqpoint{2.700318in}{2.715039in}}%
\pgfpathlineto{\pgfqpoint{2.697496in}{2.718416in}}%
\pgfpathlineto{\pgfqpoint{2.697183in}{2.718791in}}%
\pgfpathlineto{\pgfqpoint{2.694362in}{2.722167in}}%
\pgfpathlineto{\pgfqpoint{2.694048in}{2.722542in}}%
\pgfpathlineto{\pgfqpoint{2.691227in}{2.725919in}}%
\pgfpathlineto{\pgfqpoint{2.690913in}{2.726294in}}%
\pgfpathlineto{\pgfqpoint{2.688092in}{2.729670in}}%
\pgfpathlineto{\pgfqpoint{2.687779in}{2.730045in}}%
\pgfpathlineto{\pgfqpoint{2.684957in}{2.733422in}}%
\pgfpathlineto{\pgfqpoint{2.684644in}{2.733797in}}%
\pgfpathlineto{\pgfqpoint{2.681823in}{2.737173in}}%
\pgfpathlineto{\pgfqpoint{2.681509in}{2.737548in}}%
\pgfpathlineto{\pgfqpoint{2.678688in}{2.740925in}}%
\pgfpathlineto{\pgfqpoint{2.678374in}{2.741300in}}%
\pgfpathlineto{\pgfqpoint{2.675553in}{2.744676in}}%
\pgfpathlineto{\pgfqpoint{2.675240in}{2.745051in}}%
\pgfpathlineto{\pgfqpoint{2.672418in}{2.748428in}}%
\pgfpathlineto{\pgfqpoint{2.669284in}{2.748428in}}%
\pgfpathlineto{\pgfqpoint{2.666149in}{2.748428in}}%
\pgfpathlineto{\pgfqpoint{2.663014in}{2.748428in}}%
\pgfpathlineto{\pgfqpoint{2.659879in}{2.748428in}}%
\pgfpathlineto{\pgfqpoint{2.657058in}{2.745051in}}%
\pgfpathlineto{\pgfqpoint{2.656745in}{2.744676in}}%
\pgfpathlineto{\pgfqpoint{2.653610in}{2.744676in}}%
\pgfpathlineto{\pgfqpoint{2.650475in}{2.744676in}}%
\pgfpathlineto{\pgfqpoint{2.647340in}{2.744676in}}%
\pgfpathlineto{\pgfqpoint{2.644519in}{2.741300in}}%
\pgfpathlineto{\pgfqpoint{2.644206in}{2.740925in}}%
\pgfpathlineto{\pgfqpoint{2.641071in}{2.740925in}}%
\pgfpathlineto{\pgfqpoint{2.637936in}{2.740925in}}%
\pgfpathlineto{\pgfqpoint{2.634801in}{2.740925in}}%
\pgfpathlineto{\pgfqpoint{2.631980in}{2.737548in}}%
\pgfpathlineto{\pgfqpoint{2.631667in}{2.737173in}}%
\pgfpathlineto{\pgfqpoint{2.628532in}{2.737173in}}%
\pgfpathlineto{\pgfqpoint{2.625397in}{2.737173in}}%
\pgfpathlineto{\pgfqpoint{2.622262in}{2.737173in}}%
\pgfpathlineto{\pgfqpoint{2.619128in}{2.737173in}}%
\pgfpathlineto{\pgfqpoint{2.616306in}{2.733797in}}%
\pgfpathlineto{\pgfqpoint{2.615993in}{2.733422in}}%
\pgfpathlineto{\pgfqpoint{2.612858in}{2.733422in}}%
\pgfpathlineto{\pgfqpoint{2.609724in}{2.733422in}}%
\pgfpathlineto{\pgfqpoint{2.606589in}{2.733422in}}%
\pgfpathlineto{\pgfqpoint{2.603768in}{2.730045in}}%
\pgfpathlineto{\pgfqpoint{2.603454in}{2.729670in}}%
\pgfpathlineto{\pgfqpoint{2.600319in}{2.729670in}}%
\pgfpathlineto{\pgfqpoint{2.597185in}{2.729670in}}%
\pgfpathlineto{\pgfqpoint{2.594050in}{2.729670in}}%
\pgfpathlineto{\pgfqpoint{2.591229in}{2.726294in}}%
\pgfpathlineto{\pgfqpoint{2.590915in}{2.725919in}}%
\pgfpathlineto{\pgfqpoint{2.587780in}{2.725919in}}%
\pgfpathlineto{\pgfqpoint{2.584646in}{2.725919in}}%
\pgfpathlineto{\pgfqpoint{2.581511in}{2.725919in}}%
\pgfpathlineto{\pgfqpoint{2.578690in}{2.722542in}}%
\pgfpathlineto{\pgfqpoint{2.578376in}{2.722167in}}%
\pgfpathlineto{\pgfqpoint{2.575241in}{2.722167in}}%
\pgfpathlineto{\pgfqpoint{2.572107in}{2.722167in}}%
\pgfpathlineto{\pgfqpoint{2.568972in}{2.722167in}}%
\pgfpathlineto{\pgfqpoint{2.566151in}{2.718791in}}%
\pgfpathlineto{\pgfqpoint{2.565837in}{2.718416in}}%
\pgfpathlineto{\pgfqpoint{2.562702in}{2.718416in}}%
\pgfpathlineto{\pgfqpoint{2.559568in}{2.718416in}}%
\pgfpathlineto{\pgfqpoint{2.556433in}{2.718416in}}%
\pgfpathlineto{\pgfqpoint{2.553298in}{2.718416in}}%
\pgfpathlineto{\pgfqpoint{2.550477in}{2.715039in}}%
\pgfpathlineto{\pgfqpoint{2.550163in}{2.714664in}}%
\pgfpathlineto{\pgfqpoint{2.547029in}{2.714664in}}%
\pgfpathlineto{\pgfqpoint{2.543894in}{2.714664in}}%
\pgfpathlineto{\pgfqpoint{2.540759in}{2.714664in}}%
\pgfpathlineto{\pgfqpoint{2.537938in}{2.711288in}}%
\pgfpathlineto{\pgfqpoint{2.537624in}{2.710912in}}%
\pgfpathlineto{\pgfqpoint{2.534490in}{2.710912in}}%
\pgfpathlineto{\pgfqpoint{2.531355in}{2.710912in}}%
\pgfpathlineto{\pgfqpoint{2.528220in}{2.710912in}}%
\pgfpathlineto{\pgfqpoint{2.525399in}{2.707536in}}%
\pgfpathlineto{\pgfqpoint{2.525085in}{2.707161in}}%
\pgfpathlineto{\pgfqpoint{2.521951in}{2.707161in}}%
\pgfpathlineto{\pgfqpoint{2.518816in}{2.707161in}}%
\pgfpathlineto{\pgfqpoint{2.515681in}{2.707161in}}%
\pgfpathlineto{\pgfqpoint{2.512860in}{2.703785in}}%
\pgfpathlineto{\pgfqpoint{2.512547in}{2.703409in}}%
\pgfpathlineto{\pgfqpoint{2.509412in}{2.703409in}}%
\pgfpathlineto{\pgfqpoint{2.506277in}{2.703409in}}%
\pgfpathlineto{\pgfqpoint{2.503142in}{2.703409in}}%
\pgfpathlineto{\pgfqpoint{2.500008in}{2.703409in}}%
\pgfpathlineto{\pgfqpoint{2.497186in}{2.700033in}}%
\pgfpathlineto{\pgfqpoint{2.496873in}{2.699658in}}%
\pgfpathlineto{\pgfqpoint{2.493738in}{2.699658in}}%
\pgfpathlineto{\pgfqpoint{2.490603in}{2.699658in}}%
\pgfpathlineto{\pgfqpoint{2.487469in}{2.699658in}}%
\pgfpathlineto{\pgfqpoint{2.484647in}{2.696281in}}%
\pgfpathlineto{\pgfqpoint{2.484334in}{2.695906in}}%
\pgfpathlineto{\pgfqpoint{2.481199in}{2.695906in}}%
\pgfpathlineto{\pgfqpoint{2.478064in}{2.695906in}}%
\pgfpathlineto{\pgfqpoint{2.474930in}{2.695906in}}%
\pgfpathlineto{\pgfqpoint{2.472108in}{2.692530in}}%
\pgfpathlineto{\pgfqpoint{2.471795in}{2.692155in}}%
\pgfpathlineto{\pgfqpoint{2.468660in}{2.692155in}}%
\pgfpathlineto{\pgfqpoint{2.465525in}{2.692155in}}%
\pgfpathlineto{\pgfqpoint{2.462391in}{2.692155in}}%
\pgfpathlineto{\pgfqpoint{2.459569in}{2.688778in}}%
\pgfpathlineto{\pgfqpoint{2.459256in}{2.688403in}}%
\pgfpathlineto{\pgfqpoint{2.456121in}{2.688403in}}%
\pgfpathlineto{\pgfqpoint{2.452986in}{2.688403in}}%
\pgfpathlineto{\pgfqpoint{2.449852in}{2.688403in}}%
\pgfpathlineto{\pgfqpoint{2.447030in}{2.685027in}}%
\pgfpathlineto{\pgfqpoint{2.446717in}{2.684652in}}%
\pgfpathlineto{\pgfqpoint{2.443582in}{2.684652in}}%
\pgfpathlineto{\pgfqpoint{2.440447in}{2.684652in}}%
\pgfpathlineto{\pgfqpoint{2.437313in}{2.684652in}}%
\pgfpathlineto{\pgfqpoint{2.434178in}{2.684652in}}%
\pgfpathlineto{\pgfqpoint{2.431357in}{2.681275in}}%
\pgfpathlineto{\pgfqpoint{2.431043in}{2.680900in}}%
\pgfpathlineto{\pgfqpoint{2.427908in}{2.680900in}}%
\pgfpathlineto{\pgfqpoint{2.424774in}{2.680900in}}%
\pgfpathlineto{\pgfqpoint{2.421639in}{2.680900in}}%
\pgfpathlineto{\pgfqpoint{2.418818in}{2.677524in}}%
\pgfpathlineto{\pgfqpoint{2.418504in}{2.677149in}}%
\pgfpathlineto{\pgfqpoint{2.415369in}{2.677149in}}%
\pgfpathlineto{\pgfqpoint{2.412235in}{2.677149in}}%
\pgfpathlineto{\pgfqpoint{2.409100in}{2.677149in}}%
\pgfpathlineto{\pgfqpoint{2.406279in}{2.673772in}}%
\pgfpathlineto{\pgfqpoint{2.405965in}{2.673397in}}%
\pgfpathlineto{\pgfqpoint{2.402831in}{2.673397in}}%
\pgfpathlineto{\pgfqpoint{2.399696in}{2.673397in}}%
\pgfpathlineto{\pgfqpoint{2.396561in}{2.673397in}}%
\pgfpathlineto{\pgfqpoint{2.393740in}{2.670021in}}%
\pgfpathlineto{\pgfqpoint{2.393426in}{2.669646in}}%
\pgfpathlineto{\pgfqpoint{2.390292in}{2.669646in}}%
\pgfpathlineto{\pgfqpoint{2.387157in}{2.669646in}}%
\pgfpathlineto{\pgfqpoint{2.384022in}{2.669646in}}%
\pgfpathlineto{\pgfqpoint{2.380887in}{2.669646in}}%
\pgfpathlineto{\pgfqpoint{2.378066in}{2.666269in}}%
\pgfpathlineto{\pgfqpoint{2.377753in}{2.665894in}}%
\pgfpathlineto{\pgfqpoint{2.374618in}{2.665894in}}%
\pgfpathlineto{\pgfqpoint{2.371483in}{2.665894in}}%
\pgfpathlineto{\pgfqpoint{2.368348in}{2.665894in}}%
\pgfpathlineto{\pgfqpoint{2.365527in}{2.662518in}}%
\pgfpathlineto{\pgfqpoint{2.365214in}{2.662142in}}%
\pgfpathlineto{\pgfqpoint{2.362079in}{2.662142in}}%
\pgfpathlineto{\pgfqpoint{2.358944in}{2.662142in}}%
\pgfpathlineto{\pgfqpoint{2.355809in}{2.662142in}}%
\pgfpathlineto{\pgfqpoint{2.352988in}{2.658766in}}%
\pgfpathlineto{\pgfqpoint{2.352675in}{2.658391in}}%
\pgfpathlineto{\pgfqpoint{2.349540in}{2.658391in}}%
\pgfpathlineto{\pgfqpoint{2.346405in}{2.658391in}}%
\pgfpathlineto{\pgfqpoint{2.343270in}{2.658391in}}%
\pgfpathlineto{\pgfqpoint{2.340449in}{2.655015in}}%
\pgfpathlineto{\pgfqpoint{2.340136in}{2.654639in}}%
\pgfpathlineto{\pgfqpoint{2.337001in}{2.654639in}}%
\pgfpathlineto{\pgfqpoint{2.333866in}{2.654639in}}%
\pgfpathlineto{\pgfqpoint{2.330731in}{2.654639in}}%
\pgfpathlineto{\pgfqpoint{2.327597in}{2.654639in}}%
\pgfpathlineto{\pgfqpoint{2.324775in}{2.651263in}}%
\pgfpathlineto{\pgfqpoint{2.324462in}{2.650888in}}%
\pgfpathlineto{\pgfqpoint{2.321327in}{2.650888in}}%
\pgfpathlineto{\pgfqpoint{2.318192in}{2.650888in}}%
\pgfpathlineto{\pgfqpoint{2.315058in}{2.650888in}}%
\pgfpathlineto{\pgfqpoint{2.312236in}{2.647512in}}%
\pgfpathlineto{\pgfqpoint{2.311923in}{2.647136in}}%
\pgfpathlineto{\pgfqpoint{2.308788in}{2.647136in}}%
\pgfpathlineto{\pgfqpoint{2.305654in}{2.647136in}}%
\pgfpathlineto{\pgfqpoint{2.302519in}{2.647136in}}%
\pgfpathlineto{\pgfqpoint{2.299698in}{2.643760in}}%
\pgfpathlineto{\pgfqpoint{2.299384in}{2.643385in}}%
\pgfpathlineto{\pgfqpoint{2.296249in}{2.643385in}}%
\pgfpathlineto{\pgfqpoint{2.293115in}{2.643385in}}%
\pgfpathlineto{\pgfqpoint{2.289980in}{2.643385in}}%
\pgfpathlineto{\pgfqpoint{2.287159in}{2.640008in}}%
\pgfpathlineto{\pgfqpoint{2.286845in}{2.639633in}}%
\pgfpathlineto{\pgfqpoint{2.283710in}{2.639633in}}%
\pgfpathlineto{\pgfqpoint{2.280576in}{2.639633in}}%
\pgfpathlineto{\pgfqpoint{2.277441in}{2.639633in}}%
\pgfpathlineto{\pgfqpoint{2.274620in}{2.636257in}}%
\pgfpathlineto{\pgfqpoint{2.274306in}{2.635882in}}%
\pgfpathlineto{\pgfqpoint{2.271171in}{2.635882in}}%
\pgfpathlineto{\pgfqpoint{2.268037in}{2.635882in}}%
\pgfpathlineto{\pgfqpoint{2.264902in}{2.635882in}}%
\pgfpathlineto{\pgfqpoint{2.261767in}{2.635882in}}%
\pgfpathlineto{\pgfqpoint{2.258946in}{2.632505in}}%
\pgfpathlineto{\pgfqpoint{2.258632in}{2.632130in}}%
\pgfpathlineto{\pgfqpoint{2.255498in}{2.632130in}}%
\pgfpathlineto{\pgfqpoint{2.252363in}{2.632130in}}%
\pgfpathlineto{\pgfqpoint{2.249228in}{2.632130in}}%
\pgfpathlineto{\pgfqpoint{2.246407in}{2.628754in}}%
\pgfpathlineto{\pgfqpoint{2.246093in}{2.628379in}}%
\pgfpathlineto{\pgfqpoint{2.242959in}{2.628379in}}%
\pgfpathlineto{\pgfqpoint{2.239824in}{2.628379in}}%
\pgfpathlineto{\pgfqpoint{2.236689in}{2.628379in}}%
\pgfpathlineto{\pgfqpoint{2.233868in}{2.625002in}}%
\pgfpathlineto{\pgfqpoint{2.233554in}{2.624627in}}%
\pgfpathlineto{\pgfqpoint{2.230420in}{2.624627in}}%
\pgfpathlineto{\pgfqpoint{2.227285in}{2.624627in}}%
\pgfpathlineto{\pgfqpoint{2.224150in}{2.624627in}}%
\pgfpathlineto{\pgfqpoint{2.221329in}{2.621251in}}%
\pgfpathlineto{\pgfqpoint{2.221015in}{2.620876in}}%
\pgfpathlineto{\pgfqpoint{2.217881in}{2.620876in}}%
\pgfpathlineto{\pgfqpoint{2.214746in}{2.620876in}}%
\pgfpathlineto{\pgfqpoint{2.211611in}{2.620876in}}%
\pgfpathlineto{\pgfqpoint{2.208477in}{2.620876in}}%
\pgfpathlineto{\pgfqpoint{2.205655in}{2.617499in}}%
\pgfpathlineto{\pgfqpoint{2.205342in}{2.617124in}}%
\pgfpathlineto{\pgfqpoint{2.202207in}{2.617124in}}%
\pgfpathlineto{\pgfqpoint{2.199072in}{2.617124in}}%
\pgfpathlineto{\pgfqpoint{2.195938in}{2.617124in}}%
\pgfpathlineto{\pgfqpoint{2.193116in}{2.613748in}}%
\pgfpathlineto{\pgfqpoint{2.192803in}{2.613373in}}%
\pgfpathlineto{\pgfqpoint{2.189668in}{2.613373in}}%
\pgfpathlineto{\pgfqpoint{2.186533in}{2.613373in}}%
\pgfpathlineto{\pgfqpoint{2.183399in}{2.613373in}}%
\pgfpathlineto{\pgfqpoint{2.180577in}{2.609996in}}%
\pgfpathlineto{\pgfqpoint{2.180264in}{2.609621in}}%
\pgfpathlineto{\pgfqpoint{2.177129in}{2.609621in}}%
\pgfpathlineto{\pgfqpoint{2.173994in}{2.609621in}}%
\pgfpathlineto{\pgfqpoint{2.170860in}{2.609621in}}%
\pgfpathlineto{\pgfqpoint{2.168038in}{2.606245in}}%
\pgfpathlineto{\pgfqpoint{2.167725in}{2.605869in}}%
\pgfpathlineto{\pgfqpoint{2.164590in}{2.605869in}}%
\pgfpathlineto{\pgfqpoint{2.161455in}{2.605869in}}%
\pgfpathlineto{\pgfqpoint{2.158321in}{2.605869in}}%
\pgfpathlineto{\pgfqpoint{2.155186in}{2.605869in}}%
\pgfpathlineto{\pgfqpoint{2.152365in}{2.602493in}}%
\pgfpathlineto{\pgfqpoint{2.152051in}{2.602118in}}%
\pgfpathlineto{\pgfqpoint{2.148916in}{2.602118in}}%
\pgfpathlineto{\pgfqpoint{2.145782in}{2.602118in}}%
\pgfpathlineto{\pgfqpoint{2.142647in}{2.602118in}}%
\pgfpathlineto{\pgfqpoint{2.139826in}{2.598742in}}%
\pgfpathlineto{\pgfqpoint{2.139512in}{2.598366in}}%
\pgfpathlineto{\pgfqpoint{2.136377in}{2.598366in}}%
\pgfpathlineto{\pgfqpoint{2.133243in}{2.598366in}}%
\pgfpathlineto{\pgfqpoint{2.130108in}{2.598366in}}%
\pgfpathlineto{\pgfqpoint{2.127287in}{2.594990in}}%
\pgfpathlineto{\pgfqpoint{2.126973in}{2.594615in}}%
\pgfpathlineto{\pgfqpoint{2.123838in}{2.594615in}}%
\pgfpathlineto{\pgfqpoint{2.120704in}{2.594615in}}%
\pgfpathlineto{\pgfqpoint{2.117569in}{2.594615in}}%
\pgfpathlineto{\pgfqpoint{2.114748in}{2.591238in}}%
\pgfpathlineto{\pgfqpoint{2.114434in}{2.590863in}}%
\pgfpathlineto{\pgfqpoint{2.111299in}{2.590863in}}%
\pgfpathlineto{\pgfqpoint{2.108165in}{2.590863in}}%
\pgfpathlineto{\pgfqpoint{2.105030in}{2.590863in}}%
\pgfpathlineto{\pgfqpoint{2.102209in}{2.587487in}}%
\pgfpathlineto{\pgfqpoint{2.101895in}{2.587112in}}%
\pgfpathlineto{\pgfqpoint{2.098761in}{2.587112in}}%
\pgfpathlineto{\pgfqpoint{2.095626in}{2.587112in}}%
\pgfpathlineto{\pgfqpoint{2.092491in}{2.587112in}}%
\pgfpathlineto{\pgfqpoint{2.089356in}{2.587112in}}%
\pgfpathlineto{\pgfqpoint{2.086535in}{2.583735in}}%
\pgfpathlineto{\pgfqpoint{2.086222in}{2.583360in}}%
\pgfpathlineto{\pgfqpoint{2.083087in}{2.583360in}}%
\pgfpathlineto{\pgfqpoint{2.079952in}{2.583360in}}%
\pgfpathlineto{\pgfqpoint{2.076817in}{2.583360in}}%
\pgfpathlineto{\pgfqpoint{2.073996in}{2.579984in}}%
\pgfpathlineto{\pgfqpoint{2.073683in}{2.579609in}}%
\pgfpathlineto{\pgfqpoint{2.070548in}{2.579609in}}%
\pgfpathlineto{\pgfqpoint{2.067413in}{2.579609in}}%
\pgfpathlineto{\pgfqpoint{2.064278in}{2.579609in}}%
\pgfpathlineto{\pgfqpoint{2.061457in}{2.576232in}}%
\pgfpathlineto{\pgfqpoint{2.061144in}{2.575857in}}%
\pgfpathlineto{\pgfqpoint{2.058009in}{2.575857in}}%
\pgfpathlineto{\pgfqpoint{2.054874in}{2.575857in}}%
\pgfpathlineto{\pgfqpoint{2.051739in}{2.575857in}}%
\pgfpathlineto{\pgfqpoint{2.048918in}{2.572481in}}%
\pgfpathlineto{\pgfqpoint{2.048605in}{2.572106in}}%
\pgfpathlineto{\pgfqpoint{2.045470in}{2.572106in}}%
\pgfpathlineto{\pgfqpoint{2.042335in}{2.572106in}}%
\pgfpathlineto{\pgfqpoint{2.039200in}{2.572106in}}%
\pgfpathlineto{\pgfqpoint{2.036066in}{2.572106in}}%
\pgfpathlineto{\pgfqpoint{2.033244in}{2.568729in}}%
\pgfpathlineto{\pgfqpoint{2.032931in}{2.568354in}}%
\pgfpathlineto{\pgfqpoint{2.029796in}{2.568354in}}%
\pgfpathlineto{\pgfqpoint{2.026661in}{2.568354in}}%
\pgfpathlineto{\pgfqpoint{2.023527in}{2.568354in}}%
\pgfpathlineto{\pgfqpoint{2.020705in}{2.564978in}}%
\pgfpathlineto{\pgfqpoint{2.020392in}{2.564603in}}%
\pgfpathlineto{\pgfqpoint{2.017257in}{2.564603in}}%
\pgfpathlineto{\pgfqpoint{2.014122in}{2.564603in}}%
\pgfpathlineto{\pgfqpoint{2.010988in}{2.564603in}}%
\pgfpathlineto{\pgfqpoint{2.008166in}{2.561226in}}%
\pgfpathlineto{\pgfqpoint{2.007853in}{2.560851in}}%
\pgfpathlineto{\pgfqpoint{2.004718in}{2.560851in}}%
\pgfpathlineto{\pgfqpoint{2.001584in}{2.560851in}}%
\pgfpathlineto{\pgfqpoint{1.998449in}{2.560851in}}%
\pgfpathlineto{\pgfqpoint{1.995628in}{2.557475in}}%
\pgfpathlineto{\pgfqpoint{1.995314in}{2.557100in}}%
\pgfpathlineto{\pgfqpoint{1.992179in}{2.557100in}}%
\pgfpathlineto{\pgfqpoint{1.989045in}{2.557100in}}%
\pgfpathlineto{\pgfqpoint{1.985910in}{2.557100in}}%
\pgfpathlineto{\pgfqpoint{1.982775in}{2.557100in}}%
\pgfpathlineto{\pgfqpoint{1.979954in}{2.553723in}}%
\pgfpathlineto{\pgfqpoint{1.979640in}{2.553348in}}%
\pgfpathlineto{\pgfqpoint{1.976506in}{2.553348in}}%
\pgfpathlineto{\pgfqpoint{1.973371in}{2.553348in}}%
\pgfpathlineto{\pgfqpoint{1.970236in}{2.553348in}}%
\pgfpathlineto{\pgfqpoint{1.967415in}{2.549972in}}%
\pgfpathlineto{\pgfqpoint{1.967101in}{2.549596in}}%
\pgfpathlineto{\pgfqpoint{1.963967in}{2.549596in}}%
\pgfpathlineto{\pgfqpoint{1.960832in}{2.549596in}}%
\pgfpathlineto{\pgfqpoint{1.957697in}{2.549596in}}%
\pgfpathlineto{\pgfqpoint{1.954876in}{2.546220in}}%
\pgfpathlineto{\pgfqpoint{1.954562in}{2.545845in}}%
\pgfpathlineto{\pgfqpoint{1.951428in}{2.545845in}}%
\pgfpathlineto{\pgfqpoint{1.948293in}{2.545845in}}%
\pgfpathlineto{\pgfqpoint{1.945158in}{2.545845in}}%
\pgfpathlineto{\pgfqpoint{1.942337in}{2.542469in}}%
\pgfpathlineto{\pgfqpoint{1.942023in}{2.542093in}}%
\pgfpathlineto{\pgfqpoint{1.938889in}{2.542093in}}%
\pgfpathlineto{\pgfqpoint{1.935754in}{2.542093in}}%
\pgfpathlineto{\pgfqpoint{1.932619in}{2.542093in}}%
\pgfpathlineto{\pgfqpoint{1.929798in}{2.538717in}}%
\pgfpathlineto{\pgfqpoint{1.929484in}{2.538342in}}%
\pgfpathlineto{\pgfqpoint{1.926350in}{2.538342in}}%
\pgfpathlineto{\pgfqpoint{1.923215in}{2.538342in}}%
\pgfpathlineto{\pgfqpoint{1.920080in}{2.538342in}}%
\pgfpathlineto{\pgfqpoint{1.916945in}{2.538342in}}%
\pgfpathlineto{\pgfqpoint{1.914124in}{2.534965in}}%
\pgfpathlineto{\pgfqpoint{1.913811in}{2.534590in}}%
\pgfpathlineto{\pgfqpoint{1.910676in}{2.534590in}}%
\pgfpathlineto{\pgfqpoint{1.907541in}{2.534590in}}%
\pgfpathlineto{\pgfqpoint{1.904407in}{2.534590in}}%
\pgfpathlineto{\pgfqpoint{1.901585in}{2.531214in}}%
\pgfpathlineto{\pgfqpoint{1.901272in}{2.530839in}}%
\pgfpathlineto{\pgfqpoint{1.898137in}{2.530839in}}%
\pgfpathlineto{\pgfqpoint{1.895002in}{2.530839in}}%
\pgfpathlineto{\pgfqpoint{1.891868in}{2.530839in}}%
\pgfpathlineto{\pgfqpoint{1.889046in}{2.527462in}}%
\pgfpathlineto{\pgfqpoint{1.888733in}{2.527087in}}%
\pgfpathlineto{\pgfqpoint{1.885598in}{2.527087in}}%
\pgfpathlineto{\pgfqpoint{1.882463in}{2.527087in}}%
\pgfpathlineto{\pgfqpoint{1.879329in}{2.527087in}}%
\pgfpathlineto{\pgfqpoint{1.876507in}{2.523711in}}%
\pgfpathlineto{\pgfqpoint{1.876194in}{2.523336in}}%
\pgfpathlineto{\pgfqpoint{1.873059in}{2.523336in}}%
\pgfpathlineto{\pgfqpoint{1.869924in}{2.523336in}}%
\pgfpathlineto{\pgfqpoint{1.866790in}{2.523336in}}%
\pgfpathlineto{\pgfqpoint{1.863655in}{2.523336in}}%
\pgfpathlineto{\pgfqpoint{1.860834in}{2.519959in}}%
\pgfpathlineto{\pgfqpoint{1.860520in}{2.519584in}}%
\pgfpathlineto{\pgfqpoint{1.857385in}{2.519584in}}%
\pgfpathlineto{\pgfqpoint{1.854251in}{2.519584in}}%
\pgfpathlineto{\pgfqpoint{1.851116in}{2.519584in}}%
\pgfpathlineto{\pgfqpoint{1.848295in}{2.516208in}}%
\pgfpathlineto{\pgfqpoint{1.847981in}{2.515833in}}%
\pgfpathlineto{\pgfqpoint{1.844846in}{2.515833in}}%
\pgfpathlineto{\pgfqpoint{1.841712in}{2.515833in}}%
\pgfpathlineto{\pgfqpoint{1.838577in}{2.515833in}}%
\pgfpathlineto{\pgfqpoint{1.835756in}{2.512456in}}%
\pgfpathlineto{\pgfqpoint{1.835442in}{2.512081in}}%
\pgfpathlineto{\pgfqpoint{1.832307in}{2.512081in}}%
\pgfpathlineto{\pgfqpoint{1.829173in}{2.512081in}}%
\pgfpathlineto{\pgfqpoint{1.826038in}{2.512081in}}%
\pgfpathlineto{\pgfqpoint{1.823217in}{2.508705in}}%
\pgfpathlineto{\pgfqpoint{1.822903in}{2.508330in}}%
\pgfpathlineto{\pgfqpoint{1.819768in}{2.508330in}}%
\pgfpathlineto{\pgfqpoint{1.816634in}{2.508330in}}%
\pgfpathlineto{\pgfqpoint{1.813499in}{2.508330in}}%
\pgfpathlineto{\pgfqpoint{1.810678in}{2.504953in}}%
\pgfpathlineto{\pgfqpoint{1.810364in}{2.504578in}}%
\pgfpathlineto{\pgfqpoint{1.807229in}{2.504578in}}%
\pgfpathlineto{\pgfqpoint{1.804095in}{2.504578in}}%
\pgfpathlineto{\pgfqpoint{1.800960in}{2.504578in}}%
\pgfpathlineto{\pgfqpoint{1.797825in}{2.504578in}}%
\pgfpathlineto{\pgfqpoint{1.795004in}{2.501202in}}%
\pgfpathlineto{\pgfqpoint{1.794691in}{2.500827in}}%
\pgfpathlineto{\pgfqpoint{1.791556in}{2.500827in}}%
\pgfpathlineto{\pgfqpoint{1.788421in}{2.500827in}}%
\pgfpathlineto{\pgfqpoint{1.785286in}{2.500827in}}%
\pgfpathlineto{\pgfqpoint{1.782465in}{2.497450in}}%
\pgfpathlineto{\pgfqpoint{1.782152in}{2.497075in}}%
\pgfpathlineto{\pgfqpoint{1.779017in}{2.497075in}}%
\pgfpathlineto{\pgfqpoint{1.775882in}{2.497075in}}%
\pgfpathlineto{\pgfqpoint{1.772747in}{2.497075in}}%
\pgfpathlineto{\pgfqpoint{1.769926in}{2.493699in}}%
\pgfpathlineto{\pgfqpoint{1.769613in}{2.493323in}}%
\pgfpathlineto{\pgfqpoint{1.766478in}{2.493323in}}%
\pgfpathlineto{\pgfqpoint{1.763343in}{2.493323in}}%
\pgfpathlineto{\pgfqpoint{1.760208in}{2.493323in}}%
\pgfpathlineto{\pgfqpoint{1.757387in}{2.489947in}}%
\pgfpathlineto{\pgfqpoint{1.757074in}{2.489572in}}%
\pgfpathlineto{\pgfqpoint{1.753939in}{2.489572in}}%
\pgfpathlineto{\pgfqpoint{1.750804in}{2.489572in}}%
\pgfpathlineto{\pgfqpoint{1.747669in}{2.489572in}}%
\pgfpathlineto{\pgfqpoint{1.744535in}{2.489572in}}%
\pgfpathlineto{\pgfqpoint{1.741713in}{2.486196in}}%
\pgfpathlineto{\pgfqpoint{1.741400in}{2.485820in}}%
\pgfpathlineto{\pgfqpoint{1.738265in}{2.485820in}}%
\pgfpathlineto{\pgfqpoint{1.735130in}{2.485820in}}%
\pgfpathlineto{\pgfqpoint{1.731996in}{2.485820in}}%
\pgfpathlineto{\pgfqpoint{1.729174in}{2.482444in}}%
\pgfpathlineto{\pgfqpoint{1.728861in}{2.482069in}}%
\pgfpathlineto{\pgfqpoint{1.725726in}{2.482069in}}%
\pgfpathlineto{\pgfqpoint{1.722591in}{2.482069in}}%
\pgfpathlineto{\pgfqpoint{1.719457in}{2.482069in}}%
\pgfpathlineto{\pgfqpoint{1.716635in}{2.478692in}}%
\pgfpathlineto{\pgfqpoint{1.716322in}{2.478317in}}%
\pgfpathlineto{\pgfqpoint{1.713187in}{2.478317in}}%
\pgfpathlineto{\pgfqpoint{1.710052in}{2.478317in}}%
\pgfpathlineto{\pgfqpoint{1.706918in}{2.478317in}}%
\pgfpathlineto{\pgfqpoint{1.704096in}{2.474941in}}%
\pgfpathlineto{\pgfqpoint{1.703783in}{2.474566in}}%
\pgfpathlineto{\pgfqpoint{1.700648in}{2.474566in}}%
\pgfpathlineto{\pgfqpoint{1.697514in}{2.474566in}}%
\pgfpathlineto{\pgfqpoint{1.694379in}{2.474566in}}%
\pgfpathlineto{\pgfqpoint{1.691244in}{2.474566in}}%
\pgfpathlineto{\pgfqpoint{1.688423in}{2.471189in}}%
\pgfpathlineto{\pgfqpoint{1.688109in}{2.470814in}}%
\pgfpathlineto{\pgfqpoint{1.684975in}{2.470814in}}%
\pgfpathlineto{\pgfqpoint{1.681840in}{2.470814in}}%
\pgfpathlineto{\pgfqpoint{1.678705in}{2.470814in}}%
\pgfpathlineto{\pgfqpoint{1.675884in}{2.467438in}}%
\pgfpathlineto{\pgfqpoint{1.675570in}{2.467063in}}%
\pgfpathlineto{\pgfqpoint{1.672436in}{2.467063in}}%
\pgfpathlineto{\pgfqpoint{1.669301in}{2.467063in}}%
\pgfpathlineto{\pgfqpoint{1.666166in}{2.467063in}}%
\pgfpathlineto{\pgfqpoint{1.663345in}{2.463686in}}%
\pgfpathlineto{\pgfqpoint{1.663031in}{2.463311in}}%
\pgfpathlineto{\pgfqpoint{1.659897in}{2.463311in}}%
\pgfpathlineto{\pgfqpoint{1.656762in}{2.463311in}}%
\pgfpathlineto{\pgfqpoint{1.653627in}{2.463311in}}%
\pgfpathlineto{\pgfqpoint{1.650806in}{2.459935in}}%
\pgfpathlineto{\pgfqpoint{1.650492in}{2.459560in}}%
\pgfpathlineto{\pgfqpoint{1.647358in}{2.459560in}}%
\pgfpathlineto{\pgfqpoint{1.644223in}{2.459560in}}%
\pgfpathlineto{\pgfqpoint{1.641088in}{2.459560in}}%
\pgfpathlineto{\pgfqpoint{1.638267in}{2.456183in}}%
\pgfpathlineto{\pgfqpoint{1.637953in}{2.455808in}}%
\pgfpathlineto{\pgfqpoint{1.634819in}{2.455808in}}%
\pgfpathlineto{\pgfqpoint{1.631684in}{2.455808in}}%
\pgfpathlineto{\pgfqpoint{1.628549in}{2.455808in}}%
\pgfpathlineto{\pgfqpoint{1.625414in}{2.455808in}}%
\pgfpathlineto{\pgfqpoint{1.622593in}{2.452432in}}%
\pgfpathlineto{\pgfqpoint{1.622280in}{2.452057in}}%
\pgfpathlineto{\pgfqpoint{1.619145in}{2.452057in}}%
\pgfpathlineto{\pgfqpoint{1.616010in}{2.452057in}}%
\pgfpathlineto{\pgfqpoint{1.612875in}{2.452057in}}%
\pgfpathlineto{\pgfqpoint{1.610054in}{2.448680in}}%
\pgfpathlineto{\pgfqpoint{1.609741in}{2.448305in}}%
\pgfpathlineto{\pgfqpoint{1.606606in}{2.448305in}}%
\pgfpathlineto{\pgfqpoint{1.603471in}{2.448305in}}%
\pgfpathlineto{\pgfqpoint{1.600337in}{2.448305in}}%
\pgfpathlineto{\pgfqpoint{1.597515in}{2.444929in}}%
\pgfpathlineto{\pgfqpoint{1.597202in}{2.444553in}}%
\pgfpathlineto{\pgfqpoint{1.594067in}{2.444553in}}%
\pgfpathlineto{\pgfqpoint{1.590932in}{2.444553in}}%
\pgfpathlineto{\pgfqpoint{1.587798in}{2.444553in}}%
\pgfpathlineto{\pgfqpoint{1.584976in}{2.441177in}}%
\pgfpathlineto{\pgfqpoint{1.584663in}{2.440802in}}%
\pgfpathlineto{\pgfqpoint{1.581528in}{2.440802in}}%
\pgfpathlineto{\pgfqpoint{1.578393in}{2.440802in}}%
\pgfpathlineto{\pgfqpoint{1.575259in}{2.440802in}}%
\pgfpathlineto{\pgfqpoint{1.572124in}{2.440802in}}%
\pgfpathlineto{\pgfqpoint{1.569303in}{2.437426in}}%
\pgfpathlineto{\pgfqpoint{1.568989in}{2.437050in}}%
\pgfpathlineto{\pgfqpoint{1.565854in}{2.437050in}}%
\pgfpathlineto{\pgfqpoint{1.562720in}{2.437050in}}%
\pgfpathlineto{\pgfqpoint{1.559585in}{2.437050in}}%
\pgfpathlineto{\pgfqpoint{1.556764in}{2.433674in}}%
\pgfpathlineto{\pgfqpoint{1.556450in}{2.433299in}}%
\pgfpathlineto{\pgfqpoint{1.553315in}{2.433299in}}%
\pgfpathlineto{\pgfqpoint{1.550181in}{2.433299in}}%
\pgfpathlineto{\pgfqpoint{1.547046in}{2.433299in}}%
\pgfpathlineto{\pgfqpoint{1.544225in}{2.429923in}}%
\pgfpathlineto{\pgfqpoint{1.543911in}{2.429547in}}%
\pgfpathlineto{\pgfqpoint{1.540776in}{2.429547in}}%
\pgfpathlineto{\pgfqpoint{1.537642in}{2.429547in}}%
\pgfpathlineto{\pgfqpoint{1.534507in}{2.429547in}}%
\pgfpathlineto{\pgfqpoint{1.531686in}{2.426171in}}%
\pgfpathlineto{\pgfqpoint{1.531372in}{2.425796in}}%
\pgfpathlineto{\pgfqpoint{1.528237in}{2.425796in}}%
\pgfpathlineto{\pgfqpoint{1.525103in}{2.425796in}}%
\pgfpathlineto{\pgfqpoint{1.521968in}{2.425796in}}%
\pgfpathlineto{\pgfqpoint{1.518833in}{2.425796in}}%
\pgfpathlineto{\pgfqpoint{1.516012in}{2.422419in}}%
\pgfpathlineto{\pgfqpoint{1.515698in}{2.422044in}}%
\pgfpathlineto{\pgfqpoint{1.512564in}{2.422044in}}%
\pgfpathlineto{\pgfqpoint{1.509429in}{2.422044in}}%
\pgfpathlineto{\pgfqpoint{1.506294in}{2.422044in}}%
\pgfpathlineto{\pgfqpoint{1.503473in}{2.418668in}}%
\pgfpathlineto{\pgfqpoint{1.503159in}{2.418293in}}%
\pgfpathlineto{\pgfqpoint{1.500025in}{2.418293in}}%
\pgfpathlineto{\pgfqpoint{1.496890in}{2.418293in}}%
\pgfpathlineto{\pgfqpoint{1.493755in}{2.418293in}}%
\pgfpathlineto{\pgfqpoint{1.490934in}{2.414916in}}%
\pgfpathlineto{\pgfqpoint{1.490621in}{2.414541in}}%
\pgfpathlineto{\pgfqpoint{1.487486in}{2.414541in}}%
\pgfpathlineto{\pgfqpoint{1.484351in}{2.414541in}}%
\pgfpathlineto{\pgfqpoint{1.481216in}{2.414541in}}%
\pgfpathlineto{\pgfqpoint{1.478395in}{2.411165in}}%
\pgfpathlineto{\pgfqpoint{1.478082in}{2.410790in}}%
\pgfpathlineto{\pgfqpoint{1.474947in}{2.410790in}}%
\pgfpathlineto{\pgfqpoint{1.471812in}{2.410790in}}%
\pgfpathlineto{\pgfqpoint{1.468677in}{2.410790in}}%
\pgfpathlineto{\pgfqpoint{1.465856in}{2.407413in}}%
\pgfpathlineto{\pgfqpoint{1.465543in}{2.407038in}}%
\pgfpathlineto{\pgfqpoint{1.462408in}{2.407038in}}%
\pgfpathlineto{\pgfqpoint{1.459587in}{2.403662in}}%
\pgfpathlineto{\pgfqpoint{1.459587in}{2.399910in}}%
\pgfpathlineto{\pgfqpoint{1.459273in}{2.399535in}}%
\pgfpathlineto{\pgfqpoint{1.456452in}{2.396159in}}%
\pgfpathlineto{\pgfqpoint{1.456452in}{2.392407in}}%
\pgfpathlineto{\pgfqpoint{1.456452in}{2.388656in}}%
\pgfpathlineto{\pgfqpoint{1.456138in}{2.388280in}}%
\pgfpathlineto{\pgfqpoint{1.453317in}{2.384904in}}%
\pgfpathlineto{\pgfqpoint{1.453317in}{2.381153in}}%
\pgfpathlineto{\pgfqpoint{1.453317in}{2.377401in}}%
\pgfpathlineto{\pgfqpoint{1.453317in}{2.373649in}}%
\pgfpathlineto{\pgfqpoint{1.453004in}{2.373274in}}%
\pgfpathlineto{\pgfqpoint{1.450182in}{2.369898in}}%
\pgfpathlineto{\pgfqpoint{1.450182in}{2.366146in}}%
\pgfpathlineto{\pgfqpoint{1.450182in}{2.362395in}}%
\pgfpathlineto{\pgfqpoint{1.449869in}{2.362020in}}%
\pgfpathlineto{\pgfqpoint{1.447048in}{2.358643in}}%
\pgfpathlineto{\pgfqpoint{1.447048in}{2.354892in}}%
\pgfpathlineto{\pgfqpoint{1.447048in}{2.351140in}}%
\pgfpathlineto{\pgfqpoint{1.447048in}{2.347389in}}%
\pgfpathlineto{\pgfqpoint{1.446734in}{2.347014in}}%
\pgfpathlineto{\pgfqpoint{1.443913in}{2.343637in}}%
\pgfpathlineto{\pgfqpoint{1.443913in}{2.339886in}}%
\pgfpathlineto{\pgfqpoint{1.443913in}{2.336134in}}%
\pgfpathlineto{\pgfqpoint{1.443599in}{2.335759in}}%
\pgfpathlineto{\pgfqpoint{1.440778in}{2.332383in}}%
\pgfpathlineto{\pgfqpoint{1.440778in}{2.328631in}}%
\pgfpathlineto{\pgfqpoint{1.440778in}{2.324880in}}%
\pgfpathlineto{\pgfqpoint{1.440778in}{2.321128in}}%
\pgfpathlineto{\pgfqpoint{1.440465in}{2.320753in}}%
\pgfpathlineto{\pgfqpoint{1.437643in}{2.317376in}}%
\pgfpathlineto{\pgfqpoint{1.437643in}{2.313625in}}%
\pgfpathlineto{\pgfqpoint{1.437643in}{2.309873in}}%
\pgfpathlineto{\pgfqpoint{1.437330in}{2.309498in}}%
\pgfpathlineto{\pgfqpoint{1.434509in}{2.306122in}}%
\pgfpathlineto{\pgfqpoint{1.434509in}{2.302370in}}%
\pgfpathlineto{\pgfqpoint{1.434509in}{2.298619in}}%
\pgfpathlineto{\pgfqpoint{1.434509in}{2.294867in}}%
\pgfpathlineto{\pgfqpoint{1.434195in}{2.294492in}}%
\pgfpathlineto{\pgfqpoint{1.431374in}{2.291116in}}%
\pgfpathlineto{\pgfqpoint{1.431374in}{2.287364in}}%
\pgfpathlineto{\pgfqpoint{1.431374in}{2.283613in}}%
\pgfpathlineto{\pgfqpoint{1.431060in}{2.283238in}}%
\pgfpathlineto{\pgfqpoint{1.428239in}{2.279861in}}%
\pgfpathlineto{\pgfqpoint{1.428239in}{2.276110in}}%
\pgfpathlineto{\pgfqpoint{1.428239in}{2.272358in}}%
\pgfpathlineto{\pgfqpoint{1.428239in}{2.268607in}}%
\pgfpathlineto{\pgfqpoint{1.427926in}{2.268231in}}%
\pgfpathlineto{\pgfqpoint{1.425104in}{2.264855in}}%
\pgfpathlineto{\pgfqpoint{1.425104in}{2.261103in}}%
\pgfpathlineto{\pgfqpoint{1.425104in}{2.257352in}}%
\pgfpathlineto{\pgfqpoint{1.424791in}{2.256977in}}%
\pgfpathlineto{\pgfqpoint{1.421970in}{2.253600in}}%
\pgfpathlineto{\pgfqpoint{1.421970in}{2.249849in}}%
\pgfpathlineto{\pgfqpoint{1.421970in}{2.246097in}}%
\pgfpathlineto{\pgfqpoint{1.421656in}{2.245722in}}%
\pgfpathlineto{\pgfqpoint{1.418835in}{2.242346in}}%
\pgfpathlineto{\pgfqpoint{1.418835in}{2.238594in}}%
\pgfpathlineto{\pgfqpoint{1.418835in}{2.234843in}}%
\pgfpathlineto{\pgfqpoint{1.418835in}{2.231091in}}%
\pgfpathlineto{\pgfqpoint{1.418521in}{2.230716in}}%
\pgfpathlineto{\pgfqpoint{1.415700in}{2.227340in}}%
\pgfpathlineto{\pgfqpoint{1.415700in}{2.223588in}}%
\pgfpathlineto{\pgfqpoint{1.415700in}{2.219837in}}%
\pgfpathlineto{\pgfqpoint{1.415387in}{2.219461in}}%
\pgfpathlineto{\pgfqpoint{1.412565in}{2.216085in}}%
\pgfpathlineto{\pgfqpoint{1.412565in}{2.212334in}}%
\pgfpathlineto{\pgfqpoint{1.412565in}{2.208582in}}%
\pgfpathlineto{\pgfqpoint{1.412565in}{2.204830in}}%
\pgfpathlineto{\pgfqpoint{1.412252in}{2.204455in}}%
\pgfpathlineto{\pgfqpoint{1.409431in}{2.201079in}}%
\pgfpathlineto{\pgfqpoint{1.409431in}{2.197327in}}%
\pgfpathlineto{\pgfqpoint{1.409431in}{2.193576in}}%
\pgfpathlineto{\pgfqpoint{1.409117in}{2.193201in}}%
\pgfpathlineto{\pgfqpoint{1.406296in}{2.189824in}}%
\pgfpathlineto{\pgfqpoint{1.406296in}{2.186073in}}%
\pgfpathlineto{\pgfqpoint{1.406296in}{2.182321in}}%
\pgfpathlineto{\pgfqpoint{1.406296in}{2.178570in}}%
\pgfpathlineto{\pgfqpoint{1.405982in}{2.178195in}}%
\pgfpathlineto{\pgfqpoint{1.403161in}{2.174818in}}%
\pgfpathlineto{\pgfqpoint{1.403161in}{2.171067in}}%
\pgfpathlineto{\pgfqpoint{1.403161in}{2.167315in}}%
\pgfpathlineto{\pgfqpoint{1.402848in}{2.166940in}}%
\pgfpathlineto{\pgfqpoint{1.400026in}{2.163564in}}%
\pgfpathlineto{\pgfqpoint{1.400026in}{2.159812in}}%
\pgfpathlineto{\pgfqpoint{1.400026in}{2.156060in}}%
\pgfpathlineto{\pgfqpoint{1.400026in}{2.152309in}}%
\pgfpathlineto{\pgfqpoint{1.399713in}{2.151934in}}%
\pgfpathlineto{\pgfqpoint{1.396892in}{2.148557in}}%
\pgfpathlineto{\pgfqpoint{1.396892in}{2.144806in}}%
\pgfpathlineto{\pgfqpoint{1.396892in}{2.141054in}}%
\pgfpathlineto{\pgfqpoint{1.396578in}{2.140679in}}%
\pgfpathlineto{\pgfqpoint{1.393757in}{2.137303in}}%
\pgfpathlineto{\pgfqpoint{1.393757in}{2.133551in}}%
\pgfpathlineto{\pgfqpoint{1.393757in}{2.129800in}}%
\pgfpathlineto{\pgfqpoint{1.393757in}{2.126048in}}%
\pgfpathlineto{\pgfqpoint{1.393444in}{2.125673in}}%
\pgfpathlineto{\pgfqpoint{1.390622in}{2.122297in}}%
\pgfpathlineto{\pgfqpoint{1.390622in}{2.118545in}}%
\pgfpathlineto{\pgfqpoint{1.390622in}{2.114794in}}%
\pgfpathlineto{\pgfqpoint{1.390309in}{2.114418in}}%
\pgfpathlineto{\pgfqpoint{1.387487in}{2.111042in}}%
\pgfpathlineto{\pgfqpoint{1.387487in}{2.107291in}}%
\pgfpathlineto{\pgfqpoint{1.387487in}{2.103539in}}%
\pgfpathlineto{\pgfqpoint{1.387487in}{2.099787in}}%
\pgfpathlineto{\pgfqpoint{1.387174in}{2.099412in}}%
\pgfpathlineto{\pgfqpoint{1.384353in}{2.096036in}}%
\pgfpathlineto{\pgfqpoint{1.384353in}{2.092284in}}%
\pgfpathlineto{\pgfqpoint{1.384353in}{2.088533in}}%
\pgfpathlineto{\pgfqpoint{1.384039in}{2.088158in}}%
\pgfpathlineto{\pgfqpoint{1.381218in}{2.084781in}}%
\pgfpathlineto{\pgfqpoint{1.381218in}{2.081030in}}%
\pgfpathlineto{\pgfqpoint{1.381218in}{2.077278in}}%
\pgfpathlineto{\pgfqpoint{1.381218in}{2.073527in}}%
\pgfpathlineto{\pgfqpoint{1.380905in}{2.073152in}}%
\pgfpathlineto{\pgfqpoint{1.378083in}{2.069775in}}%
\pgfpathlineto{\pgfqpoint{1.378083in}{2.066024in}}%
\pgfpathlineto{\pgfqpoint{1.378083in}{2.062272in}}%
\pgfpathlineto{\pgfqpoint{1.377770in}{2.061897in}}%
\pgfpathlineto{\pgfqpoint{1.374949in}{2.058521in}}%
\pgfpathlineto{\pgfqpoint{1.374949in}{2.054769in}}%
\pgfpathlineto{\pgfqpoint{1.374949in}{2.051018in}}%
\pgfpathlineto{\pgfqpoint{1.374949in}{2.047266in}}%
\pgfpathlineto{\pgfqpoint{1.374635in}{2.046891in}}%
\pgfpathlineto{\pgfqpoint{1.371814in}{2.043514in}}%
\pgfpathlineto{\pgfqpoint{1.371814in}{2.039763in}}%
\pgfpathlineto{\pgfqpoint{1.371814in}{2.036011in}}%
\pgfpathlineto{\pgfqpoint{1.371500in}{2.035636in}}%
\pgfpathlineto{\pgfqpoint{1.368679in}{2.032260in}}%
\pgfpathlineto{\pgfqpoint{1.368679in}{2.028508in}}%
\pgfpathlineto{\pgfqpoint{1.368679in}{2.024757in}}%
\pgfpathlineto{\pgfqpoint{1.368366in}{2.024382in}}%
\pgfpathlineto{\pgfqpoint{1.365544in}{2.021005in}}%
\pgfpathlineto{\pgfqpoint{1.365544in}{2.017254in}}%
\pgfpathlineto{\pgfqpoint{1.365544in}{2.013502in}}%
\pgfpathlineto{\pgfqpoint{1.365544in}{2.009751in}}%
\pgfpathlineto{\pgfqpoint{1.365231in}{2.009375in}}%
\pgfpathlineto{\pgfqpoint{1.362410in}{2.005999in}}%
\pgfpathlineto{\pgfqpoint{1.362410in}{2.002248in}}%
\pgfpathlineto{\pgfqpoint{1.362410in}{1.998496in}}%
\pgfpathlineto{\pgfqpoint{1.362096in}{1.998121in}}%
\pgfpathlineto{\pgfqpoint{1.359275in}{1.994745in}}%
\pgfpathlineto{\pgfqpoint{1.359275in}{1.990993in}}%
\pgfpathlineto{\pgfqpoint{1.359275in}{1.987241in}}%
\pgfpathlineto{\pgfqpoint{1.359275in}{1.983490in}}%
\pgfpathlineto{\pgfqpoint{1.358961in}{1.983115in}}%
\pgfpathlineto{\pgfqpoint{1.356140in}{1.979738in}}%
\pgfpathlineto{\pgfqpoint{1.356140in}{1.975987in}}%
\pgfpathlineto{\pgfqpoint{1.356140in}{1.972235in}}%
\pgfpathlineto{\pgfqpoint{1.355827in}{1.971860in}}%
\pgfpathlineto{\pgfqpoint{1.353005in}{1.968484in}}%
\pgfpathlineto{\pgfqpoint{1.353005in}{1.964732in}}%
\pgfpathlineto{\pgfqpoint{1.353005in}{1.960981in}}%
\pgfpathlineto{\pgfqpoint{1.353005in}{1.957229in}}%
\pgfpathlineto{\pgfqpoint{1.352692in}{1.956854in}}%
\pgfpathlineto{\pgfqpoint{1.349871in}{1.953478in}}%
\pgfpathlineto{\pgfqpoint{1.349871in}{1.949726in}}%
\pgfpathlineto{\pgfqpoint{1.349871in}{1.945975in}}%
\pgfpathlineto{\pgfqpoint{1.349557in}{1.945599in}}%
\pgfpathlineto{\pgfqpoint{1.346736in}{1.942223in}}%
\pgfpathlineto{\pgfqpoint{1.346736in}{1.938471in}}%
\pgfpathlineto{\pgfqpoint{1.346736in}{1.934720in}}%
\pgfpathlineto{\pgfqpoint{1.346736in}{1.930968in}}%
\pgfpathlineto{\pgfqpoint{1.346422in}{1.930593in}}%
\pgfpathlineto{\pgfqpoint{1.343601in}{1.927217in}}%
\pgfpathlineto{\pgfqpoint{1.343601in}{1.923465in}}%
\pgfpathlineto{\pgfqpoint{1.343601in}{1.919714in}}%
\pgfpathlineto{\pgfqpoint{1.343288in}{1.919339in}}%
\pgfpathlineto{\pgfqpoint{1.340466in}{1.915962in}}%
\pgfpathlineto{\pgfqpoint{1.340466in}{1.912211in}}%
\pgfpathlineto{\pgfqpoint{1.340466in}{1.908459in}}%
\pgfpathlineto{\pgfqpoint{1.340466in}{1.904708in}}%
\pgfpathlineto{\pgfqpoint{1.340153in}{1.904333in}}%
\pgfpathlineto{\pgfqpoint{1.337332in}{1.900956in}}%
\pgfpathlineto{\pgfqpoint{1.337332in}{1.897205in}}%
\pgfpathlineto{\pgfqpoint{1.337332in}{1.893453in}}%
\pgfpathlineto{\pgfqpoint{1.337018in}{1.893078in}}%
\pgfpathlineto{\pgfqpoint{1.334197in}{1.889702in}}%
\pgfpathlineto{\pgfqpoint{1.334197in}{1.885950in}}%
\pgfpathlineto{\pgfqpoint{1.334197in}{1.882198in}}%
\pgfpathlineto{\pgfqpoint{1.334197in}{1.878447in}}%
\pgfpathlineto{\pgfqpoint{1.333883in}{1.878072in}}%
\pgfpathlineto{\pgfqpoint{1.331062in}{1.874695in}}%
\pgfpathlineto{\pgfqpoint{1.331062in}{1.870944in}}%
\pgfpathlineto{\pgfqpoint{1.331062in}{1.867192in}}%
\pgfpathlineto{\pgfqpoint{1.330749in}{1.866817in}}%
\pgfpathlineto{\pgfqpoint{1.327927in}{1.863441in}}%
\pgfpathlineto{\pgfqpoint{1.327927in}{1.859689in}}%
\pgfpathlineto{\pgfqpoint{1.327927in}{1.855938in}}%
\pgfpathlineto{\pgfqpoint{1.327927in}{1.852186in}}%
\pgfpathlineto{\pgfqpoint{1.327614in}{1.851811in}}%
\pgfpathlineto{\pgfqpoint{1.324793in}{1.848435in}}%
\pgfpathlineto{\pgfqpoint{1.324793in}{1.844683in}}%
\pgfpathlineto{\pgfqpoint{1.324793in}{1.840932in}}%
\pgfpathlineto{\pgfqpoint{1.324479in}{1.840556in}}%
\pgfpathlineto{\pgfqpoint{1.321658in}{1.837180in}}%
\pgfpathlineto{\pgfqpoint{1.321658in}{1.833429in}}%
\pgfpathlineto{\pgfqpoint{1.321658in}{1.829677in}}%
\pgfpathlineto{\pgfqpoint{1.321658in}{1.825925in}}%
\pgfpathlineto{\pgfqpoint{1.321344in}{1.825550in}}%
\pgfpathlineto{\pgfqpoint{1.318523in}{1.822174in}}%
\pgfpathlineto{\pgfqpoint{1.318523in}{1.818422in}}%
\pgfpathlineto{\pgfqpoint{1.318523in}{1.814671in}}%
\pgfpathlineto{\pgfqpoint{1.318210in}{1.814296in}}%
\pgfpathlineto{\pgfqpoint{1.315388in}{1.810919in}}%
\pgfpathlineto{\pgfqpoint{1.315388in}{1.807168in}}%
\pgfpathlineto{\pgfqpoint{1.315388in}{1.803416in}}%
\pgfpathlineto{\pgfqpoint{1.315075in}{1.803041in}}%
\pgfpathlineto{\pgfqpoint{1.312254in}{1.799665in}}%
\pgfpathlineto{\pgfqpoint{1.312254in}{1.795913in}}%
\pgfpathlineto{\pgfqpoint{1.312254in}{1.792162in}}%
\pgfpathlineto{\pgfqpoint{1.312254in}{1.788410in}}%
\pgfpathlineto{\pgfqpoint{1.311940in}{1.788035in}}%
\pgfpathlineto{\pgfqpoint{1.309119in}{1.784659in}}%
\pgfpathlineto{\pgfqpoint{1.309119in}{1.780907in}}%
\pgfpathlineto{\pgfqpoint{1.309119in}{1.777155in}}%
\pgfpathlineto{\pgfqpoint{1.308805in}{1.776780in}}%
\pgfpathlineto{\pgfqpoint{1.305984in}{1.773404in}}%
\pgfpathlineto{\pgfqpoint{1.305984in}{1.769652in}}%
\pgfpathlineto{\pgfqpoint{1.305984in}{1.765901in}}%
\pgfpathlineto{\pgfqpoint{1.305984in}{1.762149in}}%
\pgfpathlineto{\pgfqpoint{1.305671in}{1.761774in}}%
\pgfpathlineto{\pgfqpoint{1.302849in}{1.758398in}}%
\pgfpathlineto{\pgfqpoint{1.302849in}{1.754646in}}%
\pgfpathlineto{\pgfqpoint{1.302849in}{1.750895in}}%
\pgfpathlineto{\pgfqpoint{1.302536in}{1.750520in}}%
\pgfpathlineto{\pgfqpoint{1.299715in}{1.747143in}}%
\pgfpathlineto{\pgfqpoint{1.299715in}{1.743392in}}%
\pgfpathlineto{\pgfqpoint{1.299715in}{1.739640in}}%
\pgfpathlineto{\pgfqpoint{1.299715in}{1.735889in}}%
\pgfpathlineto{\pgfqpoint{1.299401in}{1.735513in}}%
\pgfpathlineto{\pgfqpoint{1.296580in}{1.732137in}}%
\pgfpathlineto{\pgfqpoint{1.296580in}{1.728386in}}%
\pgfpathlineto{\pgfqpoint{1.296580in}{1.724634in}}%
\pgfpathlineto{\pgfqpoint{1.296266in}{1.724259in}}%
\pgfpathlineto{\pgfqpoint{1.293445in}{1.720882in}}%
\pgfpathlineto{\pgfqpoint{1.293445in}{1.717131in}}%
\pgfpathlineto{\pgfqpoint{1.293445in}{1.713379in}}%
\pgfpathlineto{\pgfqpoint{1.293445in}{1.709628in}}%
\pgfpathlineto{\pgfqpoint{1.293132in}{1.709253in}}%
\pgfpathlineto{\pgfqpoint{1.290310in}{1.705876in}}%
\pgfpathlineto{\pgfqpoint{1.290310in}{1.702125in}}%
\pgfpathlineto{\pgfqpoint{1.293132in}{1.698748in}}%
\pgfpathlineto{\pgfqpoint{1.293445in}{1.698373in}}%
\pgfpathlineto{\pgfqpoint{1.296266in}{1.694997in}}%
\pgfpathlineto{\pgfqpoint{1.296580in}{1.694622in}}%
\pgfpathlineto{\pgfqpoint{1.299401in}{1.691245in}}%
\pgfpathlineto{\pgfqpoint{1.302536in}{1.691245in}}%
\pgfpathlineto{\pgfqpoint{1.302849in}{1.690870in}}%
\pgfpathlineto{\pgfqpoint{1.305671in}{1.687494in}}%
\pgfpathlineto{\pgfqpoint{1.305984in}{1.687119in}}%
\pgfpathlineto{\pgfqpoint{1.308805in}{1.683742in}}%
\pgfpathlineto{\pgfqpoint{1.309119in}{1.683367in}}%
\pgfpathlineto{\pgfqpoint{1.311940in}{1.679991in}}%
\pgfpathlineto{\pgfqpoint{1.315075in}{1.679991in}}%
\pgfpathlineto{\pgfqpoint{1.315388in}{1.679616in}}%
\pgfpathlineto{\pgfqpoint{1.318210in}{1.676239in}}%
\pgfpathlineto{\pgfqpoint{1.318523in}{1.675864in}}%
\pgfpathlineto{\pgfqpoint{1.321344in}{1.672488in}}%
\pgfpathlineto{\pgfqpoint{1.321658in}{1.672113in}}%
\pgfpathlineto{\pgfqpoint{1.324479in}{1.668736in}}%
\pgfpathlineto{\pgfqpoint{1.327614in}{1.668736in}}%
\pgfpathlineto{\pgfqpoint{1.327927in}{1.668361in}}%
\pgfpathlineto{\pgfqpoint{1.330749in}{1.664985in}}%
\pgfpathlineto{\pgfqpoint{1.331062in}{1.664609in}}%
\pgfpathlineto{\pgfqpoint{1.333883in}{1.661233in}}%
\pgfpathlineto{\pgfqpoint{1.334197in}{1.660858in}}%
\pgfpathlineto{\pgfqpoint{1.337018in}{1.657482in}}%
\pgfpathlineto{\pgfqpoint{1.337332in}{1.657106in}}%
\pgfpathlineto{\pgfqpoint{1.340153in}{1.653730in}}%
\pgfpathlineto{\pgfqpoint{1.343288in}{1.653730in}}%
\pgfpathlineto{\pgfqpoint{1.343601in}{1.653355in}}%
\pgfpathlineto{\pgfqpoint{1.346422in}{1.649978in}}%
\pgfpathlineto{\pgfqpoint{1.346736in}{1.649603in}}%
\pgfpathlineto{\pgfqpoint{1.349557in}{1.646227in}}%
\pgfpathlineto{\pgfqpoint{1.349871in}{1.645852in}}%
\pgfpathlineto{\pgfqpoint{1.352692in}{1.642475in}}%
\pgfpathlineto{\pgfqpoint{1.355827in}{1.642475in}}%
\pgfpathlineto{\pgfqpoint{1.356140in}{1.642100in}}%
\pgfpathlineto{\pgfqpoint{1.358961in}{1.638724in}}%
\pgfpathlineto{\pgfqpoint{1.359275in}{1.638349in}}%
\pgfpathlineto{\pgfqpoint{1.362096in}{1.634972in}}%
\pgfpathlineto{\pgfqpoint{1.362410in}{1.634597in}}%
\pgfpathlineto{\pgfqpoint{1.365231in}{1.631221in}}%
\pgfpathlineto{\pgfqpoint{1.368366in}{1.631221in}}%
\pgfpathlineto{\pgfqpoint{1.368679in}{1.630846in}}%
\pgfpathlineto{\pgfqpoint{1.371500in}{1.627469in}}%
\pgfpathlineto{\pgfqpoint{1.371814in}{1.627094in}}%
\pgfpathlineto{\pgfqpoint{1.374635in}{1.623718in}}%
\pgfpathlineto{\pgfqpoint{1.374949in}{1.623343in}}%
\pgfpathlineto{\pgfqpoint{1.377770in}{1.619966in}}%
\pgfpathlineto{\pgfqpoint{1.380905in}{1.619966in}}%
\pgfpathlineto{\pgfqpoint{1.381218in}{1.619591in}}%
\pgfpathlineto{\pgfqpoint{1.384039in}{1.616215in}}%
\pgfpathlineto{\pgfqpoint{1.384353in}{1.615840in}}%
\pgfpathlineto{\pgfqpoint{1.387174in}{1.612463in}}%
\pgfpathlineto{\pgfqpoint{1.387487in}{1.612088in}}%
\pgfpathlineto{\pgfqpoint{1.390309in}{1.608712in}}%
\pgfpathlineto{\pgfqpoint{1.393444in}{1.608712in}}%
\pgfpathlineto{\pgfqpoint{1.393757in}{1.608336in}}%
\pgfpathlineto{\pgfqpoint{1.396578in}{1.604960in}}%
\pgfpathlineto{\pgfqpoint{1.396892in}{1.604585in}}%
\pgfpathlineto{\pgfqpoint{1.399713in}{1.601209in}}%
\pgfpathlineto{\pgfqpoint{1.400026in}{1.600833in}}%
\pgfpathlineto{\pgfqpoint{1.402848in}{1.597457in}}%
\pgfpathlineto{\pgfqpoint{1.405982in}{1.597457in}}%
\pgfpathlineto{\pgfqpoint{1.406296in}{1.597082in}}%
\pgfpathlineto{\pgfqpoint{1.409117in}{1.593705in}}%
\pgfpathlineto{\pgfqpoint{1.409431in}{1.593330in}}%
\pgfpathlineto{\pgfqpoint{1.412252in}{1.589954in}}%
\pgfpathlineto{\pgfqpoint{1.412565in}{1.589579in}}%
\pgfpathlineto{\pgfqpoint{1.415387in}{1.586202in}}%
\pgfpathlineto{\pgfqpoint{1.415700in}{1.585827in}}%
\pgfpathlineto{\pgfqpoint{1.418521in}{1.582451in}}%
\pgfpathlineto{\pgfqpoint{1.421656in}{1.582451in}}%
\pgfpathlineto{\pgfqpoint{1.421970in}{1.582076in}}%
\pgfpathlineto{\pgfqpoint{1.424791in}{1.578699in}}%
\pgfpathlineto{\pgfqpoint{1.425104in}{1.578324in}}%
\pgfpathlineto{\pgfqpoint{1.427926in}{1.574948in}}%
\pgfpathlineto{\pgfqpoint{1.428239in}{1.574573in}}%
\pgfpathlineto{\pgfqpoint{1.431060in}{1.571196in}}%
\pgfpathlineto{\pgfqpoint{1.434195in}{1.571196in}}%
\pgfpathlineto{\pgfqpoint{1.434509in}{1.570821in}}%
\pgfpathlineto{\pgfqpoint{1.437330in}{1.567445in}}%
\pgfpathlineto{\pgfqpoint{1.437643in}{1.567070in}}%
\pgfpathlineto{\pgfqpoint{1.440465in}{1.563693in}}%
\pgfpathlineto{\pgfqpoint{1.440778in}{1.563318in}}%
\pgfpathlineto{\pgfqpoint{1.443599in}{1.559942in}}%
\pgfpathlineto{\pgfqpoint{1.446734in}{1.559942in}}%
\pgfpathlineto{\pgfqpoint{1.447048in}{1.559566in}}%
\pgfpathlineto{\pgfqpoint{1.449869in}{1.556190in}}%
\pgfpathlineto{\pgfqpoint{1.450182in}{1.555815in}}%
\pgfpathlineto{\pgfqpoint{1.453004in}{1.552439in}}%
\pgfpathlineto{\pgfqpoint{1.453317in}{1.552063in}}%
\pgfpathlineto{\pgfqpoint{1.456138in}{1.548687in}}%
\pgfpathlineto{\pgfqpoint{1.459273in}{1.548687in}}%
\pgfpathlineto{\pgfqpoint{1.459587in}{1.548312in}}%
\pgfpathlineto{\pgfqpoint{1.462408in}{1.544936in}}%
\pgfpathlineto{\pgfqpoint{1.462721in}{1.544560in}}%
\pgfpathlineto{\pgfqpoint{1.465543in}{1.541184in}}%
\pgfpathlineto{\pgfqpoint{1.465856in}{1.540809in}}%
\pgfpathlineto{\pgfqpoint{1.468677in}{1.537432in}}%
\pgfpathlineto{\pgfqpoint{1.471812in}{1.537432in}}%
\pgfpathlineto{\pgfqpoint{1.472126in}{1.537057in}}%
\pgfpathlineto{\pgfqpoint{1.474947in}{1.533681in}}%
\pgfpathlineto{\pgfqpoint{1.475260in}{1.533306in}}%
\pgfpathlineto{\pgfqpoint{1.478082in}{1.529929in}}%
\pgfpathlineto{\pgfqpoint{1.478395in}{1.529554in}}%
\pgfpathlineto{\pgfqpoint{1.481216in}{1.526178in}}%
\pgfpathlineto{\pgfqpoint{1.484351in}{1.526178in}}%
\pgfpathlineto{\pgfqpoint{1.484665in}{1.525803in}}%
\pgfpathlineto{\pgfqpoint{1.487486in}{1.522426in}}%
\pgfpathlineto{\pgfqpoint{1.487799in}{1.522051in}}%
\pgfpathlineto{\pgfqpoint{1.490621in}{1.518675in}}%
\pgfpathlineto{\pgfqpoint{1.490934in}{1.518300in}}%
\pgfpathlineto{\pgfqpoint{1.493755in}{1.514923in}}%
\pgfpathlineto{\pgfqpoint{1.494069in}{1.514548in}}%
\pgfpathlineto{\pgfqpoint{1.496890in}{1.511172in}}%
\pgfpathlineto{\pgfqpoint{1.500025in}{1.511172in}}%
\pgfpathlineto{\pgfqpoint{1.500338in}{1.510797in}}%
\pgfpathlineto{\pgfqpoint{1.503159in}{1.507420in}}%
\pgfpathlineto{\pgfqpoint{1.503473in}{1.507045in}}%
\pgfpathlineto{\pgfqpoint{1.506294in}{1.503669in}}%
\pgfpathlineto{\pgfqpoint{1.506608in}{1.503293in}}%
\pgfpathlineto{\pgfqpoint{1.509429in}{1.499917in}}%
\pgfpathlineto{\pgfqpoint{1.512564in}{1.499917in}}%
\pgfpathlineto{\pgfqpoint{1.512877in}{1.499542in}}%
\pgfpathlineto{\pgfqpoint{1.515698in}{1.496166in}}%
\pgfpathlineto{\pgfqpoint{1.516012in}{1.495790in}}%
\pgfpathlineto{\pgfqpoint{1.518833in}{1.492414in}}%
\pgfpathlineto{\pgfqpoint{1.519147in}{1.492039in}}%
\pgfpathlineto{\pgfqpoint{1.521968in}{1.488662in}}%
\pgfpathlineto{\pgfqpoint{1.525103in}{1.488662in}}%
\pgfpathlineto{\pgfqpoint{1.525416in}{1.488287in}}%
\pgfpathlineto{\pgfqpoint{1.528237in}{1.484911in}}%
\pgfpathlineto{\pgfqpoint{1.528551in}{1.484536in}}%
\pgfpathlineto{\pgfqpoint{1.531372in}{1.481159in}}%
\pgfpathlineto{\pgfqpoint{1.531686in}{1.480784in}}%
\pgfpathlineto{\pgfqpoint{1.534507in}{1.477408in}}%
\pgfpathlineto{\pgfqpoint{1.537642in}{1.477408in}}%
\pgfpathlineto{\pgfqpoint{1.537955in}{1.477033in}}%
\pgfpathlineto{\pgfqpoint{1.540776in}{1.473656in}}%
\pgfpathlineto{\pgfqpoint{1.541090in}{1.473281in}}%
\pgfpathlineto{\pgfqpoint{1.543911in}{1.469905in}}%
\pgfpathlineto{\pgfqpoint{1.544225in}{1.469530in}}%
\pgfpathlineto{\pgfqpoint{1.547046in}{1.466153in}}%
\pgfpathlineto{\pgfqpoint{1.550181in}{1.466153in}}%
\pgfpathlineto{\pgfqpoint{1.550494in}{1.465778in}}%
\pgfpathlineto{\pgfqpoint{1.553315in}{1.462402in}}%
\pgfpathlineto{\pgfqpoint{1.553629in}{1.462027in}}%
\pgfpathlineto{\pgfqpoint{1.556450in}{1.458650in}}%
\pgfpathlineto{\pgfqpoint{1.556764in}{1.458275in}}%
\pgfpathlineto{\pgfqpoint{1.559585in}{1.454899in}}%
\pgfpathlineto{\pgfqpoint{1.559898in}{1.454524in}}%
\pgfpathlineto{\pgfqpoint{1.562720in}{1.451147in}}%
\pgfpathlineto{\pgfqpoint{1.565854in}{1.451147in}}%
\pgfpathlineto{\pgfqpoint{1.566168in}{1.450772in}}%
\pgfpathlineto{\pgfqpoint{1.568989in}{1.447396in}}%
\pgfpathlineto{\pgfqpoint{1.569303in}{1.447020in}}%
\pgfpathlineto{\pgfqpoint{1.572124in}{1.443644in}}%
\pgfpathlineto{\pgfqpoint{1.572437in}{1.443269in}}%
\pgfpathlineto{\pgfqpoint{1.575259in}{1.439893in}}%
\pgfpathlineto{\pgfqpoint{1.578393in}{1.439893in}}%
\pgfpathlineto{\pgfqpoint{1.578707in}{1.439517in}}%
\pgfpathlineto{\pgfqpoint{1.581528in}{1.436141in}}%
\pgfpathlineto{\pgfqpoint{1.581842in}{1.435766in}}%
\pgfpathlineto{\pgfqpoint{1.584663in}{1.432389in}}%
\pgfpathlineto{\pgfqpoint{1.584976in}{1.432014in}}%
\pgfpathlineto{\pgfqpoint{1.587798in}{1.428638in}}%
\pgfpathlineto{\pgfqpoint{1.590932in}{1.428638in}}%
\pgfpathlineto{\pgfqpoint{1.591246in}{1.428263in}}%
\pgfpathlineto{\pgfqpoint{1.594067in}{1.424886in}}%
\pgfpathlineto{\pgfqpoint{1.594380in}{1.424511in}}%
\pgfpathlineto{\pgfqpoint{1.597202in}{1.421135in}}%
\pgfpathlineto{\pgfqpoint{1.597515in}{1.420760in}}%
\pgfpathlineto{\pgfqpoint{1.600337in}{1.417383in}}%
\pgfpathlineto{\pgfqpoint{1.603471in}{1.417383in}}%
\pgfpathlineto{\pgfqpoint{1.603785in}{1.417008in}}%
\pgfpathlineto{\pgfqpoint{1.606606in}{1.413632in}}%
\pgfpathlineto{\pgfqpoint{1.606919in}{1.413257in}}%
\pgfpathlineto{\pgfqpoint{1.609741in}{1.409880in}}%
\pgfpathlineto{\pgfqpoint{1.610054in}{1.409505in}}%
\pgfpathlineto{\pgfqpoint{1.612875in}{1.406129in}}%
\pgfpathlineto{\pgfqpoint{1.616010in}{1.406129in}}%
\pgfpathlineto{\pgfqpoint{1.616324in}{1.405754in}}%
\pgfpathlineto{\pgfqpoint{1.619145in}{1.402377in}}%
\pgfpathlineto{\pgfqpoint{1.619458in}{1.402002in}}%
\pgfpathlineto{\pgfqpoint{1.622280in}{1.398626in}}%
\pgfpathlineto{\pgfqpoint{1.622593in}{1.398251in}}%
\pgfpathlineto{\pgfqpoint{1.625414in}{1.394874in}}%
\pgfpathlineto{\pgfqpoint{1.628549in}{1.394874in}}%
\pgfpathlineto{\pgfqpoint{1.628863in}{1.394499in}}%
\pgfpathlineto{\pgfqpoint{1.631684in}{1.391123in}}%
\pgfpathlineto{\pgfqpoint{1.631997in}{1.390747in}}%
\pgfpathlineto{\pgfqpoint{1.634819in}{1.387371in}}%
\pgfpathlineto{\pgfqpoint{1.635132in}{1.386996in}}%
\pgfpathlineto{\pgfqpoint{1.637953in}{1.383620in}}%
\pgfpathlineto{\pgfqpoint{1.638267in}{1.383244in}}%
\pgfpathlineto{\pgfqpoint{1.641088in}{1.379868in}}%
\pgfpathlineto{\pgfqpoint{1.644223in}{1.379868in}}%
\pgfpathlineto{\pgfqpoint{1.644536in}{1.379493in}}%
\pgfpathlineto{\pgfqpoint{1.647358in}{1.376116in}}%
\pgfpathlineto{\pgfqpoint{1.647671in}{1.375741in}}%
\pgfpathlineto{\pgfqpoint{1.650492in}{1.372365in}}%
\pgfpathlineto{\pgfqpoint{1.650806in}{1.371990in}}%
\pgfpathlineto{\pgfqpoint{1.653627in}{1.368613in}}%
\pgfpathlineto{\pgfqpoint{1.656762in}{1.368613in}}%
\pgfpathlineto{\pgfqpoint{1.657075in}{1.368238in}}%
\pgfpathlineto{\pgfqpoint{1.659897in}{1.364862in}}%
\pgfpathlineto{\pgfqpoint{1.660210in}{1.364487in}}%
\pgfpathlineto{\pgfqpoint{1.663031in}{1.361110in}}%
\pgfpathlineto{\pgfqpoint{1.663345in}{1.360735in}}%
\pgfpathlineto{\pgfqpoint{1.666166in}{1.357359in}}%
\pgfpathlineto{\pgfqpoint{1.669301in}{1.357359in}}%
\pgfpathlineto{\pgfqpoint{1.669614in}{1.356984in}}%
\pgfpathlineto{\pgfqpoint{1.672436in}{1.353607in}}%
\pgfpathlineto{\pgfqpoint{1.672749in}{1.353232in}}%
\pgfpathlineto{\pgfqpoint{1.675570in}{1.349856in}}%
\pgfpathlineto{\pgfqpoint{1.675884in}{1.349481in}}%
\pgfpathlineto{\pgfqpoint{1.678705in}{1.346104in}}%
\pgfpathlineto{\pgfqpoint{1.681840in}{1.346104in}}%
\pgfpathlineto{\pgfqpoint{1.682153in}{1.345729in}}%
\pgfpathlineto{\pgfqpoint{1.684975in}{1.342353in}}%
\pgfpathlineto{\pgfqpoint{1.685288in}{1.341977in}}%
\pgfpathlineto{\pgfqpoint{1.688109in}{1.338601in}}%
\pgfpathlineto{\pgfqpoint{1.688423in}{1.338226in}}%
\pgfpathlineto{\pgfqpoint{1.691244in}{1.334850in}}%
\pgfpathlineto{\pgfqpoint{1.694379in}{1.334850in}}%
\pgfpathlineto{\pgfqpoint{1.694692in}{1.334474in}}%
\pgfpathlineto{\pgfqpoint{1.697514in}{1.331098in}}%
\pgfpathlineto{\pgfqpoint{1.697827in}{1.330723in}}%
\pgfpathlineto{\pgfqpoint{1.700648in}{1.327347in}}%
\pgfpathlineto{\pgfqpoint{1.700962in}{1.326971in}}%
\pgfpathlineto{\pgfqpoint{1.703783in}{1.323595in}}%
\pgfpathlineto{\pgfqpoint{1.706918in}{1.323595in}}%
\pgfpathlineto{\pgfqpoint{1.707231in}{1.323220in}}%
\pgfpathlineto{\pgfqpoint{1.710052in}{1.319843in}}%
\pgfpathlineto{\pgfqpoint{1.710366in}{1.319468in}}%
\pgfpathlineto{\pgfqpoint{1.713187in}{1.316092in}}%
\pgfpathlineto{\pgfqpoint{1.713501in}{1.315717in}}%
\pgfpathlineto{\pgfqpoint{1.716322in}{1.312340in}}%
\pgfpathlineto{\pgfqpoint{1.716635in}{1.311965in}}%
\pgfpathlineto{\pgfqpoint{1.719457in}{1.308589in}}%
\pgfpathlineto{\pgfqpoint{1.722591in}{1.308589in}}%
\pgfpathlineto{\pgfqpoint{1.722905in}{1.308214in}}%
\pgfpathlineto{\pgfqpoint{1.725726in}{1.304837in}}%
\pgfpathlineto{\pgfqpoint{1.726040in}{1.304462in}}%
\pgfpathlineto{\pgfqpoint{1.728861in}{1.301086in}}%
\pgfpathlineto{\pgfqpoint{1.729174in}{1.300711in}}%
\pgfpathlineto{\pgfqpoint{1.731996in}{1.297334in}}%
\pgfpathlineto{\pgfqpoint{1.735130in}{1.297334in}}%
\pgfpathlineto{\pgfqpoint{1.735444in}{1.296959in}}%
\pgfpathlineto{\pgfqpoint{1.738265in}{1.293583in}}%
\pgfpathlineto{\pgfqpoint{1.738579in}{1.293208in}}%
\pgfpathlineto{\pgfqpoint{1.741400in}{1.289831in}}%
\pgfpathlineto{\pgfqpoint{1.741713in}{1.289456in}}%
\pgfpathlineto{\pgfqpoint{1.744535in}{1.286080in}}%
\pgfpathlineto{\pgfqpoint{1.747669in}{1.286080in}}%
\pgfpathlineto{\pgfqpoint{1.747983in}{1.285704in}}%
\pgfpathlineto{\pgfqpoint{1.750804in}{1.282328in}}%
\pgfpathlineto{\pgfqpoint{1.751118in}{1.281953in}}%
\pgfpathlineto{\pgfqpoint{1.753939in}{1.278577in}}%
\pgfpathlineto{\pgfqpoint{1.754252in}{1.278201in}}%
\pgfpathlineto{\pgfqpoint{1.757074in}{1.274825in}}%
\pgfpathlineto{\pgfqpoint{1.760208in}{1.274825in}}%
\pgfpathlineto{\pgfqpoint{1.760522in}{1.274450in}}%
\pgfpathlineto{\pgfqpoint{1.763343in}{1.271073in}}%
\pgfpathlineto{\pgfqpoint{1.763657in}{1.270698in}}%
\pgfpathlineto{\pgfqpoint{1.766478in}{1.267322in}}%
\pgfpathlineto{\pgfqpoint{1.766791in}{1.266947in}}%
\pgfpathlineto{\pgfqpoint{1.769613in}{1.263570in}}%
\pgfpathlineto{\pgfqpoint{1.772747in}{1.263570in}}%
\pgfpathlineto{\pgfqpoint{1.773061in}{1.263195in}}%
\pgfpathlineto{\pgfqpoint{1.775882in}{1.259819in}}%
\pgfpathlineto{\pgfqpoint{1.776196in}{1.259444in}}%
\pgfpathlineto{\pgfqpoint{1.779017in}{1.256067in}}%
\pgfpathlineto{\pgfqpoint{1.779330in}{1.255692in}}%
\pgfpathlineto{\pgfqpoint{1.782152in}{1.252316in}}%
\pgfpathlineto{\pgfqpoint{1.785286in}{1.252316in}}%
\pgfpathlineto{\pgfqpoint{1.785600in}{1.251941in}}%
\pgfpathlineto{\pgfqpoint{1.788421in}{1.248564in}}%
\pgfpathlineto{\pgfqpoint{1.788735in}{1.248189in}}%
\pgfpathlineto{\pgfqpoint{1.791556in}{1.244813in}}%
\pgfpathlineto{\pgfqpoint{1.791869in}{1.244438in}}%
\pgfpathlineto{\pgfqpoint{1.794691in}{1.241061in}}%
\pgfpathlineto{\pgfqpoint{1.795004in}{1.240686in}}%
\pgfpathlineto{\pgfqpoint{1.797825in}{1.237310in}}%
\pgfpathlineto{\pgfqpoint{1.800960in}{1.237310in}}%
\pgfpathlineto{\pgfqpoint{1.801273in}{1.236935in}}%
\pgfpathlineto{\pgfqpoint{1.804095in}{1.233558in}}%
\pgfpathlineto{\pgfqpoint{1.804408in}{1.233183in}}%
\pgfpathlineto{\pgfqpoint{1.807229in}{1.229807in}}%
\pgfpathlineto{\pgfqpoint{1.807543in}{1.229431in}}%
\pgfpathlineto{\pgfqpoint{1.810364in}{1.226055in}}%
\pgfpathlineto{\pgfqpoint{1.813499in}{1.226055in}}%
\pgfpathlineto{\pgfqpoint{1.813812in}{1.225680in}}%
\pgfpathlineto{\pgfqpoint{1.816634in}{1.222304in}}%
\pgfpathlineto{\pgfqpoint{1.816947in}{1.221928in}}%
\pgfpathlineto{\pgfqpoint{1.819768in}{1.218552in}}%
\pgfpathlineto{\pgfqpoint{1.820082in}{1.218177in}}%
\pgfpathlineto{\pgfqpoint{1.822903in}{1.214800in}}%
\pgfpathlineto{\pgfqpoint{1.826038in}{1.214800in}}%
\pgfpathlineto{\pgfqpoint{1.826351in}{1.214425in}}%
\pgfpathlineto{\pgfqpoint{1.829173in}{1.211049in}}%
\pgfpathlineto{\pgfqpoint{1.829486in}{1.210674in}}%
\pgfpathlineto{\pgfqpoint{1.832307in}{1.207297in}}%
\pgfpathlineto{\pgfqpoint{1.832621in}{1.206922in}}%
\pgfpathlineto{\pgfqpoint{1.835442in}{1.203546in}}%
\pgfpathlineto{\pgfqpoint{1.838577in}{1.203546in}}%
\pgfpathlineto{\pgfqpoint{1.838890in}{1.203171in}}%
\pgfpathlineto{\pgfqpoint{1.841712in}{1.199794in}}%
\pgfpathlineto{\pgfqpoint{1.842025in}{1.199419in}}%
\pgfpathlineto{\pgfqpoint{1.844846in}{1.196043in}}%
\pgfpathlineto{\pgfqpoint{1.845160in}{1.195668in}}%
\pgfpathlineto{\pgfqpoint{1.847981in}{1.192291in}}%
\pgfpathlineto{\pgfqpoint{1.851116in}{1.192291in}}%
\pgfpathlineto{\pgfqpoint{1.851429in}{1.191916in}}%
\pgfpathlineto{\pgfqpoint{1.854251in}{1.188540in}}%
\pgfpathlineto{\pgfqpoint{1.854564in}{1.188165in}}%
\pgfpathlineto{\pgfqpoint{1.857385in}{1.184788in}}%
\pgfpathlineto{\pgfqpoint{1.857699in}{1.184413in}}%
\pgfpathlineto{\pgfqpoint{1.860520in}{1.181037in}}%
\pgfpathlineto{\pgfqpoint{1.860834in}{1.180662in}}%
\pgfpathlineto{\pgfqpoint{1.863655in}{1.177285in}}%
\pgfpathlineto{\pgfqpoint{1.866790in}{1.177285in}}%
\pgfpathlineto{\pgfqpoint{1.867103in}{1.176910in}}%
\pgfpathlineto{\pgfqpoint{1.869924in}{1.173534in}}%
\pgfpathlineto{\pgfqpoint{1.870238in}{1.173158in}}%
\pgfpathlineto{\pgfqpoint{1.873059in}{1.169782in}}%
\pgfpathlineto{\pgfqpoint{1.873373in}{1.169407in}}%
\pgfpathlineto{\pgfqpoint{1.876194in}{1.166031in}}%
\pgfpathlineto{\pgfqpoint{1.879329in}{1.166031in}}%
\pgfpathlineto{\pgfqpoint{1.879642in}{1.165655in}}%
\pgfpathlineto{\pgfqpoint{1.882463in}{1.162279in}}%
\pgfpathlineto{\pgfqpoint{1.882777in}{1.161904in}}%
\pgfpathlineto{\pgfqpoint{1.885598in}{1.158527in}}%
\pgfpathlineto{\pgfqpoint{1.885912in}{1.158152in}}%
\pgfpathlineto{\pgfqpoint{1.888733in}{1.154776in}}%
\pgfpathlineto{\pgfqpoint{1.891868in}{1.154776in}}%
\pgfpathlineto{\pgfqpoint{1.892181in}{1.154401in}}%
\pgfpathlineto{\pgfqpoint{1.895002in}{1.151024in}}%
\pgfpathlineto{\pgfqpoint{1.895316in}{1.150649in}}%
\pgfpathlineto{\pgfqpoint{1.898137in}{1.147273in}}%
\pgfpathlineto{\pgfqpoint{1.898450in}{1.146898in}}%
\pgfpathlineto{\pgfqpoint{1.901272in}{1.143521in}}%
\pgfpathlineto{\pgfqpoint{1.904407in}{1.143521in}}%
\pgfpathlineto{\pgfqpoint{1.904720in}{1.143146in}}%
\pgfpathlineto{\pgfqpoint{1.907541in}{1.139770in}}%
\pgfpathlineto{\pgfqpoint{1.907855in}{1.139395in}}%
\pgfpathlineto{\pgfqpoint{1.910676in}{1.136018in}}%
\pgfpathlineto{\pgfqpoint{1.910989in}{1.135643in}}%
\pgfpathlineto{\pgfqpoint{1.913811in}{1.132267in}}%
\pgfpathlineto{\pgfqpoint{1.916945in}{1.132267in}}%
\pgfpathlineto{\pgfqpoint{1.917259in}{1.131892in}}%
\pgfpathlineto{\pgfqpoint{1.920080in}{1.128515in}}%
\pgfpathlineto{\pgfqpoint{1.920394in}{1.128140in}}%
\pgfpathlineto{\pgfqpoint{1.923215in}{1.124764in}}%
\pgfpathlineto{\pgfqpoint{1.923528in}{1.124388in}}%
\pgfpathlineto{\pgfqpoint{1.926350in}{1.121012in}}%
\pgfpathlineto{\pgfqpoint{1.929484in}{1.121012in}}%
\pgfpathlineto{\pgfqpoint{1.929798in}{1.120637in}}%
\pgfpathlineto{\pgfqpoint{1.932619in}{1.117261in}}%
\pgfpathlineto{\pgfqpoint{1.932933in}{1.116885in}}%
\pgfpathlineto{\pgfqpoint{1.935754in}{1.113509in}}%
\pgfpathlineto{\pgfqpoint{1.936067in}{1.113134in}}%
\pgfpathlineto{\pgfqpoint{1.938889in}{1.109758in}}%
\pgfpathlineto{\pgfqpoint{1.939202in}{1.109382in}}%
\pgfpathlineto{\pgfqpoint{1.942023in}{1.106006in}}%
\pgfpathlineto{\pgfqpoint{1.945158in}{1.106006in}}%
\pgfpathlineto{\pgfqpoint{1.945472in}{1.105631in}}%
\pgfpathlineto{\pgfqpoint{1.948293in}{1.102254in}}%
\pgfpathlineto{\pgfqpoint{1.948606in}{1.101879in}}%
\pgfpathlineto{\pgfqpoint{1.951428in}{1.098503in}}%
\pgfpathlineto{\pgfqpoint{1.951741in}{1.098128in}}%
\pgfpathlineto{\pgfqpoint{1.954562in}{1.094751in}}%
\pgfpathlineto{\pgfqpoint{1.957697in}{1.094751in}}%
\pgfpathlineto{\pgfqpoint{1.958011in}{1.094376in}}%
\pgfpathlineto{\pgfqpoint{1.960832in}{1.091000in}}%
\pgfpathlineto{\pgfqpoint{1.961145in}{1.090625in}}%
\pgfpathlineto{\pgfqpoint{1.963967in}{1.087248in}}%
\pgfpathlineto{\pgfqpoint{1.964280in}{1.086873in}}%
\pgfpathlineto{\pgfqpoint{1.967101in}{1.083497in}}%
\pgfpathlineto{\pgfqpoint{1.970236in}{1.083497in}}%
\pgfpathlineto{\pgfqpoint{1.970550in}{1.083122in}}%
\pgfpathlineto{\pgfqpoint{1.973371in}{1.079745in}}%
\pgfpathlineto{\pgfqpoint{1.973684in}{1.079370in}}%
\pgfpathlineto{\pgfqpoint{1.976506in}{1.075994in}}%
\pgfpathlineto{\pgfqpoint{1.976819in}{1.075619in}}%
\pgfpathlineto{\pgfqpoint{1.979640in}{1.072242in}}%
\pgfpathlineto{\pgfqpoint{1.982775in}{1.072242in}}%
\pgfpathlineto{\pgfqpoint{1.983089in}{1.071867in}}%
\pgfpathlineto{\pgfqpoint{1.985910in}{1.068491in}}%
\pgfpathlineto{\pgfqpoint{1.986223in}{1.068115in}}%
\pgfpathlineto{\pgfqpoint{1.989045in}{1.064739in}}%
\pgfpathlineto{\pgfqpoint{1.989358in}{1.064364in}}%
\pgfpathlineto{\pgfqpoint{1.992179in}{1.060988in}}%
\pgfpathlineto{\pgfqpoint{1.995314in}{1.060988in}}%
\pgfpathlineto{\pgfqpoint{1.995628in}{1.060612in}}%
\pgfpathlineto{\pgfqpoint{1.998449in}{1.057236in}}%
\pgfpathlineto{\pgfqpoint{1.998762in}{1.056861in}}%
\pgfpathlineto{\pgfqpoint{2.001584in}{1.053484in}}%
\pgfpathlineto{\pgfqpoint{2.001897in}{1.053109in}}%
\pgfpathlineto{\pgfqpoint{2.004718in}{1.049733in}}%
\pgfpathlineto{\pgfqpoint{2.007853in}{1.049733in}}%
\pgfpathlineto{\pgfqpoint{2.008166in}{1.049358in}}%
\pgfpathlineto{\pgfqpoint{2.010988in}{1.045981in}}%
\pgfpathlineto{\pgfqpoint{2.011301in}{1.045606in}}%
\pgfpathlineto{\pgfqpoint{2.014122in}{1.042230in}}%
\pgfpathlineto{\pgfqpoint{2.014436in}{1.041855in}}%
\pgfpathlineto{\pgfqpoint{2.017257in}{1.038478in}}%
\pgfpathlineto{\pgfqpoint{2.017571in}{1.038103in}}%
\pgfpathlineto{\pgfqpoint{2.020392in}{1.034727in}}%
\pgfpathlineto{\pgfqpoint{2.023527in}{1.034727in}}%
\pgfpathlineto{\pgfqpoint{2.023840in}{1.034352in}}%
\pgfpathlineto{\pgfqpoint{2.026661in}{1.030975in}}%
\pgfpathlineto{\pgfqpoint{2.026975in}{1.030600in}}%
\pgfpathlineto{\pgfqpoint{2.029796in}{1.027224in}}%
\pgfpathlineto{\pgfqpoint{2.030110in}{1.026849in}}%
\pgfpathlineto{\pgfqpoint{2.032931in}{1.023472in}}%
\pgfpathlineto{\pgfqpoint{2.036066in}{1.023472in}}%
\pgfpathlineto{\pgfqpoint{2.036379in}{1.023097in}}%
\pgfpathlineto{\pgfqpoint{2.039200in}{1.019721in}}%
\pgfpathlineto{\pgfqpoint{2.039514in}{1.019346in}}%
\pgfpathlineto{\pgfqpoint{2.042335in}{1.015969in}}%
\pgfpathlineto{\pgfqpoint{2.042649in}{1.015594in}}%
\pgfpathlineto{\pgfqpoint{2.045470in}{1.012218in}}%
\pgfpathlineto{\pgfqpoint{2.048605in}{1.012218in}}%
\pgfpathlineto{\pgfqpoint{2.048918in}{1.011842in}}%
\pgfpathlineto{\pgfqpoint{2.051739in}{1.008466in}}%
\pgfpathlineto{\pgfqpoint{2.052053in}{1.008091in}}%
\pgfpathlineto{\pgfqpoint{2.054874in}{1.004715in}}%
\pgfpathlineto{\pgfqpoint{2.055188in}{1.004339in}}%
\pgfpathlineto{\pgfqpoint{2.058009in}{1.000963in}}%
\pgfpathlineto{\pgfqpoint{2.061144in}{1.000963in}}%
\pgfpathlineto{\pgfqpoint{2.061457in}{1.000588in}}%
\pgfpathlineto{\pgfqpoint{2.064278in}{0.997211in}}%
\pgfpathlineto{\pgfqpoint{2.064592in}{0.996836in}}%
\pgfpathlineto{\pgfqpoint{2.067413in}{0.993460in}}%
\pgfpathlineto{\pgfqpoint{2.067727in}{0.993085in}}%
\pgfpathlineto{\pgfqpoint{2.070548in}{0.989708in}}%
\pgfpathlineto{\pgfqpoint{2.073683in}{0.989708in}}%
\pgfpathlineto{\pgfqpoint{2.073996in}{0.989333in}}%
\pgfpathlineto{\pgfqpoint{2.076817in}{0.985957in}}%
\pgfpathlineto{\pgfqpoint{2.077131in}{0.985582in}}%
\pgfpathlineto{\pgfqpoint{2.079952in}{0.982205in}}%
\pgfpathlineto{\pgfqpoint{2.080266in}{0.981830in}}%
\pgfpathlineto{\pgfqpoint{2.083087in}{0.978454in}}%
\pgfpathlineto{\pgfqpoint{2.086222in}{0.978454in}}%
\pgfpathlineto{\pgfqpoint{2.086535in}{0.978079in}}%
\pgfpathlineto{\pgfqpoint{2.089356in}{0.974702in}}%
\pgfpathlineto{\pgfqpoint{2.089670in}{0.974327in}}%
\pgfpathlineto{\pgfqpoint{2.092491in}{0.970951in}}%
\pgfpathlineto{\pgfqpoint{2.092805in}{0.970576in}}%
\pgfpathlineto{\pgfqpoint{2.095626in}{0.967199in}}%
\pgfpathlineto{\pgfqpoint{2.095939in}{0.966824in}}%
\pgfpathlineto{\pgfqpoint{2.098761in}{0.963448in}}%
\pgfpathlineto{\pgfqpoint{2.101895in}{0.963448in}}%
\pgfpathlineto{\pgfqpoint{2.102209in}{0.963073in}}%
\pgfpathlineto{\pgfqpoint{2.105030in}{0.959696in}}%
\pgfpathlineto{\pgfqpoint{2.105343in}{0.959321in}}%
\pgfpathlineto{\pgfqpoint{2.108165in}{0.955945in}}%
\pgfpathlineto{\pgfqpoint{2.108478in}{0.955569in}}%
\pgfpathlineto{\pgfqpoint{2.111299in}{0.952193in}}%
\pgfpathlineto{\pgfqpoint{2.114434in}{0.952193in}}%
\pgfpathlineto{\pgfqpoint{2.114748in}{0.951818in}}%
\pgfpathlineto{\pgfqpoint{2.117569in}{0.948442in}}%
\pgfpathlineto{\pgfqpoint{2.117882in}{0.948066in}}%
\pgfpathlineto{\pgfqpoint{2.120704in}{0.944690in}}%
\pgfpathlineto{\pgfqpoint{2.121017in}{0.944315in}}%
\pgfpathlineto{\pgfqpoint{2.123838in}{0.940938in}}%
\pgfpathlineto{\pgfqpoint{2.126973in}{0.940938in}}%
\pgfpathlineto{\pgfqpoint{2.127287in}{0.940563in}}%
\pgfpathlineto{\pgfqpoint{2.130108in}{0.937187in}}%
\pgfpathlineto{\pgfqpoint{2.130421in}{0.936812in}}%
\pgfpathlineto{\pgfqpoint{2.133243in}{0.933435in}}%
\pgfpathlineto{\pgfqpoint{2.133556in}{0.933060in}}%
\pgfpathlineto{\pgfqpoint{2.136377in}{0.929684in}}%
\pgfpathlineto{\pgfqpoint{2.139512in}{0.929684in}}%
\pgfpathlineto{\pgfqpoint{2.139826in}{0.929309in}}%
\pgfpathlineto{\pgfqpoint{2.142647in}{0.925932in}}%
\pgfpathlineto{\pgfqpoint{2.142960in}{0.925557in}}%
\pgfpathlineto{\pgfqpoint{2.145782in}{0.922181in}}%
\pgfpathlineto{\pgfqpoint{2.146095in}{0.921806in}}%
\pgfpathlineto{\pgfqpoint{2.148916in}{0.918429in}}%
\pgfpathlineto{\pgfqpoint{2.152051in}{0.918429in}}%
\pgfpathlineto{\pgfqpoint{2.152365in}{0.918054in}}%
\pgfpathlineto{\pgfqpoint{2.155186in}{0.914678in}}%
\pgfpathlineto{\pgfqpoint{2.155499in}{0.914303in}}%
\pgfpathlineto{\pgfqpoint{2.158321in}{0.910926in}}%
\pgfpathlineto{\pgfqpoint{2.158634in}{0.910551in}}%
\pgfpathlineto{\pgfqpoint{2.161455in}{0.907175in}}%
\pgfpathlineto{\pgfqpoint{2.161769in}{0.906799in}}%
\pgfpathlineto{\pgfqpoint{2.164590in}{0.903423in}}%
\pgfpathlineto{\pgfqpoint{2.167725in}{0.903423in}}%
\pgfpathlineto{\pgfqpoint{2.168038in}{0.903048in}}%
\pgfpathlineto{\pgfqpoint{2.170860in}{0.899672in}}%
\pgfpathlineto{\pgfqpoint{2.171173in}{0.899296in}}%
\pgfpathclose%
\pgfpathmoveto{\pgfqpoint{2.171643in}{0.899296in}}%
\pgfpathlineto{\pgfqpoint{2.170860in}{0.900234in}}%
\pgfpathlineto{\pgfqpoint{2.168509in}{0.903048in}}%
\pgfpathlineto{\pgfqpoint{2.167725in}{0.903986in}}%
\pgfpathlineto{\pgfqpoint{2.164590in}{0.903986in}}%
\pgfpathlineto{\pgfqpoint{2.162239in}{0.906799in}}%
\pgfpathlineto{\pgfqpoint{2.161455in}{0.907737in}}%
\pgfpathlineto{\pgfqpoint{2.159104in}{0.910551in}}%
\pgfpathlineto{\pgfqpoint{2.158321in}{0.911489in}}%
\pgfpathlineto{\pgfqpoint{2.155970in}{0.914303in}}%
\pgfpathlineto{\pgfqpoint{2.155186in}{0.915240in}}%
\pgfpathlineto{\pgfqpoint{2.152835in}{0.918054in}}%
\pgfpathlineto{\pgfqpoint{2.152051in}{0.918992in}}%
\pgfpathlineto{\pgfqpoint{2.148916in}{0.918992in}}%
\pgfpathlineto{\pgfqpoint{2.146565in}{0.921806in}}%
\pgfpathlineto{\pgfqpoint{2.145782in}{0.922744in}}%
\pgfpathlineto{\pgfqpoint{2.143431in}{0.925557in}}%
\pgfpathlineto{\pgfqpoint{2.142647in}{0.926495in}}%
\pgfpathlineto{\pgfqpoint{2.140296in}{0.929309in}}%
\pgfpathlineto{\pgfqpoint{2.139512in}{0.930247in}}%
\pgfpathlineto{\pgfqpoint{2.136377in}{0.930247in}}%
\pgfpathlineto{\pgfqpoint{2.134026in}{0.933060in}}%
\pgfpathlineto{\pgfqpoint{2.133243in}{0.933998in}}%
\pgfpathlineto{\pgfqpoint{2.130892in}{0.936812in}}%
\pgfpathlineto{\pgfqpoint{2.130108in}{0.937750in}}%
\pgfpathlineto{\pgfqpoint{2.127757in}{0.940563in}}%
\pgfpathlineto{\pgfqpoint{2.126973in}{0.941501in}}%
\pgfpathlineto{\pgfqpoint{2.123838in}{0.941501in}}%
\pgfpathlineto{\pgfqpoint{2.121487in}{0.944315in}}%
\pgfpathlineto{\pgfqpoint{2.120704in}{0.945253in}}%
\pgfpathlineto{\pgfqpoint{2.118353in}{0.948066in}}%
\pgfpathlineto{\pgfqpoint{2.117569in}{0.949004in}}%
\pgfpathlineto{\pgfqpoint{2.115218in}{0.951818in}}%
\pgfpathlineto{\pgfqpoint{2.114434in}{0.952756in}}%
\pgfpathlineto{\pgfqpoint{2.111299in}{0.952756in}}%
\pgfpathlineto{\pgfqpoint{2.108948in}{0.955569in}}%
\pgfpathlineto{\pgfqpoint{2.108165in}{0.956507in}}%
\pgfpathlineto{\pgfqpoint{2.105814in}{0.959321in}}%
\pgfpathlineto{\pgfqpoint{2.105030in}{0.960259in}}%
\pgfpathlineto{\pgfqpoint{2.102679in}{0.963073in}}%
\pgfpathlineto{\pgfqpoint{2.101895in}{0.964010in}}%
\pgfpathlineto{\pgfqpoint{2.098761in}{0.964010in}}%
\pgfpathlineto{\pgfqpoint{2.096409in}{0.966824in}}%
\pgfpathlineto{\pgfqpoint{2.095626in}{0.967762in}}%
\pgfpathlineto{\pgfqpoint{2.093275in}{0.970576in}}%
\pgfpathlineto{\pgfqpoint{2.092491in}{0.971513in}}%
\pgfpathlineto{\pgfqpoint{2.090140in}{0.974327in}}%
\pgfpathlineto{\pgfqpoint{2.089356in}{0.975265in}}%
\pgfpathlineto{\pgfqpoint{2.087005in}{0.978079in}}%
\pgfpathlineto{\pgfqpoint{2.086222in}{0.979017in}}%
\pgfpathlineto{\pgfqpoint{2.083087in}{0.979017in}}%
\pgfpathlineto{\pgfqpoint{2.080736in}{0.981830in}}%
\pgfpathlineto{\pgfqpoint{2.079952in}{0.982768in}}%
\pgfpathlineto{\pgfqpoint{2.077601in}{0.985582in}}%
\pgfpathlineto{\pgfqpoint{2.076817in}{0.986520in}}%
\pgfpathlineto{\pgfqpoint{2.074466in}{0.989333in}}%
\pgfpathlineto{\pgfqpoint{2.073683in}{0.990271in}}%
\pgfpathlineto{\pgfqpoint{2.070548in}{0.990271in}}%
\pgfpathlineto{\pgfqpoint{2.068197in}{0.993085in}}%
\pgfpathlineto{\pgfqpoint{2.067413in}{0.994023in}}%
\pgfpathlineto{\pgfqpoint{2.065062in}{0.996836in}}%
\pgfpathlineto{\pgfqpoint{2.064278in}{0.997774in}}%
\pgfpathlineto{\pgfqpoint{2.061927in}{1.000588in}}%
\pgfpathlineto{\pgfqpoint{2.061144in}{1.001526in}}%
\pgfpathlineto{\pgfqpoint{2.058009in}{1.001526in}}%
\pgfpathlineto{\pgfqpoint{2.055658in}{1.004339in}}%
\pgfpathlineto{\pgfqpoint{2.054874in}{1.005277in}}%
\pgfpathlineto{\pgfqpoint{2.052523in}{1.008091in}}%
\pgfpathlineto{\pgfqpoint{2.051739in}{1.009029in}}%
\pgfpathlineto{\pgfqpoint{2.049388in}{1.011842in}}%
\pgfpathlineto{\pgfqpoint{2.048605in}{1.012780in}}%
\pgfpathlineto{\pgfqpoint{2.045470in}{1.012780in}}%
\pgfpathlineto{\pgfqpoint{2.043119in}{1.015594in}}%
\pgfpathlineto{\pgfqpoint{2.042335in}{1.016532in}}%
\pgfpathlineto{\pgfqpoint{2.039984in}{1.019346in}}%
\pgfpathlineto{\pgfqpoint{2.039200in}{1.020283in}}%
\pgfpathlineto{\pgfqpoint{2.036849in}{1.023097in}}%
\pgfpathlineto{\pgfqpoint{2.036066in}{1.024035in}}%
\pgfpathlineto{\pgfqpoint{2.032931in}{1.024035in}}%
\pgfpathlineto{\pgfqpoint{2.030580in}{1.026849in}}%
\pgfpathlineto{\pgfqpoint{2.029796in}{1.027786in}}%
\pgfpathlineto{\pgfqpoint{2.027445in}{1.030600in}}%
\pgfpathlineto{\pgfqpoint{2.026661in}{1.031538in}}%
\pgfpathlineto{\pgfqpoint{2.024310in}{1.034352in}}%
\pgfpathlineto{\pgfqpoint{2.023527in}{1.035290in}}%
\pgfpathlineto{\pgfqpoint{2.020392in}{1.035290in}}%
\pgfpathlineto{\pgfqpoint{2.018041in}{1.038103in}}%
\pgfpathlineto{\pgfqpoint{2.017257in}{1.039041in}}%
\pgfpathlineto{\pgfqpoint{2.014906in}{1.041855in}}%
\pgfpathlineto{\pgfqpoint{2.014122in}{1.042793in}}%
\pgfpathlineto{\pgfqpoint{2.011771in}{1.045606in}}%
\pgfpathlineto{\pgfqpoint{2.010988in}{1.046544in}}%
\pgfpathlineto{\pgfqpoint{2.008637in}{1.049358in}}%
\pgfpathlineto{\pgfqpoint{2.007853in}{1.050296in}}%
\pgfpathlineto{\pgfqpoint{2.004718in}{1.050296in}}%
\pgfpathlineto{\pgfqpoint{2.002367in}{1.053109in}}%
\pgfpathlineto{\pgfqpoint{2.001584in}{1.054047in}}%
\pgfpathlineto{\pgfqpoint{1.999232in}{1.056861in}}%
\pgfpathlineto{\pgfqpoint{1.998449in}{1.057799in}}%
\pgfpathlineto{\pgfqpoint{1.996098in}{1.060612in}}%
\pgfpathlineto{\pgfqpoint{1.995314in}{1.061550in}}%
\pgfpathlineto{\pgfqpoint{1.992179in}{1.061550in}}%
\pgfpathlineto{\pgfqpoint{1.989828in}{1.064364in}}%
\pgfpathlineto{\pgfqpoint{1.989045in}{1.065302in}}%
\pgfpathlineto{\pgfqpoint{1.986693in}{1.068115in}}%
\pgfpathlineto{\pgfqpoint{1.985910in}{1.069053in}}%
\pgfpathlineto{\pgfqpoint{1.983559in}{1.071867in}}%
\pgfpathlineto{\pgfqpoint{1.982775in}{1.072805in}}%
\pgfpathlineto{\pgfqpoint{1.979640in}{1.072805in}}%
\pgfpathlineto{\pgfqpoint{1.977289in}{1.075619in}}%
\pgfpathlineto{\pgfqpoint{1.976506in}{1.076556in}}%
\pgfpathlineto{\pgfqpoint{1.974155in}{1.079370in}}%
\pgfpathlineto{\pgfqpoint{1.973371in}{1.080308in}}%
\pgfpathlineto{\pgfqpoint{1.971020in}{1.083122in}}%
\pgfpathlineto{\pgfqpoint{1.970236in}{1.084059in}}%
\pgfpathlineto{\pgfqpoint{1.967101in}{1.084059in}}%
\pgfpathlineto{\pgfqpoint{1.964750in}{1.086873in}}%
\pgfpathlineto{\pgfqpoint{1.963967in}{1.087811in}}%
\pgfpathlineto{\pgfqpoint{1.961616in}{1.090625in}}%
\pgfpathlineto{\pgfqpoint{1.960832in}{1.091563in}}%
\pgfpathlineto{\pgfqpoint{1.958481in}{1.094376in}}%
\pgfpathlineto{\pgfqpoint{1.957697in}{1.095314in}}%
\pgfpathlineto{\pgfqpoint{1.954562in}{1.095314in}}%
\pgfpathlineto{\pgfqpoint{1.952211in}{1.098128in}}%
\pgfpathlineto{\pgfqpoint{1.951428in}{1.099066in}}%
\pgfpathlineto{\pgfqpoint{1.949077in}{1.101879in}}%
\pgfpathlineto{\pgfqpoint{1.948293in}{1.102817in}}%
\pgfpathlineto{\pgfqpoint{1.945942in}{1.105631in}}%
\pgfpathlineto{\pgfqpoint{1.945158in}{1.106569in}}%
\pgfpathlineto{\pgfqpoint{1.942023in}{1.106569in}}%
\pgfpathlineto{\pgfqpoint{1.939672in}{1.109382in}}%
\pgfpathlineto{\pgfqpoint{1.938889in}{1.110320in}}%
\pgfpathlineto{\pgfqpoint{1.936538in}{1.113134in}}%
\pgfpathlineto{\pgfqpoint{1.935754in}{1.114072in}}%
\pgfpathlineto{\pgfqpoint{1.933403in}{1.116885in}}%
\pgfpathlineto{\pgfqpoint{1.932619in}{1.117823in}}%
\pgfpathlineto{\pgfqpoint{1.930268in}{1.120637in}}%
\pgfpathlineto{\pgfqpoint{1.929484in}{1.121575in}}%
\pgfpathlineto{\pgfqpoint{1.926350in}{1.121575in}}%
\pgfpathlineto{\pgfqpoint{1.923999in}{1.124388in}}%
\pgfpathlineto{\pgfqpoint{1.923215in}{1.125326in}}%
\pgfpathlineto{\pgfqpoint{1.920864in}{1.128140in}}%
\pgfpathlineto{\pgfqpoint{1.920080in}{1.129078in}}%
\pgfpathlineto{\pgfqpoint{1.917729in}{1.131892in}}%
\pgfpathlineto{\pgfqpoint{1.916945in}{1.132829in}}%
\pgfpathlineto{\pgfqpoint{1.913811in}{1.132829in}}%
\pgfpathlineto{\pgfqpoint{1.911460in}{1.135643in}}%
\pgfpathlineto{\pgfqpoint{1.910676in}{1.136581in}}%
\pgfpathlineto{\pgfqpoint{1.908325in}{1.139395in}}%
\pgfpathlineto{\pgfqpoint{1.907541in}{1.140333in}}%
\pgfpathlineto{\pgfqpoint{1.905190in}{1.143146in}}%
\pgfpathlineto{\pgfqpoint{1.904407in}{1.144084in}}%
\pgfpathlineto{\pgfqpoint{1.901272in}{1.144084in}}%
\pgfpathlineto{\pgfqpoint{1.898921in}{1.146898in}}%
\pgfpathlineto{\pgfqpoint{1.898137in}{1.147836in}}%
\pgfpathlineto{\pgfqpoint{1.895786in}{1.150649in}}%
\pgfpathlineto{\pgfqpoint{1.895002in}{1.151587in}}%
\pgfpathlineto{\pgfqpoint{1.892651in}{1.154401in}}%
\pgfpathlineto{\pgfqpoint{1.891868in}{1.155339in}}%
\pgfpathlineto{\pgfqpoint{1.888733in}{1.155339in}}%
\pgfpathlineto{\pgfqpoint{1.886382in}{1.158152in}}%
\pgfpathlineto{\pgfqpoint{1.885598in}{1.159090in}}%
\pgfpathlineto{\pgfqpoint{1.883247in}{1.161904in}}%
\pgfpathlineto{\pgfqpoint{1.882463in}{1.162842in}}%
\pgfpathlineto{\pgfqpoint{1.880112in}{1.165655in}}%
\pgfpathlineto{\pgfqpoint{1.879329in}{1.166593in}}%
\pgfpathlineto{\pgfqpoint{1.876194in}{1.166593in}}%
\pgfpathlineto{\pgfqpoint{1.873843in}{1.169407in}}%
\pgfpathlineto{\pgfqpoint{1.873059in}{1.170345in}}%
\pgfpathlineto{\pgfqpoint{1.870708in}{1.173158in}}%
\pgfpathlineto{\pgfqpoint{1.869924in}{1.174096in}}%
\pgfpathlineto{\pgfqpoint{1.867573in}{1.176910in}}%
\pgfpathlineto{\pgfqpoint{1.866790in}{1.177848in}}%
\pgfpathlineto{\pgfqpoint{1.863655in}{1.177848in}}%
\pgfpathlineto{\pgfqpoint{1.861304in}{1.180662in}}%
\pgfpathlineto{\pgfqpoint{1.860520in}{1.181599in}}%
\pgfpathlineto{\pgfqpoint{1.858169in}{1.184413in}}%
\pgfpathlineto{\pgfqpoint{1.857385in}{1.185351in}}%
\pgfpathlineto{\pgfqpoint{1.855034in}{1.188165in}}%
\pgfpathlineto{\pgfqpoint{1.854251in}{1.189102in}}%
\pgfpathlineto{\pgfqpoint{1.851900in}{1.191916in}}%
\pgfpathlineto{\pgfqpoint{1.851116in}{1.192854in}}%
\pgfpathlineto{\pgfqpoint{1.847981in}{1.192854in}}%
\pgfpathlineto{\pgfqpoint{1.845630in}{1.195668in}}%
\pgfpathlineto{\pgfqpoint{1.844846in}{1.196606in}}%
\pgfpathlineto{\pgfqpoint{1.842495in}{1.199419in}}%
\pgfpathlineto{\pgfqpoint{1.841712in}{1.200357in}}%
\pgfpathlineto{\pgfqpoint{1.839361in}{1.203171in}}%
\pgfpathlineto{\pgfqpoint{1.838577in}{1.204109in}}%
\pgfpathlineto{\pgfqpoint{1.835442in}{1.204109in}}%
\pgfpathlineto{\pgfqpoint{1.833091in}{1.206922in}}%
\pgfpathlineto{\pgfqpoint{1.832307in}{1.207860in}}%
\pgfpathlineto{\pgfqpoint{1.829956in}{1.210674in}}%
\pgfpathlineto{\pgfqpoint{1.829173in}{1.211612in}}%
\pgfpathlineto{\pgfqpoint{1.826822in}{1.214425in}}%
\pgfpathlineto{\pgfqpoint{1.826038in}{1.215363in}}%
\pgfpathlineto{\pgfqpoint{1.822903in}{1.215363in}}%
\pgfpathlineto{\pgfqpoint{1.820552in}{1.218177in}}%
\pgfpathlineto{\pgfqpoint{1.819768in}{1.219115in}}%
\pgfpathlineto{\pgfqpoint{1.817417in}{1.221928in}}%
\pgfpathlineto{\pgfqpoint{1.816634in}{1.222866in}}%
\pgfpathlineto{\pgfqpoint{1.814283in}{1.225680in}}%
\pgfpathlineto{\pgfqpoint{1.813499in}{1.226618in}}%
\pgfpathlineto{\pgfqpoint{1.810364in}{1.226618in}}%
\pgfpathlineto{\pgfqpoint{1.808013in}{1.229431in}}%
\pgfpathlineto{\pgfqpoint{1.807229in}{1.230369in}}%
\pgfpathlineto{\pgfqpoint{1.804878in}{1.233183in}}%
\pgfpathlineto{\pgfqpoint{1.804095in}{1.234121in}}%
\pgfpathlineto{\pgfqpoint{1.801744in}{1.236935in}}%
\pgfpathlineto{\pgfqpoint{1.800960in}{1.237872in}}%
\pgfpathlineto{\pgfqpoint{1.797825in}{1.237872in}}%
\pgfpathlineto{\pgfqpoint{1.795474in}{1.240686in}}%
\pgfpathlineto{\pgfqpoint{1.794691in}{1.241624in}}%
\pgfpathlineto{\pgfqpoint{1.792339in}{1.244438in}}%
\pgfpathlineto{\pgfqpoint{1.791556in}{1.245375in}}%
\pgfpathlineto{\pgfqpoint{1.789205in}{1.248189in}}%
\pgfpathlineto{\pgfqpoint{1.788421in}{1.249127in}}%
\pgfpathlineto{\pgfqpoint{1.786070in}{1.251941in}}%
\pgfpathlineto{\pgfqpoint{1.785286in}{1.252879in}}%
\pgfpathlineto{\pgfqpoint{1.782152in}{1.252879in}}%
\pgfpathlineto{\pgfqpoint{1.779800in}{1.255692in}}%
\pgfpathlineto{\pgfqpoint{1.779017in}{1.256630in}}%
\pgfpathlineto{\pgfqpoint{1.776666in}{1.259444in}}%
\pgfpathlineto{\pgfqpoint{1.775882in}{1.260382in}}%
\pgfpathlineto{\pgfqpoint{1.773531in}{1.263195in}}%
\pgfpathlineto{\pgfqpoint{1.772747in}{1.264133in}}%
\pgfpathlineto{\pgfqpoint{1.769613in}{1.264133in}}%
\pgfpathlineto{\pgfqpoint{1.767262in}{1.266947in}}%
\pgfpathlineto{\pgfqpoint{1.766478in}{1.267885in}}%
\pgfpathlineto{\pgfqpoint{1.764127in}{1.270698in}}%
\pgfpathlineto{\pgfqpoint{1.763343in}{1.271636in}}%
\pgfpathlineto{\pgfqpoint{1.760992in}{1.274450in}}%
\pgfpathlineto{\pgfqpoint{1.760208in}{1.275388in}}%
\pgfpathlineto{\pgfqpoint{1.757074in}{1.275388in}}%
\pgfpathlineto{\pgfqpoint{1.754723in}{1.278201in}}%
\pgfpathlineto{\pgfqpoint{1.753939in}{1.279139in}}%
\pgfpathlineto{\pgfqpoint{1.751588in}{1.281953in}}%
\pgfpathlineto{\pgfqpoint{1.750804in}{1.282891in}}%
\pgfpathlineto{\pgfqpoint{1.748453in}{1.285704in}}%
\pgfpathlineto{\pgfqpoint{1.747669in}{1.286642in}}%
\pgfpathlineto{\pgfqpoint{1.744535in}{1.286642in}}%
\pgfpathlineto{\pgfqpoint{1.742184in}{1.289456in}}%
\pgfpathlineto{\pgfqpoint{1.741400in}{1.290394in}}%
\pgfpathlineto{\pgfqpoint{1.739049in}{1.293208in}}%
\pgfpathlineto{\pgfqpoint{1.738265in}{1.294145in}}%
\pgfpathlineto{\pgfqpoint{1.735914in}{1.296959in}}%
\pgfpathlineto{\pgfqpoint{1.735130in}{1.297897in}}%
\pgfpathlineto{\pgfqpoint{1.731996in}{1.297897in}}%
\pgfpathlineto{\pgfqpoint{1.729645in}{1.300711in}}%
\pgfpathlineto{\pgfqpoint{1.728861in}{1.301648in}}%
\pgfpathlineto{\pgfqpoint{1.726510in}{1.304462in}}%
\pgfpathlineto{\pgfqpoint{1.725726in}{1.305400in}}%
\pgfpathlineto{\pgfqpoint{1.723375in}{1.308214in}}%
\pgfpathlineto{\pgfqpoint{1.722591in}{1.309152in}}%
\pgfpathlineto{\pgfqpoint{1.719457in}{1.309152in}}%
\pgfpathlineto{\pgfqpoint{1.717106in}{1.311965in}}%
\pgfpathlineto{\pgfqpoint{1.716322in}{1.312903in}}%
\pgfpathlineto{\pgfqpoint{1.713971in}{1.315717in}}%
\pgfpathlineto{\pgfqpoint{1.713187in}{1.316655in}}%
\pgfpathlineto{\pgfqpoint{1.710836in}{1.319468in}}%
\pgfpathlineto{\pgfqpoint{1.710052in}{1.320406in}}%
\pgfpathlineto{\pgfqpoint{1.707701in}{1.323220in}}%
\pgfpathlineto{\pgfqpoint{1.706918in}{1.324158in}}%
\pgfpathlineto{\pgfqpoint{1.703783in}{1.324158in}}%
\pgfpathlineto{\pgfqpoint{1.701432in}{1.326971in}}%
\pgfpathlineto{\pgfqpoint{1.700648in}{1.327909in}}%
\pgfpathlineto{\pgfqpoint{1.698297in}{1.330723in}}%
\pgfpathlineto{\pgfqpoint{1.697514in}{1.331661in}}%
\pgfpathlineto{\pgfqpoint{1.695162in}{1.334474in}}%
\pgfpathlineto{\pgfqpoint{1.694379in}{1.335412in}}%
\pgfpathlineto{\pgfqpoint{1.691244in}{1.335412in}}%
\pgfpathlineto{\pgfqpoint{1.688893in}{1.338226in}}%
\pgfpathlineto{\pgfqpoint{1.688109in}{1.339164in}}%
\pgfpathlineto{\pgfqpoint{1.685758in}{1.341977in}}%
\pgfpathlineto{\pgfqpoint{1.684975in}{1.342915in}}%
\pgfpathlineto{\pgfqpoint{1.682623in}{1.345729in}}%
\pgfpathlineto{\pgfqpoint{1.681840in}{1.346667in}}%
\pgfpathlineto{\pgfqpoint{1.678705in}{1.346667in}}%
\pgfpathlineto{\pgfqpoint{1.676354in}{1.349481in}}%
\pgfpathlineto{\pgfqpoint{1.675570in}{1.350418in}}%
\pgfpathlineto{\pgfqpoint{1.673219in}{1.353232in}}%
\pgfpathlineto{\pgfqpoint{1.672436in}{1.354170in}}%
\pgfpathlineto{\pgfqpoint{1.670085in}{1.356984in}}%
\pgfpathlineto{\pgfqpoint{1.669301in}{1.357922in}}%
\pgfpathlineto{\pgfqpoint{1.666166in}{1.357922in}}%
\pgfpathlineto{\pgfqpoint{1.663815in}{1.360735in}}%
\pgfpathlineto{\pgfqpoint{1.663031in}{1.361673in}}%
\pgfpathlineto{\pgfqpoint{1.660680in}{1.364487in}}%
\pgfpathlineto{\pgfqpoint{1.659897in}{1.365425in}}%
\pgfpathlineto{\pgfqpoint{1.657546in}{1.368238in}}%
\pgfpathlineto{\pgfqpoint{1.656762in}{1.369176in}}%
\pgfpathlineto{\pgfqpoint{1.653627in}{1.369176in}}%
\pgfpathlineto{\pgfqpoint{1.651276in}{1.371990in}}%
\pgfpathlineto{\pgfqpoint{1.650492in}{1.372928in}}%
\pgfpathlineto{\pgfqpoint{1.648141in}{1.375741in}}%
\pgfpathlineto{\pgfqpoint{1.647358in}{1.376679in}}%
\pgfpathlineto{\pgfqpoint{1.645007in}{1.379493in}}%
\pgfpathlineto{\pgfqpoint{1.644223in}{1.380431in}}%
\pgfpathlineto{\pgfqpoint{1.641088in}{1.380431in}}%
\pgfpathlineto{\pgfqpoint{1.638737in}{1.383244in}}%
\pgfpathlineto{\pgfqpoint{1.637953in}{1.384182in}}%
\pgfpathlineto{\pgfqpoint{1.635602in}{1.386996in}}%
\pgfpathlineto{\pgfqpoint{1.634819in}{1.387934in}}%
\pgfpathlineto{\pgfqpoint{1.632468in}{1.390747in}}%
\pgfpathlineto{\pgfqpoint{1.631684in}{1.391685in}}%
\pgfpathlineto{\pgfqpoint{1.629333in}{1.394499in}}%
\pgfpathlineto{\pgfqpoint{1.628549in}{1.395437in}}%
\pgfpathlineto{\pgfqpoint{1.625414in}{1.395437in}}%
\pgfpathlineto{\pgfqpoint{1.623063in}{1.398251in}}%
\pgfpathlineto{\pgfqpoint{1.622280in}{1.399188in}}%
\pgfpathlineto{\pgfqpoint{1.619929in}{1.402002in}}%
\pgfpathlineto{\pgfqpoint{1.619145in}{1.402940in}}%
\pgfpathlineto{\pgfqpoint{1.616794in}{1.405754in}}%
\pgfpathlineto{\pgfqpoint{1.616010in}{1.406691in}}%
\pgfpathlineto{\pgfqpoint{1.612875in}{1.406691in}}%
\pgfpathlineto{\pgfqpoint{1.610524in}{1.409505in}}%
\pgfpathlineto{\pgfqpoint{1.609741in}{1.410443in}}%
\pgfpathlineto{\pgfqpoint{1.607390in}{1.413257in}}%
\pgfpathlineto{\pgfqpoint{1.606606in}{1.414195in}}%
\pgfpathlineto{\pgfqpoint{1.604255in}{1.417008in}}%
\pgfpathlineto{\pgfqpoint{1.603471in}{1.417946in}}%
\pgfpathlineto{\pgfqpoint{1.600337in}{1.417946in}}%
\pgfpathlineto{\pgfqpoint{1.597985in}{1.420760in}}%
\pgfpathlineto{\pgfqpoint{1.597202in}{1.421698in}}%
\pgfpathlineto{\pgfqpoint{1.594851in}{1.424511in}}%
\pgfpathlineto{\pgfqpoint{1.594067in}{1.425449in}}%
\pgfpathlineto{\pgfqpoint{1.591716in}{1.428263in}}%
\pgfpathlineto{\pgfqpoint{1.590932in}{1.429201in}}%
\pgfpathlineto{\pgfqpoint{1.587798in}{1.429201in}}%
\pgfpathlineto{\pgfqpoint{1.585446in}{1.432014in}}%
\pgfpathlineto{\pgfqpoint{1.584663in}{1.432952in}}%
\pgfpathlineto{\pgfqpoint{1.582312in}{1.435766in}}%
\pgfpathlineto{\pgfqpoint{1.581528in}{1.436704in}}%
\pgfpathlineto{\pgfqpoint{1.579177in}{1.439517in}}%
\pgfpathlineto{\pgfqpoint{1.578393in}{1.440455in}}%
\pgfpathlineto{\pgfqpoint{1.575259in}{1.440455in}}%
\pgfpathlineto{\pgfqpoint{1.572908in}{1.443269in}}%
\pgfpathlineto{\pgfqpoint{1.572124in}{1.444207in}}%
\pgfpathlineto{\pgfqpoint{1.569773in}{1.447020in}}%
\pgfpathlineto{\pgfqpoint{1.568989in}{1.447958in}}%
\pgfpathlineto{\pgfqpoint{1.566638in}{1.450772in}}%
\pgfpathlineto{\pgfqpoint{1.565854in}{1.451710in}}%
\pgfpathlineto{\pgfqpoint{1.562720in}{1.451710in}}%
\pgfpathlineto{\pgfqpoint{1.560369in}{1.454524in}}%
\pgfpathlineto{\pgfqpoint{1.559585in}{1.455461in}}%
\pgfpathlineto{\pgfqpoint{1.557234in}{1.458275in}}%
\pgfpathlineto{\pgfqpoint{1.556450in}{1.459213in}}%
\pgfpathlineto{\pgfqpoint{1.554099in}{1.462027in}}%
\pgfpathlineto{\pgfqpoint{1.553315in}{1.462964in}}%
\pgfpathlineto{\pgfqpoint{1.550964in}{1.465778in}}%
\pgfpathlineto{\pgfqpoint{1.550181in}{1.466716in}}%
\pgfpathlineto{\pgfqpoint{1.547046in}{1.466716in}}%
\pgfpathlineto{\pgfqpoint{1.544695in}{1.469530in}}%
\pgfpathlineto{\pgfqpoint{1.543911in}{1.470468in}}%
\pgfpathlineto{\pgfqpoint{1.541560in}{1.473281in}}%
\pgfpathlineto{\pgfqpoint{1.540776in}{1.474219in}}%
\pgfpathlineto{\pgfqpoint{1.538425in}{1.477033in}}%
\pgfpathlineto{\pgfqpoint{1.537642in}{1.477971in}}%
\pgfpathlineto{\pgfqpoint{1.534507in}{1.477971in}}%
\pgfpathlineto{\pgfqpoint{1.532156in}{1.480784in}}%
\pgfpathlineto{\pgfqpoint{1.531372in}{1.481722in}}%
\pgfpathlineto{\pgfqpoint{1.529021in}{1.484536in}}%
\pgfpathlineto{\pgfqpoint{1.528237in}{1.485474in}}%
\pgfpathlineto{\pgfqpoint{1.525886in}{1.488287in}}%
\pgfpathlineto{\pgfqpoint{1.525103in}{1.489225in}}%
\pgfpathlineto{\pgfqpoint{1.521968in}{1.489225in}}%
\pgfpathlineto{\pgfqpoint{1.519617in}{1.492039in}}%
\pgfpathlineto{\pgfqpoint{1.518833in}{1.492977in}}%
\pgfpathlineto{\pgfqpoint{1.516482in}{1.495790in}}%
\pgfpathlineto{\pgfqpoint{1.515698in}{1.496728in}}%
\pgfpathlineto{\pgfqpoint{1.513347in}{1.499542in}}%
\pgfpathlineto{\pgfqpoint{1.512564in}{1.500480in}}%
\pgfpathlineto{\pgfqpoint{1.509429in}{1.500480in}}%
\pgfpathlineto{\pgfqpoint{1.507078in}{1.503293in}}%
\pgfpathlineto{\pgfqpoint{1.506294in}{1.504231in}}%
\pgfpathlineto{\pgfqpoint{1.503943in}{1.507045in}}%
\pgfpathlineto{\pgfqpoint{1.503159in}{1.507983in}}%
\pgfpathlineto{\pgfqpoint{1.500808in}{1.510797in}}%
\pgfpathlineto{\pgfqpoint{1.500025in}{1.511734in}}%
\pgfpathlineto{\pgfqpoint{1.496890in}{1.511734in}}%
\pgfpathlineto{\pgfqpoint{1.494539in}{1.514548in}}%
\pgfpathlineto{\pgfqpoint{1.493755in}{1.515486in}}%
\pgfpathlineto{\pgfqpoint{1.491404in}{1.518300in}}%
\pgfpathlineto{\pgfqpoint{1.490621in}{1.519238in}}%
\pgfpathlineto{\pgfqpoint{1.488269in}{1.522051in}}%
\pgfpathlineto{\pgfqpoint{1.487486in}{1.522989in}}%
\pgfpathlineto{\pgfqpoint{1.485135in}{1.525803in}}%
\pgfpathlineto{\pgfqpoint{1.484351in}{1.526741in}}%
\pgfpathlineto{\pgfqpoint{1.481216in}{1.526741in}}%
\pgfpathlineto{\pgfqpoint{1.478865in}{1.529554in}}%
\pgfpathlineto{\pgfqpoint{1.478082in}{1.530492in}}%
\pgfpathlineto{\pgfqpoint{1.475730in}{1.533306in}}%
\pgfpathlineto{\pgfqpoint{1.474947in}{1.534244in}}%
\pgfpathlineto{\pgfqpoint{1.472596in}{1.537057in}}%
\pgfpathlineto{\pgfqpoint{1.471812in}{1.537995in}}%
\pgfpathlineto{\pgfqpoint{1.468677in}{1.537995in}}%
\pgfpathlineto{\pgfqpoint{1.466326in}{1.540809in}}%
\pgfpathlineto{\pgfqpoint{1.465543in}{1.541747in}}%
\pgfpathlineto{\pgfqpoint{1.463192in}{1.544560in}}%
\pgfpathlineto{\pgfqpoint{1.462408in}{1.545498in}}%
\pgfpathlineto{\pgfqpoint{1.460057in}{1.548312in}}%
\pgfpathlineto{\pgfqpoint{1.459273in}{1.549250in}}%
\pgfpathlineto{\pgfqpoint{1.456138in}{1.549250in}}%
\pgfpathlineto{\pgfqpoint{1.453787in}{1.552063in}}%
\pgfpathlineto{\pgfqpoint{1.453004in}{1.553001in}}%
\pgfpathlineto{\pgfqpoint{1.450653in}{1.555815in}}%
\pgfpathlineto{\pgfqpoint{1.449869in}{1.556753in}}%
\pgfpathlineto{\pgfqpoint{1.447518in}{1.559566in}}%
\pgfpathlineto{\pgfqpoint{1.446734in}{1.560504in}}%
\pgfpathlineto{\pgfqpoint{1.443599in}{1.560504in}}%
\pgfpathlineto{\pgfqpoint{1.441248in}{1.563318in}}%
\pgfpathlineto{\pgfqpoint{1.440465in}{1.564256in}}%
\pgfpathlineto{\pgfqpoint{1.438114in}{1.567070in}}%
\pgfpathlineto{\pgfqpoint{1.437330in}{1.568007in}}%
\pgfpathlineto{\pgfqpoint{1.434979in}{1.570821in}}%
\pgfpathlineto{\pgfqpoint{1.434195in}{1.571759in}}%
\pgfpathlineto{\pgfqpoint{1.431060in}{1.571759in}}%
\pgfpathlineto{\pgfqpoint{1.428709in}{1.574573in}}%
\pgfpathlineto{\pgfqpoint{1.427926in}{1.575511in}}%
\pgfpathlineto{\pgfqpoint{1.425575in}{1.578324in}}%
\pgfpathlineto{\pgfqpoint{1.424791in}{1.579262in}}%
\pgfpathlineto{\pgfqpoint{1.422440in}{1.582076in}}%
\pgfpathlineto{\pgfqpoint{1.421656in}{1.583014in}}%
\pgfpathlineto{\pgfqpoint{1.418521in}{1.583014in}}%
\pgfpathlineto{\pgfqpoint{1.416170in}{1.585827in}}%
\pgfpathlineto{\pgfqpoint{1.415387in}{1.586765in}}%
\pgfpathlineto{\pgfqpoint{1.413036in}{1.589579in}}%
\pgfpathlineto{\pgfqpoint{1.412252in}{1.590517in}}%
\pgfpathlineto{\pgfqpoint{1.409901in}{1.593330in}}%
\pgfpathlineto{\pgfqpoint{1.409117in}{1.594268in}}%
\pgfpathlineto{\pgfqpoint{1.406766in}{1.597082in}}%
\pgfpathlineto{\pgfqpoint{1.405982in}{1.598020in}}%
\pgfpathlineto{\pgfqpoint{1.402848in}{1.598020in}}%
\pgfpathlineto{\pgfqpoint{1.400497in}{1.600833in}}%
\pgfpathlineto{\pgfqpoint{1.399713in}{1.601771in}}%
\pgfpathlineto{\pgfqpoint{1.397362in}{1.604585in}}%
\pgfpathlineto{\pgfqpoint{1.396578in}{1.605523in}}%
\pgfpathlineto{\pgfqpoint{1.394227in}{1.608336in}}%
\pgfpathlineto{\pgfqpoint{1.393444in}{1.609274in}}%
\pgfpathlineto{\pgfqpoint{1.390309in}{1.609274in}}%
\pgfpathlineto{\pgfqpoint{1.387958in}{1.612088in}}%
\pgfpathlineto{\pgfqpoint{1.387174in}{1.613026in}}%
\pgfpathlineto{\pgfqpoint{1.384823in}{1.615840in}}%
\pgfpathlineto{\pgfqpoint{1.384039in}{1.616777in}}%
\pgfpathlineto{\pgfqpoint{1.381688in}{1.619591in}}%
\pgfpathlineto{\pgfqpoint{1.380905in}{1.620529in}}%
\pgfpathlineto{\pgfqpoint{1.377770in}{1.620529in}}%
\pgfpathlineto{\pgfqpoint{1.375419in}{1.623343in}}%
\pgfpathlineto{\pgfqpoint{1.374635in}{1.624280in}}%
\pgfpathlineto{\pgfqpoint{1.372284in}{1.627094in}}%
\pgfpathlineto{\pgfqpoint{1.371500in}{1.628032in}}%
\pgfpathlineto{\pgfqpoint{1.369149in}{1.630846in}}%
\pgfpathlineto{\pgfqpoint{1.368366in}{1.631784in}}%
\pgfpathlineto{\pgfqpoint{1.365231in}{1.631784in}}%
\pgfpathlineto{\pgfqpoint{1.362880in}{1.634597in}}%
\pgfpathlineto{\pgfqpoint{1.362096in}{1.635535in}}%
\pgfpathlineto{\pgfqpoint{1.359745in}{1.638349in}}%
\pgfpathlineto{\pgfqpoint{1.358961in}{1.639287in}}%
\pgfpathlineto{\pgfqpoint{1.356610in}{1.642100in}}%
\pgfpathlineto{\pgfqpoint{1.355827in}{1.643038in}}%
\pgfpathlineto{\pgfqpoint{1.352692in}{1.643038in}}%
\pgfpathlineto{\pgfqpoint{1.350341in}{1.645852in}}%
\pgfpathlineto{\pgfqpoint{1.349557in}{1.646790in}}%
\pgfpathlineto{\pgfqpoint{1.347206in}{1.649603in}}%
\pgfpathlineto{\pgfqpoint{1.346422in}{1.650541in}}%
\pgfpathlineto{\pgfqpoint{1.344071in}{1.653355in}}%
\pgfpathlineto{\pgfqpoint{1.343288in}{1.654293in}}%
\pgfpathlineto{\pgfqpoint{1.340153in}{1.654293in}}%
\pgfpathlineto{\pgfqpoint{1.337802in}{1.657106in}}%
\pgfpathlineto{\pgfqpoint{1.337018in}{1.658044in}}%
\pgfpathlineto{\pgfqpoint{1.334667in}{1.660858in}}%
\pgfpathlineto{\pgfqpoint{1.333883in}{1.661796in}}%
\pgfpathlineto{\pgfqpoint{1.331532in}{1.664609in}}%
\pgfpathlineto{\pgfqpoint{1.330749in}{1.665547in}}%
\pgfpathlineto{\pgfqpoint{1.328398in}{1.668361in}}%
\pgfpathlineto{\pgfqpoint{1.327614in}{1.669299in}}%
\pgfpathlineto{\pgfqpoint{1.324479in}{1.669299in}}%
\pgfpathlineto{\pgfqpoint{1.322128in}{1.672113in}}%
\pgfpathlineto{\pgfqpoint{1.321344in}{1.673050in}}%
\pgfpathlineto{\pgfqpoint{1.318993in}{1.675864in}}%
\pgfpathlineto{\pgfqpoint{1.318210in}{1.676802in}}%
\pgfpathlineto{\pgfqpoint{1.315859in}{1.679616in}}%
\pgfpathlineto{\pgfqpoint{1.315075in}{1.680553in}}%
\pgfpathlineto{\pgfqpoint{1.311940in}{1.680553in}}%
\pgfpathlineto{\pgfqpoint{1.309589in}{1.683367in}}%
\pgfpathlineto{\pgfqpoint{1.308805in}{1.684305in}}%
\pgfpathlineto{\pgfqpoint{1.306454in}{1.687119in}}%
\pgfpathlineto{\pgfqpoint{1.305671in}{1.688057in}}%
\pgfpathlineto{\pgfqpoint{1.303320in}{1.690870in}}%
\pgfpathlineto{\pgfqpoint{1.302536in}{1.691808in}}%
\pgfpathlineto{\pgfqpoint{1.299401in}{1.691808in}}%
\pgfpathlineto{\pgfqpoint{1.297050in}{1.694622in}}%
\pgfpathlineto{\pgfqpoint{1.296266in}{1.695560in}}%
\pgfpathlineto{\pgfqpoint{1.293915in}{1.698373in}}%
\pgfpathlineto{\pgfqpoint{1.293132in}{1.699311in}}%
\pgfpathlineto{\pgfqpoint{1.290781in}{1.702125in}}%
\pgfpathlineto{\pgfqpoint{1.290781in}{1.705876in}}%
\pgfpathlineto{\pgfqpoint{1.293132in}{1.708690in}}%
\pgfpathlineto{\pgfqpoint{1.293915in}{1.709628in}}%
\pgfpathlineto{\pgfqpoint{1.293915in}{1.713379in}}%
\pgfpathlineto{\pgfqpoint{1.293915in}{1.717131in}}%
\pgfpathlineto{\pgfqpoint{1.293915in}{1.720882in}}%
\pgfpathlineto{\pgfqpoint{1.296266in}{1.723696in}}%
\pgfpathlineto{\pgfqpoint{1.297050in}{1.724634in}}%
\pgfpathlineto{\pgfqpoint{1.297050in}{1.728386in}}%
\pgfpathlineto{\pgfqpoint{1.297050in}{1.732137in}}%
\pgfpathlineto{\pgfqpoint{1.299401in}{1.734951in}}%
\pgfpathlineto{\pgfqpoint{1.300185in}{1.735889in}}%
\pgfpathlineto{\pgfqpoint{1.300185in}{1.739640in}}%
\pgfpathlineto{\pgfqpoint{1.300185in}{1.743392in}}%
\pgfpathlineto{\pgfqpoint{1.300185in}{1.747143in}}%
\pgfpathlineto{\pgfqpoint{1.302536in}{1.749957in}}%
\pgfpathlineto{\pgfqpoint{1.303320in}{1.750895in}}%
\pgfpathlineto{\pgfqpoint{1.303320in}{1.754646in}}%
\pgfpathlineto{\pgfqpoint{1.303320in}{1.758398in}}%
\pgfpathlineto{\pgfqpoint{1.305671in}{1.761211in}}%
\pgfpathlineto{\pgfqpoint{1.306454in}{1.762149in}}%
\pgfpathlineto{\pgfqpoint{1.306454in}{1.765901in}}%
\pgfpathlineto{\pgfqpoint{1.306454in}{1.769652in}}%
\pgfpathlineto{\pgfqpoint{1.306454in}{1.773404in}}%
\pgfpathlineto{\pgfqpoint{1.308805in}{1.776218in}}%
\pgfpathlineto{\pgfqpoint{1.309589in}{1.777155in}}%
\pgfpathlineto{\pgfqpoint{1.309589in}{1.780907in}}%
\pgfpathlineto{\pgfqpoint{1.309589in}{1.784659in}}%
\pgfpathlineto{\pgfqpoint{1.311940in}{1.787472in}}%
\pgfpathlineto{\pgfqpoint{1.312724in}{1.788410in}}%
\pgfpathlineto{\pgfqpoint{1.312724in}{1.792162in}}%
\pgfpathlineto{\pgfqpoint{1.312724in}{1.795913in}}%
\pgfpathlineto{\pgfqpoint{1.312724in}{1.799665in}}%
\pgfpathlineto{\pgfqpoint{1.315075in}{1.802478in}}%
\pgfpathlineto{\pgfqpoint{1.315859in}{1.803416in}}%
\pgfpathlineto{\pgfqpoint{1.315859in}{1.807168in}}%
\pgfpathlineto{\pgfqpoint{1.315859in}{1.810919in}}%
\pgfpathlineto{\pgfqpoint{1.318210in}{1.813733in}}%
\pgfpathlineto{\pgfqpoint{1.318993in}{1.814671in}}%
\pgfpathlineto{\pgfqpoint{1.318993in}{1.818422in}}%
\pgfpathlineto{\pgfqpoint{1.318993in}{1.822174in}}%
\pgfpathlineto{\pgfqpoint{1.321344in}{1.824988in}}%
\pgfpathlineto{\pgfqpoint{1.322128in}{1.825925in}}%
\pgfpathlineto{\pgfqpoint{1.322128in}{1.829677in}}%
\pgfpathlineto{\pgfqpoint{1.322128in}{1.833429in}}%
\pgfpathlineto{\pgfqpoint{1.322128in}{1.837180in}}%
\pgfpathlineto{\pgfqpoint{1.324479in}{1.839994in}}%
\pgfpathlineto{\pgfqpoint{1.325263in}{1.840932in}}%
\pgfpathlineto{\pgfqpoint{1.325263in}{1.844683in}}%
\pgfpathlineto{\pgfqpoint{1.325263in}{1.848435in}}%
\pgfpathlineto{\pgfqpoint{1.327614in}{1.851248in}}%
\pgfpathlineto{\pgfqpoint{1.328398in}{1.852186in}}%
\pgfpathlineto{\pgfqpoint{1.328398in}{1.855938in}}%
\pgfpathlineto{\pgfqpoint{1.328398in}{1.859689in}}%
\pgfpathlineto{\pgfqpoint{1.328398in}{1.863441in}}%
\pgfpathlineto{\pgfqpoint{1.330749in}{1.866254in}}%
\pgfpathlineto{\pgfqpoint{1.331532in}{1.867192in}}%
\pgfpathlineto{\pgfqpoint{1.331532in}{1.870944in}}%
\pgfpathlineto{\pgfqpoint{1.331532in}{1.874695in}}%
\pgfpathlineto{\pgfqpoint{1.333883in}{1.877509in}}%
\pgfpathlineto{\pgfqpoint{1.334667in}{1.878447in}}%
\pgfpathlineto{\pgfqpoint{1.334667in}{1.882198in}}%
\pgfpathlineto{\pgfqpoint{1.334667in}{1.885950in}}%
\pgfpathlineto{\pgfqpoint{1.334667in}{1.889702in}}%
\pgfpathlineto{\pgfqpoint{1.337018in}{1.892515in}}%
\pgfpathlineto{\pgfqpoint{1.337802in}{1.893453in}}%
\pgfpathlineto{\pgfqpoint{1.337802in}{1.897205in}}%
\pgfpathlineto{\pgfqpoint{1.337802in}{1.900956in}}%
\pgfpathlineto{\pgfqpoint{1.340153in}{1.903770in}}%
\pgfpathlineto{\pgfqpoint{1.340937in}{1.904708in}}%
\pgfpathlineto{\pgfqpoint{1.340937in}{1.908459in}}%
\pgfpathlineto{\pgfqpoint{1.340937in}{1.912211in}}%
\pgfpathlineto{\pgfqpoint{1.340937in}{1.915962in}}%
\pgfpathlineto{\pgfqpoint{1.343288in}{1.918776in}}%
\pgfpathlineto{\pgfqpoint{1.344071in}{1.919714in}}%
\pgfpathlineto{\pgfqpoint{1.344071in}{1.923465in}}%
\pgfpathlineto{\pgfqpoint{1.344071in}{1.927217in}}%
\pgfpathlineto{\pgfqpoint{1.346422in}{1.930031in}}%
\pgfpathlineto{\pgfqpoint{1.347206in}{1.930968in}}%
\pgfpathlineto{\pgfqpoint{1.347206in}{1.934720in}}%
\pgfpathlineto{\pgfqpoint{1.347206in}{1.938471in}}%
\pgfpathlineto{\pgfqpoint{1.347206in}{1.942223in}}%
\pgfpathlineto{\pgfqpoint{1.349557in}{1.945037in}}%
\pgfpathlineto{\pgfqpoint{1.350341in}{1.945975in}}%
\pgfpathlineto{\pgfqpoint{1.350341in}{1.949726in}}%
\pgfpathlineto{\pgfqpoint{1.350341in}{1.953478in}}%
\pgfpathlineto{\pgfqpoint{1.352692in}{1.956291in}}%
\pgfpathlineto{\pgfqpoint{1.353476in}{1.957229in}}%
\pgfpathlineto{\pgfqpoint{1.353476in}{1.960981in}}%
\pgfpathlineto{\pgfqpoint{1.353476in}{1.964732in}}%
\pgfpathlineto{\pgfqpoint{1.353476in}{1.968484in}}%
\pgfpathlineto{\pgfqpoint{1.355827in}{1.971297in}}%
\pgfpathlineto{\pgfqpoint{1.356610in}{1.972235in}}%
\pgfpathlineto{\pgfqpoint{1.356610in}{1.975987in}}%
\pgfpathlineto{\pgfqpoint{1.356610in}{1.979738in}}%
\pgfpathlineto{\pgfqpoint{1.358961in}{1.982552in}}%
\pgfpathlineto{\pgfqpoint{1.359745in}{1.983490in}}%
\pgfpathlineto{\pgfqpoint{1.359745in}{1.987241in}}%
\pgfpathlineto{\pgfqpoint{1.359745in}{1.990993in}}%
\pgfpathlineto{\pgfqpoint{1.359745in}{1.994745in}}%
\pgfpathlineto{\pgfqpoint{1.362096in}{1.997558in}}%
\pgfpathlineto{\pgfqpoint{1.362880in}{1.998496in}}%
\pgfpathlineto{\pgfqpoint{1.362880in}{2.002248in}}%
\pgfpathlineto{\pgfqpoint{1.362880in}{2.005999in}}%
\pgfpathlineto{\pgfqpoint{1.365231in}{2.008813in}}%
\pgfpathlineto{\pgfqpoint{1.366015in}{2.009751in}}%
\pgfpathlineto{\pgfqpoint{1.366015in}{2.013502in}}%
\pgfpathlineto{\pgfqpoint{1.366015in}{2.017254in}}%
\pgfpathlineto{\pgfqpoint{1.366015in}{2.021005in}}%
\pgfpathlineto{\pgfqpoint{1.368366in}{2.023819in}}%
\pgfpathlineto{\pgfqpoint{1.369149in}{2.024757in}}%
\pgfpathlineto{\pgfqpoint{1.369149in}{2.028508in}}%
\pgfpathlineto{\pgfqpoint{1.369149in}{2.032260in}}%
\pgfpathlineto{\pgfqpoint{1.371500in}{2.035073in}}%
\pgfpathlineto{\pgfqpoint{1.372284in}{2.036011in}}%
\pgfpathlineto{\pgfqpoint{1.372284in}{2.039763in}}%
\pgfpathlineto{\pgfqpoint{1.372284in}{2.043514in}}%
\pgfpathlineto{\pgfqpoint{1.374635in}{2.046328in}}%
\pgfpathlineto{\pgfqpoint{1.375419in}{2.047266in}}%
\pgfpathlineto{\pgfqpoint{1.375419in}{2.051018in}}%
\pgfpathlineto{\pgfqpoint{1.375419in}{2.054769in}}%
\pgfpathlineto{\pgfqpoint{1.375419in}{2.058521in}}%
\pgfpathlineto{\pgfqpoint{1.377770in}{2.061334in}}%
\pgfpathlineto{\pgfqpoint{1.378553in}{2.062272in}}%
\pgfpathlineto{\pgfqpoint{1.378553in}{2.066024in}}%
\pgfpathlineto{\pgfqpoint{1.378553in}{2.069775in}}%
\pgfpathlineto{\pgfqpoint{1.380905in}{2.072589in}}%
\pgfpathlineto{\pgfqpoint{1.381688in}{2.073527in}}%
\pgfpathlineto{\pgfqpoint{1.381688in}{2.077278in}}%
\pgfpathlineto{\pgfqpoint{1.381688in}{2.081030in}}%
\pgfpathlineto{\pgfqpoint{1.381688in}{2.084781in}}%
\pgfpathlineto{\pgfqpoint{1.384039in}{2.087595in}}%
\pgfpathlineto{\pgfqpoint{1.384823in}{2.088533in}}%
\pgfpathlineto{\pgfqpoint{1.384823in}{2.092284in}}%
\pgfpathlineto{\pgfqpoint{1.384823in}{2.096036in}}%
\pgfpathlineto{\pgfqpoint{1.387174in}{2.098850in}}%
\pgfpathlineto{\pgfqpoint{1.387958in}{2.099787in}}%
\pgfpathlineto{\pgfqpoint{1.387958in}{2.103539in}}%
\pgfpathlineto{\pgfqpoint{1.387958in}{2.107291in}}%
\pgfpathlineto{\pgfqpoint{1.387958in}{2.111042in}}%
\pgfpathlineto{\pgfqpoint{1.390309in}{2.113856in}}%
\pgfpathlineto{\pgfqpoint{1.391092in}{2.114794in}}%
\pgfpathlineto{\pgfqpoint{1.391092in}{2.118545in}}%
\pgfpathlineto{\pgfqpoint{1.391092in}{2.122297in}}%
\pgfpathlineto{\pgfqpoint{1.393444in}{2.125110in}}%
\pgfpathlineto{\pgfqpoint{1.394227in}{2.126048in}}%
\pgfpathlineto{\pgfqpoint{1.394227in}{2.129800in}}%
\pgfpathlineto{\pgfqpoint{1.394227in}{2.133551in}}%
\pgfpathlineto{\pgfqpoint{1.394227in}{2.137303in}}%
\pgfpathlineto{\pgfqpoint{1.396578in}{2.140116in}}%
\pgfpathlineto{\pgfqpoint{1.397362in}{2.141054in}}%
\pgfpathlineto{\pgfqpoint{1.397362in}{2.144806in}}%
\pgfpathlineto{\pgfqpoint{1.397362in}{2.148557in}}%
\pgfpathlineto{\pgfqpoint{1.399713in}{2.151371in}}%
\pgfpathlineto{\pgfqpoint{1.400497in}{2.152309in}}%
\pgfpathlineto{\pgfqpoint{1.400497in}{2.156060in}}%
\pgfpathlineto{\pgfqpoint{1.400497in}{2.159812in}}%
\pgfpathlineto{\pgfqpoint{1.400497in}{2.163564in}}%
\pgfpathlineto{\pgfqpoint{1.402848in}{2.166377in}}%
\pgfpathlineto{\pgfqpoint{1.403631in}{2.167315in}}%
\pgfpathlineto{\pgfqpoint{1.403631in}{2.171067in}}%
\pgfpathlineto{\pgfqpoint{1.403631in}{2.174818in}}%
\pgfpathlineto{\pgfqpoint{1.405982in}{2.177632in}}%
\pgfpathlineto{\pgfqpoint{1.406766in}{2.178570in}}%
\pgfpathlineto{\pgfqpoint{1.406766in}{2.182321in}}%
\pgfpathlineto{\pgfqpoint{1.406766in}{2.186073in}}%
\pgfpathlineto{\pgfqpoint{1.406766in}{2.189824in}}%
\pgfpathlineto{\pgfqpoint{1.409117in}{2.192638in}}%
\pgfpathlineto{\pgfqpoint{1.409901in}{2.193576in}}%
\pgfpathlineto{\pgfqpoint{1.409901in}{2.197327in}}%
\pgfpathlineto{\pgfqpoint{1.409901in}{2.201079in}}%
\pgfpathlineto{\pgfqpoint{1.412252in}{2.203893in}}%
\pgfpathlineto{\pgfqpoint{1.413036in}{2.204830in}}%
\pgfpathlineto{\pgfqpoint{1.413036in}{2.208582in}}%
\pgfpathlineto{\pgfqpoint{1.413036in}{2.212334in}}%
\pgfpathlineto{\pgfqpoint{1.413036in}{2.216085in}}%
\pgfpathlineto{\pgfqpoint{1.415387in}{2.218899in}}%
\pgfpathlineto{\pgfqpoint{1.416170in}{2.219837in}}%
\pgfpathlineto{\pgfqpoint{1.416170in}{2.223588in}}%
\pgfpathlineto{\pgfqpoint{1.416170in}{2.227340in}}%
\pgfpathlineto{\pgfqpoint{1.418521in}{2.230153in}}%
\pgfpathlineto{\pgfqpoint{1.419305in}{2.231091in}}%
\pgfpathlineto{\pgfqpoint{1.419305in}{2.234843in}}%
\pgfpathlineto{\pgfqpoint{1.419305in}{2.238594in}}%
\pgfpathlineto{\pgfqpoint{1.419305in}{2.242346in}}%
\pgfpathlineto{\pgfqpoint{1.421656in}{2.245159in}}%
\pgfpathlineto{\pgfqpoint{1.422440in}{2.246097in}}%
\pgfpathlineto{\pgfqpoint{1.422440in}{2.249849in}}%
\pgfpathlineto{\pgfqpoint{1.422440in}{2.253600in}}%
\pgfpathlineto{\pgfqpoint{1.424791in}{2.256414in}}%
\pgfpathlineto{\pgfqpoint{1.425575in}{2.257352in}}%
\pgfpathlineto{\pgfqpoint{1.425575in}{2.261103in}}%
\pgfpathlineto{\pgfqpoint{1.425575in}{2.264855in}}%
\pgfpathlineto{\pgfqpoint{1.427926in}{2.267669in}}%
\pgfpathlineto{\pgfqpoint{1.428709in}{2.268607in}}%
\pgfpathlineto{\pgfqpoint{1.428709in}{2.272358in}}%
\pgfpathlineto{\pgfqpoint{1.428709in}{2.276110in}}%
\pgfpathlineto{\pgfqpoint{1.428709in}{2.279861in}}%
\pgfpathlineto{\pgfqpoint{1.431060in}{2.282675in}}%
\pgfpathlineto{\pgfqpoint{1.431844in}{2.283613in}}%
\pgfpathlineto{\pgfqpoint{1.431844in}{2.287364in}}%
\pgfpathlineto{\pgfqpoint{1.431844in}{2.291116in}}%
\pgfpathlineto{\pgfqpoint{1.434195in}{2.293929in}}%
\pgfpathlineto{\pgfqpoint{1.434979in}{2.294867in}}%
\pgfpathlineto{\pgfqpoint{1.434979in}{2.298619in}}%
\pgfpathlineto{\pgfqpoint{1.434979in}{2.302370in}}%
\pgfpathlineto{\pgfqpoint{1.434979in}{2.306122in}}%
\pgfpathlineto{\pgfqpoint{1.437330in}{2.308936in}}%
\pgfpathlineto{\pgfqpoint{1.438114in}{2.309873in}}%
\pgfpathlineto{\pgfqpoint{1.438114in}{2.313625in}}%
\pgfpathlineto{\pgfqpoint{1.438114in}{2.317376in}}%
\pgfpathlineto{\pgfqpoint{1.440465in}{2.320190in}}%
\pgfpathlineto{\pgfqpoint{1.441248in}{2.321128in}}%
\pgfpathlineto{\pgfqpoint{1.441248in}{2.324880in}}%
\pgfpathlineto{\pgfqpoint{1.441248in}{2.328631in}}%
\pgfpathlineto{\pgfqpoint{1.441248in}{2.332383in}}%
\pgfpathlineto{\pgfqpoint{1.443599in}{2.335196in}}%
\pgfpathlineto{\pgfqpoint{1.444383in}{2.336134in}}%
\pgfpathlineto{\pgfqpoint{1.444383in}{2.339886in}}%
\pgfpathlineto{\pgfqpoint{1.444383in}{2.343637in}}%
\pgfpathlineto{\pgfqpoint{1.446734in}{2.346451in}}%
\pgfpathlineto{\pgfqpoint{1.447518in}{2.347389in}}%
\pgfpathlineto{\pgfqpoint{1.447518in}{2.351140in}}%
\pgfpathlineto{\pgfqpoint{1.447518in}{2.354892in}}%
\pgfpathlineto{\pgfqpoint{1.447518in}{2.358643in}}%
\pgfpathlineto{\pgfqpoint{1.449869in}{2.361457in}}%
\pgfpathlineto{\pgfqpoint{1.450653in}{2.362395in}}%
\pgfpathlineto{\pgfqpoint{1.450653in}{2.366146in}}%
\pgfpathlineto{\pgfqpoint{1.450653in}{2.369898in}}%
\pgfpathlineto{\pgfqpoint{1.453004in}{2.372712in}}%
\pgfpathlineto{\pgfqpoint{1.453787in}{2.373649in}}%
\pgfpathlineto{\pgfqpoint{1.453787in}{2.377401in}}%
\pgfpathlineto{\pgfqpoint{1.453787in}{2.381153in}}%
\pgfpathlineto{\pgfqpoint{1.453787in}{2.384904in}}%
\pgfpathlineto{\pgfqpoint{1.456138in}{2.387718in}}%
\pgfpathlineto{\pgfqpoint{1.456922in}{2.388656in}}%
\pgfpathlineto{\pgfqpoint{1.456922in}{2.392407in}}%
\pgfpathlineto{\pgfqpoint{1.456922in}{2.396159in}}%
\pgfpathlineto{\pgfqpoint{1.459273in}{2.398972in}}%
\pgfpathlineto{\pgfqpoint{1.460057in}{2.399910in}}%
\pgfpathlineto{\pgfqpoint{1.460057in}{2.403662in}}%
\pgfpathlineto{\pgfqpoint{1.462408in}{2.406475in}}%
\pgfpathlineto{\pgfqpoint{1.465543in}{2.406475in}}%
\pgfpathlineto{\pgfqpoint{1.466326in}{2.407413in}}%
\pgfpathlineto{\pgfqpoint{1.468677in}{2.410227in}}%
\pgfpathlineto{\pgfqpoint{1.471812in}{2.410227in}}%
\pgfpathlineto{\pgfqpoint{1.474947in}{2.410227in}}%
\pgfpathlineto{\pgfqpoint{1.478082in}{2.410227in}}%
\pgfpathlineto{\pgfqpoint{1.478865in}{2.411165in}}%
\pgfpathlineto{\pgfqpoint{1.481216in}{2.413978in}}%
\pgfpathlineto{\pgfqpoint{1.484351in}{2.413978in}}%
\pgfpathlineto{\pgfqpoint{1.487486in}{2.413978in}}%
\pgfpathlineto{\pgfqpoint{1.490621in}{2.413978in}}%
\pgfpathlineto{\pgfqpoint{1.491404in}{2.414916in}}%
\pgfpathlineto{\pgfqpoint{1.493755in}{2.417730in}}%
\pgfpathlineto{\pgfqpoint{1.496890in}{2.417730in}}%
\pgfpathlineto{\pgfqpoint{1.500025in}{2.417730in}}%
\pgfpathlineto{\pgfqpoint{1.503159in}{2.417730in}}%
\pgfpathlineto{\pgfqpoint{1.503943in}{2.418668in}}%
\pgfpathlineto{\pgfqpoint{1.506294in}{2.421482in}}%
\pgfpathlineto{\pgfqpoint{1.509429in}{2.421482in}}%
\pgfpathlineto{\pgfqpoint{1.512564in}{2.421482in}}%
\pgfpathlineto{\pgfqpoint{1.515698in}{2.421482in}}%
\pgfpathlineto{\pgfqpoint{1.516482in}{2.422419in}}%
\pgfpathlineto{\pgfqpoint{1.518833in}{2.425233in}}%
\pgfpathlineto{\pgfqpoint{1.521968in}{2.425233in}}%
\pgfpathlineto{\pgfqpoint{1.525103in}{2.425233in}}%
\pgfpathlineto{\pgfqpoint{1.528237in}{2.425233in}}%
\pgfpathlineto{\pgfqpoint{1.531372in}{2.425233in}}%
\pgfpathlineto{\pgfqpoint{1.532156in}{2.426171in}}%
\pgfpathlineto{\pgfqpoint{1.534507in}{2.428985in}}%
\pgfpathlineto{\pgfqpoint{1.537642in}{2.428985in}}%
\pgfpathlineto{\pgfqpoint{1.540776in}{2.428985in}}%
\pgfpathlineto{\pgfqpoint{1.543911in}{2.428985in}}%
\pgfpathlineto{\pgfqpoint{1.544695in}{2.429923in}}%
\pgfpathlineto{\pgfqpoint{1.547046in}{2.432736in}}%
\pgfpathlineto{\pgfqpoint{1.550181in}{2.432736in}}%
\pgfpathlineto{\pgfqpoint{1.553315in}{2.432736in}}%
\pgfpathlineto{\pgfqpoint{1.556450in}{2.432736in}}%
\pgfpathlineto{\pgfqpoint{1.557234in}{2.433674in}}%
\pgfpathlineto{\pgfqpoint{1.559585in}{2.436488in}}%
\pgfpathlineto{\pgfqpoint{1.562720in}{2.436488in}}%
\pgfpathlineto{\pgfqpoint{1.565854in}{2.436488in}}%
\pgfpathlineto{\pgfqpoint{1.568989in}{2.436488in}}%
\pgfpathlineto{\pgfqpoint{1.569773in}{2.437426in}}%
\pgfpathlineto{\pgfqpoint{1.572124in}{2.440239in}}%
\pgfpathlineto{\pgfqpoint{1.575259in}{2.440239in}}%
\pgfpathlineto{\pgfqpoint{1.578393in}{2.440239in}}%
\pgfpathlineto{\pgfqpoint{1.581528in}{2.440239in}}%
\pgfpathlineto{\pgfqpoint{1.584663in}{2.440239in}}%
\pgfpathlineto{\pgfqpoint{1.585446in}{2.441177in}}%
\pgfpathlineto{\pgfqpoint{1.587798in}{2.443991in}}%
\pgfpathlineto{\pgfqpoint{1.590932in}{2.443991in}}%
\pgfpathlineto{\pgfqpoint{1.594067in}{2.443991in}}%
\pgfpathlineto{\pgfqpoint{1.597202in}{2.443991in}}%
\pgfpathlineto{\pgfqpoint{1.597985in}{2.444929in}}%
\pgfpathlineto{\pgfqpoint{1.600337in}{2.447742in}}%
\pgfpathlineto{\pgfqpoint{1.603471in}{2.447742in}}%
\pgfpathlineto{\pgfqpoint{1.606606in}{2.447742in}}%
\pgfpathlineto{\pgfqpoint{1.609741in}{2.447742in}}%
\pgfpathlineto{\pgfqpoint{1.610524in}{2.448680in}}%
\pgfpathlineto{\pgfqpoint{1.612875in}{2.451494in}}%
\pgfpathlineto{\pgfqpoint{1.616010in}{2.451494in}}%
\pgfpathlineto{\pgfqpoint{1.619145in}{2.451494in}}%
\pgfpathlineto{\pgfqpoint{1.622280in}{2.451494in}}%
\pgfpathlineto{\pgfqpoint{1.623063in}{2.452432in}}%
\pgfpathlineto{\pgfqpoint{1.625414in}{2.455245in}}%
\pgfpathlineto{\pgfqpoint{1.628549in}{2.455245in}}%
\pgfpathlineto{\pgfqpoint{1.631684in}{2.455245in}}%
\pgfpathlineto{\pgfqpoint{1.634819in}{2.455245in}}%
\pgfpathlineto{\pgfqpoint{1.637953in}{2.455245in}}%
\pgfpathlineto{\pgfqpoint{1.638737in}{2.456183in}}%
\pgfpathlineto{\pgfqpoint{1.641088in}{2.458997in}}%
\pgfpathlineto{\pgfqpoint{1.644223in}{2.458997in}}%
\pgfpathlineto{\pgfqpoint{1.647358in}{2.458997in}}%
\pgfpathlineto{\pgfqpoint{1.650492in}{2.458997in}}%
\pgfpathlineto{\pgfqpoint{1.651276in}{2.459935in}}%
\pgfpathlineto{\pgfqpoint{1.653627in}{2.462748in}}%
\pgfpathlineto{\pgfqpoint{1.656762in}{2.462748in}}%
\pgfpathlineto{\pgfqpoint{1.659897in}{2.462748in}}%
\pgfpathlineto{\pgfqpoint{1.663031in}{2.462748in}}%
\pgfpathlineto{\pgfqpoint{1.663815in}{2.463686in}}%
\pgfpathlineto{\pgfqpoint{1.666166in}{2.466500in}}%
\pgfpathlineto{\pgfqpoint{1.669301in}{2.466500in}}%
\pgfpathlineto{\pgfqpoint{1.672436in}{2.466500in}}%
\pgfpathlineto{\pgfqpoint{1.675570in}{2.466500in}}%
\pgfpathlineto{\pgfqpoint{1.676354in}{2.467438in}}%
\pgfpathlineto{\pgfqpoint{1.678705in}{2.470252in}}%
\pgfpathlineto{\pgfqpoint{1.681840in}{2.470252in}}%
\pgfpathlineto{\pgfqpoint{1.684975in}{2.470252in}}%
\pgfpathlineto{\pgfqpoint{1.688109in}{2.470252in}}%
\pgfpathlineto{\pgfqpoint{1.688893in}{2.471189in}}%
\pgfpathlineto{\pgfqpoint{1.691244in}{2.474003in}}%
\pgfpathlineto{\pgfqpoint{1.694379in}{2.474003in}}%
\pgfpathlineto{\pgfqpoint{1.697514in}{2.474003in}}%
\pgfpathlineto{\pgfqpoint{1.700648in}{2.474003in}}%
\pgfpathlineto{\pgfqpoint{1.703783in}{2.474003in}}%
\pgfpathlineto{\pgfqpoint{1.704567in}{2.474941in}}%
\pgfpathlineto{\pgfqpoint{1.706918in}{2.477755in}}%
\pgfpathlineto{\pgfqpoint{1.710052in}{2.477755in}}%
\pgfpathlineto{\pgfqpoint{1.713187in}{2.477755in}}%
\pgfpathlineto{\pgfqpoint{1.716322in}{2.477755in}}%
\pgfpathlineto{\pgfqpoint{1.717106in}{2.478692in}}%
\pgfpathlineto{\pgfqpoint{1.719457in}{2.481506in}}%
\pgfpathlineto{\pgfqpoint{1.722591in}{2.481506in}}%
\pgfpathlineto{\pgfqpoint{1.725726in}{2.481506in}}%
\pgfpathlineto{\pgfqpoint{1.728861in}{2.481506in}}%
\pgfpathlineto{\pgfqpoint{1.729645in}{2.482444in}}%
\pgfpathlineto{\pgfqpoint{1.731996in}{2.485258in}}%
\pgfpathlineto{\pgfqpoint{1.735130in}{2.485258in}}%
\pgfpathlineto{\pgfqpoint{1.738265in}{2.485258in}}%
\pgfpathlineto{\pgfqpoint{1.741400in}{2.485258in}}%
\pgfpathlineto{\pgfqpoint{1.742184in}{2.486196in}}%
\pgfpathlineto{\pgfqpoint{1.744535in}{2.489009in}}%
\pgfpathlineto{\pgfqpoint{1.747669in}{2.489009in}}%
\pgfpathlineto{\pgfqpoint{1.750804in}{2.489009in}}%
\pgfpathlineto{\pgfqpoint{1.753939in}{2.489009in}}%
\pgfpathlineto{\pgfqpoint{1.757074in}{2.489009in}}%
\pgfpathlineto{\pgfqpoint{1.757857in}{2.489947in}}%
\pgfpathlineto{\pgfqpoint{1.760208in}{2.492761in}}%
\pgfpathlineto{\pgfqpoint{1.763343in}{2.492761in}}%
\pgfpathlineto{\pgfqpoint{1.766478in}{2.492761in}}%
\pgfpathlineto{\pgfqpoint{1.769613in}{2.492761in}}%
\pgfpathlineto{\pgfqpoint{1.770396in}{2.493699in}}%
\pgfpathlineto{\pgfqpoint{1.772747in}{2.496512in}}%
\pgfpathlineto{\pgfqpoint{1.775882in}{2.496512in}}%
\pgfpathlineto{\pgfqpoint{1.779017in}{2.496512in}}%
\pgfpathlineto{\pgfqpoint{1.782152in}{2.496512in}}%
\pgfpathlineto{\pgfqpoint{1.782935in}{2.497450in}}%
\pgfpathlineto{\pgfqpoint{1.785286in}{2.500264in}}%
\pgfpathlineto{\pgfqpoint{1.788421in}{2.500264in}}%
\pgfpathlineto{\pgfqpoint{1.791556in}{2.500264in}}%
\pgfpathlineto{\pgfqpoint{1.794691in}{2.500264in}}%
\pgfpathlineto{\pgfqpoint{1.795474in}{2.501202in}}%
\pgfpathlineto{\pgfqpoint{1.797825in}{2.504015in}}%
\pgfpathlineto{\pgfqpoint{1.800960in}{2.504015in}}%
\pgfpathlineto{\pgfqpoint{1.804095in}{2.504015in}}%
\pgfpathlineto{\pgfqpoint{1.807229in}{2.504015in}}%
\pgfpathlineto{\pgfqpoint{1.810364in}{2.504015in}}%
\pgfpathlineto{\pgfqpoint{1.811148in}{2.504953in}}%
\pgfpathlineto{\pgfqpoint{1.813499in}{2.507767in}}%
\pgfpathlineto{\pgfqpoint{1.816634in}{2.507767in}}%
\pgfpathlineto{\pgfqpoint{1.819768in}{2.507767in}}%
\pgfpathlineto{\pgfqpoint{1.822903in}{2.507767in}}%
\pgfpathlineto{\pgfqpoint{1.823687in}{2.508705in}}%
\pgfpathlineto{\pgfqpoint{1.826038in}{2.511518in}}%
\pgfpathlineto{\pgfqpoint{1.829173in}{2.511518in}}%
\pgfpathlineto{\pgfqpoint{1.832307in}{2.511518in}}%
\pgfpathlineto{\pgfqpoint{1.835442in}{2.511518in}}%
\pgfpathlineto{\pgfqpoint{1.836226in}{2.512456in}}%
\pgfpathlineto{\pgfqpoint{1.838577in}{2.515270in}}%
\pgfpathlineto{\pgfqpoint{1.841712in}{2.515270in}}%
\pgfpathlineto{\pgfqpoint{1.844846in}{2.515270in}}%
\pgfpathlineto{\pgfqpoint{1.847981in}{2.515270in}}%
\pgfpathlineto{\pgfqpoint{1.848765in}{2.516208in}}%
\pgfpathlineto{\pgfqpoint{1.851116in}{2.519021in}}%
\pgfpathlineto{\pgfqpoint{1.854251in}{2.519021in}}%
\pgfpathlineto{\pgfqpoint{1.857385in}{2.519021in}}%
\pgfpathlineto{\pgfqpoint{1.860520in}{2.519021in}}%
\pgfpathlineto{\pgfqpoint{1.861304in}{2.519959in}}%
\pgfpathlineto{\pgfqpoint{1.863655in}{2.522773in}}%
\pgfpathlineto{\pgfqpoint{1.866790in}{2.522773in}}%
\pgfpathlineto{\pgfqpoint{1.869924in}{2.522773in}}%
\pgfpathlineto{\pgfqpoint{1.873059in}{2.522773in}}%
\pgfpathlineto{\pgfqpoint{1.876194in}{2.522773in}}%
\pgfpathlineto{\pgfqpoint{1.876978in}{2.523711in}}%
\pgfpathlineto{\pgfqpoint{1.879329in}{2.526525in}}%
\pgfpathlineto{\pgfqpoint{1.882463in}{2.526525in}}%
\pgfpathlineto{\pgfqpoint{1.885598in}{2.526525in}}%
\pgfpathlineto{\pgfqpoint{1.888733in}{2.526525in}}%
\pgfpathlineto{\pgfqpoint{1.889516in}{2.527462in}}%
\pgfpathlineto{\pgfqpoint{1.891868in}{2.530276in}}%
\pgfpathlineto{\pgfqpoint{1.895002in}{2.530276in}}%
\pgfpathlineto{\pgfqpoint{1.898137in}{2.530276in}}%
\pgfpathlineto{\pgfqpoint{1.901272in}{2.530276in}}%
\pgfpathlineto{\pgfqpoint{1.902055in}{2.531214in}}%
\pgfpathlineto{\pgfqpoint{1.904407in}{2.534028in}}%
\pgfpathlineto{\pgfqpoint{1.907541in}{2.534028in}}%
\pgfpathlineto{\pgfqpoint{1.910676in}{2.534028in}}%
\pgfpathlineto{\pgfqpoint{1.913811in}{2.534028in}}%
\pgfpathlineto{\pgfqpoint{1.914594in}{2.534965in}}%
\pgfpathlineto{\pgfqpoint{1.916945in}{2.537779in}}%
\pgfpathlineto{\pgfqpoint{1.920080in}{2.537779in}}%
\pgfpathlineto{\pgfqpoint{1.923215in}{2.537779in}}%
\pgfpathlineto{\pgfqpoint{1.926350in}{2.537779in}}%
\pgfpathlineto{\pgfqpoint{1.929484in}{2.537779in}}%
\pgfpathlineto{\pgfqpoint{1.930268in}{2.538717in}}%
\pgfpathlineto{\pgfqpoint{1.932619in}{2.541531in}}%
\pgfpathlineto{\pgfqpoint{1.935754in}{2.541531in}}%
\pgfpathlineto{\pgfqpoint{1.938889in}{2.541531in}}%
\pgfpathlineto{\pgfqpoint{1.942023in}{2.541531in}}%
\pgfpathlineto{\pgfqpoint{1.942807in}{2.542469in}}%
\pgfpathlineto{\pgfqpoint{1.945158in}{2.545282in}}%
\pgfpathlineto{\pgfqpoint{1.948293in}{2.545282in}}%
\pgfpathlineto{\pgfqpoint{1.951428in}{2.545282in}}%
\pgfpathlineto{\pgfqpoint{1.954562in}{2.545282in}}%
\pgfpathlineto{\pgfqpoint{1.955346in}{2.546220in}}%
\pgfpathlineto{\pgfqpoint{1.957697in}{2.549034in}}%
\pgfpathlineto{\pgfqpoint{1.960832in}{2.549034in}}%
\pgfpathlineto{\pgfqpoint{1.963967in}{2.549034in}}%
\pgfpathlineto{\pgfqpoint{1.967101in}{2.549034in}}%
\pgfpathlineto{\pgfqpoint{1.967885in}{2.549972in}}%
\pgfpathlineto{\pgfqpoint{1.970236in}{2.552785in}}%
\pgfpathlineto{\pgfqpoint{1.973371in}{2.552785in}}%
\pgfpathlineto{\pgfqpoint{1.976506in}{2.552785in}}%
\pgfpathlineto{\pgfqpoint{1.979640in}{2.552785in}}%
\pgfpathlineto{\pgfqpoint{1.980424in}{2.553723in}}%
\pgfpathlineto{\pgfqpoint{1.982775in}{2.556537in}}%
\pgfpathlineto{\pgfqpoint{1.985910in}{2.556537in}}%
\pgfpathlineto{\pgfqpoint{1.989045in}{2.556537in}}%
\pgfpathlineto{\pgfqpoint{1.992179in}{2.556537in}}%
\pgfpathlineto{\pgfqpoint{1.995314in}{2.556537in}}%
\pgfpathlineto{\pgfqpoint{1.996098in}{2.557475in}}%
\pgfpathlineto{\pgfqpoint{1.998449in}{2.560288in}}%
\pgfpathlineto{\pgfqpoint{2.001584in}{2.560288in}}%
\pgfpathlineto{\pgfqpoint{2.004718in}{2.560288in}}%
\pgfpathlineto{\pgfqpoint{2.007853in}{2.560288in}}%
\pgfpathlineto{\pgfqpoint{2.008637in}{2.561226in}}%
\pgfpathlineto{\pgfqpoint{2.010988in}{2.564040in}}%
\pgfpathlineto{\pgfqpoint{2.014122in}{2.564040in}}%
\pgfpathlineto{\pgfqpoint{2.017257in}{2.564040in}}%
\pgfpathlineto{\pgfqpoint{2.020392in}{2.564040in}}%
\pgfpathlineto{\pgfqpoint{2.021176in}{2.564978in}}%
\pgfpathlineto{\pgfqpoint{2.023527in}{2.567791in}}%
\pgfpathlineto{\pgfqpoint{2.026661in}{2.567791in}}%
\pgfpathlineto{\pgfqpoint{2.029796in}{2.567791in}}%
\pgfpathlineto{\pgfqpoint{2.032931in}{2.567791in}}%
\pgfpathlineto{\pgfqpoint{2.033715in}{2.568729in}}%
\pgfpathlineto{\pgfqpoint{2.036066in}{2.571543in}}%
\pgfpathlineto{\pgfqpoint{2.039200in}{2.571543in}}%
\pgfpathlineto{\pgfqpoint{2.042335in}{2.571543in}}%
\pgfpathlineto{\pgfqpoint{2.045470in}{2.571543in}}%
\pgfpathlineto{\pgfqpoint{2.048605in}{2.571543in}}%
\pgfpathlineto{\pgfqpoint{2.049388in}{2.572481in}}%
\pgfpathlineto{\pgfqpoint{2.051739in}{2.575294in}}%
\pgfpathlineto{\pgfqpoint{2.054874in}{2.575294in}}%
\pgfpathlineto{\pgfqpoint{2.058009in}{2.575294in}}%
\pgfpathlineto{\pgfqpoint{2.061144in}{2.575294in}}%
\pgfpathlineto{\pgfqpoint{2.061927in}{2.576232in}}%
\pgfpathlineto{\pgfqpoint{2.064278in}{2.579046in}}%
\pgfpathlineto{\pgfqpoint{2.067413in}{2.579046in}}%
\pgfpathlineto{\pgfqpoint{2.070548in}{2.579046in}}%
\pgfpathlineto{\pgfqpoint{2.073683in}{2.579046in}}%
\pgfpathlineto{\pgfqpoint{2.074466in}{2.579984in}}%
\pgfpathlineto{\pgfqpoint{2.076817in}{2.582798in}}%
\pgfpathlineto{\pgfqpoint{2.079952in}{2.582798in}}%
\pgfpathlineto{\pgfqpoint{2.083087in}{2.582798in}}%
\pgfpathlineto{\pgfqpoint{2.086222in}{2.582798in}}%
\pgfpathlineto{\pgfqpoint{2.087005in}{2.583735in}}%
\pgfpathlineto{\pgfqpoint{2.089356in}{2.586549in}}%
\pgfpathlineto{\pgfqpoint{2.092491in}{2.586549in}}%
\pgfpathlineto{\pgfqpoint{2.095626in}{2.586549in}}%
\pgfpathlineto{\pgfqpoint{2.098761in}{2.586549in}}%
\pgfpathlineto{\pgfqpoint{2.101895in}{2.586549in}}%
\pgfpathlineto{\pgfqpoint{2.102679in}{2.587487in}}%
\pgfpathlineto{\pgfqpoint{2.105030in}{2.590301in}}%
\pgfpathlineto{\pgfqpoint{2.108165in}{2.590301in}}%
\pgfpathlineto{\pgfqpoint{2.111299in}{2.590301in}}%
\pgfpathlineto{\pgfqpoint{2.114434in}{2.590301in}}%
\pgfpathlineto{\pgfqpoint{2.115218in}{2.591238in}}%
\pgfpathlineto{\pgfqpoint{2.117569in}{2.594052in}}%
\pgfpathlineto{\pgfqpoint{2.120704in}{2.594052in}}%
\pgfpathlineto{\pgfqpoint{2.123838in}{2.594052in}}%
\pgfpathlineto{\pgfqpoint{2.126973in}{2.594052in}}%
\pgfpathlineto{\pgfqpoint{2.127757in}{2.594990in}}%
\pgfpathlineto{\pgfqpoint{2.130108in}{2.597804in}}%
\pgfpathlineto{\pgfqpoint{2.133243in}{2.597804in}}%
\pgfpathlineto{\pgfqpoint{2.136377in}{2.597804in}}%
\pgfpathlineto{\pgfqpoint{2.139512in}{2.597804in}}%
\pgfpathlineto{\pgfqpoint{2.140296in}{2.598742in}}%
\pgfpathlineto{\pgfqpoint{2.142647in}{2.601555in}}%
\pgfpathlineto{\pgfqpoint{2.145782in}{2.601555in}}%
\pgfpathlineto{\pgfqpoint{2.148916in}{2.601555in}}%
\pgfpathlineto{\pgfqpoint{2.152051in}{2.601555in}}%
\pgfpathlineto{\pgfqpoint{2.152835in}{2.602493in}}%
\pgfpathlineto{\pgfqpoint{2.155186in}{2.605307in}}%
\pgfpathlineto{\pgfqpoint{2.158321in}{2.605307in}}%
\pgfpathlineto{\pgfqpoint{2.161455in}{2.605307in}}%
\pgfpathlineto{\pgfqpoint{2.164590in}{2.605307in}}%
\pgfpathlineto{\pgfqpoint{2.167725in}{2.605307in}}%
\pgfpathlineto{\pgfqpoint{2.168509in}{2.606245in}}%
\pgfpathlineto{\pgfqpoint{2.170860in}{2.609058in}}%
\pgfpathlineto{\pgfqpoint{2.173994in}{2.609058in}}%
\pgfpathlineto{\pgfqpoint{2.177129in}{2.609058in}}%
\pgfpathlineto{\pgfqpoint{2.180264in}{2.609058in}}%
\pgfpathlineto{\pgfqpoint{2.181048in}{2.609996in}}%
\pgfpathlineto{\pgfqpoint{2.183399in}{2.612810in}}%
\pgfpathlineto{\pgfqpoint{2.186533in}{2.612810in}}%
\pgfpathlineto{\pgfqpoint{2.189668in}{2.612810in}}%
\pgfpathlineto{\pgfqpoint{2.192803in}{2.612810in}}%
\pgfpathlineto{\pgfqpoint{2.193586in}{2.613748in}}%
\pgfpathlineto{\pgfqpoint{2.195938in}{2.616561in}}%
\pgfpathlineto{\pgfqpoint{2.199072in}{2.616561in}}%
\pgfpathlineto{\pgfqpoint{2.202207in}{2.616561in}}%
\pgfpathlineto{\pgfqpoint{2.205342in}{2.616561in}}%
\pgfpathlineto{\pgfqpoint{2.206125in}{2.617499in}}%
\pgfpathlineto{\pgfqpoint{2.208477in}{2.620313in}}%
\pgfpathlineto{\pgfqpoint{2.211611in}{2.620313in}}%
\pgfpathlineto{\pgfqpoint{2.214746in}{2.620313in}}%
\pgfpathlineto{\pgfqpoint{2.217881in}{2.620313in}}%
\pgfpathlineto{\pgfqpoint{2.221015in}{2.620313in}}%
\pgfpathlineto{\pgfqpoint{2.221799in}{2.621251in}}%
\pgfpathlineto{\pgfqpoint{2.224150in}{2.624064in}}%
\pgfpathlineto{\pgfqpoint{2.227285in}{2.624064in}}%
\pgfpathlineto{\pgfqpoint{2.230420in}{2.624064in}}%
\pgfpathlineto{\pgfqpoint{2.233554in}{2.624064in}}%
\pgfpathlineto{\pgfqpoint{2.234338in}{2.625002in}}%
\pgfpathlineto{\pgfqpoint{2.236689in}{2.627816in}}%
\pgfpathlineto{\pgfqpoint{2.239824in}{2.627816in}}%
\pgfpathlineto{\pgfqpoint{2.242959in}{2.627816in}}%
\pgfpathlineto{\pgfqpoint{2.246093in}{2.627816in}}%
\pgfpathlineto{\pgfqpoint{2.246877in}{2.628754in}}%
\pgfpathlineto{\pgfqpoint{2.249228in}{2.631567in}}%
\pgfpathlineto{\pgfqpoint{2.252363in}{2.631567in}}%
\pgfpathlineto{\pgfqpoint{2.255498in}{2.631567in}}%
\pgfpathlineto{\pgfqpoint{2.258632in}{2.631567in}}%
\pgfpathlineto{\pgfqpoint{2.259416in}{2.632505in}}%
\pgfpathlineto{\pgfqpoint{2.261767in}{2.635319in}}%
\pgfpathlineto{\pgfqpoint{2.264902in}{2.635319in}}%
\pgfpathlineto{\pgfqpoint{2.268037in}{2.635319in}}%
\pgfpathlineto{\pgfqpoint{2.271171in}{2.635319in}}%
\pgfpathlineto{\pgfqpoint{2.274306in}{2.635319in}}%
\pgfpathlineto{\pgfqpoint{2.275090in}{2.636257in}}%
\pgfpathlineto{\pgfqpoint{2.277441in}{2.639071in}}%
\pgfpathlineto{\pgfqpoint{2.280576in}{2.639071in}}%
\pgfpathlineto{\pgfqpoint{2.283710in}{2.639071in}}%
\pgfpathlineto{\pgfqpoint{2.286845in}{2.639071in}}%
\pgfpathlineto{\pgfqpoint{2.287629in}{2.640008in}}%
\pgfpathlineto{\pgfqpoint{2.289980in}{2.642822in}}%
\pgfpathlineto{\pgfqpoint{2.293115in}{2.642822in}}%
\pgfpathlineto{\pgfqpoint{2.296249in}{2.642822in}}%
\pgfpathlineto{\pgfqpoint{2.299384in}{2.642822in}}%
\pgfpathlineto{\pgfqpoint{2.300168in}{2.643760in}}%
\pgfpathlineto{\pgfqpoint{2.302519in}{2.646574in}}%
\pgfpathlineto{\pgfqpoint{2.305654in}{2.646574in}}%
\pgfpathlineto{\pgfqpoint{2.308788in}{2.646574in}}%
\pgfpathlineto{\pgfqpoint{2.311923in}{2.646574in}}%
\pgfpathlineto{\pgfqpoint{2.312707in}{2.647512in}}%
\pgfpathlineto{\pgfqpoint{2.315058in}{2.650325in}}%
\pgfpathlineto{\pgfqpoint{2.318192in}{2.650325in}}%
\pgfpathlineto{\pgfqpoint{2.321327in}{2.650325in}}%
\pgfpathlineto{\pgfqpoint{2.324462in}{2.650325in}}%
\pgfpathlineto{\pgfqpoint{2.325246in}{2.651263in}}%
\pgfpathlineto{\pgfqpoint{2.327597in}{2.654077in}}%
\pgfpathlineto{\pgfqpoint{2.330731in}{2.654077in}}%
\pgfpathlineto{\pgfqpoint{2.333866in}{2.654077in}}%
\pgfpathlineto{\pgfqpoint{2.337001in}{2.654077in}}%
\pgfpathlineto{\pgfqpoint{2.340136in}{2.654077in}}%
\pgfpathlineto{\pgfqpoint{2.340919in}{2.655015in}}%
\pgfpathlineto{\pgfqpoint{2.343270in}{2.657828in}}%
\pgfpathlineto{\pgfqpoint{2.346405in}{2.657828in}}%
\pgfpathlineto{\pgfqpoint{2.349540in}{2.657828in}}%
\pgfpathlineto{\pgfqpoint{2.352675in}{2.657828in}}%
\pgfpathlineto{\pgfqpoint{2.353458in}{2.658766in}}%
\pgfpathlineto{\pgfqpoint{2.355809in}{2.661580in}}%
\pgfpathlineto{\pgfqpoint{2.358944in}{2.661580in}}%
\pgfpathlineto{\pgfqpoint{2.362079in}{2.661580in}}%
\pgfpathlineto{\pgfqpoint{2.365214in}{2.661580in}}%
\pgfpathlineto{\pgfqpoint{2.365997in}{2.662518in}}%
\pgfpathlineto{\pgfqpoint{2.368348in}{2.665331in}}%
\pgfpathlineto{\pgfqpoint{2.371483in}{2.665331in}}%
\pgfpathlineto{\pgfqpoint{2.374618in}{2.665331in}}%
\pgfpathlineto{\pgfqpoint{2.377753in}{2.665331in}}%
\pgfpathlineto{\pgfqpoint{2.378536in}{2.666269in}}%
\pgfpathlineto{\pgfqpoint{2.380887in}{2.669083in}}%
\pgfpathlineto{\pgfqpoint{2.384022in}{2.669083in}}%
\pgfpathlineto{\pgfqpoint{2.387157in}{2.669083in}}%
\pgfpathlineto{\pgfqpoint{2.390292in}{2.669083in}}%
\pgfpathlineto{\pgfqpoint{2.393426in}{2.669083in}}%
\pgfpathlineto{\pgfqpoint{2.394210in}{2.670021in}}%
\pgfpathlineto{\pgfqpoint{2.396561in}{2.672834in}}%
\pgfpathlineto{\pgfqpoint{2.399696in}{2.672834in}}%
\pgfpathlineto{\pgfqpoint{2.402831in}{2.672834in}}%
\pgfpathlineto{\pgfqpoint{2.405965in}{2.672834in}}%
\pgfpathlineto{\pgfqpoint{2.406749in}{2.673772in}}%
\pgfpathlineto{\pgfqpoint{2.409100in}{2.676586in}}%
\pgfpathlineto{\pgfqpoint{2.412235in}{2.676586in}}%
\pgfpathlineto{\pgfqpoint{2.415369in}{2.676586in}}%
\pgfpathlineto{\pgfqpoint{2.418504in}{2.676586in}}%
\pgfpathlineto{\pgfqpoint{2.419288in}{2.677524in}}%
\pgfpathlineto{\pgfqpoint{2.421639in}{2.680337in}}%
\pgfpathlineto{\pgfqpoint{2.424774in}{2.680337in}}%
\pgfpathlineto{\pgfqpoint{2.427908in}{2.680337in}}%
\pgfpathlineto{\pgfqpoint{2.431043in}{2.680337in}}%
\pgfpathlineto{\pgfqpoint{2.431827in}{2.681275in}}%
\pgfpathlineto{\pgfqpoint{2.434178in}{2.684089in}}%
\pgfpathlineto{\pgfqpoint{2.437313in}{2.684089in}}%
\pgfpathlineto{\pgfqpoint{2.440447in}{2.684089in}}%
\pgfpathlineto{\pgfqpoint{2.443582in}{2.684089in}}%
\pgfpathlineto{\pgfqpoint{2.446717in}{2.684089in}}%
\pgfpathlineto{\pgfqpoint{2.447501in}{2.685027in}}%
\pgfpathlineto{\pgfqpoint{2.449852in}{2.687841in}}%
\pgfpathlineto{\pgfqpoint{2.452986in}{2.687841in}}%
\pgfpathlineto{\pgfqpoint{2.456121in}{2.687841in}}%
\pgfpathlineto{\pgfqpoint{2.459256in}{2.687841in}}%
\pgfpathlineto{\pgfqpoint{2.460040in}{2.688778in}}%
\pgfpathlineto{\pgfqpoint{2.462391in}{2.691592in}}%
\pgfpathlineto{\pgfqpoint{2.465525in}{2.691592in}}%
\pgfpathlineto{\pgfqpoint{2.468660in}{2.691592in}}%
\pgfpathlineto{\pgfqpoint{2.471795in}{2.691592in}}%
\pgfpathlineto{\pgfqpoint{2.472579in}{2.692530in}}%
\pgfpathlineto{\pgfqpoint{2.474930in}{2.695344in}}%
\pgfpathlineto{\pgfqpoint{2.478064in}{2.695344in}}%
\pgfpathlineto{\pgfqpoint{2.481199in}{2.695344in}}%
\pgfpathlineto{\pgfqpoint{2.484334in}{2.695344in}}%
\pgfpathlineto{\pgfqpoint{2.485118in}{2.696281in}}%
\pgfpathlineto{\pgfqpoint{2.487469in}{2.699095in}}%
\pgfpathlineto{\pgfqpoint{2.490603in}{2.699095in}}%
\pgfpathlineto{\pgfqpoint{2.493738in}{2.699095in}}%
\pgfpathlineto{\pgfqpoint{2.496873in}{2.699095in}}%
\pgfpathlineto{\pgfqpoint{2.497656in}{2.700033in}}%
\pgfpathlineto{\pgfqpoint{2.500008in}{2.702847in}}%
\pgfpathlineto{\pgfqpoint{2.503142in}{2.702847in}}%
\pgfpathlineto{\pgfqpoint{2.506277in}{2.702847in}}%
\pgfpathlineto{\pgfqpoint{2.509412in}{2.702847in}}%
\pgfpathlineto{\pgfqpoint{2.512547in}{2.702847in}}%
\pgfpathlineto{\pgfqpoint{2.513330in}{2.703785in}}%
\pgfpathlineto{\pgfqpoint{2.515681in}{2.706598in}}%
\pgfpathlineto{\pgfqpoint{2.518816in}{2.706598in}}%
\pgfpathlineto{\pgfqpoint{2.521951in}{2.706598in}}%
\pgfpathlineto{\pgfqpoint{2.525085in}{2.706598in}}%
\pgfpathlineto{\pgfqpoint{2.525869in}{2.707536in}}%
\pgfpathlineto{\pgfqpoint{2.528220in}{2.710350in}}%
\pgfpathlineto{\pgfqpoint{2.531355in}{2.710350in}}%
\pgfpathlineto{\pgfqpoint{2.534490in}{2.710350in}}%
\pgfpathlineto{\pgfqpoint{2.537624in}{2.710350in}}%
\pgfpathlineto{\pgfqpoint{2.538408in}{2.711288in}}%
\pgfpathlineto{\pgfqpoint{2.540759in}{2.714101in}}%
\pgfpathlineto{\pgfqpoint{2.543894in}{2.714101in}}%
\pgfpathlineto{\pgfqpoint{2.547029in}{2.714101in}}%
\pgfpathlineto{\pgfqpoint{2.550163in}{2.714101in}}%
\pgfpathlineto{\pgfqpoint{2.550947in}{2.715039in}}%
\pgfpathlineto{\pgfqpoint{2.553298in}{2.717853in}}%
\pgfpathlineto{\pgfqpoint{2.556433in}{2.717853in}}%
\pgfpathlineto{\pgfqpoint{2.559568in}{2.717853in}}%
\pgfpathlineto{\pgfqpoint{2.562702in}{2.717853in}}%
\pgfpathlineto{\pgfqpoint{2.565837in}{2.717853in}}%
\pgfpathlineto{\pgfqpoint{2.566621in}{2.718791in}}%
\pgfpathlineto{\pgfqpoint{2.568972in}{2.721604in}}%
\pgfpathlineto{\pgfqpoint{2.572107in}{2.721604in}}%
\pgfpathlineto{\pgfqpoint{2.575241in}{2.721604in}}%
\pgfpathlineto{\pgfqpoint{2.578376in}{2.721604in}}%
\pgfpathlineto{\pgfqpoint{2.579160in}{2.722542in}}%
\pgfpathlineto{\pgfqpoint{2.581511in}{2.725356in}}%
\pgfpathlineto{\pgfqpoint{2.584646in}{2.725356in}}%
\pgfpathlineto{\pgfqpoint{2.587780in}{2.725356in}}%
\pgfpathlineto{\pgfqpoint{2.590915in}{2.725356in}}%
\pgfpathlineto{\pgfqpoint{2.591699in}{2.726294in}}%
\pgfpathlineto{\pgfqpoint{2.594050in}{2.729107in}}%
\pgfpathlineto{\pgfqpoint{2.597185in}{2.729107in}}%
\pgfpathlineto{\pgfqpoint{2.600319in}{2.729107in}}%
\pgfpathlineto{\pgfqpoint{2.603454in}{2.729107in}}%
\pgfpathlineto{\pgfqpoint{2.604238in}{2.730045in}}%
\pgfpathlineto{\pgfqpoint{2.606589in}{2.732859in}}%
\pgfpathlineto{\pgfqpoint{2.609724in}{2.732859in}}%
\pgfpathlineto{\pgfqpoint{2.612858in}{2.732859in}}%
\pgfpathlineto{\pgfqpoint{2.615993in}{2.732859in}}%
\pgfpathlineto{\pgfqpoint{2.616777in}{2.733797in}}%
\pgfpathlineto{\pgfqpoint{2.619128in}{2.736610in}}%
\pgfpathlineto{\pgfqpoint{2.622262in}{2.736610in}}%
\pgfpathlineto{\pgfqpoint{2.625397in}{2.736610in}}%
\pgfpathlineto{\pgfqpoint{2.628532in}{2.736610in}}%
\pgfpathlineto{\pgfqpoint{2.631667in}{2.736610in}}%
\pgfpathlineto{\pgfqpoint{2.632450in}{2.737548in}}%
\pgfpathlineto{\pgfqpoint{2.634801in}{2.740362in}}%
\pgfpathlineto{\pgfqpoint{2.637936in}{2.740362in}}%
\pgfpathlineto{\pgfqpoint{2.641071in}{2.740362in}}%
\pgfpathlineto{\pgfqpoint{2.644206in}{2.740362in}}%
\pgfpathlineto{\pgfqpoint{2.644989in}{2.741300in}}%
\pgfpathlineto{\pgfqpoint{2.647340in}{2.744114in}}%
\pgfpathlineto{\pgfqpoint{2.650475in}{2.744114in}}%
\pgfpathlineto{\pgfqpoint{2.653610in}{2.744114in}}%
\pgfpathlineto{\pgfqpoint{2.656745in}{2.744114in}}%
\pgfpathlineto{\pgfqpoint{2.657528in}{2.745051in}}%
\pgfpathlineto{\pgfqpoint{2.659879in}{2.747865in}}%
\pgfpathlineto{\pgfqpoint{2.663014in}{2.747865in}}%
\pgfpathlineto{\pgfqpoint{2.666149in}{2.747865in}}%
\pgfpathlineto{\pgfqpoint{2.669284in}{2.747865in}}%
\pgfpathlineto{\pgfqpoint{2.672418in}{2.747865in}}%
\pgfpathlineto{\pgfqpoint{2.674769in}{2.745051in}}%
\pgfpathlineto{\pgfqpoint{2.675553in}{2.744114in}}%
\pgfpathlineto{\pgfqpoint{2.677904in}{2.741300in}}%
\pgfpathlineto{\pgfqpoint{2.678688in}{2.740362in}}%
\pgfpathlineto{\pgfqpoint{2.681039in}{2.737548in}}%
\pgfpathlineto{\pgfqpoint{2.681823in}{2.736610in}}%
\pgfpathlineto{\pgfqpoint{2.684174in}{2.733797in}}%
\pgfpathlineto{\pgfqpoint{2.684957in}{2.732859in}}%
\pgfpathlineto{\pgfqpoint{2.687308in}{2.730045in}}%
\pgfpathlineto{\pgfqpoint{2.688092in}{2.729107in}}%
\pgfpathlineto{\pgfqpoint{2.690443in}{2.726294in}}%
\pgfpathlineto{\pgfqpoint{2.691227in}{2.725356in}}%
\pgfpathlineto{\pgfqpoint{2.693578in}{2.722542in}}%
\pgfpathlineto{\pgfqpoint{2.694362in}{2.721604in}}%
\pgfpathlineto{\pgfqpoint{2.696713in}{2.718791in}}%
\pgfpathlineto{\pgfqpoint{2.697496in}{2.717853in}}%
\pgfpathlineto{\pgfqpoint{2.699847in}{2.715039in}}%
\pgfpathlineto{\pgfqpoint{2.700631in}{2.714101in}}%
\pgfpathlineto{\pgfqpoint{2.702982in}{2.711288in}}%
\pgfpathlineto{\pgfqpoint{2.703766in}{2.710350in}}%
\pgfpathlineto{\pgfqpoint{2.706117in}{2.707536in}}%
\pgfpathlineto{\pgfqpoint{2.706901in}{2.706598in}}%
\pgfpathlineto{\pgfqpoint{2.709252in}{2.703785in}}%
\pgfpathlineto{\pgfqpoint{2.710035in}{2.702847in}}%
\pgfpathlineto{\pgfqpoint{2.712386in}{2.700033in}}%
\pgfpathlineto{\pgfqpoint{2.712386in}{2.696281in}}%
\pgfpathlineto{\pgfqpoint{2.713170in}{2.695344in}}%
\pgfpathlineto{\pgfqpoint{2.715521in}{2.692530in}}%
\pgfpathlineto{\pgfqpoint{2.716305in}{2.691592in}}%
\pgfpathlineto{\pgfqpoint{2.718656in}{2.688778in}}%
\pgfpathlineto{\pgfqpoint{2.719439in}{2.687841in}}%
\pgfpathlineto{\pgfqpoint{2.721791in}{2.685027in}}%
\pgfpathlineto{\pgfqpoint{2.722574in}{2.684089in}}%
\pgfpathlineto{\pgfqpoint{2.724925in}{2.681275in}}%
\pgfpathlineto{\pgfqpoint{2.725709in}{2.680337in}}%
\pgfpathlineto{\pgfqpoint{2.728060in}{2.677524in}}%
\pgfpathlineto{\pgfqpoint{2.728844in}{2.676586in}}%
\pgfpathlineto{\pgfqpoint{2.731195in}{2.673772in}}%
\pgfpathlineto{\pgfqpoint{2.731978in}{2.672834in}}%
\pgfpathlineto{\pgfqpoint{2.734330in}{2.670021in}}%
\pgfpathlineto{\pgfqpoint{2.735113in}{2.669083in}}%
\pgfpathlineto{\pgfqpoint{2.737464in}{2.666269in}}%
\pgfpathlineto{\pgfqpoint{2.738248in}{2.665331in}}%
\pgfpathlineto{\pgfqpoint{2.740599in}{2.662518in}}%
\pgfpathlineto{\pgfqpoint{2.741383in}{2.661580in}}%
\pgfpathlineto{\pgfqpoint{2.743734in}{2.658766in}}%
\pgfpathlineto{\pgfqpoint{2.744517in}{2.657828in}}%
\pgfpathlineto{\pgfqpoint{2.746868in}{2.655015in}}%
\pgfpathlineto{\pgfqpoint{2.747652in}{2.654077in}}%
\pgfpathlineto{\pgfqpoint{2.750003in}{2.651263in}}%
\pgfpathlineto{\pgfqpoint{2.750787in}{2.650325in}}%
\pgfpathlineto{\pgfqpoint{2.753138in}{2.647512in}}%
\pgfpathlineto{\pgfqpoint{2.753922in}{2.646574in}}%
\pgfpathlineto{\pgfqpoint{2.756273in}{2.643760in}}%
\pgfpathlineto{\pgfqpoint{2.757056in}{2.642822in}}%
\pgfpathlineto{\pgfqpoint{2.759407in}{2.640008in}}%
\pgfpathlineto{\pgfqpoint{2.760191in}{2.639071in}}%
\pgfpathlineto{\pgfqpoint{2.762542in}{2.636257in}}%
\pgfpathlineto{\pgfqpoint{2.762542in}{2.632505in}}%
\pgfpathlineto{\pgfqpoint{2.763326in}{2.631567in}}%
\pgfpathlineto{\pgfqpoint{2.765677in}{2.628754in}}%
\pgfpathlineto{\pgfqpoint{2.766461in}{2.627816in}}%
\pgfpathlineto{\pgfqpoint{2.768812in}{2.625002in}}%
\pgfpathlineto{\pgfqpoint{2.769595in}{2.624064in}}%
\pgfpathlineto{\pgfqpoint{2.771946in}{2.621251in}}%
\pgfpathlineto{\pgfqpoint{2.772730in}{2.620313in}}%
\pgfpathlineto{\pgfqpoint{2.775081in}{2.617499in}}%
\pgfpathlineto{\pgfqpoint{2.775865in}{2.616561in}}%
\pgfpathlineto{\pgfqpoint{2.778216in}{2.613748in}}%
\pgfpathlineto{\pgfqpoint{2.779000in}{2.612810in}}%
\pgfpathlineto{\pgfqpoint{2.781351in}{2.609996in}}%
\pgfpathlineto{\pgfqpoint{2.782134in}{2.609058in}}%
\pgfpathlineto{\pgfqpoint{2.784485in}{2.606245in}}%
\pgfpathlineto{\pgfqpoint{2.785269in}{2.605307in}}%
\pgfpathlineto{\pgfqpoint{2.787620in}{2.602493in}}%
\pgfpathlineto{\pgfqpoint{2.788404in}{2.601555in}}%
\pgfpathlineto{\pgfqpoint{2.790755in}{2.598742in}}%
\pgfpathlineto{\pgfqpoint{2.791539in}{2.597804in}}%
\pgfpathlineto{\pgfqpoint{2.793890in}{2.594990in}}%
\pgfpathlineto{\pgfqpoint{2.794673in}{2.594052in}}%
\pgfpathlineto{\pgfqpoint{2.797024in}{2.591238in}}%
\pgfpathlineto{\pgfqpoint{2.797808in}{2.590301in}}%
\pgfpathlineto{\pgfqpoint{2.800159in}{2.587487in}}%
\pgfpathlineto{\pgfqpoint{2.800943in}{2.586549in}}%
\pgfpathlineto{\pgfqpoint{2.803294in}{2.583735in}}%
\pgfpathlineto{\pgfqpoint{2.804078in}{2.582798in}}%
\pgfpathlineto{\pgfqpoint{2.806429in}{2.579984in}}%
\pgfpathlineto{\pgfqpoint{2.807212in}{2.579046in}}%
\pgfpathlineto{\pgfqpoint{2.809563in}{2.576232in}}%
\pgfpathlineto{\pgfqpoint{2.809563in}{2.572481in}}%
\pgfpathlineto{\pgfqpoint{2.810347in}{2.571543in}}%
\pgfpathlineto{\pgfqpoint{2.812698in}{2.568729in}}%
\pgfpathlineto{\pgfqpoint{2.813482in}{2.567791in}}%
\pgfpathlineto{\pgfqpoint{2.815833in}{2.564978in}}%
\pgfpathlineto{\pgfqpoint{2.816617in}{2.564040in}}%
\pgfpathlineto{\pgfqpoint{2.818968in}{2.561226in}}%
\pgfpathlineto{\pgfqpoint{2.819751in}{2.560288in}}%
\pgfpathlineto{\pgfqpoint{2.822102in}{2.557475in}}%
\pgfpathlineto{\pgfqpoint{2.822886in}{2.556537in}}%
\pgfpathlineto{\pgfqpoint{2.825237in}{2.553723in}}%
\pgfpathlineto{\pgfqpoint{2.826021in}{2.552785in}}%
\pgfpathlineto{\pgfqpoint{2.828372in}{2.549972in}}%
\pgfpathlineto{\pgfqpoint{2.829155in}{2.549034in}}%
\pgfpathlineto{\pgfqpoint{2.831507in}{2.546220in}}%
\pgfpathlineto{\pgfqpoint{2.832290in}{2.545282in}}%
\pgfpathlineto{\pgfqpoint{2.834641in}{2.542469in}}%
\pgfpathlineto{\pgfqpoint{2.835425in}{2.541531in}}%
\pgfpathlineto{\pgfqpoint{2.837776in}{2.538717in}}%
\pgfpathlineto{\pgfqpoint{2.838560in}{2.537779in}}%
\pgfpathlineto{\pgfqpoint{2.840911in}{2.534965in}}%
\pgfpathlineto{\pgfqpoint{2.841694in}{2.534028in}}%
\pgfpathlineto{\pgfqpoint{2.844046in}{2.531214in}}%
\pgfpathlineto{\pgfqpoint{2.844829in}{2.530276in}}%
\pgfpathlineto{\pgfqpoint{2.847180in}{2.527462in}}%
\pgfpathlineto{\pgfqpoint{2.847964in}{2.526525in}}%
\pgfpathlineto{\pgfqpoint{2.850315in}{2.523711in}}%
\pgfpathlineto{\pgfqpoint{2.851099in}{2.522773in}}%
\pgfpathlineto{\pgfqpoint{2.853450in}{2.519959in}}%
\pgfpathlineto{\pgfqpoint{2.854233in}{2.519021in}}%
\pgfpathlineto{\pgfqpoint{2.856584in}{2.516208in}}%
\pgfpathlineto{\pgfqpoint{2.857368in}{2.515270in}}%
\pgfpathlineto{\pgfqpoint{2.859719in}{2.512456in}}%
\pgfpathlineto{\pgfqpoint{2.859719in}{2.508705in}}%
\pgfpathlineto{\pgfqpoint{2.860503in}{2.507767in}}%
\pgfpathlineto{\pgfqpoint{2.862854in}{2.504953in}}%
\pgfpathlineto{\pgfqpoint{2.863638in}{2.504015in}}%
\pgfpathlineto{\pgfqpoint{2.865989in}{2.501202in}}%
\pgfpathlineto{\pgfqpoint{2.866772in}{2.500264in}}%
\pgfpathlineto{\pgfqpoint{2.869123in}{2.497450in}}%
\pgfpathlineto{\pgfqpoint{2.869907in}{2.496512in}}%
\pgfpathlineto{\pgfqpoint{2.872258in}{2.493699in}}%
\pgfpathlineto{\pgfqpoint{2.873042in}{2.492761in}}%
\pgfpathlineto{\pgfqpoint{2.875393in}{2.489947in}}%
\pgfpathlineto{\pgfqpoint{2.876177in}{2.489009in}}%
\pgfpathlineto{\pgfqpoint{2.878528in}{2.486196in}}%
\pgfpathlineto{\pgfqpoint{2.879311in}{2.485258in}}%
\pgfpathlineto{\pgfqpoint{2.881662in}{2.482444in}}%
\pgfpathlineto{\pgfqpoint{2.882446in}{2.481506in}}%
\pgfpathlineto{\pgfqpoint{2.884797in}{2.478692in}}%
\pgfpathlineto{\pgfqpoint{2.885581in}{2.477755in}}%
\pgfpathlineto{\pgfqpoint{2.887932in}{2.474941in}}%
\pgfpathlineto{\pgfqpoint{2.888716in}{2.474003in}}%
\pgfpathlineto{\pgfqpoint{2.891067in}{2.471189in}}%
\pgfpathlineto{\pgfqpoint{2.891850in}{2.470252in}}%
\pgfpathlineto{\pgfqpoint{2.894201in}{2.467438in}}%
\pgfpathlineto{\pgfqpoint{2.894985in}{2.466500in}}%
\pgfpathlineto{\pgfqpoint{2.897336in}{2.463686in}}%
\pgfpathlineto{\pgfqpoint{2.898120in}{2.462748in}}%
\pgfpathlineto{\pgfqpoint{2.900471in}{2.459935in}}%
\pgfpathlineto{\pgfqpoint{2.901255in}{2.458997in}}%
\pgfpathlineto{\pgfqpoint{2.903606in}{2.456183in}}%
\pgfpathlineto{\pgfqpoint{2.904389in}{2.455245in}}%
\pgfpathlineto{\pgfqpoint{2.906740in}{2.452432in}}%
\pgfpathlineto{\pgfqpoint{2.906740in}{2.448680in}}%
\pgfpathlineto{\pgfqpoint{2.907524in}{2.447742in}}%
\pgfpathlineto{\pgfqpoint{2.909875in}{2.444929in}}%
\pgfpathlineto{\pgfqpoint{2.910659in}{2.443991in}}%
\pgfpathlineto{\pgfqpoint{2.913010in}{2.441177in}}%
\pgfpathlineto{\pgfqpoint{2.913794in}{2.440239in}}%
\pgfpathlineto{\pgfqpoint{2.916145in}{2.437426in}}%
\pgfpathlineto{\pgfqpoint{2.916928in}{2.436488in}}%
\pgfpathlineto{\pgfqpoint{2.919279in}{2.433674in}}%
\pgfpathlineto{\pgfqpoint{2.920063in}{2.432736in}}%
\pgfpathlineto{\pgfqpoint{2.922414in}{2.429923in}}%
\pgfpathlineto{\pgfqpoint{2.923198in}{2.428985in}}%
\pgfpathlineto{\pgfqpoint{2.925549in}{2.426171in}}%
\pgfpathlineto{\pgfqpoint{2.926332in}{2.425233in}}%
\pgfpathlineto{\pgfqpoint{2.928684in}{2.422419in}}%
\pgfpathlineto{\pgfqpoint{2.929467in}{2.421482in}}%
\pgfpathlineto{\pgfqpoint{2.931818in}{2.418668in}}%
\pgfpathlineto{\pgfqpoint{2.932602in}{2.417730in}}%
\pgfpathlineto{\pgfqpoint{2.934953in}{2.414916in}}%
\pgfpathlineto{\pgfqpoint{2.935737in}{2.413978in}}%
\pgfpathlineto{\pgfqpoint{2.938088in}{2.411165in}}%
\pgfpathlineto{\pgfqpoint{2.938871in}{2.410227in}}%
\pgfpathlineto{\pgfqpoint{2.941223in}{2.407413in}}%
\pgfpathlineto{\pgfqpoint{2.942006in}{2.406475in}}%
\pgfpathlineto{\pgfqpoint{2.944357in}{2.403662in}}%
\pgfpathlineto{\pgfqpoint{2.945141in}{2.402724in}}%
\pgfpathlineto{\pgfqpoint{2.947492in}{2.399910in}}%
\pgfpathlineto{\pgfqpoint{2.948276in}{2.398972in}}%
\pgfpathlineto{\pgfqpoint{2.950627in}{2.396159in}}%
\pgfpathlineto{\pgfqpoint{2.951410in}{2.395221in}}%
\pgfpathlineto{\pgfqpoint{2.953761in}{2.392407in}}%
\pgfpathlineto{\pgfqpoint{2.954545in}{2.391469in}}%
\pgfpathlineto{\pgfqpoint{2.956896in}{2.388656in}}%
\pgfpathlineto{\pgfqpoint{2.956896in}{2.384904in}}%
\pgfpathlineto{\pgfqpoint{2.957680in}{2.383966in}}%
\pgfpathlineto{\pgfqpoint{2.960031in}{2.381153in}}%
\pgfpathlineto{\pgfqpoint{2.960815in}{2.380215in}}%
\pgfpathlineto{\pgfqpoint{2.963166in}{2.377401in}}%
\pgfpathlineto{\pgfqpoint{2.963949in}{2.376463in}}%
\pgfpathlineto{\pgfqpoint{2.966300in}{2.373649in}}%
\pgfpathlineto{\pgfqpoint{2.967084in}{2.372712in}}%
\pgfpathlineto{\pgfqpoint{2.969435in}{2.369898in}}%
\pgfpathlineto{\pgfqpoint{2.970219in}{2.368960in}}%
\pgfpathlineto{\pgfqpoint{2.972570in}{2.366146in}}%
\pgfpathlineto{\pgfqpoint{2.973354in}{2.365209in}}%
\pgfpathlineto{\pgfqpoint{2.975705in}{2.362395in}}%
\pgfpathlineto{\pgfqpoint{2.976488in}{2.361457in}}%
\pgfpathlineto{\pgfqpoint{2.978839in}{2.358643in}}%
\pgfpathlineto{\pgfqpoint{2.979623in}{2.357705in}}%
\pgfpathlineto{\pgfqpoint{2.981974in}{2.354892in}}%
\pgfpathlineto{\pgfqpoint{2.982758in}{2.353954in}}%
\pgfpathlineto{\pgfqpoint{2.985109in}{2.351140in}}%
\pgfpathlineto{\pgfqpoint{2.985893in}{2.350202in}}%
\pgfpathlineto{\pgfqpoint{2.988244in}{2.347389in}}%
\pgfpathlineto{\pgfqpoint{2.989027in}{2.346451in}}%
\pgfpathlineto{\pgfqpoint{2.991378in}{2.343637in}}%
\pgfpathlineto{\pgfqpoint{2.992162in}{2.342699in}}%
\pgfpathlineto{\pgfqpoint{2.994513in}{2.339886in}}%
\pgfpathlineto{\pgfqpoint{2.995297in}{2.338948in}}%
\pgfpathlineto{\pgfqpoint{2.997648in}{2.336134in}}%
\pgfpathlineto{\pgfqpoint{2.998432in}{2.335196in}}%
\pgfpathlineto{\pgfqpoint{3.000783in}{2.332383in}}%
\pgfpathlineto{\pgfqpoint{3.001566in}{2.331445in}}%
\pgfpathlineto{\pgfqpoint{3.003917in}{2.328631in}}%
\pgfpathlineto{\pgfqpoint{3.003917in}{2.324880in}}%
\pgfpathlineto{\pgfqpoint{3.004701in}{2.323942in}}%
\pgfpathlineto{\pgfqpoint{3.007052in}{2.321128in}}%
\pgfpathlineto{\pgfqpoint{3.007836in}{2.320190in}}%
\pgfpathlineto{\pgfqpoint{3.010187in}{2.317376in}}%
\pgfpathlineto{\pgfqpoint{3.010971in}{2.316439in}}%
\pgfpathlineto{\pgfqpoint{3.013322in}{2.313625in}}%
\pgfpathlineto{\pgfqpoint{3.014105in}{2.312687in}}%
\pgfpathlineto{\pgfqpoint{3.016456in}{2.309873in}}%
\pgfpathlineto{\pgfqpoint{3.017240in}{2.308936in}}%
\pgfpathlineto{\pgfqpoint{3.019591in}{2.306122in}}%
\pgfpathlineto{\pgfqpoint{3.020375in}{2.305184in}}%
\pgfpathlineto{\pgfqpoint{3.022726in}{2.302370in}}%
\pgfpathlineto{\pgfqpoint{3.023510in}{2.301432in}}%
\pgfpathlineto{\pgfqpoint{3.025861in}{2.298619in}}%
\pgfpathlineto{\pgfqpoint{3.026644in}{2.297681in}}%
\pgfpathlineto{\pgfqpoint{3.028995in}{2.294867in}}%
\pgfpathlineto{\pgfqpoint{3.029779in}{2.293929in}}%
\pgfpathlineto{\pgfqpoint{3.032130in}{2.291116in}}%
\pgfpathlineto{\pgfqpoint{3.032914in}{2.290178in}}%
\pgfpathlineto{\pgfqpoint{3.035265in}{2.287364in}}%
\pgfpathlineto{\pgfqpoint{3.036048in}{2.286426in}}%
\pgfpathlineto{\pgfqpoint{3.038400in}{2.283613in}}%
\pgfpathlineto{\pgfqpoint{3.039183in}{2.282675in}}%
\pgfpathlineto{\pgfqpoint{3.041534in}{2.279861in}}%
\pgfpathlineto{\pgfqpoint{3.042318in}{2.278923in}}%
\pgfpathlineto{\pgfqpoint{3.044669in}{2.276110in}}%
\pgfpathlineto{\pgfqpoint{3.045453in}{2.275172in}}%
\pgfpathlineto{\pgfqpoint{3.047804in}{2.272358in}}%
\pgfpathlineto{\pgfqpoint{3.048587in}{2.271420in}}%
\pgfpathlineto{\pgfqpoint{3.050938in}{2.268607in}}%
\pgfpathlineto{\pgfqpoint{3.051722in}{2.267669in}}%
\pgfpathlineto{\pgfqpoint{3.054073in}{2.264855in}}%
\pgfpathlineto{\pgfqpoint{3.054073in}{2.261103in}}%
\pgfpathlineto{\pgfqpoint{3.054857in}{2.260166in}}%
\pgfpathlineto{\pgfqpoint{3.057208in}{2.257352in}}%
\pgfpathlineto{\pgfqpoint{3.057992in}{2.256414in}}%
\pgfpathlineto{\pgfqpoint{3.060343in}{2.253600in}}%
\pgfpathlineto{\pgfqpoint{3.061126in}{2.252663in}}%
\pgfpathlineto{\pgfqpoint{3.063477in}{2.249849in}}%
\pgfpathlineto{\pgfqpoint{3.064261in}{2.248911in}}%
\pgfpathlineto{\pgfqpoint{3.066612in}{2.246097in}}%
\pgfpathlineto{\pgfqpoint{3.067396in}{2.245159in}}%
\pgfpathlineto{\pgfqpoint{3.069747in}{2.242346in}}%
\pgfpathlineto{\pgfqpoint{3.070531in}{2.241408in}}%
\pgfpathlineto{\pgfqpoint{3.072882in}{2.238594in}}%
\pgfpathlineto{\pgfqpoint{3.073665in}{2.237656in}}%
\pgfpathlineto{\pgfqpoint{3.076016in}{2.234843in}}%
\pgfpathlineto{\pgfqpoint{3.076800in}{2.233905in}}%
\pgfpathlineto{\pgfqpoint{3.079151in}{2.231091in}}%
\pgfpathlineto{\pgfqpoint{3.079935in}{2.230153in}}%
\pgfpathlineto{\pgfqpoint{3.082286in}{2.227340in}}%
\pgfpathlineto{\pgfqpoint{3.083070in}{2.226402in}}%
\pgfpathlineto{\pgfqpoint{3.085421in}{2.223588in}}%
\pgfpathlineto{\pgfqpoint{3.086204in}{2.222650in}}%
\pgfpathlineto{\pgfqpoint{3.088555in}{2.219837in}}%
\pgfpathlineto{\pgfqpoint{3.089339in}{2.218899in}}%
\pgfpathlineto{\pgfqpoint{3.091690in}{2.216085in}}%
\pgfpathlineto{\pgfqpoint{3.092474in}{2.215147in}}%
\pgfpathlineto{\pgfqpoint{3.094825in}{2.212334in}}%
\pgfpathlineto{\pgfqpoint{3.094825in}{2.208582in}}%
\pgfpathlineto{\pgfqpoint{3.092474in}{2.205768in}}%
\pgfpathlineto{\pgfqpoint{3.091690in}{2.204830in}}%
\pgfpathlineto{\pgfqpoint{3.091690in}{2.201079in}}%
\pgfpathlineto{\pgfqpoint{3.091690in}{2.197327in}}%
\pgfpathlineto{\pgfqpoint{3.089339in}{2.194514in}}%
\pgfpathlineto{\pgfqpoint{3.088555in}{2.193576in}}%
\pgfpathlineto{\pgfqpoint{3.088555in}{2.189824in}}%
\pgfpathlineto{\pgfqpoint{3.088555in}{2.186073in}}%
\pgfpathlineto{\pgfqpoint{3.086204in}{2.183259in}}%
\pgfpathlineto{\pgfqpoint{3.085421in}{2.182321in}}%
\pgfpathlineto{\pgfqpoint{3.085421in}{2.178570in}}%
\pgfpathlineto{\pgfqpoint{3.085421in}{2.174818in}}%
\pgfpathlineto{\pgfqpoint{3.083070in}{2.172005in}}%
\pgfpathlineto{\pgfqpoint{3.082286in}{2.171067in}}%
\pgfpathlineto{\pgfqpoint{3.082286in}{2.167315in}}%
\pgfpathlineto{\pgfqpoint{3.082286in}{2.163564in}}%
\pgfpathlineto{\pgfqpoint{3.079935in}{2.160750in}}%
\pgfpathlineto{\pgfqpoint{3.079151in}{2.159812in}}%
\pgfpathlineto{\pgfqpoint{3.079151in}{2.156060in}}%
\pgfpathlineto{\pgfqpoint{3.079151in}{2.152309in}}%
\pgfpathlineto{\pgfqpoint{3.079151in}{2.148557in}}%
\pgfpathlineto{\pgfqpoint{3.076800in}{2.145744in}}%
\pgfpathlineto{\pgfqpoint{3.076016in}{2.144806in}}%
\pgfpathlineto{\pgfqpoint{3.076016in}{2.141054in}}%
\pgfpathlineto{\pgfqpoint{3.076016in}{2.137303in}}%
\pgfpathlineto{\pgfqpoint{3.073665in}{2.134489in}}%
\pgfpathlineto{\pgfqpoint{3.072882in}{2.133551in}}%
\pgfpathlineto{\pgfqpoint{3.072882in}{2.129800in}}%
\pgfpathlineto{\pgfqpoint{3.072882in}{2.126048in}}%
\pgfpathlineto{\pgfqpoint{3.070531in}{2.123235in}}%
\pgfpathlineto{\pgfqpoint{3.069747in}{2.122297in}}%
\pgfpathlineto{\pgfqpoint{3.069747in}{2.118545in}}%
\pgfpathlineto{\pgfqpoint{3.069747in}{2.114794in}}%
\pgfpathlineto{\pgfqpoint{3.067396in}{2.111980in}}%
\pgfpathlineto{\pgfqpoint{3.066612in}{2.111042in}}%
\pgfpathlineto{\pgfqpoint{3.066612in}{2.107291in}}%
\pgfpathlineto{\pgfqpoint{3.066612in}{2.103539in}}%
\pgfpathlineto{\pgfqpoint{3.064261in}{2.100725in}}%
\pgfpathlineto{\pgfqpoint{3.063477in}{2.099787in}}%
\pgfpathlineto{\pgfqpoint{3.063477in}{2.096036in}}%
\pgfpathlineto{\pgfqpoint{3.063477in}{2.092284in}}%
\pgfpathlineto{\pgfqpoint{3.061126in}{2.089471in}}%
\pgfpathlineto{\pgfqpoint{3.060343in}{2.088533in}}%
\pgfpathlineto{\pgfqpoint{3.060343in}{2.084781in}}%
\pgfpathlineto{\pgfqpoint{3.060343in}{2.081030in}}%
\pgfpathlineto{\pgfqpoint{3.057992in}{2.078216in}}%
\pgfpathlineto{\pgfqpoint{3.057208in}{2.077278in}}%
\pgfpathlineto{\pgfqpoint{3.057208in}{2.073527in}}%
\pgfpathlineto{\pgfqpoint{3.057208in}{2.069775in}}%
\pgfpathlineto{\pgfqpoint{3.054857in}{2.066962in}}%
\pgfpathlineto{\pgfqpoint{3.054073in}{2.066024in}}%
\pgfpathlineto{\pgfqpoint{3.054073in}{2.062272in}}%
\pgfpathlineto{\pgfqpoint{3.054073in}{2.058521in}}%
\pgfpathlineto{\pgfqpoint{3.051722in}{2.055707in}}%
\pgfpathlineto{\pgfqpoint{3.050938in}{2.054769in}}%
\pgfpathlineto{\pgfqpoint{3.050938in}{2.051018in}}%
\pgfpathlineto{\pgfqpoint{3.050938in}{2.047266in}}%
\pgfpathlineto{\pgfqpoint{3.050938in}{2.043514in}}%
\pgfpathlineto{\pgfqpoint{3.048587in}{2.040701in}}%
\pgfpathlineto{\pgfqpoint{3.047804in}{2.039763in}}%
\pgfpathlineto{\pgfqpoint{3.047804in}{2.036011in}}%
\pgfpathlineto{\pgfqpoint{3.047804in}{2.032260in}}%
\pgfpathlineto{\pgfqpoint{3.045453in}{2.029446in}}%
\pgfpathlineto{\pgfqpoint{3.044669in}{2.028508in}}%
\pgfpathlineto{\pgfqpoint{3.044669in}{2.024757in}}%
\pgfpathlineto{\pgfqpoint{3.044669in}{2.021005in}}%
\pgfpathlineto{\pgfqpoint{3.042318in}{2.018192in}}%
\pgfpathlineto{\pgfqpoint{3.041534in}{2.017254in}}%
\pgfpathlineto{\pgfqpoint{3.041534in}{2.013502in}}%
\pgfpathlineto{\pgfqpoint{3.041534in}{2.009751in}}%
\pgfpathlineto{\pgfqpoint{3.039183in}{2.006937in}}%
\pgfpathlineto{\pgfqpoint{3.038400in}{2.005999in}}%
\pgfpathlineto{\pgfqpoint{3.038400in}{2.002248in}}%
\pgfpathlineto{\pgfqpoint{3.038400in}{1.998496in}}%
\pgfpathlineto{\pgfqpoint{3.036048in}{1.995682in}}%
\pgfpathlineto{\pgfqpoint{3.035265in}{1.994745in}}%
\pgfpathlineto{\pgfqpoint{3.035265in}{1.990993in}}%
\pgfpathlineto{\pgfqpoint{3.035265in}{1.987241in}}%
\pgfpathlineto{\pgfqpoint{3.032914in}{1.984428in}}%
\pgfpathlineto{\pgfqpoint{3.032130in}{1.983490in}}%
\pgfpathlineto{\pgfqpoint{3.032130in}{1.979738in}}%
\pgfpathlineto{\pgfqpoint{3.032130in}{1.975987in}}%
\pgfpathlineto{\pgfqpoint{3.029779in}{1.973173in}}%
\pgfpathlineto{\pgfqpoint{3.028995in}{1.972235in}}%
\pgfpathlineto{\pgfqpoint{3.028995in}{1.968484in}}%
\pgfpathlineto{\pgfqpoint{3.028995in}{1.964732in}}%
\pgfpathlineto{\pgfqpoint{3.026644in}{1.961919in}}%
\pgfpathlineto{\pgfqpoint{3.025861in}{1.960981in}}%
\pgfpathlineto{\pgfqpoint{3.025861in}{1.957229in}}%
\pgfpathlineto{\pgfqpoint{3.025861in}{1.953478in}}%
\pgfpathlineto{\pgfqpoint{3.025861in}{1.949726in}}%
\pgfpathlineto{\pgfqpoint{3.023510in}{1.946912in}}%
\pgfpathlineto{\pgfqpoint{3.022726in}{1.945975in}}%
\pgfpathlineto{\pgfqpoint{3.022726in}{1.942223in}}%
\pgfpathlineto{\pgfqpoint{3.022726in}{1.938471in}}%
\pgfpathlineto{\pgfqpoint{3.020375in}{1.935658in}}%
\pgfpathlineto{\pgfqpoint{3.019591in}{1.934720in}}%
\pgfpathlineto{\pgfqpoint{3.019591in}{1.930968in}}%
\pgfpathlineto{\pgfqpoint{3.019591in}{1.927217in}}%
\pgfpathlineto{\pgfqpoint{3.017240in}{1.924403in}}%
\pgfpathlineto{\pgfqpoint{3.016456in}{1.923465in}}%
\pgfpathlineto{\pgfqpoint{3.016456in}{1.919714in}}%
\pgfpathlineto{\pgfqpoint{3.016456in}{1.915962in}}%
\pgfpathlineto{\pgfqpoint{3.014105in}{1.913149in}}%
\pgfpathlineto{\pgfqpoint{3.013322in}{1.912211in}}%
\pgfpathlineto{\pgfqpoint{3.013322in}{1.908459in}}%
\pgfpathlineto{\pgfqpoint{3.013322in}{1.904708in}}%
\pgfpathlineto{\pgfqpoint{3.010971in}{1.901894in}}%
\pgfpathlineto{\pgfqpoint{3.010187in}{1.900956in}}%
\pgfpathlineto{\pgfqpoint{3.010187in}{1.897205in}}%
\pgfpathlineto{\pgfqpoint{3.010187in}{1.893453in}}%
\pgfpathlineto{\pgfqpoint{3.007836in}{1.890639in}}%
\pgfpathlineto{\pgfqpoint{3.007052in}{1.889702in}}%
\pgfpathlineto{\pgfqpoint{3.007052in}{1.885950in}}%
\pgfpathlineto{\pgfqpoint{3.007052in}{1.882198in}}%
\pgfpathlineto{\pgfqpoint{3.004701in}{1.879385in}}%
\pgfpathlineto{\pgfqpoint{3.003917in}{1.878447in}}%
\pgfpathlineto{\pgfqpoint{3.003917in}{1.874695in}}%
\pgfpathlineto{\pgfqpoint{3.003917in}{1.870944in}}%
\pgfpathlineto{\pgfqpoint{3.001566in}{1.868130in}}%
\pgfpathlineto{\pgfqpoint{3.000783in}{1.867192in}}%
\pgfpathlineto{\pgfqpoint{3.000783in}{1.863441in}}%
\pgfpathlineto{\pgfqpoint{3.000783in}{1.859689in}}%
\pgfpathlineto{\pgfqpoint{2.998432in}{1.856876in}}%
\pgfpathlineto{\pgfqpoint{2.997648in}{1.855938in}}%
\pgfpathlineto{\pgfqpoint{2.997648in}{1.852186in}}%
\pgfpathlineto{\pgfqpoint{2.997648in}{1.848435in}}%
\pgfpathlineto{\pgfqpoint{2.997648in}{1.844683in}}%
\pgfpathlineto{\pgfqpoint{2.995297in}{1.841869in}}%
\pgfpathlineto{\pgfqpoint{2.994513in}{1.840932in}}%
\pgfpathlineto{\pgfqpoint{2.994513in}{1.837180in}}%
\pgfpathlineto{\pgfqpoint{2.994513in}{1.833429in}}%
\pgfpathlineto{\pgfqpoint{2.992162in}{1.830615in}}%
\pgfpathlineto{\pgfqpoint{2.991378in}{1.829677in}}%
\pgfpathlineto{\pgfqpoint{2.991378in}{1.825925in}}%
\pgfpathlineto{\pgfqpoint{2.991378in}{1.822174in}}%
\pgfpathlineto{\pgfqpoint{2.989027in}{1.819360in}}%
\pgfpathlineto{\pgfqpoint{2.988244in}{1.818422in}}%
\pgfpathlineto{\pgfqpoint{2.988244in}{1.814671in}}%
\pgfpathlineto{\pgfqpoint{2.988244in}{1.810919in}}%
\pgfpathlineto{\pgfqpoint{2.985893in}{1.808106in}}%
\pgfpathlineto{\pgfqpoint{2.985109in}{1.807168in}}%
\pgfpathlineto{\pgfqpoint{2.985109in}{1.803416in}}%
\pgfpathlineto{\pgfqpoint{2.985109in}{1.799665in}}%
\pgfpathlineto{\pgfqpoint{2.982758in}{1.796851in}}%
\pgfpathlineto{\pgfqpoint{2.981974in}{1.795913in}}%
\pgfpathlineto{\pgfqpoint{2.981974in}{1.792162in}}%
\pgfpathlineto{\pgfqpoint{2.981974in}{1.788410in}}%
\pgfpathlineto{\pgfqpoint{2.979623in}{1.785596in}}%
\pgfpathlineto{\pgfqpoint{2.978839in}{1.784659in}}%
\pgfpathlineto{\pgfqpoint{2.978839in}{1.780907in}}%
\pgfpathlineto{\pgfqpoint{2.978839in}{1.777155in}}%
\pgfpathlineto{\pgfqpoint{2.976488in}{1.774342in}}%
\pgfpathlineto{\pgfqpoint{2.975705in}{1.773404in}}%
\pgfpathlineto{\pgfqpoint{2.975705in}{1.769652in}}%
\pgfpathlineto{\pgfqpoint{2.975705in}{1.765901in}}%
\pgfpathlineto{\pgfqpoint{2.973354in}{1.763087in}}%
\pgfpathlineto{\pgfqpoint{2.972570in}{1.762149in}}%
\pgfpathlineto{\pgfqpoint{2.972570in}{1.758398in}}%
\pgfpathlineto{\pgfqpoint{2.972570in}{1.754646in}}%
\pgfpathlineto{\pgfqpoint{2.972570in}{1.750895in}}%
\pgfpathlineto{\pgfqpoint{2.970219in}{1.748081in}}%
\pgfpathlineto{\pgfqpoint{2.969435in}{1.747143in}}%
\pgfpathlineto{\pgfqpoint{2.969435in}{1.743392in}}%
\pgfpathlineto{\pgfqpoint{2.969435in}{1.739640in}}%
\pgfpathlineto{\pgfqpoint{2.967084in}{1.736827in}}%
\pgfpathlineto{\pgfqpoint{2.966300in}{1.735889in}}%
\pgfpathlineto{\pgfqpoint{2.966300in}{1.732137in}}%
\pgfpathlineto{\pgfqpoint{2.966300in}{1.728386in}}%
\pgfpathlineto{\pgfqpoint{2.963949in}{1.725572in}}%
\pgfpathlineto{\pgfqpoint{2.963166in}{1.724634in}}%
\pgfpathlineto{\pgfqpoint{2.963166in}{1.720882in}}%
\pgfpathlineto{\pgfqpoint{2.963166in}{1.717131in}}%
\pgfpathlineto{\pgfqpoint{2.960815in}{1.714317in}}%
\pgfpathlineto{\pgfqpoint{2.960031in}{1.713379in}}%
\pgfpathlineto{\pgfqpoint{2.960031in}{1.709628in}}%
\pgfpathlineto{\pgfqpoint{2.960031in}{1.705876in}}%
\pgfpathlineto{\pgfqpoint{2.957680in}{1.703063in}}%
\pgfpathlineto{\pgfqpoint{2.956896in}{1.702125in}}%
\pgfpathlineto{\pgfqpoint{2.956896in}{1.698373in}}%
\pgfpathlineto{\pgfqpoint{2.956896in}{1.694622in}}%
\pgfpathlineto{\pgfqpoint{2.954545in}{1.691808in}}%
\pgfpathlineto{\pgfqpoint{2.953761in}{1.690870in}}%
\pgfpathlineto{\pgfqpoint{2.953761in}{1.687119in}}%
\pgfpathlineto{\pgfqpoint{2.953761in}{1.683367in}}%
\pgfpathlineto{\pgfqpoint{2.951410in}{1.680553in}}%
\pgfpathlineto{\pgfqpoint{2.950627in}{1.679616in}}%
\pgfpathlineto{\pgfqpoint{2.950627in}{1.675864in}}%
\pgfpathlineto{\pgfqpoint{2.950627in}{1.672113in}}%
\pgfpathlineto{\pgfqpoint{2.948276in}{1.669299in}}%
\pgfpathlineto{\pgfqpoint{2.947492in}{1.668361in}}%
\pgfpathlineto{\pgfqpoint{2.947492in}{1.664609in}}%
\pgfpathlineto{\pgfqpoint{2.947492in}{1.660858in}}%
\pgfpathlineto{\pgfqpoint{2.945141in}{1.658044in}}%
\pgfpathlineto{\pgfqpoint{2.944357in}{1.657106in}}%
\pgfpathlineto{\pgfqpoint{2.944357in}{1.653355in}}%
\pgfpathlineto{\pgfqpoint{2.944357in}{1.649603in}}%
\pgfpathlineto{\pgfqpoint{2.944357in}{1.645852in}}%
\pgfpathlineto{\pgfqpoint{2.942006in}{1.643038in}}%
\pgfpathlineto{\pgfqpoint{2.941223in}{1.642100in}}%
\pgfpathlineto{\pgfqpoint{2.941223in}{1.638349in}}%
\pgfpathlineto{\pgfqpoint{2.941223in}{1.634597in}}%
\pgfpathlineto{\pgfqpoint{2.938871in}{1.631784in}}%
\pgfpathlineto{\pgfqpoint{2.938088in}{1.630846in}}%
\pgfpathlineto{\pgfqpoint{2.938088in}{1.627094in}}%
\pgfpathlineto{\pgfqpoint{2.938088in}{1.623343in}}%
\pgfpathlineto{\pgfqpoint{2.935737in}{1.620529in}}%
\pgfpathlineto{\pgfqpoint{2.934953in}{1.619591in}}%
\pgfpathlineto{\pgfqpoint{2.934953in}{1.615840in}}%
\pgfpathlineto{\pgfqpoint{2.934953in}{1.612088in}}%
\pgfpathlineto{\pgfqpoint{2.932602in}{1.609274in}}%
\pgfpathlineto{\pgfqpoint{2.931818in}{1.608336in}}%
\pgfpathlineto{\pgfqpoint{2.931818in}{1.604585in}}%
\pgfpathlineto{\pgfqpoint{2.931818in}{1.600833in}}%
\pgfpathlineto{\pgfqpoint{2.929467in}{1.598020in}}%
\pgfpathlineto{\pgfqpoint{2.928684in}{1.597082in}}%
\pgfpathlineto{\pgfqpoint{2.928684in}{1.593330in}}%
\pgfpathlineto{\pgfqpoint{2.928684in}{1.589579in}}%
\pgfpathlineto{\pgfqpoint{2.926332in}{1.586765in}}%
\pgfpathlineto{\pgfqpoint{2.925549in}{1.585827in}}%
\pgfpathlineto{\pgfqpoint{2.925549in}{1.582076in}}%
\pgfpathlineto{\pgfqpoint{2.925549in}{1.578324in}}%
\pgfpathlineto{\pgfqpoint{2.923198in}{1.575511in}}%
\pgfpathlineto{\pgfqpoint{2.922414in}{1.574573in}}%
\pgfpathlineto{\pgfqpoint{2.922414in}{1.570821in}}%
\pgfpathlineto{\pgfqpoint{2.922414in}{1.567070in}}%
\pgfpathlineto{\pgfqpoint{2.920063in}{1.564256in}}%
\pgfpathlineto{\pgfqpoint{2.919279in}{1.563318in}}%
\pgfpathlineto{\pgfqpoint{2.919279in}{1.559566in}}%
\pgfpathlineto{\pgfqpoint{2.919279in}{1.555815in}}%
\pgfpathlineto{\pgfqpoint{2.919279in}{1.552063in}}%
\pgfpathlineto{\pgfqpoint{2.916928in}{1.549250in}}%
\pgfpathlineto{\pgfqpoint{2.916145in}{1.548312in}}%
\pgfpathlineto{\pgfqpoint{2.916145in}{1.544560in}}%
\pgfpathlineto{\pgfqpoint{2.916145in}{1.540809in}}%
\pgfpathlineto{\pgfqpoint{2.913794in}{1.537995in}}%
\pgfpathlineto{\pgfqpoint{2.913010in}{1.537057in}}%
\pgfpathlineto{\pgfqpoint{2.913010in}{1.533306in}}%
\pgfpathlineto{\pgfqpoint{2.913010in}{1.529554in}}%
\pgfpathlineto{\pgfqpoint{2.910659in}{1.526741in}}%
\pgfpathlineto{\pgfqpoint{2.909875in}{1.525803in}}%
\pgfpathlineto{\pgfqpoint{2.909875in}{1.522051in}}%
\pgfpathlineto{\pgfqpoint{2.909875in}{1.518300in}}%
\pgfpathlineto{\pgfqpoint{2.907524in}{1.515486in}}%
\pgfpathlineto{\pgfqpoint{2.906740in}{1.514548in}}%
\pgfpathlineto{\pgfqpoint{2.906740in}{1.510797in}}%
\pgfpathlineto{\pgfqpoint{2.906740in}{1.507045in}}%
\pgfpathlineto{\pgfqpoint{2.904389in}{1.504231in}}%
\pgfpathlineto{\pgfqpoint{2.903606in}{1.503293in}}%
\pgfpathlineto{\pgfqpoint{2.903606in}{1.499542in}}%
\pgfpathlineto{\pgfqpoint{2.903606in}{1.495790in}}%
\pgfpathlineto{\pgfqpoint{2.901255in}{1.492977in}}%
\pgfpathlineto{\pgfqpoint{2.900471in}{1.492039in}}%
\pgfpathlineto{\pgfqpoint{2.900471in}{1.488287in}}%
\pgfpathlineto{\pgfqpoint{2.900471in}{1.484536in}}%
\pgfpathlineto{\pgfqpoint{2.898120in}{1.481722in}}%
\pgfpathlineto{\pgfqpoint{2.897336in}{1.480784in}}%
\pgfpathlineto{\pgfqpoint{2.897336in}{1.477033in}}%
\pgfpathlineto{\pgfqpoint{2.897336in}{1.473281in}}%
\pgfpathlineto{\pgfqpoint{2.894985in}{1.470468in}}%
\pgfpathlineto{\pgfqpoint{2.894201in}{1.469530in}}%
\pgfpathlineto{\pgfqpoint{2.894201in}{1.465778in}}%
\pgfpathlineto{\pgfqpoint{2.894201in}{1.462027in}}%
\pgfpathlineto{\pgfqpoint{2.891850in}{1.459213in}}%
\pgfpathlineto{\pgfqpoint{2.891067in}{1.458275in}}%
\pgfpathlineto{\pgfqpoint{2.891067in}{1.454524in}}%
\pgfpathlineto{\pgfqpoint{2.891067in}{1.450772in}}%
\pgfpathlineto{\pgfqpoint{2.891067in}{1.447020in}}%
\pgfpathlineto{\pgfqpoint{2.888716in}{1.444207in}}%
\pgfpathlineto{\pgfqpoint{2.887932in}{1.443269in}}%
\pgfpathlineto{\pgfqpoint{2.887932in}{1.439517in}}%
\pgfpathlineto{\pgfqpoint{2.887932in}{1.435766in}}%
\pgfpathlineto{\pgfqpoint{2.885581in}{1.432952in}}%
\pgfpathlineto{\pgfqpoint{2.884797in}{1.432014in}}%
\pgfpathlineto{\pgfqpoint{2.884797in}{1.428263in}}%
\pgfpathlineto{\pgfqpoint{2.884797in}{1.424511in}}%
\pgfpathlineto{\pgfqpoint{2.882446in}{1.421698in}}%
\pgfpathlineto{\pgfqpoint{2.881662in}{1.420760in}}%
\pgfpathlineto{\pgfqpoint{2.881662in}{1.417008in}}%
\pgfpathlineto{\pgfqpoint{2.881662in}{1.413257in}}%
\pgfpathlineto{\pgfqpoint{2.879311in}{1.410443in}}%
\pgfpathlineto{\pgfqpoint{2.878528in}{1.409505in}}%
\pgfpathlineto{\pgfqpoint{2.878528in}{1.405754in}}%
\pgfpathlineto{\pgfqpoint{2.878528in}{1.402002in}}%
\pgfpathlineto{\pgfqpoint{2.876177in}{1.399188in}}%
\pgfpathlineto{\pgfqpoint{2.875393in}{1.398251in}}%
\pgfpathlineto{\pgfqpoint{2.875393in}{1.394499in}}%
\pgfpathlineto{\pgfqpoint{2.875393in}{1.390747in}}%
\pgfpathlineto{\pgfqpoint{2.873042in}{1.387934in}}%
\pgfpathlineto{\pgfqpoint{2.872258in}{1.386996in}}%
\pgfpathlineto{\pgfqpoint{2.872258in}{1.383244in}}%
\pgfpathlineto{\pgfqpoint{2.872258in}{1.379493in}}%
\pgfpathlineto{\pgfqpoint{2.869907in}{1.376679in}}%
\pgfpathlineto{\pgfqpoint{2.869123in}{1.375741in}}%
\pgfpathlineto{\pgfqpoint{2.869123in}{1.371990in}}%
\pgfpathlineto{\pgfqpoint{2.869123in}{1.368238in}}%
\pgfpathlineto{\pgfqpoint{2.866772in}{1.365425in}}%
\pgfpathlineto{\pgfqpoint{2.865989in}{1.364487in}}%
\pgfpathlineto{\pgfqpoint{2.865989in}{1.360735in}}%
\pgfpathlineto{\pgfqpoint{2.865989in}{1.356984in}}%
\pgfpathlineto{\pgfqpoint{2.865989in}{1.353232in}}%
\pgfpathlineto{\pgfqpoint{2.863638in}{1.350418in}}%
\pgfpathlineto{\pgfqpoint{2.862854in}{1.349481in}}%
\pgfpathlineto{\pgfqpoint{2.862854in}{1.345729in}}%
\pgfpathlineto{\pgfqpoint{2.862854in}{1.341977in}}%
\pgfpathlineto{\pgfqpoint{2.860503in}{1.339164in}}%
\pgfpathlineto{\pgfqpoint{2.859719in}{1.338226in}}%
\pgfpathlineto{\pgfqpoint{2.859719in}{1.334474in}}%
\pgfpathlineto{\pgfqpoint{2.859719in}{1.330723in}}%
\pgfpathlineto{\pgfqpoint{2.857368in}{1.327909in}}%
\pgfpathlineto{\pgfqpoint{2.856584in}{1.326971in}}%
\pgfpathlineto{\pgfqpoint{2.856584in}{1.323220in}}%
\pgfpathlineto{\pgfqpoint{2.856584in}{1.319468in}}%
\pgfpathlineto{\pgfqpoint{2.854233in}{1.316655in}}%
\pgfpathlineto{\pgfqpoint{2.853450in}{1.315717in}}%
\pgfpathlineto{\pgfqpoint{2.853450in}{1.311965in}}%
\pgfpathlineto{\pgfqpoint{2.853450in}{1.308214in}}%
\pgfpathlineto{\pgfqpoint{2.851099in}{1.305400in}}%
\pgfpathlineto{\pgfqpoint{2.850315in}{1.304462in}}%
\pgfpathlineto{\pgfqpoint{2.850315in}{1.300711in}}%
\pgfpathlineto{\pgfqpoint{2.850315in}{1.296959in}}%
\pgfpathlineto{\pgfqpoint{2.847964in}{1.294145in}}%
\pgfpathlineto{\pgfqpoint{2.847180in}{1.293208in}}%
\pgfpathlineto{\pgfqpoint{2.847180in}{1.289456in}}%
\pgfpathlineto{\pgfqpoint{2.847180in}{1.285704in}}%
\pgfpathlineto{\pgfqpoint{2.844829in}{1.282891in}}%
\pgfpathlineto{\pgfqpoint{2.844046in}{1.281953in}}%
\pgfpathlineto{\pgfqpoint{2.844046in}{1.278201in}}%
\pgfpathlineto{\pgfqpoint{2.844046in}{1.274450in}}%
\pgfpathlineto{\pgfqpoint{2.841694in}{1.271636in}}%
\pgfpathlineto{\pgfqpoint{2.840911in}{1.270698in}}%
\pgfpathlineto{\pgfqpoint{2.840911in}{1.266947in}}%
\pgfpathlineto{\pgfqpoint{2.840911in}{1.263195in}}%
\pgfpathlineto{\pgfqpoint{2.838560in}{1.260382in}}%
\pgfpathlineto{\pgfqpoint{2.837776in}{1.259444in}}%
\pgfpathlineto{\pgfqpoint{2.837776in}{1.255692in}}%
\pgfpathlineto{\pgfqpoint{2.837776in}{1.251941in}}%
\pgfpathlineto{\pgfqpoint{2.837776in}{1.248189in}}%
\pgfpathlineto{\pgfqpoint{2.835425in}{1.245375in}}%
\pgfpathlineto{\pgfqpoint{2.834641in}{1.244438in}}%
\pgfpathlineto{\pgfqpoint{2.834641in}{1.240686in}}%
\pgfpathlineto{\pgfqpoint{2.834641in}{1.236935in}}%
\pgfpathlineto{\pgfqpoint{2.832290in}{1.234121in}}%
\pgfpathlineto{\pgfqpoint{2.831507in}{1.233183in}}%
\pgfpathlineto{\pgfqpoint{2.831507in}{1.229431in}}%
\pgfpathlineto{\pgfqpoint{2.831507in}{1.225680in}}%
\pgfpathlineto{\pgfqpoint{2.829155in}{1.222866in}}%
\pgfpathlineto{\pgfqpoint{2.828372in}{1.221928in}}%
\pgfpathlineto{\pgfqpoint{2.828372in}{1.218177in}}%
\pgfpathlineto{\pgfqpoint{2.828372in}{1.214425in}}%
\pgfpathlineto{\pgfqpoint{2.826021in}{1.211612in}}%
\pgfpathlineto{\pgfqpoint{2.825237in}{1.210674in}}%
\pgfpathlineto{\pgfqpoint{2.825237in}{1.206922in}}%
\pgfpathlineto{\pgfqpoint{2.825237in}{1.203171in}}%
\pgfpathlineto{\pgfqpoint{2.822886in}{1.200357in}}%
\pgfpathlineto{\pgfqpoint{2.822102in}{1.199419in}}%
\pgfpathlineto{\pgfqpoint{2.822102in}{1.195668in}}%
\pgfpathlineto{\pgfqpoint{2.822102in}{1.191916in}}%
\pgfpathlineto{\pgfqpoint{2.819751in}{1.189102in}}%
\pgfpathlineto{\pgfqpoint{2.818968in}{1.188165in}}%
\pgfpathlineto{\pgfqpoint{2.818968in}{1.184413in}}%
\pgfpathlineto{\pgfqpoint{2.818968in}{1.180662in}}%
\pgfpathlineto{\pgfqpoint{2.816617in}{1.177848in}}%
\pgfpathlineto{\pgfqpoint{2.815833in}{1.176910in}}%
\pgfpathlineto{\pgfqpoint{2.815833in}{1.173158in}}%
\pgfpathlineto{\pgfqpoint{2.813482in}{1.170345in}}%
\pgfpathlineto{\pgfqpoint{2.810347in}{1.170345in}}%
\pgfpathlineto{\pgfqpoint{2.807212in}{1.170345in}}%
\pgfpathlineto{\pgfqpoint{2.806429in}{1.169407in}}%
\pgfpathlineto{\pgfqpoint{2.804078in}{1.166593in}}%
\pgfpathlineto{\pgfqpoint{2.800943in}{1.166593in}}%
\pgfpathlineto{\pgfqpoint{2.797808in}{1.166593in}}%
\pgfpathlineto{\pgfqpoint{2.797024in}{1.165655in}}%
\pgfpathlineto{\pgfqpoint{2.794673in}{1.162842in}}%
\pgfpathlineto{\pgfqpoint{2.791539in}{1.162842in}}%
\pgfpathlineto{\pgfqpoint{2.790755in}{1.161904in}}%
\pgfpathlineto{\pgfqpoint{2.788404in}{1.159090in}}%
\pgfpathlineto{\pgfqpoint{2.785269in}{1.159090in}}%
\pgfpathlineto{\pgfqpoint{2.782134in}{1.159090in}}%
\pgfpathlineto{\pgfqpoint{2.781351in}{1.158152in}}%
\pgfpathlineto{\pgfqpoint{2.779000in}{1.155339in}}%
\pgfpathlineto{\pgfqpoint{2.775865in}{1.155339in}}%
\pgfpathlineto{\pgfqpoint{2.772730in}{1.155339in}}%
\pgfpathlineto{\pgfqpoint{2.771946in}{1.154401in}}%
\pgfpathlineto{\pgfqpoint{2.769595in}{1.151587in}}%
\pgfpathlineto{\pgfqpoint{2.766461in}{1.151587in}}%
\pgfpathlineto{\pgfqpoint{2.763326in}{1.151587in}}%
\pgfpathlineto{\pgfqpoint{2.762542in}{1.150649in}}%
\pgfpathlineto{\pgfqpoint{2.760191in}{1.147836in}}%
\pgfpathlineto{\pgfqpoint{2.757056in}{1.147836in}}%
\pgfpathlineto{\pgfqpoint{2.756273in}{1.146898in}}%
\pgfpathlineto{\pgfqpoint{2.753922in}{1.144084in}}%
\pgfpathlineto{\pgfqpoint{2.750787in}{1.144084in}}%
\pgfpathlineto{\pgfqpoint{2.747652in}{1.144084in}}%
\pgfpathlineto{\pgfqpoint{2.746868in}{1.143146in}}%
\pgfpathlineto{\pgfqpoint{2.744517in}{1.140333in}}%
\pgfpathlineto{\pgfqpoint{2.741383in}{1.140333in}}%
\pgfpathlineto{\pgfqpoint{2.738248in}{1.140333in}}%
\pgfpathlineto{\pgfqpoint{2.737464in}{1.139395in}}%
\pgfpathlineto{\pgfqpoint{2.735113in}{1.136581in}}%
\pgfpathlineto{\pgfqpoint{2.731978in}{1.136581in}}%
\pgfpathlineto{\pgfqpoint{2.728844in}{1.136581in}}%
\pgfpathlineto{\pgfqpoint{2.728060in}{1.135643in}}%
\pgfpathlineto{\pgfqpoint{2.725709in}{1.132829in}}%
\pgfpathlineto{\pgfqpoint{2.722574in}{1.132829in}}%
\pgfpathlineto{\pgfqpoint{2.719439in}{1.132829in}}%
\pgfpathlineto{\pgfqpoint{2.718656in}{1.131892in}}%
\pgfpathlineto{\pgfqpoint{2.716305in}{1.129078in}}%
\pgfpathlineto{\pgfqpoint{2.713170in}{1.129078in}}%
\pgfpathlineto{\pgfqpoint{2.712386in}{1.128140in}}%
\pgfpathlineto{\pgfqpoint{2.710035in}{1.125326in}}%
\pgfpathlineto{\pgfqpoint{2.706901in}{1.125326in}}%
\pgfpathlineto{\pgfqpoint{2.703766in}{1.125326in}}%
\pgfpathlineto{\pgfqpoint{2.702982in}{1.124388in}}%
\pgfpathlineto{\pgfqpoint{2.700631in}{1.121575in}}%
\pgfpathlineto{\pgfqpoint{2.697496in}{1.121575in}}%
\pgfpathlineto{\pgfqpoint{2.694362in}{1.121575in}}%
\pgfpathlineto{\pgfqpoint{2.693578in}{1.120637in}}%
\pgfpathlineto{\pgfqpoint{2.691227in}{1.117823in}}%
\pgfpathlineto{\pgfqpoint{2.688092in}{1.117823in}}%
\pgfpathlineto{\pgfqpoint{2.684957in}{1.117823in}}%
\pgfpathlineto{\pgfqpoint{2.684174in}{1.116885in}}%
\pgfpathlineto{\pgfqpoint{2.681823in}{1.114072in}}%
\pgfpathlineto{\pgfqpoint{2.678688in}{1.114072in}}%
\pgfpathlineto{\pgfqpoint{2.677904in}{1.113134in}}%
\pgfpathlineto{\pgfqpoint{2.675553in}{1.110320in}}%
\pgfpathlineto{\pgfqpoint{2.672418in}{1.110320in}}%
\pgfpathlineto{\pgfqpoint{2.669284in}{1.110320in}}%
\pgfpathlineto{\pgfqpoint{2.668500in}{1.109382in}}%
\pgfpathlineto{\pgfqpoint{2.666149in}{1.106569in}}%
\pgfpathlineto{\pgfqpoint{2.663014in}{1.106569in}}%
\pgfpathlineto{\pgfqpoint{2.659879in}{1.106569in}}%
\pgfpathlineto{\pgfqpoint{2.659096in}{1.105631in}}%
\pgfpathlineto{\pgfqpoint{2.656745in}{1.102817in}}%
\pgfpathlineto{\pgfqpoint{2.653610in}{1.102817in}}%
\pgfpathlineto{\pgfqpoint{2.650475in}{1.102817in}}%
\pgfpathlineto{\pgfqpoint{2.649691in}{1.101879in}}%
\pgfpathlineto{\pgfqpoint{2.647340in}{1.099066in}}%
\pgfpathlineto{\pgfqpoint{2.644206in}{1.099066in}}%
\pgfpathlineto{\pgfqpoint{2.643422in}{1.098128in}}%
\pgfpathlineto{\pgfqpoint{2.641071in}{1.095314in}}%
\pgfpathlineto{\pgfqpoint{2.637936in}{1.095314in}}%
\pgfpathlineto{\pgfqpoint{2.634801in}{1.095314in}}%
\pgfpathlineto{\pgfqpoint{2.634018in}{1.094376in}}%
\pgfpathlineto{\pgfqpoint{2.631667in}{1.091563in}}%
\pgfpathlineto{\pgfqpoint{2.628532in}{1.091563in}}%
\pgfpathlineto{\pgfqpoint{2.625397in}{1.091563in}}%
\pgfpathlineto{\pgfqpoint{2.624614in}{1.090625in}}%
\pgfpathlineto{\pgfqpoint{2.622262in}{1.087811in}}%
\pgfpathlineto{\pgfqpoint{2.619128in}{1.087811in}}%
\pgfpathlineto{\pgfqpoint{2.615993in}{1.087811in}}%
\pgfpathlineto{\pgfqpoint{2.615209in}{1.086873in}}%
\pgfpathlineto{\pgfqpoint{2.612858in}{1.084059in}}%
\pgfpathlineto{\pgfqpoint{2.609724in}{1.084059in}}%
\pgfpathlineto{\pgfqpoint{2.606589in}{1.084059in}}%
\pgfpathlineto{\pgfqpoint{2.605805in}{1.083122in}}%
\pgfpathlineto{\pgfqpoint{2.603454in}{1.080308in}}%
\pgfpathlineto{\pgfqpoint{2.600319in}{1.080308in}}%
\pgfpathlineto{\pgfqpoint{2.599536in}{1.079370in}}%
\pgfpathlineto{\pgfqpoint{2.597185in}{1.076556in}}%
\pgfpathlineto{\pgfqpoint{2.594050in}{1.076556in}}%
\pgfpathlineto{\pgfqpoint{2.590915in}{1.076556in}}%
\pgfpathlineto{\pgfqpoint{2.590131in}{1.075619in}}%
\pgfpathlineto{\pgfqpoint{2.587780in}{1.072805in}}%
\pgfpathlineto{\pgfqpoint{2.584646in}{1.072805in}}%
\pgfpathlineto{\pgfqpoint{2.581511in}{1.072805in}}%
\pgfpathlineto{\pgfqpoint{2.580727in}{1.071867in}}%
\pgfpathlineto{\pgfqpoint{2.578376in}{1.069053in}}%
\pgfpathlineto{\pgfqpoint{2.575241in}{1.069053in}}%
\pgfpathlineto{\pgfqpoint{2.572107in}{1.069053in}}%
\pgfpathlineto{\pgfqpoint{2.571323in}{1.068115in}}%
\pgfpathlineto{\pgfqpoint{2.568972in}{1.065302in}}%
\pgfpathlineto{\pgfqpoint{2.565837in}{1.065302in}}%
\pgfpathlineto{\pgfqpoint{2.565053in}{1.064364in}}%
\pgfpathlineto{\pgfqpoint{2.562702in}{1.061550in}}%
\pgfpathlineto{\pgfqpoint{2.559568in}{1.061550in}}%
\pgfpathlineto{\pgfqpoint{2.556433in}{1.061550in}}%
\pgfpathlineto{\pgfqpoint{2.555649in}{1.060612in}}%
\pgfpathlineto{\pgfqpoint{2.553298in}{1.057799in}}%
\pgfpathlineto{\pgfqpoint{2.550163in}{1.057799in}}%
\pgfpathlineto{\pgfqpoint{2.547029in}{1.057799in}}%
\pgfpathlineto{\pgfqpoint{2.546245in}{1.056861in}}%
\pgfpathlineto{\pgfqpoint{2.543894in}{1.054047in}}%
\pgfpathlineto{\pgfqpoint{2.540759in}{1.054047in}}%
\pgfpathlineto{\pgfqpoint{2.537624in}{1.054047in}}%
\pgfpathlineto{\pgfqpoint{2.536841in}{1.053109in}}%
\pgfpathlineto{\pgfqpoint{2.534490in}{1.050296in}}%
\pgfpathlineto{\pgfqpoint{2.531355in}{1.050296in}}%
\pgfpathlineto{\pgfqpoint{2.530571in}{1.049358in}}%
\pgfpathlineto{\pgfqpoint{2.528220in}{1.046544in}}%
\pgfpathlineto{\pgfqpoint{2.525085in}{1.046544in}}%
\pgfpathlineto{\pgfqpoint{2.521951in}{1.046544in}}%
\pgfpathlineto{\pgfqpoint{2.521167in}{1.045606in}}%
\pgfpathlineto{\pgfqpoint{2.518816in}{1.042793in}}%
\pgfpathlineto{\pgfqpoint{2.515681in}{1.042793in}}%
\pgfpathlineto{\pgfqpoint{2.512547in}{1.042793in}}%
\pgfpathlineto{\pgfqpoint{2.511763in}{1.041855in}}%
\pgfpathlineto{\pgfqpoint{2.509412in}{1.039041in}}%
\pgfpathlineto{\pgfqpoint{2.506277in}{1.039041in}}%
\pgfpathlineto{\pgfqpoint{2.503142in}{1.039041in}}%
\pgfpathlineto{\pgfqpoint{2.502359in}{1.038103in}}%
\pgfpathlineto{\pgfqpoint{2.500008in}{1.035290in}}%
\pgfpathlineto{\pgfqpoint{2.496873in}{1.035290in}}%
\pgfpathlineto{\pgfqpoint{2.493738in}{1.035290in}}%
\pgfpathlineto{\pgfqpoint{2.492954in}{1.034352in}}%
\pgfpathlineto{\pgfqpoint{2.490603in}{1.031538in}}%
\pgfpathlineto{\pgfqpoint{2.487469in}{1.031538in}}%
\pgfpathlineto{\pgfqpoint{2.486685in}{1.030600in}}%
\pgfpathlineto{\pgfqpoint{2.484334in}{1.027786in}}%
\pgfpathlineto{\pgfqpoint{2.481199in}{1.027786in}}%
\pgfpathlineto{\pgfqpoint{2.478064in}{1.027786in}}%
\pgfpathlineto{\pgfqpoint{2.477281in}{1.026849in}}%
\pgfpathlineto{\pgfqpoint{2.474930in}{1.024035in}}%
\pgfpathlineto{\pgfqpoint{2.471795in}{1.024035in}}%
\pgfpathlineto{\pgfqpoint{2.468660in}{1.024035in}}%
\pgfpathlineto{\pgfqpoint{2.467876in}{1.023097in}}%
\pgfpathlineto{\pgfqpoint{2.465525in}{1.020283in}}%
\pgfpathlineto{\pgfqpoint{2.462391in}{1.020283in}}%
\pgfpathlineto{\pgfqpoint{2.459256in}{1.020283in}}%
\pgfpathlineto{\pgfqpoint{2.458472in}{1.019346in}}%
\pgfpathlineto{\pgfqpoint{2.456121in}{1.016532in}}%
\pgfpathlineto{\pgfqpoint{2.452986in}{1.016532in}}%
\pgfpathlineto{\pgfqpoint{2.452203in}{1.015594in}}%
\pgfpathlineto{\pgfqpoint{2.449852in}{1.012780in}}%
\pgfpathlineto{\pgfqpoint{2.446717in}{1.012780in}}%
\pgfpathlineto{\pgfqpoint{2.443582in}{1.012780in}}%
\pgfpathlineto{\pgfqpoint{2.442798in}{1.011842in}}%
\pgfpathlineto{\pgfqpoint{2.440447in}{1.009029in}}%
\pgfpathlineto{\pgfqpoint{2.437313in}{1.009029in}}%
\pgfpathlineto{\pgfqpoint{2.434178in}{1.009029in}}%
\pgfpathlineto{\pgfqpoint{2.433394in}{1.008091in}}%
\pgfpathlineto{\pgfqpoint{2.431043in}{1.005277in}}%
\pgfpathlineto{\pgfqpoint{2.427908in}{1.005277in}}%
\pgfpathlineto{\pgfqpoint{2.424774in}{1.005277in}}%
\pgfpathlineto{\pgfqpoint{2.423990in}{1.004339in}}%
\pgfpathlineto{\pgfqpoint{2.421639in}{1.001526in}}%
\pgfpathlineto{\pgfqpoint{2.418504in}{1.001526in}}%
\pgfpathlineto{\pgfqpoint{2.417721in}{1.000588in}}%
\pgfpathlineto{\pgfqpoint{2.415369in}{0.997774in}}%
\pgfpathlineto{\pgfqpoint{2.412235in}{0.997774in}}%
\pgfpathlineto{\pgfqpoint{2.409100in}{0.997774in}}%
\pgfpathlineto{\pgfqpoint{2.408316in}{0.996836in}}%
\pgfpathlineto{\pgfqpoint{2.405965in}{0.994023in}}%
\pgfpathlineto{\pgfqpoint{2.402831in}{0.994023in}}%
\pgfpathlineto{\pgfqpoint{2.399696in}{0.994023in}}%
\pgfpathlineto{\pgfqpoint{2.398912in}{0.993085in}}%
\pgfpathlineto{\pgfqpoint{2.396561in}{0.990271in}}%
\pgfpathlineto{\pgfqpoint{2.393426in}{0.990271in}}%
\pgfpathlineto{\pgfqpoint{2.390292in}{0.990271in}}%
\pgfpathlineto{\pgfqpoint{2.389508in}{0.989333in}}%
\pgfpathlineto{\pgfqpoint{2.387157in}{0.986520in}}%
\pgfpathlineto{\pgfqpoint{2.384022in}{0.986520in}}%
\pgfpathlineto{\pgfqpoint{2.380887in}{0.986520in}}%
\pgfpathlineto{\pgfqpoint{2.380104in}{0.985582in}}%
\pgfpathlineto{\pgfqpoint{2.377753in}{0.982768in}}%
\pgfpathlineto{\pgfqpoint{2.374618in}{0.982768in}}%
\pgfpathlineto{\pgfqpoint{2.373834in}{0.981830in}}%
\pgfpathlineto{\pgfqpoint{2.371483in}{0.979017in}}%
\pgfpathlineto{\pgfqpoint{2.368348in}{0.979017in}}%
\pgfpathlineto{\pgfqpoint{2.365214in}{0.979017in}}%
\pgfpathlineto{\pgfqpoint{2.364430in}{0.978079in}}%
\pgfpathlineto{\pgfqpoint{2.362079in}{0.975265in}}%
\pgfpathlineto{\pgfqpoint{2.358944in}{0.975265in}}%
\pgfpathlineto{\pgfqpoint{2.355809in}{0.975265in}}%
\pgfpathlineto{\pgfqpoint{2.355026in}{0.974327in}}%
\pgfpathlineto{\pgfqpoint{2.352675in}{0.971513in}}%
\pgfpathlineto{\pgfqpoint{2.349540in}{0.971513in}}%
\pgfpathlineto{\pgfqpoint{2.346405in}{0.971513in}}%
\pgfpathlineto{\pgfqpoint{2.345621in}{0.970576in}}%
\pgfpathlineto{\pgfqpoint{2.343270in}{0.967762in}}%
\pgfpathlineto{\pgfqpoint{2.340136in}{0.967762in}}%
\pgfpathlineto{\pgfqpoint{2.339352in}{0.966824in}}%
\pgfpathlineto{\pgfqpoint{2.337001in}{0.964010in}}%
\pgfpathlineto{\pgfqpoint{2.333866in}{0.964010in}}%
\pgfpathlineto{\pgfqpoint{2.330731in}{0.964010in}}%
\pgfpathlineto{\pgfqpoint{2.329948in}{0.963073in}}%
\pgfpathlineto{\pgfqpoint{2.327597in}{0.960259in}}%
\pgfpathlineto{\pgfqpoint{2.324462in}{0.960259in}}%
\pgfpathlineto{\pgfqpoint{2.321327in}{0.960259in}}%
\pgfpathlineto{\pgfqpoint{2.320544in}{0.959321in}}%
\pgfpathlineto{\pgfqpoint{2.318192in}{0.956507in}}%
\pgfpathlineto{\pgfqpoint{2.315058in}{0.956507in}}%
\pgfpathlineto{\pgfqpoint{2.311923in}{0.956507in}}%
\pgfpathlineto{\pgfqpoint{2.311139in}{0.955569in}}%
\pgfpathlineto{\pgfqpoint{2.308788in}{0.952756in}}%
\pgfpathlineto{\pgfqpoint{2.305654in}{0.952756in}}%
\pgfpathlineto{\pgfqpoint{2.304870in}{0.951818in}}%
\pgfpathlineto{\pgfqpoint{2.302519in}{0.949004in}}%
\pgfpathlineto{\pgfqpoint{2.299384in}{0.949004in}}%
\pgfpathlineto{\pgfqpoint{2.296249in}{0.949004in}}%
\pgfpathlineto{\pgfqpoint{2.295466in}{0.948066in}}%
\pgfpathlineto{\pgfqpoint{2.293115in}{0.945253in}}%
\pgfpathlineto{\pgfqpoint{2.289980in}{0.945253in}}%
\pgfpathlineto{\pgfqpoint{2.286845in}{0.945253in}}%
\pgfpathlineto{\pgfqpoint{2.286061in}{0.944315in}}%
\pgfpathlineto{\pgfqpoint{2.283710in}{0.941501in}}%
\pgfpathlineto{\pgfqpoint{2.280576in}{0.941501in}}%
\pgfpathlineto{\pgfqpoint{2.277441in}{0.941501in}}%
\pgfpathlineto{\pgfqpoint{2.276657in}{0.940563in}}%
\pgfpathlineto{\pgfqpoint{2.274306in}{0.937750in}}%
\pgfpathlineto{\pgfqpoint{2.271171in}{0.937750in}}%
\pgfpathlineto{\pgfqpoint{2.270388in}{0.936812in}}%
\pgfpathlineto{\pgfqpoint{2.268037in}{0.933998in}}%
\pgfpathlineto{\pgfqpoint{2.264902in}{0.933998in}}%
\pgfpathlineto{\pgfqpoint{2.261767in}{0.933998in}}%
\pgfpathlineto{\pgfqpoint{2.260983in}{0.933060in}}%
\pgfpathlineto{\pgfqpoint{2.258632in}{0.930247in}}%
\pgfpathlineto{\pgfqpoint{2.255498in}{0.930247in}}%
\pgfpathlineto{\pgfqpoint{2.252363in}{0.930247in}}%
\pgfpathlineto{\pgfqpoint{2.251579in}{0.929309in}}%
\pgfpathlineto{\pgfqpoint{2.249228in}{0.926495in}}%
\pgfpathlineto{\pgfqpoint{2.246093in}{0.926495in}}%
\pgfpathlineto{\pgfqpoint{2.242959in}{0.926495in}}%
\pgfpathlineto{\pgfqpoint{2.242175in}{0.925557in}}%
\pgfpathlineto{\pgfqpoint{2.239824in}{0.922744in}}%
\pgfpathlineto{\pgfqpoint{2.236689in}{0.922744in}}%
\pgfpathlineto{\pgfqpoint{2.233554in}{0.922744in}}%
\pgfpathlineto{\pgfqpoint{2.232771in}{0.921806in}}%
\pgfpathlineto{\pgfqpoint{2.230420in}{0.918992in}}%
\pgfpathlineto{\pgfqpoint{2.227285in}{0.918992in}}%
\pgfpathlineto{\pgfqpoint{2.226501in}{0.918054in}}%
\pgfpathlineto{\pgfqpoint{2.224150in}{0.915240in}}%
\pgfpathlineto{\pgfqpoint{2.221015in}{0.915240in}}%
\pgfpathlineto{\pgfqpoint{2.217881in}{0.915240in}}%
\pgfpathlineto{\pgfqpoint{2.217097in}{0.914303in}}%
\pgfpathlineto{\pgfqpoint{2.214746in}{0.911489in}}%
\pgfpathlineto{\pgfqpoint{2.211611in}{0.911489in}}%
\pgfpathlineto{\pgfqpoint{2.208477in}{0.911489in}}%
\pgfpathlineto{\pgfqpoint{2.207693in}{0.910551in}}%
\pgfpathlineto{\pgfqpoint{2.205342in}{0.907737in}}%
\pgfpathlineto{\pgfqpoint{2.202207in}{0.907737in}}%
\pgfpathlineto{\pgfqpoint{2.199072in}{0.907737in}}%
\pgfpathlineto{\pgfqpoint{2.198289in}{0.906799in}}%
\pgfpathlineto{\pgfqpoint{2.195938in}{0.903986in}}%
\pgfpathlineto{\pgfqpoint{2.192803in}{0.903986in}}%
\pgfpathlineto{\pgfqpoint{2.192019in}{0.903048in}}%
\pgfpathlineto{\pgfqpoint{2.189668in}{0.900234in}}%
\pgfpathlineto{\pgfqpoint{2.186533in}{0.900234in}}%
\pgfpathlineto{\pgfqpoint{2.183399in}{0.900234in}}%
\pgfpathlineto{\pgfqpoint{2.182615in}{0.899296in}}%
\pgfpathlineto{\pgfqpoint{2.180264in}{0.896483in}}%
\pgfpathlineto{\pgfqpoint{2.177129in}{0.896483in}}%
\pgfpathlineto{\pgfqpoint{2.173994in}{0.896483in}}%
\pgfpathclose%
\pgfusepath{fill}%
\end{pgfscope}%
\begin{pgfscope}%
\pgfpathrectangle{\pgfqpoint{0.888750in}{0.419100in}}{\pgfqpoint{2.504659in}{2.933700in}} %
\pgfusepath{clip}%
\pgfsetbuttcap%
\pgfsetroundjoin%
\definecolor{currentfill}{rgb}{0.494124,0.000000,0.560765}%
\pgfsetfillcolor{currentfill}%
\pgfsetfillopacity{0.300000}%
\pgfsetlinewidth{0.000000pt}%
\definecolor{currentstroke}{rgb}{0.000000,0.000000,0.000000}%
\pgfsetstrokecolor{currentstroke}%
\pgfsetdash{}{0pt}%
\pgfpathmoveto{\pgfqpoint{0.891885in}{0.419100in}}%
\pgfpathlineto{\pgfqpoint{0.895019in}{0.419100in}}%
\pgfpathlineto{\pgfqpoint{0.898154in}{0.419100in}}%
\pgfpathlineto{\pgfqpoint{0.901289in}{0.419100in}}%
\pgfpathlineto{\pgfqpoint{0.904424in}{0.419100in}}%
\pgfpathlineto{\pgfqpoint{0.907558in}{0.419100in}}%
\pgfpathlineto{\pgfqpoint{0.910693in}{0.419100in}}%
\pgfpathlineto{\pgfqpoint{0.913828in}{0.419100in}}%
\pgfpathlineto{\pgfqpoint{0.916963in}{0.419100in}}%
\pgfpathlineto{\pgfqpoint{0.920097in}{0.419100in}}%
\pgfpathlineto{\pgfqpoint{0.923232in}{0.419100in}}%
\pgfpathlineto{\pgfqpoint{0.926367in}{0.419100in}}%
\pgfpathlineto{\pgfqpoint{0.929502in}{0.419100in}}%
\pgfpathlineto{\pgfqpoint{0.932636in}{0.419100in}}%
\pgfpathlineto{\pgfqpoint{0.935771in}{0.419100in}}%
\pgfpathlineto{\pgfqpoint{0.938906in}{0.419100in}}%
\pgfpathlineto{\pgfqpoint{0.942041in}{0.419100in}}%
\pgfpathlineto{\pgfqpoint{0.945175in}{0.419100in}}%
\pgfpathlineto{\pgfqpoint{0.948310in}{0.419100in}}%
\pgfpathlineto{\pgfqpoint{0.951445in}{0.419100in}}%
\pgfpathlineto{\pgfqpoint{0.954580in}{0.419100in}}%
\pgfpathlineto{\pgfqpoint{0.957714in}{0.419100in}}%
\pgfpathlineto{\pgfqpoint{0.960849in}{0.419100in}}%
\pgfpathlineto{\pgfqpoint{0.963984in}{0.419100in}}%
\pgfpathlineto{\pgfqpoint{0.967119in}{0.419100in}}%
\pgfpathlineto{\pgfqpoint{0.970253in}{0.419100in}}%
\pgfpathlineto{\pgfqpoint{0.973388in}{0.419100in}}%
\pgfpathlineto{\pgfqpoint{0.976523in}{0.419100in}}%
\pgfpathlineto{\pgfqpoint{0.979658in}{0.419100in}}%
\pgfpathlineto{\pgfqpoint{0.982792in}{0.419100in}}%
\pgfpathlineto{\pgfqpoint{0.985927in}{0.419100in}}%
\pgfpathlineto{\pgfqpoint{0.989062in}{0.419100in}}%
\pgfpathlineto{\pgfqpoint{0.992196in}{0.419100in}}%
\pgfpathlineto{\pgfqpoint{0.995331in}{0.419100in}}%
\pgfpathlineto{\pgfqpoint{0.998466in}{0.419100in}}%
\pgfpathlineto{\pgfqpoint{1.001601in}{0.419100in}}%
\pgfpathlineto{\pgfqpoint{1.004735in}{0.419100in}}%
\pgfpathlineto{\pgfqpoint{1.007870in}{0.419100in}}%
\pgfpathlineto{\pgfqpoint{1.011005in}{0.419100in}}%
\pgfpathlineto{\pgfqpoint{1.014140in}{0.419100in}}%
\pgfpathlineto{\pgfqpoint{1.017274in}{0.419100in}}%
\pgfpathlineto{\pgfqpoint{1.020409in}{0.419100in}}%
\pgfpathlineto{\pgfqpoint{1.023544in}{0.419100in}}%
\pgfpathlineto{\pgfqpoint{1.026679in}{0.419100in}}%
\pgfpathlineto{\pgfqpoint{1.029813in}{0.419100in}}%
\pgfpathlineto{\pgfqpoint{1.032948in}{0.419100in}}%
\pgfpathlineto{\pgfqpoint{1.036083in}{0.419100in}}%
\pgfpathlineto{\pgfqpoint{1.039218in}{0.419100in}}%
\pgfpathlineto{\pgfqpoint{1.042352in}{0.419100in}}%
\pgfpathlineto{\pgfqpoint{1.045487in}{0.419100in}}%
\pgfpathlineto{\pgfqpoint{1.048622in}{0.419100in}}%
\pgfpathlineto{\pgfqpoint{1.051757in}{0.419100in}}%
\pgfpathlineto{\pgfqpoint{1.054891in}{0.419100in}}%
\pgfpathlineto{\pgfqpoint{1.058026in}{0.419100in}}%
\pgfpathlineto{\pgfqpoint{1.061161in}{0.419100in}}%
\pgfpathlineto{\pgfqpoint{1.064296in}{0.419100in}}%
\pgfpathlineto{\pgfqpoint{1.067430in}{0.419100in}}%
\pgfpathlineto{\pgfqpoint{1.070565in}{0.419100in}}%
\pgfpathlineto{\pgfqpoint{1.073700in}{0.419100in}}%
\pgfpathlineto{\pgfqpoint{1.076835in}{0.419100in}}%
\pgfpathlineto{\pgfqpoint{1.079969in}{0.419100in}}%
\pgfpathlineto{\pgfqpoint{1.083104in}{0.419100in}}%
\pgfpathlineto{\pgfqpoint{1.086239in}{0.419100in}}%
\pgfpathlineto{\pgfqpoint{1.089374in}{0.419100in}}%
\pgfpathlineto{\pgfqpoint{1.092508in}{0.419100in}}%
\pgfpathlineto{\pgfqpoint{1.095643in}{0.419100in}}%
\pgfpathlineto{\pgfqpoint{1.098778in}{0.419100in}}%
\pgfpathlineto{\pgfqpoint{1.101912in}{0.419100in}}%
\pgfpathlineto{\pgfqpoint{1.105047in}{0.419100in}}%
\pgfpathlineto{\pgfqpoint{1.108182in}{0.419100in}}%
\pgfpathlineto{\pgfqpoint{1.111317in}{0.419100in}}%
\pgfpathlineto{\pgfqpoint{1.114451in}{0.419100in}}%
\pgfpathlineto{\pgfqpoint{1.117586in}{0.419100in}}%
\pgfpathlineto{\pgfqpoint{1.120721in}{0.419100in}}%
\pgfpathlineto{\pgfqpoint{1.123856in}{0.419100in}}%
\pgfpathlineto{\pgfqpoint{1.126990in}{0.419100in}}%
\pgfpathlineto{\pgfqpoint{1.130125in}{0.419100in}}%
\pgfpathlineto{\pgfqpoint{1.133260in}{0.419100in}}%
\pgfpathlineto{\pgfqpoint{1.136395in}{0.419100in}}%
\pgfpathlineto{\pgfqpoint{1.139529in}{0.419100in}}%
\pgfpathlineto{\pgfqpoint{1.142664in}{0.419100in}}%
\pgfpathlineto{\pgfqpoint{1.145799in}{0.419100in}}%
\pgfpathlineto{\pgfqpoint{1.148934in}{0.419100in}}%
\pgfpathlineto{\pgfqpoint{1.152068in}{0.419100in}}%
\pgfpathlineto{\pgfqpoint{1.155203in}{0.419100in}}%
\pgfpathlineto{\pgfqpoint{1.158338in}{0.419100in}}%
\pgfpathlineto{\pgfqpoint{1.161473in}{0.419100in}}%
\pgfpathlineto{\pgfqpoint{1.164607in}{0.419100in}}%
\pgfpathlineto{\pgfqpoint{1.167742in}{0.419100in}}%
\pgfpathlineto{\pgfqpoint{1.170877in}{0.419100in}}%
\pgfpathlineto{\pgfqpoint{1.174012in}{0.419100in}}%
\pgfpathlineto{\pgfqpoint{1.177146in}{0.419100in}}%
\pgfpathlineto{\pgfqpoint{1.180281in}{0.419100in}}%
\pgfpathlineto{\pgfqpoint{1.183416in}{0.419100in}}%
\pgfpathlineto{\pgfqpoint{1.186551in}{0.419100in}}%
\pgfpathlineto{\pgfqpoint{1.189685in}{0.419100in}}%
\pgfpathlineto{\pgfqpoint{1.192820in}{0.419100in}}%
\pgfpathlineto{\pgfqpoint{1.195955in}{0.419100in}}%
\pgfpathlineto{\pgfqpoint{1.199089in}{0.419100in}}%
\pgfpathlineto{\pgfqpoint{1.202224in}{0.419100in}}%
\pgfpathlineto{\pgfqpoint{1.205359in}{0.419100in}}%
\pgfpathlineto{\pgfqpoint{1.208494in}{0.419100in}}%
\pgfpathlineto{\pgfqpoint{1.211628in}{0.419100in}}%
\pgfpathlineto{\pgfqpoint{1.214763in}{0.419100in}}%
\pgfpathlineto{\pgfqpoint{1.217898in}{0.419100in}}%
\pgfpathlineto{\pgfqpoint{1.221033in}{0.419100in}}%
\pgfpathlineto{\pgfqpoint{1.224167in}{0.419100in}}%
\pgfpathlineto{\pgfqpoint{1.227302in}{0.419100in}}%
\pgfpathlineto{\pgfqpoint{1.230437in}{0.419100in}}%
\pgfpathlineto{\pgfqpoint{1.233572in}{0.419100in}}%
\pgfpathlineto{\pgfqpoint{1.236706in}{0.419100in}}%
\pgfpathlineto{\pgfqpoint{1.239841in}{0.419100in}}%
\pgfpathlineto{\pgfqpoint{1.242976in}{0.419100in}}%
\pgfpathlineto{\pgfqpoint{1.246111in}{0.419100in}}%
\pgfpathlineto{\pgfqpoint{1.249245in}{0.419100in}}%
\pgfpathlineto{\pgfqpoint{1.252380in}{0.419100in}}%
\pgfpathlineto{\pgfqpoint{1.255515in}{0.419100in}}%
\pgfpathlineto{\pgfqpoint{1.258650in}{0.419100in}}%
\pgfpathlineto{\pgfqpoint{1.261784in}{0.419100in}}%
\pgfpathlineto{\pgfqpoint{1.264919in}{0.419100in}}%
\pgfpathlineto{\pgfqpoint{1.268054in}{0.419100in}}%
\pgfpathlineto{\pgfqpoint{1.271189in}{0.419100in}}%
\pgfpathlineto{\pgfqpoint{1.274323in}{0.419100in}}%
\pgfpathlineto{\pgfqpoint{1.277458in}{0.419100in}}%
\pgfpathlineto{\pgfqpoint{1.280593in}{0.419100in}}%
\pgfpathlineto{\pgfqpoint{1.283728in}{0.419100in}}%
\pgfpathlineto{\pgfqpoint{1.286862in}{0.419100in}}%
\pgfpathlineto{\pgfqpoint{1.289997in}{0.419100in}}%
\pgfpathlineto{\pgfqpoint{1.293132in}{0.419100in}}%
\pgfpathlineto{\pgfqpoint{1.296266in}{0.419100in}}%
\pgfpathlineto{\pgfqpoint{1.299401in}{0.419100in}}%
\pgfpathlineto{\pgfqpoint{1.302536in}{0.419100in}}%
\pgfpathlineto{\pgfqpoint{1.305671in}{0.419100in}}%
\pgfpathlineto{\pgfqpoint{1.308805in}{0.419100in}}%
\pgfpathlineto{\pgfqpoint{1.311940in}{0.419100in}}%
\pgfpathlineto{\pgfqpoint{1.315075in}{0.419100in}}%
\pgfpathlineto{\pgfqpoint{1.318210in}{0.419100in}}%
\pgfpathlineto{\pgfqpoint{1.321344in}{0.419100in}}%
\pgfpathlineto{\pgfqpoint{1.324479in}{0.419100in}}%
\pgfpathlineto{\pgfqpoint{1.327614in}{0.419100in}}%
\pgfpathlineto{\pgfqpoint{1.330749in}{0.419100in}}%
\pgfpathlineto{\pgfqpoint{1.333883in}{0.419100in}}%
\pgfpathlineto{\pgfqpoint{1.337018in}{0.419100in}}%
\pgfpathlineto{\pgfqpoint{1.340153in}{0.419100in}}%
\pgfpathlineto{\pgfqpoint{1.343288in}{0.419100in}}%
\pgfpathlineto{\pgfqpoint{1.346422in}{0.419100in}}%
\pgfpathlineto{\pgfqpoint{1.349557in}{0.419100in}}%
\pgfpathlineto{\pgfqpoint{1.352692in}{0.419100in}}%
\pgfpathlineto{\pgfqpoint{1.355827in}{0.419100in}}%
\pgfpathlineto{\pgfqpoint{1.358961in}{0.419100in}}%
\pgfpathlineto{\pgfqpoint{1.362096in}{0.419100in}}%
\pgfpathlineto{\pgfqpoint{1.365231in}{0.419100in}}%
\pgfpathlineto{\pgfqpoint{1.368366in}{0.419100in}}%
\pgfpathlineto{\pgfqpoint{1.371500in}{0.419100in}}%
\pgfpathlineto{\pgfqpoint{1.374635in}{0.419100in}}%
\pgfpathlineto{\pgfqpoint{1.377770in}{0.419100in}}%
\pgfpathlineto{\pgfqpoint{1.380905in}{0.419100in}}%
\pgfpathlineto{\pgfqpoint{1.384039in}{0.419100in}}%
\pgfpathlineto{\pgfqpoint{1.387174in}{0.419100in}}%
\pgfpathlineto{\pgfqpoint{1.390309in}{0.419100in}}%
\pgfpathlineto{\pgfqpoint{1.393444in}{0.419100in}}%
\pgfpathlineto{\pgfqpoint{1.396578in}{0.419100in}}%
\pgfpathlineto{\pgfqpoint{1.399713in}{0.419100in}}%
\pgfpathlineto{\pgfqpoint{1.402848in}{0.419100in}}%
\pgfpathlineto{\pgfqpoint{1.405982in}{0.419100in}}%
\pgfpathlineto{\pgfqpoint{1.409117in}{0.419100in}}%
\pgfpathlineto{\pgfqpoint{1.412252in}{0.419100in}}%
\pgfpathlineto{\pgfqpoint{1.415387in}{0.419100in}}%
\pgfpathlineto{\pgfqpoint{1.418521in}{0.419100in}}%
\pgfpathlineto{\pgfqpoint{1.421656in}{0.419100in}}%
\pgfpathlineto{\pgfqpoint{1.424791in}{0.419100in}}%
\pgfpathlineto{\pgfqpoint{1.427926in}{0.419100in}}%
\pgfpathlineto{\pgfqpoint{1.431060in}{0.419100in}}%
\pgfpathlineto{\pgfqpoint{1.434195in}{0.419100in}}%
\pgfpathlineto{\pgfqpoint{1.437330in}{0.419100in}}%
\pgfpathlineto{\pgfqpoint{1.440465in}{0.419100in}}%
\pgfpathlineto{\pgfqpoint{1.443599in}{0.419100in}}%
\pgfpathlineto{\pgfqpoint{1.446734in}{0.419100in}}%
\pgfpathlineto{\pgfqpoint{1.449869in}{0.419100in}}%
\pgfpathlineto{\pgfqpoint{1.453004in}{0.419100in}}%
\pgfpathlineto{\pgfqpoint{1.456138in}{0.419100in}}%
\pgfpathlineto{\pgfqpoint{1.459273in}{0.419100in}}%
\pgfpathlineto{\pgfqpoint{1.462408in}{0.419100in}}%
\pgfpathlineto{\pgfqpoint{1.465543in}{0.419100in}}%
\pgfpathlineto{\pgfqpoint{1.468677in}{0.419100in}}%
\pgfpathlineto{\pgfqpoint{1.471812in}{0.419100in}}%
\pgfpathlineto{\pgfqpoint{1.474947in}{0.419100in}}%
\pgfpathlineto{\pgfqpoint{1.478082in}{0.419100in}}%
\pgfpathlineto{\pgfqpoint{1.481216in}{0.419100in}}%
\pgfpathlineto{\pgfqpoint{1.484351in}{0.419100in}}%
\pgfpathlineto{\pgfqpoint{1.487486in}{0.419100in}}%
\pgfpathlineto{\pgfqpoint{1.490621in}{0.419100in}}%
\pgfpathlineto{\pgfqpoint{1.493755in}{0.419100in}}%
\pgfpathlineto{\pgfqpoint{1.496890in}{0.419100in}}%
\pgfpathlineto{\pgfqpoint{1.500025in}{0.419100in}}%
\pgfpathlineto{\pgfqpoint{1.503159in}{0.419100in}}%
\pgfpathlineto{\pgfqpoint{1.506294in}{0.419100in}}%
\pgfpathlineto{\pgfqpoint{1.509429in}{0.419100in}}%
\pgfpathlineto{\pgfqpoint{1.512564in}{0.419100in}}%
\pgfpathlineto{\pgfqpoint{1.515698in}{0.419100in}}%
\pgfpathlineto{\pgfqpoint{1.518833in}{0.419100in}}%
\pgfpathlineto{\pgfqpoint{1.521968in}{0.419100in}}%
\pgfpathlineto{\pgfqpoint{1.525103in}{0.419100in}}%
\pgfpathlineto{\pgfqpoint{1.528237in}{0.419100in}}%
\pgfpathlineto{\pgfqpoint{1.531372in}{0.419100in}}%
\pgfpathlineto{\pgfqpoint{1.534507in}{0.419100in}}%
\pgfpathlineto{\pgfqpoint{1.537642in}{0.419100in}}%
\pgfpathlineto{\pgfqpoint{1.540776in}{0.419100in}}%
\pgfpathlineto{\pgfqpoint{1.543911in}{0.419100in}}%
\pgfpathlineto{\pgfqpoint{1.547046in}{0.419100in}}%
\pgfpathlineto{\pgfqpoint{1.550181in}{0.419100in}}%
\pgfpathlineto{\pgfqpoint{1.553315in}{0.419100in}}%
\pgfpathlineto{\pgfqpoint{1.556450in}{0.419100in}}%
\pgfpathlineto{\pgfqpoint{1.559585in}{0.419100in}}%
\pgfpathlineto{\pgfqpoint{1.562720in}{0.419100in}}%
\pgfpathlineto{\pgfqpoint{1.565854in}{0.419100in}}%
\pgfpathlineto{\pgfqpoint{1.568989in}{0.419100in}}%
\pgfpathlineto{\pgfqpoint{1.572124in}{0.419100in}}%
\pgfpathlineto{\pgfqpoint{1.575259in}{0.419100in}}%
\pgfpathlineto{\pgfqpoint{1.578393in}{0.419100in}}%
\pgfpathlineto{\pgfqpoint{1.581528in}{0.419100in}}%
\pgfpathlineto{\pgfqpoint{1.584663in}{0.419100in}}%
\pgfpathlineto{\pgfqpoint{1.587798in}{0.419100in}}%
\pgfpathlineto{\pgfqpoint{1.590932in}{0.419100in}}%
\pgfpathlineto{\pgfqpoint{1.594067in}{0.419100in}}%
\pgfpathlineto{\pgfqpoint{1.597202in}{0.419100in}}%
\pgfpathlineto{\pgfqpoint{1.600337in}{0.419100in}}%
\pgfpathlineto{\pgfqpoint{1.603471in}{0.419100in}}%
\pgfpathlineto{\pgfqpoint{1.606606in}{0.419100in}}%
\pgfpathlineto{\pgfqpoint{1.609741in}{0.419100in}}%
\pgfpathlineto{\pgfqpoint{1.612875in}{0.419100in}}%
\pgfpathlineto{\pgfqpoint{1.616010in}{0.419100in}}%
\pgfpathlineto{\pgfqpoint{1.619145in}{0.419100in}}%
\pgfpathlineto{\pgfqpoint{1.622280in}{0.419100in}}%
\pgfpathlineto{\pgfqpoint{1.625414in}{0.419100in}}%
\pgfpathlineto{\pgfqpoint{1.628549in}{0.419100in}}%
\pgfpathlineto{\pgfqpoint{1.631684in}{0.419100in}}%
\pgfpathlineto{\pgfqpoint{1.634819in}{0.419100in}}%
\pgfpathlineto{\pgfqpoint{1.637953in}{0.419100in}}%
\pgfpathlineto{\pgfqpoint{1.641088in}{0.419100in}}%
\pgfpathlineto{\pgfqpoint{1.644223in}{0.419100in}}%
\pgfpathlineto{\pgfqpoint{1.647358in}{0.419100in}}%
\pgfpathlineto{\pgfqpoint{1.650492in}{0.419100in}}%
\pgfpathlineto{\pgfqpoint{1.653627in}{0.419100in}}%
\pgfpathlineto{\pgfqpoint{1.656762in}{0.419100in}}%
\pgfpathlineto{\pgfqpoint{1.659897in}{0.419100in}}%
\pgfpathlineto{\pgfqpoint{1.663031in}{0.419100in}}%
\pgfpathlineto{\pgfqpoint{1.666166in}{0.419100in}}%
\pgfpathlineto{\pgfqpoint{1.669301in}{0.419100in}}%
\pgfpathlineto{\pgfqpoint{1.672436in}{0.419100in}}%
\pgfpathlineto{\pgfqpoint{1.675570in}{0.419100in}}%
\pgfpathlineto{\pgfqpoint{1.678705in}{0.419100in}}%
\pgfpathlineto{\pgfqpoint{1.681840in}{0.419100in}}%
\pgfpathlineto{\pgfqpoint{1.684975in}{0.419100in}}%
\pgfpathlineto{\pgfqpoint{1.688109in}{0.419100in}}%
\pgfpathlineto{\pgfqpoint{1.691244in}{0.419100in}}%
\pgfpathlineto{\pgfqpoint{1.694379in}{0.419100in}}%
\pgfpathlineto{\pgfqpoint{1.697514in}{0.419100in}}%
\pgfpathlineto{\pgfqpoint{1.700648in}{0.419100in}}%
\pgfpathlineto{\pgfqpoint{1.703783in}{0.419100in}}%
\pgfpathlineto{\pgfqpoint{1.706918in}{0.419100in}}%
\pgfpathlineto{\pgfqpoint{1.710052in}{0.419100in}}%
\pgfpathlineto{\pgfqpoint{1.713187in}{0.419100in}}%
\pgfpathlineto{\pgfqpoint{1.716322in}{0.419100in}}%
\pgfpathlineto{\pgfqpoint{1.719457in}{0.419100in}}%
\pgfpathlineto{\pgfqpoint{1.722591in}{0.419100in}}%
\pgfpathlineto{\pgfqpoint{1.725726in}{0.419100in}}%
\pgfpathlineto{\pgfqpoint{1.728861in}{0.419100in}}%
\pgfpathlineto{\pgfqpoint{1.731996in}{0.419100in}}%
\pgfpathlineto{\pgfqpoint{1.735130in}{0.419100in}}%
\pgfpathlineto{\pgfqpoint{1.738265in}{0.419100in}}%
\pgfpathlineto{\pgfqpoint{1.741400in}{0.419100in}}%
\pgfpathlineto{\pgfqpoint{1.744535in}{0.419100in}}%
\pgfpathlineto{\pgfqpoint{1.747669in}{0.419100in}}%
\pgfpathlineto{\pgfqpoint{1.750804in}{0.419100in}}%
\pgfpathlineto{\pgfqpoint{1.753939in}{0.419100in}}%
\pgfpathlineto{\pgfqpoint{1.757074in}{0.419100in}}%
\pgfpathlineto{\pgfqpoint{1.760208in}{0.419100in}}%
\pgfpathlineto{\pgfqpoint{1.763343in}{0.419100in}}%
\pgfpathlineto{\pgfqpoint{1.766478in}{0.419100in}}%
\pgfpathlineto{\pgfqpoint{1.769613in}{0.419100in}}%
\pgfpathlineto{\pgfqpoint{1.772747in}{0.419100in}}%
\pgfpathlineto{\pgfqpoint{1.775882in}{0.419100in}}%
\pgfpathlineto{\pgfqpoint{1.779017in}{0.419100in}}%
\pgfpathlineto{\pgfqpoint{1.782152in}{0.419100in}}%
\pgfpathlineto{\pgfqpoint{1.785286in}{0.419100in}}%
\pgfpathlineto{\pgfqpoint{1.788421in}{0.419100in}}%
\pgfpathlineto{\pgfqpoint{1.791556in}{0.419100in}}%
\pgfpathlineto{\pgfqpoint{1.794691in}{0.419100in}}%
\pgfpathlineto{\pgfqpoint{1.797825in}{0.419100in}}%
\pgfpathlineto{\pgfqpoint{1.800960in}{0.419100in}}%
\pgfpathlineto{\pgfqpoint{1.804095in}{0.419100in}}%
\pgfpathlineto{\pgfqpoint{1.807229in}{0.419100in}}%
\pgfpathlineto{\pgfqpoint{1.810364in}{0.419100in}}%
\pgfpathlineto{\pgfqpoint{1.813499in}{0.419100in}}%
\pgfpathlineto{\pgfqpoint{1.816634in}{0.419100in}}%
\pgfpathlineto{\pgfqpoint{1.819768in}{0.419100in}}%
\pgfpathlineto{\pgfqpoint{1.822903in}{0.419100in}}%
\pgfpathlineto{\pgfqpoint{1.826038in}{0.419100in}}%
\pgfpathlineto{\pgfqpoint{1.829173in}{0.419100in}}%
\pgfpathlineto{\pgfqpoint{1.832307in}{0.419100in}}%
\pgfpathlineto{\pgfqpoint{1.835442in}{0.419100in}}%
\pgfpathlineto{\pgfqpoint{1.838577in}{0.419100in}}%
\pgfpathlineto{\pgfqpoint{1.841712in}{0.419100in}}%
\pgfpathlineto{\pgfqpoint{1.844846in}{0.419100in}}%
\pgfpathlineto{\pgfqpoint{1.847981in}{0.419100in}}%
\pgfpathlineto{\pgfqpoint{1.851116in}{0.419100in}}%
\pgfpathlineto{\pgfqpoint{1.854251in}{0.419100in}}%
\pgfpathlineto{\pgfqpoint{1.857385in}{0.419100in}}%
\pgfpathlineto{\pgfqpoint{1.860520in}{0.419100in}}%
\pgfpathlineto{\pgfqpoint{1.863655in}{0.419100in}}%
\pgfpathlineto{\pgfqpoint{1.866790in}{0.419100in}}%
\pgfpathlineto{\pgfqpoint{1.869924in}{0.419100in}}%
\pgfpathlineto{\pgfqpoint{1.873059in}{0.419100in}}%
\pgfpathlineto{\pgfqpoint{1.876194in}{0.419100in}}%
\pgfpathlineto{\pgfqpoint{1.879329in}{0.419100in}}%
\pgfpathlineto{\pgfqpoint{1.882463in}{0.419100in}}%
\pgfpathlineto{\pgfqpoint{1.885598in}{0.419100in}}%
\pgfpathlineto{\pgfqpoint{1.888733in}{0.419100in}}%
\pgfpathlineto{\pgfqpoint{1.891868in}{0.419100in}}%
\pgfpathlineto{\pgfqpoint{1.895002in}{0.419100in}}%
\pgfpathlineto{\pgfqpoint{1.898137in}{0.419100in}}%
\pgfpathlineto{\pgfqpoint{1.901272in}{0.419100in}}%
\pgfpathlineto{\pgfqpoint{1.904407in}{0.419100in}}%
\pgfpathlineto{\pgfqpoint{1.907541in}{0.419100in}}%
\pgfpathlineto{\pgfqpoint{1.910676in}{0.419100in}}%
\pgfpathlineto{\pgfqpoint{1.913811in}{0.419100in}}%
\pgfpathlineto{\pgfqpoint{1.916945in}{0.419100in}}%
\pgfpathlineto{\pgfqpoint{1.920080in}{0.419100in}}%
\pgfpathlineto{\pgfqpoint{1.923215in}{0.419100in}}%
\pgfpathlineto{\pgfqpoint{1.926350in}{0.419100in}}%
\pgfpathlineto{\pgfqpoint{1.929484in}{0.419100in}}%
\pgfpathlineto{\pgfqpoint{1.932619in}{0.419100in}}%
\pgfpathlineto{\pgfqpoint{1.935754in}{0.419100in}}%
\pgfpathlineto{\pgfqpoint{1.938889in}{0.419100in}}%
\pgfpathlineto{\pgfqpoint{1.942023in}{0.419100in}}%
\pgfpathlineto{\pgfqpoint{1.945158in}{0.419100in}}%
\pgfpathlineto{\pgfqpoint{1.948293in}{0.419100in}}%
\pgfpathlineto{\pgfqpoint{1.951428in}{0.419100in}}%
\pgfpathlineto{\pgfqpoint{1.954562in}{0.419100in}}%
\pgfpathlineto{\pgfqpoint{1.957697in}{0.419100in}}%
\pgfpathlineto{\pgfqpoint{1.960832in}{0.419100in}}%
\pgfpathlineto{\pgfqpoint{1.963967in}{0.419100in}}%
\pgfpathlineto{\pgfqpoint{1.967101in}{0.419100in}}%
\pgfpathlineto{\pgfqpoint{1.970236in}{0.419100in}}%
\pgfpathlineto{\pgfqpoint{1.973371in}{0.419100in}}%
\pgfpathlineto{\pgfqpoint{1.976506in}{0.419100in}}%
\pgfpathlineto{\pgfqpoint{1.979640in}{0.419100in}}%
\pgfpathlineto{\pgfqpoint{1.982775in}{0.419100in}}%
\pgfpathlineto{\pgfqpoint{1.985910in}{0.419100in}}%
\pgfpathlineto{\pgfqpoint{1.989045in}{0.419100in}}%
\pgfpathlineto{\pgfqpoint{1.992179in}{0.419100in}}%
\pgfpathlineto{\pgfqpoint{1.995314in}{0.419100in}}%
\pgfpathlineto{\pgfqpoint{1.998449in}{0.419100in}}%
\pgfpathlineto{\pgfqpoint{2.001584in}{0.419100in}}%
\pgfpathlineto{\pgfqpoint{2.004718in}{0.419100in}}%
\pgfpathlineto{\pgfqpoint{2.007853in}{0.419100in}}%
\pgfpathlineto{\pgfqpoint{2.010988in}{0.419100in}}%
\pgfpathlineto{\pgfqpoint{2.014122in}{0.419100in}}%
\pgfpathlineto{\pgfqpoint{2.017257in}{0.419100in}}%
\pgfpathlineto{\pgfqpoint{2.020392in}{0.419100in}}%
\pgfpathlineto{\pgfqpoint{2.023527in}{0.419100in}}%
\pgfpathlineto{\pgfqpoint{2.026661in}{0.419100in}}%
\pgfpathlineto{\pgfqpoint{2.029796in}{0.419100in}}%
\pgfpathlineto{\pgfqpoint{2.032931in}{0.419100in}}%
\pgfpathlineto{\pgfqpoint{2.036066in}{0.419100in}}%
\pgfpathlineto{\pgfqpoint{2.039200in}{0.419100in}}%
\pgfpathlineto{\pgfqpoint{2.042335in}{0.419100in}}%
\pgfpathlineto{\pgfqpoint{2.045470in}{0.419100in}}%
\pgfpathlineto{\pgfqpoint{2.048605in}{0.419100in}}%
\pgfpathlineto{\pgfqpoint{2.051739in}{0.419100in}}%
\pgfpathlineto{\pgfqpoint{2.054874in}{0.419100in}}%
\pgfpathlineto{\pgfqpoint{2.058009in}{0.419100in}}%
\pgfpathlineto{\pgfqpoint{2.061144in}{0.419100in}}%
\pgfpathlineto{\pgfqpoint{2.064278in}{0.419100in}}%
\pgfpathlineto{\pgfqpoint{2.067413in}{0.419100in}}%
\pgfpathlineto{\pgfqpoint{2.070548in}{0.419100in}}%
\pgfpathlineto{\pgfqpoint{2.073683in}{0.419100in}}%
\pgfpathlineto{\pgfqpoint{2.076817in}{0.419100in}}%
\pgfpathlineto{\pgfqpoint{2.079952in}{0.419100in}}%
\pgfpathlineto{\pgfqpoint{2.083087in}{0.419100in}}%
\pgfpathlineto{\pgfqpoint{2.086222in}{0.419100in}}%
\pgfpathlineto{\pgfqpoint{2.089356in}{0.419100in}}%
\pgfpathlineto{\pgfqpoint{2.092491in}{0.419100in}}%
\pgfpathlineto{\pgfqpoint{2.095626in}{0.419100in}}%
\pgfpathlineto{\pgfqpoint{2.098761in}{0.419100in}}%
\pgfpathlineto{\pgfqpoint{2.101895in}{0.419100in}}%
\pgfpathlineto{\pgfqpoint{2.105030in}{0.419100in}}%
\pgfpathlineto{\pgfqpoint{2.108165in}{0.419100in}}%
\pgfpathlineto{\pgfqpoint{2.111299in}{0.419100in}}%
\pgfpathlineto{\pgfqpoint{2.114434in}{0.419100in}}%
\pgfpathlineto{\pgfqpoint{2.117569in}{0.419100in}}%
\pgfpathlineto{\pgfqpoint{2.120704in}{0.419100in}}%
\pgfpathlineto{\pgfqpoint{2.123838in}{0.419100in}}%
\pgfpathlineto{\pgfqpoint{2.126973in}{0.419100in}}%
\pgfpathlineto{\pgfqpoint{2.130108in}{0.419100in}}%
\pgfpathlineto{\pgfqpoint{2.133243in}{0.419100in}}%
\pgfpathlineto{\pgfqpoint{2.136377in}{0.419100in}}%
\pgfpathlineto{\pgfqpoint{2.139512in}{0.419100in}}%
\pgfpathlineto{\pgfqpoint{2.142647in}{0.419100in}}%
\pgfpathlineto{\pgfqpoint{2.145782in}{0.419100in}}%
\pgfpathlineto{\pgfqpoint{2.148916in}{0.419100in}}%
\pgfpathlineto{\pgfqpoint{2.152051in}{0.419100in}}%
\pgfpathlineto{\pgfqpoint{2.155186in}{0.419100in}}%
\pgfpathlineto{\pgfqpoint{2.158321in}{0.419100in}}%
\pgfpathlineto{\pgfqpoint{2.161455in}{0.419100in}}%
\pgfpathlineto{\pgfqpoint{2.164590in}{0.419100in}}%
\pgfpathlineto{\pgfqpoint{2.167725in}{0.419100in}}%
\pgfpathlineto{\pgfqpoint{2.170860in}{0.419100in}}%
\pgfpathlineto{\pgfqpoint{2.173994in}{0.419100in}}%
\pgfpathlineto{\pgfqpoint{2.177129in}{0.419100in}}%
\pgfpathlineto{\pgfqpoint{2.180264in}{0.419100in}}%
\pgfpathlineto{\pgfqpoint{2.183399in}{0.419100in}}%
\pgfpathlineto{\pgfqpoint{2.186533in}{0.419100in}}%
\pgfpathlineto{\pgfqpoint{2.189668in}{0.419100in}}%
\pgfpathlineto{\pgfqpoint{2.192803in}{0.419100in}}%
\pgfpathlineto{\pgfqpoint{2.195938in}{0.419100in}}%
\pgfpathlineto{\pgfqpoint{2.199072in}{0.419100in}}%
\pgfpathlineto{\pgfqpoint{2.202207in}{0.419100in}}%
\pgfpathlineto{\pgfqpoint{2.205342in}{0.419100in}}%
\pgfpathlineto{\pgfqpoint{2.208477in}{0.419100in}}%
\pgfpathlineto{\pgfqpoint{2.211611in}{0.419100in}}%
\pgfpathlineto{\pgfqpoint{2.214746in}{0.419100in}}%
\pgfpathlineto{\pgfqpoint{2.217881in}{0.419100in}}%
\pgfpathlineto{\pgfqpoint{2.221015in}{0.419100in}}%
\pgfpathlineto{\pgfqpoint{2.224150in}{0.419100in}}%
\pgfpathlineto{\pgfqpoint{2.227285in}{0.419100in}}%
\pgfpathlineto{\pgfqpoint{2.230420in}{0.419100in}}%
\pgfpathlineto{\pgfqpoint{2.233554in}{0.419100in}}%
\pgfpathlineto{\pgfqpoint{2.236689in}{0.419100in}}%
\pgfpathlineto{\pgfqpoint{2.239824in}{0.419100in}}%
\pgfpathlineto{\pgfqpoint{2.242959in}{0.419100in}}%
\pgfpathlineto{\pgfqpoint{2.246093in}{0.419100in}}%
\pgfpathlineto{\pgfqpoint{2.249228in}{0.419100in}}%
\pgfpathlineto{\pgfqpoint{2.252363in}{0.419100in}}%
\pgfpathlineto{\pgfqpoint{2.255498in}{0.419100in}}%
\pgfpathlineto{\pgfqpoint{2.258632in}{0.419100in}}%
\pgfpathlineto{\pgfqpoint{2.261767in}{0.419100in}}%
\pgfpathlineto{\pgfqpoint{2.264902in}{0.419100in}}%
\pgfpathlineto{\pgfqpoint{2.268037in}{0.419100in}}%
\pgfpathlineto{\pgfqpoint{2.271171in}{0.419100in}}%
\pgfpathlineto{\pgfqpoint{2.274306in}{0.419100in}}%
\pgfpathlineto{\pgfqpoint{2.277441in}{0.419100in}}%
\pgfpathlineto{\pgfqpoint{2.280576in}{0.419100in}}%
\pgfpathlineto{\pgfqpoint{2.283710in}{0.419100in}}%
\pgfpathlineto{\pgfqpoint{2.286845in}{0.419100in}}%
\pgfpathlineto{\pgfqpoint{2.289980in}{0.419100in}}%
\pgfpathlineto{\pgfqpoint{2.293115in}{0.419100in}}%
\pgfpathlineto{\pgfqpoint{2.296249in}{0.419100in}}%
\pgfpathlineto{\pgfqpoint{2.299384in}{0.419100in}}%
\pgfpathlineto{\pgfqpoint{2.302519in}{0.419100in}}%
\pgfpathlineto{\pgfqpoint{2.305654in}{0.419100in}}%
\pgfpathlineto{\pgfqpoint{2.308788in}{0.419100in}}%
\pgfpathlineto{\pgfqpoint{2.311923in}{0.419100in}}%
\pgfpathlineto{\pgfqpoint{2.315058in}{0.419100in}}%
\pgfpathlineto{\pgfqpoint{2.318192in}{0.419100in}}%
\pgfpathlineto{\pgfqpoint{2.321327in}{0.419100in}}%
\pgfpathlineto{\pgfqpoint{2.324462in}{0.419100in}}%
\pgfpathlineto{\pgfqpoint{2.327597in}{0.419100in}}%
\pgfpathlineto{\pgfqpoint{2.330731in}{0.419100in}}%
\pgfpathlineto{\pgfqpoint{2.333866in}{0.419100in}}%
\pgfpathlineto{\pgfqpoint{2.337001in}{0.419100in}}%
\pgfpathlineto{\pgfqpoint{2.340136in}{0.419100in}}%
\pgfpathlineto{\pgfqpoint{2.343270in}{0.419100in}}%
\pgfpathlineto{\pgfqpoint{2.346405in}{0.419100in}}%
\pgfpathlineto{\pgfqpoint{2.349540in}{0.419100in}}%
\pgfpathlineto{\pgfqpoint{2.352675in}{0.419100in}}%
\pgfpathlineto{\pgfqpoint{2.355809in}{0.419100in}}%
\pgfpathlineto{\pgfqpoint{2.358944in}{0.419100in}}%
\pgfpathlineto{\pgfqpoint{2.362079in}{0.419100in}}%
\pgfpathlineto{\pgfqpoint{2.365214in}{0.419100in}}%
\pgfpathlineto{\pgfqpoint{2.368348in}{0.419100in}}%
\pgfpathlineto{\pgfqpoint{2.371483in}{0.419100in}}%
\pgfpathlineto{\pgfqpoint{2.374618in}{0.419100in}}%
\pgfpathlineto{\pgfqpoint{2.377753in}{0.419100in}}%
\pgfpathlineto{\pgfqpoint{2.380887in}{0.419100in}}%
\pgfpathlineto{\pgfqpoint{2.384022in}{0.419100in}}%
\pgfpathlineto{\pgfqpoint{2.387157in}{0.419100in}}%
\pgfpathlineto{\pgfqpoint{2.390292in}{0.419100in}}%
\pgfpathlineto{\pgfqpoint{2.393426in}{0.419100in}}%
\pgfpathlineto{\pgfqpoint{2.396561in}{0.419100in}}%
\pgfpathlineto{\pgfqpoint{2.399696in}{0.419100in}}%
\pgfpathlineto{\pgfqpoint{2.402831in}{0.419100in}}%
\pgfpathlineto{\pgfqpoint{2.405965in}{0.419100in}}%
\pgfpathlineto{\pgfqpoint{2.409100in}{0.419100in}}%
\pgfpathlineto{\pgfqpoint{2.412235in}{0.419100in}}%
\pgfpathlineto{\pgfqpoint{2.415369in}{0.419100in}}%
\pgfpathlineto{\pgfqpoint{2.418504in}{0.419100in}}%
\pgfpathlineto{\pgfqpoint{2.421639in}{0.419100in}}%
\pgfpathlineto{\pgfqpoint{2.424774in}{0.419100in}}%
\pgfpathlineto{\pgfqpoint{2.427908in}{0.419100in}}%
\pgfpathlineto{\pgfqpoint{2.431043in}{0.419100in}}%
\pgfpathlineto{\pgfqpoint{2.434178in}{0.419100in}}%
\pgfpathlineto{\pgfqpoint{2.437313in}{0.419100in}}%
\pgfpathlineto{\pgfqpoint{2.440447in}{0.419100in}}%
\pgfpathlineto{\pgfqpoint{2.443582in}{0.419100in}}%
\pgfpathlineto{\pgfqpoint{2.446717in}{0.419100in}}%
\pgfpathlineto{\pgfqpoint{2.449852in}{0.419100in}}%
\pgfpathlineto{\pgfqpoint{2.452986in}{0.419100in}}%
\pgfpathlineto{\pgfqpoint{2.456121in}{0.419100in}}%
\pgfpathlineto{\pgfqpoint{2.459256in}{0.419100in}}%
\pgfpathlineto{\pgfqpoint{2.462391in}{0.419100in}}%
\pgfpathlineto{\pgfqpoint{2.465525in}{0.419100in}}%
\pgfpathlineto{\pgfqpoint{2.468660in}{0.419100in}}%
\pgfpathlineto{\pgfqpoint{2.471795in}{0.419100in}}%
\pgfpathlineto{\pgfqpoint{2.474930in}{0.419100in}}%
\pgfpathlineto{\pgfqpoint{2.478064in}{0.419100in}}%
\pgfpathlineto{\pgfqpoint{2.481199in}{0.419100in}}%
\pgfpathlineto{\pgfqpoint{2.484334in}{0.419100in}}%
\pgfpathlineto{\pgfqpoint{2.487469in}{0.419100in}}%
\pgfpathlineto{\pgfqpoint{2.490603in}{0.419100in}}%
\pgfpathlineto{\pgfqpoint{2.493738in}{0.419100in}}%
\pgfpathlineto{\pgfqpoint{2.496873in}{0.419100in}}%
\pgfpathlineto{\pgfqpoint{2.500008in}{0.419100in}}%
\pgfpathlineto{\pgfqpoint{2.503142in}{0.419100in}}%
\pgfpathlineto{\pgfqpoint{2.506277in}{0.419100in}}%
\pgfpathlineto{\pgfqpoint{2.509412in}{0.419100in}}%
\pgfpathlineto{\pgfqpoint{2.512547in}{0.419100in}}%
\pgfpathlineto{\pgfqpoint{2.515681in}{0.419100in}}%
\pgfpathlineto{\pgfqpoint{2.518816in}{0.419100in}}%
\pgfpathlineto{\pgfqpoint{2.521951in}{0.419100in}}%
\pgfpathlineto{\pgfqpoint{2.525085in}{0.419100in}}%
\pgfpathlineto{\pgfqpoint{2.528220in}{0.419100in}}%
\pgfpathlineto{\pgfqpoint{2.531355in}{0.419100in}}%
\pgfpathlineto{\pgfqpoint{2.534490in}{0.419100in}}%
\pgfpathlineto{\pgfqpoint{2.537624in}{0.419100in}}%
\pgfpathlineto{\pgfqpoint{2.540759in}{0.419100in}}%
\pgfpathlineto{\pgfqpoint{2.543894in}{0.419100in}}%
\pgfpathlineto{\pgfqpoint{2.547029in}{0.419100in}}%
\pgfpathlineto{\pgfqpoint{2.550163in}{0.419100in}}%
\pgfpathlineto{\pgfqpoint{2.553298in}{0.419100in}}%
\pgfpathlineto{\pgfqpoint{2.556433in}{0.419100in}}%
\pgfpathlineto{\pgfqpoint{2.559568in}{0.419100in}}%
\pgfpathlineto{\pgfqpoint{2.562702in}{0.419100in}}%
\pgfpathlineto{\pgfqpoint{2.565837in}{0.419100in}}%
\pgfpathlineto{\pgfqpoint{2.568972in}{0.419100in}}%
\pgfpathlineto{\pgfqpoint{2.572107in}{0.419100in}}%
\pgfpathlineto{\pgfqpoint{2.575241in}{0.419100in}}%
\pgfpathlineto{\pgfqpoint{2.578376in}{0.419100in}}%
\pgfpathlineto{\pgfqpoint{2.581511in}{0.419100in}}%
\pgfpathlineto{\pgfqpoint{2.584646in}{0.419100in}}%
\pgfpathlineto{\pgfqpoint{2.587780in}{0.419100in}}%
\pgfpathlineto{\pgfqpoint{2.590915in}{0.419100in}}%
\pgfpathlineto{\pgfqpoint{2.594050in}{0.419100in}}%
\pgfpathlineto{\pgfqpoint{2.597185in}{0.419100in}}%
\pgfpathlineto{\pgfqpoint{2.600319in}{0.419100in}}%
\pgfpathlineto{\pgfqpoint{2.603454in}{0.419100in}}%
\pgfpathlineto{\pgfqpoint{2.606589in}{0.419100in}}%
\pgfpathlineto{\pgfqpoint{2.609724in}{0.419100in}}%
\pgfpathlineto{\pgfqpoint{2.612858in}{0.419100in}}%
\pgfpathlineto{\pgfqpoint{2.615993in}{0.419100in}}%
\pgfpathlineto{\pgfqpoint{2.619128in}{0.419100in}}%
\pgfpathlineto{\pgfqpoint{2.622262in}{0.419100in}}%
\pgfpathlineto{\pgfqpoint{2.625397in}{0.419100in}}%
\pgfpathlineto{\pgfqpoint{2.628532in}{0.419100in}}%
\pgfpathlineto{\pgfqpoint{2.631667in}{0.419100in}}%
\pgfpathlineto{\pgfqpoint{2.634801in}{0.419100in}}%
\pgfpathlineto{\pgfqpoint{2.637936in}{0.419100in}}%
\pgfpathlineto{\pgfqpoint{2.641071in}{0.419100in}}%
\pgfpathlineto{\pgfqpoint{2.644206in}{0.419100in}}%
\pgfpathlineto{\pgfqpoint{2.647340in}{0.419100in}}%
\pgfpathlineto{\pgfqpoint{2.650475in}{0.419100in}}%
\pgfpathlineto{\pgfqpoint{2.653610in}{0.419100in}}%
\pgfpathlineto{\pgfqpoint{2.656745in}{0.419100in}}%
\pgfpathlineto{\pgfqpoint{2.659879in}{0.419100in}}%
\pgfpathlineto{\pgfqpoint{2.663014in}{0.419100in}}%
\pgfpathlineto{\pgfqpoint{2.666149in}{0.419100in}}%
\pgfpathlineto{\pgfqpoint{2.669284in}{0.419100in}}%
\pgfpathlineto{\pgfqpoint{2.672418in}{0.419100in}}%
\pgfpathlineto{\pgfqpoint{2.675553in}{0.419100in}}%
\pgfpathlineto{\pgfqpoint{2.678688in}{0.419100in}}%
\pgfpathlineto{\pgfqpoint{2.681823in}{0.419100in}}%
\pgfpathlineto{\pgfqpoint{2.684957in}{0.419100in}}%
\pgfpathlineto{\pgfqpoint{2.688092in}{0.419100in}}%
\pgfpathlineto{\pgfqpoint{2.691227in}{0.419100in}}%
\pgfpathlineto{\pgfqpoint{2.694362in}{0.419100in}}%
\pgfpathlineto{\pgfqpoint{2.697496in}{0.419100in}}%
\pgfpathlineto{\pgfqpoint{2.700631in}{0.419100in}}%
\pgfpathlineto{\pgfqpoint{2.703766in}{0.419100in}}%
\pgfpathlineto{\pgfqpoint{2.706901in}{0.419100in}}%
\pgfpathlineto{\pgfqpoint{2.710035in}{0.419100in}}%
\pgfpathlineto{\pgfqpoint{2.713170in}{0.419100in}}%
\pgfpathlineto{\pgfqpoint{2.716305in}{0.419100in}}%
\pgfpathlineto{\pgfqpoint{2.719439in}{0.419100in}}%
\pgfpathlineto{\pgfqpoint{2.722574in}{0.419100in}}%
\pgfpathlineto{\pgfqpoint{2.725709in}{0.419100in}}%
\pgfpathlineto{\pgfqpoint{2.728844in}{0.419100in}}%
\pgfpathlineto{\pgfqpoint{2.731978in}{0.419100in}}%
\pgfpathlineto{\pgfqpoint{2.735113in}{0.419100in}}%
\pgfpathlineto{\pgfqpoint{2.738248in}{0.419100in}}%
\pgfpathlineto{\pgfqpoint{2.741383in}{0.419100in}}%
\pgfpathlineto{\pgfqpoint{2.744517in}{0.419100in}}%
\pgfpathlineto{\pgfqpoint{2.747652in}{0.419100in}}%
\pgfpathlineto{\pgfqpoint{2.750787in}{0.419100in}}%
\pgfpathlineto{\pgfqpoint{2.753922in}{0.419100in}}%
\pgfpathlineto{\pgfqpoint{2.757056in}{0.419100in}}%
\pgfpathlineto{\pgfqpoint{2.760191in}{0.419100in}}%
\pgfpathlineto{\pgfqpoint{2.763326in}{0.419100in}}%
\pgfpathlineto{\pgfqpoint{2.766461in}{0.419100in}}%
\pgfpathlineto{\pgfqpoint{2.769595in}{0.419100in}}%
\pgfpathlineto{\pgfqpoint{2.772730in}{0.419100in}}%
\pgfpathlineto{\pgfqpoint{2.775865in}{0.419100in}}%
\pgfpathlineto{\pgfqpoint{2.779000in}{0.419100in}}%
\pgfpathlineto{\pgfqpoint{2.782134in}{0.419100in}}%
\pgfpathlineto{\pgfqpoint{2.785269in}{0.419100in}}%
\pgfpathlineto{\pgfqpoint{2.788404in}{0.419100in}}%
\pgfpathlineto{\pgfqpoint{2.791539in}{0.419100in}}%
\pgfpathlineto{\pgfqpoint{2.794673in}{0.419100in}}%
\pgfpathlineto{\pgfqpoint{2.797808in}{0.419100in}}%
\pgfpathlineto{\pgfqpoint{2.800943in}{0.419100in}}%
\pgfpathlineto{\pgfqpoint{2.804078in}{0.419100in}}%
\pgfpathlineto{\pgfqpoint{2.807212in}{0.419100in}}%
\pgfpathlineto{\pgfqpoint{2.810347in}{0.419100in}}%
\pgfpathlineto{\pgfqpoint{2.813482in}{0.419100in}}%
\pgfpathlineto{\pgfqpoint{2.816617in}{0.419100in}}%
\pgfpathlineto{\pgfqpoint{2.819751in}{0.419100in}}%
\pgfpathlineto{\pgfqpoint{2.822886in}{0.419100in}}%
\pgfpathlineto{\pgfqpoint{2.826021in}{0.419100in}}%
\pgfpathlineto{\pgfqpoint{2.829155in}{0.419100in}}%
\pgfpathlineto{\pgfqpoint{2.832290in}{0.419100in}}%
\pgfpathlineto{\pgfqpoint{2.835425in}{0.419100in}}%
\pgfpathlineto{\pgfqpoint{2.838560in}{0.419100in}}%
\pgfpathlineto{\pgfqpoint{2.841694in}{0.419100in}}%
\pgfpathlineto{\pgfqpoint{2.844829in}{0.419100in}}%
\pgfpathlineto{\pgfqpoint{2.847964in}{0.419100in}}%
\pgfpathlineto{\pgfqpoint{2.851099in}{0.419100in}}%
\pgfpathlineto{\pgfqpoint{2.854233in}{0.419100in}}%
\pgfpathlineto{\pgfqpoint{2.857368in}{0.419100in}}%
\pgfpathlineto{\pgfqpoint{2.860503in}{0.419100in}}%
\pgfpathlineto{\pgfqpoint{2.863638in}{0.419100in}}%
\pgfpathlineto{\pgfqpoint{2.866772in}{0.419100in}}%
\pgfpathlineto{\pgfqpoint{2.869907in}{0.419100in}}%
\pgfpathlineto{\pgfqpoint{2.873042in}{0.419100in}}%
\pgfpathlineto{\pgfqpoint{2.876177in}{0.419100in}}%
\pgfpathlineto{\pgfqpoint{2.879311in}{0.419100in}}%
\pgfpathlineto{\pgfqpoint{2.882446in}{0.419100in}}%
\pgfpathlineto{\pgfqpoint{2.885581in}{0.419100in}}%
\pgfpathlineto{\pgfqpoint{2.888716in}{0.419100in}}%
\pgfpathlineto{\pgfqpoint{2.891850in}{0.419100in}}%
\pgfpathlineto{\pgfqpoint{2.894985in}{0.419100in}}%
\pgfpathlineto{\pgfqpoint{2.898120in}{0.419100in}}%
\pgfpathlineto{\pgfqpoint{2.901255in}{0.419100in}}%
\pgfpathlineto{\pgfqpoint{2.904389in}{0.419100in}}%
\pgfpathlineto{\pgfqpoint{2.907524in}{0.419100in}}%
\pgfpathlineto{\pgfqpoint{2.910659in}{0.419100in}}%
\pgfpathlineto{\pgfqpoint{2.913794in}{0.419100in}}%
\pgfpathlineto{\pgfqpoint{2.916928in}{0.419100in}}%
\pgfpathlineto{\pgfqpoint{2.920063in}{0.419100in}}%
\pgfpathlineto{\pgfqpoint{2.923198in}{0.419100in}}%
\pgfpathlineto{\pgfqpoint{2.926332in}{0.419100in}}%
\pgfpathlineto{\pgfqpoint{2.929467in}{0.419100in}}%
\pgfpathlineto{\pgfqpoint{2.932602in}{0.419100in}}%
\pgfpathlineto{\pgfqpoint{2.935737in}{0.419100in}}%
\pgfpathlineto{\pgfqpoint{2.938871in}{0.419100in}}%
\pgfpathlineto{\pgfqpoint{2.942006in}{0.419100in}}%
\pgfpathlineto{\pgfqpoint{2.945141in}{0.419100in}}%
\pgfpathlineto{\pgfqpoint{2.948276in}{0.419100in}}%
\pgfpathlineto{\pgfqpoint{2.951410in}{0.419100in}}%
\pgfpathlineto{\pgfqpoint{2.954545in}{0.419100in}}%
\pgfpathlineto{\pgfqpoint{2.957680in}{0.419100in}}%
\pgfpathlineto{\pgfqpoint{2.960815in}{0.419100in}}%
\pgfpathlineto{\pgfqpoint{2.963949in}{0.419100in}}%
\pgfpathlineto{\pgfqpoint{2.967084in}{0.419100in}}%
\pgfpathlineto{\pgfqpoint{2.970219in}{0.419100in}}%
\pgfpathlineto{\pgfqpoint{2.973354in}{0.419100in}}%
\pgfpathlineto{\pgfqpoint{2.976488in}{0.419100in}}%
\pgfpathlineto{\pgfqpoint{2.979623in}{0.419100in}}%
\pgfpathlineto{\pgfqpoint{2.982758in}{0.419100in}}%
\pgfpathlineto{\pgfqpoint{2.985893in}{0.419100in}}%
\pgfpathlineto{\pgfqpoint{2.989027in}{0.419100in}}%
\pgfpathlineto{\pgfqpoint{2.992162in}{0.419100in}}%
\pgfpathlineto{\pgfqpoint{2.995297in}{0.419100in}}%
\pgfpathlineto{\pgfqpoint{2.998432in}{0.419100in}}%
\pgfpathlineto{\pgfqpoint{3.001566in}{0.419100in}}%
\pgfpathlineto{\pgfqpoint{3.004701in}{0.419100in}}%
\pgfpathlineto{\pgfqpoint{3.007836in}{0.419100in}}%
\pgfpathlineto{\pgfqpoint{3.010971in}{0.419100in}}%
\pgfpathlineto{\pgfqpoint{3.014105in}{0.419100in}}%
\pgfpathlineto{\pgfqpoint{3.017240in}{0.419100in}}%
\pgfpathlineto{\pgfqpoint{3.020375in}{0.419100in}}%
\pgfpathlineto{\pgfqpoint{3.023510in}{0.419100in}}%
\pgfpathlineto{\pgfqpoint{3.026644in}{0.419100in}}%
\pgfpathlineto{\pgfqpoint{3.029779in}{0.419100in}}%
\pgfpathlineto{\pgfqpoint{3.032914in}{0.419100in}}%
\pgfpathlineto{\pgfqpoint{3.036048in}{0.419100in}}%
\pgfpathlineto{\pgfqpoint{3.039183in}{0.419100in}}%
\pgfpathlineto{\pgfqpoint{3.042318in}{0.419100in}}%
\pgfpathlineto{\pgfqpoint{3.045453in}{0.419100in}}%
\pgfpathlineto{\pgfqpoint{3.048587in}{0.419100in}}%
\pgfpathlineto{\pgfqpoint{3.051722in}{0.419100in}}%
\pgfpathlineto{\pgfqpoint{3.054857in}{0.419100in}}%
\pgfpathlineto{\pgfqpoint{3.057992in}{0.419100in}}%
\pgfpathlineto{\pgfqpoint{3.061126in}{0.419100in}}%
\pgfpathlineto{\pgfqpoint{3.064261in}{0.419100in}}%
\pgfpathlineto{\pgfqpoint{3.067396in}{0.419100in}}%
\pgfpathlineto{\pgfqpoint{3.070531in}{0.419100in}}%
\pgfpathlineto{\pgfqpoint{3.073665in}{0.419100in}}%
\pgfpathlineto{\pgfqpoint{3.076800in}{0.419100in}}%
\pgfpathlineto{\pgfqpoint{3.079935in}{0.419100in}}%
\pgfpathlineto{\pgfqpoint{3.083070in}{0.419100in}}%
\pgfpathlineto{\pgfqpoint{3.086204in}{0.419100in}}%
\pgfpathlineto{\pgfqpoint{3.089339in}{0.419100in}}%
\pgfpathlineto{\pgfqpoint{3.092474in}{0.419100in}}%
\pgfpathlineto{\pgfqpoint{3.095609in}{0.419100in}}%
\pgfpathlineto{\pgfqpoint{3.098743in}{0.419100in}}%
\pgfpathlineto{\pgfqpoint{3.101878in}{0.419100in}}%
\pgfpathlineto{\pgfqpoint{3.105013in}{0.419100in}}%
\pgfpathlineto{\pgfqpoint{3.108148in}{0.419100in}}%
\pgfpathlineto{\pgfqpoint{3.111282in}{0.419100in}}%
\pgfpathlineto{\pgfqpoint{3.114417in}{0.419100in}}%
\pgfpathlineto{\pgfqpoint{3.117552in}{0.419100in}}%
\pgfpathlineto{\pgfqpoint{3.120687in}{0.419100in}}%
\pgfpathlineto{\pgfqpoint{3.123821in}{0.419100in}}%
\pgfpathlineto{\pgfqpoint{3.126956in}{0.419100in}}%
\pgfpathlineto{\pgfqpoint{3.130091in}{0.419100in}}%
\pgfpathlineto{\pgfqpoint{3.133225in}{0.419100in}}%
\pgfpathlineto{\pgfqpoint{3.136360in}{0.419100in}}%
\pgfpathlineto{\pgfqpoint{3.139495in}{0.419100in}}%
\pgfpathlineto{\pgfqpoint{3.142630in}{0.419100in}}%
\pgfpathlineto{\pgfqpoint{3.145764in}{0.419100in}}%
\pgfpathlineto{\pgfqpoint{3.148899in}{0.419100in}}%
\pgfpathlineto{\pgfqpoint{3.152034in}{0.419100in}}%
\pgfpathlineto{\pgfqpoint{3.155169in}{0.419100in}}%
\pgfpathlineto{\pgfqpoint{3.158303in}{0.419100in}}%
\pgfpathlineto{\pgfqpoint{3.161438in}{0.419100in}}%
\pgfpathlineto{\pgfqpoint{3.164573in}{0.419100in}}%
\pgfpathlineto{\pgfqpoint{3.167708in}{0.419100in}}%
\pgfpathlineto{\pgfqpoint{3.170842in}{0.419100in}}%
\pgfpathlineto{\pgfqpoint{3.173977in}{0.419100in}}%
\pgfpathlineto{\pgfqpoint{3.177112in}{0.419100in}}%
\pgfpathlineto{\pgfqpoint{3.180247in}{0.419100in}}%
\pgfpathlineto{\pgfqpoint{3.183381in}{0.419100in}}%
\pgfpathlineto{\pgfqpoint{3.186516in}{0.419100in}}%
\pgfpathlineto{\pgfqpoint{3.189651in}{0.419100in}}%
\pgfpathlineto{\pgfqpoint{3.192786in}{0.419100in}}%
\pgfpathlineto{\pgfqpoint{3.195920in}{0.419100in}}%
\pgfpathlineto{\pgfqpoint{3.199055in}{0.419100in}}%
\pgfpathlineto{\pgfqpoint{3.202190in}{0.419100in}}%
\pgfpathlineto{\pgfqpoint{3.205325in}{0.419100in}}%
\pgfpathlineto{\pgfqpoint{3.208459in}{0.419100in}}%
\pgfpathlineto{\pgfqpoint{3.211594in}{0.419100in}}%
\pgfpathlineto{\pgfqpoint{3.214729in}{0.419100in}}%
\pgfpathlineto{\pgfqpoint{3.217864in}{0.419100in}}%
\pgfpathlineto{\pgfqpoint{3.220998in}{0.419100in}}%
\pgfpathlineto{\pgfqpoint{3.224133in}{0.419100in}}%
\pgfpathlineto{\pgfqpoint{3.227268in}{0.419100in}}%
\pgfpathlineto{\pgfqpoint{3.230402in}{0.419100in}}%
\pgfpathlineto{\pgfqpoint{3.233537in}{0.419100in}}%
\pgfpathlineto{\pgfqpoint{3.236672in}{0.419100in}}%
\pgfpathlineto{\pgfqpoint{3.239807in}{0.419100in}}%
\pgfpathlineto{\pgfqpoint{3.242941in}{0.419100in}}%
\pgfpathlineto{\pgfqpoint{3.246076in}{0.419100in}}%
\pgfpathlineto{\pgfqpoint{3.249211in}{0.419100in}}%
\pgfpathlineto{\pgfqpoint{3.252346in}{0.419100in}}%
\pgfpathlineto{\pgfqpoint{3.255480in}{0.419100in}}%
\pgfpathlineto{\pgfqpoint{3.258615in}{0.419100in}}%
\pgfpathlineto{\pgfqpoint{3.261750in}{0.419100in}}%
\pgfpathlineto{\pgfqpoint{3.264885in}{0.419100in}}%
\pgfpathlineto{\pgfqpoint{3.268019in}{0.419100in}}%
\pgfpathlineto{\pgfqpoint{3.271154in}{0.419100in}}%
\pgfpathlineto{\pgfqpoint{3.274289in}{0.419100in}}%
\pgfpathlineto{\pgfqpoint{3.277424in}{0.419100in}}%
\pgfpathlineto{\pgfqpoint{3.280558in}{0.419100in}}%
\pgfpathlineto{\pgfqpoint{3.283693in}{0.419100in}}%
\pgfpathlineto{\pgfqpoint{3.286828in}{0.419100in}}%
\pgfpathlineto{\pgfqpoint{3.289963in}{0.419100in}}%
\pgfpathlineto{\pgfqpoint{3.293097in}{0.419100in}}%
\pgfpathlineto{\pgfqpoint{3.296232in}{0.419100in}}%
\pgfpathlineto{\pgfqpoint{3.299367in}{0.419100in}}%
\pgfpathlineto{\pgfqpoint{3.302502in}{0.419100in}}%
\pgfpathlineto{\pgfqpoint{3.305636in}{0.419100in}}%
\pgfpathlineto{\pgfqpoint{3.308771in}{0.419100in}}%
\pgfpathlineto{\pgfqpoint{3.311906in}{0.419100in}}%
\pgfpathlineto{\pgfqpoint{3.315041in}{0.419100in}}%
\pgfpathlineto{\pgfqpoint{3.318175in}{0.419100in}}%
\pgfpathlineto{\pgfqpoint{3.321310in}{0.419100in}}%
\pgfpathlineto{\pgfqpoint{3.324445in}{0.419100in}}%
\pgfpathlineto{\pgfqpoint{3.327580in}{0.419100in}}%
\pgfpathlineto{\pgfqpoint{3.330714in}{0.419100in}}%
\pgfpathlineto{\pgfqpoint{3.333849in}{0.419100in}}%
\pgfpathlineto{\pgfqpoint{3.336984in}{0.419100in}}%
\pgfpathlineto{\pgfqpoint{3.340118in}{0.419100in}}%
\pgfpathlineto{\pgfqpoint{3.343253in}{0.419100in}}%
\pgfpathlineto{\pgfqpoint{3.346388in}{0.419100in}}%
\pgfpathlineto{\pgfqpoint{3.349523in}{0.419100in}}%
\pgfpathlineto{\pgfqpoint{3.352657in}{0.419100in}}%
\pgfpathlineto{\pgfqpoint{3.355792in}{0.419100in}}%
\pgfpathlineto{\pgfqpoint{3.358927in}{0.419100in}}%
\pgfpathlineto{\pgfqpoint{3.362062in}{0.419100in}}%
\pgfpathlineto{\pgfqpoint{3.365196in}{0.419100in}}%
\pgfpathlineto{\pgfqpoint{3.368331in}{0.419100in}}%
\pgfpathlineto{\pgfqpoint{3.371466in}{0.419100in}}%
\pgfpathlineto{\pgfqpoint{3.374601in}{0.419100in}}%
\pgfpathlineto{\pgfqpoint{3.377735in}{0.419100in}}%
\pgfpathlineto{\pgfqpoint{3.380870in}{0.419100in}}%
\pgfpathlineto{\pgfqpoint{3.384005in}{0.419100in}}%
\pgfpathlineto{\pgfqpoint{3.387140in}{0.419100in}}%
\pgfpathlineto{\pgfqpoint{3.390274in}{0.419100in}}%
\pgfpathlineto{\pgfqpoint{3.393409in}{0.419100in}}%
\pgfpathlineto{\pgfqpoint{3.393409in}{0.422852in}}%
\pgfpathlineto{\pgfqpoint{3.393409in}{0.426603in}}%
\pgfpathlineto{\pgfqpoint{3.393409in}{0.430355in}}%
\pgfpathlineto{\pgfqpoint{3.393409in}{0.434106in}}%
\pgfpathlineto{\pgfqpoint{3.393409in}{0.437858in}}%
\pgfpathlineto{\pgfqpoint{3.393409in}{0.441609in}}%
\pgfpathlineto{\pgfqpoint{3.393409in}{0.445361in}}%
\pgfpathlineto{\pgfqpoint{3.393409in}{0.449112in}}%
\pgfpathlineto{\pgfqpoint{3.393409in}{0.452864in}}%
\pgfpathlineto{\pgfqpoint{3.393409in}{0.456615in}}%
\pgfpathlineto{\pgfqpoint{3.393409in}{0.460367in}}%
\pgfpathlineto{\pgfqpoint{3.393409in}{0.464118in}}%
\pgfpathlineto{\pgfqpoint{3.393409in}{0.467870in}}%
\pgfpathlineto{\pgfqpoint{3.393409in}{0.471621in}}%
\pgfpathlineto{\pgfqpoint{3.393409in}{0.475373in}}%
\pgfpathlineto{\pgfqpoint{3.393409in}{0.479125in}}%
\pgfpathlineto{\pgfqpoint{3.393409in}{0.482876in}}%
\pgfpathlineto{\pgfqpoint{3.393409in}{0.486628in}}%
\pgfpathlineto{\pgfqpoint{3.393409in}{0.490379in}}%
\pgfpathlineto{\pgfqpoint{3.393409in}{0.494131in}}%
\pgfpathlineto{\pgfqpoint{3.393409in}{0.497882in}}%
\pgfpathlineto{\pgfqpoint{3.393409in}{0.501634in}}%
\pgfpathlineto{\pgfqpoint{3.393409in}{0.505385in}}%
\pgfpathlineto{\pgfqpoint{3.393409in}{0.509137in}}%
\pgfpathlineto{\pgfqpoint{3.393409in}{0.512888in}}%
\pgfpathlineto{\pgfqpoint{3.393409in}{0.516640in}}%
\pgfpathlineto{\pgfqpoint{3.393409in}{0.520391in}}%
\pgfpathlineto{\pgfqpoint{3.393409in}{0.524143in}}%
\pgfpathlineto{\pgfqpoint{3.393409in}{0.527895in}}%
\pgfpathlineto{\pgfqpoint{3.393409in}{0.531646in}}%
\pgfpathlineto{\pgfqpoint{3.393409in}{0.535398in}}%
\pgfpathlineto{\pgfqpoint{3.393409in}{0.539149in}}%
\pgfpathlineto{\pgfqpoint{3.393409in}{0.542901in}}%
\pgfpathlineto{\pgfqpoint{3.393409in}{0.546652in}}%
\pgfpathlineto{\pgfqpoint{3.393409in}{0.550404in}}%
\pgfpathlineto{\pgfqpoint{3.393409in}{0.554155in}}%
\pgfpathlineto{\pgfqpoint{3.393409in}{0.557907in}}%
\pgfpathlineto{\pgfqpoint{3.393409in}{0.561658in}}%
\pgfpathlineto{\pgfqpoint{3.393409in}{0.565410in}}%
\pgfpathlineto{\pgfqpoint{3.393409in}{0.569161in}}%
\pgfpathlineto{\pgfqpoint{3.393409in}{0.572913in}}%
\pgfpathlineto{\pgfqpoint{3.393409in}{0.576664in}}%
\pgfpathlineto{\pgfqpoint{3.393409in}{0.580416in}}%
\pgfpathlineto{\pgfqpoint{3.393409in}{0.584168in}}%
\pgfpathlineto{\pgfqpoint{3.393409in}{0.587919in}}%
\pgfpathlineto{\pgfqpoint{3.393409in}{0.591671in}}%
\pgfpathlineto{\pgfqpoint{3.393409in}{0.595422in}}%
\pgfpathlineto{\pgfqpoint{3.393409in}{0.599174in}}%
\pgfpathlineto{\pgfqpoint{3.393409in}{0.602925in}}%
\pgfpathlineto{\pgfqpoint{3.393409in}{0.606677in}}%
\pgfpathlineto{\pgfqpoint{3.393409in}{0.610428in}}%
\pgfpathlineto{\pgfqpoint{3.393409in}{0.614180in}}%
\pgfpathlineto{\pgfqpoint{3.393409in}{0.617931in}}%
\pgfpathlineto{\pgfqpoint{3.393409in}{0.621683in}}%
\pgfpathlineto{\pgfqpoint{3.393409in}{0.625434in}}%
\pgfpathlineto{\pgfqpoint{3.393409in}{0.629186in}}%
\pgfpathlineto{\pgfqpoint{3.393409in}{0.632937in}}%
\pgfpathlineto{\pgfqpoint{3.393409in}{0.636689in}}%
\pgfpathlineto{\pgfqpoint{3.393409in}{0.640441in}}%
\pgfpathlineto{\pgfqpoint{3.393409in}{0.644192in}}%
\pgfpathlineto{\pgfqpoint{3.393409in}{0.647944in}}%
\pgfpathlineto{\pgfqpoint{3.393409in}{0.651695in}}%
\pgfpathlineto{\pgfqpoint{3.393409in}{0.655447in}}%
\pgfpathlineto{\pgfqpoint{3.393409in}{0.659198in}}%
\pgfpathlineto{\pgfqpoint{3.393409in}{0.662950in}}%
\pgfpathlineto{\pgfqpoint{3.393409in}{0.666701in}}%
\pgfpathlineto{\pgfqpoint{3.393409in}{0.670453in}}%
\pgfpathlineto{\pgfqpoint{3.393409in}{0.674204in}}%
\pgfpathlineto{\pgfqpoint{3.393409in}{0.677956in}}%
\pgfpathlineto{\pgfqpoint{3.393409in}{0.681707in}}%
\pgfpathlineto{\pgfqpoint{3.393409in}{0.685459in}}%
\pgfpathlineto{\pgfqpoint{3.393409in}{0.689210in}}%
\pgfpathlineto{\pgfqpoint{3.393409in}{0.692962in}}%
\pgfpathlineto{\pgfqpoint{3.393409in}{0.696714in}}%
\pgfpathlineto{\pgfqpoint{3.393409in}{0.700465in}}%
\pgfpathlineto{\pgfqpoint{3.393409in}{0.704217in}}%
\pgfpathlineto{\pgfqpoint{3.393409in}{0.707968in}}%
\pgfpathlineto{\pgfqpoint{3.393409in}{0.711720in}}%
\pgfpathlineto{\pgfqpoint{3.393409in}{0.715471in}}%
\pgfpathlineto{\pgfqpoint{3.393409in}{0.719223in}}%
\pgfpathlineto{\pgfqpoint{3.393409in}{0.722974in}}%
\pgfpathlineto{\pgfqpoint{3.393409in}{0.726726in}}%
\pgfpathlineto{\pgfqpoint{3.393409in}{0.730477in}}%
\pgfpathlineto{\pgfqpoint{3.393409in}{0.734229in}}%
\pgfpathlineto{\pgfqpoint{3.393409in}{0.737980in}}%
\pgfpathlineto{\pgfqpoint{3.393409in}{0.741732in}}%
\pgfpathlineto{\pgfqpoint{3.393409in}{0.745484in}}%
\pgfpathlineto{\pgfqpoint{3.393409in}{0.749235in}}%
\pgfpathlineto{\pgfqpoint{3.393409in}{0.752987in}}%
\pgfpathlineto{\pgfqpoint{3.393409in}{0.756738in}}%
\pgfpathlineto{\pgfqpoint{3.393409in}{0.760490in}}%
\pgfpathlineto{\pgfqpoint{3.393409in}{0.764241in}}%
\pgfpathlineto{\pgfqpoint{3.393409in}{0.767993in}}%
\pgfpathlineto{\pgfqpoint{3.393409in}{0.771744in}}%
\pgfpathlineto{\pgfqpoint{3.393409in}{0.775496in}}%
\pgfpathlineto{\pgfqpoint{3.393409in}{0.779247in}}%
\pgfpathlineto{\pgfqpoint{3.393409in}{0.782999in}}%
\pgfpathlineto{\pgfqpoint{3.393409in}{0.786750in}}%
\pgfpathlineto{\pgfqpoint{3.393409in}{0.790502in}}%
\pgfpathlineto{\pgfqpoint{3.393409in}{0.794253in}}%
\pgfpathlineto{\pgfqpoint{3.393409in}{0.798005in}}%
\pgfpathlineto{\pgfqpoint{3.393409in}{0.801757in}}%
\pgfpathlineto{\pgfqpoint{3.393409in}{0.805508in}}%
\pgfpathlineto{\pgfqpoint{3.393409in}{0.809260in}}%
\pgfpathlineto{\pgfqpoint{3.393409in}{0.813011in}}%
\pgfpathlineto{\pgfqpoint{3.393409in}{0.816763in}}%
\pgfpathlineto{\pgfqpoint{3.393409in}{0.820514in}}%
\pgfpathlineto{\pgfqpoint{3.393409in}{0.824266in}}%
\pgfpathlineto{\pgfqpoint{3.393409in}{0.828017in}}%
\pgfpathlineto{\pgfqpoint{3.393409in}{0.831769in}}%
\pgfpathlineto{\pgfqpoint{3.393409in}{0.835520in}}%
\pgfpathlineto{\pgfqpoint{3.393409in}{0.839272in}}%
\pgfpathlineto{\pgfqpoint{3.393409in}{0.843023in}}%
\pgfpathlineto{\pgfqpoint{3.393409in}{0.846775in}}%
\pgfpathlineto{\pgfqpoint{3.393409in}{0.850526in}}%
\pgfpathlineto{\pgfqpoint{3.393409in}{0.854278in}}%
\pgfpathlineto{\pgfqpoint{3.393409in}{0.858030in}}%
\pgfpathlineto{\pgfqpoint{3.393409in}{0.861781in}}%
\pgfpathlineto{\pgfqpoint{3.393409in}{0.865533in}}%
\pgfpathlineto{\pgfqpoint{3.393409in}{0.869284in}}%
\pgfpathlineto{\pgfqpoint{3.393409in}{0.873036in}}%
\pgfpathlineto{\pgfqpoint{3.393409in}{0.876787in}}%
\pgfpathlineto{\pgfqpoint{3.393409in}{0.880539in}}%
\pgfpathlineto{\pgfqpoint{3.393409in}{0.884290in}}%
\pgfpathlineto{\pgfqpoint{3.393409in}{0.888042in}}%
\pgfpathlineto{\pgfqpoint{3.393409in}{0.891793in}}%
\pgfpathlineto{\pgfqpoint{3.393409in}{0.895545in}}%
\pgfpathlineto{\pgfqpoint{3.393409in}{0.899296in}}%
\pgfpathlineto{\pgfqpoint{3.393409in}{0.903048in}}%
\pgfpathlineto{\pgfqpoint{3.393409in}{0.906799in}}%
\pgfpathlineto{\pgfqpoint{3.393409in}{0.910551in}}%
\pgfpathlineto{\pgfqpoint{3.393409in}{0.914303in}}%
\pgfpathlineto{\pgfqpoint{3.393409in}{0.918054in}}%
\pgfpathlineto{\pgfqpoint{3.393409in}{0.921806in}}%
\pgfpathlineto{\pgfqpoint{3.393409in}{0.925557in}}%
\pgfpathlineto{\pgfqpoint{3.393409in}{0.929309in}}%
\pgfpathlineto{\pgfqpoint{3.393409in}{0.933060in}}%
\pgfpathlineto{\pgfqpoint{3.393409in}{0.936812in}}%
\pgfpathlineto{\pgfqpoint{3.393409in}{0.940563in}}%
\pgfpathlineto{\pgfqpoint{3.393409in}{0.944315in}}%
\pgfpathlineto{\pgfqpoint{3.393409in}{0.948066in}}%
\pgfpathlineto{\pgfqpoint{3.393409in}{0.951818in}}%
\pgfpathlineto{\pgfqpoint{3.393409in}{0.955569in}}%
\pgfpathlineto{\pgfqpoint{3.393409in}{0.959321in}}%
\pgfpathlineto{\pgfqpoint{3.393409in}{0.963073in}}%
\pgfpathlineto{\pgfqpoint{3.393409in}{0.966824in}}%
\pgfpathlineto{\pgfqpoint{3.393409in}{0.970576in}}%
\pgfpathlineto{\pgfqpoint{3.393409in}{0.974327in}}%
\pgfpathlineto{\pgfqpoint{3.393409in}{0.978079in}}%
\pgfpathlineto{\pgfqpoint{3.393409in}{0.981830in}}%
\pgfpathlineto{\pgfqpoint{3.393409in}{0.985582in}}%
\pgfpathlineto{\pgfqpoint{3.393409in}{0.989333in}}%
\pgfpathlineto{\pgfqpoint{3.393409in}{0.993085in}}%
\pgfpathlineto{\pgfqpoint{3.393409in}{0.996836in}}%
\pgfpathlineto{\pgfqpoint{3.393409in}{1.000588in}}%
\pgfpathlineto{\pgfqpoint{3.393409in}{1.004339in}}%
\pgfpathlineto{\pgfqpoint{3.393409in}{1.008091in}}%
\pgfpathlineto{\pgfqpoint{3.393409in}{1.011842in}}%
\pgfpathlineto{\pgfqpoint{3.393409in}{1.015594in}}%
\pgfpathlineto{\pgfqpoint{3.393409in}{1.019346in}}%
\pgfpathlineto{\pgfqpoint{3.393409in}{1.023097in}}%
\pgfpathlineto{\pgfqpoint{3.393409in}{1.026849in}}%
\pgfpathlineto{\pgfqpoint{3.393409in}{1.030600in}}%
\pgfpathlineto{\pgfqpoint{3.393409in}{1.034352in}}%
\pgfpathlineto{\pgfqpoint{3.393409in}{1.038103in}}%
\pgfpathlineto{\pgfqpoint{3.393409in}{1.041855in}}%
\pgfpathlineto{\pgfqpoint{3.393409in}{1.045606in}}%
\pgfpathlineto{\pgfqpoint{3.393409in}{1.049358in}}%
\pgfpathlineto{\pgfqpoint{3.393409in}{1.053109in}}%
\pgfpathlineto{\pgfqpoint{3.393409in}{1.056861in}}%
\pgfpathlineto{\pgfqpoint{3.393409in}{1.060612in}}%
\pgfpathlineto{\pgfqpoint{3.393409in}{1.064364in}}%
\pgfpathlineto{\pgfqpoint{3.393409in}{1.068115in}}%
\pgfpathlineto{\pgfqpoint{3.393409in}{1.071867in}}%
\pgfpathlineto{\pgfqpoint{3.393409in}{1.075619in}}%
\pgfpathlineto{\pgfqpoint{3.393409in}{1.079370in}}%
\pgfpathlineto{\pgfqpoint{3.393409in}{1.083122in}}%
\pgfpathlineto{\pgfqpoint{3.393409in}{1.086873in}}%
\pgfpathlineto{\pgfqpoint{3.393409in}{1.090625in}}%
\pgfpathlineto{\pgfqpoint{3.393409in}{1.094376in}}%
\pgfpathlineto{\pgfqpoint{3.393409in}{1.098128in}}%
\pgfpathlineto{\pgfqpoint{3.393409in}{1.101879in}}%
\pgfpathlineto{\pgfqpoint{3.393409in}{1.105631in}}%
\pgfpathlineto{\pgfqpoint{3.393409in}{1.109382in}}%
\pgfpathlineto{\pgfqpoint{3.393409in}{1.113134in}}%
\pgfpathlineto{\pgfqpoint{3.393409in}{1.116885in}}%
\pgfpathlineto{\pgfqpoint{3.393409in}{1.120637in}}%
\pgfpathlineto{\pgfqpoint{3.393409in}{1.124388in}}%
\pgfpathlineto{\pgfqpoint{3.393409in}{1.128140in}}%
\pgfpathlineto{\pgfqpoint{3.393409in}{1.131892in}}%
\pgfpathlineto{\pgfqpoint{3.393409in}{1.135643in}}%
\pgfpathlineto{\pgfqpoint{3.393409in}{1.139395in}}%
\pgfpathlineto{\pgfqpoint{3.393409in}{1.143146in}}%
\pgfpathlineto{\pgfqpoint{3.393409in}{1.146898in}}%
\pgfpathlineto{\pgfqpoint{3.393409in}{1.150649in}}%
\pgfpathlineto{\pgfqpoint{3.393409in}{1.154401in}}%
\pgfpathlineto{\pgfqpoint{3.393409in}{1.158152in}}%
\pgfpathlineto{\pgfqpoint{3.393409in}{1.161904in}}%
\pgfpathlineto{\pgfqpoint{3.393409in}{1.165655in}}%
\pgfpathlineto{\pgfqpoint{3.393409in}{1.169407in}}%
\pgfpathlineto{\pgfqpoint{3.393409in}{1.173158in}}%
\pgfpathlineto{\pgfqpoint{3.393409in}{1.176910in}}%
\pgfpathlineto{\pgfqpoint{3.393409in}{1.180662in}}%
\pgfpathlineto{\pgfqpoint{3.393409in}{1.184413in}}%
\pgfpathlineto{\pgfqpoint{3.393409in}{1.188165in}}%
\pgfpathlineto{\pgfqpoint{3.393409in}{1.191916in}}%
\pgfpathlineto{\pgfqpoint{3.393409in}{1.195668in}}%
\pgfpathlineto{\pgfqpoint{3.393409in}{1.199419in}}%
\pgfpathlineto{\pgfqpoint{3.393409in}{1.203171in}}%
\pgfpathlineto{\pgfqpoint{3.393409in}{1.206922in}}%
\pgfpathlineto{\pgfqpoint{3.393409in}{1.210674in}}%
\pgfpathlineto{\pgfqpoint{3.393409in}{1.214425in}}%
\pgfpathlineto{\pgfqpoint{3.393409in}{1.218177in}}%
\pgfpathlineto{\pgfqpoint{3.393409in}{1.221928in}}%
\pgfpathlineto{\pgfqpoint{3.393409in}{1.225680in}}%
\pgfpathlineto{\pgfqpoint{3.393409in}{1.229431in}}%
\pgfpathlineto{\pgfqpoint{3.393409in}{1.233183in}}%
\pgfpathlineto{\pgfqpoint{3.393409in}{1.236935in}}%
\pgfpathlineto{\pgfqpoint{3.393409in}{1.240686in}}%
\pgfpathlineto{\pgfqpoint{3.393409in}{1.244438in}}%
\pgfpathlineto{\pgfqpoint{3.393409in}{1.248189in}}%
\pgfpathlineto{\pgfqpoint{3.393409in}{1.251941in}}%
\pgfpathlineto{\pgfqpoint{3.393409in}{1.255692in}}%
\pgfpathlineto{\pgfqpoint{3.393409in}{1.259444in}}%
\pgfpathlineto{\pgfqpoint{3.393409in}{1.263195in}}%
\pgfpathlineto{\pgfqpoint{3.393409in}{1.266947in}}%
\pgfpathlineto{\pgfqpoint{3.393409in}{1.270698in}}%
\pgfpathlineto{\pgfqpoint{3.393409in}{1.274450in}}%
\pgfpathlineto{\pgfqpoint{3.393409in}{1.278201in}}%
\pgfpathlineto{\pgfqpoint{3.393409in}{1.281953in}}%
\pgfpathlineto{\pgfqpoint{3.393409in}{1.285704in}}%
\pgfpathlineto{\pgfqpoint{3.393409in}{1.289456in}}%
\pgfpathlineto{\pgfqpoint{3.393409in}{1.293208in}}%
\pgfpathlineto{\pgfqpoint{3.393409in}{1.296959in}}%
\pgfpathlineto{\pgfqpoint{3.393409in}{1.300711in}}%
\pgfpathlineto{\pgfqpoint{3.393409in}{1.304462in}}%
\pgfpathlineto{\pgfqpoint{3.393409in}{1.308214in}}%
\pgfpathlineto{\pgfqpoint{3.393409in}{1.311965in}}%
\pgfpathlineto{\pgfqpoint{3.393409in}{1.315717in}}%
\pgfpathlineto{\pgfqpoint{3.393409in}{1.319468in}}%
\pgfpathlineto{\pgfqpoint{3.393409in}{1.323220in}}%
\pgfpathlineto{\pgfqpoint{3.393409in}{1.326971in}}%
\pgfpathlineto{\pgfqpoint{3.393409in}{1.330723in}}%
\pgfpathlineto{\pgfqpoint{3.393409in}{1.334474in}}%
\pgfpathlineto{\pgfqpoint{3.393409in}{1.338226in}}%
\pgfpathlineto{\pgfqpoint{3.393409in}{1.341977in}}%
\pgfpathlineto{\pgfqpoint{3.393409in}{1.345729in}}%
\pgfpathlineto{\pgfqpoint{3.393409in}{1.349481in}}%
\pgfpathlineto{\pgfqpoint{3.393409in}{1.353232in}}%
\pgfpathlineto{\pgfqpoint{3.393409in}{1.356984in}}%
\pgfpathlineto{\pgfqpoint{3.393409in}{1.360735in}}%
\pgfpathlineto{\pgfqpoint{3.393409in}{1.364487in}}%
\pgfpathlineto{\pgfqpoint{3.393409in}{1.368238in}}%
\pgfpathlineto{\pgfqpoint{3.393409in}{1.371990in}}%
\pgfpathlineto{\pgfqpoint{3.393409in}{1.375741in}}%
\pgfpathlineto{\pgfqpoint{3.393409in}{1.379493in}}%
\pgfpathlineto{\pgfqpoint{3.393409in}{1.383244in}}%
\pgfpathlineto{\pgfqpoint{3.393409in}{1.386996in}}%
\pgfpathlineto{\pgfqpoint{3.393409in}{1.390747in}}%
\pgfpathlineto{\pgfqpoint{3.393409in}{1.394499in}}%
\pgfpathlineto{\pgfqpoint{3.393409in}{1.398251in}}%
\pgfpathlineto{\pgfqpoint{3.393409in}{1.402002in}}%
\pgfpathlineto{\pgfqpoint{3.393409in}{1.405754in}}%
\pgfpathlineto{\pgfqpoint{3.393409in}{1.409505in}}%
\pgfpathlineto{\pgfqpoint{3.393409in}{1.413257in}}%
\pgfpathlineto{\pgfqpoint{3.393409in}{1.417008in}}%
\pgfpathlineto{\pgfqpoint{3.393409in}{1.420760in}}%
\pgfpathlineto{\pgfqpoint{3.393409in}{1.424511in}}%
\pgfpathlineto{\pgfqpoint{3.393409in}{1.428263in}}%
\pgfpathlineto{\pgfqpoint{3.393409in}{1.432014in}}%
\pgfpathlineto{\pgfqpoint{3.393409in}{1.435766in}}%
\pgfpathlineto{\pgfqpoint{3.393409in}{1.439517in}}%
\pgfpathlineto{\pgfqpoint{3.393409in}{1.443269in}}%
\pgfpathlineto{\pgfqpoint{3.393409in}{1.447020in}}%
\pgfpathlineto{\pgfqpoint{3.393409in}{1.450772in}}%
\pgfpathlineto{\pgfqpoint{3.393409in}{1.454524in}}%
\pgfpathlineto{\pgfqpoint{3.393409in}{1.458275in}}%
\pgfpathlineto{\pgfqpoint{3.393409in}{1.462027in}}%
\pgfpathlineto{\pgfqpoint{3.393409in}{1.465778in}}%
\pgfpathlineto{\pgfqpoint{3.393409in}{1.469530in}}%
\pgfpathlineto{\pgfqpoint{3.393409in}{1.473281in}}%
\pgfpathlineto{\pgfqpoint{3.393409in}{1.477033in}}%
\pgfpathlineto{\pgfqpoint{3.393409in}{1.480784in}}%
\pgfpathlineto{\pgfqpoint{3.393409in}{1.484536in}}%
\pgfpathlineto{\pgfqpoint{3.393409in}{1.488287in}}%
\pgfpathlineto{\pgfqpoint{3.393409in}{1.492039in}}%
\pgfpathlineto{\pgfqpoint{3.393409in}{1.495790in}}%
\pgfpathlineto{\pgfqpoint{3.393409in}{1.499542in}}%
\pgfpathlineto{\pgfqpoint{3.393409in}{1.503293in}}%
\pgfpathlineto{\pgfqpoint{3.393409in}{1.507045in}}%
\pgfpathlineto{\pgfqpoint{3.393409in}{1.510797in}}%
\pgfpathlineto{\pgfqpoint{3.393409in}{1.514548in}}%
\pgfpathlineto{\pgfqpoint{3.393409in}{1.518300in}}%
\pgfpathlineto{\pgfqpoint{3.393409in}{1.522051in}}%
\pgfpathlineto{\pgfqpoint{3.393409in}{1.525803in}}%
\pgfpathlineto{\pgfqpoint{3.393409in}{1.529554in}}%
\pgfpathlineto{\pgfqpoint{3.393409in}{1.533306in}}%
\pgfpathlineto{\pgfqpoint{3.393409in}{1.537057in}}%
\pgfpathlineto{\pgfqpoint{3.393409in}{1.540809in}}%
\pgfpathlineto{\pgfqpoint{3.393409in}{1.544560in}}%
\pgfpathlineto{\pgfqpoint{3.393409in}{1.548312in}}%
\pgfpathlineto{\pgfqpoint{3.393409in}{1.552063in}}%
\pgfpathlineto{\pgfqpoint{3.393409in}{1.555815in}}%
\pgfpathlineto{\pgfqpoint{3.393409in}{1.559566in}}%
\pgfpathlineto{\pgfqpoint{3.393409in}{1.563318in}}%
\pgfpathlineto{\pgfqpoint{3.393409in}{1.567070in}}%
\pgfpathlineto{\pgfqpoint{3.393409in}{1.570821in}}%
\pgfpathlineto{\pgfqpoint{3.393409in}{1.574573in}}%
\pgfpathlineto{\pgfqpoint{3.393409in}{1.578324in}}%
\pgfpathlineto{\pgfqpoint{3.393409in}{1.582076in}}%
\pgfpathlineto{\pgfqpoint{3.393409in}{1.585827in}}%
\pgfpathlineto{\pgfqpoint{3.393409in}{1.589579in}}%
\pgfpathlineto{\pgfqpoint{3.393409in}{1.593330in}}%
\pgfpathlineto{\pgfqpoint{3.393409in}{1.597082in}}%
\pgfpathlineto{\pgfqpoint{3.393409in}{1.600833in}}%
\pgfpathlineto{\pgfqpoint{3.393409in}{1.604585in}}%
\pgfpathlineto{\pgfqpoint{3.393409in}{1.608336in}}%
\pgfpathlineto{\pgfqpoint{3.393409in}{1.612088in}}%
\pgfpathlineto{\pgfqpoint{3.393409in}{1.615840in}}%
\pgfpathlineto{\pgfqpoint{3.393409in}{1.619591in}}%
\pgfpathlineto{\pgfqpoint{3.393409in}{1.623343in}}%
\pgfpathlineto{\pgfqpoint{3.393409in}{1.627094in}}%
\pgfpathlineto{\pgfqpoint{3.393409in}{1.630846in}}%
\pgfpathlineto{\pgfqpoint{3.393409in}{1.634597in}}%
\pgfpathlineto{\pgfqpoint{3.393409in}{1.638349in}}%
\pgfpathlineto{\pgfqpoint{3.393409in}{1.642100in}}%
\pgfpathlineto{\pgfqpoint{3.393409in}{1.645852in}}%
\pgfpathlineto{\pgfqpoint{3.393409in}{1.649603in}}%
\pgfpathlineto{\pgfqpoint{3.393409in}{1.653355in}}%
\pgfpathlineto{\pgfqpoint{3.393409in}{1.657106in}}%
\pgfpathlineto{\pgfqpoint{3.393409in}{1.660858in}}%
\pgfpathlineto{\pgfqpoint{3.393409in}{1.664609in}}%
\pgfpathlineto{\pgfqpoint{3.393409in}{1.668361in}}%
\pgfpathlineto{\pgfqpoint{3.393409in}{1.672113in}}%
\pgfpathlineto{\pgfqpoint{3.393409in}{1.675864in}}%
\pgfpathlineto{\pgfqpoint{3.393409in}{1.679616in}}%
\pgfpathlineto{\pgfqpoint{3.393409in}{1.683367in}}%
\pgfpathlineto{\pgfqpoint{3.393409in}{1.687119in}}%
\pgfpathlineto{\pgfqpoint{3.393409in}{1.690870in}}%
\pgfpathlineto{\pgfqpoint{3.393409in}{1.694622in}}%
\pgfpathlineto{\pgfqpoint{3.393409in}{1.698373in}}%
\pgfpathlineto{\pgfqpoint{3.393409in}{1.702125in}}%
\pgfpathlineto{\pgfqpoint{3.393409in}{1.705876in}}%
\pgfpathlineto{\pgfqpoint{3.393409in}{1.709628in}}%
\pgfpathlineto{\pgfqpoint{3.393409in}{1.713379in}}%
\pgfpathlineto{\pgfqpoint{3.393409in}{1.717131in}}%
\pgfpathlineto{\pgfqpoint{3.393409in}{1.720882in}}%
\pgfpathlineto{\pgfqpoint{3.393409in}{1.724634in}}%
\pgfpathlineto{\pgfqpoint{3.393409in}{1.728386in}}%
\pgfpathlineto{\pgfqpoint{3.393409in}{1.732137in}}%
\pgfpathlineto{\pgfqpoint{3.393409in}{1.735889in}}%
\pgfpathlineto{\pgfqpoint{3.393409in}{1.739640in}}%
\pgfpathlineto{\pgfqpoint{3.393409in}{1.743392in}}%
\pgfpathlineto{\pgfqpoint{3.393409in}{1.747143in}}%
\pgfpathlineto{\pgfqpoint{3.393409in}{1.750895in}}%
\pgfpathlineto{\pgfqpoint{3.393409in}{1.754646in}}%
\pgfpathlineto{\pgfqpoint{3.393409in}{1.758398in}}%
\pgfpathlineto{\pgfqpoint{3.393409in}{1.762149in}}%
\pgfpathlineto{\pgfqpoint{3.393409in}{1.765901in}}%
\pgfpathlineto{\pgfqpoint{3.393409in}{1.769652in}}%
\pgfpathlineto{\pgfqpoint{3.393409in}{1.773404in}}%
\pgfpathlineto{\pgfqpoint{3.393409in}{1.777155in}}%
\pgfpathlineto{\pgfqpoint{3.393409in}{1.780907in}}%
\pgfpathlineto{\pgfqpoint{3.393409in}{1.784659in}}%
\pgfpathlineto{\pgfqpoint{3.393409in}{1.788410in}}%
\pgfpathlineto{\pgfqpoint{3.393409in}{1.792162in}}%
\pgfpathlineto{\pgfqpoint{3.393409in}{1.795913in}}%
\pgfpathlineto{\pgfqpoint{3.393409in}{1.799665in}}%
\pgfpathlineto{\pgfqpoint{3.393409in}{1.803416in}}%
\pgfpathlineto{\pgfqpoint{3.393409in}{1.807168in}}%
\pgfpathlineto{\pgfqpoint{3.393409in}{1.810919in}}%
\pgfpathlineto{\pgfqpoint{3.393409in}{1.814671in}}%
\pgfpathlineto{\pgfqpoint{3.393409in}{1.818422in}}%
\pgfpathlineto{\pgfqpoint{3.393409in}{1.822174in}}%
\pgfpathlineto{\pgfqpoint{3.393409in}{1.825925in}}%
\pgfpathlineto{\pgfqpoint{3.393409in}{1.829677in}}%
\pgfpathlineto{\pgfqpoint{3.393409in}{1.833429in}}%
\pgfpathlineto{\pgfqpoint{3.393409in}{1.837180in}}%
\pgfpathlineto{\pgfqpoint{3.393409in}{1.840932in}}%
\pgfpathlineto{\pgfqpoint{3.393409in}{1.844683in}}%
\pgfpathlineto{\pgfqpoint{3.393409in}{1.848435in}}%
\pgfpathlineto{\pgfqpoint{3.393409in}{1.852186in}}%
\pgfpathlineto{\pgfqpoint{3.393409in}{1.855938in}}%
\pgfpathlineto{\pgfqpoint{3.393409in}{1.859689in}}%
\pgfpathlineto{\pgfqpoint{3.393409in}{1.863441in}}%
\pgfpathlineto{\pgfqpoint{3.393409in}{1.867192in}}%
\pgfpathlineto{\pgfqpoint{3.393409in}{1.870944in}}%
\pgfpathlineto{\pgfqpoint{3.393409in}{1.874695in}}%
\pgfpathlineto{\pgfqpoint{3.393409in}{1.878447in}}%
\pgfpathlineto{\pgfqpoint{3.393409in}{1.882198in}}%
\pgfpathlineto{\pgfqpoint{3.393409in}{1.885950in}}%
\pgfpathlineto{\pgfqpoint{3.393409in}{1.889702in}}%
\pgfpathlineto{\pgfqpoint{3.393409in}{1.893453in}}%
\pgfpathlineto{\pgfqpoint{3.393409in}{1.897205in}}%
\pgfpathlineto{\pgfqpoint{3.393409in}{1.900956in}}%
\pgfpathlineto{\pgfqpoint{3.393409in}{1.904708in}}%
\pgfpathlineto{\pgfqpoint{3.393409in}{1.908459in}}%
\pgfpathlineto{\pgfqpoint{3.393409in}{1.912211in}}%
\pgfpathlineto{\pgfqpoint{3.393409in}{1.915962in}}%
\pgfpathlineto{\pgfqpoint{3.393409in}{1.919714in}}%
\pgfpathlineto{\pgfqpoint{3.393409in}{1.923465in}}%
\pgfpathlineto{\pgfqpoint{3.393409in}{1.927217in}}%
\pgfpathlineto{\pgfqpoint{3.393409in}{1.930968in}}%
\pgfpathlineto{\pgfqpoint{3.393409in}{1.934720in}}%
\pgfpathlineto{\pgfqpoint{3.393409in}{1.938471in}}%
\pgfpathlineto{\pgfqpoint{3.393409in}{1.942223in}}%
\pgfpathlineto{\pgfqpoint{3.393409in}{1.945975in}}%
\pgfpathlineto{\pgfqpoint{3.393409in}{1.949726in}}%
\pgfpathlineto{\pgfqpoint{3.393409in}{1.953478in}}%
\pgfpathlineto{\pgfqpoint{3.393409in}{1.957229in}}%
\pgfpathlineto{\pgfqpoint{3.393409in}{1.960981in}}%
\pgfpathlineto{\pgfqpoint{3.393409in}{1.964732in}}%
\pgfpathlineto{\pgfqpoint{3.393409in}{1.968484in}}%
\pgfpathlineto{\pgfqpoint{3.393409in}{1.972235in}}%
\pgfpathlineto{\pgfqpoint{3.393409in}{1.975987in}}%
\pgfpathlineto{\pgfqpoint{3.393409in}{1.979738in}}%
\pgfpathlineto{\pgfqpoint{3.393409in}{1.983490in}}%
\pgfpathlineto{\pgfqpoint{3.393409in}{1.987241in}}%
\pgfpathlineto{\pgfqpoint{3.393409in}{1.990993in}}%
\pgfpathlineto{\pgfqpoint{3.393409in}{1.994745in}}%
\pgfpathlineto{\pgfqpoint{3.393409in}{1.998496in}}%
\pgfpathlineto{\pgfqpoint{3.393409in}{2.002248in}}%
\pgfpathlineto{\pgfqpoint{3.393409in}{2.005999in}}%
\pgfpathlineto{\pgfqpoint{3.393409in}{2.009751in}}%
\pgfpathlineto{\pgfqpoint{3.393409in}{2.013502in}}%
\pgfpathlineto{\pgfqpoint{3.393409in}{2.017254in}}%
\pgfpathlineto{\pgfqpoint{3.393409in}{2.021005in}}%
\pgfpathlineto{\pgfqpoint{3.393409in}{2.024757in}}%
\pgfpathlineto{\pgfqpoint{3.393409in}{2.028508in}}%
\pgfpathlineto{\pgfqpoint{3.393409in}{2.032260in}}%
\pgfpathlineto{\pgfqpoint{3.393409in}{2.036011in}}%
\pgfpathlineto{\pgfqpoint{3.393409in}{2.039763in}}%
\pgfpathlineto{\pgfqpoint{3.393409in}{2.043514in}}%
\pgfpathlineto{\pgfqpoint{3.393409in}{2.047266in}}%
\pgfpathlineto{\pgfqpoint{3.393409in}{2.051018in}}%
\pgfpathlineto{\pgfqpoint{3.393409in}{2.054769in}}%
\pgfpathlineto{\pgfqpoint{3.393409in}{2.058521in}}%
\pgfpathlineto{\pgfqpoint{3.393409in}{2.062272in}}%
\pgfpathlineto{\pgfqpoint{3.393409in}{2.066024in}}%
\pgfpathlineto{\pgfqpoint{3.393409in}{2.069775in}}%
\pgfpathlineto{\pgfqpoint{3.393409in}{2.073527in}}%
\pgfpathlineto{\pgfqpoint{3.393409in}{2.077278in}}%
\pgfpathlineto{\pgfqpoint{3.393409in}{2.081030in}}%
\pgfpathlineto{\pgfqpoint{3.393409in}{2.084781in}}%
\pgfpathlineto{\pgfqpoint{3.393409in}{2.088533in}}%
\pgfpathlineto{\pgfqpoint{3.393409in}{2.092284in}}%
\pgfpathlineto{\pgfqpoint{3.393409in}{2.096036in}}%
\pgfpathlineto{\pgfqpoint{3.393409in}{2.099787in}}%
\pgfpathlineto{\pgfqpoint{3.393409in}{2.103539in}}%
\pgfpathlineto{\pgfqpoint{3.393409in}{2.107291in}}%
\pgfpathlineto{\pgfqpoint{3.393409in}{2.111042in}}%
\pgfpathlineto{\pgfqpoint{3.393409in}{2.114794in}}%
\pgfpathlineto{\pgfqpoint{3.393409in}{2.118545in}}%
\pgfpathlineto{\pgfqpoint{3.393409in}{2.122297in}}%
\pgfpathlineto{\pgfqpoint{3.393409in}{2.126048in}}%
\pgfpathlineto{\pgfqpoint{3.393409in}{2.129800in}}%
\pgfpathlineto{\pgfqpoint{3.393409in}{2.133551in}}%
\pgfpathlineto{\pgfqpoint{3.393409in}{2.137303in}}%
\pgfpathlineto{\pgfqpoint{3.393409in}{2.141054in}}%
\pgfpathlineto{\pgfqpoint{3.393409in}{2.144806in}}%
\pgfpathlineto{\pgfqpoint{3.393409in}{2.148557in}}%
\pgfpathlineto{\pgfqpoint{3.393409in}{2.152309in}}%
\pgfpathlineto{\pgfqpoint{3.393409in}{2.156060in}}%
\pgfpathlineto{\pgfqpoint{3.393409in}{2.159812in}}%
\pgfpathlineto{\pgfqpoint{3.393409in}{2.163564in}}%
\pgfpathlineto{\pgfqpoint{3.393409in}{2.167315in}}%
\pgfpathlineto{\pgfqpoint{3.393409in}{2.171067in}}%
\pgfpathlineto{\pgfqpoint{3.393409in}{2.174818in}}%
\pgfpathlineto{\pgfqpoint{3.393409in}{2.178570in}}%
\pgfpathlineto{\pgfqpoint{3.393409in}{2.182321in}}%
\pgfpathlineto{\pgfqpoint{3.393409in}{2.186073in}}%
\pgfpathlineto{\pgfqpoint{3.393409in}{2.189824in}}%
\pgfpathlineto{\pgfqpoint{3.393409in}{2.193576in}}%
\pgfpathlineto{\pgfqpoint{3.393409in}{2.197327in}}%
\pgfpathlineto{\pgfqpoint{3.393409in}{2.201079in}}%
\pgfpathlineto{\pgfqpoint{3.393409in}{2.204830in}}%
\pgfpathlineto{\pgfqpoint{3.393409in}{2.208582in}}%
\pgfpathlineto{\pgfqpoint{3.393409in}{2.212334in}}%
\pgfpathlineto{\pgfqpoint{3.393409in}{2.216085in}}%
\pgfpathlineto{\pgfqpoint{3.393409in}{2.219837in}}%
\pgfpathlineto{\pgfqpoint{3.393409in}{2.223588in}}%
\pgfpathlineto{\pgfqpoint{3.393409in}{2.227340in}}%
\pgfpathlineto{\pgfqpoint{3.393409in}{2.231091in}}%
\pgfpathlineto{\pgfqpoint{3.393409in}{2.234843in}}%
\pgfpathlineto{\pgfqpoint{3.393409in}{2.238594in}}%
\pgfpathlineto{\pgfqpoint{3.393409in}{2.242346in}}%
\pgfpathlineto{\pgfqpoint{3.393409in}{2.246097in}}%
\pgfpathlineto{\pgfqpoint{3.393409in}{2.249849in}}%
\pgfpathlineto{\pgfqpoint{3.393409in}{2.253600in}}%
\pgfpathlineto{\pgfqpoint{3.393409in}{2.257352in}}%
\pgfpathlineto{\pgfqpoint{3.393409in}{2.261103in}}%
\pgfpathlineto{\pgfqpoint{3.393409in}{2.264855in}}%
\pgfpathlineto{\pgfqpoint{3.393409in}{2.268607in}}%
\pgfpathlineto{\pgfqpoint{3.393409in}{2.272358in}}%
\pgfpathlineto{\pgfqpoint{3.393409in}{2.276110in}}%
\pgfpathlineto{\pgfqpoint{3.393409in}{2.279861in}}%
\pgfpathlineto{\pgfqpoint{3.393409in}{2.283613in}}%
\pgfpathlineto{\pgfqpoint{3.393409in}{2.287364in}}%
\pgfpathlineto{\pgfqpoint{3.393409in}{2.291116in}}%
\pgfpathlineto{\pgfqpoint{3.393409in}{2.294867in}}%
\pgfpathlineto{\pgfqpoint{3.393409in}{2.298619in}}%
\pgfpathlineto{\pgfqpoint{3.393409in}{2.302370in}}%
\pgfpathlineto{\pgfqpoint{3.393409in}{2.306122in}}%
\pgfpathlineto{\pgfqpoint{3.393409in}{2.309873in}}%
\pgfpathlineto{\pgfqpoint{3.393409in}{2.313625in}}%
\pgfpathlineto{\pgfqpoint{3.393409in}{2.317376in}}%
\pgfpathlineto{\pgfqpoint{3.393409in}{2.321128in}}%
\pgfpathlineto{\pgfqpoint{3.393409in}{2.324880in}}%
\pgfpathlineto{\pgfqpoint{3.393409in}{2.328631in}}%
\pgfpathlineto{\pgfqpoint{3.393409in}{2.332383in}}%
\pgfpathlineto{\pgfqpoint{3.393409in}{2.336134in}}%
\pgfpathlineto{\pgfqpoint{3.393409in}{2.339886in}}%
\pgfpathlineto{\pgfqpoint{3.393409in}{2.343637in}}%
\pgfpathlineto{\pgfqpoint{3.393409in}{2.347389in}}%
\pgfpathlineto{\pgfqpoint{3.393409in}{2.351140in}}%
\pgfpathlineto{\pgfqpoint{3.393409in}{2.354892in}}%
\pgfpathlineto{\pgfqpoint{3.393409in}{2.358643in}}%
\pgfpathlineto{\pgfqpoint{3.393409in}{2.362395in}}%
\pgfpathlineto{\pgfqpoint{3.393409in}{2.366146in}}%
\pgfpathlineto{\pgfqpoint{3.393409in}{2.369898in}}%
\pgfpathlineto{\pgfqpoint{3.393409in}{2.373649in}}%
\pgfpathlineto{\pgfqpoint{3.393409in}{2.377401in}}%
\pgfpathlineto{\pgfqpoint{3.393409in}{2.381153in}}%
\pgfpathlineto{\pgfqpoint{3.393409in}{2.384904in}}%
\pgfpathlineto{\pgfqpoint{3.393409in}{2.388656in}}%
\pgfpathlineto{\pgfqpoint{3.393409in}{2.392407in}}%
\pgfpathlineto{\pgfqpoint{3.393409in}{2.396159in}}%
\pgfpathlineto{\pgfqpoint{3.393409in}{2.399910in}}%
\pgfpathlineto{\pgfqpoint{3.393409in}{2.403662in}}%
\pgfpathlineto{\pgfqpoint{3.393409in}{2.407413in}}%
\pgfpathlineto{\pgfqpoint{3.393409in}{2.411165in}}%
\pgfpathlineto{\pgfqpoint{3.393409in}{2.414916in}}%
\pgfpathlineto{\pgfqpoint{3.393409in}{2.418668in}}%
\pgfpathlineto{\pgfqpoint{3.393409in}{2.422419in}}%
\pgfpathlineto{\pgfqpoint{3.393409in}{2.426171in}}%
\pgfpathlineto{\pgfqpoint{3.393409in}{2.429923in}}%
\pgfpathlineto{\pgfqpoint{3.393409in}{2.433674in}}%
\pgfpathlineto{\pgfqpoint{3.393409in}{2.437426in}}%
\pgfpathlineto{\pgfqpoint{3.393409in}{2.441177in}}%
\pgfpathlineto{\pgfqpoint{3.393409in}{2.444929in}}%
\pgfpathlineto{\pgfqpoint{3.393409in}{2.448680in}}%
\pgfpathlineto{\pgfqpoint{3.393409in}{2.452432in}}%
\pgfpathlineto{\pgfqpoint{3.393409in}{2.456183in}}%
\pgfpathlineto{\pgfqpoint{3.393409in}{2.459935in}}%
\pgfpathlineto{\pgfqpoint{3.393409in}{2.463686in}}%
\pgfpathlineto{\pgfqpoint{3.393409in}{2.467438in}}%
\pgfpathlineto{\pgfqpoint{3.393409in}{2.471189in}}%
\pgfpathlineto{\pgfqpoint{3.393409in}{2.474941in}}%
\pgfpathlineto{\pgfqpoint{3.393409in}{2.478692in}}%
\pgfpathlineto{\pgfqpoint{3.393409in}{2.482444in}}%
\pgfpathlineto{\pgfqpoint{3.393409in}{2.486196in}}%
\pgfpathlineto{\pgfqpoint{3.393409in}{2.489947in}}%
\pgfpathlineto{\pgfqpoint{3.393409in}{2.493699in}}%
\pgfpathlineto{\pgfqpoint{3.393409in}{2.497450in}}%
\pgfpathlineto{\pgfqpoint{3.393409in}{2.501202in}}%
\pgfpathlineto{\pgfqpoint{3.393409in}{2.504953in}}%
\pgfpathlineto{\pgfqpoint{3.393409in}{2.508705in}}%
\pgfpathlineto{\pgfqpoint{3.393409in}{2.512456in}}%
\pgfpathlineto{\pgfqpoint{3.393409in}{2.516208in}}%
\pgfpathlineto{\pgfqpoint{3.393409in}{2.519959in}}%
\pgfpathlineto{\pgfqpoint{3.393409in}{2.523711in}}%
\pgfpathlineto{\pgfqpoint{3.393409in}{2.527462in}}%
\pgfpathlineto{\pgfqpoint{3.393409in}{2.531214in}}%
\pgfpathlineto{\pgfqpoint{3.393409in}{2.534965in}}%
\pgfpathlineto{\pgfqpoint{3.393409in}{2.538717in}}%
\pgfpathlineto{\pgfqpoint{3.393409in}{2.542469in}}%
\pgfpathlineto{\pgfqpoint{3.393409in}{2.546220in}}%
\pgfpathlineto{\pgfqpoint{3.393409in}{2.549972in}}%
\pgfpathlineto{\pgfqpoint{3.393409in}{2.553723in}}%
\pgfpathlineto{\pgfqpoint{3.393409in}{2.557475in}}%
\pgfpathlineto{\pgfqpoint{3.393409in}{2.561226in}}%
\pgfpathlineto{\pgfqpoint{3.393409in}{2.564978in}}%
\pgfpathlineto{\pgfqpoint{3.393409in}{2.568729in}}%
\pgfpathlineto{\pgfqpoint{3.393409in}{2.572481in}}%
\pgfpathlineto{\pgfqpoint{3.393409in}{2.576232in}}%
\pgfpathlineto{\pgfqpoint{3.393409in}{2.579984in}}%
\pgfpathlineto{\pgfqpoint{3.393409in}{2.583735in}}%
\pgfpathlineto{\pgfqpoint{3.393409in}{2.587487in}}%
\pgfpathlineto{\pgfqpoint{3.393409in}{2.591238in}}%
\pgfpathlineto{\pgfqpoint{3.393409in}{2.594990in}}%
\pgfpathlineto{\pgfqpoint{3.393409in}{2.598742in}}%
\pgfpathlineto{\pgfqpoint{3.393409in}{2.602493in}}%
\pgfpathlineto{\pgfqpoint{3.393409in}{2.606245in}}%
\pgfpathlineto{\pgfqpoint{3.393409in}{2.609996in}}%
\pgfpathlineto{\pgfqpoint{3.393409in}{2.613748in}}%
\pgfpathlineto{\pgfqpoint{3.393409in}{2.617499in}}%
\pgfpathlineto{\pgfqpoint{3.393409in}{2.621251in}}%
\pgfpathlineto{\pgfqpoint{3.393409in}{2.625002in}}%
\pgfpathlineto{\pgfqpoint{3.393409in}{2.628754in}}%
\pgfpathlineto{\pgfqpoint{3.393409in}{2.632505in}}%
\pgfpathlineto{\pgfqpoint{3.393409in}{2.636257in}}%
\pgfpathlineto{\pgfqpoint{3.393409in}{2.640008in}}%
\pgfpathlineto{\pgfqpoint{3.393409in}{2.643760in}}%
\pgfpathlineto{\pgfqpoint{3.393409in}{2.647512in}}%
\pgfpathlineto{\pgfqpoint{3.393409in}{2.651263in}}%
\pgfpathlineto{\pgfqpoint{3.393409in}{2.655015in}}%
\pgfpathlineto{\pgfqpoint{3.393409in}{2.658766in}}%
\pgfpathlineto{\pgfqpoint{3.393409in}{2.662518in}}%
\pgfpathlineto{\pgfqpoint{3.393409in}{2.666269in}}%
\pgfpathlineto{\pgfqpoint{3.393409in}{2.670021in}}%
\pgfpathlineto{\pgfqpoint{3.393409in}{2.673772in}}%
\pgfpathlineto{\pgfqpoint{3.393409in}{2.677524in}}%
\pgfpathlineto{\pgfqpoint{3.393409in}{2.681275in}}%
\pgfpathlineto{\pgfqpoint{3.393409in}{2.685027in}}%
\pgfpathlineto{\pgfqpoint{3.393409in}{2.688778in}}%
\pgfpathlineto{\pgfqpoint{3.393409in}{2.692530in}}%
\pgfpathlineto{\pgfqpoint{3.393409in}{2.696281in}}%
\pgfpathlineto{\pgfqpoint{3.393409in}{2.700033in}}%
\pgfpathlineto{\pgfqpoint{3.393409in}{2.703785in}}%
\pgfpathlineto{\pgfqpoint{3.393409in}{2.707536in}}%
\pgfpathlineto{\pgfqpoint{3.393409in}{2.711288in}}%
\pgfpathlineto{\pgfqpoint{3.393409in}{2.715039in}}%
\pgfpathlineto{\pgfqpoint{3.393409in}{2.718791in}}%
\pgfpathlineto{\pgfqpoint{3.393409in}{2.722542in}}%
\pgfpathlineto{\pgfqpoint{3.393409in}{2.726294in}}%
\pgfpathlineto{\pgfqpoint{3.393409in}{2.730045in}}%
\pgfpathlineto{\pgfqpoint{3.393409in}{2.733797in}}%
\pgfpathlineto{\pgfqpoint{3.393409in}{2.737548in}}%
\pgfpathlineto{\pgfqpoint{3.393409in}{2.741300in}}%
\pgfpathlineto{\pgfqpoint{3.393409in}{2.745051in}}%
\pgfpathlineto{\pgfqpoint{3.393409in}{2.748803in}}%
\pgfpathlineto{\pgfqpoint{3.393409in}{2.752554in}}%
\pgfpathlineto{\pgfqpoint{3.393409in}{2.756306in}}%
\pgfpathlineto{\pgfqpoint{3.393409in}{2.760058in}}%
\pgfpathlineto{\pgfqpoint{3.393409in}{2.763809in}}%
\pgfpathlineto{\pgfqpoint{3.393409in}{2.767561in}}%
\pgfpathlineto{\pgfqpoint{3.393409in}{2.771312in}}%
\pgfpathlineto{\pgfqpoint{3.393409in}{2.775064in}}%
\pgfpathlineto{\pgfqpoint{3.393409in}{2.778815in}}%
\pgfpathlineto{\pgfqpoint{3.393409in}{2.782567in}}%
\pgfpathlineto{\pgfqpoint{3.393409in}{2.786318in}}%
\pgfpathlineto{\pgfqpoint{3.393409in}{2.790070in}}%
\pgfpathlineto{\pgfqpoint{3.393409in}{2.793821in}}%
\pgfpathlineto{\pgfqpoint{3.393409in}{2.797573in}}%
\pgfpathlineto{\pgfqpoint{3.393409in}{2.801324in}}%
\pgfpathlineto{\pgfqpoint{3.393409in}{2.805076in}}%
\pgfpathlineto{\pgfqpoint{3.393409in}{2.808827in}}%
\pgfpathlineto{\pgfqpoint{3.393409in}{2.812579in}}%
\pgfpathlineto{\pgfqpoint{3.393409in}{2.816331in}}%
\pgfpathlineto{\pgfqpoint{3.393409in}{2.820082in}}%
\pgfpathlineto{\pgfqpoint{3.393409in}{2.823834in}}%
\pgfpathlineto{\pgfqpoint{3.393409in}{2.827585in}}%
\pgfpathlineto{\pgfqpoint{3.393409in}{2.831337in}}%
\pgfpathlineto{\pgfqpoint{3.393409in}{2.835088in}}%
\pgfpathlineto{\pgfqpoint{3.393409in}{2.838840in}}%
\pgfpathlineto{\pgfqpoint{3.393409in}{2.842591in}}%
\pgfpathlineto{\pgfqpoint{3.393409in}{2.846343in}}%
\pgfpathlineto{\pgfqpoint{3.393409in}{2.850094in}}%
\pgfpathlineto{\pgfqpoint{3.393409in}{2.853846in}}%
\pgfpathlineto{\pgfqpoint{3.393409in}{2.857597in}}%
\pgfpathlineto{\pgfqpoint{3.393409in}{2.861349in}}%
\pgfpathlineto{\pgfqpoint{3.393409in}{2.865101in}}%
\pgfpathlineto{\pgfqpoint{3.393409in}{2.868852in}}%
\pgfpathlineto{\pgfqpoint{3.393409in}{2.872604in}}%
\pgfpathlineto{\pgfqpoint{3.393409in}{2.876355in}}%
\pgfpathlineto{\pgfqpoint{3.393409in}{2.880107in}}%
\pgfpathlineto{\pgfqpoint{3.393409in}{2.883858in}}%
\pgfpathlineto{\pgfqpoint{3.393409in}{2.887610in}}%
\pgfpathlineto{\pgfqpoint{3.393409in}{2.891361in}}%
\pgfpathlineto{\pgfqpoint{3.393409in}{2.895113in}}%
\pgfpathlineto{\pgfqpoint{3.393409in}{2.898864in}}%
\pgfpathlineto{\pgfqpoint{3.393409in}{2.902616in}}%
\pgfpathlineto{\pgfqpoint{3.393409in}{2.906367in}}%
\pgfpathlineto{\pgfqpoint{3.393409in}{2.910119in}}%
\pgfpathlineto{\pgfqpoint{3.393409in}{2.913870in}}%
\pgfpathlineto{\pgfqpoint{3.393409in}{2.917622in}}%
\pgfpathlineto{\pgfqpoint{3.393409in}{2.921374in}}%
\pgfpathlineto{\pgfqpoint{3.393409in}{2.925125in}}%
\pgfpathlineto{\pgfqpoint{3.393409in}{2.928877in}}%
\pgfpathlineto{\pgfqpoint{3.393409in}{2.932628in}}%
\pgfpathlineto{\pgfqpoint{3.393409in}{2.936380in}}%
\pgfpathlineto{\pgfqpoint{3.393409in}{2.940131in}}%
\pgfpathlineto{\pgfqpoint{3.393409in}{2.943883in}}%
\pgfpathlineto{\pgfqpoint{3.393409in}{2.947634in}}%
\pgfpathlineto{\pgfqpoint{3.393409in}{2.951386in}}%
\pgfpathlineto{\pgfqpoint{3.393409in}{2.955137in}}%
\pgfpathlineto{\pgfqpoint{3.393409in}{2.958889in}}%
\pgfpathlineto{\pgfqpoint{3.393409in}{2.962640in}}%
\pgfpathlineto{\pgfqpoint{3.393409in}{2.966392in}}%
\pgfpathlineto{\pgfqpoint{3.393409in}{2.970143in}}%
\pgfpathlineto{\pgfqpoint{3.393409in}{2.973895in}}%
\pgfpathlineto{\pgfqpoint{3.393409in}{2.977647in}}%
\pgfpathlineto{\pgfqpoint{3.393409in}{2.981398in}}%
\pgfpathlineto{\pgfqpoint{3.393409in}{2.985150in}}%
\pgfpathlineto{\pgfqpoint{3.393409in}{2.988901in}}%
\pgfpathlineto{\pgfqpoint{3.393409in}{2.992653in}}%
\pgfpathlineto{\pgfqpoint{3.393409in}{2.996404in}}%
\pgfpathlineto{\pgfqpoint{3.393409in}{3.000156in}}%
\pgfpathlineto{\pgfqpoint{3.393409in}{3.003907in}}%
\pgfpathlineto{\pgfqpoint{3.393409in}{3.007659in}}%
\pgfpathlineto{\pgfqpoint{3.393409in}{3.011410in}}%
\pgfpathlineto{\pgfqpoint{3.393409in}{3.015162in}}%
\pgfpathlineto{\pgfqpoint{3.393409in}{3.018913in}}%
\pgfpathlineto{\pgfqpoint{3.393409in}{3.022665in}}%
\pgfpathlineto{\pgfqpoint{3.393409in}{3.026416in}}%
\pgfpathlineto{\pgfqpoint{3.393409in}{3.030168in}}%
\pgfpathlineto{\pgfqpoint{3.393409in}{3.033920in}}%
\pgfpathlineto{\pgfqpoint{3.393409in}{3.037671in}}%
\pgfpathlineto{\pgfqpoint{3.393409in}{3.041423in}}%
\pgfpathlineto{\pgfqpoint{3.393409in}{3.045174in}}%
\pgfpathlineto{\pgfqpoint{3.393409in}{3.048926in}}%
\pgfpathlineto{\pgfqpoint{3.393409in}{3.052677in}}%
\pgfpathlineto{\pgfqpoint{3.393409in}{3.056429in}}%
\pgfpathlineto{\pgfqpoint{3.393409in}{3.060180in}}%
\pgfpathlineto{\pgfqpoint{3.393409in}{3.063932in}}%
\pgfpathlineto{\pgfqpoint{3.393409in}{3.067683in}}%
\pgfpathlineto{\pgfqpoint{3.393409in}{3.071435in}}%
\pgfpathlineto{\pgfqpoint{3.393409in}{3.075186in}}%
\pgfpathlineto{\pgfqpoint{3.393409in}{3.078938in}}%
\pgfpathlineto{\pgfqpoint{3.393409in}{3.082690in}}%
\pgfpathlineto{\pgfqpoint{3.393409in}{3.086441in}}%
\pgfpathlineto{\pgfqpoint{3.393409in}{3.090193in}}%
\pgfpathlineto{\pgfqpoint{3.393409in}{3.093944in}}%
\pgfpathlineto{\pgfqpoint{3.393409in}{3.097696in}}%
\pgfpathlineto{\pgfqpoint{3.393409in}{3.101447in}}%
\pgfpathlineto{\pgfqpoint{3.393409in}{3.105199in}}%
\pgfpathlineto{\pgfqpoint{3.393409in}{3.108950in}}%
\pgfpathlineto{\pgfqpoint{3.393409in}{3.112702in}}%
\pgfpathlineto{\pgfqpoint{3.393409in}{3.116453in}}%
\pgfpathlineto{\pgfqpoint{3.393409in}{3.120205in}}%
\pgfpathlineto{\pgfqpoint{3.393409in}{3.123956in}}%
\pgfpathlineto{\pgfqpoint{3.393409in}{3.127708in}}%
\pgfpathlineto{\pgfqpoint{3.393409in}{3.131459in}}%
\pgfpathlineto{\pgfqpoint{3.393409in}{3.135211in}}%
\pgfpathlineto{\pgfqpoint{3.393409in}{3.138963in}}%
\pgfpathlineto{\pgfqpoint{3.393409in}{3.142714in}}%
\pgfpathlineto{\pgfqpoint{3.393409in}{3.146466in}}%
\pgfpathlineto{\pgfqpoint{3.393409in}{3.150217in}}%
\pgfpathlineto{\pgfqpoint{3.393409in}{3.153969in}}%
\pgfpathlineto{\pgfqpoint{3.393409in}{3.157720in}}%
\pgfpathlineto{\pgfqpoint{3.393409in}{3.161472in}}%
\pgfpathlineto{\pgfqpoint{3.393409in}{3.165223in}}%
\pgfpathlineto{\pgfqpoint{3.393409in}{3.168975in}}%
\pgfpathlineto{\pgfqpoint{3.393409in}{3.172726in}}%
\pgfpathlineto{\pgfqpoint{3.393409in}{3.176478in}}%
\pgfpathlineto{\pgfqpoint{3.393409in}{3.180229in}}%
\pgfpathlineto{\pgfqpoint{3.393409in}{3.183981in}}%
\pgfpathlineto{\pgfqpoint{3.393409in}{3.187732in}}%
\pgfpathlineto{\pgfqpoint{3.393409in}{3.191484in}}%
\pgfpathlineto{\pgfqpoint{3.393409in}{3.195236in}}%
\pgfpathlineto{\pgfqpoint{3.393409in}{3.198987in}}%
\pgfpathlineto{\pgfqpoint{3.393409in}{3.202739in}}%
\pgfpathlineto{\pgfqpoint{3.393409in}{3.206490in}}%
\pgfpathlineto{\pgfqpoint{3.393409in}{3.210242in}}%
\pgfpathlineto{\pgfqpoint{3.393409in}{3.213993in}}%
\pgfpathlineto{\pgfqpoint{3.393409in}{3.217745in}}%
\pgfpathlineto{\pgfqpoint{3.393409in}{3.221496in}}%
\pgfpathlineto{\pgfqpoint{3.393409in}{3.225248in}}%
\pgfpathlineto{\pgfqpoint{3.393409in}{3.228999in}}%
\pgfpathlineto{\pgfqpoint{3.393409in}{3.232751in}}%
\pgfpathlineto{\pgfqpoint{3.393409in}{3.236502in}}%
\pgfpathlineto{\pgfqpoint{3.393409in}{3.240254in}}%
\pgfpathlineto{\pgfqpoint{3.393409in}{3.244005in}}%
\pgfpathlineto{\pgfqpoint{3.393409in}{3.247757in}}%
\pgfpathlineto{\pgfqpoint{3.393409in}{3.251509in}}%
\pgfpathlineto{\pgfqpoint{3.393409in}{3.255260in}}%
\pgfpathlineto{\pgfqpoint{3.393409in}{3.259012in}}%
\pgfpathlineto{\pgfqpoint{3.393409in}{3.262763in}}%
\pgfpathlineto{\pgfqpoint{3.393409in}{3.266515in}}%
\pgfpathlineto{\pgfqpoint{3.393409in}{3.270266in}}%
\pgfpathlineto{\pgfqpoint{3.393409in}{3.274018in}}%
\pgfpathlineto{\pgfqpoint{3.393409in}{3.277769in}}%
\pgfpathlineto{\pgfqpoint{3.393409in}{3.281521in}}%
\pgfpathlineto{\pgfqpoint{3.393409in}{3.285272in}}%
\pgfpathlineto{\pgfqpoint{3.393409in}{3.289024in}}%
\pgfpathlineto{\pgfqpoint{3.393409in}{3.292775in}}%
\pgfpathlineto{\pgfqpoint{3.393409in}{3.296527in}}%
\pgfpathlineto{\pgfqpoint{3.393409in}{3.300279in}}%
\pgfpathlineto{\pgfqpoint{3.393409in}{3.304030in}}%
\pgfpathlineto{\pgfqpoint{3.393409in}{3.307782in}}%
\pgfpathlineto{\pgfqpoint{3.393409in}{3.311533in}}%
\pgfpathlineto{\pgfqpoint{3.393409in}{3.315285in}}%
\pgfpathlineto{\pgfqpoint{3.393409in}{3.319036in}}%
\pgfpathlineto{\pgfqpoint{3.393409in}{3.322788in}}%
\pgfpathlineto{\pgfqpoint{3.393409in}{3.326539in}}%
\pgfpathlineto{\pgfqpoint{3.393409in}{3.330291in}}%
\pgfpathlineto{\pgfqpoint{3.393409in}{3.334042in}}%
\pgfpathlineto{\pgfqpoint{3.393409in}{3.337794in}}%
\pgfpathlineto{\pgfqpoint{3.393409in}{3.341545in}}%
\pgfpathlineto{\pgfqpoint{3.393409in}{3.345297in}}%
\pgfpathlineto{\pgfqpoint{3.393409in}{3.349048in}}%
\pgfpathlineto{\pgfqpoint{3.393409in}{3.352800in}}%
\pgfpathlineto{\pgfqpoint{3.390274in}{3.352800in}}%
\pgfpathlineto{\pgfqpoint{3.387140in}{3.352800in}}%
\pgfpathlineto{\pgfqpoint{3.384005in}{3.352800in}}%
\pgfpathlineto{\pgfqpoint{3.380870in}{3.352800in}}%
\pgfpathlineto{\pgfqpoint{3.377735in}{3.352800in}}%
\pgfpathlineto{\pgfqpoint{3.374601in}{3.352800in}}%
\pgfpathlineto{\pgfqpoint{3.371466in}{3.352800in}}%
\pgfpathlineto{\pgfqpoint{3.368331in}{3.352800in}}%
\pgfpathlineto{\pgfqpoint{3.365196in}{3.352800in}}%
\pgfpathlineto{\pgfqpoint{3.362062in}{3.352800in}}%
\pgfpathlineto{\pgfqpoint{3.358927in}{3.352800in}}%
\pgfpathlineto{\pgfqpoint{3.355792in}{3.352800in}}%
\pgfpathlineto{\pgfqpoint{3.352657in}{3.352800in}}%
\pgfpathlineto{\pgfqpoint{3.349523in}{3.352800in}}%
\pgfpathlineto{\pgfqpoint{3.346388in}{3.352800in}}%
\pgfpathlineto{\pgfqpoint{3.343253in}{3.352800in}}%
\pgfpathlineto{\pgfqpoint{3.340118in}{3.352800in}}%
\pgfpathlineto{\pgfqpoint{3.336984in}{3.352800in}}%
\pgfpathlineto{\pgfqpoint{3.333849in}{3.352800in}}%
\pgfpathlineto{\pgfqpoint{3.330714in}{3.352800in}}%
\pgfpathlineto{\pgfqpoint{3.327580in}{3.352800in}}%
\pgfpathlineto{\pgfqpoint{3.324445in}{3.352800in}}%
\pgfpathlineto{\pgfqpoint{3.321310in}{3.352800in}}%
\pgfpathlineto{\pgfqpoint{3.318175in}{3.352800in}}%
\pgfpathlineto{\pgfqpoint{3.315041in}{3.352800in}}%
\pgfpathlineto{\pgfqpoint{3.311906in}{3.352800in}}%
\pgfpathlineto{\pgfqpoint{3.308771in}{3.352800in}}%
\pgfpathlineto{\pgfqpoint{3.305636in}{3.352800in}}%
\pgfpathlineto{\pgfqpoint{3.302502in}{3.352800in}}%
\pgfpathlineto{\pgfqpoint{3.299367in}{3.352800in}}%
\pgfpathlineto{\pgfqpoint{3.296232in}{3.352800in}}%
\pgfpathlineto{\pgfqpoint{3.293097in}{3.352800in}}%
\pgfpathlineto{\pgfqpoint{3.289963in}{3.352800in}}%
\pgfpathlineto{\pgfqpoint{3.286828in}{3.352800in}}%
\pgfpathlineto{\pgfqpoint{3.283693in}{3.352800in}}%
\pgfpathlineto{\pgfqpoint{3.280558in}{3.352800in}}%
\pgfpathlineto{\pgfqpoint{3.277424in}{3.352800in}}%
\pgfpathlineto{\pgfqpoint{3.274289in}{3.352800in}}%
\pgfpathlineto{\pgfqpoint{3.271154in}{3.352800in}}%
\pgfpathlineto{\pgfqpoint{3.268019in}{3.352800in}}%
\pgfpathlineto{\pgfqpoint{3.264885in}{3.352800in}}%
\pgfpathlineto{\pgfqpoint{3.261750in}{3.352800in}}%
\pgfpathlineto{\pgfqpoint{3.258615in}{3.352800in}}%
\pgfpathlineto{\pgfqpoint{3.255480in}{3.352800in}}%
\pgfpathlineto{\pgfqpoint{3.252346in}{3.352800in}}%
\pgfpathlineto{\pgfqpoint{3.249211in}{3.352800in}}%
\pgfpathlineto{\pgfqpoint{3.246076in}{3.352800in}}%
\pgfpathlineto{\pgfqpoint{3.242941in}{3.352800in}}%
\pgfpathlineto{\pgfqpoint{3.239807in}{3.352800in}}%
\pgfpathlineto{\pgfqpoint{3.236672in}{3.352800in}}%
\pgfpathlineto{\pgfqpoint{3.233537in}{3.352800in}}%
\pgfpathlineto{\pgfqpoint{3.230402in}{3.352800in}}%
\pgfpathlineto{\pgfqpoint{3.227268in}{3.352800in}}%
\pgfpathlineto{\pgfqpoint{3.224133in}{3.352800in}}%
\pgfpathlineto{\pgfqpoint{3.220998in}{3.352800in}}%
\pgfpathlineto{\pgfqpoint{3.217864in}{3.352800in}}%
\pgfpathlineto{\pgfqpoint{3.214729in}{3.352800in}}%
\pgfpathlineto{\pgfqpoint{3.211594in}{3.352800in}}%
\pgfpathlineto{\pgfqpoint{3.208459in}{3.352800in}}%
\pgfpathlineto{\pgfqpoint{3.205325in}{3.352800in}}%
\pgfpathlineto{\pgfqpoint{3.202190in}{3.352800in}}%
\pgfpathlineto{\pgfqpoint{3.199055in}{3.352800in}}%
\pgfpathlineto{\pgfqpoint{3.195920in}{3.352800in}}%
\pgfpathlineto{\pgfqpoint{3.192786in}{3.352800in}}%
\pgfpathlineto{\pgfqpoint{3.189651in}{3.352800in}}%
\pgfpathlineto{\pgfqpoint{3.186516in}{3.352800in}}%
\pgfpathlineto{\pgfqpoint{3.183381in}{3.352800in}}%
\pgfpathlineto{\pgfqpoint{3.180247in}{3.352800in}}%
\pgfpathlineto{\pgfqpoint{3.177112in}{3.352800in}}%
\pgfpathlineto{\pgfqpoint{3.173977in}{3.352800in}}%
\pgfpathlineto{\pgfqpoint{3.170842in}{3.352800in}}%
\pgfpathlineto{\pgfqpoint{3.167708in}{3.352800in}}%
\pgfpathlineto{\pgfqpoint{3.164573in}{3.352800in}}%
\pgfpathlineto{\pgfqpoint{3.161438in}{3.352800in}}%
\pgfpathlineto{\pgfqpoint{3.158303in}{3.352800in}}%
\pgfpathlineto{\pgfqpoint{3.155169in}{3.352800in}}%
\pgfpathlineto{\pgfqpoint{3.152034in}{3.352800in}}%
\pgfpathlineto{\pgfqpoint{3.148899in}{3.352800in}}%
\pgfpathlineto{\pgfqpoint{3.145764in}{3.352800in}}%
\pgfpathlineto{\pgfqpoint{3.142630in}{3.352800in}}%
\pgfpathlineto{\pgfqpoint{3.139495in}{3.352800in}}%
\pgfpathlineto{\pgfqpoint{3.136360in}{3.352800in}}%
\pgfpathlineto{\pgfqpoint{3.133225in}{3.352800in}}%
\pgfpathlineto{\pgfqpoint{3.130091in}{3.352800in}}%
\pgfpathlineto{\pgfqpoint{3.126956in}{3.352800in}}%
\pgfpathlineto{\pgfqpoint{3.123821in}{3.352800in}}%
\pgfpathlineto{\pgfqpoint{3.120687in}{3.352800in}}%
\pgfpathlineto{\pgfqpoint{3.117552in}{3.352800in}}%
\pgfpathlineto{\pgfqpoint{3.114417in}{3.352800in}}%
\pgfpathlineto{\pgfqpoint{3.111282in}{3.352800in}}%
\pgfpathlineto{\pgfqpoint{3.108148in}{3.352800in}}%
\pgfpathlineto{\pgfqpoint{3.105013in}{3.352800in}}%
\pgfpathlineto{\pgfqpoint{3.101878in}{3.352800in}}%
\pgfpathlineto{\pgfqpoint{3.098743in}{3.352800in}}%
\pgfpathlineto{\pgfqpoint{3.095609in}{3.352800in}}%
\pgfpathlineto{\pgfqpoint{3.092474in}{3.352800in}}%
\pgfpathlineto{\pgfqpoint{3.089339in}{3.352800in}}%
\pgfpathlineto{\pgfqpoint{3.086204in}{3.352800in}}%
\pgfpathlineto{\pgfqpoint{3.083070in}{3.352800in}}%
\pgfpathlineto{\pgfqpoint{3.079935in}{3.352800in}}%
\pgfpathlineto{\pgfqpoint{3.076800in}{3.352800in}}%
\pgfpathlineto{\pgfqpoint{3.073665in}{3.352800in}}%
\pgfpathlineto{\pgfqpoint{3.070531in}{3.352800in}}%
\pgfpathlineto{\pgfqpoint{3.067396in}{3.352800in}}%
\pgfpathlineto{\pgfqpoint{3.064261in}{3.352800in}}%
\pgfpathlineto{\pgfqpoint{3.061126in}{3.352800in}}%
\pgfpathlineto{\pgfqpoint{3.057992in}{3.352800in}}%
\pgfpathlineto{\pgfqpoint{3.054857in}{3.352800in}}%
\pgfpathlineto{\pgfqpoint{3.051722in}{3.352800in}}%
\pgfpathlineto{\pgfqpoint{3.048587in}{3.352800in}}%
\pgfpathlineto{\pgfqpoint{3.045453in}{3.352800in}}%
\pgfpathlineto{\pgfqpoint{3.042318in}{3.352800in}}%
\pgfpathlineto{\pgfqpoint{3.039183in}{3.352800in}}%
\pgfpathlineto{\pgfqpoint{3.036048in}{3.352800in}}%
\pgfpathlineto{\pgfqpoint{3.032914in}{3.352800in}}%
\pgfpathlineto{\pgfqpoint{3.029779in}{3.352800in}}%
\pgfpathlineto{\pgfqpoint{3.026644in}{3.352800in}}%
\pgfpathlineto{\pgfqpoint{3.023510in}{3.352800in}}%
\pgfpathlineto{\pgfqpoint{3.020375in}{3.352800in}}%
\pgfpathlineto{\pgfqpoint{3.017240in}{3.352800in}}%
\pgfpathlineto{\pgfqpoint{3.014105in}{3.352800in}}%
\pgfpathlineto{\pgfqpoint{3.010971in}{3.352800in}}%
\pgfpathlineto{\pgfqpoint{3.007836in}{3.352800in}}%
\pgfpathlineto{\pgfqpoint{3.004701in}{3.352800in}}%
\pgfpathlineto{\pgfqpoint{3.001566in}{3.352800in}}%
\pgfpathlineto{\pgfqpoint{2.998432in}{3.352800in}}%
\pgfpathlineto{\pgfqpoint{2.995297in}{3.352800in}}%
\pgfpathlineto{\pgfqpoint{2.992162in}{3.352800in}}%
\pgfpathlineto{\pgfqpoint{2.989027in}{3.352800in}}%
\pgfpathlineto{\pgfqpoint{2.985893in}{3.352800in}}%
\pgfpathlineto{\pgfqpoint{2.982758in}{3.352800in}}%
\pgfpathlineto{\pgfqpoint{2.979623in}{3.352800in}}%
\pgfpathlineto{\pgfqpoint{2.976488in}{3.352800in}}%
\pgfpathlineto{\pgfqpoint{2.973354in}{3.352800in}}%
\pgfpathlineto{\pgfqpoint{2.970219in}{3.352800in}}%
\pgfpathlineto{\pgfqpoint{2.967084in}{3.352800in}}%
\pgfpathlineto{\pgfqpoint{2.963949in}{3.352800in}}%
\pgfpathlineto{\pgfqpoint{2.960815in}{3.352800in}}%
\pgfpathlineto{\pgfqpoint{2.957680in}{3.352800in}}%
\pgfpathlineto{\pgfqpoint{2.954545in}{3.352800in}}%
\pgfpathlineto{\pgfqpoint{2.951410in}{3.352800in}}%
\pgfpathlineto{\pgfqpoint{2.948276in}{3.352800in}}%
\pgfpathlineto{\pgfqpoint{2.945141in}{3.352800in}}%
\pgfpathlineto{\pgfqpoint{2.942006in}{3.352800in}}%
\pgfpathlineto{\pgfqpoint{2.938871in}{3.352800in}}%
\pgfpathlineto{\pgfqpoint{2.935737in}{3.352800in}}%
\pgfpathlineto{\pgfqpoint{2.932602in}{3.352800in}}%
\pgfpathlineto{\pgfqpoint{2.929467in}{3.352800in}}%
\pgfpathlineto{\pgfqpoint{2.926332in}{3.352800in}}%
\pgfpathlineto{\pgfqpoint{2.923198in}{3.352800in}}%
\pgfpathlineto{\pgfqpoint{2.920063in}{3.352800in}}%
\pgfpathlineto{\pgfqpoint{2.916928in}{3.352800in}}%
\pgfpathlineto{\pgfqpoint{2.913794in}{3.352800in}}%
\pgfpathlineto{\pgfqpoint{2.910659in}{3.352800in}}%
\pgfpathlineto{\pgfqpoint{2.907524in}{3.352800in}}%
\pgfpathlineto{\pgfqpoint{2.904389in}{3.352800in}}%
\pgfpathlineto{\pgfqpoint{2.901255in}{3.352800in}}%
\pgfpathlineto{\pgfqpoint{2.898120in}{3.352800in}}%
\pgfpathlineto{\pgfqpoint{2.894985in}{3.352800in}}%
\pgfpathlineto{\pgfqpoint{2.891850in}{3.352800in}}%
\pgfpathlineto{\pgfqpoint{2.888716in}{3.352800in}}%
\pgfpathlineto{\pgfqpoint{2.885581in}{3.352800in}}%
\pgfpathlineto{\pgfqpoint{2.882446in}{3.352800in}}%
\pgfpathlineto{\pgfqpoint{2.879311in}{3.352800in}}%
\pgfpathlineto{\pgfqpoint{2.876177in}{3.352800in}}%
\pgfpathlineto{\pgfqpoint{2.873042in}{3.352800in}}%
\pgfpathlineto{\pgfqpoint{2.869907in}{3.352800in}}%
\pgfpathlineto{\pgfqpoint{2.866772in}{3.352800in}}%
\pgfpathlineto{\pgfqpoint{2.863638in}{3.352800in}}%
\pgfpathlineto{\pgfqpoint{2.860503in}{3.352800in}}%
\pgfpathlineto{\pgfqpoint{2.857368in}{3.352800in}}%
\pgfpathlineto{\pgfqpoint{2.854233in}{3.352800in}}%
\pgfpathlineto{\pgfqpoint{2.851099in}{3.352800in}}%
\pgfpathlineto{\pgfqpoint{2.847964in}{3.352800in}}%
\pgfpathlineto{\pgfqpoint{2.844829in}{3.352800in}}%
\pgfpathlineto{\pgfqpoint{2.841694in}{3.352800in}}%
\pgfpathlineto{\pgfqpoint{2.838560in}{3.352800in}}%
\pgfpathlineto{\pgfqpoint{2.835425in}{3.352800in}}%
\pgfpathlineto{\pgfqpoint{2.832290in}{3.352800in}}%
\pgfpathlineto{\pgfqpoint{2.829155in}{3.352800in}}%
\pgfpathlineto{\pgfqpoint{2.826021in}{3.352800in}}%
\pgfpathlineto{\pgfqpoint{2.822886in}{3.352800in}}%
\pgfpathlineto{\pgfqpoint{2.819751in}{3.352800in}}%
\pgfpathlineto{\pgfqpoint{2.816617in}{3.352800in}}%
\pgfpathlineto{\pgfqpoint{2.813482in}{3.352800in}}%
\pgfpathlineto{\pgfqpoint{2.810347in}{3.352800in}}%
\pgfpathlineto{\pgfqpoint{2.807212in}{3.352800in}}%
\pgfpathlineto{\pgfqpoint{2.804078in}{3.352800in}}%
\pgfpathlineto{\pgfqpoint{2.800943in}{3.352800in}}%
\pgfpathlineto{\pgfqpoint{2.797808in}{3.352800in}}%
\pgfpathlineto{\pgfqpoint{2.794673in}{3.352800in}}%
\pgfpathlineto{\pgfqpoint{2.791539in}{3.352800in}}%
\pgfpathlineto{\pgfqpoint{2.788404in}{3.352800in}}%
\pgfpathlineto{\pgfqpoint{2.785269in}{3.352800in}}%
\pgfpathlineto{\pgfqpoint{2.782134in}{3.352800in}}%
\pgfpathlineto{\pgfqpoint{2.779000in}{3.352800in}}%
\pgfpathlineto{\pgfqpoint{2.775865in}{3.352800in}}%
\pgfpathlineto{\pgfqpoint{2.772730in}{3.352800in}}%
\pgfpathlineto{\pgfqpoint{2.769595in}{3.352800in}}%
\pgfpathlineto{\pgfqpoint{2.766461in}{3.352800in}}%
\pgfpathlineto{\pgfqpoint{2.763326in}{3.352800in}}%
\pgfpathlineto{\pgfqpoint{2.760191in}{3.352800in}}%
\pgfpathlineto{\pgfqpoint{2.757056in}{3.352800in}}%
\pgfpathlineto{\pgfqpoint{2.753922in}{3.352800in}}%
\pgfpathlineto{\pgfqpoint{2.750787in}{3.352800in}}%
\pgfpathlineto{\pgfqpoint{2.747652in}{3.352800in}}%
\pgfpathlineto{\pgfqpoint{2.744517in}{3.352800in}}%
\pgfpathlineto{\pgfqpoint{2.741383in}{3.352800in}}%
\pgfpathlineto{\pgfqpoint{2.738248in}{3.352800in}}%
\pgfpathlineto{\pgfqpoint{2.735113in}{3.352800in}}%
\pgfpathlineto{\pgfqpoint{2.731978in}{3.352800in}}%
\pgfpathlineto{\pgfqpoint{2.728844in}{3.352800in}}%
\pgfpathlineto{\pgfqpoint{2.725709in}{3.352800in}}%
\pgfpathlineto{\pgfqpoint{2.722574in}{3.352800in}}%
\pgfpathlineto{\pgfqpoint{2.719439in}{3.352800in}}%
\pgfpathlineto{\pgfqpoint{2.716305in}{3.352800in}}%
\pgfpathlineto{\pgfqpoint{2.713170in}{3.352800in}}%
\pgfpathlineto{\pgfqpoint{2.710035in}{3.352800in}}%
\pgfpathlineto{\pgfqpoint{2.706901in}{3.352800in}}%
\pgfpathlineto{\pgfqpoint{2.703766in}{3.352800in}}%
\pgfpathlineto{\pgfqpoint{2.700631in}{3.352800in}}%
\pgfpathlineto{\pgfqpoint{2.697496in}{3.352800in}}%
\pgfpathlineto{\pgfqpoint{2.694362in}{3.352800in}}%
\pgfpathlineto{\pgfqpoint{2.691227in}{3.352800in}}%
\pgfpathlineto{\pgfqpoint{2.688092in}{3.352800in}}%
\pgfpathlineto{\pgfqpoint{2.684957in}{3.352800in}}%
\pgfpathlineto{\pgfqpoint{2.681823in}{3.352800in}}%
\pgfpathlineto{\pgfqpoint{2.678688in}{3.352800in}}%
\pgfpathlineto{\pgfqpoint{2.675553in}{3.352800in}}%
\pgfpathlineto{\pgfqpoint{2.672418in}{3.352800in}}%
\pgfpathlineto{\pgfqpoint{2.669284in}{3.352800in}}%
\pgfpathlineto{\pgfqpoint{2.666149in}{3.352800in}}%
\pgfpathlineto{\pgfqpoint{2.663014in}{3.352800in}}%
\pgfpathlineto{\pgfqpoint{2.659879in}{3.352800in}}%
\pgfpathlineto{\pgfqpoint{2.656745in}{3.352800in}}%
\pgfpathlineto{\pgfqpoint{2.653610in}{3.352800in}}%
\pgfpathlineto{\pgfqpoint{2.650475in}{3.352800in}}%
\pgfpathlineto{\pgfqpoint{2.647340in}{3.352800in}}%
\pgfpathlineto{\pgfqpoint{2.644206in}{3.352800in}}%
\pgfpathlineto{\pgfqpoint{2.641071in}{3.352800in}}%
\pgfpathlineto{\pgfqpoint{2.637936in}{3.352800in}}%
\pgfpathlineto{\pgfqpoint{2.634801in}{3.352800in}}%
\pgfpathlineto{\pgfqpoint{2.631667in}{3.352800in}}%
\pgfpathlineto{\pgfqpoint{2.628532in}{3.352800in}}%
\pgfpathlineto{\pgfqpoint{2.625397in}{3.352800in}}%
\pgfpathlineto{\pgfqpoint{2.622262in}{3.352800in}}%
\pgfpathlineto{\pgfqpoint{2.619128in}{3.352800in}}%
\pgfpathlineto{\pgfqpoint{2.615993in}{3.352800in}}%
\pgfpathlineto{\pgfqpoint{2.612858in}{3.352800in}}%
\pgfpathlineto{\pgfqpoint{2.609724in}{3.352800in}}%
\pgfpathlineto{\pgfqpoint{2.606589in}{3.352800in}}%
\pgfpathlineto{\pgfqpoint{2.603454in}{3.352800in}}%
\pgfpathlineto{\pgfqpoint{2.600319in}{3.352800in}}%
\pgfpathlineto{\pgfqpoint{2.597185in}{3.352800in}}%
\pgfpathlineto{\pgfqpoint{2.594050in}{3.352800in}}%
\pgfpathlineto{\pgfqpoint{2.590915in}{3.352800in}}%
\pgfpathlineto{\pgfqpoint{2.587780in}{3.352800in}}%
\pgfpathlineto{\pgfqpoint{2.584646in}{3.352800in}}%
\pgfpathlineto{\pgfqpoint{2.581511in}{3.352800in}}%
\pgfpathlineto{\pgfqpoint{2.578376in}{3.352800in}}%
\pgfpathlineto{\pgfqpoint{2.575241in}{3.352800in}}%
\pgfpathlineto{\pgfqpoint{2.572107in}{3.352800in}}%
\pgfpathlineto{\pgfqpoint{2.568972in}{3.352800in}}%
\pgfpathlineto{\pgfqpoint{2.565837in}{3.352800in}}%
\pgfpathlineto{\pgfqpoint{2.562702in}{3.352800in}}%
\pgfpathlineto{\pgfqpoint{2.559568in}{3.352800in}}%
\pgfpathlineto{\pgfqpoint{2.556433in}{3.352800in}}%
\pgfpathlineto{\pgfqpoint{2.553298in}{3.352800in}}%
\pgfpathlineto{\pgfqpoint{2.550163in}{3.352800in}}%
\pgfpathlineto{\pgfqpoint{2.547029in}{3.352800in}}%
\pgfpathlineto{\pgfqpoint{2.543894in}{3.352800in}}%
\pgfpathlineto{\pgfqpoint{2.540759in}{3.352800in}}%
\pgfpathlineto{\pgfqpoint{2.537624in}{3.352800in}}%
\pgfpathlineto{\pgfqpoint{2.534490in}{3.352800in}}%
\pgfpathlineto{\pgfqpoint{2.531355in}{3.352800in}}%
\pgfpathlineto{\pgfqpoint{2.528220in}{3.352800in}}%
\pgfpathlineto{\pgfqpoint{2.525085in}{3.352800in}}%
\pgfpathlineto{\pgfqpoint{2.521951in}{3.352800in}}%
\pgfpathlineto{\pgfqpoint{2.518816in}{3.352800in}}%
\pgfpathlineto{\pgfqpoint{2.515681in}{3.352800in}}%
\pgfpathlineto{\pgfqpoint{2.512547in}{3.352800in}}%
\pgfpathlineto{\pgfqpoint{2.509412in}{3.352800in}}%
\pgfpathlineto{\pgfqpoint{2.506277in}{3.352800in}}%
\pgfpathlineto{\pgfqpoint{2.503142in}{3.352800in}}%
\pgfpathlineto{\pgfqpoint{2.500008in}{3.352800in}}%
\pgfpathlineto{\pgfqpoint{2.496873in}{3.352800in}}%
\pgfpathlineto{\pgfqpoint{2.493738in}{3.352800in}}%
\pgfpathlineto{\pgfqpoint{2.490603in}{3.352800in}}%
\pgfpathlineto{\pgfqpoint{2.487469in}{3.352800in}}%
\pgfpathlineto{\pgfqpoint{2.484334in}{3.352800in}}%
\pgfpathlineto{\pgfqpoint{2.481199in}{3.352800in}}%
\pgfpathlineto{\pgfqpoint{2.478064in}{3.352800in}}%
\pgfpathlineto{\pgfqpoint{2.474930in}{3.352800in}}%
\pgfpathlineto{\pgfqpoint{2.471795in}{3.352800in}}%
\pgfpathlineto{\pgfqpoint{2.468660in}{3.352800in}}%
\pgfpathlineto{\pgfqpoint{2.465525in}{3.352800in}}%
\pgfpathlineto{\pgfqpoint{2.462391in}{3.352800in}}%
\pgfpathlineto{\pgfqpoint{2.459256in}{3.352800in}}%
\pgfpathlineto{\pgfqpoint{2.456121in}{3.352800in}}%
\pgfpathlineto{\pgfqpoint{2.452986in}{3.352800in}}%
\pgfpathlineto{\pgfqpoint{2.449852in}{3.352800in}}%
\pgfpathlineto{\pgfqpoint{2.446717in}{3.352800in}}%
\pgfpathlineto{\pgfqpoint{2.443582in}{3.352800in}}%
\pgfpathlineto{\pgfqpoint{2.440447in}{3.352800in}}%
\pgfpathlineto{\pgfqpoint{2.437313in}{3.352800in}}%
\pgfpathlineto{\pgfqpoint{2.434178in}{3.352800in}}%
\pgfpathlineto{\pgfqpoint{2.431043in}{3.352800in}}%
\pgfpathlineto{\pgfqpoint{2.427908in}{3.352800in}}%
\pgfpathlineto{\pgfqpoint{2.424774in}{3.352800in}}%
\pgfpathlineto{\pgfqpoint{2.421639in}{3.352800in}}%
\pgfpathlineto{\pgfqpoint{2.418504in}{3.352800in}}%
\pgfpathlineto{\pgfqpoint{2.415369in}{3.352800in}}%
\pgfpathlineto{\pgfqpoint{2.412235in}{3.352800in}}%
\pgfpathlineto{\pgfqpoint{2.409100in}{3.352800in}}%
\pgfpathlineto{\pgfqpoint{2.405965in}{3.352800in}}%
\pgfpathlineto{\pgfqpoint{2.402831in}{3.352800in}}%
\pgfpathlineto{\pgfqpoint{2.399696in}{3.352800in}}%
\pgfpathlineto{\pgfqpoint{2.396561in}{3.352800in}}%
\pgfpathlineto{\pgfqpoint{2.393426in}{3.352800in}}%
\pgfpathlineto{\pgfqpoint{2.390292in}{3.352800in}}%
\pgfpathlineto{\pgfqpoint{2.387157in}{3.352800in}}%
\pgfpathlineto{\pgfqpoint{2.384022in}{3.352800in}}%
\pgfpathlineto{\pgfqpoint{2.380887in}{3.352800in}}%
\pgfpathlineto{\pgfqpoint{2.377753in}{3.352800in}}%
\pgfpathlineto{\pgfqpoint{2.374618in}{3.352800in}}%
\pgfpathlineto{\pgfqpoint{2.371483in}{3.352800in}}%
\pgfpathlineto{\pgfqpoint{2.368348in}{3.352800in}}%
\pgfpathlineto{\pgfqpoint{2.365214in}{3.352800in}}%
\pgfpathlineto{\pgfqpoint{2.362079in}{3.352800in}}%
\pgfpathlineto{\pgfqpoint{2.358944in}{3.352800in}}%
\pgfpathlineto{\pgfqpoint{2.355809in}{3.352800in}}%
\pgfpathlineto{\pgfqpoint{2.352675in}{3.352800in}}%
\pgfpathlineto{\pgfqpoint{2.349540in}{3.352800in}}%
\pgfpathlineto{\pgfqpoint{2.346405in}{3.352800in}}%
\pgfpathlineto{\pgfqpoint{2.343270in}{3.352800in}}%
\pgfpathlineto{\pgfqpoint{2.340136in}{3.352800in}}%
\pgfpathlineto{\pgfqpoint{2.337001in}{3.352800in}}%
\pgfpathlineto{\pgfqpoint{2.333866in}{3.352800in}}%
\pgfpathlineto{\pgfqpoint{2.330731in}{3.352800in}}%
\pgfpathlineto{\pgfqpoint{2.327597in}{3.352800in}}%
\pgfpathlineto{\pgfqpoint{2.324462in}{3.352800in}}%
\pgfpathlineto{\pgfqpoint{2.321327in}{3.352800in}}%
\pgfpathlineto{\pgfqpoint{2.318192in}{3.352800in}}%
\pgfpathlineto{\pgfqpoint{2.315058in}{3.352800in}}%
\pgfpathlineto{\pgfqpoint{2.311923in}{3.352800in}}%
\pgfpathlineto{\pgfqpoint{2.308788in}{3.352800in}}%
\pgfpathlineto{\pgfqpoint{2.305654in}{3.352800in}}%
\pgfpathlineto{\pgfqpoint{2.302519in}{3.352800in}}%
\pgfpathlineto{\pgfqpoint{2.299384in}{3.352800in}}%
\pgfpathlineto{\pgfqpoint{2.296249in}{3.352800in}}%
\pgfpathlineto{\pgfqpoint{2.293115in}{3.352800in}}%
\pgfpathlineto{\pgfqpoint{2.289980in}{3.352800in}}%
\pgfpathlineto{\pgfqpoint{2.286845in}{3.352800in}}%
\pgfpathlineto{\pgfqpoint{2.283710in}{3.352800in}}%
\pgfpathlineto{\pgfqpoint{2.280576in}{3.352800in}}%
\pgfpathlineto{\pgfqpoint{2.277441in}{3.352800in}}%
\pgfpathlineto{\pgfqpoint{2.274306in}{3.352800in}}%
\pgfpathlineto{\pgfqpoint{2.271171in}{3.352800in}}%
\pgfpathlineto{\pgfqpoint{2.268037in}{3.352800in}}%
\pgfpathlineto{\pgfqpoint{2.264902in}{3.352800in}}%
\pgfpathlineto{\pgfqpoint{2.261767in}{3.352800in}}%
\pgfpathlineto{\pgfqpoint{2.258632in}{3.352800in}}%
\pgfpathlineto{\pgfqpoint{2.255498in}{3.352800in}}%
\pgfpathlineto{\pgfqpoint{2.252363in}{3.352800in}}%
\pgfpathlineto{\pgfqpoint{2.249228in}{3.352800in}}%
\pgfpathlineto{\pgfqpoint{2.246093in}{3.352800in}}%
\pgfpathlineto{\pgfqpoint{2.242959in}{3.352800in}}%
\pgfpathlineto{\pgfqpoint{2.239824in}{3.352800in}}%
\pgfpathlineto{\pgfqpoint{2.236689in}{3.352800in}}%
\pgfpathlineto{\pgfqpoint{2.233554in}{3.352800in}}%
\pgfpathlineto{\pgfqpoint{2.230420in}{3.352800in}}%
\pgfpathlineto{\pgfqpoint{2.227285in}{3.352800in}}%
\pgfpathlineto{\pgfqpoint{2.224150in}{3.352800in}}%
\pgfpathlineto{\pgfqpoint{2.221015in}{3.352800in}}%
\pgfpathlineto{\pgfqpoint{2.217881in}{3.352800in}}%
\pgfpathlineto{\pgfqpoint{2.214746in}{3.352800in}}%
\pgfpathlineto{\pgfqpoint{2.211611in}{3.352800in}}%
\pgfpathlineto{\pgfqpoint{2.208477in}{3.352800in}}%
\pgfpathlineto{\pgfqpoint{2.205342in}{3.352800in}}%
\pgfpathlineto{\pgfqpoint{2.202207in}{3.352800in}}%
\pgfpathlineto{\pgfqpoint{2.199072in}{3.352800in}}%
\pgfpathlineto{\pgfqpoint{2.195938in}{3.352800in}}%
\pgfpathlineto{\pgfqpoint{2.192803in}{3.352800in}}%
\pgfpathlineto{\pgfqpoint{2.189668in}{3.352800in}}%
\pgfpathlineto{\pgfqpoint{2.186533in}{3.352800in}}%
\pgfpathlineto{\pgfqpoint{2.183399in}{3.352800in}}%
\pgfpathlineto{\pgfqpoint{2.180264in}{3.352800in}}%
\pgfpathlineto{\pgfqpoint{2.177129in}{3.352800in}}%
\pgfpathlineto{\pgfqpoint{2.173994in}{3.352800in}}%
\pgfpathlineto{\pgfqpoint{2.170860in}{3.352800in}}%
\pgfpathlineto{\pgfqpoint{2.167725in}{3.352800in}}%
\pgfpathlineto{\pgfqpoint{2.164590in}{3.352800in}}%
\pgfpathlineto{\pgfqpoint{2.161455in}{3.352800in}}%
\pgfpathlineto{\pgfqpoint{2.158321in}{3.352800in}}%
\pgfpathlineto{\pgfqpoint{2.155186in}{3.352800in}}%
\pgfpathlineto{\pgfqpoint{2.152051in}{3.352800in}}%
\pgfpathlineto{\pgfqpoint{2.148916in}{3.352800in}}%
\pgfpathlineto{\pgfqpoint{2.145782in}{3.352800in}}%
\pgfpathlineto{\pgfqpoint{2.142647in}{3.352800in}}%
\pgfpathlineto{\pgfqpoint{2.139512in}{3.352800in}}%
\pgfpathlineto{\pgfqpoint{2.136377in}{3.352800in}}%
\pgfpathlineto{\pgfqpoint{2.133243in}{3.352800in}}%
\pgfpathlineto{\pgfqpoint{2.130108in}{3.352800in}}%
\pgfpathlineto{\pgfqpoint{2.126973in}{3.352800in}}%
\pgfpathlineto{\pgfqpoint{2.123838in}{3.352800in}}%
\pgfpathlineto{\pgfqpoint{2.120704in}{3.352800in}}%
\pgfpathlineto{\pgfqpoint{2.117569in}{3.352800in}}%
\pgfpathlineto{\pgfqpoint{2.114434in}{3.352800in}}%
\pgfpathlineto{\pgfqpoint{2.111299in}{3.352800in}}%
\pgfpathlineto{\pgfqpoint{2.108165in}{3.352800in}}%
\pgfpathlineto{\pgfqpoint{2.105030in}{3.352800in}}%
\pgfpathlineto{\pgfqpoint{2.101895in}{3.352800in}}%
\pgfpathlineto{\pgfqpoint{2.098761in}{3.352800in}}%
\pgfpathlineto{\pgfqpoint{2.095626in}{3.352800in}}%
\pgfpathlineto{\pgfqpoint{2.092491in}{3.352800in}}%
\pgfpathlineto{\pgfqpoint{2.089356in}{3.352800in}}%
\pgfpathlineto{\pgfqpoint{2.086222in}{3.352800in}}%
\pgfpathlineto{\pgfqpoint{2.083087in}{3.352800in}}%
\pgfpathlineto{\pgfqpoint{2.079952in}{3.352800in}}%
\pgfpathlineto{\pgfqpoint{2.076817in}{3.352800in}}%
\pgfpathlineto{\pgfqpoint{2.073683in}{3.352800in}}%
\pgfpathlineto{\pgfqpoint{2.070548in}{3.352800in}}%
\pgfpathlineto{\pgfqpoint{2.067413in}{3.352800in}}%
\pgfpathlineto{\pgfqpoint{2.064278in}{3.352800in}}%
\pgfpathlineto{\pgfqpoint{2.061144in}{3.352800in}}%
\pgfpathlineto{\pgfqpoint{2.058009in}{3.352800in}}%
\pgfpathlineto{\pgfqpoint{2.054874in}{3.352800in}}%
\pgfpathlineto{\pgfqpoint{2.051739in}{3.352800in}}%
\pgfpathlineto{\pgfqpoint{2.048605in}{3.352800in}}%
\pgfpathlineto{\pgfqpoint{2.045470in}{3.352800in}}%
\pgfpathlineto{\pgfqpoint{2.042335in}{3.352800in}}%
\pgfpathlineto{\pgfqpoint{2.039200in}{3.352800in}}%
\pgfpathlineto{\pgfqpoint{2.036066in}{3.352800in}}%
\pgfpathlineto{\pgfqpoint{2.032931in}{3.352800in}}%
\pgfpathlineto{\pgfqpoint{2.029796in}{3.352800in}}%
\pgfpathlineto{\pgfqpoint{2.026661in}{3.352800in}}%
\pgfpathlineto{\pgfqpoint{2.023527in}{3.352800in}}%
\pgfpathlineto{\pgfqpoint{2.020392in}{3.352800in}}%
\pgfpathlineto{\pgfqpoint{2.017257in}{3.352800in}}%
\pgfpathlineto{\pgfqpoint{2.014122in}{3.352800in}}%
\pgfpathlineto{\pgfqpoint{2.010988in}{3.352800in}}%
\pgfpathlineto{\pgfqpoint{2.007853in}{3.352800in}}%
\pgfpathlineto{\pgfqpoint{2.004718in}{3.352800in}}%
\pgfpathlineto{\pgfqpoint{2.001584in}{3.352800in}}%
\pgfpathlineto{\pgfqpoint{1.998449in}{3.352800in}}%
\pgfpathlineto{\pgfqpoint{1.995314in}{3.352800in}}%
\pgfpathlineto{\pgfqpoint{1.992179in}{3.352800in}}%
\pgfpathlineto{\pgfqpoint{1.989045in}{3.352800in}}%
\pgfpathlineto{\pgfqpoint{1.985910in}{3.352800in}}%
\pgfpathlineto{\pgfqpoint{1.982775in}{3.352800in}}%
\pgfpathlineto{\pgfqpoint{1.979640in}{3.352800in}}%
\pgfpathlineto{\pgfqpoint{1.976506in}{3.352800in}}%
\pgfpathlineto{\pgfqpoint{1.973371in}{3.352800in}}%
\pgfpathlineto{\pgfqpoint{1.970236in}{3.352800in}}%
\pgfpathlineto{\pgfqpoint{1.967101in}{3.352800in}}%
\pgfpathlineto{\pgfqpoint{1.963967in}{3.352800in}}%
\pgfpathlineto{\pgfqpoint{1.960832in}{3.352800in}}%
\pgfpathlineto{\pgfqpoint{1.957697in}{3.352800in}}%
\pgfpathlineto{\pgfqpoint{1.954562in}{3.352800in}}%
\pgfpathlineto{\pgfqpoint{1.951428in}{3.352800in}}%
\pgfpathlineto{\pgfqpoint{1.948293in}{3.352800in}}%
\pgfpathlineto{\pgfqpoint{1.945158in}{3.352800in}}%
\pgfpathlineto{\pgfqpoint{1.942023in}{3.352800in}}%
\pgfpathlineto{\pgfqpoint{1.938889in}{3.352800in}}%
\pgfpathlineto{\pgfqpoint{1.935754in}{3.352800in}}%
\pgfpathlineto{\pgfqpoint{1.932619in}{3.352800in}}%
\pgfpathlineto{\pgfqpoint{1.929484in}{3.352800in}}%
\pgfpathlineto{\pgfqpoint{1.926350in}{3.352800in}}%
\pgfpathlineto{\pgfqpoint{1.923215in}{3.352800in}}%
\pgfpathlineto{\pgfqpoint{1.920080in}{3.352800in}}%
\pgfpathlineto{\pgfqpoint{1.916945in}{3.352800in}}%
\pgfpathlineto{\pgfqpoint{1.913811in}{3.352800in}}%
\pgfpathlineto{\pgfqpoint{1.910676in}{3.352800in}}%
\pgfpathlineto{\pgfqpoint{1.907541in}{3.352800in}}%
\pgfpathlineto{\pgfqpoint{1.904407in}{3.352800in}}%
\pgfpathlineto{\pgfqpoint{1.901272in}{3.352800in}}%
\pgfpathlineto{\pgfqpoint{1.898137in}{3.352800in}}%
\pgfpathlineto{\pgfqpoint{1.895002in}{3.352800in}}%
\pgfpathlineto{\pgfqpoint{1.891868in}{3.352800in}}%
\pgfpathlineto{\pgfqpoint{1.888733in}{3.352800in}}%
\pgfpathlineto{\pgfqpoint{1.885598in}{3.352800in}}%
\pgfpathlineto{\pgfqpoint{1.882463in}{3.352800in}}%
\pgfpathlineto{\pgfqpoint{1.879329in}{3.352800in}}%
\pgfpathlineto{\pgfqpoint{1.876194in}{3.352800in}}%
\pgfpathlineto{\pgfqpoint{1.873059in}{3.352800in}}%
\pgfpathlineto{\pgfqpoint{1.869924in}{3.352800in}}%
\pgfpathlineto{\pgfqpoint{1.866790in}{3.352800in}}%
\pgfpathlineto{\pgfqpoint{1.863655in}{3.352800in}}%
\pgfpathlineto{\pgfqpoint{1.860520in}{3.352800in}}%
\pgfpathlineto{\pgfqpoint{1.857385in}{3.352800in}}%
\pgfpathlineto{\pgfqpoint{1.854251in}{3.352800in}}%
\pgfpathlineto{\pgfqpoint{1.851116in}{3.352800in}}%
\pgfpathlineto{\pgfqpoint{1.847981in}{3.352800in}}%
\pgfpathlineto{\pgfqpoint{1.844846in}{3.352800in}}%
\pgfpathlineto{\pgfqpoint{1.841712in}{3.352800in}}%
\pgfpathlineto{\pgfqpoint{1.838577in}{3.352800in}}%
\pgfpathlineto{\pgfqpoint{1.835442in}{3.352800in}}%
\pgfpathlineto{\pgfqpoint{1.832307in}{3.352800in}}%
\pgfpathlineto{\pgfqpoint{1.829173in}{3.352800in}}%
\pgfpathlineto{\pgfqpoint{1.826038in}{3.352800in}}%
\pgfpathlineto{\pgfqpoint{1.822903in}{3.352800in}}%
\pgfpathlineto{\pgfqpoint{1.819768in}{3.352800in}}%
\pgfpathlineto{\pgfqpoint{1.816634in}{3.352800in}}%
\pgfpathlineto{\pgfqpoint{1.813499in}{3.352800in}}%
\pgfpathlineto{\pgfqpoint{1.810364in}{3.352800in}}%
\pgfpathlineto{\pgfqpoint{1.807229in}{3.352800in}}%
\pgfpathlineto{\pgfqpoint{1.804095in}{3.352800in}}%
\pgfpathlineto{\pgfqpoint{1.800960in}{3.352800in}}%
\pgfpathlineto{\pgfqpoint{1.797825in}{3.352800in}}%
\pgfpathlineto{\pgfqpoint{1.794691in}{3.352800in}}%
\pgfpathlineto{\pgfqpoint{1.791556in}{3.352800in}}%
\pgfpathlineto{\pgfqpoint{1.788421in}{3.352800in}}%
\pgfpathlineto{\pgfqpoint{1.785286in}{3.352800in}}%
\pgfpathlineto{\pgfqpoint{1.782152in}{3.352800in}}%
\pgfpathlineto{\pgfqpoint{1.779017in}{3.352800in}}%
\pgfpathlineto{\pgfqpoint{1.775882in}{3.352800in}}%
\pgfpathlineto{\pgfqpoint{1.772747in}{3.352800in}}%
\pgfpathlineto{\pgfqpoint{1.769613in}{3.352800in}}%
\pgfpathlineto{\pgfqpoint{1.766478in}{3.352800in}}%
\pgfpathlineto{\pgfqpoint{1.763343in}{3.352800in}}%
\pgfpathlineto{\pgfqpoint{1.760208in}{3.352800in}}%
\pgfpathlineto{\pgfqpoint{1.757074in}{3.352800in}}%
\pgfpathlineto{\pgfqpoint{1.753939in}{3.352800in}}%
\pgfpathlineto{\pgfqpoint{1.750804in}{3.352800in}}%
\pgfpathlineto{\pgfqpoint{1.747669in}{3.352800in}}%
\pgfpathlineto{\pgfqpoint{1.744535in}{3.352800in}}%
\pgfpathlineto{\pgfqpoint{1.741400in}{3.352800in}}%
\pgfpathlineto{\pgfqpoint{1.738265in}{3.352800in}}%
\pgfpathlineto{\pgfqpoint{1.735130in}{3.352800in}}%
\pgfpathlineto{\pgfqpoint{1.731996in}{3.352800in}}%
\pgfpathlineto{\pgfqpoint{1.728861in}{3.352800in}}%
\pgfpathlineto{\pgfqpoint{1.725726in}{3.352800in}}%
\pgfpathlineto{\pgfqpoint{1.722591in}{3.352800in}}%
\pgfpathlineto{\pgfqpoint{1.719457in}{3.352800in}}%
\pgfpathlineto{\pgfqpoint{1.716322in}{3.352800in}}%
\pgfpathlineto{\pgfqpoint{1.713187in}{3.352800in}}%
\pgfpathlineto{\pgfqpoint{1.710052in}{3.352800in}}%
\pgfpathlineto{\pgfqpoint{1.706918in}{3.352800in}}%
\pgfpathlineto{\pgfqpoint{1.703783in}{3.352800in}}%
\pgfpathlineto{\pgfqpoint{1.700648in}{3.352800in}}%
\pgfpathlineto{\pgfqpoint{1.697514in}{3.352800in}}%
\pgfpathlineto{\pgfqpoint{1.694379in}{3.352800in}}%
\pgfpathlineto{\pgfqpoint{1.691244in}{3.352800in}}%
\pgfpathlineto{\pgfqpoint{1.688109in}{3.352800in}}%
\pgfpathlineto{\pgfqpoint{1.684975in}{3.352800in}}%
\pgfpathlineto{\pgfqpoint{1.681840in}{3.352800in}}%
\pgfpathlineto{\pgfqpoint{1.678705in}{3.352800in}}%
\pgfpathlineto{\pgfqpoint{1.675570in}{3.352800in}}%
\pgfpathlineto{\pgfqpoint{1.672436in}{3.352800in}}%
\pgfpathlineto{\pgfqpoint{1.669301in}{3.352800in}}%
\pgfpathlineto{\pgfqpoint{1.666166in}{3.352800in}}%
\pgfpathlineto{\pgfqpoint{1.663031in}{3.352800in}}%
\pgfpathlineto{\pgfqpoint{1.659897in}{3.352800in}}%
\pgfpathlineto{\pgfqpoint{1.656762in}{3.352800in}}%
\pgfpathlineto{\pgfqpoint{1.653627in}{3.352800in}}%
\pgfpathlineto{\pgfqpoint{1.650492in}{3.352800in}}%
\pgfpathlineto{\pgfqpoint{1.647358in}{3.352800in}}%
\pgfpathlineto{\pgfqpoint{1.644223in}{3.352800in}}%
\pgfpathlineto{\pgfqpoint{1.641088in}{3.352800in}}%
\pgfpathlineto{\pgfqpoint{1.637953in}{3.352800in}}%
\pgfpathlineto{\pgfqpoint{1.634819in}{3.352800in}}%
\pgfpathlineto{\pgfqpoint{1.631684in}{3.352800in}}%
\pgfpathlineto{\pgfqpoint{1.628549in}{3.352800in}}%
\pgfpathlineto{\pgfqpoint{1.625414in}{3.352800in}}%
\pgfpathlineto{\pgfqpoint{1.622280in}{3.352800in}}%
\pgfpathlineto{\pgfqpoint{1.619145in}{3.352800in}}%
\pgfpathlineto{\pgfqpoint{1.616010in}{3.352800in}}%
\pgfpathlineto{\pgfqpoint{1.612875in}{3.352800in}}%
\pgfpathlineto{\pgfqpoint{1.609741in}{3.352800in}}%
\pgfpathlineto{\pgfqpoint{1.606606in}{3.352800in}}%
\pgfpathlineto{\pgfqpoint{1.603471in}{3.352800in}}%
\pgfpathlineto{\pgfqpoint{1.600337in}{3.352800in}}%
\pgfpathlineto{\pgfqpoint{1.597202in}{3.352800in}}%
\pgfpathlineto{\pgfqpoint{1.594067in}{3.352800in}}%
\pgfpathlineto{\pgfqpoint{1.590932in}{3.352800in}}%
\pgfpathlineto{\pgfqpoint{1.587798in}{3.352800in}}%
\pgfpathlineto{\pgfqpoint{1.584663in}{3.352800in}}%
\pgfpathlineto{\pgfqpoint{1.581528in}{3.352800in}}%
\pgfpathlineto{\pgfqpoint{1.578393in}{3.352800in}}%
\pgfpathlineto{\pgfqpoint{1.575259in}{3.352800in}}%
\pgfpathlineto{\pgfqpoint{1.572124in}{3.352800in}}%
\pgfpathlineto{\pgfqpoint{1.568989in}{3.352800in}}%
\pgfpathlineto{\pgfqpoint{1.565854in}{3.352800in}}%
\pgfpathlineto{\pgfqpoint{1.562720in}{3.352800in}}%
\pgfpathlineto{\pgfqpoint{1.559585in}{3.352800in}}%
\pgfpathlineto{\pgfqpoint{1.556450in}{3.352800in}}%
\pgfpathlineto{\pgfqpoint{1.553315in}{3.352800in}}%
\pgfpathlineto{\pgfqpoint{1.550181in}{3.352800in}}%
\pgfpathlineto{\pgfqpoint{1.547046in}{3.352800in}}%
\pgfpathlineto{\pgfqpoint{1.543911in}{3.352800in}}%
\pgfpathlineto{\pgfqpoint{1.540776in}{3.352800in}}%
\pgfpathlineto{\pgfqpoint{1.537642in}{3.352800in}}%
\pgfpathlineto{\pgfqpoint{1.534507in}{3.352800in}}%
\pgfpathlineto{\pgfqpoint{1.531372in}{3.352800in}}%
\pgfpathlineto{\pgfqpoint{1.528237in}{3.352800in}}%
\pgfpathlineto{\pgfqpoint{1.525103in}{3.352800in}}%
\pgfpathlineto{\pgfqpoint{1.521968in}{3.352800in}}%
\pgfpathlineto{\pgfqpoint{1.518833in}{3.352800in}}%
\pgfpathlineto{\pgfqpoint{1.515698in}{3.352800in}}%
\pgfpathlineto{\pgfqpoint{1.512564in}{3.352800in}}%
\pgfpathlineto{\pgfqpoint{1.509429in}{3.352800in}}%
\pgfpathlineto{\pgfqpoint{1.506294in}{3.352800in}}%
\pgfpathlineto{\pgfqpoint{1.503159in}{3.352800in}}%
\pgfpathlineto{\pgfqpoint{1.500025in}{3.352800in}}%
\pgfpathlineto{\pgfqpoint{1.496890in}{3.352800in}}%
\pgfpathlineto{\pgfqpoint{1.493755in}{3.352800in}}%
\pgfpathlineto{\pgfqpoint{1.490621in}{3.352800in}}%
\pgfpathlineto{\pgfqpoint{1.487486in}{3.352800in}}%
\pgfpathlineto{\pgfqpoint{1.484351in}{3.352800in}}%
\pgfpathlineto{\pgfqpoint{1.481216in}{3.352800in}}%
\pgfpathlineto{\pgfqpoint{1.478082in}{3.352800in}}%
\pgfpathlineto{\pgfqpoint{1.474947in}{3.352800in}}%
\pgfpathlineto{\pgfqpoint{1.471812in}{3.352800in}}%
\pgfpathlineto{\pgfqpoint{1.468677in}{3.352800in}}%
\pgfpathlineto{\pgfqpoint{1.465543in}{3.352800in}}%
\pgfpathlineto{\pgfqpoint{1.462408in}{3.352800in}}%
\pgfpathlineto{\pgfqpoint{1.459273in}{3.352800in}}%
\pgfpathlineto{\pgfqpoint{1.456138in}{3.352800in}}%
\pgfpathlineto{\pgfqpoint{1.453004in}{3.352800in}}%
\pgfpathlineto{\pgfqpoint{1.449869in}{3.352800in}}%
\pgfpathlineto{\pgfqpoint{1.446734in}{3.352800in}}%
\pgfpathlineto{\pgfqpoint{1.443599in}{3.352800in}}%
\pgfpathlineto{\pgfqpoint{1.440465in}{3.352800in}}%
\pgfpathlineto{\pgfqpoint{1.437330in}{3.352800in}}%
\pgfpathlineto{\pgfqpoint{1.434195in}{3.352800in}}%
\pgfpathlineto{\pgfqpoint{1.431060in}{3.352800in}}%
\pgfpathlineto{\pgfqpoint{1.427926in}{3.352800in}}%
\pgfpathlineto{\pgfqpoint{1.424791in}{3.352800in}}%
\pgfpathlineto{\pgfqpoint{1.421656in}{3.352800in}}%
\pgfpathlineto{\pgfqpoint{1.418521in}{3.352800in}}%
\pgfpathlineto{\pgfqpoint{1.415387in}{3.352800in}}%
\pgfpathlineto{\pgfqpoint{1.412252in}{3.352800in}}%
\pgfpathlineto{\pgfqpoint{1.409117in}{3.352800in}}%
\pgfpathlineto{\pgfqpoint{1.405982in}{3.352800in}}%
\pgfpathlineto{\pgfqpoint{1.402848in}{3.352800in}}%
\pgfpathlineto{\pgfqpoint{1.399713in}{3.352800in}}%
\pgfpathlineto{\pgfqpoint{1.396578in}{3.352800in}}%
\pgfpathlineto{\pgfqpoint{1.393444in}{3.352800in}}%
\pgfpathlineto{\pgfqpoint{1.390309in}{3.352800in}}%
\pgfpathlineto{\pgfqpoint{1.387174in}{3.352800in}}%
\pgfpathlineto{\pgfqpoint{1.384039in}{3.352800in}}%
\pgfpathlineto{\pgfqpoint{1.380905in}{3.352800in}}%
\pgfpathlineto{\pgfqpoint{1.377770in}{3.352800in}}%
\pgfpathlineto{\pgfqpoint{1.374635in}{3.352800in}}%
\pgfpathlineto{\pgfqpoint{1.371500in}{3.352800in}}%
\pgfpathlineto{\pgfqpoint{1.368366in}{3.352800in}}%
\pgfpathlineto{\pgfqpoint{1.365231in}{3.352800in}}%
\pgfpathlineto{\pgfqpoint{1.362096in}{3.352800in}}%
\pgfpathlineto{\pgfqpoint{1.358961in}{3.352800in}}%
\pgfpathlineto{\pgfqpoint{1.355827in}{3.352800in}}%
\pgfpathlineto{\pgfqpoint{1.352692in}{3.352800in}}%
\pgfpathlineto{\pgfqpoint{1.349557in}{3.352800in}}%
\pgfpathlineto{\pgfqpoint{1.346422in}{3.352800in}}%
\pgfpathlineto{\pgfqpoint{1.343288in}{3.352800in}}%
\pgfpathlineto{\pgfqpoint{1.340153in}{3.352800in}}%
\pgfpathlineto{\pgfqpoint{1.337018in}{3.352800in}}%
\pgfpathlineto{\pgfqpoint{1.333883in}{3.352800in}}%
\pgfpathlineto{\pgfqpoint{1.330749in}{3.352800in}}%
\pgfpathlineto{\pgfqpoint{1.327614in}{3.352800in}}%
\pgfpathlineto{\pgfqpoint{1.324479in}{3.352800in}}%
\pgfpathlineto{\pgfqpoint{1.321344in}{3.352800in}}%
\pgfpathlineto{\pgfqpoint{1.318210in}{3.352800in}}%
\pgfpathlineto{\pgfqpoint{1.315075in}{3.352800in}}%
\pgfpathlineto{\pgfqpoint{1.311940in}{3.352800in}}%
\pgfpathlineto{\pgfqpoint{1.308805in}{3.352800in}}%
\pgfpathlineto{\pgfqpoint{1.305671in}{3.352800in}}%
\pgfpathlineto{\pgfqpoint{1.302536in}{3.352800in}}%
\pgfpathlineto{\pgfqpoint{1.299401in}{3.352800in}}%
\pgfpathlineto{\pgfqpoint{1.296266in}{3.352800in}}%
\pgfpathlineto{\pgfqpoint{1.293132in}{3.352800in}}%
\pgfpathlineto{\pgfqpoint{1.289997in}{3.352800in}}%
\pgfpathlineto{\pgfqpoint{1.286862in}{3.352800in}}%
\pgfpathlineto{\pgfqpoint{1.283728in}{3.352800in}}%
\pgfpathlineto{\pgfqpoint{1.280593in}{3.352800in}}%
\pgfpathlineto{\pgfqpoint{1.277458in}{3.352800in}}%
\pgfpathlineto{\pgfqpoint{1.274323in}{3.352800in}}%
\pgfpathlineto{\pgfqpoint{1.271189in}{3.352800in}}%
\pgfpathlineto{\pgfqpoint{1.268054in}{3.352800in}}%
\pgfpathlineto{\pgfqpoint{1.264919in}{3.352800in}}%
\pgfpathlineto{\pgfqpoint{1.261784in}{3.352800in}}%
\pgfpathlineto{\pgfqpoint{1.258650in}{3.352800in}}%
\pgfpathlineto{\pgfqpoint{1.255515in}{3.352800in}}%
\pgfpathlineto{\pgfqpoint{1.252380in}{3.352800in}}%
\pgfpathlineto{\pgfqpoint{1.249245in}{3.352800in}}%
\pgfpathlineto{\pgfqpoint{1.246111in}{3.352800in}}%
\pgfpathlineto{\pgfqpoint{1.242976in}{3.352800in}}%
\pgfpathlineto{\pgfqpoint{1.239841in}{3.352800in}}%
\pgfpathlineto{\pgfqpoint{1.236706in}{3.352800in}}%
\pgfpathlineto{\pgfqpoint{1.233572in}{3.352800in}}%
\pgfpathlineto{\pgfqpoint{1.230437in}{3.352800in}}%
\pgfpathlineto{\pgfqpoint{1.227302in}{3.352800in}}%
\pgfpathlineto{\pgfqpoint{1.224167in}{3.352800in}}%
\pgfpathlineto{\pgfqpoint{1.221033in}{3.352800in}}%
\pgfpathlineto{\pgfqpoint{1.217898in}{3.352800in}}%
\pgfpathlineto{\pgfqpoint{1.214763in}{3.352800in}}%
\pgfpathlineto{\pgfqpoint{1.211628in}{3.352800in}}%
\pgfpathlineto{\pgfqpoint{1.208494in}{3.352800in}}%
\pgfpathlineto{\pgfqpoint{1.205359in}{3.352800in}}%
\pgfpathlineto{\pgfqpoint{1.202224in}{3.352800in}}%
\pgfpathlineto{\pgfqpoint{1.199089in}{3.352800in}}%
\pgfpathlineto{\pgfqpoint{1.195955in}{3.352800in}}%
\pgfpathlineto{\pgfqpoint{1.192820in}{3.352800in}}%
\pgfpathlineto{\pgfqpoint{1.189685in}{3.352800in}}%
\pgfpathlineto{\pgfqpoint{1.186551in}{3.352800in}}%
\pgfpathlineto{\pgfqpoint{1.183416in}{3.352800in}}%
\pgfpathlineto{\pgfqpoint{1.180281in}{3.352800in}}%
\pgfpathlineto{\pgfqpoint{1.177146in}{3.352800in}}%
\pgfpathlineto{\pgfqpoint{1.174012in}{3.352800in}}%
\pgfpathlineto{\pgfqpoint{1.170877in}{3.352800in}}%
\pgfpathlineto{\pgfqpoint{1.167742in}{3.352800in}}%
\pgfpathlineto{\pgfqpoint{1.164607in}{3.352800in}}%
\pgfpathlineto{\pgfqpoint{1.161473in}{3.352800in}}%
\pgfpathlineto{\pgfqpoint{1.158338in}{3.352800in}}%
\pgfpathlineto{\pgfqpoint{1.155203in}{3.352800in}}%
\pgfpathlineto{\pgfqpoint{1.152068in}{3.352800in}}%
\pgfpathlineto{\pgfqpoint{1.148934in}{3.352800in}}%
\pgfpathlineto{\pgfqpoint{1.145799in}{3.352800in}}%
\pgfpathlineto{\pgfqpoint{1.142664in}{3.352800in}}%
\pgfpathlineto{\pgfqpoint{1.139529in}{3.352800in}}%
\pgfpathlineto{\pgfqpoint{1.136395in}{3.352800in}}%
\pgfpathlineto{\pgfqpoint{1.133260in}{3.352800in}}%
\pgfpathlineto{\pgfqpoint{1.130125in}{3.352800in}}%
\pgfpathlineto{\pgfqpoint{1.126990in}{3.352800in}}%
\pgfpathlineto{\pgfqpoint{1.123856in}{3.352800in}}%
\pgfpathlineto{\pgfqpoint{1.120721in}{3.352800in}}%
\pgfpathlineto{\pgfqpoint{1.117586in}{3.352800in}}%
\pgfpathlineto{\pgfqpoint{1.114451in}{3.352800in}}%
\pgfpathlineto{\pgfqpoint{1.111317in}{3.352800in}}%
\pgfpathlineto{\pgfqpoint{1.108182in}{3.352800in}}%
\pgfpathlineto{\pgfqpoint{1.105047in}{3.352800in}}%
\pgfpathlineto{\pgfqpoint{1.101912in}{3.352800in}}%
\pgfpathlineto{\pgfqpoint{1.098778in}{3.352800in}}%
\pgfpathlineto{\pgfqpoint{1.095643in}{3.352800in}}%
\pgfpathlineto{\pgfqpoint{1.092508in}{3.352800in}}%
\pgfpathlineto{\pgfqpoint{1.089374in}{3.352800in}}%
\pgfpathlineto{\pgfqpoint{1.086239in}{3.352800in}}%
\pgfpathlineto{\pgfqpoint{1.083104in}{3.352800in}}%
\pgfpathlineto{\pgfqpoint{1.079969in}{3.352800in}}%
\pgfpathlineto{\pgfqpoint{1.076835in}{3.352800in}}%
\pgfpathlineto{\pgfqpoint{1.073700in}{3.352800in}}%
\pgfpathlineto{\pgfqpoint{1.070565in}{3.352800in}}%
\pgfpathlineto{\pgfqpoint{1.067430in}{3.352800in}}%
\pgfpathlineto{\pgfqpoint{1.064296in}{3.352800in}}%
\pgfpathlineto{\pgfqpoint{1.061161in}{3.352800in}}%
\pgfpathlineto{\pgfqpoint{1.058026in}{3.352800in}}%
\pgfpathlineto{\pgfqpoint{1.054891in}{3.352800in}}%
\pgfpathlineto{\pgfqpoint{1.051757in}{3.352800in}}%
\pgfpathlineto{\pgfqpoint{1.048622in}{3.352800in}}%
\pgfpathlineto{\pgfqpoint{1.045487in}{3.352800in}}%
\pgfpathlineto{\pgfqpoint{1.042352in}{3.352800in}}%
\pgfpathlineto{\pgfqpoint{1.039218in}{3.352800in}}%
\pgfpathlineto{\pgfqpoint{1.036083in}{3.352800in}}%
\pgfpathlineto{\pgfqpoint{1.032948in}{3.352800in}}%
\pgfpathlineto{\pgfqpoint{1.029813in}{3.352800in}}%
\pgfpathlineto{\pgfqpoint{1.026679in}{3.352800in}}%
\pgfpathlineto{\pgfqpoint{1.023544in}{3.352800in}}%
\pgfpathlineto{\pgfqpoint{1.020409in}{3.352800in}}%
\pgfpathlineto{\pgfqpoint{1.017274in}{3.352800in}}%
\pgfpathlineto{\pgfqpoint{1.014140in}{3.352800in}}%
\pgfpathlineto{\pgfqpoint{1.011005in}{3.352800in}}%
\pgfpathlineto{\pgfqpoint{1.007870in}{3.352800in}}%
\pgfpathlineto{\pgfqpoint{1.004735in}{3.352800in}}%
\pgfpathlineto{\pgfqpoint{1.001601in}{3.352800in}}%
\pgfpathlineto{\pgfqpoint{0.998466in}{3.352800in}}%
\pgfpathlineto{\pgfqpoint{0.995331in}{3.352800in}}%
\pgfpathlineto{\pgfqpoint{0.992196in}{3.352800in}}%
\pgfpathlineto{\pgfqpoint{0.989062in}{3.352800in}}%
\pgfpathlineto{\pgfqpoint{0.985927in}{3.352800in}}%
\pgfpathlineto{\pgfqpoint{0.982792in}{3.352800in}}%
\pgfpathlineto{\pgfqpoint{0.979658in}{3.352800in}}%
\pgfpathlineto{\pgfqpoint{0.976523in}{3.352800in}}%
\pgfpathlineto{\pgfqpoint{0.973388in}{3.352800in}}%
\pgfpathlineto{\pgfqpoint{0.970253in}{3.352800in}}%
\pgfpathlineto{\pgfqpoint{0.967119in}{3.352800in}}%
\pgfpathlineto{\pgfqpoint{0.963984in}{3.352800in}}%
\pgfpathlineto{\pgfqpoint{0.960849in}{3.352800in}}%
\pgfpathlineto{\pgfqpoint{0.957714in}{3.352800in}}%
\pgfpathlineto{\pgfqpoint{0.954580in}{3.352800in}}%
\pgfpathlineto{\pgfqpoint{0.951445in}{3.352800in}}%
\pgfpathlineto{\pgfqpoint{0.948310in}{3.352800in}}%
\pgfpathlineto{\pgfqpoint{0.945175in}{3.352800in}}%
\pgfpathlineto{\pgfqpoint{0.942041in}{3.352800in}}%
\pgfpathlineto{\pgfqpoint{0.938906in}{3.352800in}}%
\pgfpathlineto{\pgfqpoint{0.935771in}{3.352800in}}%
\pgfpathlineto{\pgfqpoint{0.932636in}{3.352800in}}%
\pgfpathlineto{\pgfqpoint{0.929502in}{3.352800in}}%
\pgfpathlineto{\pgfqpoint{0.926367in}{3.352800in}}%
\pgfpathlineto{\pgfqpoint{0.923232in}{3.352800in}}%
\pgfpathlineto{\pgfqpoint{0.920097in}{3.352800in}}%
\pgfpathlineto{\pgfqpoint{0.916963in}{3.352800in}}%
\pgfpathlineto{\pgfqpoint{0.913828in}{3.352800in}}%
\pgfpathlineto{\pgfqpoint{0.910693in}{3.352800in}}%
\pgfpathlineto{\pgfqpoint{0.907558in}{3.352800in}}%
\pgfpathlineto{\pgfqpoint{0.904424in}{3.352800in}}%
\pgfpathlineto{\pgfqpoint{0.901289in}{3.352800in}}%
\pgfpathlineto{\pgfqpoint{0.898154in}{3.352800in}}%
\pgfpathlineto{\pgfqpoint{0.895019in}{3.352800in}}%
\pgfpathlineto{\pgfqpoint{0.891885in}{3.352800in}}%
\pgfpathlineto{\pgfqpoint{0.888750in}{3.352800in}}%
\pgfpathlineto{\pgfqpoint{0.888750in}{3.349048in}}%
\pgfpathlineto{\pgfqpoint{0.888750in}{3.345297in}}%
\pgfpathlineto{\pgfqpoint{0.888750in}{3.341545in}}%
\pgfpathlineto{\pgfqpoint{0.888750in}{3.337794in}}%
\pgfpathlineto{\pgfqpoint{0.888750in}{3.334042in}}%
\pgfpathlineto{\pgfqpoint{0.888750in}{3.330291in}}%
\pgfpathlineto{\pgfqpoint{0.888750in}{3.326539in}}%
\pgfpathlineto{\pgfqpoint{0.888750in}{3.322788in}}%
\pgfpathlineto{\pgfqpoint{0.888750in}{3.319036in}}%
\pgfpathlineto{\pgfqpoint{0.888750in}{3.315285in}}%
\pgfpathlineto{\pgfqpoint{0.888750in}{3.311533in}}%
\pgfpathlineto{\pgfqpoint{0.888750in}{3.307782in}}%
\pgfpathlineto{\pgfqpoint{0.888750in}{3.304030in}}%
\pgfpathlineto{\pgfqpoint{0.888750in}{3.300279in}}%
\pgfpathlineto{\pgfqpoint{0.888750in}{3.296527in}}%
\pgfpathlineto{\pgfqpoint{0.888750in}{3.292775in}}%
\pgfpathlineto{\pgfqpoint{0.888750in}{3.289024in}}%
\pgfpathlineto{\pgfqpoint{0.888750in}{3.285272in}}%
\pgfpathlineto{\pgfqpoint{0.888750in}{3.281521in}}%
\pgfpathlineto{\pgfqpoint{0.888750in}{3.277769in}}%
\pgfpathlineto{\pgfqpoint{0.888750in}{3.274018in}}%
\pgfpathlineto{\pgfqpoint{0.888750in}{3.270266in}}%
\pgfpathlineto{\pgfqpoint{0.888750in}{3.266515in}}%
\pgfpathlineto{\pgfqpoint{0.888750in}{3.262763in}}%
\pgfpathlineto{\pgfqpoint{0.888750in}{3.259012in}}%
\pgfpathlineto{\pgfqpoint{0.888750in}{3.255260in}}%
\pgfpathlineto{\pgfqpoint{0.888750in}{3.251509in}}%
\pgfpathlineto{\pgfqpoint{0.888750in}{3.247757in}}%
\pgfpathlineto{\pgfqpoint{0.888750in}{3.244005in}}%
\pgfpathlineto{\pgfqpoint{0.888750in}{3.240254in}}%
\pgfpathlineto{\pgfqpoint{0.888750in}{3.236502in}}%
\pgfpathlineto{\pgfqpoint{0.888750in}{3.232751in}}%
\pgfpathlineto{\pgfqpoint{0.888750in}{3.228999in}}%
\pgfpathlineto{\pgfqpoint{0.888750in}{3.225248in}}%
\pgfpathlineto{\pgfqpoint{0.888750in}{3.221496in}}%
\pgfpathlineto{\pgfqpoint{0.888750in}{3.217745in}}%
\pgfpathlineto{\pgfqpoint{0.888750in}{3.213993in}}%
\pgfpathlineto{\pgfqpoint{0.888750in}{3.210242in}}%
\pgfpathlineto{\pgfqpoint{0.888750in}{3.206490in}}%
\pgfpathlineto{\pgfqpoint{0.888750in}{3.202739in}}%
\pgfpathlineto{\pgfqpoint{0.888750in}{3.198987in}}%
\pgfpathlineto{\pgfqpoint{0.888750in}{3.195236in}}%
\pgfpathlineto{\pgfqpoint{0.888750in}{3.191484in}}%
\pgfpathlineto{\pgfqpoint{0.888750in}{3.187732in}}%
\pgfpathlineto{\pgfqpoint{0.888750in}{3.183981in}}%
\pgfpathlineto{\pgfqpoint{0.888750in}{3.180229in}}%
\pgfpathlineto{\pgfqpoint{0.888750in}{3.176478in}}%
\pgfpathlineto{\pgfqpoint{0.888750in}{3.172726in}}%
\pgfpathlineto{\pgfqpoint{0.888750in}{3.168975in}}%
\pgfpathlineto{\pgfqpoint{0.888750in}{3.165223in}}%
\pgfpathlineto{\pgfqpoint{0.888750in}{3.161472in}}%
\pgfpathlineto{\pgfqpoint{0.888750in}{3.157720in}}%
\pgfpathlineto{\pgfqpoint{0.888750in}{3.153969in}}%
\pgfpathlineto{\pgfqpoint{0.888750in}{3.150217in}}%
\pgfpathlineto{\pgfqpoint{0.888750in}{3.146466in}}%
\pgfpathlineto{\pgfqpoint{0.888750in}{3.142714in}}%
\pgfpathlineto{\pgfqpoint{0.888750in}{3.138963in}}%
\pgfpathlineto{\pgfqpoint{0.888750in}{3.135211in}}%
\pgfpathlineto{\pgfqpoint{0.888750in}{3.131459in}}%
\pgfpathlineto{\pgfqpoint{0.888750in}{3.127708in}}%
\pgfpathlineto{\pgfqpoint{0.888750in}{3.123956in}}%
\pgfpathlineto{\pgfqpoint{0.888750in}{3.120205in}}%
\pgfpathlineto{\pgfqpoint{0.888750in}{3.116453in}}%
\pgfpathlineto{\pgfqpoint{0.888750in}{3.112702in}}%
\pgfpathlineto{\pgfqpoint{0.888750in}{3.108950in}}%
\pgfpathlineto{\pgfqpoint{0.888750in}{3.105199in}}%
\pgfpathlineto{\pgfqpoint{0.888750in}{3.101447in}}%
\pgfpathlineto{\pgfqpoint{0.888750in}{3.097696in}}%
\pgfpathlineto{\pgfqpoint{0.888750in}{3.093944in}}%
\pgfpathlineto{\pgfqpoint{0.888750in}{3.090193in}}%
\pgfpathlineto{\pgfqpoint{0.888750in}{3.086441in}}%
\pgfpathlineto{\pgfqpoint{0.888750in}{3.082690in}}%
\pgfpathlineto{\pgfqpoint{0.888750in}{3.078938in}}%
\pgfpathlineto{\pgfqpoint{0.888750in}{3.075186in}}%
\pgfpathlineto{\pgfqpoint{0.888750in}{3.071435in}}%
\pgfpathlineto{\pgfqpoint{0.888750in}{3.067683in}}%
\pgfpathlineto{\pgfqpoint{0.888750in}{3.063932in}}%
\pgfpathlineto{\pgfqpoint{0.888750in}{3.060180in}}%
\pgfpathlineto{\pgfqpoint{0.888750in}{3.056429in}}%
\pgfpathlineto{\pgfqpoint{0.888750in}{3.052677in}}%
\pgfpathlineto{\pgfqpoint{0.888750in}{3.048926in}}%
\pgfpathlineto{\pgfqpoint{0.888750in}{3.045174in}}%
\pgfpathlineto{\pgfqpoint{0.888750in}{3.041423in}}%
\pgfpathlineto{\pgfqpoint{0.888750in}{3.037671in}}%
\pgfpathlineto{\pgfqpoint{0.888750in}{3.033920in}}%
\pgfpathlineto{\pgfqpoint{0.888750in}{3.030168in}}%
\pgfpathlineto{\pgfqpoint{0.888750in}{3.026416in}}%
\pgfpathlineto{\pgfqpoint{0.888750in}{3.022665in}}%
\pgfpathlineto{\pgfqpoint{0.888750in}{3.018913in}}%
\pgfpathlineto{\pgfqpoint{0.888750in}{3.015162in}}%
\pgfpathlineto{\pgfqpoint{0.888750in}{3.011410in}}%
\pgfpathlineto{\pgfqpoint{0.888750in}{3.007659in}}%
\pgfpathlineto{\pgfqpoint{0.888750in}{3.003907in}}%
\pgfpathlineto{\pgfqpoint{0.888750in}{3.000156in}}%
\pgfpathlineto{\pgfqpoint{0.888750in}{2.996404in}}%
\pgfpathlineto{\pgfqpoint{0.888750in}{2.992653in}}%
\pgfpathlineto{\pgfqpoint{0.888750in}{2.988901in}}%
\pgfpathlineto{\pgfqpoint{0.888750in}{2.985150in}}%
\pgfpathlineto{\pgfqpoint{0.888750in}{2.981398in}}%
\pgfpathlineto{\pgfqpoint{0.888750in}{2.977647in}}%
\pgfpathlineto{\pgfqpoint{0.888750in}{2.973895in}}%
\pgfpathlineto{\pgfqpoint{0.888750in}{2.970143in}}%
\pgfpathlineto{\pgfqpoint{0.888750in}{2.966392in}}%
\pgfpathlineto{\pgfqpoint{0.888750in}{2.962640in}}%
\pgfpathlineto{\pgfqpoint{0.888750in}{2.958889in}}%
\pgfpathlineto{\pgfqpoint{0.888750in}{2.955137in}}%
\pgfpathlineto{\pgfqpoint{0.888750in}{2.951386in}}%
\pgfpathlineto{\pgfqpoint{0.888750in}{2.947634in}}%
\pgfpathlineto{\pgfqpoint{0.888750in}{2.943883in}}%
\pgfpathlineto{\pgfqpoint{0.888750in}{2.940131in}}%
\pgfpathlineto{\pgfqpoint{0.888750in}{2.936380in}}%
\pgfpathlineto{\pgfqpoint{0.888750in}{2.932628in}}%
\pgfpathlineto{\pgfqpoint{0.888750in}{2.928877in}}%
\pgfpathlineto{\pgfqpoint{0.888750in}{2.925125in}}%
\pgfpathlineto{\pgfqpoint{0.888750in}{2.921374in}}%
\pgfpathlineto{\pgfqpoint{0.888750in}{2.917622in}}%
\pgfpathlineto{\pgfqpoint{0.888750in}{2.913870in}}%
\pgfpathlineto{\pgfqpoint{0.888750in}{2.910119in}}%
\pgfpathlineto{\pgfqpoint{0.888750in}{2.906367in}}%
\pgfpathlineto{\pgfqpoint{0.888750in}{2.902616in}}%
\pgfpathlineto{\pgfqpoint{0.888750in}{2.898864in}}%
\pgfpathlineto{\pgfqpoint{0.888750in}{2.895113in}}%
\pgfpathlineto{\pgfqpoint{0.888750in}{2.891361in}}%
\pgfpathlineto{\pgfqpoint{0.888750in}{2.887610in}}%
\pgfpathlineto{\pgfqpoint{0.888750in}{2.883858in}}%
\pgfpathlineto{\pgfqpoint{0.888750in}{2.880107in}}%
\pgfpathlineto{\pgfqpoint{0.888750in}{2.876355in}}%
\pgfpathlineto{\pgfqpoint{0.888750in}{2.872604in}}%
\pgfpathlineto{\pgfqpoint{0.888750in}{2.868852in}}%
\pgfpathlineto{\pgfqpoint{0.888750in}{2.865101in}}%
\pgfpathlineto{\pgfqpoint{0.888750in}{2.861349in}}%
\pgfpathlineto{\pgfqpoint{0.888750in}{2.857597in}}%
\pgfpathlineto{\pgfqpoint{0.888750in}{2.853846in}}%
\pgfpathlineto{\pgfqpoint{0.888750in}{2.850094in}}%
\pgfpathlineto{\pgfqpoint{0.888750in}{2.846343in}}%
\pgfpathlineto{\pgfqpoint{0.888750in}{2.842591in}}%
\pgfpathlineto{\pgfqpoint{0.888750in}{2.838840in}}%
\pgfpathlineto{\pgfqpoint{0.888750in}{2.835088in}}%
\pgfpathlineto{\pgfqpoint{0.888750in}{2.831337in}}%
\pgfpathlineto{\pgfqpoint{0.888750in}{2.827585in}}%
\pgfpathlineto{\pgfqpoint{0.888750in}{2.823834in}}%
\pgfpathlineto{\pgfqpoint{0.888750in}{2.820082in}}%
\pgfpathlineto{\pgfqpoint{0.888750in}{2.816331in}}%
\pgfpathlineto{\pgfqpoint{0.888750in}{2.812579in}}%
\pgfpathlineto{\pgfqpoint{0.888750in}{2.808827in}}%
\pgfpathlineto{\pgfqpoint{0.888750in}{2.805076in}}%
\pgfpathlineto{\pgfqpoint{0.888750in}{2.801324in}}%
\pgfpathlineto{\pgfqpoint{0.888750in}{2.797573in}}%
\pgfpathlineto{\pgfqpoint{0.888750in}{2.793821in}}%
\pgfpathlineto{\pgfqpoint{0.888750in}{2.790070in}}%
\pgfpathlineto{\pgfqpoint{0.888750in}{2.786318in}}%
\pgfpathlineto{\pgfqpoint{0.888750in}{2.782567in}}%
\pgfpathlineto{\pgfqpoint{0.888750in}{2.778815in}}%
\pgfpathlineto{\pgfqpoint{0.888750in}{2.775064in}}%
\pgfpathlineto{\pgfqpoint{0.888750in}{2.771312in}}%
\pgfpathlineto{\pgfqpoint{0.888750in}{2.767561in}}%
\pgfpathlineto{\pgfqpoint{0.888750in}{2.763809in}}%
\pgfpathlineto{\pgfqpoint{0.888750in}{2.760058in}}%
\pgfpathlineto{\pgfqpoint{0.888750in}{2.756306in}}%
\pgfpathlineto{\pgfqpoint{0.888750in}{2.752554in}}%
\pgfpathlineto{\pgfqpoint{0.888750in}{2.748803in}}%
\pgfpathlineto{\pgfqpoint{0.888750in}{2.745051in}}%
\pgfpathlineto{\pgfqpoint{0.888750in}{2.741300in}}%
\pgfpathlineto{\pgfqpoint{0.888750in}{2.737548in}}%
\pgfpathlineto{\pgfqpoint{0.888750in}{2.733797in}}%
\pgfpathlineto{\pgfqpoint{0.888750in}{2.730045in}}%
\pgfpathlineto{\pgfqpoint{0.888750in}{2.726294in}}%
\pgfpathlineto{\pgfqpoint{0.888750in}{2.722542in}}%
\pgfpathlineto{\pgfqpoint{0.888750in}{2.718791in}}%
\pgfpathlineto{\pgfqpoint{0.888750in}{2.715039in}}%
\pgfpathlineto{\pgfqpoint{0.888750in}{2.711288in}}%
\pgfpathlineto{\pgfqpoint{0.888750in}{2.707536in}}%
\pgfpathlineto{\pgfqpoint{0.888750in}{2.703785in}}%
\pgfpathlineto{\pgfqpoint{0.888750in}{2.700033in}}%
\pgfpathlineto{\pgfqpoint{0.888750in}{2.696281in}}%
\pgfpathlineto{\pgfqpoint{0.888750in}{2.692530in}}%
\pgfpathlineto{\pgfqpoint{0.888750in}{2.688778in}}%
\pgfpathlineto{\pgfqpoint{0.888750in}{2.685027in}}%
\pgfpathlineto{\pgfqpoint{0.888750in}{2.681275in}}%
\pgfpathlineto{\pgfqpoint{0.888750in}{2.677524in}}%
\pgfpathlineto{\pgfqpoint{0.888750in}{2.673772in}}%
\pgfpathlineto{\pgfqpoint{0.888750in}{2.670021in}}%
\pgfpathlineto{\pgfqpoint{0.888750in}{2.666269in}}%
\pgfpathlineto{\pgfqpoint{0.888750in}{2.662518in}}%
\pgfpathlineto{\pgfqpoint{0.888750in}{2.658766in}}%
\pgfpathlineto{\pgfqpoint{0.888750in}{2.655015in}}%
\pgfpathlineto{\pgfqpoint{0.888750in}{2.651263in}}%
\pgfpathlineto{\pgfqpoint{0.888750in}{2.647512in}}%
\pgfpathlineto{\pgfqpoint{0.888750in}{2.643760in}}%
\pgfpathlineto{\pgfqpoint{0.888750in}{2.640008in}}%
\pgfpathlineto{\pgfqpoint{0.888750in}{2.636257in}}%
\pgfpathlineto{\pgfqpoint{0.888750in}{2.632505in}}%
\pgfpathlineto{\pgfqpoint{0.888750in}{2.628754in}}%
\pgfpathlineto{\pgfqpoint{0.888750in}{2.625002in}}%
\pgfpathlineto{\pgfqpoint{0.888750in}{2.621251in}}%
\pgfpathlineto{\pgfqpoint{0.888750in}{2.617499in}}%
\pgfpathlineto{\pgfqpoint{0.888750in}{2.613748in}}%
\pgfpathlineto{\pgfqpoint{0.888750in}{2.609996in}}%
\pgfpathlineto{\pgfqpoint{0.888750in}{2.606245in}}%
\pgfpathlineto{\pgfqpoint{0.888750in}{2.602493in}}%
\pgfpathlineto{\pgfqpoint{0.888750in}{2.598742in}}%
\pgfpathlineto{\pgfqpoint{0.888750in}{2.594990in}}%
\pgfpathlineto{\pgfqpoint{0.888750in}{2.591238in}}%
\pgfpathlineto{\pgfqpoint{0.888750in}{2.587487in}}%
\pgfpathlineto{\pgfqpoint{0.888750in}{2.583735in}}%
\pgfpathlineto{\pgfqpoint{0.888750in}{2.579984in}}%
\pgfpathlineto{\pgfqpoint{0.888750in}{2.576232in}}%
\pgfpathlineto{\pgfqpoint{0.888750in}{2.572481in}}%
\pgfpathlineto{\pgfqpoint{0.888750in}{2.568729in}}%
\pgfpathlineto{\pgfqpoint{0.888750in}{2.564978in}}%
\pgfpathlineto{\pgfqpoint{0.888750in}{2.561226in}}%
\pgfpathlineto{\pgfqpoint{0.888750in}{2.557475in}}%
\pgfpathlineto{\pgfqpoint{0.888750in}{2.553723in}}%
\pgfpathlineto{\pgfqpoint{0.888750in}{2.549972in}}%
\pgfpathlineto{\pgfqpoint{0.888750in}{2.546220in}}%
\pgfpathlineto{\pgfqpoint{0.888750in}{2.542469in}}%
\pgfpathlineto{\pgfqpoint{0.888750in}{2.538717in}}%
\pgfpathlineto{\pgfqpoint{0.888750in}{2.534965in}}%
\pgfpathlineto{\pgfqpoint{0.888750in}{2.531214in}}%
\pgfpathlineto{\pgfqpoint{0.888750in}{2.527462in}}%
\pgfpathlineto{\pgfqpoint{0.888750in}{2.523711in}}%
\pgfpathlineto{\pgfqpoint{0.888750in}{2.519959in}}%
\pgfpathlineto{\pgfqpoint{0.888750in}{2.516208in}}%
\pgfpathlineto{\pgfqpoint{0.888750in}{2.512456in}}%
\pgfpathlineto{\pgfqpoint{0.888750in}{2.508705in}}%
\pgfpathlineto{\pgfqpoint{0.888750in}{2.504953in}}%
\pgfpathlineto{\pgfqpoint{0.888750in}{2.501202in}}%
\pgfpathlineto{\pgfqpoint{0.888750in}{2.497450in}}%
\pgfpathlineto{\pgfqpoint{0.888750in}{2.493699in}}%
\pgfpathlineto{\pgfqpoint{0.888750in}{2.489947in}}%
\pgfpathlineto{\pgfqpoint{0.888750in}{2.486196in}}%
\pgfpathlineto{\pgfqpoint{0.888750in}{2.482444in}}%
\pgfpathlineto{\pgfqpoint{0.888750in}{2.478692in}}%
\pgfpathlineto{\pgfqpoint{0.888750in}{2.474941in}}%
\pgfpathlineto{\pgfqpoint{0.888750in}{2.471189in}}%
\pgfpathlineto{\pgfqpoint{0.888750in}{2.467438in}}%
\pgfpathlineto{\pgfqpoint{0.888750in}{2.463686in}}%
\pgfpathlineto{\pgfqpoint{0.888750in}{2.459935in}}%
\pgfpathlineto{\pgfqpoint{0.888750in}{2.456183in}}%
\pgfpathlineto{\pgfqpoint{0.888750in}{2.452432in}}%
\pgfpathlineto{\pgfqpoint{0.888750in}{2.448680in}}%
\pgfpathlineto{\pgfqpoint{0.888750in}{2.444929in}}%
\pgfpathlineto{\pgfqpoint{0.888750in}{2.441177in}}%
\pgfpathlineto{\pgfqpoint{0.888750in}{2.437426in}}%
\pgfpathlineto{\pgfqpoint{0.888750in}{2.433674in}}%
\pgfpathlineto{\pgfqpoint{0.888750in}{2.429923in}}%
\pgfpathlineto{\pgfqpoint{0.888750in}{2.426171in}}%
\pgfpathlineto{\pgfqpoint{0.888750in}{2.422419in}}%
\pgfpathlineto{\pgfqpoint{0.888750in}{2.418668in}}%
\pgfpathlineto{\pgfqpoint{0.888750in}{2.414916in}}%
\pgfpathlineto{\pgfqpoint{0.888750in}{2.411165in}}%
\pgfpathlineto{\pgfqpoint{0.888750in}{2.407413in}}%
\pgfpathlineto{\pgfqpoint{0.888750in}{2.403662in}}%
\pgfpathlineto{\pgfqpoint{0.888750in}{2.399910in}}%
\pgfpathlineto{\pgfqpoint{0.888750in}{2.396159in}}%
\pgfpathlineto{\pgfqpoint{0.888750in}{2.392407in}}%
\pgfpathlineto{\pgfqpoint{0.888750in}{2.388656in}}%
\pgfpathlineto{\pgfqpoint{0.888750in}{2.384904in}}%
\pgfpathlineto{\pgfqpoint{0.888750in}{2.381153in}}%
\pgfpathlineto{\pgfqpoint{0.888750in}{2.377401in}}%
\pgfpathlineto{\pgfqpoint{0.888750in}{2.373649in}}%
\pgfpathlineto{\pgfqpoint{0.888750in}{2.369898in}}%
\pgfpathlineto{\pgfqpoint{0.888750in}{2.366146in}}%
\pgfpathlineto{\pgfqpoint{0.888750in}{2.362395in}}%
\pgfpathlineto{\pgfqpoint{0.888750in}{2.358643in}}%
\pgfpathlineto{\pgfqpoint{0.888750in}{2.354892in}}%
\pgfpathlineto{\pgfqpoint{0.888750in}{2.351140in}}%
\pgfpathlineto{\pgfqpoint{0.888750in}{2.347389in}}%
\pgfpathlineto{\pgfqpoint{0.888750in}{2.343637in}}%
\pgfpathlineto{\pgfqpoint{0.888750in}{2.339886in}}%
\pgfpathlineto{\pgfqpoint{0.888750in}{2.336134in}}%
\pgfpathlineto{\pgfqpoint{0.888750in}{2.332383in}}%
\pgfpathlineto{\pgfqpoint{0.888750in}{2.328631in}}%
\pgfpathlineto{\pgfqpoint{0.888750in}{2.324880in}}%
\pgfpathlineto{\pgfqpoint{0.888750in}{2.321128in}}%
\pgfpathlineto{\pgfqpoint{0.888750in}{2.317376in}}%
\pgfpathlineto{\pgfqpoint{0.888750in}{2.313625in}}%
\pgfpathlineto{\pgfqpoint{0.888750in}{2.309873in}}%
\pgfpathlineto{\pgfqpoint{0.888750in}{2.306122in}}%
\pgfpathlineto{\pgfqpoint{0.888750in}{2.302370in}}%
\pgfpathlineto{\pgfqpoint{0.888750in}{2.298619in}}%
\pgfpathlineto{\pgfqpoint{0.888750in}{2.294867in}}%
\pgfpathlineto{\pgfqpoint{0.888750in}{2.291116in}}%
\pgfpathlineto{\pgfqpoint{0.888750in}{2.287364in}}%
\pgfpathlineto{\pgfqpoint{0.888750in}{2.283613in}}%
\pgfpathlineto{\pgfqpoint{0.888750in}{2.279861in}}%
\pgfpathlineto{\pgfqpoint{0.888750in}{2.276110in}}%
\pgfpathlineto{\pgfqpoint{0.888750in}{2.272358in}}%
\pgfpathlineto{\pgfqpoint{0.888750in}{2.268607in}}%
\pgfpathlineto{\pgfqpoint{0.888750in}{2.264855in}}%
\pgfpathlineto{\pgfqpoint{0.888750in}{2.261103in}}%
\pgfpathlineto{\pgfqpoint{0.888750in}{2.257352in}}%
\pgfpathlineto{\pgfqpoint{0.888750in}{2.253600in}}%
\pgfpathlineto{\pgfqpoint{0.888750in}{2.249849in}}%
\pgfpathlineto{\pgfqpoint{0.888750in}{2.246097in}}%
\pgfpathlineto{\pgfqpoint{0.888750in}{2.242346in}}%
\pgfpathlineto{\pgfqpoint{0.888750in}{2.238594in}}%
\pgfpathlineto{\pgfqpoint{0.888750in}{2.234843in}}%
\pgfpathlineto{\pgfqpoint{0.888750in}{2.231091in}}%
\pgfpathlineto{\pgfqpoint{0.888750in}{2.227340in}}%
\pgfpathlineto{\pgfqpoint{0.888750in}{2.223588in}}%
\pgfpathlineto{\pgfqpoint{0.888750in}{2.219837in}}%
\pgfpathlineto{\pgfqpoint{0.888750in}{2.216085in}}%
\pgfpathlineto{\pgfqpoint{0.888750in}{2.212334in}}%
\pgfpathlineto{\pgfqpoint{0.888750in}{2.208582in}}%
\pgfpathlineto{\pgfqpoint{0.888750in}{2.204830in}}%
\pgfpathlineto{\pgfqpoint{0.888750in}{2.201079in}}%
\pgfpathlineto{\pgfqpoint{0.888750in}{2.197327in}}%
\pgfpathlineto{\pgfqpoint{0.888750in}{2.193576in}}%
\pgfpathlineto{\pgfqpoint{0.888750in}{2.189824in}}%
\pgfpathlineto{\pgfqpoint{0.888750in}{2.186073in}}%
\pgfpathlineto{\pgfqpoint{0.888750in}{2.182321in}}%
\pgfpathlineto{\pgfqpoint{0.888750in}{2.178570in}}%
\pgfpathlineto{\pgfqpoint{0.888750in}{2.174818in}}%
\pgfpathlineto{\pgfqpoint{0.888750in}{2.171067in}}%
\pgfpathlineto{\pgfqpoint{0.888750in}{2.167315in}}%
\pgfpathlineto{\pgfqpoint{0.888750in}{2.163564in}}%
\pgfpathlineto{\pgfqpoint{0.888750in}{2.159812in}}%
\pgfpathlineto{\pgfqpoint{0.888750in}{2.156060in}}%
\pgfpathlineto{\pgfqpoint{0.888750in}{2.152309in}}%
\pgfpathlineto{\pgfqpoint{0.888750in}{2.148557in}}%
\pgfpathlineto{\pgfqpoint{0.888750in}{2.144806in}}%
\pgfpathlineto{\pgfqpoint{0.888750in}{2.141054in}}%
\pgfpathlineto{\pgfqpoint{0.888750in}{2.137303in}}%
\pgfpathlineto{\pgfqpoint{0.888750in}{2.133551in}}%
\pgfpathlineto{\pgfqpoint{0.888750in}{2.129800in}}%
\pgfpathlineto{\pgfqpoint{0.888750in}{2.126048in}}%
\pgfpathlineto{\pgfqpoint{0.888750in}{2.122297in}}%
\pgfpathlineto{\pgfqpoint{0.888750in}{2.118545in}}%
\pgfpathlineto{\pgfqpoint{0.888750in}{2.114794in}}%
\pgfpathlineto{\pgfqpoint{0.888750in}{2.111042in}}%
\pgfpathlineto{\pgfqpoint{0.888750in}{2.107291in}}%
\pgfpathlineto{\pgfqpoint{0.888750in}{2.103539in}}%
\pgfpathlineto{\pgfqpoint{0.888750in}{2.099787in}}%
\pgfpathlineto{\pgfqpoint{0.888750in}{2.096036in}}%
\pgfpathlineto{\pgfqpoint{0.888750in}{2.092284in}}%
\pgfpathlineto{\pgfqpoint{0.888750in}{2.088533in}}%
\pgfpathlineto{\pgfqpoint{0.888750in}{2.084781in}}%
\pgfpathlineto{\pgfqpoint{0.888750in}{2.081030in}}%
\pgfpathlineto{\pgfqpoint{0.888750in}{2.077278in}}%
\pgfpathlineto{\pgfqpoint{0.888750in}{2.073527in}}%
\pgfpathlineto{\pgfqpoint{0.888750in}{2.069775in}}%
\pgfpathlineto{\pgfqpoint{0.888750in}{2.066024in}}%
\pgfpathlineto{\pgfqpoint{0.888750in}{2.062272in}}%
\pgfpathlineto{\pgfqpoint{0.888750in}{2.058521in}}%
\pgfpathlineto{\pgfqpoint{0.888750in}{2.054769in}}%
\pgfpathlineto{\pgfqpoint{0.888750in}{2.051018in}}%
\pgfpathlineto{\pgfqpoint{0.888750in}{2.047266in}}%
\pgfpathlineto{\pgfqpoint{0.888750in}{2.043514in}}%
\pgfpathlineto{\pgfqpoint{0.888750in}{2.039763in}}%
\pgfpathlineto{\pgfqpoint{0.888750in}{2.036011in}}%
\pgfpathlineto{\pgfqpoint{0.888750in}{2.032260in}}%
\pgfpathlineto{\pgfqpoint{0.888750in}{2.028508in}}%
\pgfpathlineto{\pgfqpoint{0.888750in}{2.024757in}}%
\pgfpathlineto{\pgfqpoint{0.888750in}{2.021005in}}%
\pgfpathlineto{\pgfqpoint{0.888750in}{2.017254in}}%
\pgfpathlineto{\pgfqpoint{0.888750in}{2.013502in}}%
\pgfpathlineto{\pgfqpoint{0.888750in}{2.009751in}}%
\pgfpathlineto{\pgfqpoint{0.888750in}{2.005999in}}%
\pgfpathlineto{\pgfqpoint{0.888750in}{2.002248in}}%
\pgfpathlineto{\pgfqpoint{0.888750in}{1.998496in}}%
\pgfpathlineto{\pgfqpoint{0.888750in}{1.994745in}}%
\pgfpathlineto{\pgfqpoint{0.888750in}{1.990993in}}%
\pgfpathlineto{\pgfqpoint{0.888750in}{1.987241in}}%
\pgfpathlineto{\pgfqpoint{0.888750in}{1.983490in}}%
\pgfpathlineto{\pgfqpoint{0.888750in}{1.979738in}}%
\pgfpathlineto{\pgfqpoint{0.888750in}{1.975987in}}%
\pgfpathlineto{\pgfqpoint{0.888750in}{1.972235in}}%
\pgfpathlineto{\pgfqpoint{0.888750in}{1.968484in}}%
\pgfpathlineto{\pgfqpoint{0.888750in}{1.964732in}}%
\pgfpathlineto{\pgfqpoint{0.888750in}{1.960981in}}%
\pgfpathlineto{\pgfqpoint{0.888750in}{1.957229in}}%
\pgfpathlineto{\pgfqpoint{0.888750in}{1.953478in}}%
\pgfpathlineto{\pgfqpoint{0.888750in}{1.949726in}}%
\pgfpathlineto{\pgfqpoint{0.888750in}{1.945975in}}%
\pgfpathlineto{\pgfqpoint{0.888750in}{1.942223in}}%
\pgfpathlineto{\pgfqpoint{0.888750in}{1.938471in}}%
\pgfpathlineto{\pgfqpoint{0.888750in}{1.934720in}}%
\pgfpathlineto{\pgfqpoint{0.888750in}{1.930968in}}%
\pgfpathlineto{\pgfqpoint{0.888750in}{1.927217in}}%
\pgfpathlineto{\pgfqpoint{0.888750in}{1.923465in}}%
\pgfpathlineto{\pgfqpoint{0.888750in}{1.919714in}}%
\pgfpathlineto{\pgfqpoint{0.888750in}{1.915962in}}%
\pgfpathlineto{\pgfqpoint{0.888750in}{1.912211in}}%
\pgfpathlineto{\pgfqpoint{0.888750in}{1.908459in}}%
\pgfpathlineto{\pgfqpoint{0.888750in}{1.904708in}}%
\pgfpathlineto{\pgfqpoint{0.888750in}{1.900956in}}%
\pgfpathlineto{\pgfqpoint{0.888750in}{1.897205in}}%
\pgfpathlineto{\pgfqpoint{0.888750in}{1.893453in}}%
\pgfpathlineto{\pgfqpoint{0.888750in}{1.889702in}}%
\pgfpathlineto{\pgfqpoint{0.888750in}{1.885950in}}%
\pgfpathlineto{\pgfqpoint{0.888750in}{1.882198in}}%
\pgfpathlineto{\pgfqpoint{0.888750in}{1.878447in}}%
\pgfpathlineto{\pgfqpoint{0.888750in}{1.874695in}}%
\pgfpathlineto{\pgfqpoint{0.888750in}{1.870944in}}%
\pgfpathlineto{\pgfqpoint{0.888750in}{1.867192in}}%
\pgfpathlineto{\pgfqpoint{0.888750in}{1.863441in}}%
\pgfpathlineto{\pgfqpoint{0.888750in}{1.859689in}}%
\pgfpathlineto{\pgfqpoint{0.888750in}{1.855938in}}%
\pgfpathlineto{\pgfqpoint{0.888750in}{1.852186in}}%
\pgfpathlineto{\pgfqpoint{0.888750in}{1.848435in}}%
\pgfpathlineto{\pgfqpoint{0.888750in}{1.844683in}}%
\pgfpathlineto{\pgfqpoint{0.888750in}{1.840932in}}%
\pgfpathlineto{\pgfqpoint{0.888750in}{1.837180in}}%
\pgfpathlineto{\pgfqpoint{0.888750in}{1.833429in}}%
\pgfpathlineto{\pgfqpoint{0.888750in}{1.829677in}}%
\pgfpathlineto{\pgfqpoint{0.888750in}{1.825925in}}%
\pgfpathlineto{\pgfqpoint{0.888750in}{1.822174in}}%
\pgfpathlineto{\pgfqpoint{0.888750in}{1.818422in}}%
\pgfpathlineto{\pgfqpoint{0.888750in}{1.814671in}}%
\pgfpathlineto{\pgfqpoint{0.888750in}{1.810919in}}%
\pgfpathlineto{\pgfqpoint{0.888750in}{1.807168in}}%
\pgfpathlineto{\pgfqpoint{0.888750in}{1.803416in}}%
\pgfpathlineto{\pgfqpoint{0.888750in}{1.799665in}}%
\pgfpathlineto{\pgfqpoint{0.888750in}{1.795913in}}%
\pgfpathlineto{\pgfqpoint{0.888750in}{1.792162in}}%
\pgfpathlineto{\pgfqpoint{0.888750in}{1.788410in}}%
\pgfpathlineto{\pgfqpoint{0.888750in}{1.784659in}}%
\pgfpathlineto{\pgfqpoint{0.888750in}{1.780907in}}%
\pgfpathlineto{\pgfqpoint{0.888750in}{1.777155in}}%
\pgfpathlineto{\pgfqpoint{0.888750in}{1.773404in}}%
\pgfpathlineto{\pgfqpoint{0.888750in}{1.769652in}}%
\pgfpathlineto{\pgfqpoint{0.888750in}{1.765901in}}%
\pgfpathlineto{\pgfqpoint{0.888750in}{1.762149in}}%
\pgfpathlineto{\pgfqpoint{0.888750in}{1.758398in}}%
\pgfpathlineto{\pgfqpoint{0.888750in}{1.754646in}}%
\pgfpathlineto{\pgfqpoint{0.888750in}{1.750895in}}%
\pgfpathlineto{\pgfqpoint{0.888750in}{1.747143in}}%
\pgfpathlineto{\pgfqpoint{0.888750in}{1.743392in}}%
\pgfpathlineto{\pgfqpoint{0.888750in}{1.739640in}}%
\pgfpathlineto{\pgfqpoint{0.888750in}{1.735889in}}%
\pgfpathlineto{\pgfqpoint{0.888750in}{1.732137in}}%
\pgfpathlineto{\pgfqpoint{0.888750in}{1.728386in}}%
\pgfpathlineto{\pgfqpoint{0.888750in}{1.724634in}}%
\pgfpathlineto{\pgfqpoint{0.888750in}{1.720882in}}%
\pgfpathlineto{\pgfqpoint{0.888750in}{1.717131in}}%
\pgfpathlineto{\pgfqpoint{0.888750in}{1.713379in}}%
\pgfpathlineto{\pgfqpoint{0.888750in}{1.709628in}}%
\pgfpathlineto{\pgfqpoint{0.888750in}{1.705876in}}%
\pgfpathlineto{\pgfqpoint{0.888750in}{1.702125in}}%
\pgfpathlineto{\pgfqpoint{0.888750in}{1.698373in}}%
\pgfpathlineto{\pgfqpoint{0.888750in}{1.694622in}}%
\pgfpathlineto{\pgfqpoint{0.888750in}{1.690870in}}%
\pgfpathlineto{\pgfqpoint{0.888750in}{1.687119in}}%
\pgfpathlineto{\pgfqpoint{0.888750in}{1.683367in}}%
\pgfpathlineto{\pgfqpoint{0.888750in}{1.679616in}}%
\pgfpathlineto{\pgfqpoint{0.888750in}{1.675864in}}%
\pgfpathlineto{\pgfqpoint{0.888750in}{1.672113in}}%
\pgfpathlineto{\pgfqpoint{0.888750in}{1.668361in}}%
\pgfpathlineto{\pgfqpoint{0.888750in}{1.664609in}}%
\pgfpathlineto{\pgfqpoint{0.888750in}{1.660858in}}%
\pgfpathlineto{\pgfqpoint{0.888750in}{1.657106in}}%
\pgfpathlineto{\pgfqpoint{0.888750in}{1.653355in}}%
\pgfpathlineto{\pgfqpoint{0.888750in}{1.649603in}}%
\pgfpathlineto{\pgfqpoint{0.888750in}{1.645852in}}%
\pgfpathlineto{\pgfqpoint{0.888750in}{1.642100in}}%
\pgfpathlineto{\pgfqpoint{0.888750in}{1.638349in}}%
\pgfpathlineto{\pgfqpoint{0.888750in}{1.634597in}}%
\pgfpathlineto{\pgfqpoint{0.888750in}{1.630846in}}%
\pgfpathlineto{\pgfqpoint{0.888750in}{1.627094in}}%
\pgfpathlineto{\pgfqpoint{0.888750in}{1.623343in}}%
\pgfpathlineto{\pgfqpoint{0.888750in}{1.619591in}}%
\pgfpathlineto{\pgfqpoint{0.888750in}{1.615840in}}%
\pgfpathlineto{\pgfqpoint{0.888750in}{1.612088in}}%
\pgfpathlineto{\pgfqpoint{0.888750in}{1.608336in}}%
\pgfpathlineto{\pgfqpoint{0.888750in}{1.604585in}}%
\pgfpathlineto{\pgfqpoint{0.888750in}{1.600833in}}%
\pgfpathlineto{\pgfqpoint{0.888750in}{1.597082in}}%
\pgfpathlineto{\pgfqpoint{0.888750in}{1.593330in}}%
\pgfpathlineto{\pgfqpoint{0.888750in}{1.589579in}}%
\pgfpathlineto{\pgfqpoint{0.888750in}{1.585827in}}%
\pgfpathlineto{\pgfqpoint{0.888750in}{1.582076in}}%
\pgfpathlineto{\pgfqpoint{0.888750in}{1.578324in}}%
\pgfpathlineto{\pgfqpoint{0.888750in}{1.574573in}}%
\pgfpathlineto{\pgfqpoint{0.888750in}{1.570821in}}%
\pgfpathlineto{\pgfqpoint{0.888750in}{1.567070in}}%
\pgfpathlineto{\pgfqpoint{0.888750in}{1.563318in}}%
\pgfpathlineto{\pgfqpoint{0.888750in}{1.559566in}}%
\pgfpathlineto{\pgfqpoint{0.888750in}{1.555815in}}%
\pgfpathlineto{\pgfqpoint{0.888750in}{1.552063in}}%
\pgfpathlineto{\pgfqpoint{0.888750in}{1.548312in}}%
\pgfpathlineto{\pgfqpoint{0.888750in}{1.544560in}}%
\pgfpathlineto{\pgfqpoint{0.888750in}{1.540809in}}%
\pgfpathlineto{\pgfqpoint{0.888750in}{1.537057in}}%
\pgfpathlineto{\pgfqpoint{0.888750in}{1.533306in}}%
\pgfpathlineto{\pgfqpoint{0.888750in}{1.529554in}}%
\pgfpathlineto{\pgfqpoint{0.888750in}{1.525803in}}%
\pgfpathlineto{\pgfqpoint{0.888750in}{1.522051in}}%
\pgfpathlineto{\pgfqpoint{0.888750in}{1.518300in}}%
\pgfpathlineto{\pgfqpoint{0.888750in}{1.514548in}}%
\pgfpathlineto{\pgfqpoint{0.888750in}{1.510797in}}%
\pgfpathlineto{\pgfqpoint{0.888750in}{1.507045in}}%
\pgfpathlineto{\pgfqpoint{0.888750in}{1.503293in}}%
\pgfpathlineto{\pgfqpoint{0.888750in}{1.499542in}}%
\pgfpathlineto{\pgfqpoint{0.888750in}{1.495790in}}%
\pgfpathlineto{\pgfqpoint{0.888750in}{1.492039in}}%
\pgfpathlineto{\pgfqpoint{0.888750in}{1.488287in}}%
\pgfpathlineto{\pgfqpoint{0.888750in}{1.484536in}}%
\pgfpathlineto{\pgfqpoint{0.888750in}{1.480784in}}%
\pgfpathlineto{\pgfqpoint{0.888750in}{1.477033in}}%
\pgfpathlineto{\pgfqpoint{0.888750in}{1.473281in}}%
\pgfpathlineto{\pgfqpoint{0.888750in}{1.469530in}}%
\pgfpathlineto{\pgfqpoint{0.888750in}{1.465778in}}%
\pgfpathlineto{\pgfqpoint{0.888750in}{1.462027in}}%
\pgfpathlineto{\pgfqpoint{0.888750in}{1.458275in}}%
\pgfpathlineto{\pgfqpoint{0.888750in}{1.454524in}}%
\pgfpathlineto{\pgfqpoint{0.888750in}{1.450772in}}%
\pgfpathlineto{\pgfqpoint{0.888750in}{1.447020in}}%
\pgfpathlineto{\pgfqpoint{0.888750in}{1.443269in}}%
\pgfpathlineto{\pgfqpoint{0.888750in}{1.439517in}}%
\pgfpathlineto{\pgfqpoint{0.888750in}{1.435766in}}%
\pgfpathlineto{\pgfqpoint{0.888750in}{1.432014in}}%
\pgfpathlineto{\pgfqpoint{0.888750in}{1.428263in}}%
\pgfpathlineto{\pgfqpoint{0.888750in}{1.424511in}}%
\pgfpathlineto{\pgfqpoint{0.888750in}{1.420760in}}%
\pgfpathlineto{\pgfqpoint{0.888750in}{1.417008in}}%
\pgfpathlineto{\pgfqpoint{0.888750in}{1.413257in}}%
\pgfpathlineto{\pgfqpoint{0.888750in}{1.409505in}}%
\pgfpathlineto{\pgfqpoint{0.888750in}{1.405754in}}%
\pgfpathlineto{\pgfqpoint{0.888750in}{1.402002in}}%
\pgfpathlineto{\pgfqpoint{0.888750in}{1.398251in}}%
\pgfpathlineto{\pgfqpoint{0.888750in}{1.394499in}}%
\pgfpathlineto{\pgfqpoint{0.888750in}{1.390747in}}%
\pgfpathlineto{\pgfqpoint{0.888750in}{1.386996in}}%
\pgfpathlineto{\pgfqpoint{0.888750in}{1.383244in}}%
\pgfpathlineto{\pgfqpoint{0.888750in}{1.379493in}}%
\pgfpathlineto{\pgfqpoint{0.888750in}{1.375741in}}%
\pgfpathlineto{\pgfqpoint{0.888750in}{1.371990in}}%
\pgfpathlineto{\pgfqpoint{0.888750in}{1.368238in}}%
\pgfpathlineto{\pgfqpoint{0.888750in}{1.364487in}}%
\pgfpathlineto{\pgfqpoint{0.888750in}{1.360735in}}%
\pgfpathlineto{\pgfqpoint{0.888750in}{1.356984in}}%
\pgfpathlineto{\pgfqpoint{0.888750in}{1.353232in}}%
\pgfpathlineto{\pgfqpoint{0.888750in}{1.349481in}}%
\pgfpathlineto{\pgfqpoint{0.888750in}{1.345729in}}%
\pgfpathlineto{\pgfqpoint{0.888750in}{1.341977in}}%
\pgfpathlineto{\pgfqpoint{0.888750in}{1.338226in}}%
\pgfpathlineto{\pgfqpoint{0.888750in}{1.334474in}}%
\pgfpathlineto{\pgfqpoint{0.888750in}{1.330723in}}%
\pgfpathlineto{\pgfqpoint{0.888750in}{1.326971in}}%
\pgfpathlineto{\pgfqpoint{0.888750in}{1.323220in}}%
\pgfpathlineto{\pgfqpoint{0.888750in}{1.319468in}}%
\pgfpathlineto{\pgfqpoint{0.888750in}{1.315717in}}%
\pgfpathlineto{\pgfqpoint{0.888750in}{1.311965in}}%
\pgfpathlineto{\pgfqpoint{0.888750in}{1.308214in}}%
\pgfpathlineto{\pgfqpoint{0.888750in}{1.304462in}}%
\pgfpathlineto{\pgfqpoint{0.888750in}{1.300711in}}%
\pgfpathlineto{\pgfqpoint{0.888750in}{1.296959in}}%
\pgfpathlineto{\pgfqpoint{0.888750in}{1.293208in}}%
\pgfpathlineto{\pgfqpoint{0.888750in}{1.289456in}}%
\pgfpathlineto{\pgfqpoint{0.888750in}{1.285704in}}%
\pgfpathlineto{\pgfqpoint{0.888750in}{1.281953in}}%
\pgfpathlineto{\pgfqpoint{0.888750in}{1.278201in}}%
\pgfpathlineto{\pgfqpoint{0.888750in}{1.274450in}}%
\pgfpathlineto{\pgfqpoint{0.888750in}{1.270698in}}%
\pgfpathlineto{\pgfqpoint{0.888750in}{1.266947in}}%
\pgfpathlineto{\pgfqpoint{0.888750in}{1.263195in}}%
\pgfpathlineto{\pgfqpoint{0.888750in}{1.259444in}}%
\pgfpathlineto{\pgfqpoint{0.888750in}{1.255692in}}%
\pgfpathlineto{\pgfqpoint{0.888750in}{1.251941in}}%
\pgfpathlineto{\pgfqpoint{0.888750in}{1.248189in}}%
\pgfpathlineto{\pgfqpoint{0.888750in}{1.244438in}}%
\pgfpathlineto{\pgfqpoint{0.888750in}{1.240686in}}%
\pgfpathlineto{\pgfqpoint{0.888750in}{1.236935in}}%
\pgfpathlineto{\pgfqpoint{0.888750in}{1.233183in}}%
\pgfpathlineto{\pgfqpoint{0.888750in}{1.229431in}}%
\pgfpathlineto{\pgfqpoint{0.888750in}{1.225680in}}%
\pgfpathlineto{\pgfqpoint{0.888750in}{1.221928in}}%
\pgfpathlineto{\pgfqpoint{0.888750in}{1.218177in}}%
\pgfpathlineto{\pgfqpoint{0.888750in}{1.214425in}}%
\pgfpathlineto{\pgfqpoint{0.888750in}{1.210674in}}%
\pgfpathlineto{\pgfqpoint{0.888750in}{1.206922in}}%
\pgfpathlineto{\pgfqpoint{0.888750in}{1.203171in}}%
\pgfpathlineto{\pgfqpoint{0.888750in}{1.199419in}}%
\pgfpathlineto{\pgfqpoint{0.888750in}{1.195668in}}%
\pgfpathlineto{\pgfqpoint{0.888750in}{1.191916in}}%
\pgfpathlineto{\pgfqpoint{0.888750in}{1.188165in}}%
\pgfpathlineto{\pgfqpoint{0.888750in}{1.184413in}}%
\pgfpathlineto{\pgfqpoint{0.888750in}{1.180662in}}%
\pgfpathlineto{\pgfqpoint{0.888750in}{1.176910in}}%
\pgfpathlineto{\pgfqpoint{0.888750in}{1.173158in}}%
\pgfpathlineto{\pgfqpoint{0.888750in}{1.169407in}}%
\pgfpathlineto{\pgfqpoint{0.888750in}{1.165655in}}%
\pgfpathlineto{\pgfqpoint{0.888750in}{1.161904in}}%
\pgfpathlineto{\pgfqpoint{0.888750in}{1.158152in}}%
\pgfpathlineto{\pgfqpoint{0.888750in}{1.154401in}}%
\pgfpathlineto{\pgfqpoint{0.888750in}{1.150649in}}%
\pgfpathlineto{\pgfqpoint{0.888750in}{1.146898in}}%
\pgfpathlineto{\pgfqpoint{0.888750in}{1.143146in}}%
\pgfpathlineto{\pgfqpoint{0.888750in}{1.139395in}}%
\pgfpathlineto{\pgfqpoint{0.888750in}{1.135643in}}%
\pgfpathlineto{\pgfqpoint{0.888750in}{1.131892in}}%
\pgfpathlineto{\pgfqpoint{0.888750in}{1.128140in}}%
\pgfpathlineto{\pgfqpoint{0.888750in}{1.124388in}}%
\pgfpathlineto{\pgfqpoint{0.888750in}{1.120637in}}%
\pgfpathlineto{\pgfqpoint{0.888750in}{1.116885in}}%
\pgfpathlineto{\pgfqpoint{0.888750in}{1.113134in}}%
\pgfpathlineto{\pgfqpoint{0.888750in}{1.109382in}}%
\pgfpathlineto{\pgfqpoint{0.888750in}{1.105631in}}%
\pgfpathlineto{\pgfqpoint{0.888750in}{1.101879in}}%
\pgfpathlineto{\pgfqpoint{0.888750in}{1.098128in}}%
\pgfpathlineto{\pgfqpoint{0.888750in}{1.094376in}}%
\pgfpathlineto{\pgfqpoint{0.888750in}{1.090625in}}%
\pgfpathlineto{\pgfqpoint{0.888750in}{1.086873in}}%
\pgfpathlineto{\pgfqpoint{0.888750in}{1.083122in}}%
\pgfpathlineto{\pgfqpoint{0.888750in}{1.079370in}}%
\pgfpathlineto{\pgfqpoint{0.888750in}{1.075619in}}%
\pgfpathlineto{\pgfqpoint{0.888750in}{1.071867in}}%
\pgfpathlineto{\pgfqpoint{0.888750in}{1.068115in}}%
\pgfpathlineto{\pgfqpoint{0.888750in}{1.064364in}}%
\pgfpathlineto{\pgfqpoint{0.888750in}{1.060612in}}%
\pgfpathlineto{\pgfqpoint{0.888750in}{1.056861in}}%
\pgfpathlineto{\pgfqpoint{0.888750in}{1.053109in}}%
\pgfpathlineto{\pgfqpoint{0.888750in}{1.049358in}}%
\pgfpathlineto{\pgfqpoint{0.888750in}{1.045606in}}%
\pgfpathlineto{\pgfqpoint{0.888750in}{1.041855in}}%
\pgfpathlineto{\pgfqpoint{0.888750in}{1.038103in}}%
\pgfpathlineto{\pgfqpoint{0.888750in}{1.034352in}}%
\pgfpathlineto{\pgfqpoint{0.888750in}{1.030600in}}%
\pgfpathlineto{\pgfqpoint{0.888750in}{1.026849in}}%
\pgfpathlineto{\pgfqpoint{0.888750in}{1.023097in}}%
\pgfpathlineto{\pgfqpoint{0.888750in}{1.019346in}}%
\pgfpathlineto{\pgfqpoint{0.888750in}{1.015594in}}%
\pgfpathlineto{\pgfqpoint{0.888750in}{1.011842in}}%
\pgfpathlineto{\pgfqpoint{0.888750in}{1.008091in}}%
\pgfpathlineto{\pgfqpoint{0.888750in}{1.004339in}}%
\pgfpathlineto{\pgfqpoint{0.888750in}{1.000588in}}%
\pgfpathlineto{\pgfqpoint{0.888750in}{0.996836in}}%
\pgfpathlineto{\pgfqpoint{0.888750in}{0.993085in}}%
\pgfpathlineto{\pgfqpoint{0.888750in}{0.989333in}}%
\pgfpathlineto{\pgfqpoint{0.888750in}{0.985582in}}%
\pgfpathlineto{\pgfqpoint{0.888750in}{0.981830in}}%
\pgfpathlineto{\pgfqpoint{0.888750in}{0.978079in}}%
\pgfpathlineto{\pgfqpoint{0.888750in}{0.974327in}}%
\pgfpathlineto{\pgfqpoint{0.888750in}{0.970576in}}%
\pgfpathlineto{\pgfqpoint{0.888750in}{0.966824in}}%
\pgfpathlineto{\pgfqpoint{0.888750in}{0.963073in}}%
\pgfpathlineto{\pgfqpoint{0.888750in}{0.959321in}}%
\pgfpathlineto{\pgfqpoint{0.888750in}{0.955569in}}%
\pgfpathlineto{\pgfqpoint{0.888750in}{0.951818in}}%
\pgfpathlineto{\pgfqpoint{0.888750in}{0.948066in}}%
\pgfpathlineto{\pgfqpoint{0.888750in}{0.944315in}}%
\pgfpathlineto{\pgfqpoint{0.888750in}{0.940563in}}%
\pgfpathlineto{\pgfqpoint{0.888750in}{0.936812in}}%
\pgfpathlineto{\pgfqpoint{0.888750in}{0.933060in}}%
\pgfpathlineto{\pgfqpoint{0.888750in}{0.929309in}}%
\pgfpathlineto{\pgfqpoint{0.888750in}{0.925557in}}%
\pgfpathlineto{\pgfqpoint{0.888750in}{0.921806in}}%
\pgfpathlineto{\pgfqpoint{0.888750in}{0.918054in}}%
\pgfpathlineto{\pgfqpoint{0.888750in}{0.914303in}}%
\pgfpathlineto{\pgfqpoint{0.888750in}{0.910551in}}%
\pgfpathlineto{\pgfqpoint{0.888750in}{0.906799in}}%
\pgfpathlineto{\pgfqpoint{0.888750in}{0.903048in}}%
\pgfpathlineto{\pgfqpoint{0.888750in}{0.899296in}}%
\pgfpathlineto{\pgfqpoint{0.888750in}{0.895545in}}%
\pgfpathlineto{\pgfqpoint{0.888750in}{0.891793in}}%
\pgfpathlineto{\pgfqpoint{0.888750in}{0.888042in}}%
\pgfpathlineto{\pgfqpoint{0.888750in}{0.884290in}}%
\pgfpathlineto{\pgfqpoint{0.888750in}{0.880539in}}%
\pgfpathlineto{\pgfqpoint{0.888750in}{0.876787in}}%
\pgfpathlineto{\pgfqpoint{0.888750in}{0.873036in}}%
\pgfpathlineto{\pgfqpoint{0.888750in}{0.869284in}}%
\pgfpathlineto{\pgfqpoint{0.888750in}{0.865533in}}%
\pgfpathlineto{\pgfqpoint{0.888750in}{0.861781in}}%
\pgfpathlineto{\pgfqpoint{0.888750in}{0.858030in}}%
\pgfpathlineto{\pgfqpoint{0.888750in}{0.854278in}}%
\pgfpathlineto{\pgfqpoint{0.888750in}{0.850526in}}%
\pgfpathlineto{\pgfqpoint{0.888750in}{0.846775in}}%
\pgfpathlineto{\pgfqpoint{0.888750in}{0.843023in}}%
\pgfpathlineto{\pgfqpoint{0.888750in}{0.839272in}}%
\pgfpathlineto{\pgfqpoint{0.888750in}{0.835520in}}%
\pgfpathlineto{\pgfqpoint{0.888750in}{0.831769in}}%
\pgfpathlineto{\pgfqpoint{0.888750in}{0.828017in}}%
\pgfpathlineto{\pgfqpoint{0.888750in}{0.824266in}}%
\pgfpathlineto{\pgfqpoint{0.888750in}{0.820514in}}%
\pgfpathlineto{\pgfqpoint{0.888750in}{0.816763in}}%
\pgfpathlineto{\pgfqpoint{0.888750in}{0.813011in}}%
\pgfpathlineto{\pgfqpoint{0.888750in}{0.809260in}}%
\pgfpathlineto{\pgfqpoint{0.888750in}{0.805508in}}%
\pgfpathlineto{\pgfqpoint{0.888750in}{0.801757in}}%
\pgfpathlineto{\pgfqpoint{0.888750in}{0.798005in}}%
\pgfpathlineto{\pgfqpoint{0.888750in}{0.794253in}}%
\pgfpathlineto{\pgfqpoint{0.888750in}{0.790502in}}%
\pgfpathlineto{\pgfqpoint{0.888750in}{0.786750in}}%
\pgfpathlineto{\pgfqpoint{0.888750in}{0.782999in}}%
\pgfpathlineto{\pgfqpoint{0.888750in}{0.779247in}}%
\pgfpathlineto{\pgfqpoint{0.888750in}{0.775496in}}%
\pgfpathlineto{\pgfqpoint{0.888750in}{0.771744in}}%
\pgfpathlineto{\pgfqpoint{0.888750in}{0.767993in}}%
\pgfpathlineto{\pgfqpoint{0.888750in}{0.764241in}}%
\pgfpathlineto{\pgfqpoint{0.888750in}{0.760490in}}%
\pgfpathlineto{\pgfqpoint{0.888750in}{0.756738in}}%
\pgfpathlineto{\pgfqpoint{0.888750in}{0.752987in}}%
\pgfpathlineto{\pgfqpoint{0.888750in}{0.749235in}}%
\pgfpathlineto{\pgfqpoint{0.888750in}{0.745484in}}%
\pgfpathlineto{\pgfqpoint{0.888750in}{0.741732in}}%
\pgfpathlineto{\pgfqpoint{0.888750in}{0.737980in}}%
\pgfpathlineto{\pgfqpoint{0.888750in}{0.734229in}}%
\pgfpathlineto{\pgfqpoint{0.888750in}{0.730477in}}%
\pgfpathlineto{\pgfqpoint{0.888750in}{0.726726in}}%
\pgfpathlineto{\pgfqpoint{0.888750in}{0.722974in}}%
\pgfpathlineto{\pgfqpoint{0.888750in}{0.719223in}}%
\pgfpathlineto{\pgfqpoint{0.888750in}{0.715471in}}%
\pgfpathlineto{\pgfqpoint{0.888750in}{0.711720in}}%
\pgfpathlineto{\pgfqpoint{0.888750in}{0.707968in}}%
\pgfpathlineto{\pgfqpoint{0.888750in}{0.704217in}}%
\pgfpathlineto{\pgfqpoint{0.888750in}{0.700465in}}%
\pgfpathlineto{\pgfqpoint{0.888750in}{0.696714in}}%
\pgfpathlineto{\pgfqpoint{0.888750in}{0.692962in}}%
\pgfpathlineto{\pgfqpoint{0.888750in}{0.689210in}}%
\pgfpathlineto{\pgfqpoint{0.888750in}{0.685459in}}%
\pgfpathlineto{\pgfqpoint{0.888750in}{0.681707in}}%
\pgfpathlineto{\pgfqpoint{0.888750in}{0.677956in}}%
\pgfpathlineto{\pgfqpoint{0.888750in}{0.674204in}}%
\pgfpathlineto{\pgfqpoint{0.888750in}{0.670453in}}%
\pgfpathlineto{\pgfqpoint{0.888750in}{0.666701in}}%
\pgfpathlineto{\pgfqpoint{0.888750in}{0.662950in}}%
\pgfpathlineto{\pgfqpoint{0.888750in}{0.659198in}}%
\pgfpathlineto{\pgfqpoint{0.888750in}{0.655447in}}%
\pgfpathlineto{\pgfqpoint{0.888750in}{0.651695in}}%
\pgfpathlineto{\pgfqpoint{0.888750in}{0.647944in}}%
\pgfpathlineto{\pgfqpoint{0.888750in}{0.644192in}}%
\pgfpathlineto{\pgfqpoint{0.888750in}{0.640441in}}%
\pgfpathlineto{\pgfqpoint{0.888750in}{0.636689in}}%
\pgfpathlineto{\pgfqpoint{0.888750in}{0.632937in}}%
\pgfpathlineto{\pgfqpoint{0.888750in}{0.629186in}}%
\pgfpathlineto{\pgfqpoint{0.888750in}{0.625434in}}%
\pgfpathlineto{\pgfqpoint{0.888750in}{0.621683in}}%
\pgfpathlineto{\pgfqpoint{0.888750in}{0.617931in}}%
\pgfpathlineto{\pgfqpoint{0.888750in}{0.614180in}}%
\pgfpathlineto{\pgfqpoint{0.888750in}{0.610428in}}%
\pgfpathlineto{\pgfqpoint{0.888750in}{0.606677in}}%
\pgfpathlineto{\pgfqpoint{0.888750in}{0.602925in}}%
\pgfpathlineto{\pgfqpoint{0.888750in}{0.599174in}}%
\pgfpathlineto{\pgfqpoint{0.888750in}{0.595422in}}%
\pgfpathlineto{\pgfqpoint{0.888750in}{0.591671in}}%
\pgfpathlineto{\pgfqpoint{0.888750in}{0.587919in}}%
\pgfpathlineto{\pgfqpoint{0.888750in}{0.584168in}}%
\pgfpathlineto{\pgfqpoint{0.888750in}{0.580416in}}%
\pgfpathlineto{\pgfqpoint{0.888750in}{0.576664in}}%
\pgfpathlineto{\pgfqpoint{0.888750in}{0.572913in}}%
\pgfpathlineto{\pgfqpoint{0.888750in}{0.569161in}}%
\pgfpathlineto{\pgfqpoint{0.888750in}{0.565410in}}%
\pgfpathlineto{\pgfqpoint{0.888750in}{0.561658in}}%
\pgfpathlineto{\pgfqpoint{0.888750in}{0.557907in}}%
\pgfpathlineto{\pgfqpoint{0.888750in}{0.554155in}}%
\pgfpathlineto{\pgfqpoint{0.888750in}{0.550404in}}%
\pgfpathlineto{\pgfqpoint{0.888750in}{0.546652in}}%
\pgfpathlineto{\pgfqpoint{0.888750in}{0.542901in}}%
\pgfpathlineto{\pgfqpoint{0.888750in}{0.539149in}}%
\pgfpathlineto{\pgfqpoint{0.888750in}{0.535398in}}%
\pgfpathlineto{\pgfqpoint{0.888750in}{0.531646in}}%
\pgfpathlineto{\pgfqpoint{0.888750in}{0.527895in}}%
\pgfpathlineto{\pgfqpoint{0.888750in}{0.524143in}}%
\pgfpathlineto{\pgfqpoint{0.888750in}{0.520391in}}%
\pgfpathlineto{\pgfqpoint{0.888750in}{0.516640in}}%
\pgfpathlineto{\pgfqpoint{0.888750in}{0.512888in}}%
\pgfpathlineto{\pgfqpoint{0.888750in}{0.509137in}}%
\pgfpathlineto{\pgfqpoint{0.888750in}{0.505385in}}%
\pgfpathlineto{\pgfqpoint{0.888750in}{0.501634in}}%
\pgfpathlineto{\pgfqpoint{0.888750in}{0.497882in}}%
\pgfpathlineto{\pgfqpoint{0.888750in}{0.494131in}}%
\pgfpathlineto{\pgfqpoint{0.888750in}{0.490379in}}%
\pgfpathlineto{\pgfqpoint{0.888750in}{0.486628in}}%
\pgfpathlineto{\pgfqpoint{0.888750in}{0.482876in}}%
\pgfpathlineto{\pgfqpoint{0.888750in}{0.479125in}}%
\pgfpathlineto{\pgfqpoint{0.888750in}{0.475373in}}%
\pgfpathlineto{\pgfqpoint{0.888750in}{0.471621in}}%
\pgfpathlineto{\pgfqpoint{0.888750in}{0.467870in}}%
\pgfpathlineto{\pgfqpoint{0.888750in}{0.464118in}}%
\pgfpathlineto{\pgfqpoint{0.888750in}{0.460367in}}%
\pgfpathlineto{\pgfqpoint{0.888750in}{0.456615in}}%
\pgfpathlineto{\pgfqpoint{0.888750in}{0.452864in}}%
\pgfpathlineto{\pgfqpoint{0.888750in}{0.449112in}}%
\pgfpathlineto{\pgfqpoint{0.888750in}{0.445361in}}%
\pgfpathlineto{\pgfqpoint{0.888750in}{0.441609in}}%
\pgfpathlineto{\pgfqpoint{0.888750in}{0.437858in}}%
\pgfpathlineto{\pgfqpoint{0.888750in}{0.434106in}}%
\pgfpathlineto{\pgfqpoint{0.888750in}{0.430355in}}%
\pgfpathlineto{\pgfqpoint{0.888750in}{0.426603in}}%
\pgfpathlineto{\pgfqpoint{0.888750in}{0.422852in}}%
\pgfpathlineto{\pgfqpoint{0.888750in}{0.419100in}}%
\pgfpathclose%
\pgfpathmoveto{\pgfqpoint{2.171173in}{0.899296in}}%
\pgfpathlineto{\pgfqpoint{2.170860in}{0.899672in}}%
\pgfpathlineto{\pgfqpoint{2.168038in}{0.903048in}}%
\pgfpathlineto{\pgfqpoint{2.167725in}{0.903423in}}%
\pgfpathlineto{\pgfqpoint{2.164590in}{0.903423in}}%
\pgfpathlineto{\pgfqpoint{2.161769in}{0.906799in}}%
\pgfpathlineto{\pgfqpoint{2.161455in}{0.907175in}}%
\pgfpathlineto{\pgfqpoint{2.158634in}{0.910551in}}%
\pgfpathlineto{\pgfqpoint{2.158321in}{0.910926in}}%
\pgfpathlineto{\pgfqpoint{2.155499in}{0.914303in}}%
\pgfpathlineto{\pgfqpoint{2.155186in}{0.914678in}}%
\pgfpathlineto{\pgfqpoint{2.152365in}{0.918054in}}%
\pgfpathlineto{\pgfqpoint{2.152051in}{0.918429in}}%
\pgfpathlineto{\pgfqpoint{2.148916in}{0.918429in}}%
\pgfpathlineto{\pgfqpoint{2.146095in}{0.921806in}}%
\pgfpathlineto{\pgfqpoint{2.145782in}{0.922181in}}%
\pgfpathlineto{\pgfqpoint{2.142960in}{0.925557in}}%
\pgfpathlineto{\pgfqpoint{2.142647in}{0.925932in}}%
\pgfpathlineto{\pgfqpoint{2.139826in}{0.929309in}}%
\pgfpathlineto{\pgfqpoint{2.139512in}{0.929684in}}%
\pgfpathlineto{\pgfqpoint{2.136377in}{0.929684in}}%
\pgfpathlineto{\pgfqpoint{2.133556in}{0.933060in}}%
\pgfpathlineto{\pgfqpoint{2.133243in}{0.933435in}}%
\pgfpathlineto{\pgfqpoint{2.130421in}{0.936812in}}%
\pgfpathlineto{\pgfqpoint{2.130108in}{0.937187in}}%
\pgfpathlineto{\pgfqpoint{2.127287in}{0.940563in}}%
\pgfpathlineto{\pgfqpoint{2.126973in}{0.940938in}}%
\pgfpathlineto{\pgfqpoint{2.123838in}{0.940938in}}%
\pgfpathlineto{\pgfqpoint{2.121017in}{0.944315in}}%
\pgfpathlineto{\pgfqpoint{2.120704in}{0.944690in}}%
\pgfpathlineto{\pgfqpoint{2.117882in}{0.948066in}}%
\pgfpathlineto{\pgfqpoint{2.117569in}{0.948442in}}%
\pgfpathlineto{\pgfqpoint{2.114748in}{0.951818in}}%
\pgfpathlineto{\pgfqpoint{2.114434in}{0.952193in}}%
\pgfpathlineto{\pgfqpoint{2.111299in}{0.952193in}}%
\pgfpathlineto{\pgfqpoint{2.108478in}{0.955569in}}%
\pgfpathlineto{\pgfqpoint{2.108165in}{0.955945in}}%
\pgfpathlineto{\pgfqpoint{2.105343in}{0.959321in}}%
\pgfpathlineto{\pgfqpoint{2.105030in}{0.959696in}}%
\pgfpathlineto{\pgfqpoint{2.102209in}{0.963073in}}%
\pgfpathlineto{\pgfqpoint{2.101895in}{0.963448in}}%
\pgfpathlineto{\pgfqpoint{2.098761in}{0.963448in}}%
\pgfpathlineto{\pgfqpoint{2.095939in}{0.966824in}}%
\pgfpathlineto{\pgfqpoint{2.095626in}{0.967199in}}%
\pgfpathlineto{\pgfqpoint{2.092805in}{0.970576in}}%
\pgfpathlineto{\pgfqpoint{2.092491in}{0.970951in}}%
\pgfpathlineto{\pgfqpoint{2.089670in}{0.974327in}}%
\pgfpathlineto{\pgfqpoint{2.089356in}{0.974702in}}%
\pgfpathlineto{\pgfqpoint{2.086535in}{0.978079in}}%
\pgfpathlineto{\pgfqpoint{2.086222in}{0.978454in}}%
\pgfpathlineto{\pgfqpoint{2.083087in}{0.978454in}}%
\pgfpathlineto{\pgfqpoint{2.080266in}{0.981830in}}%
\pgfpathlineto{\pgfqpoint{2.079952in}{0.982205in}}%
\pgfpathlineto{\pgfqpoint{2.077131in}{0.985582in}}%
\pgfpathlineto{\pgfqpoint{2.076817in}{0.985957in}}%
\pgfpathlineto{\pgfqpoint{2.073996in}{0.989333in}}%
\pgfpathlineto{\pgfqpoint{2.073683in}{0.989708in}}%
\pgfpathlineto{\pgfqpoint{2.070548in}{0.989708in}}%
\pgfpathlineto{\pgfqpoint{2.067727in}{0.993085in}}%
\pgfpathlineto{\pgfqpoint{2.067413in}{0.993460in}}%
\pgfpathlineto{\pgfqpoint{2.064592in}{0.996836in}}%
\pgfpathlineto{\pgfqpoint{2.064278in}{0.997211in}}%
\pgfpathlineto{\pgfqpoint{2.061457in}{1.000588in}}%
\pgfpathlineto{\pgfqpoint{2.061144in}{1.000963in}}%
\pgfpathlineto{\pgfqpoint{2.058009in}{1.000963in}}%
\pgfpathlineto{\pgfqpoint{2.055188in}{1.004339in}}%
\pgfpathlineto{\pgfqpoint{2.054874in}{1.004715in}}%
\pgfpathlineto{\pgfqpoint{2.052053in}{1.008091in}}%
\pgfpathlineto{\pgfqpoint{2.051739in}{1.008466in}}%
\pgfpathlineto{\pgfqpoint{2.048918in}{1.011842in}}%
\pgfpathlineto{\pgfqpoint{2.048605in}{1.012218in}}%
\pgfpathlineto{\pgfqpoint{2.045470in}{1.012218in}}%
\pgfpathlineto{\pgfqpoint{2.042649in}{1.015594in}}%
\pgfpathlineto{\pgfqpoint{2.042335in}{1.015969in}}%
\pgfpathlineto{\pgfqpoint{2.039514in}{1.019346in}}%
\pgfpathlineto{\pgfqpoint{2.039200in}{1.019721in}}%
\pgfpathlineto{\pgfqpoint{2.036379in}{1.023097in}}%
\pgfpathlineto{\pgfqpoint{2.036066in}{1.023472in}}%
\pgfpathlineto{\pgfqpoint{2.032931in}{1.023472in}}%
\pgfpathlineto{\pgfqpoint{2.030110in}{1.026849in}}%
\pgfpathlineto{\pgfqpoint{2.029796in}{1.027224in}}%
\pgfpathlineto{\pgfqpoint{2.026975in}{1.030600in}}%
\pgfpathlineto{\pgfqpoint{2.026661in}{1.030975in}}%
\pgfpathlineto{\pgfqpoint{2.023840in}{1.034352in}}%
\pgfpathlineto{\pgfqpoint{2.023527in}{1.034727in}}%
\pgfpathlineto{\pgfqpoint{2.020392in}{1.034727in}}%
\pgfpathlineto{\pgfqpoint{2.017571in}{1.038103in}}%
\pgfpathlineto{\pgfqpoint{2.017257in}{1.038478in}}%
\pgfpathlineto{\pgfqpoint{2.014436in}{1.041855in}}%
\pgfpathlineto{\pgfqpoint{2.014122in}{1.042230in}}%
\pgfpathlineto{\pgfqpoint{2.011301in}{1.045606in}}%
\pgfpathlineto{\pgfqpoint{2.010988in}{1.045981in}}%
\pgfpathlineto{\pgfqpoint{2.008166in}{1.049358in}}%
\pgfpathlineto{\pgfqpoint{2.007853in}{1.049733in}}%
\pgfpathlineto{\pgfqpoint{2.004718in}{1.049733in}}%
\pgfpathlineto{\pgfqpoint{2.001897in}{1.053109in}}%
\pgfpathlineto{\pgfqpoint{2.001584in}{1.053484in}}%
\pgfpathlineto{\pgfqpoint{1.998762in}{1.056861in}}%
\pgfpathlineto{\pgfqpoint{1.998449in}{1.057236in}}%
\pgfpathlineto{\pgfqpoint{1.995628in}{1.060612in}}%
\pgfpathlineto{\pgfqpoint{1.995314in}{1.060988in}}%
\pgfpathlineto{\pgfqpoint{1.992179in}{1.060988in}}%
\pgfpathlineto{\pgfqpoint{1.989358in}{1.064364in}}%
\pgfpathlineto{\pgfqpoint{1.989045in}{1.064739in}}%
\pgfpathlineto{\pgfqpoint{1.986223in}{1.068115in}}%
\pgfpathlineto{\pgfqpoint{1.985910in}{1.068491in}}%
\pgfpathlineto{\pgfqpoint{1.983089in}{1.071867in}}%
\pgfpathlineto{\pgfqpoint{1.982775in}{1.072242in}}%
\pgfpathlineto{\pgfqpoint{1.979640in}{1.072242in}}%
\pgfpathlineto{\pgfqpoint{1.976819in}{1.075619in}}%
\pgfpathlineto{\pgfqpoint{1.976506in}{1.075994in}}%
\pgfpathlineto{\pgfqpoint{1.973684in}{1.079370in}}%
\pgfpathlineto{\pgfqpoint{1.973371in}{1.079745in}}%
\pgfpathlineto{\pgfqpoint{1.970550in}{1.083122in}}%
\pgfpathlineto{\pgfqpoint{1.970236in}{1.083497in}}%
\pgfpathlineto{\pgfqpoint{1.967101in}{1.083497in}}%
\pgfpathlineto{\pgfqpoint{1.964280in}{1.086873in}}%
\pgfpathlineto{\pgfqpoint{1.963967in}{1.087248in}}%
\pgfpathlineto{\pgfqpoint{1.961145in}{1.090625in}}%
\pgfpathlineto{\pgfqpoint{1.960832in}{1.091000in}}%
\pgfpathlineto{\pgfqpoint{1.958011in}{1.094376in}}%
\pgfpathlineto{\pgfqpoint{1.957697in}{1.094751in}}%
\pgfpathlineto{\pgfqpoint{1.954562in}{1.094751in}}%
\pgfpathlineto{\pgfqpoint{1.951741in}{1.098128in}}%
\pgfpathlineto{\pgfqpoint{1.951428in}{1.098503in}}%
\pgfpathlineto{\pgfqpoint{1.948606in}{1.101879in}}%
\pgfpathlineto{\pgfqpoint{1.948293in}{1.102254in}}%
\pgfpathlineto{\pgfqpoint{1.945472in}{1.105631in}}%
\pgfpathlineto{\pgfqpoint{1.945158in}{1.106006in}}%
\pgfpathlineto{\pgfqpoint{1.942023in}{1.106006in}}%
\pgfpathlineto{\pgfqpoint{1.939202in}{1.109382in}}%
\pgfpathlineto{\pgfqpoint{1.938889in}{1.109758in}}%
\pgfpathlineto{\pgfqpoint{1.936067in}{1.113134in}}%
\pgfpathlineto{\pgfqpoint{1.935754in}{1.113509in}}%
\pgfpathlineto{\pgfqpoint{1.932933in}{1.116885in}}%
\pgfpathlineto{\pgfqpoint{1.932619in}{1.117261in}}%
\pgfpathlineto{\pgfqpoint{1.929798in}{1.120637in}}%
\pgfpathlineto{\pgfqpoint{1.929484in}{1.121012in}}%
\pgfpathlineto{\pgfqpoint{1.926350in}{1.121012in}}%
\pgfpathlineto{\pgfqpoint{1.923528in}{1.124388in}}%
\pgfpathlineto{\pgfqpoint{1.923215in}{1.124764in}}%
\pgfpathlineto{\pgfqpoint{1.920394in}{1.128140in}}%
\pgfpathlineto{\pgfqpoint{1.920080in}{1.128515in}}%
\pgfpathlineto{\pgfqpoint{1.917259in}{1.131892in}}%
\pgfpathlineto{\pgfqpoint{1.916945in}{1.132267in}}%
\pgfpathlineto{\pgfqpoint{1.913811in}{1.132267in}}%
\pgfpathlineto{\pgfqpoint{1.910989in}{1.135643in}}%
\pgfpathlineto{\pgfqpoint{1.910676in}{1.136018in}}%
\pgfpathlineto{\pgfqpoint{1.907855in}{1.139395in}}%
\pgfpathlineto{\pgfqpoint{1.907541in}{1.139770in}}%
\pgfpathlineto{\pgfqpoint{1.904720in}{1.143146in}}%
\pgfpathlineto{\pgfqpoint{1.904407in}{1.143521in}}%
\pgfpathlineto{\pgfqpoint{1.901272in}{1.143521in}}%
\pgfpathlineto{\pgfqpoint{1.898450in}{1.146898in}}%
\pgfpathlineto{\pgfqpoint{1.898137in}{1.147273in}}%
\pgfpathlineto{\pgfqpoint{1.895316in}{1.150649in}}%
\pgfpathlineto{\pgfqpoint{1.895002in}{1.151024in}}%
\pgfpathlineto{\pgfqpoint{1.892181in}{1.154401in}}%
\pgfpathlineto{\pgfqpoint{1.891868in}{1.154776in}}%
\pgfpathlineto{\pgfqpoint{1.888733in}{1.154776in}}%
\pgfpathlineto{\pgfqpoint{1.885912in}{1.158152in}}%
\pgfpathlineto{\pgfqpoint{1.885598in}{1.158527in}}%
\pgfpathlineto{\pgfqpoint{1.882777in}{1.161904in}}%
\pgfpathlineto{\pgfqpoint{1.882463in}{1.162279in}}%
\pgfpathlineto{\pgfqpoint{1.879642in}{1.165655in}}%
\pgfpathlineto{\pgfqpoint{1.879329in}{1.166031in}}%
\pgfpathlineto{\pgfqpoint{1.876194in}{1.166031in}}%
\pgfpathlineto{\pgfqpoint{1.873373in}{1.169407in}}%
\pgfpathlineto{\pgfqpoint{1.873059in}{1.169782in}}%
\pgfpathlineto{\pgfqpoint{1.870238in}{1.173158in}}%
\pgfpathlineto{\pgfqpoint{1.869924in}{1.173534in}}%
\pgfpathlineto{\pgfqpoint{1.867103in}{1.176910in}}%
\pgfpathlineto{\pgfqpoint{1.866790in}{1.177285in}}%
\pgfpathlineto{\pgfqpoint{1.863655in}{1.177285in}}%
\pgfpathlineto{\pgfqpoint{1.860834in}{1.180662in}}%
\pgfpathlineto{\pgfqpoint{1.860520in}{1.181037in}}%
\pgfpathlineto{\pgfqpoint{1.857699in}{1.184413in}}%
\pgfpathlineto{\pgfqpoint{1.857385in}{1.184788in}}%
\pgfpathlineto{\pgfqpoint{1.854564in}{1.188165in}}%
\pgfpathlineto{\pgfqpoint{1.854251in}{1.188540in}}%
\pgfpathlineto{\pgfqpoint{1.851429in}{1.191916in}}%
\pgfpathlineto{\pgfqpoint{1.851116in}{1.192291in}}%
\pgfpathlineto{\pgfqpoint{1.847981in}{1.192291in}}%
\pgfpathlineto{\pgfqpoint{1.845160in}{1.195668in}}%
\pgfpathlineto{\pgfqpoint{1.844846in}{1.196043in}}%
\pgfpathlineto{\pgfqpoint{1.842025in}{1.199419in}}%
\pgfpathlineto{\pgfqpoint{1.841712in}{1.199794in}}%
\pgfpathlineto{\pgfqpoint{1.838890in}{1.203171in}}%
\pgfpathlineto{\pgfqpoint{1.838577in}{1.203546in}}%
\pgfpathlineto{\pgfqpoint{1.835442in}{1.203546in}}%
\pgfpathlineto{\pgfqpoint{1.832621in}{1.206922in}}%
\pgfpathlineto{\pgfqpoint{1.832307in}{1.207297in}}%
\pgfpathlineto{\pgfqpoint{1.829486in}{1.210674in}}%
\pgfpathlineto{\pgfqpoint{1.829173in}{1.211049in}}%
\pgfpathlineto{\pgfqpoint{1.826351in}{1.214425in}}%
\pgfpathlineto{\pgfqpoint{1.826038in}{1.214800in}}%
\pgfpathlineto{\pgfqpoint{1.822903in}{1.214800in}}%
\pgfpathlineto{\pgfqpoint{1.820082in}{1.218177in}}%
\pgfpathlineto{\pgfqpoint{1.819768in}{1.218552in}}%
\pgfpathlineto{\pgfqpoint{1.816947in}{1.221928in}}%
\pgfpathlineto{\pgfqpoint{1.816634in}{1.222304in}}%
\pgfpathlineto{\pgfqpoint{1.813812in}{1.225680in}}%
\pgfpathlineto{\pgfqpoint{1.813499in}{1.226055in}}%
\pgfpathlineto{\pgfqpoint{1.810364in}{1.226055in}}%
\pgfpathlineto{\pgfqpoint{1.807543in}{1.229431in}}%
\pgfpathlineto{\pgfqpoint{1.807229in}{1.229807in}}%
\pgfpathlineto{\pgfqpoint{1.804408in}{1.233183in}}%
\pgfpathlineto{\pgfqpoint{1.804095in}{1.233558in}}%
\pgfpathlineto{\pgfqpoint{1.801273in}{1.236935in}}%
\pgfpathlineto{\pgfqpoint{1.800960in}{1.237310in}}%
\pgfpathlineto{\pgfqpoint{1.797825in}{1.237310in}}%
\pgfpathlineto{\pgfqpoint{1.795004in}{1.240686in}}%
\pgfpathlineto{\pgfqpoint{1.794691in}{1.241061in}}%
\pgfpathlineto{\pgfqpoint{1.791869in}{1.244438in}}%
\pgfpathlineto{\pgfqpoint{1.791556in}{1.244813in}}%
\pgfpathlineto{\pgfqpoint{1.788735in}{1.248189in}}%
\pgfpathlineto{\pgfqpoint{1.788421in}{1.248564in}}%
\pgfpathlineto{\pgfqpoint{1.785600in}{1.251941in}}%
\pgfpathlineto{\pgfqpoint{1.785286in}{1.252316in}}%
\pgfpathlineto{\pgfqpoint{1.782152in}{1.252316in}}%
\pgfpathlineto{\pgfqpoint{1.779330in}{1.255692in}}%
\pgfpathlineto{\pgfqpoint{1.779017in}{1.256067in}}%
\pgfpathlineto{\pgfqpoint{1.776196in}{1.259444in}}%
\pgfpathlineto{\pgfqpoint{1.775882in}{1.259819in}}%
\pgfpathlineto{\pgfqpoint{1.773061in}{1.263195in}}%
\pgfpathlineto{\pgfqpoint{1.772747in}{1.263570in}}%
\pgfpathlineto{\pgfqpoint{1.769613in}{1.263570in}}%
\pgfpathlineto{\pgfqpoint{1.766791in}{1.266947in}}%
\pgfpathlineto{\pgfqpoint{1.766478in}{1.267322in}}%
\pgfpathlineto{\pgfqpoint{1.763657in}{1.270698in}}%
\pgfpathlineto{\pgfqpoint{1.763343in}{1.271073in}}%
\pgfpathlineto{\pgfqpoint{1.760522in}{1.274450in}}%
\pgfpathlineto{\pgfqpoint{1.760208in}{1.274825in}}%
\pgfpathlineto{\pgfqpoint{1.757074in}{1.274825in}}%
\pgfpathlineto{\pgfqpoint{1.754252in}{1.278201in}}%
\pgfpathlineto{\pgfqpoint{1.753939in}{1.278577in}}%
\pgfpathlineto{\pgfqpoint{1.751118in}{1.281953in}}%
\pgfpathlineto{\pgfqpoint{1.750804in}{1.282328in}}%
\pgfpathlineto{\pgfqpoint{1.747983in}{1.285704in}}%
\pgfpathlineto{\pgfqpoint{1.747669in}{1.286080in}}%
\pgfpathlineto{\pgfqpoint{1.744535in}{1.286080in}}%
\pgfpathlineto{\pgfqpoint{1.741713in}{1.289456in}}%
\pgfpathlineto{\pgfqpoint{1.741400in}{1.289831in}}%
\pgfpathlineto{\pgfqpoint{1.738579in}{1.293208in}}%
\pgfpathlineto{\pgfqpoint{1.738265in}{1.293583in}}%
\pgfpathlineto{\pgfqpoint{1.735444in}{1.296959in}}%
\pgfpathlineto{\pgfqpoint{1.735130in}{1.297334in}}%
\pgfpathlineto{\pgfqpoint{1.731996in}{1.297334in}}%
\pgfpathlineto{\pgfqpoint{1.729174in}{1.300711in}}%
\pgfpathlineto{\pgfqpoint{1.728861in}{1.301086in}}%
\pgfpathlineto{\pgfqpoint{1.726040in}{1.304462in}}%
\pgfpathlineto{\pgfqpoint{1.725726in}{1.304837in}}%
\pgfpathlineto{\pgfqpoint{1.722905in}{1.308214in}}%
\pgfpathlineto{\pgfqpoint{1.722591in}{1.308589in}}%
\pgfpathlineto{\pgfqpoint{1.719457in}{1.308589in}}%
\pgfpathlineto{\pgfqpoint{1.716635in}{1.311965in}}%
\pgfpathlineto{\pgfqpoint{1.716322in}{1.312340in}}%
\pgfpathlineto{\pgfqpoint{1.713501in}{1.315717in}}%
\pgfpathlineto{\pgfqpoint{1.713187in}{1.316092in}}%
\pgfpathlineto{\pgfqpoint{1.710366in}{1.319468in}}%
\pgfpathlineto{\pgfqpoint{1.710052in}{1.319843in}}%
\pgfpathlineto{\pgfqpoint{1.707231in}{1.323220in}}%
\pgfpathlineto{\pgfqpoint{1.706918in}{1.323595in}}%
\pgfpathlineto{\pgfqpoint{1.703783in}{1.323595in}}%
\pgfpathlineto{\pgfqpoint{1.700962in}{1.326971in}}%
\pgfpathlineto{\pgfqpoint{1.700648in}{1.327347in}}%
\pgfpathlineto{\pgfqpoint{1.697827in}{1.330723in}}%
\pgfpathlineto{\pgfqpoint{1.697514in}{1.331098in}}%
\pgfpathlineto{\pgfqpoint{1.694692in}{1.334474in}}%
\pgfpathlineto{\pgfqpoint{1.694379in}{1.334850in}}%
\pgfpathlineto{\pgfqpoint{1.691244in}{1.334850in}}%
\pgfpathlineto{\pgfqpoint{1.688423in}{1.338226in}}%
\pgfpathlineto{\pgfqpoint{1.688109in}{1.338601in}}%
\pgfpathlineto{\pgfqpoint{1.685288in}{1.341977in}}%
\pgfpathlineto{\pgfqpoint{1.684975in}{1.342353in}}%
\pgfpathlineto{\pgfqpoint{1.682153in}{1.345729in}}%
\pgfpathlineto{\pgfqpoint{1.681840in}{1.346104in}}%
\pgfpathlineto{\pgfqpoint{1.678705in}{1.346104in}}%
\pgfpathlineto{\pgfqpoint{1.675884in}{1.349481in}}%
\pgfpathlineto{\pgfqpoint{1.675570in}{1.349856in}}%
\pgfpathlineto{\pgfqpoint{1.672749in}{1.353232in}}%
\pgfpathlineto{\pgfqpoint{1.672436in}{1.353607in}}%
\pgfpathlineto{\pgfqpoint{1.669614in}{1.356984in}}%
\pgfpathlineto{\pgfqpoint{1.669301in}{1.357359in}}%
\pgfpathlineto{\pgfqpoint{1.666166in}{1.357359in}}%
\pgfpathlineto{\pgfqpoint{1.663345in}{1.360735in}}%
\pgfpathlineto{\pgfqpoint{1.663031in}{1.361110in}}%
\pgfpathlineto{\pgfqpoint{1.660210in}{1.364487in}}%
\pgfpathlineto{\pgfqpoint{1.659897in}{1.364862in}}%
\pgfpathlineto{\pgfqpoint{1.657075in}{1.368238in}}%
\pgfpathlineto{\pgfqpoint{1.656762in}{1.368613in}}%
\pgfpathlineto{\pgfqpoint{1.653627in}{1.368613in}}%
\pgfpathlineto{\pgfqpoint{1.650806in}{1.371990in}}%
\pgfpathlineto{\pgfqpoint{1.650492in}{1.372365in}}%
\pgfpathlineto{\pgfqpoint{1.647671in}{1.375741in}}%
\pgfpathlineto{\pgfqpoint{1.647358in}{1.376116in}}%
\pgfpathlineto{\pgfqpoint{1.644536in}{1.379493in}}%
\pgfpathlineto{\pgfqpoint{1.644223in}{1.379868in}}%
\pgfpathlineto{\pgfqpoint{1.641088in}{1.379868in}}%
\pgfpathlineto{\pgfqpoint{1.638267in}{1.383244in}}%
\pgfpathlineto{\pgfqpoint{1.637953in}{1.383620in}}%
\pgfpathlineto{\pgfqpoint{1.635132in}{1.386996in}}%
\pgfpathlineto{\pgfqpoint{1.634819in}{1.387371in}}%
\pgfpathlineto{\pgfqpoint{1.631997in}{1.390747in}}%
\pgfpathlineto{\pgfqpoint{1.631684in}{1.391123in}}%
\pgfpathlineto{\pgfqpoint{1.628863in}{1.394499in}}%
\pgfpathlineto{\pgfqpoint{1.628549in}{1.394874in}}%
\pgfpathlineto{\pgfqpoint{1.625414in}{1.394874in}}%
\pgfpathlineto{\pgfqpoint{1.622593in}{1.398251in}}%
\pgfpathlineto{\pgfqpoint{1.622280in}{1.398626in}}%
\pgfpathlineto{\pgfqpoint{1.619458in}{1.402002in}}%
\pgfpathlineto{\pgfqpoint{1.619145in}{1.402377in}}%
\pgfpathlineto{\pgfqpoint{1.616324in}{1.405754in}}%
\pgfpathlineto{\pgfqpoint{1.616010in}{1.406129in}}%
\pgfpathlineto{\pgfqpoint{1.612875in}{1.406129in}}%
\pgfpathlineto{\pgfqpoint{1.610054in}{1.409505in}}%
\pgfpathlineto{\pgfqpoint{1.609741in}{1.409880in}}%
\pgfpathlineto{\pgfqpoint{1.606919in}{1.413257in}}%
\pgfpathlineto{\pgfqpoint{1.606606in}{1.413632in}}%
\pgfpathlineto{\pgfqpoint{1.603785in}{1.417008in}}%
\pgfpathlineto{\pgfqpoint{1.603471in}{1.417383in}}%
\pgfpathlineto{\pgfqpoint{1.600337in}{1.417383in}}%
\pgfpathlineto{\pgfqpoint{1.597515in}{1.420760in}}%
\pgfpathlineto{\pgfqpoint{1.597202in}{1.421135in}}%
\pgfpathlineto{\pgfqpoint{1.594380in}{1.424511in}}%
\pgfpathlineto{\pgfqpoint{1.594067in}{1.424886in}}%
\pgfpathlineto{\pgfqpoint{1.591246in}{1.428263in}}%
\pgfpathlineto{\pgfqpoint{1.590932in}{1.428638in}}%
\pgfpathlineto{\pgfqpoint{1.587798in}{1.428638in}}%
\pgfpathlineto{\pgfqpoint{1.584976in}{1.432014in}}%
\pgfpathlineto{\pgfqpoint{1.584663in}{1.432389in}}%
\pgfpathlineto{\pgfqpoint{1.581842in}{1.435766in}}%
\pgfpathlineto{\pgfqpoint{1.581528in}{1.436141in}}%
\pgfpathlineto{\pgfqpoint{1.578707in}{1.439517in}}%
\pgfpathlineto{\pgfqpoint{1.578393in}{1.439893in}}%
\pgfpathlineto{\pgfqpoint{1.575259in}{1.439893in}}%
\pgfpathlineto{\pgfqpoint{1.572437in}{1.443269in}}%
\pgfpathlineto{\pgfqpoint{1.572124in}{1.443644in}}%
\pgfpathlineto{\pgfqpoint{1.569303in}{1.447020in}}%
\pgfpathlineto{\pgfqpoint{1.568989in}{1.447396in}}%
\pgfpathlineto{\pgfqpoint{1.566168in}{1.450772in}}%
\pgfpathlineto{\pgfqpoint{1.565854in}{1.451147in}}%
\pgfpathlineto{\pgfqpoint{1.562720in}{1.451147in}}%
\pgfpathlineto{\pgfqpoint{1.559898in}{1.454524in}}%
\pgfpathlineto{\pgfqpoint{1.559585in}{1.454899in}}%
\pgfpathlineto{\pgfqpoint{1.556764in}{1.458275in}}%
\pgfpathlineto{\pgfqpoint{1.556450in}{1.458650in}}%
\pgfpathlineto{\pgfqpoint{1.553629in}{1.462027in}}%
\pgfpathlineto{\pgfqpoint{1.553315in}{1.462402in}}%
\pgfpathlineto{\pgfqpoint{1.550494in}{1.465778in}}%
\pgfpathlineto{\pgfqpoint{1.550181in}{1.466153in}}%
\pgfpathlineto{\pgfqpoint{1.547046in}{1.466153in}}%
\pgfpathlineto{\pgfqpoint{1.544225in}{1.469530in}}%
\pgfpathlineto{\pgfqpoint{1.543911in}{1.469905in}}%
\pgfpathlineto{\pgfqpoint{1.541090in}{1.473281in}}%
\pgfpathlineto{\pgfqpoint{1.540776in}{1.473656in}}%
\pgfpathlineto{\pgfqpoint{1.537955in}{1.477033in}}%
\pgfpathlineto{\pgfqpoint{1.537642in}{1.477408in}}%
\pgfpathlineto{\pgfqpoint{1.534507in}{1.477408in}}%
\pgfpathlineto{\pgfqpoint{1.531686in}{1.480784in}}%
\pgfpathlineto{\pgfqpoint{1.531372in}{1.481159in}}%
\pgfpathlineto{\pgfqpoint{1.528551in}{1.484536in}}%
\pgfpathlineto{\pgfqpoint{1.528237in}{1.484911in}}%
\pgfpathlineto{\pgfqpoint{1.525416in}{1.488287in}}%
\pgfpathlineto{\pgfqpoint{1.525103in}{1.488662in}}%
\pgfpathlineto{\pgfqpoint{1.521968in}{1.488662in}}%
\pgfpathlineto{\pgfqpoint{1.519147in}{1.492039in}}%
\pgfpathlineto{\pgfqpoint{1.518833in}{1.492414in}}%
\pgfpathlineto{\pgfqpoint{1.516012in}{1.495790in}}%
\pgfpathlineto{\pgfqpoint{1.515698in}{1.496166in}}%
\pgfpathlineto{\pgfqpoint{1.512877in}{1.499542in}}%
\pgfpathlineto{\pgfqpoint{1.512564in}{1.499917in}}%
\pgfpathlineto{\pgfqpoint{1.509429in}{1.499917in}}%
\pgfpathlineto{\pgfqpoint{1.506608in}{1.503293in}}%
\pgfpathlineto{\pgfqpoint{1.506294in}{1.503669in}}%
\pgfpathlineto{\pgfqpoint{1.503473in}{1.507045in}}%
\pgfpathlineto{\pgfqpoint{1.503159in}{1.507420in}}%
\pgfpathlineto{\pgfqpoint{1.500338in}{1.510797in}}%
\pgfpathlineto{\pgfqpoint{1.500025in}{1.511172in}}%
\pgfpathlineto{\pgfqpoint{1.496890in}{1.511172in}}%
\pgfpathlineto{\pgfqpoint{1.494069in}{1.514548in}}%
\pgfpathlineto{\pgfqpoint{1.493755in}{1.514923in}}%
\pgfpathlineto{\pgfqpoint{1.490934in}{1.518300in}}%
\pgfpathlineto{\pgfqpoint{1.490621in}{1.518675in}}%
\pgfpathlineto{\pgfqpoint{1.487799in}{1.522051in}}%
\pgfpathlineto{\pgfqpoint{1.487486in}{1.522426in}}%
\pgfpathlineto{\pgfqpoint{1.484665in}{1.525803in}}%
\pgfpathlineto{\pgfqpoint{1.484351in}{1.526178in}}%
\pgfpathlineto{\pgfqpoint{1.481216in}{1.526178in}}%
\pgfpathlineto{\pgfqpoint{1.478395in}{1.529554in}}%
\pgfpathlineto{\pgfqpoint{1.478082in}{1.529929in}}%
\pgfpathlineto{\pgfqpoint{1.475260in}{1.533306in}}%
\pgfpathlineto{\pgfqpoint{1.474947in}{1.533681in}}%
\pgfpathlineto{\pgfqpoint{1.472126in}{1.537057in}}%
\pgfpathlineto{\pgfqpoint{1.471812in}{1.537432in}}%
\pgfpathlineto{\pgfqpoint{1.468677in}{1.537432in}}%
\pgfpathlineto{\pgfqpoint{1.465856in}{1.540809in}}%
\pgfpathlineto{\pgfqpoint{1.465543in}{1.541184in}}%
\pgfpathlineto{\pgfqpoint{1.462721in}{1.544560in}}%
\pgfpathlineto{\pgfqpoint{1.462408in}{1.544936in}}%
\pgfpathlineto{\pgfqpoint{1.459587in}{1.548312in}}%
\pgfpathlineto{\pgfqpoint{1.459273in}{1.548687in}}%
\pgfpathlineto{\pgfqpoint{1.456138in}{1.548687in}}%
\pgfpathlineto{\pgfqpoint{1.453317in}{1.552063in}}%
\pgfpathlineto{\pgfqpoint{1.453004in}{1.552439in}}%
\pgfpathlineto{\pgfqpoint{1.450182in}{1.555815in}}%
\pgfpathlineto{\pgfqpoint{1.449869in}{1.556190in}}%
\pgfpathlineto{\pgfqpoint{1.447048in}{1.559566in}}%
\pgfpathlineto{\pgfqpoint{1.446734in}{1.559942in}}%
\pgfpathlineto{\pgfqpoint{1.443599in}{1.559942in}}%
\pgfpathlineto{\pgfqpoint{1.440778in}{1.563318in}}%
\pgfpathlineto{\pgfqpoint{1.440465in}{1.563693in}}%
\pgfpathlineto{\pgfqpoint{1.437643in}{1.567070in}}%
\pgfpathlineto{\pgfqpoint{1.437330in}{1.567445in}}%
\pgfpathlineto{\pgfqpoint{1.434509in}{1.570821in}}%
\pgfpathlineto{\pgfqpoint{1.434195in}{1.571196in}}%
\pgfpathlineto{\pgfqpoint{1.431060in}{1.571196in}}%
\pgfpathlineto{\pgfqpoint{1.428239in}{1.574573in}}%
\pgfpathlineto{\pgfqpoint{1.427926in}{1.574948in}}%
\pgfpathlineto{\pgfqpoint{1.425104in}{1.578324in}}%
\pgfpathlineto{\pgfqpoint{1.424791in}{1.578699in}}%
\pgfpathlineto{\pgfqpoint{1.421970in}{1.582076in}}%
\pgfpathlineto{\pgfqpoint{1.421656in}{1.582451in}}%
\pgfpathlineto{\pgfqpoint{1.418521in}{1.582451in}}%
\pgfpathlineto{\pgfqpoint{1.415700in}{1.585827in}}%
\pgfpathlineto{\pgfqpoint{1.415387in}{1.586202in}}%
\pgfpathlineto{\pgfqpoint{1.412565in}{1.589579in}}%
\pgfpathlineto{\pgfqpoint{1.412252in}{1.589954in}}%
\pgfpathlineto{\pgfqpoint{1.409431in}{1.593330in}}%
\pgfpathlineto{\pgfqpoint{1.409117in}{1.593705in}}%
\pgfpathlineto{\pgfqpoint{1.406296in}{1.597082in}}%
\pgfpathlineto{\pgfqpoint{1.405982in}{1.597457in}}%
\pgfpathlineto{\pgfqpoint{1.402848in}{1.597457in}}%
\pgfpathlineto{\pgfqpoint{1.400026in}{1.600833in}}%
\pgfpathlineto{\pgfqpoint{1.399713in}{1.601209in}}%
\pgfpathlineto{\pgfqpoint{1.396892in}{1.604585in}}%
\pgfpathlineto{\pgfqpoint{1.396578in}{1.604960in}}%
\pgfpathlineto{\pgfqpoint{1.393757in}{1.608336in}}%
\pgfpathlineto{\pgfqpoint{1.393444in}{1.608712in}}%
\pgfpathlineto{\pgfqpoint{1.390309in}{1.608712in}}%
\pgfpathlineto{\pgfqpoint{1.387487in}{1.612088in}}%
\pgfpathlineto{\pgfqpoint{1.387174in}{1.612463in}}%
\pgfpathlineto{\pgfqpoint{1.384353in}{1.615840in}}%
\pgfpathlineto{\pgfqpoint{1.384039in}{1.616215in}}%
\pgfpathlineto{\pgfqpoint{1.381218in}{1.619591in}}%
\pgfpathlineto{\pgfqpoint{1.380905in}{1.619966in}}%
\pgfpathlineto{\pgfqpoint{1.377770in}{1.619966in}}%
\pgfpathlineto{\pgfqpoint{1.374949in}{1.623343in}}%
\pgfpathlineto{\pgfqpoint{1.374635in}{1.623718in}}%
\pgfpathlineto{\pgfqpoint{1.371814in}{1.627094in}}%
\pgfpathlineto{\pgfqpoint{1.371500in}{1.627469in}}%
\pgfpathlineto{\pgfqpoint{1.368679in}{1.630846in}}%
\pgfpathlineto{\pgfqpoint{1.368366in}{1.631221in}}%
\pgfpathlineto{\pgfqpoint{1.365231in}{1.631221in}}%
\pgfpathlineto{\pgfqpoint{1.362410in}{1.634597in}}%
\pgfpathlineto{\pgfqpoint{1.362096in}{1.634972in}}%
\pgfpathlineto{\pgfqpoint{1.359275in}{1.638349in}}%
\pgfpathlineto{\pgfqpoint{1.358961in}{1.638724in}}%
\pgfpathlineto{\pgfqpoint{1.356140in}{1.642100in}}%
\pgfpathlineto{\pgfqpoint{1.355827in}{1.642475in}}%
\pgfpathlineto{\pgfqpoint{1.352692in}{1.642475in}}%
\pgfpathlineto{\pgfqpoint{1.349871in}{1.645852in}}%
\pgfpathlineto{\pgfqpoint{1.349557in}{1.646227in}}%
\pgfpathlineto{\pgfqpoint{1.346736in}{1.649603in}}%
\pgfpathlineto{\pgfqpoint{1.346422in}{1.649978in}}%
\pgfpathlineto{\pgfqpoint{1.343601in}{1.653355in}}%
\pgfpathlineto{\pgfqpoint{1.343288in}{1.653730in}}%
\pgfpathlineto{\pgfqpoint{1.340153in}{1.653730in}}%
\pgfpathlineto{\pgfqpoint{1.337332in}{1.657106in}}%
\pgfpathlineto{\pgfqpoint{1.337018in}{1.657482in}}%
\pgfpathlineto{\pgfqpoint{1.334197in}{1.660858in}}%
\pgfpathlineto{\pgfqpoint{1.333883in}{1.661233in}}%
\pgfpathlineto{\pgfqpoint{1.331062in}{1.664609in}}%
\pgfpathlineto{\pgfqpoint{1.330749in}{1.664985in}}%
\pgfpathlineto{\pgfqpoint{1.327927in}{1.668361in}}%
\pgfpathlineto{\pgfqpoint{1.327614in}{1.668736in}}%
\pgfpathlineto{\pgfqpoint{1.324479in}{1.668736in}}%
\pgfpathlineto{\pgfqpoint{1.321658in}{1.672113in}}%
\pgfpathlineto{\pgfqpoint{1.321344in}{1.672488in}}%
\pgfpathlineto{\pgfqpoint{1.318523in}{1.675864in}}%
\pgfpathlineto{\pgfqpoint{1.318210in}{1.676239in}}%
\pgfpathlineto{\pgfqpoint{1.315388in}{1.679616in}}%
\pgfpathlineto{\pgfqpoint{1.315075in}{1.679991in}}%
\pgfpathlineto{\pgfqpoint{1.311940in}{1.679991in}}%
\pgfpathlineto{\pgfqpoint{1.309119in}{1.683367in}}%
\pgfpathlineto{\pgfqpoint{1.308805in}{1.683742in}}%
\pgfpathlineto{\pgfqpoint{1.305984in}{1.687119in}}%
\pgfpathlineto{\pgfqpoint{1.305671in}{1.687494in}}%
\pgfpathlineto{\pgfqpoint{1.302849in}{1.690870in}}%
\pgfpathlineto{\pgfqpoint{1.302536in}{1.691245in}}%
\pgfpathlineto{\pgfqpoint{1.299401in}{1.691245in}}%
\pgfpathlineto{\pgfqpoint{1.296580in}{1.694622in}}%
\pgfpathlineto{\pgfqpoint{1.296266in}{1.694997in}}%
\pgfpathlineto{\pgfqpoint{1.293445in}{1.698373in}}%
\pgfpathlineto{\pgfqpoint{1.293132in}{1.698748in}}%
\pgfpathlineto{\pgfqpoint{1.290310in}{1.702125in}}%
\pgfpathlineto{\pgfqpoint{1.290310in}{1.705876in}}%
\pgfpathlineto{\pgfqpoint{1.293132in}{1.709253in}}%
\pgfpathlineto{\pgfqpoint{1.293445in}{1.709628in}}%
\pgfpathlineto{\pgfqpoint{1.293445in}{1.713379in}}%
\pgfpathlineto{\pgfqpoint{1.293445in}{1.717131in}}%
\pgfpathlineto{\pgfqpoint{1.293445in}{1.720882in}}%
\pgfpathlineto{\pgfqpoint{1.296266in}{1.724259in}}%
\pgfpathlineto{\pgfqpoint{1.296580in}{1.724634in}}%
\pgfpathlineto{\pgfqpoint{1.296580in}{1.728386in}}%
\pgfpathlineto{\pgfqpoint{1.296580in}{1.732137in}}%
\pgfpathlineto{\pgfqpoint{1.299401in}{1.735513in}}%
\pgfpathlineto{\pgfqpoint{1.299715in}{1.735889in}}%
\pgfpathlineto{\pgfqpoint{1.299715in}{1.739640in}}%
\pgfpathlineto{\pgfqpoint{1.299715in}{1.743392in}}%
\pgfpathlineto{\pgfqpoint{1.299715in}{1.747143in}}%
\pgfpathlineto{\pgfqpoint{1.302536in}{1.750520in}}%
\pgfpathlineto{\pgfqpoint{1.302849in}{1.750895in}}%
\pgfpathlineto{\pgfqpoint{1.302849in}{1.754646in}}%
\pgfpathlineto{\pgfqpoint{1.302849in}{1.758398in}}%
\pgfpathlineto{\pgfqpoint{1.305671in}{1.761774in}}%
\pgfpathlineto{\pgfqpoint{1.305984in}{1.762149in}}%
\pgfpathlineto{\pgfqpoint{1.305984in}{1.765901in}}%
\pgfpathlineto{\pgfqpoint{1.305984in}{1.769652in}}%
\pgfpathlineto{\pgfqpoint{1.305984in}{1.773404in}}%
\pgfpathlineto{\pgfqpoint{1.308805in}{1.776780in}}%
\pgfpathlineto{\pgfqpoint{1.309119in}{1.777155in}}%
\pgfpathlineto{\pgfqpoint{1.309119in}{1.780907in}}%
\pgfpathlineto{\pgfqpoint{1.309119in}{1.784659in}}%
\pgfpathlineto{\pgfqpoint{1.311940in}{1.788035in}}%
\pgfpathlineto{\pgfqpoint{1.312254in}{1.788410in}}%
\pgfpathlineto{\pgfqpoint{1.312254in}{1.792162in}}%
\pgfpathlineto{\pgfqpoint{1.312254in}{1.795913in}}%
\pgfpathlineto{\pgfqpoint{1.312254in}{1.799665in}}%
\pgfpathlineto{\pgfqpoint{1.315075in}{1.803041in}}%
\pgfpathlineto{\pgfqpoint{1.315388in}{1.803416in}}%
\pgfpathlineto{\pgfqpoint{1.315388in}{1.807168in}}%
\pgfpathlineto{\pgfqpoint{1.315388in}{1.810919in}}%
\pgfpathlineto{\pgfqpoint{1.318210in}{1.814296in}}%
\pgfpathlineto{\pgfqpoint{1.318523in}{1.814671in}}%
\pgfpathlineto{\pgfqpoint{1.318523in}{1.818422in}}%
\pgfpathlineto{\pgfqpoint{1.318523in}{1.822174in}}%
\pgfpathlineto{\pgfqpoint{1.321344in}{1.825550in}}%
\pgfpathlineto{\pgfqpoint{1.321658in}{1.825925in}}%
\pgfpathlineto{\pgfqpoint{1.321658in}{1.829677in}}%
\pgfpathlineto{\pgfqpoint{1.321658in}{1.833429in}}%
\pgfpathlineto{\pgfqpoint{1.321658in}{1.837180in}}%
\pgfpathlineto{\pgfqpoint{1.324479in}{1.840556in}}%
\pgfpathlineto{\pgfqpoint{1.324793in}{1.840932in}}%
\pgfpathlineto{\pgfqpoint{1.324793in}{1.844683in}}%
\pgfpathlineto{\pgfqpoint{1.324793in}{1.848435in}}%
\pgfpathlineto{\pgfqpoint{1.327614in}{1.851811in}}%
\pgfpathlineto{\pgfqpoint{1.327927in}{1.852186in}}%
\pgfpathlineto{\pgfqpoint{1.327927in}{1.855938in}}%
\pgfpathlineto{\pgfqpoint{1.327927in}{1.859689in}}%
\pgfpathlineto{\pgfqpoint{1.327927in}{1.863441in}}%
\pgfpathlineto{\pgfqpoint{1.330749in}{1.866817in}}%
\pgfpathlineto{\pgfqpoint{1.331062in}{1.867192in}}%
\pgfpathlineto{\pgfqpoint{1.331062in}{1.870944in}}%
\pgfpathlineto{\pgfqpoint{1.331062in}{1.874695in}}%
\pgfpathlineto{\pgfqpoint{1.333883in}{1.878072in}}%
\pgfpathlineto{\pgfqpoint{1.334197in}{1.878447in}}%
\pgfpathlineto{\pgfqpoint{1.334197in}{1.882198in}}%
\pgfpathlineto{\pgfqpoint{1.334197in}{1.885950in}}%
\pgfpathlineto{\pgfqpoint{1.334197in}{1.889702in}}%
\pgfpathlineto{\pgfqpoint{1.337018in}{1.893078in}}%
\pgfpathlineto{\pgfqpoint{1.337332in}{1.893453in}}%
\pgfpathlineto{\pgfqpoint{1.337332in}{1.897205in}}%
\pgfpathlineto{\pgfqpoint{1.337332in}{1.900956in}}%
\pgfpathlineto{\pgfqpoint{1.340153in}{1.904333in}}%
\pgfpathlineto{\pgfqpoint{1.340466in}{1.904708in}}%
\pgfpathlineto{\pgfqpoint{1.340466in}{1.908459in}}%
\pgfpathlineto{\pgfqpoint{1.340466in}{1.912211in}}%
\pgfpathlineto{\pgfqpoint{1.340466in}{1.915962in}}%
\pgfpathlineto{\pgfqpoint{1.343288in}{1.919339in}}%
\pgfpathlineto{\pgfqpoint{1.343601in}{1.919714in}}%
\pgfpathlineto{\pgfqpoint{1.343601in}{1.923465in}}%
\pgfpathlineto{\pgfqpoint{1.343601in}{1.927217in}}%
\pgfpathlineto{\pgfqpoint{1.346422in}{1.930593in}}%
\pgfpathlineto{\pgfqpoint{1.346736in}{1.930968in}}%
\pgfpathlineto{\pgfqpoint{1.346736in}{1.934720in}}%
\pgfpathlineto{\pgfqpoint{1.346736in}{1.938471in}}%
\pgfpathlineto{\pgfqpoint{1.346736in}{1.942223in}}%
\pgfpathlineto{\pgfqpoint{1.349557in}{1.945599in}}%
\pgfpathlineto{\pgfqpoint{1.349871in}{1.945975in}}%
\pgfpathlineto{\pgfqpoint{1.349871in}{1.949726in}}%
\pgfpathlineto{\pgfqpoint{1.349871in}{1.953478in}}%
\pgfpathlineto{\pgfqpoint{1.352692in}{1.956854in}}%
\pgfpathlineto{\pgfqpoint{1.353005in}{1.957229in}}%
\pgfpathlineto{\pgfqpoint{1.353005in}{1.960981in}}%
\pgfpathlineto{\pgfqpoint{1.353005in}{1.964732in}}%
\pgfpathlineto{\pgfqpoint{1.353005in}{1.968484in}}%
\pgfpathlineto{\pgfqpoint{1.355827in}{1.971860in}}%
\pgfpathlineto{\pgfqpoint{1.356140in}{1.972235in}}%
\pgfpathlineto{\pgfqpoint{1.356140in}{1.975987in}}%
\pgfpathlineto{\pgfqpoint{1.356140in}{1.979738in}}%
\pgfpathlineto{\pgfqpoint{1.358961in}{1.983115in}}%
\pgfpathlineto{\pgfqpoint{1.359275in}{1.983490in}}%
\pgfpathlineto{\pgfqpoint{1.359275in}{1.987241in}}%
\pgfpathlineto{\pgfqpoint{1.359275in}{1.990993in}}%
\pgfpathlineto{\pgfqpoint{1.359275in}{1.994745in}}%
\pgfpathlineto{\pgfqpoint{1.362096in}{1.998121in}}%
\pgfpathlineto{\pgfqpoint{1.362410in}{1.998496in}}%
\pgfpathlineto{\pgfqpoint{1.362410in}{2.002248in}}%
\pgfpathlineto{\pgfqpoint{1.362410in}{2.005999in}}%
\pgfpathlineto{\pgfqpoint{1.365231in}{2.009375in}}%
\pgfpathlineto{\pgfqpoint{1.365544in}{2.009751in}}%
\pgfpathlineto{\pgfqpoint{1.365544in}{2.013502in}}%
\pgfpathlineto{\pgfqpoint{1.365544in}{2.017254in}}%
\pgfpathlineto{\pgfqpoint{1.365544in}{2.021005in}}%
\pgfpathlineto{\pgfqpoint{1.368366in}{2.024382in}}%
\pgfpathlineto{\pgfqpoint{1.368679in}{2.024757in}}%
\pgfpathlineto{\pgfqpoint{1.368679in}{2.028508in}}%
\pgfpathlineto{\pgfqpoint{1.368679in}{2.032260in}}%
\pgfpathlineto{\pgfqpoint{1.371500in}{2.035636in}}%
\pgfpathlineto{\pgfqpoint{1.371814in}{2.036011in}}%
\pgfpathlineto{\pgfqpoint{1.371814in}{2.039763in}}%
\pgfpathlineto{\pgfqpoint{1.371814in}{2.043514in}}%
\pgfpathlineto{\pgfqpoint{1.374635in}{2.046891in}}%
\pgfpathlineto{\pgfqpoint{1.374949in}{2.047266in}}%
\pgfpathlineto{\pgfqpoint{1.374949in}{2.051018in}}%
\pgfpathlineto{\pgfqpoint{1.374949in}{2.054769in}}%
\pgfpathlineto{\pgfqpoint{1.374949in}{2.058521in}}%
\pgfpathlineto{\pgfqpoint{1.377770in}{2.061897in}}%
\pgfpathlineto{\pgfqpoint{1.378083in}{2.062272in}}%
\pgfpathlineto{\pgfqpoint{1.378083in}{2.066024in}}%
\pgfpathlineto{\pgfqpoint{1.378083in}{2.069775in}}%
\pgfpathlineto{\pgfqpoint{1.380905in}{2.073152in}}%
\pgfpathlineto{\pgfqpoint{1.381218in}{2.073527in}}%
\pgfpathlineto{\pgfqpoint{1.381218in}{2.077278in}}%
\pgfpathlineto{\pgfqpoint{1.381218in}{2.081030in}}%
\pgfpathlineto{\pgfqpoint{1.381218in}{2.084781in}}%
\pgfpathlineto{\pgfqpoint{1.384039in}{2.088158in}}%
\pgfpathlineto{\pgfqpoint{1.384353in}{2.088533in}}%
\pgfpathlineto{\pgfqpoint{1.384353in}{2.092284in}}%
\pgfpathlineto{\pgfqpoint{1.384353in}{2.096036in}}%
\pgfpathlineto{\pgfqpoint{1.387174in}{2.099412in}}%
\pgfpathlineto{\pgfqpoint{1.387487in}{2.099787in}}%
\pgfpathlineto{\pgfqpoint{1.387487in}{2.103539in}}%
\pgfpathlineto{\pgfqpoint{1.387487in}{2.107291in}}%
\pgfpathlineto{\pgfqpoint{1.387487in}{2.111042in}}%
\pgfpathlineto{\pgfqpoint{1.390309in}{2.114418in}}%
\pgfpathlineto{\pgfqpoint{1.390622in}{2.114794in}}%
\pgfpathlineto{\pgfqpoint{1.390622in}{2.118545in}}%
\pgfpathlineto{\pgfqpoint{1.390622in}{2.122297in}}%
\pgfpathlineto{\pgfqpoint{1.393444in}{2.125673in}}%
\pgfpathlineto{\pgfqpoint{1.393757in}{2.126048in}}%
\pgfpathlineto{\pgfqpoint{1.393757in}{2.129800in}}%
\pgfpathlineto{\pgfqpoint{1.393757in}{2.133551in}}%
\pgfpathlineto{\pgfqpoint{1.393757in}{2.137303in}}%
\pgfpathlineto{\pgfqpoint{1.396578in}{2.140679in}}%
\pgfpathlineto{\pgfqpoint{1.396892in}{2.141054in}}%
\pgfpathlineto{\pgfqpoint{1.396892in}{2.144806in}}%
\pgfpathlineto{\pgfqpoint{1.396892in}{2.148557in}}%
\pgfpathlineto{\pgfqpoint{1.399713in}{2.151934in}}%
\pgfpathlineto{\pgfqpoint{1.400026in}{2.152309in}}%
\pgfpathlineto{\pgfqpoint{1.400026in}{2.156060in}}%
\pgfpathlineto{\pgfqpoint{1.400026in}{2.159812in}}%
\pgfpathlineto{\pgfqpoint{1.400026in}{2.163564in}}%
\pgfpathlineto{\pgfqpoint{1.402848in}{2.166940in}}%
\pgfpathlineto{\pgfqpoint{1.403161in}{2.167315in}}%
\pgfpathlineto{\pgfqpoint{1.403161in}{2.171067in}}%
\pgfpathlineto{\pgfqpoint{1.403161in}{2.174818in}}%
\pgfpathlineto{\pgfqpoint{1.405982in}{2.178195in}}%
\pgfpathlineto{\pgfqpoint{1.406296in}{2.178570in}}%
\pgfpathlineto{\pgfqpoint{1.406296in}{2.182321in}}%
\pgfpathlineto{\pgfqpoint{1.406296in}{2.186073in}}%
\pgfpathlineto{\pgfqpoint{1.406296in}{2.189824in}}%
\pgfpathlineto{\pgfqpoint{1.409117in}{2.193201in}}%
\pgfpathlineto{\pgfqpoint{1.409431in}{2.193576in}}%
\pgfpathlineto{\pgfqpoint{1.409431in}{2.197327in}}%
\pgfpathlineto{\pgfqpoint{1.409431in}{2.201079in}}%
\pgfpathlineto{\pgfqpoint{1.412252in}{2.204455in}}%
\pgfpathlineto{\pgfqpoint{1.412565in}{2.204830in}}%
\pgfpathlineto{\pgfqpoint{1.412565in}{2.208582in}}%
\pgfpathlineto{\pgfqpoint{1.412565in}{2.212334in}}%
\pgfpathlineto{\pgfqpoint{1.412565in}{2.216085in}}%
\pgfpathlineto{\pgfqpoint{1.415387in}{2.219461in}}%
\pgfpathlineto{\pgfqpoint{1.415700in}{2.219837in}}%
\pgfpathlineto{\pgfqpoint{1.415700in}{2.223588in}}%
\pgfpathlineto{\pgfqpoint{1.415700in}{2.227340in}}%
\pgfpathlineto{\pgfqpoint{1.418521in}{2.230716in}}%
\pgfpathlineto{\pgfqpoint{1.418835in}{2.231091in}}%
\pgfpathlineto{\pgfqpoint{1.418835in}{2.234843in}}%
\pgfpathlineto{\pgfqpoint{1.418835in}{2.238594in}}%
\pgfpathlineto{\pgfqpoint{1.418835in}{2.242346in}}%
\pgfpathlineto{\pgfqpoint{1.421656in}{2.245722in}}%
\pgfpathlineto{\pgfqpoint{1.421970in}{2.246097in}}%
\pgfpathlineto{\pgfqpoint{1.421970in}{2.249849in}}%
\pgfpathlineto{\pgfqpoint{1.421970in}{2.253600in}}%
\pgfpathlineto{\pgfqpoint{1.424791in}{2.256977in}}%
\pgfpathlineto{\pgfqpoint{1.425104in}{2.257352in}}%
\pgfpathlineto{\pgfqpoint{1.425104in}{2.261103in}}%
\pgfpathlineto{\pgfqpoint{1.425104in}{2.264855in}}%
\pgfpathlineto{\pgfqpoint{1.427926in}{2.268231in}}%
\pgfpathlineto{\pgfqpoint{1.428239in}{2.268607in}}%
\pgfpathlineto{\pgfqpoint{1.428239in}{2.272358in}}%
\pgfpathlineto{\pgfqpoint{1.428239in}{2.276110in}}%
\pgfpathlineto{\pgfqpoint{1.428239in}{2.279861in}}%
\pgfpathlineto{\pgfqpoint{1.431060in}{2.283238in}}%
\pgfpathlineto{\pgfqpoint{1.431374in}{2.283613in}}%
\pgfpathlineto{\pgfqpoint{1.431374in}{2.287364in}}%
\pgfpathlineto{\pgfqpoint{1.431374in}{2.291116in}}%
\pgfpathlineto{\pgfqpoint{1.434195in}{2.294492in}}%
\pgfpathlineto{\pgfqpoint{1.434509in}{2.294867in}}%
\pgfpathlineto{\pgfqpoint{1.434509in}{2.298619in}}%
\pgfpathlineto{\pgfqpoint{1.434509in}{2.302370in}}%
\pgfpathlineto{\pgfqpoint{1.434509in}{2.306122in}}%
\pgfpathlineto{\pgfqpoint{1.437330in}{2.309498in}}%
\pgfpathlineto{\pgfqpoint{1.437643in}{2.309873in}}%
\pgfpathlineto{\pgfqpoint{1.437643in}{2.313625in}}%
\pgfpathlineto{\pgfqpoint{1.437643in}{2.317376in}}%
\pgfpathlineto{\pgfqpoint{1.440465in}{2.320753in}}%
\pgfpathlineto{\pgfqpoint{1.440778in}{2.321128in}}%
\pgfpathlineto{\pgfqpoint{1.440778in}{2.324880in}}%
\pgfpathlineto{\pgfqpoint{1.440778in}{2.328631in}}%
\pgfpathlineto{\pgfqpoint{1.440778in}{2.332383in}}%
\pgfpathlineto{\pgfqpoint{1.443599in}{2.335759in}}%
\pgfpathlineto{\pgfqpoint{1.443913in}{2.336134in}}%
\pgfpathlineto{\pgfqpoint{1.443913in}{2.339886in}}%
\pgfpathlineto{\pgfqpoint{1.443913in}{2.343637in}}%
\pgfpathlineto{\pgfqpoint{1.446734in}{2.347014in}}%
\pgfpathlineto{\pgfqpoint{1.447048in}{2.347389in}}%
\pgfpathlineto{\pgfqpoint{1.447048in}{2.351140in}}%
\pgfpathlineto{\pgfqpoint{1.447048in}{2.354892in}}%
\pgfpathlineto{\pgfqpoint{1.447048in}{2.358643in}}%
\pgfpathlineto{\pgfqpoint{1.449869in}{2.362020in}}%
\pgfpathlineto{\pgfqpoint{1.450182in}{2.362395in}}%
\pgfpathlineto{\pgfqpoint{1.450182in}{2.366146in}}%
\pgfpathlineto{\pgfqpoint{1.450182in}{2.369898in}}%
\pgfpathlineto{\pgfqpoint{1.453004in}{2.373274in}}%
\pgfpathlineto{\pgfqpoint{1.453317in}{2.373649in}}%
\pgfpathlineto{\pgfqpoint{1.453317in}{2.377401in}}%
\pgfpathlineto{\pgfqpoint{1.453317in}{2.381153in}}%
\pgfpathlineto{\pgfqpoint{1.453317in}{2.384904in}}%
\pgfpathlineto{\pgfqpoint{1.456138in}{2.388280in}}%
\pgfpathlineto{\pgfqpoint{1.456452in}{2.388656in}}%
\pgfpathlineto{\pgfqpoint{1.456452in}{2.392407in}}%
\pgfpathlineto{\pgfqpoint{1.456452in}{2.396159in}}%
\pgfpathlineto{\pgfqpoint{1.459273in}{2.399535in}}%
\pgfpathlineto{\pgfqpoint{1.459587in}{2.399910in}}%
\pgfpathlineto{\pgfqpoint{1.459587in}{2.403662in}}%
\pgfpathlineto{\pgfqpoint{1.462408in}{2.407038in}}%
\pgfpathlineto{\pgfqpoint{1.465543in}{2.407038in}}%
\pgfpathlineto{\pgfqpoint{1.465856in}{2.407413in}}%
\pgfpathlineto{\pgfqpoint{1.468677in}{2.410790in}}%
\pgfpathlineto{\pgfqpoint{1.471812in}{2.410790in}}%
\pgfpathlineto{\pgfqpoint{1.474947in}{2.410790in}}%
\pgfpathlineto{\pgfqpoint{1.478082in}{2.410790in}}%
\pgfpathlineto{\pgfqpoint{1.478395in}{2.411165in}}%
\pgfpathlineto{\pgfqpoint{1.481216in}{2.414541in}}%
\pgfpathlineto{\pgfqpoint{1.484351in}{2.414541in}}%
\pgfpathlineto{\pgfqpoint{1.487486in}{2.414541in}}%
\pgfpathlineto{\pgfqpoint{1.490621in}{2.414541in}}%
\pgfpathlineto{\pgfqpoint{1.490934in}{2.414916in}}%
\pgfpathlineto{\pgfqpoint{1.493755in}{2.418293in}}%
\pgfpathlineto{\pgfqpoint{1.496890in}{2.418293in}}%
\pgfpathlineto{\pgfqpoint{1.500025in}{2.418293in}}%
\pgfpathlineto{\pgfqpoint{1.503159in}{2.418293in}}%
\pgfpathlineto{\pgfqpoint{1.503473in}{2.418668in}}%
\pgfpathlineto{\pgfqpoint{1.506294in}{2.422044in}}%
\pgfpathlineto{\pgfqpoint{1.509429in}{2.422044in}}%
\pgfpathlineto{\pgfqpoint{1.512564in}{2.422044in}}%
\pgfpathlineto{\pgfqpoint{1.515698in}{2.422044in}}%
\pgfpathlineto{\pgfqpoint{1.516012in}{2.422419in}}%
\pgfpathlineto{\pgfqpoint{1.518833in}{2.425796in}}%
\pgfpathlineto{\pgfqpoint{1.521968in}{2.425796in}}%
\pgfpathlineto{\pgfqpoint{1.525103in}{2.425796in}}%
\pgfpathlineto{\pgfqpoint{1.528237in}{2.425796in}}%
\pgfpathlineto{\pgfqpoint{1.531372in}{2.425796in}}%
\pgfpathlineto{\pgfqpoint{1.531686in}{2.426171in}}%
\pgfpathlineto{\pgfqpoint{1.534507in}{2.429547in}}%
\pgfpathlineto{\pgfqpoint{1.537642in}{2.429547in}}%
\pgfpathlineto{\pgfqpoint{1.540776in}{2.429547in}}%
\pgfpathlineto{\pgfqpoint{1.543911in}{2.429547in}}%
\pgfpathlineto{\pgfqpoint{1.544225in}{2.429923in}}%
\pgfpathlineto{\pgfqpoint{1.547046in}{2.433299in}}%
\pgfpathlineto{\pgfqpoint{1.550181in}{2.433299in}}%
\pgfpathlineto{\pgfqpoint{1.553315in}{2.433299in}}%
\pgfpathlineto{\pgfqpoint{1.556450in}{2.433299in}}%
\pgfpathlineto{\pgfqpoint{1.556764in}{2.433674in}}%
\pgfpathlineto{\pgfqpoint{1.559585in}{2.437050in}}%
\pgfpathlineto{\pgfqpoint{1.562720in}{2.437050in}}%
\pgfpathlineto{\pgfqpoint{1.565854in}{2.437050in}}%
\pgfpathlineto{\pgfqpoint{1.568989in}{2.437050in}}%
\pgfpathlineto{\pgfqpoint{1.569303in}{2.437426in}}%
\pgfpathlineto{\pgfqpoint{1.572124in}{2.440802in}}%
\pgfpathlineto{\pgfqpoint{1.575259in}{2.440802in}}%
\pgfpathlineto{\pgfqpoint{1.578393in}{2.440802in}}%
\pgfpathlineto{\pgfqpoint{1.581528in}{2.440802in}}%
\pgfpathlineto{\pgfqpoint{1.584663in}{2.440802in}}%
\pgfpathlineto{\pgfqpoint{1.584976in}{2.441177in}}%
\pgfpathlineto{\pgfqpoint{1.587798in}{2.444553in}}%
\pgfpathlineto{\pgfqpoint{1.590932in}{2.444553in}}%
\pgfpathlineto{\pgfqpoint{1.594067in}{2.444553in}}%
\pgfpathlineto{\pgfqpoint{1.597202in}{2.444553in}}%
\pgfpathlineto{\pgfqpoint{1.597515in}{2.444929in}}%
\pgfpathlineto{\pgfqpoint{1.600337in}{2.448305in}}%
\pgfpathlineto{\pgfqpoint{1.603471in}{2.448305in}}%
\pgfpathlineto{\pgfqpoint{1.606606in}{2.448305in}}%
\pgfpathlineto{\pgfqpoint{1.609741in}{2.448305in}}%
\pgfpathlineto{\pgfqpoint{1.610054in}{2.448680in}}%
\pgfpathlineto{\pgfqpoint{1.612875in}{2.452057in}}%
\pgfpathlineto{\pgfqpoint{1.616010in}{2.452057in}}%
\pgfpathlineto{\pgfqpoint{1.619145in}{2.452057in}}%
\pgfpathlineto{\pgfqpoint{1.622280in}{2.452057in}}%
\pgfpathlineto{\pgfqpoint{1.622593in}{2.452432in}}%
\pgfpathlineto{\pgfqpoint{1.625414in}{2.455808in}}%
\pgfpathlineto{\pgfqpoint{1.628549in}{2.455808in}}%
\pgfpathlineto{\pgfqpoint{1.631684in}{2.455808in}}%
\pgfpathlineto{\pgfqpoint{1.634819in}{2.455808in}}%
\pgfpathlineto{\pgfqpoint{1.637953in}{2.455808in}}%
\pgfpathlineto{\pgfqpoint{1.638267in}{2.456183in}}%
\pgfpathlineto{\pgfqpoint{1.641088in}{2.459560in}}%
\pgfpathlineto{\pgfqpoint{1.644223in}{2.459560in}}%
\pgfpathlineto{\pgfqpoint{1.647358in}{2.459560in}}%
\pgfpathlineto{\pgfqpoint{1.650492in}{2.459560in}}%
\pgfpathlineto{\pgfqpoint{1.650806in}{2.459935in}}%
\pgfpathlineto{\pgfqpoint{1.653627in}{2.463311in}}%
\pgfpathlineto{\pgfqpoint{1.656762in}{2.463311in}}%
\pgfpathlineto{\pgfqpoint{1.659897in}{2.463311in}}%
\pgfpathlineto{\pgfqpoint{1.663031in}{2.463311in}}%
\pgfpathlineto{\pgfqpoint{1.663345in}{2.463686in}}%
\pgfpathlineto{\pgfqpoint{1.666166in}{2.467063in}}%
\pgfpathlineto{\pgfqpoint{1.669301in}{2.467063in}}%
\pgfpathlineto{\pgfqpoint{1.672436in}{2.467063in}}%
\pgfpathlineto{\pgfqpoint{1.675570in}{2.467063in}}%
\pgfpathlineto{\pgfqpoint{1.675884in}{2.467438in}}%
\pgfpathlineto{\pgfqpoint{1.678705in}{2.470814in}}%
\pgfpathlineto{\pgfqpoint{1.681840in}{2.470814in}}%
\pgfpathlineto{\pgfqpoint{1.684975in}{2.470814in}}%
\pgfpathlineto{\pgfqpoint{1.688109in}{2.470814in}}%
\pgfpathlineto{\pgfqpoint{1.688423in}{2.471189in}}%
\pgfpathlineto{\pgfqpoint{1.691244in}{2.474566in}}%
\pgfpathlineto{\pgfqpoint{1.694379in}{2.474566in}}%
\pgfpathlineto{\pgfqpoint{1.697514in}{2.474566in}}%
\pgfpathlineto{\pgfqpoint{1.700648in}{2.474566in}}%
\pgfpathlineto{\pgfqpoint{1.703783in}{2.474566in}}%
\pgfpathlineto{\pgfqpoint{1.704096in}{2.474941in}}%
\pgfpathlineto{\pgfqpoint{1.706918in}{2.478317in}}%
\pgfpathlineto{\pgfqpoint{1.710052in}{2.478317in}}%
\pgfpathlineto{\pgfqpoint{1.713187in}{2.478317in}}%
\pgfpathlineto{\pgfqpoint{1.716322in}{2.478317in}}%
\pgfpathlineto{\pgfqpoint{1.716635in}{2.478692in}}%
\pgfpathlineto{\pgfqpoint{1.719457in}{2.482069in}}%
\pgfpathlineto{\pgfqpoint{1.722591in}{2.482069in}}%
\pgfpathlineto{\pgfqpoint{1.725726in}{2.482069in}}%
\pgfpathlineto{\pgfqpoint{1.728861in}{2.482069in}}%
\pgfpathlineto{\pgfqpoint{1.729174in}{2.482444in}}%
\pgfpathlineto{\pgfqpoint{1.731996in}{2.485820in}}%
\pgfpathlineto{\pgfqpoint{1.735130in}{2.485820in}}%
\pgfpathlineto{\pgfqpoint{1.738265in}{2.485820in}}%
\pgfpathlineto{\pgfqpoint{1.741400in}{2.485820in}}%
\pgfpathlineto{\pgfqpoint{1.741713in}{2.486196in}}%
\pgfpathlineto{\pgfqpoint{1.744535in}{2.489572in}}%
\pgfpathlineto{\pgfqpoint{1.747669in}{2.489572in}}%
\pgfpathlineto{\pgfqpoint{1.750804in}{2.489572in}}%
\pgfpathlineto{\pgfqpoint{1.753939in}{2.489572in}}%
\pgfpathlineto{\pgfqpoint{1.757074in}{2.489572in}}%
\pgfpathlineto{\pgfqpoint{1.757387in}{2.489947in}}%
\pgfpathlineto{\pgfqpoint{1.760208in}{2.493323in}}%
\pgfpathlineto{\pgfqpoint{1.763343in}{2.493323in}}%
\pgfpathlineto{\pgfqpoint{1.766478in}{2.493323in}}%
\pgfpathlineto{\pgfqpoint{1.769613in}{2.493323in}}%
\pgfpathlineto{\pgfqpoint{1.769926in}{2.493699in}}%
\pgfpathlineto{\pgfqpoint{1.772747in}{2.497075in}}%
\pgfpathlineto{\pgfqpoint{1.775882in}{2.497075in}}%
\pgfpathlineto{\pgfqpoint{1.779017in}{2.497075in}}%
\pgfpathlineto{\pgfqpoint{1.782152in}{2.497075in}}%
\pgfpathlineto{\pgfqpoint{1.782465in}{2.497450in}}%
\pgfpathlineto{\pgfqpoint{1.785286in}{2.500827in}}%
\pgfpathlineto{\pgfqpoint{1.788421in}{2.500827in}}%
\pgfpathlineto{\pgfqpoint{1.791556in}{2.500827in}}%
\pgfpathlineto{\pgfqpoint{1.794691in}{2.500827in}}%
\pgfpathlineto{\pgfqpoint{1.795004in}{2.501202in}}%
\pgfpathlineto{\pgfqpoint{1.797825in}{2.504578in}}%
\pgfpathlineto{\pgfqpoint{1.800960in}{2.504578in}}%
\pgfpathlineto{\pgfqpoint{1.804095in}{2.504578in}}%
\pgfpathlineto{\pgfqpoint{1.807229in}{2.504578in}}%
\pgfpathlineto{\pgfqpoint{1.810364in}{2.504578in}}%
\pgfpathlineto{\pgfqpoint{1.810678in}{2.504953in}}%
\pgfpathlineto{\pgfqpoint{1.813499in}{2.508330in}}%
\pgfpathlineto{\pgfqpoint{1.816634in}{2.508330in}}%
\pgfpathlineto{\pgfqpoint{1.819768in}{2.508330in}}%
\pgfpathlineto{\pgfqpoint{1.822903in}{2.508330in}}%
\pgfpathlineto{\pgfqpoint{1.823217in}{2.508705in}}%
\pgfpathlineto{\pgfqpoint{1.826038in}{2.512081in}}%
\pgfpathlineto{\pgfqpoint{1.829173in}{2.512081in}}%
\pgfpathlineto{\pgfqpoint{1.832307in}{2.512081in}}%
\pgfpathlineto{\pgfqpoint{1.835442in}{2.512081in}}%
\pgfpathlineto{\pgfqpoint{1.835756in}{2.512456in}}%
\pgfpathlineto{\pgfqpoint{1.838577in}{2.515833in}}%
\pgfpathlineto{\pgfqpoint{1.841712in}{2.515833in}}%
\pgfpathlineto{\pgfqpoint{1.844846in}{2.515833in}}%
\pgfpathlineto{\pgfqpoint{1.847981in}{2.515833in}}%
\pgfpathlineto{\pgfqpoint{1.848295in}{2.516208in}}%
\pgfpathlineto{\pgfqpoint{1.851116in}{2.519584in}}%
\pgfpathlineto{\pgfqpoint{1.854251in}{2.519584in}}%
\pgfpathlineto{\pgfqpoint{1.857385in}{2.519584in}}%
\pgfpathlineto{\pgfqpoint{1.860520in}{2.519584in}}%
\pgfpathlineto{\pgfqpoint{1.860834in}{2.519959in}}%
\pgfpathlineto{\pgfqpoint{1.863655in}{2.523336in}}%
\pgfpathlineto{\pgfqpoint{1.866790in}{2.523336in}}%
\pgfpathlineto{\pgfqpoint{1.869924in}{2.523336in}}%
\pgfpathlineto{\pgfqpoint{1.873059in}{2.523336in}}%
\pgfpathlineto{\pgfqpoint{1.876194in}{2.523336in}}%
\pgfpathlineto{\pgfqpoint{1.876507in}{2.523711in}}%
\pgfpathlineto{\pgfqpoint{1.879329in}{2.527087in}}%
\pgfpathlineto{\pgfqpoint{1.882463in}{2.527087in}}%
\pgfpathlineto{\pgfqpoint{1.885598in}{2.527087in}}%
\pgfpathlineto{\pgfqpoint{1.888733in}{2.527087in}}%
\pgfpathlineto{\pgfqpoint{1.889046in}{2.527462in}}%
\pgfpathlineto{\pgfqpoint{1.891868in}{2.530839in}}%
\pgfpathlineto{\pgfqpoint{1.895002in}{2.530839in}}%
\pgfpathlineto{\pgfqpoint{1.898137in}{2.530839in}}%
\pgfpathlineto{\pgfqpoint{1.901272in}{2.530839in}}%
\pgfpathlineto{\pgfqpoint{1.901585in}{2.531214in}}%
\pgfpathlineto{\pgfqpoint{1.904407in}{2.534590in}}%
\pgfpathlineto{\pgfqpoint{1.907541in}{2.534590in}}%
\pgfpathlineto{\pgfqpoint{1.910676in}{2.534590in}}%
\pgfpathlineto{\pgfqpoint{1.913811in}{2.534590in}}%
\pgfpathlineto{\pgfqpoint{1.914124in}{2.534965in}}%
\pgfpathlineto{\pgfqpoint{1.916945in}{2.538342in}}%
\pgfpathlineto{\pgfqpoint{1.920080in}{2.538342in}}%
\pgfpathlineto{\pgfqpoint{1.923215in}{2.538342in}}%
\pgfpathlineto{\pgfqpoint{1.926350in}{2.538342in}}%
\pgfpathlineto{\pgfqpoint{1.929484in}{2.538342in}}%
\pgfpathlineto{\pgfqpoint{1.929798in}{2.538717in}}%
\pgfpathlineto{\pgfqpoint{1.932619in}{2.542093in}}%
\pgfpathlineto{\pgfqpoint{1.935754in}{2.542093in}}%
\pgfpathlineto{\pgfqpoint{1.938889in}{2.542093in}}%
\pgfpathlineto{\pgfqpoint{1.942023in}{2.542093in}}%
\pgfpathlineto{\pgfqpoint{1.942337in}{2.542469in}}%
\pgfpathlineto{\pgfqpoint{1.945158in}{2.545845in}}%
\pgfpathlineto{\pgfqpoint{1.948293in}{2.545845in}}%
\pgfpathlineto{\pgfqpoint{1.951428in}{2.545845in}}%
\pgfpathlineto{\pgfqpoint{1.954562in}{2.545845in}}%
\pgfpathlineto{\pgfqpoint{1.954876in}{2.546220in}}%
\pgfpathlineto{\pgfqpoint{1.957697in}{2.549596in}}%
\pgfpathlineto{\pgfqpoint{1.960832in}{2.549596in}}%
\pgfpathlineto{\pgfqpoint{1.963967in}{2.549596in}}%
\pgfpathlineto{\pgfqpoint{1.967101in}{2.549596in}}%
\pgfpathlineto{\pgfqpoint{1.967415in}{2.549972in}}%
\pgfpathlineto{\pgfqpoint{1.970236in}{2.553348in}}%
\pgfpathlineto{\pgfqpoint{1.973371in}{2.553348in}}%
\pgfpathlineto{\pgfqpoint{1.976506in}{2.553348in}}%
\pgfpathlineto{\pgfqpoint{1.979640in}{2.553348in}}%
\pgfpathlineto{\pgfqpoint{1.979954in}{2.553723in}}%
\pgfpathlineto{\pgfqpoint{1.982775in}{2.557100in}}%
\pgfpathlineto{\pgfqpoint{1.985910in}{2.557100in}}%
\pgfpathlineto{\pgfqpoint{1.989045in}{2.557100in}}%
\pgfpathlineto{\pgfqpoint{1.992179in}{2.557100in}}%
\pgfpathlineto{\pgfqpoint{1.995314in}{2.557100in}}%
\pgfpathlineto{\pgfqpoint{1.995628in}{2.557475in}}%
\pgfpathlineto{\pgfqpoint{1.998449in}{2.560851in}}%
\pgfpathlineto{\pgfqpoint{2.001584in}{2.560851in}}%
\pgfpathlineto{\pgfqpoint{2.004718in}{2.560851in}}%
\pgfpathlineto{\pgfqpoint{2.007853in}{2.560851in}}%
\pgfpathlineto{\pgfqpoint{2.008166in}{2.561226in}}%
\pgfpathlineto{\pgfqpoint{2.010988in}{2.564603in}}%
\pgfpathlineto{\pgfqpoint{2.014122in}{2.564603in}}%
\pgfpathlineto{\pgfqpoint{2.017257in}{2.564603in}}%
\pgfpathlineto{\pgfqpoint{2.020392in}{2.564603in}}%
\pgfpathlineto{\pgfqpoint{2.020705in}{2.564978in}}%
\pgfpathlineto{\pgfqpoint{2.023527in}{2.568354in}}%
\pgfpathlineto{\pgfqpoint{2.026661in}{2.568354in}}%
\pgfpathlineto{\pgfqpoint{2.029796in}{2.568354in}}%
\pgfpathlineto{\pgfqpoint{2.032931in}{2.568354in}}%
\pgfpathlineto{\pgfqpoint{2.033244in}{2.568729in}}%
\pgfpathlineto{\pgfqpoint{2.036066in}{2.572106in}}%
\pgfpathlineto{\pgfqpoint{2.039200in}{2.572106in}}%
\pgfpathlineto{\pgfqpoint{2.042335in}{2.572106in}}%
\pgfpathlineto{\pgfqpoint{2.045470in}{2.572106in}}%
\pgfpathlineto{\pgfqpoint{2.048605in}{2.572106in}}%
\pgfpathlineto{\pgfqpoint{2.048918in}{2.572481in}}%
\pgfpathlineto{\pgfqpoint{2.051739in}{2.575857in}}%
\pgfpathlineto{\pgfqpoint{2.054874in}{2.575857in}}%
\pgfpathlineto{\pgfqpoint{2.058009in}{2.575857in}}%
\pgfpathlineto{\pgfqpoint{2.061144in}{2.575857in}}%
\pgfpathlineto{\pgfqpoint{2.061457in}{2.576232in}}%
\pgfpathlineto{\pgfqpoint{2.064278in}{2.579609in}}%
\pgfpathlineto{\pgfqpoint{2.067413in}{2.579609in}}%
\pgfpathlineto{\pgfqpoint{2.070548in}{2.579609in}}%
\pgfpathlineto{\pgfqpoint{2.073683in}{2.579609in}}%
\pgfpathlineto{\pgfqpoint{2.073996in}{2.579984in}}%
\pgfpathlineto{\pgfqpoint{2.076817in}{2.583360in}}%
\pgfpathlineto{\pgfqpoint{2.079952in}{2.583360in}}%
\pgfpathlineto{\pgfqpoint{2.083087in}{2.583360in}}%
\pgfpathlineto{\pgfqpoint{2.086222in}{2.583360in}}%
\pgfpathlineto{\pgfqpoint{2.086535in}{2.583735in}}%
\pgfpathlineto{\pgfqpoint{2.089356in}{2.587112in}}%
\pgfpathlineto{\pgfqpoint{2.092491in}{2.587112in}}%
\pgfpathlineto{\pgfqpoint{2.095626in}{2.587112in}}%
\pgfpathlineto{\pgfqpoint{2.098761in}{2.587112in}}%
\pgfpathlineto{\pgfqpoint{2.101895in}{2.587112in}}%
\pgfpathlineto{\pgfqpoint{2.102209in}{2.587487in}}%
\pgfpathlineto{\pgfqpoint{2.105030in}{2.590863in}}%
\pgfpathlineto{\pgfqpoint{2.108165in}{2.590863in}}%
\pgfpathlineto{\pgfqpoint{2.111299in}{2.590863in}}%
\pgfpathlineto{\pgfqpoint{2.114434in}{2.590863in}}%
\pgfpathlineto{\pgfqpoint{2.114748in}{2.591238in}}%
\pgfpathlineto{\pgfqpoint{2.117569in}{2.594615in}}%
\pgfpathlineto{\pgfqpoint{2.120704in}{2.594615in}}%
\pgfpathlineto{\pgfqpoint{2.123838in}{2.594615in}}%
\pgfpathlineto{\pgfqpoint{2.126973in}{2.594615in}}%
\pgfpathlineto{\pgfqpoint{2.127287in}{2.594990in}}%
\pgfpathlineto{\pgfqpoint{2.130108in}{2.598366in}}%
\pgfpathlineto{\pgfqpoint{2.133243in}{2.598366in}}%
\pgfpathlineto{\pgfqpoint{2.136377in}{2.598366in}}%
\pgfpathlineto{\pgfqpoint{2.139512in}{2.598366in}}%
\pgfpathlineto{\pgfqpoint{2.139826in}{2.598742in}}%
\pgfpathlineto{\pgfqpoint{2.142647in}{2.602118in}}%
\pgfpathlineto{\pgfqpoint{2.145782in}{2.602118in}}%
\pgfpathlineto{\pgfqpoint{2.148916in}{2.602118in}}%
\pgfpathlineto{\pgfqpoint{2.152051in}{2.602118in}}%
\pgfpathlineto{\pgfqpoint{2.152365in}{2.602493in}}%
\pgfpathlineto{\pgfqpoint{2.155186in}{2.605869in}}%
\pgfpathlineto{\pgfqpoint{2.158321in}{2.605869in}}%
\pgfpathlineto{\pgfqpoint{2.161455in}{2.605869in}}%
\pgfpathlineto{\pgfqpoint{2.164590in}{2.605869in}}%
\pgfpathlineto{\pgfqpoint{2.167725in}{2.605869in}}%
\pgfpathlineto{\pgfqpoint{2.168038in}{2.606245in}}%
\pgfpathlineto{\pgfqpoint{2.170860in}{2.609621in}}%
\pgfpathlineto{\pgfqpoint{2.173994in}{2.609621in}}%
\pgfpathlineto{\pgfqpoint{2.177129in}{2.609621in}}%
\pgfpathlineto{\pgfqpoint{2.180264in}{2.609621in}}%
\pgfpathlineto{\pgfqpoint{2.180577in}{2.609996in}}%
\pgfpathlineto{\pgfqpoint{2.183399in}{2.613373in}}%
\pgfpathlineto{\pgfqpoint{2.186533in}{2.613373in}}%
\pgfpathlineto{\pgfqpoint{2.189668in}{2.613373in}}%
\pgfpathlineto{\pgfqpoint{2.192803in}{2.613373in}}%
\pgfpathlineto{\pgfqpoint{2.193116in}{2.613748in}}%
\pgfpathlineto{\pgfqpoint{2.195938in}{2.617124in}}%
\pgfpathlineto{\pgfqpoint{2.199072in}{2.617124in}}%
\pgfpathlineto{\pgfqpoint{2.202207in}{2.617124in}}%
\pgfpathlineto{\pgfqpoint{2.205342in}{2.617124in}}%
\pgfpathlineto{\pgfqpoint{2.205655in}{2.617499in}}%
\pgfpathlineto{\pgfqpoint{2.208477in}{2.620876in}}%
\pgfpathlineto{\pgfqpoint{2.211611in}{2.620876in}}%
\pgfpathlineto{\pgfqpoint{2.214746in}{2.620876in}}%
\pgfpathlineto{\pgfqpoint{2.217881in}{2.620876in}}%
\pgfpathlineto{\pgfqpoint{2.221015in}{2.620876in}}%
\pgfpathlineto{\pgfqpoint{2.221329in}{2.621251in}}%
\pgfpathlineto{\pgfqpoint{2.224150in}{2.624627in}}%
\pgfpathlineto{\pgfqpoint{2.227285in}{2.624627in}}%
\pgfpathlineto{\pgfqpoint{2.230420in}{2.624627in}}%
\pgfpathlineto{\pgfqpoint{2.233554in}{2.624627in}}%
\pgfpathlineto{\pgfqpoint{2.233868in}{2.625002in}}%
\pgfpathlineto{\pgfqpoint{2.236689in}{2.628379in}}%
\pgfpathlineto{\pgfqpoint{2.239824in}{2.628379in}}%
\pgfpathlineto{\pgfqpoint{2.242959in}{2.628379in}}%
\pgfpathlineto{\pgfqpoint{2.246093in}{2.628379in}}%
\pgfpathlineto{\pgfqpoint{2.246407in}{2.628754in}}%
\pgfpathlineto{\pgfqpoint{2.249228in}{2.632130in}}%
\pgfpathlineto{\pgfqpoint{2.252363in}{2.632130in}}%
\pgfpathlineto{\pgfqpoint{2.255498in}{2.632130in}}%
\pgfpathlineto{\pgfqpoint{2.258632in}{2.632130in}}%
\pgfpathlineto{\pgfqpoint{2.258946in}{2.632505in}}%
\pgfpathlineto{\pgfqpoint{2.261767in}{2.635882in}}%
\pgfpathlineto{\pgfqpoint{2.264902in}{2.635882in}}%
\pgfpathlineto{\pgfqpoint{2.268037in}{2.635882in}}%
\pgfpathlineto{\pgfqpoint{2.271171in}{2.635882in}}%
\pgfpathlineto{\pgfqpoint{2.274306in}{2.635882in}}%
\pgfpathlineto{\pgfqpoint{2.274620in}{2.636257in}}%
\pgfpathlineto{\pgfqpoint{2.277441in}{2.639633in}}%
\pgfpathlineto{\pgfqpoint{2.280576in}{2.639633in}}%
\pgfpathlineto{\pgfqpoint{2.283710in}{2.639633in}}%
\pgfpathlineto{\pgfqpoint{2.286845in}{2.639633in}}%
\pgfpathlineto{\pgfqpoint{2.287159in}{2.640008in}}%
\pgfpathlineto{\pgfqpoint{2.289980in}{2.643385in}}%
\pgfpathlineto{\pgfqpoint{2.293115in}{2.643385in}}%
\pgfpathlineto{\pgfqpoint{2.296249in}{2.643385in}}%
\pgfpathlineto{\pgfqpoint{2.299384in}{2.643385in}}%
\pgfpathlineto{\pgfqpoint{2.299698in}{2.643760in}}%
\pgfpathlineto{\pgfqpoint{2.302519in}{2.647136in}}%
\pgfpathlineto{\pgfqpoint{2.305654in}{2.647136in}}%
\pgfpathlineto{\pgfqpoint{2.308788in}{2.647136in}}%
\pgfpathlineto{\pgfqpoint{2.311923in}{2.647136in}}%
\pgfpathlineto{\pgfqpoint{2.312236in}{2.647512in}}%
\pgfpathlineto{\pgfqpoint{2.315058in}{2.650888in}}%
\pgfpathlineto{\pgfqpoint{2.318192in}{2.650888in}}%
\pgfpathlineto{\pgfqpoint{2.321327in}{2.650888in}}%
\pgfpathlineto{\pgfqpoint{2.324462in}{2.650888in}}%
\pgfpathlineto{\pgfqpoint{2.324775in}{2.651263in}}%
\pgfpathlineto{\pgfqpoint{2.327597in}{2.654639in}}%
\pgfpathlineto{\pgfqpoint{2.330731in}{2.654639in}}%
\pgfpathlineto{\pgfqpoint{2.333866in}{2.654639in}}%
\pgfpathlineto{\pgfqpoint{2.337001in}{2.654639in}}%
\pgfpathlineto{\pgfqpoint{2.340136in}{2.654639in}}%
\pgfpathlineto{\pgfqpoint{2.340449in}{2.655015in}}%
\pgfpathlineto{\pgfqpoint{2.343270in}{2.658391in}}%
\pgfpathlineto{\pgfqpoint{2.346405in}{2.658391in}}%
\pgfpathlineto{\pgfqpoint{2.349540in}{2.658391in}}%
\pgfpathlineto{\pgfqpoint{2.352675in}{2.658391in}}%
\pgfpathlineto{\pgfqpoint{2.352988in}{2.658766in}}%
\pgfpathlineto{\pgfqpoint{2.355809in}{2.662142in}}%
\pgfpathlineto{\pgfqpoint{2.358944in}{2.662142in}}%
\pgfpathlineto{\pgfqpoint{2.362079in}{2.662142in}}%
\pgfpathlineto{\pgfqpoint{2.365214in}{2.662142in}}%
\pgfpathlineto{\pgfqpoint{2.365527in}{2.662518in}}%
\pgfpathlineto{\pgfqpoint{2.368348in}{2.665894in}}%
\pgfpathlineto{\pgfqpoint{2.371483in}{2.665894in}}%
\pgfpathlineto{\pgfqpoint{2.374618in}{2.665894in}}%
\pgfpathlineto{\pgfqpoint{2.377753in}{2.665894in}}%
\pgfpathlineto{\pgfqpoint{2.378066in}{2.666269in}}%
\pgfpathlineto{\pgfqpoint{2.380887in}{2.669646in}}%
\pgfpathlineto{\pgfqpoint{2.384022in}{2.669646in}}%
\pgfpathlineto{\pgfqpoint{2.387157in}{2.669646in}}%
\pgfpathlineto{\pgfqpoint{2.390292in}{2.669646in}}%
\pgfpathlineto{\pgfqpoint{2.393426in}{2.669646in}}%
\pgfpathlineto{\pgfqpoint{2.393740in}{2.670021in}}%
\pgfpathlineto{\pgfqpoint{2.396561in}{2.673397in}}%
\pgfpathlineto{\pgfqpoint{2.399696in}{2.673397in}}%
\pgfpathlineto{\pgfqpoint{2.402831in}{2.673397in}}%
\pgfpathlineto{\pgfqpoint{2.405965in}{2.673397in}}%
\pgfpathlineto{\pgfqpoint{2.406279in}{2.673772in}}%
\pgfpathlineto{\pgfqpoint{2.409100in}{2.677149in}}%
\pgfpathlineto{\pgfqpoint{2.412235in}{2.677149in}}%
\pgfpathlineto{\pgfqpoint{2.415369in}{2.677149in}}%
\pgfpathlineto{\pgfqpoint{2.418504in}{2.677149in}}%
\pgfpathlineto{\pgfqpoint{2.418818in}{2.677524in}}%
\pgfpathlineto{\pgfqpoint{2.421639in}{2.680900in}}%
\pgfpathlineto{\pgfqpoint{2.424774in}{2.680900in}}%
\pgfpathlineto{\pgfqpoint{2.427908in}{2.680900in}}%
\pgfpathlineto{\pgfqpoint{2.431043in}{2.680900in}}%
\pgfpathlineto{\pgfqpoint{2.431357in}{2.681275in}}%
\pgfpathlineto{\pgfqpoint{2.434178in}{2.684652in}}%
\pgfpathlineto{\pgfqpoint{2.437313in}{2.684652in}}%
\pgfpathlineto{\pgfqpoint{2.440447in}{2.684652in}}%
\pgfpathlineto{\pgfqpoint{2.443582in}{2.684652in}}%
\pgfpathlineto{\pgfqpoint{2.446717in}{2.684652in}}%
\pgfpathlineto{\pgfqpoint{2.447030in}{2.685027in}}%
\pgfpathlineto{\pgfqpoint{2.449852in}{2.688403in}}%
\pgfpathlineto{\pgfqpoint{2.452986in}{2.688403in}}%
\pgfpathlineto{\pgfqpoint{2.456121in}{2.688403in}}%
\pgfpathlineto{\pgfqpoint{2.459256in}{2.688403in}}%
\pgfpathlineto{\pgfqpoint{2.459569in}{2.688778in}}%
\pgfpathlineto{\pgfqpoint{2.462391in}{2.692155in}}%
\pgfpathlineto{\pgfqpoint{2.465525in}{2.692155in}}%
\pgfpathlineto{\pgfqpoint{2.468660in}{2.692155in}}%
\pgfpathlineto{\pgfqpoint{2.471795in}{2.692155in}}%
\pgfpathlineto{\pgfqpoint{2.472108in}{2.692530in}}%
\pgfpathlineto{\pgfqpoint{2.474930in}{2.695906in}}%
\pgfpathlineto{\pgfqpoint{2.478064in}{2.695906in}}%
\pgfpathlineto{\pgfqpoint{2.481199in}{2.695906in}}%
\pgfpathlineto{\pgfqpoint{2.484334in}{2.695906in}}%
\pgfpathlineto{\pgfqpoint{2.484647in}{2.696281in}}%
\pgfpathlineto{\pgfqpoint{2.487469in}{2.699658in}}%
\pgfpathlineto{\pgfqpoint{2.490603in}{2.699658in}}%
\pgfpathlineto{\pgfqpoint{2.493738in}{2.699658in}}%
\pgfpathlineto{\pgfqpoint{2.496873in}{2.699658in}}%
\pgfpathlineto{\pgfqpoint{2.497186in}{2.700033in}}%
\pgfpathlineto{\pgfqpoint{2.500008in}{2.703409in}}%
\pgfpathlineto{\pgfqpoint{2.503142in}{2.703409in}}%
\pgfpathlineto{\pgfqpoint{2.506277in}{2.703409in}}%
\pgfpathlineto{\pgfqpoint{2.509412in}{2.703409in}}%
\pgfpathlineto{\pgfqpoint{2.512547in}{2.703409in}}%
\pgfpathlineto{\pgfqpoint{2.512860in}{2.703785in}}%
\pgfpathlineto{\pgfqpoint{2.515681in}{2.707161in}}%
\pgfpathlineto{\pgfqpoint{2.518816in}{2.707161in}}%
\pgfpathlineto{\pgfqpoint{2.521951in}{2.707161in}}%
\pgfpathlineto{\pgfqpoint{2.525085in}{2.707161in}}%
\pgfpathlineto{\pgfqpoint{2.525399in}{2.707536in}}%
\pgfpathlineto{\pgfqpoint{2.528220in}{2.710912in}}%
\pgfpathlineto{\pgfqpoint{2.531355in}{2.710912in}}%
\pgfpathlineto{\pgfqpoint{2.534490in}{2.710912in}}%
\pgfpathlineto{\pgfqpoint{2.537624in}{2.710912in}}%
\pgfpathlineto{\pgfqpoint{2.537938in}{2.711288in}}%
\pgfpathlineto{\pgfqpoint{2.540759in}{2.714664in}}%
\pgfpathlineto{\pgfqpoint{2.543894in}{2.714664in}}%
\pgfpathlineto{\pgfqpoint{2.547029in}{2.714664in}}%
\pgfpathlineto{\pgfqpoint{2.550163in}{2.714664in}}%
\pgfpathlineto{\pgfqpoint{2.550477in}{2.715039in}}%
\pgfpathlineto{\pgfqpoint{2.553298in}{2.718416in}}%
\pgfpathlineto{\pgfqpoint{2.556433in}{2.718416in}}%
\pgfpathlineto{\pgfqpoint{2.559568in}{2.718416in}}%
\pgfpathlineto{\pgfqpoint{2.562702in}{2.718416in}}%
\pgfpathlineto{\pgfqpoint{2.565837in}{2.718416in}}%
\pgfpathlineto{\pgfqpoint{2.566151in}{2.718791in}}%
\pgfpathlineto{\pgfqpoint{2.568972in}{2.722167in}}%
\pgfpathlineto{\pgfqpoint{2.572107in}{2.722167in}}%
\pgfpathlineto{\pgfqpoint{2.575241in}{2.722167in}}%
\pgfpathlineto{\pgfqpoint{2.578376in}{2.722167in}}%
\pgfpathlineto{\pgfqpoint{2.578690in}{2.722542in}}%
\pgfpathlineto{\pgfqpoint{2.581511in}{2.725919in}}%
\pgfpathlineto{\pgfqpoint{2.584646in}{2.725919in}}%
\pgfpathlineto{\pgfqpoint{2.587780in}{2.725919in}}%
\pgfpathlineto{\pgfqpoint{2.590915in}{2.725919in}}%
\pgfpathlineto{\pgfqpoint{2.591229in}{2.726294in}}%
\pgfpathlineto{\pgfqpoint{2.594050in}{2.729670in}}%
\pgfpathlineto{\pgfqpoint{2.597185in}{2.729670in}}%
\pgfpathlineto{\pgfqpoint{2.600319in}{2.729670in}}%
\pgfpathlineto{\pgfqpoint{2.603454in}{2.729670in}}%
\pgfpathlineto{\pgfqpoint{2.603768in}{2.730045in}}%
\pgfpathlineto{\pgfqpoint{2.606589in}{2.733422in}}%
\pgfpathlineto{\pgfqpoint{2.609724in}{2.733422in}}%
\pgfpathlineto{\pgfqpoint{2.612858in}{2.733422in}}%
\pgfpathlineto{\pgfqpoint{2.615993in}{2.733422in}}%
\pgfpathlineto{\pgfqpoint{2.616306in}{2.733797in}}%
\pgfpathlineto{\pgfqpoint{2.619128in}{2.737173in}}%
\pgfpathlineto{\pgfqpoint{2.622262in}{2.737173in}}%
\pgfpathlineto{\pgfqpoint{2.625397in}{2.737173in}}%
\pgfpathlineto{\pgfqpoint{2.628532in}{2.737173in}}%
\pgfpathlineto{\pgfqpoint{2.631667in}{2.737173in}}%
\pgfpathlineto{\pgfqpoint{2.631980in}{2.737548in}}%
\pgfpathlineto{\pgfqpoint{2.634801in}{2.740925in}}%
\pgfpathlineto{\pgfqpoint{2.637936in}{2.740925in}}%
\pgfpathlineto{\pgfqpoint{2.641071in}{2.740925in}}%
\pgfpathlineto{\pgfqpoint{2.644206in}{2.740925in}}%
\pgfpathlineto{\pgfqpoint{2.644519in}{2.741300in}}%
\pgfpathlineto{\pgfqpoint{2.647340in}{2.744676in}}%
\pgfpathlineto{\pgfqpoint{2.650475in}{2.744676in}}%
\pgfpathlineto{\pgfqpoint{2.653610in}{2.744676in}}%
\pgfpathlineto{\pgfqpoint{2.656745in}{2.744676in}}%
\pgfpathlineto{\pgfqpoint{2.657058in}{2.745051in}}%
\pgfpathlineto{\pgfqpoint{2.659879in}{2.748428in}}%
\pgfpathlineto{\pgfqpoint{2.663014in}{2.748428in}}%
\pgfpathlineto{\pgfqpoint{2.666149in}{2.748428in}}%
\pgfpathlineto{\pgfqpoint{2.669284in}{2.748428in}}%
\pgfpathlineto{\pgfqpoint{2.672418in}{2.748428in}}%
\pgfpathlineto{\pgfqpoint{2.675240in}{2.745051in}}%
\pgfpathlineto{\pgfqpoint{2.675553in}{2.744676in}}%
\pgfpathlineto{\pgfqpoint{2.678374in}{2.741300in}}%
\pgfpathlineto{\pgfqpoint{2.678688in}{2.740925in}}%
\pgfpathlineto{\pgfqpoint{2.681509in}{2.737548in}}%
\pgfpathlineto{\pgfqpoint{2.681823in}{2.737173in}}%
\pgfpathlineto{\pgfqpoint{2.684644in}{2.733797in}}%
\pgfpathlineto{\pgfqpoint{2.684957in}{2.733422in}}%
\pgfpathlineto{\pgfqpoint{2.687779in}{2.730045in}}%
\pgfpathlineto{\pgfqpoint{2.688092in}{2.729670in}}%
\pgfpathlineto{\pgfqpoint{2.690913in}{2.726294in}}%
\pgfpathlineto{\pgfqpoint{2.691227in}{2.725919in}}%
\pgfpathlineto{\pgfqpoint{2.694048in}{2.722542in}}%
\pgfpathlineto{\pgfqpoint{2.694362in}{2.722167in}}%
\pgfpathlineto{\pgfqpoint{2.697183in}{2.718791in}}%
\pgfpathlineto{\pgfqpoint{2.697496in}{2.718416in}}%
\pgfpathlineto{\pgfqpoint{2.700318in}{2.715039in}}%
\pgfpathlineto{\pgfqpoint{2.700631in}{2.714664in}}%
\pgfpathlineto{\pgfqpoint{2.703452in}{2.711288in}}%
\pgfpathlineto{\pgfqpoint{2.703766in}{2.710912in}}%
\pgfpathlineto{\pgfqpoint{2.706587in}{2.707536in}}%
\pgfpathlineto{\pgfqpoint{2.706901in}{2.707161in}}%
\pgfpathlineto{\pgfqpoint{2.709722in}{2.703785in}}%
\pgfpathlineto{\pgfqpoint{2.710035in}{2.703409in}}%
\pgfpathlineto{\pgfqpoint{2.712857in}{2.700033in}}%
\pgfpathlineto{\pgfqpoint{2.712857in}{2.696281in}}%
\pgfpathlineto{\pgfqpoint{2.713170in}{2.695906in}}%
\pgfpathlineto{\pgfqpoint{2.715991in}{2.692530in}}%
\pgfpathlineto{\pgfqpoint{2.716305in}{2.692155in}}%
\pgfpathlineto{\pgfqpoint{2.719126in}{2.688778in}}%
\pgfpathlineto{\pgfqpoint{2.719439in}{2.688403in}}%
\pgfpathlineto{\pgfqpoint{2.722261in}{2.685027in}}%
\pgfpathlineto{\pgfqpoint{2.722574in}{2.684652in}}%
\pgfpathlineto{\pgfqpoint{2.725396in}{2.681275in}}%
\pgfpathlineto{\pgfqpoint{2.725709in}{2.680900in}}%
\pgfpathlineto{\pgfqpoint{2.728530in}{2.677524in}}%
\pgfpathlineto{\pgfqpoint{2.728844in}{2.677149in}}%
\pgfpathlineto{\pgfqpoint{2.731665in}{2.673772in}}%
\pgfpathlineto{\pgfqpoint{2.731978in}{2.673397in}}%
\pgfpathlineto{\pgfqpoint{2.734800in}{2.670021in}}%
\pgfpathlineto{\pgfqpoint{2.735113in}{2.669646in}}%
\pgfpathlineto{\pgfqpoint{2.737934in}{2.666269in}}%
\pgfpathlineto{\pgfqpoint{2.738248in}{2.665894in}}%
\pgfpathlineto{\pgfqpoint{2.741069in}{2.662518in}}%
\pgfpathlineto{\pgfqpoint{2.741383in}{2.662142in}}%
\pgfpathlineto{\pgfqpoint{2.744204in}{2.658766in}}%
\pgfpathlineto{\pgfqpoint{2.744517in}{2.658391in}}%
\pgfpathlineto{\pgfqpoint{2.747339in}{2.655015in}}%
\pgfpathlineto{\pgfqpoint{2.747652in}{2.654639in}}%
\pgfpathlineto{\pgfqpoint{2.750473in}{2.651263in}}%
\pgfpathlineto{\pgfqpoint{2.750787in}{2.650888in}}%
\pgfpathlineto{\pgfqpoint{2.753608in}{2.647512in}}%
\pgfpathlineto{\pgfqpoint{2.753922in}{2.647136in}}%
\pgfpathlineto{\pgfqpoint{2.756743in}{2.643760in}}%
\pgfpathlineto{\pgfqpoint{2.757056in}{2.643385in}}%
\pgfpathlineto{\pgfqpoint{2.759878in}{2.640008in}}%
\pgfpathlineto{\pgfqpoint{2.760191in}{2.639633in}}%
\pgfpathlineto{\pgfqpoint{2.763012in}{2.636257in}}%
\pgfpathlineto{\pgfqpoint{2.763012in}{2.632505in}}%
\pgfpathlineto{\pgfqpoint{2.763326in}{2.632130in}}%
\pgfpathlineto{\pgfqpoint{2.766147in}{2.628754in}}%
\pgfpathlineto{\pgfqpoint{2.766461in}{2.628379in}}%
\pgfpathlineto{\pgfqpoint{2.769282in}{2.625002in}}%
\pgfpathlineto{\pgfqpoint{2.769595in}{2.624627in}}%
\pgfpathlineto{\pgfqpoint{2.772417in}{2.621251in}}%
\pgfpathlineto{\pgfqpoint{2.772730in}{2.620876in}}%
\pgfpathlineto{\pgfqpoint{2.775551in}{2.617499in}}%
\pgfpathlineto{\pgfqpoint{2.775865in}{2.617124in}}%
\pgfpathlineto{\pgfqpoint{2.778686in}{2.613748in}}%
\pgfpathlineto{\pgfqpoint{2.779000in}{2.613373in}}%
\pgfpathlineto{\pgfqpoint{2.781821in}{2.609996in}}%
\pgfpathlineto{\pgfqpoint{2.782134in}{2.609621in}}%
\pgfpathlineto{\pgfqpoint{2.784956in}{2.606245in}}%
\pgfpathlineto{\pgfqpoint{2.785269in}{2.605869in}}%
\pgfpathlineto{\pgfqpoint{2.788090in}{2.602493in}}%
\pgfpathlineto{\pgfqpoint{2.788404in}{2.602118in}}%
\pgfpathlineto{\pgfqpoint{2.791225in}{2.598742in}}%
\pgfpathlineto{\pgfqpoint{2.791539in}{2.598366in}}%
\pgfpathlineto{\pgfqpoint{2.794360in}{2.594990in}}%
\pgfpathlineto{\pgfqpoint{2.794673in}{2.594615in}}%
\pgfpathlineto{\pgfqpoint{2.797495in}{2.591238in}}%
\pgfpathlineto{\pgfqpoint{2.797808in}{2.590863in}}%
\pgfpathlineto{\pgfqpoint{2.800629in}{2.587487in}}%
\pgfpathlineto{\pgfqpoint{2.800943in}{2.587112in}}%
\pgfpathlineto{\pgfqpoint{2.803764in}{2.583735in}}%
\pgfpathlineto{\pgfqpoint{2.804078in}{2.583360in}}%
\pgfpathlineto{\pgfqpoint{2.806899in}{2.579984in}}%
\pgfpathlineto{\pgfqpoint{2.807212in}{2.579609in}}%
\pgfpathlineto{\pgfqpoint{2.810034in}{2.576232in}}%
\pgfpathlineto{\pgfqpoint{2.810034in}{2.572481in}}%
\pgfpathlineto{\pgfqpoint{2.810347in}{2.572106in}}%
\pgfpathlineto{\pgfqpoint{2.813168in}{2.568729in}}%
\pgfpathlineto{\pgfqpoint{2.813482in}{2.568354in}}%
\pgfpathlineto{\pgfqpoint{2.816303in}{2.564978in}}%
\pgfpathlineto{\pgfqpoint{2.816617in}{2.564603in}}%
\pgfpathlineto{\pgfqpoint{2.819438in}{2.561226in}}%
\pgfpathlineto{\pgfqpoint{2.819751in}{2.560851in}}%
\pgfpathlineto{\pgfqpoint{2.822573in}{2.557475in}}%
\pgfpathlineto{\pgfqpoint{2.822886in}{2.557100in}}%
\pgfpathlineto{\pgfqpoint{2.825707in}{2.553723in}}%
\pgfpathlineto{\pgfqpoint{2.826021in}{2.553348in}}%
\pgfpathlineto{\pgfqpoint{2.828842in}{2.549972in}}%
\pgfpathlineto{\pgfqpoint{2.829155in}{2.549596in}}%
\pgfpathlineto{\pgfqpoint{2.831977in}{2.546220in}}%
\pgfpathlineto{\pgfqpoint{2.832290in}{2.545845in}}%
\pgfpathlineto{\pgfqpoint{2.835111in}{2.542469in}}%
\pgfpathlineto{\pgfqpoint{2.835425in}{2.542093in}}%
\pgfpathlineto{\pgfqpoint{2.838246in}{2.538717in}}%
\pgfpathlineto{\pgfqpoint{2.838560in}{2.538342in}}%
\pgfpathlineto{\pgfqpoint{2.841381in}{2.534965in}}%
\pgfpathlineto{\pgfqpoint{2.841694in}{2.534590in}}%
\pgfpathlineto{\pgfqpoint{2.844516in}{2.531214in}}%
\pgfpathlineto{\pgfqpoint{2.844829in}{2.530839in}}%
\pgfpathlineto{\pgfqpoint{2.847650in}{2.527462in}}%
\pgfpathlineto{\pgfqpoint{2.847964in}{2.527087in}}%
\pgfpathlineto{\pgfqpoint{2.850785in}{2.523711in}}%
\pgfpathlineto{\pgfqpoint{2.851099in}{2.523336in}}%
\pgfpathlineto{\pgfqpoint{2.853920in}{2.519959in}}%
\pgfpathlineto{\pgfqpoint{2.854233in}{2.519584in}}%
\pgfpathlineto{\pgfqpoint{2.857055in}{2.516208in}}%
\pgfpathlineto{\pgfqpoint{2.857368in}{2.515833in}}%
\pgfpathlineto{\pgfqpoint{2.860189in}{2.512456in}}%
\pgfpathlineto{\pgfqpoint{2.860189in}{2.508705in}}%
\pgfpathlineto{\pgfqpoint{2.860503in}{2.508330in}}%
\pgfpathlineto{\pgfqpoint{2.863324in}{2.504953in}}%
\pgfpathlineto{\pgfqpoint{2.863638in}{2.504578in}}%
\pgfpathlineto{\pgfqpoint{2.866459in}{2.501202in}}%
\pgfpathlineto{\pgfqpoint{2.866772in}{2.500827in}}%
\pgfpathlineto{\pgfqpoint{2.869594in}{2.497450in}}%
\pgfpathlineto{\pgfqpoint{2.869907in}{2.497075in}}%
\pgfpathlineto{\pgfqpoint{2.872728in}{2.493699in}}%
\pgfpathlineto{\pgfqpoint{2.873042in}{2.493323in}}%
\pgfpathlineto{\pgfqpoint{2.875863in}{2.489947in}}%
\pgfpathlineto{\pgfqpoint{2.876177in}{2.489572in}}%
\pgfpathlineto{\pgfqpoint{2.878998in}{2.486196in}}%
\pgfpathlineto{\pgfqpoint{2.879311in}{2.485820in}}%
\pgfpathlineto{\pgfqpoint{2.882133in}{2.482444in}}%
\pgfpathlineto{\pgfqpoint{2.882446in}{2.482069in}}%
\pgfpathlineto{\pgfqpoint{2.885267in}{2.478692in}}%
\pgfpathlineto{\pgfqpoint{2.885581in}{2.478317in}}%
\pgfpathlineto{\pgfqpoint{2.888402in}{2.474941in}}%
\pgfpathlineto{\pgfqpoint{2.888716in}{2.474566in}}%
\pgfpathlineto{\pgfqpoint{2.891537in}{2.471189in}}%
\pgfpathlineto{\pgfqpoint{2.891850in}{2.470814in}}%
\pgfpathlineto{\pgfqpoint{2.894672in}{2.467438in}}%
\pgfpathlineto{\pgfqpoint{2.894985in}{2.467063in}}%
\pgfpathlineto{\pgfqpoint{2.897806in}{2.463686in}}%
\pgfpathlineto{\pgfqpoint{2.898120in}{2.463311in}}%
\pgfpathlineto{\pgfqpoint{2.900941in}{2.459935in}}%
\pgfpathlineto{\pgfqpoint{2.901255in}{2.459560in}}%
\pgfpathlineto{\pgfqpoint{2.904076in}{2.456183in}}%
\pgfpathlineto{\pgfqpoint{2.904389in}{2.455808in}}%
\pgfpathlineto{\pgfqpoint{2.907211in}{2.452432in}}%
\pgfpathlineto{\pgfqpoint{2.907211in}{2.448680in}}%
\pgfpathlineto{\pgfqpoint{2.907524in}{2.448305in}}%
\pgfpathlineto{\pgfqpoint{2.910345in}{2.444929in}}%
\pgfpathlineto{\pgfqpoint{2.910659in}{2.444553in}}%
\pgfpathlineto{\pgfqpoint{2.913480in}{2.441177in}}%
\pgfpathlineto{\pgfqpoint{2.913794in}{2.440802in}}%
\pgfpathlineto{\pgfqpoint{2.916615in}{2.437426in}}%
\pgfpathlineto{\pgfqpoint{2.916928in}{2.437050in}}%
\pgfpathlineto{\pgfqpoint{2.919750in}{2.433674in}}%
\pgfpathlineto{\pgfqpoint{2.920063in}{2.433299in}}%
\pgfpathlineto{\pgfqpoint{2.922884in}{2.429923in}}%
\pgfpathlineto{\pgfqpoint{2.923198in}{2.429547in}}%
\pgfpathlineto{\pgfqpoint{2.926019in}{2.426171in}}%
\pgfpathlineto{\pgfqpoint{2.926332in}{2.425796in}}%
\pgfpathlineto{\pgfqpoint{2.929154in}{2.422419in}}%
\pgfpathlineto{\pgfqpoint{2.929467in}{2.422044in}}%
\pgfpathlineto{\pgfqpoint{2.932288in}{2.418668in}}%
\pgfpathlineto{\pgfqpoint{2.932602in}{2.418293in}}%
\pgfpathlineto{\pgfqpoint{2.935423in}{2.414916in}}%
\pgfpathlineto{\pgfqpoint{2.935737in}{2.414541in}}%
\pgfpathlineto{\pgfqpoint{2.938558in}{2.411165in}}%
\pgfpathlineto{\pgfqpoint{2.938871in}{2.410790in}}%
\pgfpathlineto{\pgfqpoint{2.941693in}{2.407413in}}%
\pgfpathlineto{\pgfqpoint{2.942006in}{2.407038in}}%
\pgfpathlineto{\pgfqpoint{2.944827in}{2.403662in}}%
\pgfpathlineto{\pgfqpoint{2.945141in}{2.403287in}}%
\pgfpathlineto{\pgfqpoint{2.947962in}{2.399910in}}%
\pgfpathlineto{\pgfqpoint{2.948276in}{2.399535in}}%
\pgfpathlineto{\pgfqpoint{2.951097in}{2.396159in}}%
\pgfpathlineto{\pgfqpoint{2.951410in}{2.395784in}}%
\pgfpathlineto{\pgfqpoint{2.954232in}{2.392407in}}%
\pgfpathlineto{\pgfqpoint{2.954545in}{2.392032in}}%
\pgfpathlineto{\pgfqpoint{2.957366in}{2.388656in}}%
\pgfpathlineto{\pgfqpoint{2.957366in}{2.384904in}}%
\pgfpathlineto{\pgfqpoint{2.957680in}{2.384529in}}%
\pgfpathlineto{\pgfqpoint{2.960501in}{2.381153in}}%
\pgfpathlineto{\pgfqpoint{2.960815in}{2.380777in}}%
\pgfpathlineto{\pgfqpoint{2.963636in}{2.377401in}}%
\pgfpathlineto{\pgfqpoint{2.963949in}{2.377026in}}%
\pgfpathlineto{\pgfqpoint{2.966771in}{2.373649in}}%
\pgfpathlineto{\pgfqpoint{2.967084in}{2.373274in}}%
\pgfpathlineto{\pgfqpoint{2.969905in}{2.369898in}}%
\pgfpathlineto{\pgfqpoint{2.970219in}{2.369523in}}%
\pgfpathlineto{\pgfqpoint{2.973040in}{2.366146in}}%
\pgfpathlineto{\pgfqpoint{2.973354in}{2.365771in}}%
\pgfpathlineto{\pgfqpoint{2.976175in}{2.362395in}}%
\pgfpathlineto{\pgfqpoint{2.976488in}{2.362020in}}%
\pgfpathlineto{\pgfqpoint{2.979310in}{2.358643in}}%
\pgfpathlineto{\pgfqpoint{2.979623in}{2.358268in}}%
\pgfpathlineto{\pgfqpoint{2.982444in}{2.354892in}}%
\pgfpathlineto{\pgfqpoint{2.982758in}{2.354517in}}%
\pgfpathlineto{\pgfqpoint{2.985579in}{2.351140in}}%
\pgfpathlineto{\pgfqpoint{2.985893in}{2.350765in}}%
\pgfpathlineto{\pgfqpoint{2.988714in}{2.347389in}}%
\pgfpathlineto{\pgfqpoint{2.989027in}{2.347014in}}%
\pgfpathlineto{\pgfqpoint{2.991849in}{2.343637in}}%
\pgfpathlineto{\pgfqpoint{2.992162in}{2.343262in}}%
\pgfpathlineto{\pgfqpoint{2.994983in}{2.339886in}}%
\pgfpathlineto{\pgfqpoint{2.995297in}{2.339511in}}%
\pgfpathlineto{\pgfqpoint{2.998118in}{2.336134in}}%
\pgfpathlineto{\pgfqpoint{2.998432in}{2.335759in}}%
\pgfpathlineto{\pgfqpoint{3.001253in}{2.332383in}}%
\pgfpathlineto{\pgfqpoint{3.001566in}{2.332007in}}%
\pgfpathlineto{\pgfqpoint{3.004388in}{2.328631in}}%
\pgfpathlineto{\pgfqpoint{3.004388in}{2.324880in}}%
\pgfpathlineto{\pgfqpoint{3.004701in}{2.324504in}}%
\pgfpathlineto{\pgfqpoint{3.007522in}{2.321128in}}%
\pgfpathlineto{\pgfqpoint{3.007836in}{2.320753in}}%
\pgfpathlineto{\pgfqpoint{3.010657in}{2.317376in}}%
\pgfpathlineto{\pgfqpoint{3.010971in}{2.317001in}}%
\pgfpathlineto{\pgfqpoint{3.013792in}{2.313625in}}%
\pgfpathlineto{\pgfqpoint{3.014105in}{2.313250in}}%
\pgfpathlineto{\pgfqpoint{3.016927in}{2.309873in}}%
\pgfpathlineto{\pgfqpoint{3.017240in}{2.309498in}}%
\pgfpathlineto{\pgfqpoint{3.020061in}{2.306122in}}%
\pgfpathlineto{\pgfqpoint{3.020375in}{2.305747in}}%
\pgfpathlineto{\pgfqpoint{3.023196in}{2.302370in}}%
\pgfpathlineto{\pgfqpoint{3.023510in}{2.301995in}}%
\pgfpathlineto{\pgfqpoint{3.026331in}{2.298619in}}%
\pgfpathlineto{\pgfqpoint{3.026644in}{2.298244in}}%
\pgfpathlineto{\pgfqpoint{3.029466in}{2.294867in}}%
\pgfpathlineto{\pgfqpoint{3.029779in}{2.294492in}}%
\pgfpathlineto{\pgfqpoint{3.032600in}{2.291116in}}%
\pgfpathlineto{\pgfqpoint{3.032914in}{2.290741in}}%
\pgfpathlineto{\pgfqpoint{3.035735in}{2.287364in}}%
\pgfpathlineto{\pgfqpoint{3.036048in}{2.286989in}}%
\pgfpathlineto{\pgfqpoint{3.038870in}{2.283613in}}%
\pgfpathlineto{\pgfqpoint{3.039183in}{2.283238in}}%
\pgfpathlineto{\pgfqpoint{3.042004in}{2.279861in}}%
\pgfpathlineto{\pgfqpoint{3.042318in}{2.279486in}}%
\pgfpathlineto{\pgfqpoint{3.045139in}{2.276110in}}%
\pgfpathlineto{\pgfqpoint{3.045453in}{2.275734in}}%
\pgfpathlineto{\pgfqpoint{3.048274in}{2.272358in}}%
\pgfpathlineto{\pgfqpoint{3.048587in}{2.271983in}}%
\pgfpathlineto{\pgfqpoint{3.051409in}{2.268607in}}%
\pgfpathlineto{\pgfqpoint{3.051722in}{2.268231in}}%
\pgfpathlineto{\pgfqpoint{3.054543in}{2.264855in}}%
\pgfpathlineto{\pgfqpoint{3.054543in}{2.261103in}}%
\pgfpathlineto{\pgfqpoint{3.054857in}{2.260728in}}%
\pgfpathlineto{\pgfqpoint{3.057678in}{2.257352in}}%
\pgfpathlineto{\pgfqpoint{3.057992in}{2.256977in}}%
\pgfpathlineto{\pgfqpoint{3.060813in}{2.253600in}}%
\pgfpathlineto{\pgfqpoint{3.061126in}{2.253225in}}%
\pgfpathlineto{\pgfqpoint{3.063948in}{2.249849in}}%
\pgfpathlineto{\pgfqpoint{3.064261in}{2.249474in}}%
\pgfpathlineto{\pgfqpoint{3.067082in}{2.246097in}}%
\pgfpathlineto{\pgfqpoint{3.067396in}{2.245722in}}%
\pgfpathlineto{\pgfqpoint{3.070217in}{2.242346in}}%
\pgfpathlineto{\pgfqpoint{3.070531in}{2.241971in}}%
\pgfpathlineto{\pgfqpoint{3.073352in}{2.238594in}}%
\pgfpathlineto{\pgfqpoint{3.073665in}{2.238219in}}%
\pgfpathlineto{\pgfqpoint{3.076487in}{2.234843in}}%
\pgfpathlineto{\pgfqpoint{3.076800in}{2.234468in}}%
\pgfpathlineto{\pgfqpoint{3.079621in}{2.231091in}}%
\pgfpathlineto{\pgfqpoint{3.079935in}{2.230716in}}%
\pgfpathlineto{\pgfqpoint{3.082756in}{2.227340in}}%
\pgfpathlineto{\pgfqpoint{3.083070in}{2.226964in}}%
\pgfpathlineto{\pgfqpoint{3.085891in}{2.223588in}}%
\pgfpathlineto{\pgfqpoint{3.086204in}{2.223213in}}%
\pgfpathlineto{\pgfqpoint{3.089026in}{2.219837in}}%
\pgfpathlineto{\pgfqpoint{3.089339in}{2.219461in}}%
\pgfpathlineto{\pgfqpoint{3.092160in}{2.216085in}}%
\pgfpathlineto{\pgfqpoint{3.092474in}{2.215710in}}%
\pgfpathlineto{\pgfqpoint{3.095295in}{2.212334in}}%
\pgfpathlineto{\pgfqpoint{3.095295in}{2.208582in}}%
\pgfpathlineto{\pgfqpoint{3.092474in}{2.205206in}}%
\pgfpathlineto{\pgfqpoint{3.092160in}{2.204830in}}%
\pgfpathlineto{\pgfqpoint{3.092160in}{2.201079in}}%
\pgfpathlineto{\pgfqpoint{3.092160in}{2.197327in}}%
\pgfpathlineto{\pgfqpoint{3.089339in}{2.193951in}}%
\pgfpathlineto{\pgfqpoint{3.089026in}{2.193576in}}%
\pgfpathlineto{\pgfqpoint{3.089026in}{2.189824in}}%
\pgfpathlineto{\pgfqpoint{3.089026in}{2.186073in}}%
\pgfpathlineto{\pgfqpoint{3.086204in}{2.182696in}}%
\pgfpathlineto{\pgfqpoint{3.085891in}{2.182321in}}%
\pgfpathlineto{\pgfqpoint{3.085891in}{2.178570in}}%
\pgfpathlineto{\pgfqpoint{3.085891in}{2.174818in}}%
\pgfpathlineto{\pgfqpoint{3.083070in}{2.171442in}}%
\pgfpathlineto{\pgfqpoint{3.082756in}{2.171067in}}%
\pgfpathlineto{\pgfqpoint{3.082756in}{2.167315in}}%
\pgfpathlineto{\pgfqpoint{3.082756in}{2.163564in}}%
\pgfpathlineto{\pgfqpoint{3.079935in}{2.160187in}}%
\pgfpathlineto{\pgfqpoint{3.079621in}{2.159812in}}%
\pgfpathlineto{\pgfqpoint{3.079621in}{2.156060in}}%
\pgfpathlineto{\pgfqpoint{3.079621in}{2.152309in}}%
\pgfpathlineto{\pgfqpoint{3.079621in}{2.148557in}}%
\pgfpathlineto{\pgfqpoint{3.076800in}{2.145181in}}%
\pgfpathlineto{\pgfqpoint{3.076487in}{2.144806in}}%
\pgfpathlineto{\pgfqpoint{3.076487in}{2.141054in}}%
\pgfpathlineto{\pgfqpoint{3.076487in}{2.137303in}}%
\pgfpathlineto{\pgfqpoint{3.073665in}{2.133926in}}%
\pgfpathlineto{\pgfqpoint{3.073352in}{2.133551in}}%
\pgfpathlineto{\pgfqpoint{3.073352in}{2.129800in}}%
\pgfpathlineto{\pgfqpoint{3.073352in}{2.126048in}}%
\pgfpathlineto{\pgfqpoint{3.070531in}{2.122672in}}%
\pgfpathlineto{\pgfqpoint{3.070217in}{2.122297in}}%
\pgfpathlineto{\pgfqpoint{3.070217in}{2.118545in}}%
\pgfpathlineto{\pgfqpoint{3.070217in}{2.114794in}}%
\pgfpathlineto{\pgfqpoint{3.067396in}{2.111417in}}%
\pgfpathlineto{\pgfqpoint{3.067082in}{2.111042in}}%
\pgfpathlineto{\pgfqpoint{3.067082in}{2.107291in}}%
\pgfpathlineto{\pgfqpoint{3.067082in}{2.103539in}}%
\pgfpathlineto{\pgfqpoint{3.064261in}{2.100163in}}%
\pgfpathlineto{\pgfqpoint{3.063948in}{2.099787in}}%
\pgfpathlineto{\pgfqpoint{3.063948in}{2.096036in}}%
\pgfpathlineto{\pgfqpoint{3.063948in}{2.092284in}}%
\pgfpathlineto{\pgfqpoint{3.061126in}{2.088908in}}%
\pgfpathlineto{\pgfqpoint{3.060813in}{2.088533in}}%
\pgfpathlineto{\pgfqpoint{3.060813in}{2.084781in}}%
\pgfpathlineto{\pgfqpoint{3.060813in}{2.081030in}}%
\pgfpathlineto{\pgfqpoint{3.057992in}{2.077653in}}%
\pgfpathlineto{\pgfqpoint{3.057678in}{2.077278in}}%
\pgfpathlineto{\pgfqpoint{3.057678in}{2.073527in}}%
\pgfpathlineto{\pgfqpoint{3.057678in}{2.069775in}}%
\pgfpathlineto{\pgfqpoint{3.054857in}{2.066399in}}%
\pgfpathlineto{\pgfqpoint{3.054543in}{2.066024in}}%
\pgfpathlineto{\pgfqpoint{3.054543in}{2.062272in}}%
\pgfpathlineto{\pgfqpoint{3.054543in}{2.058521in}}%
\pgfpathlineto{\pgfqpoint{3.051722in}{2.055144in}}%
\pgfpathlineto{\pgfqpoint{3.051409in}{2.054769in}}%
\pgfpathlineto{\pgfqpoint{3.051409in}{2.051018in}}%
\pgfpathlineto{\pgfqpoint{3.051409in}{2.047266in}}%
\pgfpathlineto{\pgfqpoint{3.051409in}{2.043514in}}%
\pgfpathlineto{\pgfqpoint{3.048587in}{2.040138in}}%
\pgfpathlineto{\pgfqpoint{3.048274in}{2.039763in}}%
\pgfpathlineto{\pgfqpoint{3.048274in}{2.036011in}}%
\pgfpathlineto{\pgfqpoint{3.048274in}{2.032260in}}%
\pgfpathlineto{\pgfqpoint{3.045453in}{2.028883in}}%
\pgfpathlineto{\pgfqpoint{3.045139in}{2.028508in}}%
\pgfpathlineto{\pgfqpoint{3.045139in}{2.024757in}}%
\pgfpathlineto{\pgfqpoint{3.045139in}{2.021005in}}%
\pgfpathlineto{\pgfqpoint{3.042318in}{2.017629in}}%
\pgfpathlineto{\pgfqpoint{3.042004in}{2.017254in}}%
\pgfpathlineto{\pgfqpoint{3.042004in}{2.013502in}}%
\pgfpathlineto{\pgfqpoint{3.042004in}{2.009751in}}%
\pgfpathlineto{\pgfqpoint{3.039183in}{2.006374in}}%
\pgfpathlineto{\pgfqpoint{3.038870in}{2.005999in}}%
\pgfpathlineto{\pgfqpoint{3.038870in}{2.002248in}}%
\pgfpathlineto{\pgfqpoint{3.038870in}{1.998496in}}%
\pgfpathlineto{\pgfqpoint{3.036048in}{1.995120in}}%
\pgfpathlineto{\pgfqpoint{3.035735in}{1.994745in}}%
\pgfpathlineto{\pgfqpoint{3.035735in}{1.990993in}}%
\pgfpathlineto{\pgfqpoint{3.035735in}{1.987241in}}%
\pgfpathlineto{\pgfqpoint{3.032914in}{1.983865in}}%
\pgfpathlineto{\pgfqpoint{3.032600in}{1.983490in}}%
\pgfpathlineto{\pgfqpoint{3.032600in}{1.979738in}}%
\pgfpathlineto{\pgfqpoint{3.032600in}{1.975987in}}%
\pgfpathlineto{\pgfqpoint{3.029779in}{1.972610in}}%
\pgfpathlineto{\pgfqpoint{3.029466in}{1.972235in}}%
\pgfpathlineto{\pgfqpoint{3.029466in}{1.968484in}}%
\pgfpathlineto{\pgfqpoint{3.029466in}{1.964732in}}%
\pgfpathlineto{\pgfqpoint{3.026644in}{1.961356in}}%
\pgfpathlineto{\pgfqpoint{3.026331in}{1.960981in}}%
\pgfpathlineto{\pgfqpoint{3.026331in}{1.957229in}}%
\pgfpathlineto{\pgfqpoint{3.026331in}{1.953478in}}%
\pgfpathlineto{\pgfqpoint{3.026331in}{1.949726in}}%
\pgfpathlineto{\pgfqpoint{3.023510in}{1.946350in}}%
\pgfpathlineto{\pgfqpoint{3.023196in}{1.945975in}}%
\pgfpathlineto{\pgfqpoint{3.023196in}{1.942223in}}%
\pgfpathlineto{\pgfqpoint{3.023196in}{1.938471in}}%
\pgfpathlineto{\pgfqpoint{3.020375in}{1.935095in}}%
\pgfpathlineto{\pgfqpoint{3.020061in}{1.934720in}}%
\pgfpathlineto{\pgfqpoint{3.020061in}{1.930968in}}%
\pgfpathlineto{\pgfqpoint{3.020061in}{1.927217in}}%
\pgfpathlineto{\pgfqpoint{3.017240in}{1.923840in}}%
\pgfpathlineto{\pgfqpoint{3.016927in}{1.923465in}}%
\pgfpathlineto{\pgfqpoint{3.016927in}{1.919714in}}%
\pgfpathlineto{\pgfqpoint{3.016927in}{1.915962in}}%
\pgfpathlineto{\pgfqpoint{3.014105in}{1.912586in}}%
\pgfpathlineto{\pgfqpoint{3.013792in}{1.912211in}}%
\pgfpathlineto{\pgfqpoint{3.013792in}{1.908459in}}%
\pgfpathlineto{\pgfqpoint{3.013792in}{1.904708in}}%
\pgfpathlineto{\pgfqpoint{3.010971in}{1.901331in}}%
\pgfpathlineto{\pgfqpoint{3.010657in}{1.900956in}}%
\pgfpathlineto{\pgfqpoint{3.010657in}{1.897205in}}%
\pgfpathlineto{\pgfqpoint{3.010657in}{1.893453in}}%
\pgfpathlineto{\pgfqpoint{3.007836in}{1.890077in}}%
\pgfpathlineto{\pgfqpoint{3.007522in}{1.889702in}}%
\pgfpathlineto{\pgfqpoint{3.007522in}{1.885950in}}%
\pgfpathlineto{\pgfqpoint{3.007522in}{1.882198in}}%
\pgfpathlineto{\pgfqpoint{3.004701in}{1.878822in}}%
\pgfpathlineto{\pgfqpoint{3.004388in}{1.878447in}}%
\pgfpathlineto{\pgfqpoint{3.004388in}{1.874695in}}%
\pgfpathlineto{\pgfqpoint{3.004388in}{1.870944in}}%
\pgfpathlineto{\pgfqpoint{3.001566in}{1.867567in}}%
\pgfpathlineto{\pgfqpoint{3.001253in}{1.867192in}}%
\pgfpathlineto{\pgfqpoint{3.001253in}{1.863441in}}%
\pgfpathlineto{\pgfqpoint{3.001253in}{1.859689in}}%
\pgfpathlineto{\pgfqpoint{2.998432in}{1.856313in}}%
\pgfpathlineto{\pgfqpoint{2.998118in}{1.855938in}}%
\pgfpathlineto{\pgfqpoint{2.998118in}{1.852186in}}%
\pgfpathlineto{\pgfqpoint{2.998118in}{1.848435in}}%
\pgfpathlineto{\pgfqpoint{2.998118in}{1.844683in}}%
\pgfpathlineto{\pgfqpoint{2.995297in}{1.841307in}}%
\pgfpathlineto{\pgfqpoint{2.994983in}{1.840932in}}%
\pgfpathlineto{\pgfqpoint{2.994983in}{1.837180in}}%
\pgfpathlineto{\pgfqpoint{2.994983in}{1.833429in}}%
\pgfpathlineto{\pgfqpoint{2.992162in}{1.830052in}}%
\pgfpathlineto{\pgfqpoint{2.991849in}{1.829677in}}%
\pgfpathlineto{\pgfqpoint{2.991849in}{1.825925in}}%
\pgfpathlineto{\pgfqpoint{2.991849in}{1.822174in}}%
\pgfpathlineto{\pgfqpoint{2.989027in}{1.818798in}}%
\pgfpathlineto{\pgfqpoint{2.988714in}{1.818422in}}%
\pgfpathlineto{\pgfqpoint{2.988714in}{1.814671in}}%
\pgfpathlineto{\pgfqpoint{2.988714in}{1.810919in}}%
\pgfpathlineto{\pgfqpoint{2.985893in}{1.807543in}}%
\pgfpathlineto{\pgfqpoint{2.985579in}{1.807168in}}%
\pgfpathlineto{\pgfqpoint{2.985579in}{1.803416in}}%
\pgfpathlineto{\pgfqpoint{2.985579in}{1.799665in}}%
\pgfpathlineto{\pgfqpoint{2.982758in}{1.796288in}}%
\pgfpathlineto{\pgfqpoint{2.982444in}{1.795913in}}%
\pgfpathlineto{\pgfqpoint{2.982444in}{1.792162in}}%
\pgfpathlineto{\pgfqpoint{2.982444in}{1.788410in}}%
\pgfpathlineto{\pgfqpoint{2.979623in}{1.785034in}}%
\pgfpathlineto{\pgfqpoint{2.979310in}{1.784659in}}%
\pgfpathlineto{\pgfqpoint{2.979310in}{1.780907in}}%
\pgfpathlineto{\pgfqpoint{2.979310in}{1.777155in}}%
\pgfpathlineto{\pgfqpoint{2.976488in}{1.773779in}}%
\pgfpathlineto{\pgfqpoint{2.976175in}{1.773404in}}%
\pgfpathlineto{\pgfqpoint{2.976175in}{1.769652in}}%
\pgfpathlineto{\pgfqpoint{2.976175in}{1.765901in}}%
\pgfpathlineto{\pgfqpoint{2.973354in}{1.762525in}}%
\pgfpathlineto{\pgfqpoint{2.973040in}{1.762149in}}%
\pgfpathlineto{\pgfqpoint{2.973040in}{1.758398in}}%
\pgfpathlineto{\pgfqpoint{2.973040in}{1.754646in}}%
\pgfpathlineto{\pgfqpoint{2.973040in}{1.750895in}}%
\pgfpathlineto{\pgfqpoint{2.970219in}{1.747518in}}%
\pgfpathlineto{\pgfqpoint{2.969905in}{1.747143in}}%
\pgfpathlineto{\pgfqpoint{2.969905in}{1.743392in}}%
\pgfpathlineto{\pgfqpoint{2.969905in}{1.739640in}}%
\pgfpathlineto{\pgfqpoint{2.967084in}{1.736264in}}%
\pgfpathlineto{\pgfqpoint{2.966771in}{1.735889in}}%
\pgfpathlineto{\pgfqpoint{2.966771in}{1.732137in}}%
\pgfpathlineto{\pgfqpoint{2.966771in}{1.728386in}}%
\pgfpathlineto{\pgfqpoint{2.963949in}{1.725009in}}%
\pgfpathlineto{\pgfqpoint{2.963636in}{1.724634in}}%
\pgfpathlineto{\pgfqpoint{2.963636in}{1.720882in}}%
\pgfpathlineto{\pgfqpoint{2.963636in}{1.717131in}}%
\pgfpathlineto{\pgfqpoint{2.960815in}{1.713755in}}%
\pgfpathlineto{\pgfqpoint{2.960501in}{1.713379in}}%
\pgfpathlineto{\pgfqpoint{2.960501in}{1.709628in}}%
\pgfpathlineto{\pgfqpoint{2.960501in}{1.705876in}}%
\pgfpathlineto{\pgfqpoint{2.957680in}{1.702500in}}%
\pgfpathlineto{\pgfqpoint{2.957366in}{1.702125in}}%
\pgfpathlineto{\pgfqpoint{2.957366in}{1.698373in}}%
\pgfpathlineto{\pgfqpoint{2.957366in}{1.694622in}}%
\pgfpathlineto{\pgfqpoint{2.954545in}{1.691245in}}%
\pgfpathlineto{\pgfqpoint{2.954232in}{1.690870in}}%
\pgfpathlineto{\pgfqpoint{2.954232in}{1.687119in}}%
\pgfpathlineto{\pgfqpoint{2.954232in}{1.683367in}}%
\pgfpathlineto{\pgfqpoint{2.951410in}{1.679991in}}%
\pgfpathlineto{\pgfqpoint{2.951097in}{1.679616in}}%
\pgfpathlineto{\pgfqpoint{2.951097in}{1.675864in}}%
\pgfpathlineto{\pgfqpoint{2.951097in}{1.672113in}}%
\pgfpathlineto{\pgfqpoint{2.948276in}{1.668736in}}%
\pgfpathlineto{\pgfqpoint{2.947962in}{1.668361in}}%
\pgfpathlineto{\pgfqpoint{2.947962in}{1.664609in}}%
\pgfpathlineto{\pgfqpoint{2.947962in}{1.660858in}}%
\pgfpathlineto{\pgfqpoint{2.945141in}{1.657482in}}%
\pgfpathlineto{\pgfqpoint{2.944827in}{1.657106in}}%
\pgfpathlineto{\pgfqpoint{2.944827in}{1.653355in}}%
\pgfpathlineto{\pgfqpoint{2.944827in}{1.649603in}}%
\pgfpathlineto{\pgfqpoint{2.944827in}{1.645852in}}%
\pgfpathlineto{\pgfqpoint{2.942006in}{1.642475in}}%
\pgfpathlineto{\pgfqpoint{2.941693in}{1.642100in}}%
\pgfpathlineto{\pgfqpoint{2.941693in}{1.638349in}}%
\pgfpathlineto{\pgfqpoint{2.941693in}{1.634597in}}%
\pgfpathlineto{\pgfqpoint{2.938871in}{1.631221in}}%
\pgfpathlineto{\pgfqpoint{2.938558in}{1.630846in}}%
\pgfpathlineto{\pgfqpoint{2.938558in}{1.627094in}}%
\pgfpathlineto{\pgfqpoint{2.938558in}{1.623343in}}%
\pgfpathlineto{\pgfqpoint{2.935737in}{1.619966in}}%
\pgfpathlineto{\pgfqpoint{2.935423in}{1.619591in}}%
\pgfpathlineto{\pgfqpoint{2.935423in}{1.615840in}}%
\pgfpathlineto{\pgfqpoint{2.935423in}{1.612088in}}%
\pgfpathlineto{\pgfqpoint{2.932602in}{1.608712in}}%
\pgfpathlineto{\pgfqpoint{2.932288in}{1.608336in}}%
\pgfpathlineto{\pgfqpoint{2.932288in}{1.604585in}}%
\pgfpathlineto{\pgfqpoint{2.932288in}{1.600833in}}%
\pgfpathlineto{\pgfqpoint{2.929467in}{1.597457in}}%
\pgfpathlineto{\pgfqpoint{2.929154in}{1.597082in}}%
\pgfpathlineto{\pgfqpoint{2.929154in}{1.593330in}}%
\pgfpathlineto{\pgfqpoint{2.929154in}{1.589579in}}%
\pgfpathlineto{\pgfqpoint{2.926332in}{1.586202in}}%
\pgfpathlineto{\pgfqpoint{2.926019in}{1.585827in}}%
\pgfpathlineto{\pgfqpoint{2.926019in}{1.582076in}}%
\pgfpathlineto{\pgfqpoint{2.926019in}{1.578324in}}%
\pgfpathlineto{\pgfqpoint{2.923198in}{1.574948in}}%
\pgfpathlineto{\pgfqpoint{2.922884in}{1.574573in}}%
\pgfpathlineto{\pgfqpoint{2.922884in}{1.570821in}}%
\pgfpathlineto{\pgfqpoint{2.922884in}{1.567070in}}%
\pgfpathlineto{\pgfqpoint{2.920063in}{1.563693in}}%
\pgfpathlineto{\pgfqpoint{2.919750in}{1.563318in}}%
\pgfpathlineto{\pgfqpoint{2.919750in}{1.559566in}}%
\pgfpathlineto{\pgfqpoint{2.919750in}{1.555815in}}%
\pgfpathlineto{\pgfqpoint{2.919750in}{1.552063in}}%
\pgfpathlineto{\pgfqpoint{2.916928in}{1.548687in}}%
\pgfpathlineto{\pgfqpoint{2.916615in}{1.548312in}}%
\pgfpathlineto{\pgfqpoint{2.916615in}{1.544560in}}%
\pgfpathlineto{\pgfqpoint{2.916615in}{1.540809in}}%
\pgfpathlineto{\pgfqpoint{2.913794in}{1.537432in}}%
\pgfpathlineto{\pgfqpoint{2.913480in}{1.537057in}}%
\pgfpathlineto{\pgfqpoint{2.913480in}{1.533306in}}%
\pgfpathlineto{\pgfqpoint{2.913480in}{1.529554in}}%
\pgfpathlineto{\pgfqpoint{2.910659in}{1.526178in}}%
\pgfpathlineto{\pgfqpoint{2.910345in}{1.525803in}}%
\pgfpathlineto{\pgfqpoint{2.910345in}{1.522051in}}%
\pgfpathlineto{\pgfqpoint{2.910345in}{1.518300in}}%
\pgfpathlineto{\pgfqpoint{2.907524in}{1.514923in}}%
\pgfpathlineto{\pgfqpoint{2.907211in}{1.514548in}}%
\pgfpathlineto{\pgfqpoint{2.907211in}{1.510797in}}%
\pgfpathlineto{\pgfqpoint{2.907211in}{1.507045in}}%
\pgfpathlineto{\pgfqpoint{2.904389in}{1.503669in}}%
\pgfpathlineto{\pgfqpoint{2.904076in}{1.503293in}}%
\pgfpathlineto{\pgfqpoint{2.904076in}{1.499542in}}%
\pgfpathlineto{\pgfqpoint{2.904076in}{1.495790in}}%
\pgfpathlineto{\pgfqpoint{2.901255in}{1.492414in}}%
\pgfpathlineto{\pgfqpoint{2.900941in}{1.492039in}}%
\pgfpathlineto{\pgfqpoint{2.900941in}{1.488287in}}%
\pgfpathlineto{\pgfqpoint{2.900941in}{1.484536in}}%
\pgfpathlineto{\pgfqpoint{2.898120in}{1.481159in}}%
\pgfpathlineto{\pgfqpoint{2.897806in}{1.480784in}}%
\pgfpathlineto{\pgfqpoint{2.897806in}{1.477033in}}%
\pgfpathlineto{\pgfqpoint{2.897806in}{1.473281in}}%
\pgfpathlineto{\pgfqpoint{2.894985in}{1.469905in}}%
\pgfpathlineto{\pgfqpoint{2.894672in}{1.469530in}}%
\pgfpathlineto{\pgfqpoint{2.894672in}{1.465778in}}%
\pgfpathlineto{\pgfqpoint{2.894672in}{1.462027in}}%
\pgfpathlineto{\pgfqpoint{2.891850in}{1.458650in}}%
\pgfpathlineto{\pgfqpoint{2.891537in}{1.458275in}}%
\pgfpathlineto{\pgfqpoint{2.891537in}{1.454524in}}%
\pgfpathlineto{\pgfqpoint{2.891537in}{1.450772in}}%
\pgfpathlineto{\pgfqpoint{2.891537in}{1.447020in}}%
\pgfpathlineto{\pgfqpoint{2.888716in}{1.443644in}}%
\pgfpathlineto{\pgfqpoint{2.888402in}{1.443269in}}%
\pgfpathlineto{\pgfqpoint{2.888402in}{1.439517in}}%
\pgfpathlineto{\pgfqpoint{2.888402in}{1.435766in}}%
\pgfpathlineto{\pgfqpoint{2.885581in}{1.432389in}}%
\pgfpathlineto{\pgfqpoint{2.885267in}{1.432014in}}%
\pgfpathlineto{\pgfqpoint{2.885267in}{1.428263in}}%
\pgfpathlineto{\pgfqpoint{2.885267in}{1.424511in}}%
\pgfpathlineto{\pgfqpoint{2.882446in}{1.421135in}}%
\pgfpathlineto{\pgfqpoint{2.882133in}{1.420760in}}%
\pgfpathlineto{\pgfqpoint{2.882133in}{1.417008in}}%
\pgfpathlineto{\pgfqpoint{2.882133in}{1.413257in}}%
\pgfpathlineto{\pgfqpoint{2.879311in}{1.409880in}}%
\pgfpathlineto{\pgfqpoint{2.878998in}{1.409505in}}%
\pgfpathlineto{\pgfqpoint{2.878998in}{1.405754in}}%
\pgfpathlineto{\pgfqpoint{2.878998in}{1.402002in}}%
\pgfpathlineto{\pgfqpoint{2.876177in}{1.398626in}}%
\pgfpathlineto{\pgfqpoint{2.875863in}{1.398251in}}%
\pgfpathlineto{\pgfqpoint{2.875863in}{1.394499in}}%
\pgfpathlineto{\pgfqpoint{2.875863in}{1.390747in}}%
\pgfpathlineto{\pgfqpoint{2.873042in}{1.387371in}}%
\pgfpathlineto{\pgfqpoint{2.872728in}{1.386996in}}%
\pgfpathlineto{\pgfqpoint{2.872728in}{1.383244in}}%
\pgfpathlineto{\pgfqpoint{2.872728in}{1.379493in}}%
\pgfpathlineto{\pgfqpoint{2.869907in}{1.376116in}}%
\pgfpathlineto{\pgfqpoint{2.869594in}{1.375741in}}%
\pgfpathlineto{\pgfqpoint{2.869594in}{1.371990in}}%
\pgfpathlineto{\pgfqpoint{2.869594in}{1.368238in}}%
\pgfpathlineto{\pgfqpoint{2.866772in}{1.364862in}}%
\pgfpathlineto{\pgfqpoint{2.866459in}{1.364487in}}%
\pgfpathlineto{\pgfqpoint{2.866459in}{1.360735in}}%
\pgfpathlineto{\pgfqpoint{2.866459in}{1.356984in}}%
\pgfpathlineto{\pgfqpoint{2.866459in}{1.353232in}}%
\pgfpathlineto{\pgfqpoint{2.863638in}{1.349856in}}%
\pgfpathlineto{\pgfqpoint{2.863324in}{1.349481in}}%
\pgfpathlineto{\pgfqpoint{2.863324in}{1.345729in}}%
\pgfpathlineto{\pgfqpoint{2.863324in}{1.341977in}}%
\pgfpathlineto{\pgfqpoint{2.860503in}{1.338601in}}%
\pgfpathlineto{\pgfqpoint{2.860189in}{1.338226in}}%
\pgfpathlineto{\pgfqpoint{2.860189in}{1.334474in}}%
\pgfpathlineto{\pgfqpoint{2.860189in}{1.330723in}}%
\pgfpathlineto{\pgfqpoint{2.857368in}{1.327347in}}%
\pgfpathlineto{\pgfqpoint{2.857055in}{1.326971in}}%
\pgfpathlineto{\pgfqpoint{2.857055in}{1.323220in}}%
\pgfpathlineto{\pgfqpoint{2.857055in}{1.319468in}}%
\pgfpathlineto{\pgfqpoint{2.854233in}{1.316092in}}%
\pgfpathlineto{\pgfqpoint{2.853920in}{1.315717in}}%
\pgfpathlineto{\pgfqpoint{2.853920in}{1.311965in}}%
\pgfpathlineto{\pgfqpoint{2.853920in}{1.308214in}}%
\pgfpathlineto{\pgfqpoint{2.851099in}{1.304837in}}%
\pgfpathlineto{\pgfqpoint{2.850785in}{1.304462in}}%
\pgfpathlineto{\pgfqpoint{2.850785in}{1.300711in}}%
\pgfpathlineto{\pgfqpoint{2.850785in}{1.296959in}}%
\pgfpathlineto{\pgfqpoint{2.847964in}{1.293583in}}%
\pgfpathlineto{\pgfqpoint{2.847650in}{1.293208in}}%
\pgfpathlineto{\pgfqpoint{2.847650in}{1.289456in}}%
\pgfpathlineto{\pgfqpoint{2.847650in}{1.285704in}}%
\pgfpathlineto{\pgfqpoint{2.844829in}{1.282328in}}%
\pgfpathlineto{\pgfqpoint{2.844516in}{1.281953in}}%
\pgfpathlineto{\pgfqpoint{2.844516in}{1.278201in}}%
\pgfpathlineto{\pgfqpoint{2.844516in}{1.274450in}}%
\pgfpathlineto{\pgfqpoint{2.841694in}{1.271073in}}%
\pgfpathlineto{\pgfqpoint{2.841381in}{1.270698in}}%
\pgfpathlineto{\pgfqpoint{2.841381in}{1.266947in}}%
\pgfpathlineto{\pgfqpoint{2.841381in}{1.263195in}}%
\pgfpathlineto{\pgfqpoint{2.838560in}{1.259819in}}%
\pgfpathlineto{\pgfqpoint{2.838246in}{1.259444in}}%
\pgfpathlineto{\pgfqpoint{2.838246in}{1.255692in}}%
\pgfpathlineto{\pgfqpoint{2.838246in}{1.251941in}}%
\pgfpathlineto{\pgfqpoint{2.838246in}{1.248189in}}%
\pgfpathlineto{\pgfqpoint{2.835425in}{1.244813in}}%
\pgfpathlineto{\pgfqpoint{2.835111in}{1.244438in}}%
\pgfpathlineto{\pgfqpoint{2.835111in}{1.240686in}}%
\pgfpathlineto{\pgfqpoint{2.835111in}{1.236935in}}%
\pgfpathlineto{\pgfqpoint{2.832290in}{1.233558in}}%
\pgfpathlineto{\pgfqpoint{2.831977in}{1.233183in}}%
\pgfpathlineto{\pgfqpoint{2.831977in}{1.229431in}}%
\pgfpathlineto{\pgfqpoint{2.831977in}{1.225680in}}%
\pgfpathlineto{\pgfqpoint{2.829155in}{1.222304in}}%
\pgfpathlineto{\pgfqpoint{2.828842in}{1.221928in}}%
\pgfpathlineto{\pgfqpoint{2.828842in}{1.218177in}}%
\pgfpathlineto{\pgfqpoint{2.828842in}{1.214425in}}%
\pgfpathlineto{\pgfqpoint{2.826021in}{1.211049in}}%
\pgfpathlineto{\pgfqpoint{2.825707in}{1.210674in}}%
\pgfpathlineto{\pgfqpoint{2.825707in}{1.206922in}}%
\pgfpathlineto{\pgfqpoint{2.825707in}{1.203171in}}%
\pgfpathlineto{\pgfqpoint{2.822886in}{1.199794in}}%
\pgfpathlineto{\pgfqpoint{2.822573in}{1.199419in}}%
\pgfpathlineto{\pgfqpoint{2.822573in}{1.195668in}}%
\pgfpathlineto{\pgfqpoint{2.822573in}{1.191916in}}%
\pgfpathlineto{\pgfqpoint{2.819751in}{1.188540in}}%
\pgfpathlineto{\pgfqpoint{2.819438in}{1.188165in}}%
\pgfpathlineto{\pgfqpoint{2.819438in}{1.184413in}}%
\pgfpathlineto{\pgfqpoint{2.819438in}{1.180662in}}%
\pgfpathlineto{\pgfqpoint{2.816617in}{1.177285in}}%
\pgfpathlineto{\pgfqpoint{2.816303in}{1.176910in}}%
\pgfpathlineto{\pgfqpoint{2.816303in}{1.173158in}}%
\pgfpathlineto{\pgfqpoint{2.813482in}{1.169782in}}%
\pgfpathlineto{\pgfqpoint{2.810347in}{1.169782in}}%
\pgfpathlineto{\pgfqpoint{2.807212in}{1.169782in}}%
\pgfpathlineto{\pgfqpoint{2.806899in}{1.169407in}}%
\pgfpathlineto{\pgfqpoint{2.804078in}{1.166031in}}%
\pgfpathlineto{\pgfqpoint{2.800943in}{1.166031in}}%
\pgfpathlineto{\pgfqpoint{2.797808in}{1.166031in}}%
\pgfpathlineto{\pgfqpoint{2.797495in}{1.165655in}}%
\pgfpathlineto{\pgfqpoint{2.794673in}{1.162279in}}%
\pgfpathlineto{\pgfqpoint{2.791539in}{1.162279in}}%
\pgfpathlineto{\pgfqpoint{2.791225in}{1.161904in}}%
\pgfpathlineto{\pgfqpoint{2.788404in}{1.158527in}}%
\pgfpathlineto{\pgfqpoint{2.785269in}{1.158527in}}%
\pgfpathlineto{\pgfqpoint{2.782134in}{1.158527in}}%
\pgfpathlineto{\pgfqpoint{2.781821in}{1.158152in}}%
\pgfpathlineto{\pgfqpoint{2.779000in}{1.154776in}}%
\pgfpathlineto{\pgfqpoint{2.775865in}{1.154776in}}%
\pgfpathlineto{\pgfqpoint{2.772730in}{1.154776in}}%
\pgfpathlineto{\pgfqpoint{2.772417in}{1.154401in}}%
\pgfpathlineto{\pgfqpoint{2.769595in}{1.151024in}}%
\pgfpathlineto{\pgfqpoint{2.766461in}{1.151024in}}%
\pgfpathlineto{\pgfqpoint{2.763326in}{1.151024in}}%
\pgfpathlineto{\pgfqpoint{2.763012in}{1.150649in}}%
\pgfpathlineto{\pgfqpoint{2.760191in}{1.147273in}}%
\pgfpathlineto{\pgfqpoint{2.757056in}{1.147273in}}%
\pgfpathlineto{\pgfqpoint{2.756743in}{1.146898in}}%
\pgfpathlineto{\pgfqpoint{2.753922in}{1.143521in}}%
\pgfpathlineto{\pgfqpoint{2.750787in}{1.143521in}}%
\pgfpathlineto{\pgfqpoint{2.747652in}{1.143521in}}%
\pgfpathlineto{\pgfqpoint{2.747339in}{1.143146in}}%
\pgfpathlineto{\pgfqpoint{2.744517in}{1.139770in}}%
\pgfpathlineto{\pgfqpoint{2.741383in}{1.139770in}}%
\pgfpathlineto{\pgfqpoint{2.738248in}{1.139770in}}%
\pgfpathlineto{\pgfqpoint{2.737934in}{1.139395in}}%
\pgfpathlineto{\pgfqpoint{2.735113in}{1.136018in}}%
\pgfpathlineto{\pgfqpoint{2.731978in}{1.136018in}}%
\pgfpathlineto{\pgfqpoint{2.728844in}{1.136018in}}%
\pgfpathlineto{\pgfqpoint{2.728530in}{1.135643in}}%
\pgfpathlineto{\pgfqpoint{2.725709in}{1.132267in}}%
\pgfpathlineto{\pgfqpoint{2.722574in}{1.132267in}}%
\pgfpathlineto{\pgfqpoint{2.719439in}{1.132267in}}%
\pgfpathlineto{\pgfqpoint{2.719126in}{1.131892in}}%
\pgfpathlineto{\pgfqpoint{2.716305in}{1.128515in}}%
\pgfpathlineto{\pgfqpoint{2.713170in}{1.128515in}}%
\pgfpathlineto{\pgfqpoint{2.712857in}{1.128140in}}%
\pgfpathlineto{\pgfqpoint{2.710035in}{1.124764in}}%
\pgfpathlineto{\pgfqpoint{2.706901in}{1.124764in}}%
\pgfpathlineto{\pgfqpoint{2.703766in}{1.124764in}}%
\pgfpathlineto{\pgfqpoint{2.703452in}{1.124388in}}%
\pgfpathlineto{\pgfqpoint{2.700631in}{1.121012in}}%
\pgfpathlineto{\pgfqpoint{2.697496in}{1.121012in}}%
\pgfpathlineto{\pgfqpoint{2.694362in}{1.121012in}}%
\pgfpathlineto{\pgfqpoint{2.694048in}{1.120637in}}%
\pgfpathlineto{\pgfqpoint{2.691227in}{1.117261in}}%
\pgfpathlineto{\pgfqpoint{2.688092in}{1.117261in}}%
\pgfpathlineto{\pgfqpoint{2.684957in}{1.117261in}}%
\pgfpathlineto{\pgfqpoint{2.684644in}{1.116885in}}%
\pgfpathlineto{\pgfqpoint{2.681823in}{1.113509in}}%
\pgfpathlineto{\pgfqpoint{2.678688in}{1.113509in}}%
\pgfpathlineto{\pgfqpoint{2.678374in}{1.113134in}}%
\pgfpathlineto{\pgfqpoint{2.675553in}{1.109758in}}%
\pgfpathlineto{\pgfqpoint{2.672418in}{1.109758in}}%
\pgfpathlineto{\pgfqpoint{2.669284in}{1.109758in}}%
\pgfpathlineto{\pgfqpoint{2.668970in}{1.109382in}}%
\pgfpathlineto{\pgfqpoint{2.666149in}{1.106006in}}%
\pgfpathlineto{\pgfqpoint{2.663014in}{1.106006in}}%
\pgfpathlineto{\pgfqpoint{2.659879in}{1.106006in}}%
\pgfpathlineto{\pgfqpoint{2.659566in}{1.105631in}}%
\pgfpathlineto{\pgfqpoint{2.656745in}{1.102254in}}%
\pgfpathlineto{\pgfqpoint{2.653610in}{1.102254in}}%
\pgfpathlineto{\pgfqpoint{2.650475in}{1.102254in}}%
\pgfpathlineto{\pgfqpoint{2.650162in}{1.101879in}}%
\pgfpathlineto{\pgfqpoint{2.647340in}{1.098503in}}%
\pgfpathlineto{\pgfqpoint{2.644206in}{1.098503in}}%
\pgfpathlineto{\pgfqpoint{2.643892in}{1.098128in}}%
\pgfpathlineto{\pgfqpoint{2.641071in}{1.094751in}}%
\pgfpathlineto{\pgfqpoint{2.637936in}{1.094751in}}%
\pgfpathlineto{\pgfqpoint{2.634801in}{1.094751in}}%
\pgfpathlineto{\pgfqpoint{2.634488in}{1.094376in}}%
\pgfpathlineto{\pgfqpoint{2.631667in}{1.091000in}}%
\pgfpathlineto{\pgfqpoint{2.628532in}{1.091000in}}%
\pgfpathlineto{\pgfqpoint{2.625397in}{1.091000in}}%
\pgfpathlineto{\pgfqpoint{2.625084in}{1.090625in}}%
\pgfpathlineto{\pgfqpoint{2.622262in}{1.087248in}}%
\pgfpathlineto{\pgfqpoint{2.619128in}{1.087248in}}%
\pgfpathlineto{\pgfqpoint{2.615993in}{1.087248in}}%
\pgfpathlineto{\pgfqpoint{2.615680in}{1.086873in}}%
\pgfpathlineto{\pgfqpoint{2.612858in}{1.083497in}}%
\pgfpathlineto{\pgfqpoint{2.609724in}{1.083497in}}%
\pgfpathlineto{\pgfqpoint{2.606589in}{1.083497in}}%
\pgfpathlineto{\pgfqpoint{2.606275in}{1.083122in}}%
\pgfpathlineto{\pgfqpoint{2.603454in}{1.079745in}}%
\pgfpathlineto{\pgfqpoint{2.600319in}{1.079745in}}%
\pgfpathlineto{\pgfqpoint{2.600006in}{1.079370in}}%
\pgfpathlineto{\pgfqpoint{2.597185in}{1.075994in}}%
\pgfpathlineto{\pgfqpoint{2.594050in}{1.075994in}}%
\pgfpathlineto{\pgfqpoint{2.590915in}{1.075994in}}%
\pgfpathlineto{\pgfqpoint{2.590602in}{1.075619in}}%
\pgfpathlineto{\pgfqpoint{2.587780in}{1.072242in}}%
\pgfpathlineto{\pgfqpoint{2.584646in}{1.072242in}}%
\pgfpathlineto{\pgfqpoint{2.581511in}{1.072242in}}%
\pgfpathlineto{\pgfqpoint{2.581197in}{1.071867in}}%
\pgfpathlineto{\pgfqpoint{2.578376in}{1.068491in}}%
\pgfpathlineto{\pgfqpoint{2.575241in}{1.068491in}}%
\pgfpathlineto{\pgfqpoint{2.572107in}{1.068491in}}%
\pgfpathlineto{\pgfqpoint{2.571793in}{1.068115in}}%
\pgfpathlineto{\pgfqpoint{2.568972in}{1.064739in}}%
\pgfpathlineto{\pgfqpoint{2.565837in}{1.064739in}}%
\pgfpathlineto{\pgfqpoint{2.565524in}{1.064364in}}%
\pgfpathlineto{\pgfqpoint{2.562702in}{1.060988in}}%
\pgfpathlineto{\pgfqpoint{2.559568in}{1.060988in}}%
\pgfpathlineto{\pgfqpoint{2.556433in}{1.060988in}}%
\pgfpathlineto{\pgfqpoint{2.556119in}{1.060612in}}%
\pgfpathlineto{\pgfqpoint{2.553298in}{1.057236in}}%
\pgfpathlineto{\pgfqpoint{2.550163in}{1.057236in}}%
\pgfpathlineto{\pgfqpoint{2.547029in}{1.057236in}}%
\pgfpathlineto{\pgfqpoint{2.546715in}{1.056861in}}%
\pgfpathlineto{\pgfqpoint{2.543894in}{1.053484in}}%
\pgfpathlineto{\pgfqpoint{2.540759in}{1.053484in}}%
\pgfpathlineto{\pgfqpoint{2.537624in}{1.053484in}}%
\pgfpathlineto{\pgfqpoint{2.537311in}{1.053109in}}%
\pgfpathlineto{\pgfqpoint{2.534490in}{1.049733in}}%
\pgfpathlineto{\pgfqpoint{2.531355in}{1.049733in}}%
\pgfpathlineto{\pgfqpoint{2.531041in}{1.049358in}}%
\pgfpathlineto{\pgfqpoint{2.528220in}{1.045981in}}%
\pgfpathlineto{\pgfqpoint{2.525085in}{1.045981in}}%
\pgfpathlineto{\pgfqpoint{2.521951in}{1.045981in}}%
\pgfpathlineto{\pgfqpoint{2.521637in}{1.045606in}}%
\pgfpathlineto{\pgfqpoint{2.518816in}{1.042230in}}%
\pgfpathlineto{\pgfqpoint{2.515681in}{1.042230in}}%
\pgfpathlineto{\pgfqpoint{2.512547in}{1.042230in}}%
\pgfpathlineto{\pgfqpoint{2.512233in}{1.041855in}}%
\pgfpathlineto{\pgfqpoint{2.509412in}{1.038478in}}%
\pgfpathlineto{\pgfqpoint{2.506277in}{1.038478in}}%
\pgfpathlineto{\pgfqpoint{2.503142in}{1.038478in}}%
\pgfpathlineto{\pgfqpoint{2.502829in}{1.038103in}}%
\pgfpathlineto{\pgfqpoint{2.500008in}{1.034727in}}%
\pgfpathlineto{\pgfqpoint{2.496873in}{1.034727in}}%
\pgfpathlineto{\pgfqpoint{2.493738in}{1.034727in}}%
\pgfpathlineto{\pgfqpoint{2.493425in}{1.034352in}}%
\pgfpathlineto{\pgfqpoint{2.490603in}{1.030975in}}%
\pgfpathlineto{\pgfqpoint{2.487469in}{1.030975in}}%
\pgfpathlineto{\pgfqpoint{2.487155in}{1.030600in}}%
\pgfpathlineto{\pgfqpoint{2.484334in}{1.027224in}}%
\pgfpathlineto{\pgfqpoint{2.481199in}{1.027224in}}%
\pgfpathlineto{\pgfqpoint{2.478064in}{1.027224in}}%
\pgfpathlineto{\pgfqpoint{2.477751in}{1.026849in}}%
\pgfpathlineto{\pgfqpoint{2.474930in}{1.023472in}}%
\pgfpathlineto{\pgfqpoint{2.471795in}{1.023472in}}%
\pgfpathlineto{\pgfqpoint{2.468660in}{1.023472in}}%
\pgfpathlineto{\pgfqpoint{2.468347in}{1.023097in}}%
\pgfpathlineto{\pgfqpoint{2.465525in}{1.019721in}}%
\pgfpathlineto{\pgfqpoint{2.462391in}{1.019721in}}%
\pgfpathlineto{\pgfqpoint{2.459256in}{1.019721in}}%
\pgfpathlineto{\pgfqpoint{2.458942in}{1.019346in}}%
\pgfpathlineto{\pgfqpoint{2.456121in}{1.015969in}}%
\pgfpathlineto{\pgfqpoint{2.452986in}{1.015969in}}%
\pgfpathlineto{\pgfqpoint{2.452673in}{1.015594in}}%
\pgfpathlineto{\pgfqpoint{2.449852in}{1.012218in}}%
\pgfpathlineto{\pgfqpoint{2.446717in}{1.012218in}}%
\pgfpathlineto{\pgfqpoint{2.443582in}{1.012218in}}%
\pgfpathlineto{\pgfqpoint{2.443269in}{1.011842in}}%
\pgfpathlineto{\pgfqpoint{2.440447in}{1.008466in}}%
\pgfpathlineto{\pgfqpoint{2.437313in}{1.008466in}}%
\pgfpathlineto{\pgfqpoint{2.434178in}{1.008466in}}%
\pgfpathlineto{\pgfqpoint{2.433864in}{1.008091in}}%
\pgfpathlineto{\pgfqpoint{2.431043in}{1.004715in}}%
\pgfpathlineto{\pgfqpoint{2.427908in}{1.004715in}}%
\pgfpathlineto{\pgfqpoint{2.424774in}{1.004715in}}%
\pgfpathlineto{\pgfqpoint{2.424460in}{1.004339in}}%
\pgfpathlineto{\pgfqpoint{2.421639in}{1.000963in}}%
\pgfpathlineto{\pgfqpoint{2.418504in}{1.000963in}}%
\pgfpathlineto{\pgfqpoint{2.418191in}{1.000588in}}%
\pgfpathlineto{\pgfqpoint{2.415369in}{0.997211in}}%
\pgfpathlineto{\pgfqpoint{2.412235in}{0.997211in}}%
\pgfpathlineto{\pgfqpoint{2.409100in}{0.997211in}}%
\pgfpathlineto{\pgfqpoint{2.408787in}{0.996836in}}%
\pgfpathlineto{\pgfqpoint{2.405965in}{0.993460in}}%
\pgfpathlineto{\pgfqpoint{2.402831in}{0.993460in}}%
\pgfpathlineto{\pgfqpoint{2.399696in}{0.993460in}}%
\pgfpathlineto{\pgfqpoint{2.399382in}{0.993085in}}%
\pgfpathlineto{\pgfqpoint{2.396561in}{0.989708in}}%
\pgfpathlineto{\pgfqpoint{2.393426in}{0.989708in}}%
\pgfpathlineto{\pgfqpoint{2.390292in}{0.989708in}}%
\pgfpathlineto{\pgfqpoint{2.389978in}{0.989333in}}%
\pgfpathlineto{\pgfqpoint{2.387157in}{0.985957in}}%
\pgfpathlineto{\pgfqpoint{2.384022in}{0.985957in}}%
\pgfpathlineto{\pgfqpoint{2.380887in}{0.985957in}}%
\pgfpathlineto{\pgfqpoint{2.380574in}{0.985582in}}%
\pgfpathlineto{\pgfqpoint{2.377753in}{0.982205in}}%
\pgfpathlineto{\pgfqpoint{2.374618in}{0.982205in}}%
\pgfpathlineto{\pgfqpoint{2.374304in}{0.981830in}}%
\pgfpathlineto{\pgfqpoint{2.371483in}{0.978454in}}%
\pgfpathlineto{\pgfqpoint{2.368348in}{0.978454in}}%
\pgfpathlineto{\pgfqpoint{2.365214in}{0.978454in}}%
\pgfpathlineto{\pgfqpoint{2.364900in}{0.978079in}}%
\pgfpathlineto{\pgfqpoint{2.362079in}{0.974702in}}%
\pgfpathlineto{\pgfqpoint{2.358944in}{0.974702in}}%
\pgfpathlineto{\pgfqpoint{2.355809in}{0.974702in}}%
\pgfpathlineto{\pgfqpoint{2.355496in}{0.974327in}}%
\pgfpathlineto{\pgfqpoint{2.352675in}{0.970951in}}%
\pgfpathlineto{\pgfqpoint{2.349540in}{0.970951in}}%
\pgfpathlineto{\pgfqpoint{2.346405in}{0.970951in}}%
\pgfpathlineto{\pgfqpoint{2.346092in}{0.970576in}}%
\pgfpathlineto{\pgfqpoint{2.343270in}{0.967199in}}%
\pgfpathlineto{\pgfqpoint{2.340136in}{0.967199in}}%
\pgfpathlineto{\pgfqpoint{2.339822in}{0.966824in}}%
\pgfpathlineto{\pgfqpoint{2.337001in}{0.963448in}}%
\pgfpathlineto{\pgfqpoint{2.333866in}{0.963448in}}%
\pgfpathlineto{\pgfqpoint{2.330731in}{0.963448in}}%
\pgfpathlineto{\pgfqpoint{2.330418in}{0.963073in}}%
\pgfpathlineto{\pgfqpoint{2.327597in}{0.959696in}}%
\pgfpathlineto{\pgfqpoint{2.324462in}{0.959696in}}%
\pgfpathlineto{\pgfqpoint{2.321327in}{0.959696in}}%
\pgfpathlineto{\pgfqpoint{2.321014in}{0.959321in}}%
\pgfpathlineto{\pgfqpoint{2.318192in}{0.955945in}}%
\pgfpathlineto{\pgfqpoint{2.315058in}{0.955945in}}%
\pgfpathlineto{\pgfqpoint{2.311923in}{0.955945in}}%
\pgfpathlineto{\pgfqpoint{2.311610in}{0.955569in}}%
\pgfpathlineto{\pgfqpoint{2.308788in}{0.952193in}}%
\pgfpathlineto{\pgfqpoint{2.305654in}{0.952193in}}%
\pgfpathlineto{\pgfqpoint{2.305340in}{0.951818in}}%
\pgfpathlineto{\pgfqpoint{2.302519in}{0.948442in}}%
\pgfpathlineto{\pgfqpoint{2.299384in}{0.948442in}}%
\pgfpathlineto{\pgfqpoint{2.296249in}{0.948442in}}%
\pgfpathlineto{\pgfqpoint{2.295936in}{0.948066in}}%
\pgfpathlineto{\pgfqpoint{2.293115in}{0.944690in}}%
\pgfpathlineto{\pgfqpoint{2.289980in}{0.944690in}}%
\pgfpathlineto{\pgfqpoint{2.286845in}{0.944690in}}%
\pgfpathlineto{\pgfqpoint{2.286532in}{0.944315in}}%
\pgfpathlineto{\pgfqpoint{2.283710in}{0.940938in}}%
\pgfpathlineto{\pgfqpoint{2.280576in}{0.940938in}}%
\pgfpathlineto{\pgfqpoint{2.277441in}{0.940938in}}%
\pgfpathlineto{\pgfqpoint{2.277127in}{0.940563in}}%
\pgfpathlineto{\pgfqpoint{2.274306in}{0.937187in}}%
\pgfpathlineto{\pgfqpoint{2.271171in}{0.937187in}}%
\pgfpathlineto{\pgfqpoint{2.270858in}{0.936812in}}%
\pgfpathlineto{\pgfqpoint{2.268037in}{0.933435in}}%
\pgfpathlineto{\pgfqpoint{2.264902in}{0.933435in}}%
\pgfpathlineto{\pgfqpoint{2.261767in}{0.933435in}}%
\pgfpathlineto{\pgfqpoint{2.261454in}{0.933060in}}%
\pgfpathlineto{\pgfqpoint{2.258632in}{0.929684in}}%
\pgfpathlineto{\pgfqpoint{2.255498in}{0.929684in}}%
\pgfpathlineto{\pgfqpoint{2.252363in}{0.929684in}}%
\pgfpathlineto{\pgfqpoint{2.252049in}{0.929309in}}%
\pgfpathlineto{\pgfqpoint{2.249228in}{0.925932in}}%
\pgfpathlineto{\pgfqpoint{2.246093in}{0.925932in}}%
\pgfpathlineto{\pgfqpoint{2.242959in}{0.925932in}}%
\pgfpathlineto{\pgfqpoint{2.242645in}{0.925557in}}%
\pgfpathlineto{\pgfqpoint{2.239824in}{0.922181in}}%
\pgfpathlineto{\pgfqpoint{2.236689in}{0.922181in}}%
\pgfpathlineto{\pgfqpoint{2.233554in}{0.922181in}}%
\pgfpathlineto{\pgfqpoint{2.233241in}{0.921806in}}%
\pgfpathlineto{\pgfqpoint{2.230420in}{0.918429in}}%
\pgfpathlineto{\pgfqpoint{2.227285in}{0.918429in}}%
\pgfpathlineto{\pgfqpoint{2.226971in}{0.918054in}}%
\pgfpathlineto{\pgfqpoint{2.224150in}{0.914678in}}%
\pgfpathlineto{\pgfqpoint{2.221015in}{0.914678in}}%
\pgfpathlineto{\pgfqpoint{2.217881in}{0.914678in}}%
\pgfpathlineto{\pgfqpoint{2.217567in}{0.914303in}}%
\pgfpathlineto{\pgfqpoint{2.214746in}{0.910926in}}%
\pgfpathlineto{\pgfqpoint{2.211611in}{0.910926in}}%
\pgfpathlineto{\pgfqpoint{2.208477in}{0.910926in}}%
\pgfpathlineto{\pgfqpoint{2.208163in}{0.910551in}}%
\pgfpathlineto{\pgfqpoint{2.205342in}{0.907175in}}%
\pgfpathlineto{\pgfqpoint{2.202207in}{0.907175in}}%
\pgfpathlineto{\pgfqpoint{2.199072in}{0.907175in}}%
\pgfpathlineto{\pgfqpoint{2.198759in}{0.906799in}}%
\pgfpathlineto{\pgfqpoint{2.195938in}{0.903423in}}%
\pgfpathlineto{\pgfqpoint{2.192803in}{0.903423in}}%
\pgfpathlineto{\pgfqpoint{2.192489in}{0.903048in}}%
\pgfpathlineto{\pgfqpoint{2.189668in}{0.899672in}}%
\pgfpathlineto{\pgfqpoint{2.186533in}{0.899672in}}%
\pgfpathlineto{\pgfqpoint{2.183399in}{0.899672in}}%
\pgfpathlineto{\pgfqpoint{2.183085in}{0.899296in}}%
\pgfpathlineto{\pgfqpoint{2.180264in}{0.895920in}}%
\pgfpathlineto{\pgfqpoint{2.177129in}{0.895920in}}%
\pgfpathlineto{\pgfqpoint{2.173994in}{0.895920in}}%
\pgfpathclose%
\pgfusepath{fill}%
\end{pgfscope}%
\begin{pgfscope}%
\pgfpathrectangle{\pgfqpoint{0.888750in}{0.419100in}}{\pgfqpoint{2.504659in}{2.933700in}} %
\pgfusepath{clip}%
\pgfsetbuttcap%
\pgfsetroundjoin%
\definecolor{currentfill}{rgb}{1.000000,0.549020,0.000000}%
\pgfsetfillcolor{currentfill}%
\pgfsetlinewidth{1.003750pt}%
\definecolor{currentstroke}{rgb}{1.000000,0.549020,0.000000}%
\pgfsetstrokecolor{currentstroke}%
\pgfsetdash{}{0pt}%
\pgfpathmoveto{\pgfqpoint{2.340036in}{1.940535in}}%
\pgfpathcurveto{\pgfqpoint{2.351086in}{1.940535in}}{\pgfqpoint{2.361685in}{1.944926in}}{\pgfqpoint{2.369499in}{1.952739in}}%
\pgfpathcurveto{\pgfqpoint{2.377312in}{1.960553in}}{\pgfqpoint{2.381702in}{1.971152in}}{\pgfqpoint{2.381702in}{1.982202in}}%
\pgfpathcurveto{\pgfqpoint{2.381702in}{1.993252in}}{\pgfqpoint{2.377312in}{2.003851in}}{\pgfqpoint{2.369499in}{2.011665in}}%
\pgfpathcurveto{\pgfqpoint{2.361685in}{2.019478in}}{\pgfqpoint{2.351086in}{2.023869in}}{\pgfqpoint{2.340036in}{2.023869in}}%
\pgfpathcurveto{\pgfqpoint{2.328986in}{2.023869in}}{\pgfqpoint{2.318387in}{2.019478in}}{\pgfqpoint{2.310573in}{2.011665in}}%
\pgfpathcurveto{\pgfqpoint{2.302759in}{2.003851in}}{\pgfqpoint{2.298369in}{1.993252in}}{\pgfqpoint{2.298369in}{1.982202in}}%
\pgfpathcurveto{\pgfqpoint{2.298369in}{1.971152in}}{\pgfqpoint{2.302759in}{1.960553in}}{\pgfqpoint{2.310573in}{1.952739in}}%
\pgfpathcurveto{\pgfqpoint{2.318387in}{1.944926in}}{\pgfqpoint{2.328986in}{1.940535in}}{\pgfqpoint{2.340036in}{1.940535in}}%
\pgfpathclose%
\pgfusepath{stroke,fill}%
\end{pgfscope}%
\begin{pgfscope}%
\pgfpathrectangle{\pgfqpoint{0.888750in}{0.419100in}}{\pgfqpoint{2.504659in}{2.933700in}} %
\pgfusepath{clip}%
\pgfsetbuttcap%
\pgfsetroundjoin%
\definecolor{currentfill}{rgb}{1.000000,0.549020,0.000000}%
\pgfsetfillcolor{currentfill}%
\pgfsetlinewidth{1.003750pt}%
\definecolor{currentstroke}{rgb}{1.000000,0.549020,0.000000}%
\pgfsetstrokecolor{currentstroke}%
\pgfsetdash{}{0pt}%
\pgfpathmoveto{\pgfqpoint{1.922833in}{1.922929in}}%
\pgfpathcurveto{\pgfqpoint{1.933883in}{1.922929in}}{\pgfqpoint{1.944482in}{1.927319in}}{\pgfqpoint{1.952296in}{1.935133in}}%
\pgfpathcurveto{\pgfqpoint{1.960110in}{1.942947in}}{\pgfqpoint{1.964500in}{1.953546in}}{\pgfqpoint{1.964500in}{1.964596in}}%
\pgfpathcurveto{\pgfqpoint{1.964500in}{1.975646in}}{\pgfqpoint{1.960110in}{1.986245in}}{\pgfqpoint{1.952296in}{1.994059in}}%
\pgfpathcurveto{\pgfqpoint{1.944482in}{2.001872in}}{\pgfqpoint{1.933883in}{2.006262in}}{\pgfqpoint{1.922833in}{2.006262in}}%
\pgfpathcurveto{\pgfqpoint{1.911783in}{2.006262in}}{\pgfqpoint{1.901184in}{2.001872in}}{\pgfqpoint{1.893370in}{1.994059in}}%
\pgfpathcurveto{\pgfqpoint{1.885557in}{1.986245in}}{\pgfqpoint{1.881167in}{1.975646in}}{\pgfqpoint{1.881167in}{1.964596in}}%
\pgfpathcurveto{\pgfqpoint{1.881167in}{1.953546in}}{\pgfqpoint{1.885557in}{1.942947in}}{\pgfqpoint{1.893370in}{1.935133in}}%
\pgfpathcurveto{\pgfqpoint{1.901184in}{1.927319in}}{\pgfqpoint{1.911783in}{1.922929in}}{\pgfqpoint{1.922833in}{1.922929in}}%
\pgfpathclose%
\pgfusepath{stroke,fill}%
\end{pgfscope}%
\begin{pgfscope}%
\pgfpathrectangle{\pgfqpoint{0.888750in}{0.419100in}}{\pgfqpoint{2.504659in}{2.933700in}} %
\pgfusepath{clip}%
\pgfsetbuttcap%
\pgfsetroundjoin%
\definecolor{currentfill}{rgb}{1.000000,0.549020,0.000000}%
\pgfsetfillcolor{currentfill}%
\pgfsetlinewidth{1.003750pt}%
\definecolor{currentstroke}{rgb}{1.000000,0.549020,0.000000}%
\pgfsetstrokecolor{currentstroke}%
\pgfsetdash{}{0pt}%
\pgfpathmoveto{\pgfqpoint{2.054464in}{1.230951in}}%
\pgfpathcurveto{\pgfqpoint{2.065514in}{1.230951in}}{\pgfqpoint{2.076113in}{1.235341in}}{\pgfqpoint{2.083926in}{1.243155in}}%
\pgfpathcurveto{\pgfqpoint{2.091740in}{1.250968in}}{\pgfqpoint{2.096130in}{1.261568in}}{\pgfqpoint{2.096130in}{1.272618in}}%
\pgfpathcurveto{\pgfqpoint{2.096130in}{1.283668in}}{\pgfqpoint{2.091740in}{1.294267in}}{\pgfqpoint{2.083926in}{1.302080in}}%
\pgfpathcurveto{\pgfqpoint{2.076113in}{1.309894in}}{\pgfqpoint{2.065514in}{1.314284in}}{\pgfqpoint{2.054464in}{1.314284in}}%
\pgfpathcurveto{\pgfqpoint{2.043413in}{1.314284in}}{\pgfqpoint{2.032814in}{1.309894in}}{\pgfqpoint{2.025001in}{1.302080in}}%
\pgfpathcurveto{\pgfqpoint{2.017187in}{1.294267in}}{\pgfqpoint{2.012797in}{1.283668in}}{\pgfqpoint{2.012797in}{1.272618in}}%
\pgfpathcurveto{\pgfqpoint{2.012797in}{1.261568in}}{\pgfqpoint{2.017187in}{1.250968in}}{\pgfqpoint{2.025001in}{1.243155in}}%
\pgfpathcurveto{\pgfqpoint{2.032814in}{1.235341in}}{\pgfqpoint{2.043413in}{1.230951in}}{\pgfqpoint{2.054464in}{1.230951in}}%
\pgfpathclose%
\pgfusepath{stroke,fill}%
\end{pgfscope}%
\begin{pgfscope}%
\pgfpathrectangle{\pgfqpoint{0.888750in}{0.419100in}}{\pgfqpoint{2.504659in}{2.933700in}} %
\pgfusepath{clip}%
\pgfsetbuttcap%
\pgfsetroundjoin%
\definecolor{currentfill}{rgb}{1.000000,0.549020,0.000000}%
\pgfsetfillcolor{currentfill}%
\pgfsetlinewidth{1.003750pt}%
\definecolor{currentstroke}{rgb}{1.000000,0.549020,0.000000}%
\pgfsetstrokecolor{currentstroke}%
\pgfsetdash{}{0pt}%
\pgfpathmoveto{\pgfqpoint{1.994492in}{1.959016in}}%
\pgfpathcurveto{\pgfqpoint{2.005542in}{1.959016in}}{\pgfqpoint{2.016141in}{1.963407in}}{\pgfqpoint{2.023955in}{1.971220in}}%
\pgfpathcurveto{\pgfqpoint{2.031768in}{1.979034in}}{\pgfqpoint{2.036159in}{1.989633in}}{\pgfqpoint{2.036159in}{2.000683in}}%
\pgfpathcurveto{\pgfqpoint{2.036159in}{2.011733in}}{\pgfqpoint{2.031768in}{2.022332in}}{\pgfqpoint{2.023955in}{2.030146in}}%
\pgfpathcurveto{\pgfqpoint{2.016141in}{2.037959in}}{\pgfqpoint{2.005542in}{2.042350in}}{\pgfqpoint{1.994492in}{2.042350in}}%
\pgfpathcurveto{\pgfqpoint{1.983442in}{2.042350in}}{\pgfqpoint{1.972843in}{2.037959in}}{\pgfqpoint{1.965029in}{2.030146in}}%
\pgfpathcurveto{\pgfqpoint{1.957215in}{2.022332in}}{\pgfqpoint{1.952825in}{2.011733in}}{\pgfqpoint{1.952825in}{2.000683in}}%
\pgfpathcurveto{\pgfqpoint{1.952825in}{1.989633in}}{\pgfqpoint{1.957215in}{1.979034in}}{\pgfqpoint{1.965029in}{1.971220in}}%
\pgfpathcurveto{\pgfqpoint{1.972843in}{1.963407in}}{\pgfqpoint{1.983442in}{1.959016in}}{\pgfqpoint{1.994492in}{1.959016in}}%
\pgfpathclose%
\pgfusepath{stroke,fill}%
\end{pgfscope}%
\begin{pgfscope}%
\pgfpathrectangle{\pgfqpoint{0.888750in}{0.419100in}}{\pgfqpoint{2.504659in}{2.933700in}} %
\pgfusepath{clip}%
\pgfsetbuttcap%
\pgfsetroundjoin%
\definecolor{currentfill}{rgb}{1.000000,0.549020,0.000000}%
\pgfsetfillcolor{currentfill}%
\pgfsetlinewidth{1.003750pt}%
\definecolor{currentstroke}{rgb}{1.000000,0.549020,0.000000}%
\pgfsetstrokecolor{currentstroke}%
\pgfsetdash{}{0pt}%
\pgfpathmoveto{\pgfqpoint{1.984072in}{1.677513in}}%
\pgfpathcurveto{\pgfqpoint{1.995122in}{1.677513in}}{\pgfqpoint{2.005721in}{1.681904in}}{\pgfqpoint{2.013535in}{1.689717in}}%
\pgfpathcurveto{\pgfqpoint{2.021348in}{1.697531in}}{\pgfqpoint{2.025738in}{1.708130in}}{\pgfqpoint{2.025738in}{1.719180in}}%
\pgfpathcurveto{\pgfqpoint{2.025738in}{1.730230in}}{\pgfqpoint{2.021348in}{1.740829in}}{\pgfqpoint{2.013535in}{1.748643in}}%
\pgfpathcurveto{\pgfqpoint{2.005721in}{1.756456in}}{\pgfqpoint{1.995122in}{1.760847in}}{\pgfqpoint{1.984072in}{1.760847in}}%
\pgfpathcurveto{\pgfqpoint{1.973022in}{1.760847in}}{\pgfqpoint{1.962423in}{1.756456in}}{\pgfqpoint{1.954609in}{1.748643in}}%
\pgfpathcurveto{\pgfqpoint{1.946795in}{1.740829in}}{\pgfqpoint{1.942405in}{1.730230in}}{\pgfqpoint{1.942405in}{1.719180in}}%
\pgfpathcurveto{\pgfqpoint{1.942405in}{1.708130in}}{\pgfqpoint{1.946795in}{1.697531in}}{\pgfqpoint{1.954609in}{1.689717in}}%
\pgfpathcurveto{\pgfqpoint{1.962423in}{1.681904in}}{\pgfqpoint{1.973022in}{1.677513in}}{\pgfqpoint{1.984072in}{1.677513in}}%
\pgfpathclose%
\pgfusepath{stroke,fill}%
\end{pgfscope}%
\begin{pgfscope}%
\pgfpathrectangle{\pgfqpoint{0.888750in}{0.419100in}}{\pgfqpoint{2.504659in}{2.933700in}} %
\pgfusepath{clip}%
\pgfsetbuttcap%
\pgfsetroundjoin%
\definecolor{currentfill}{rgb}{1.000000,0.549020,0.000000}%
\pgfsetfillcolor{currentfill}%
\pgfsetlinewidth{1.003750pt}%
\definecolor{currentstroke}{rgb}{1.000000,0.549020,0.000000}%
\pgfsetstrokecolor{currentstroke}%
\pgfsetdash{}{0pt}%
\pgfpathmoveto{\pgfqpoint{2.617877in}{2.523505in}}%
\pgfpathcurveto{\pgfqpoint{2.628928in}{2.523505in}}{\pgfqpoint{2.639527in}{2.527895in}}{\pgfqpoint{2.647340in}{2.535709in}}%
\pgfpathcurveto{\pgfqpoint{2.655154in}{2.543523in}}{\pgfqpoint{2.659544in}{2.554122in}}{\pgfqpoint{2.659544in}{2.565172in}}%
\pgfpathcurveto{\pgfqpoint{2.659544in}{2.576222in}}{\pgfqpoint{2.655154in}{2.586821in}}{\pgfqpoint{2.647340in}{2.594634in}}%
\pgfpathcurveto{\pgfqpoint{2.639527in}{2.602448in}}{\pgfqpoint{2.628928in}{2.606838in}}{\pgfqpoint{2.617877in}{2.606838in}}%
\pgfpathcurveto{\pgfqpoint{2.606827in}{2.606838in}}{\pgfqpoint{2.596228in}{2.602448in}}{\pgfqpoint{2.588415in}{2.594634in}}%
\pgfpathcurveto{\pgfqpoint{2.580601in}{2.586821in}}{\pgfqpoint{2.576211in}{2.576222in}}{\pgfqpoint{2.576211in}{2.565172in}}%
\pgfpathcurveto{\pgfqpoint{2.576211in}{2.554122in}}{\pgfqpoint{2.580601in}{2.543523in}}{\pgfqpoint{2.588415in}{2.535709in}}%
\pgfpathcurveto{\pgfqpoint{2.596228in}{2.527895in}}{\pgfqpoint{2.606827in}{2.523505in}}{\pgfqpoint{2.617877in}{2.523505in}}%
\pgfpathclose%
\pgfusepath{stroke,fill}%
\end{pgfscope}%
\begin{pgfscope}%
\pgfpathrectangle{\pgfqpoint{0.888750in}{0.419100in}}{\pgfqpoint{2.504659in}{2.933700in}} %
\pgfusepath{clip}%
\pgfsetbuttcap%
\pgfsetroundjoin%
\definecolor{currentfill}{rgb}{1.000000,0.549020,0.000000}%
\pgfsetfillcolor{currentfill}%
\pgfsetlinewidth{1.003750pt}%
\definecolor{currentstroke}{rgb}{1.000000,0.549020,0.000000}%
\pgfsetstrokecolor{currentstroke}%
\pgfsetdash{}{0pt}%
\pgfpathmoveto{\pgfqpoint{2.054470in}{1.269670in}}%
\pgfpathcurveto{\pgfqpoint{2.065520in}{1.269670in}}{\pgfqpoint{2.076119in}{1.274061in}}{\pgfqpoint{2.083932in}{1.281874in}}%
\pgfpathcurveto{\pgfqpoint{2.091746in}{1.289688in}}{\pgfqpoint{2.096136in}{1.300287in}}{\pgfqpoint{2.096136in}{1.311337in}}%
\pgfpathcurveto{\pgfqpoint{2.096136in}{1.322387in}}{\pgfqpoint{2.091746in}{1.332986in}}{\pgfqpoint{2.083932in}{1.340800in}}%
\pgfpathcurveto{\pgfqpoint{2.076119in}{1.348613in}}{\pgfqpoint{2.065520in}{1.353004in}}{\pgfqpoint{2.054470in}{1.353004in}}%
\pgfpathcurveto{\pgfqpoint{2.043419in}{1.353004in}}{\pgfqpoint{2.032820in}{1.348613in}}{\pgfqpoint{2.025007in}{1.340800in}}%
\pgfpathcurveto{\pgfqpoint{2.017193in}{1.332986in}}{\pgfqpoint{2.012803in}{1.322387in}}{\pgfqpoint{2.012803in}{1.311337in}}%
\pgfpathcurveto{\pgfqpoint{2.012803in}{1.300287in}}{\pgfqpoint{2.017193in}{1.289688in}}{\pgfqpoint{2.025007in}{1.281874in}}%
\pgfpathcurveto{\pgfqpoint{2.032820in}{1.274061in}}{\pgfqpoint{2.043419in}{1.269670in}}{\pgfqpoint{2.054470in}{1.269670in}}%
\pgfpathclose%
\pgfusepath{stroke,fill}%
\end{pgfscope}%
\begin{pgfscope}%
\pgfpathrectangle{\pgfqpoint{0.888750in}{0.419100in}}{\pgfqpoint{2.504659in}{2.933700in}} %
\pgfusepath{clip}%
\pgfsetbuttcap%
\pgfsetroundjoin%
\definecolor{currentfill}{rgb}{1.000000,0.549020,0.000000}%
\pgfsetfillcolor{currentfill}%
\pgfsetlinewidth{1.003750pt}%
\definecolor{currentstroke}{rgb}{1.000000,0.549020,0.000000}%
\pgfsetstrokecolor{currentstroke}%
\pgfsetdash{}{0pt}%
\pgfpathmoveto{\pgfqpoint{2.303705in}{1.805059in}}%
\pgfpathcurveto{\pgfqpoint{2.314756in}{1.805059in}}{\pgfqpoint{2.325355in}{1.809449in}}{\pgfqpoint{2.333168in}{1.817262in}}%
\pgfpathcurveto{\pgfqpoint{2.340982in}{1.825076in}}{\pgfqpoint{2.345372in}{1.835675in}}{\pgfqpoint{2.345372in}{1.846725in}}%
\pgfpathcurveto{\pgfqpoint{2.345372in}{1.857775in}}{\pgfqpoint{2.340982in}{1.868374in}}{\pgfqpoint{2.333168in}{1.876188in}}%
\pgfpathcurveto{\pgfqpoint{2.325355in}{1.884002in}}{\pgfqpoint{2.314756in}{1.888392in}}{\pgfqpoint{2.303705in}{1.888392in}}%
\pgfpathcurveto{\pgfqpoint{2.292655in}{1.888392in}}{\pgfqpoint{2.282056in}{1.884002in}}{\pgfqpoint{2.274243in}{1.876188in}}%
\pgfpathcurveto{\pgfqpoint{2.266429in}{1.868374in}}{\pgfqpoint{2.262039in}{1.857775in}}{\pgfqpoint{2.262039in}{1.846725in}}%
\pgfpathcurveto{\pgfqpoint{2.262039in}{1.835675in}}{\pgfqpoint{2.266429in}{1.825076in}}{\pgfqpoint{2.274243in}{1.817262in}}%
\pgfpathcurveto{\pgfqpoint{2.282056in}{1.809449in}}{\pgfqpoint{2.292655in}{1.805059in}}{\pgfqpoint{2.303705in}{1.805059in}}%
\pgfpathclose%
\pgfusepath{stroke,fill}%
\end{pgfscope}%
\begin{pgfscope}%
\pgfpathrectangle{\pgfqpoint{0.888750in}{0.419100in}}{\pgfqpoint{2.504659in}{2.933700in}} %
\pgfusepath{clip}%
\pgfsetbuttcap%
\pgfsetroundjoin%
\definecolor{currentfill}{rgb}{1.000000,0.549020,0.000000}%
\pgfsetfillcolor{currentfill}%
\pgfsetlinewidth{1.003750pt}%
\definecolor{currentstroke}{rgb}{1.000000,0.549020,0.000000}%
\pgfsetstrokecolor{currentstroke}%
\pgfsetdash{}{0pt}%
\pgfpathmoveto{\pgfqpoint{2.107759in}{1.701534in}}%
\pgfpathcurveto{\pgfqpoint{2.118809in}{1.701534in}}{\pgfqpoint{2.129408in}{1.705924in}}{\pgfqpoint{2.137222in}{1.713738in}}%
\pgfpathcurveto{\pgfqpoint{2.145036in}{1.721551in}}{\pgfqpoint{2.149426in}{1.732151in}}{\pgfqpoint{2.149426in}{1.743201in}}%
\pgfpathcurveto{\pgfqpoint{2.149426in}{1.754251in}}{\pgfqpoint{2.145036in}{1.764850in}}{\pgfqpoint{2.137222in}{1.772663in}}%
\pgfpathcurveto{\pgfqpoint{2.129408in}{1.780477in}}{\pgfqpoint{2.118809in}{1.784867in}}{\pgfqpoint{2.107759in}{1.784867in}}%
\pgfpathcurveto{\pgfqpoint{2.096709in}{1.784867in}}{\pgfqpoint{2.086110in}{1.780477in}}{\pgfqpoint{2.078296in}{1.772663in}}%
\pgfpathcurveto{\pgfqpoint{2.070483in}{1.764850in}}{\pgfqpoint{2.066093in}{1.754251in}}{\pgfqpoint{2.066093in}{1.743201in}}%
\pgfpathcurveto{\pgfqpoint{2.066093in}{1.732151in}}{\pgfqpoint{2.070483in}{1.721551in}}{\pgfqpoint{2.078296in}{1.713738in}}%
\pgfpathcurveto{\pgfqpoint{2.086110in}{1.705924in}}{\pgfqpoint{2.096709in}{1.701534in}}{\pgfqpoint{2.107759in}{1.701534in}}%
\pgfpathclose%
\pgfusepath{stroke,fill}%
\end{pgfscope}%
\begin{pgfscope}%
\pgfpathrectangle{\pgfqpoint{0.888750in}{0.419100in}}{\pgfqpoint{2.504659in}{2.933700in}} %
\pgfusepath{clip}%
\pgfsetbuttcap%
\pgfsetroundjoin%
\definecolor{currentfill}{rgb}{1.000000,0.549020,0.000000}%
\pgfsetfillcolor{currentfill}%
\pgfsetlinewidth{1.003750pt}%
\definecolor{currentstroke}{rgb}{1.000000,0.549020,0.000000}%
\pgfsetstrokecolor{currentstroke}%
\pgfsetdash{}{0pt}%
\pgfpathmoveto{\pgfqpoint{2.297212in}{1.290458in}}%
\pgfpathcurveto{\pgfqpoint{2.308262in}{1.290458in}}{\pgfqpoint{2.318861in}{1.294848in}}{\pgfqpoint{2.326675in}{1.302662in}}%
\pgfpathcurveto{\pgfqpoint{2.334488in}{1.310476in}}{\pgfqpoint{2.338879in}{1.321075in}}{\pgfqpoint{2.338879in}{1.332125in}}%
\pgfpathcurveto{\pgfqpoint{2.338879in}{1.343175in}}{\pgfqpoint{2.334488in}{1.353774in}}{\pgfqpoint{2.326675in}{1.361588in}}%
\pgfpathcurveto{\pgfqpoint{2.318861in}{1.369401in}}{\pgfqpoint{2.308262in}{1.373791in}}{\pgfqpoint{2.297212in}{1.373791in}}%
\pgfpathcurveto{\pgfqpoint{2.286162in}{1.373791in}}{\pgfqpoint{2.275563in}{1.369401in}}{\pgfqpoint{2.267749in}{1.361588in}}%
\pgfpathcurveto{\pgfqpoint{2.259936in}{1.353774in}}{\pgfqpoint{2.255545in}{1.343175in}}{\pgfqpoint{2.255545in}{1.332125in}}%
\pgfpathcurveto{\pgfqpoint{2.255545in}{1.321075in}}{\pgfqpoint{2.259936in}{1.310476in}}{\pgfqpoint{2.267749in}{1.302662in}}%
\pgfpathcurveto{\pgfqpoint{2.275563in}{1.294848in}}{\pgfqpoint{2.286162in}{1.290458in}}{\pgfqpoint{2.297212in}{1.290458in}}%
\pgfpathclose%
\pgfusepath{stroke,fill}%
\end{pgfscope}%
\begin{pgfscope}%
\pgfpathrectangle{\pgfqpoint{0.888750in}{0.419100in}}{\pgfqpoint{2.504659in}{2.933700in}} %
\pgfusepath{clip}%
\pgfsetbuttcap%
\pgfsetroundjoin%
\definecolor{currentfill}{rgb}{1.000000,0.549020,0.000000}%
\pgfsetfillcolor{currentfill}%
\pgfsetlinewidth{1.003750pt}%
\definecolor{currentstroke}{rgb}{1.000000,0.549020,0.000000}%
\pgfsetstrokecolor{currentstroke}%
\pgfsetdash{}{0pt}%
\pgfpathmoveto{\pgfqpoint{1.908971in}{1.895820in}}%
\pgfpathcurveto{\pgfqpoint{1.920021in}{1.895820in}}{\pgfqpoint{1.930620in}{1.900210in}}{\pgfqpoint{1.938434in}{1.908024in}}%
\pgfpathcurveto{\pgfqpoint{1.946247in}{1.915837in}}{\pgfqpoint{1.950638in}{1.926436in}}{\pgfqpoint{1.950638in}{1.937487in}}%
\pgfpathcurveto{\pgfqpoint{1.950638in}{1.948537in}}{\pgfqpoint{1.946247in}{1.959136in}}{\pgfqpoint{1.938434in}{1.966949in}}%
\pgfpathcurveto{\pgfqpoint{1.930620in}{1.974763in}}{\pgfqpoint{1.920021in}{1.979153in}}{\pgfqpoint{1.908971in}{1.979153in}}%
\pgfpathcurveto{\pgfqpoint{1.897921in}{1.979153in}}{\pgfqpoint{1.887322in}{1.974763in}}{\pgfqpoint{1.879508in}{1.966949in}}%
\pgfpathcurveto{\pgfqpoint{1.871695in}{1.959136in}}{\pgfqpoint{1.867304in}{1.948537in}}{\pgfqpoint{1.867304in}{1.937487in}}%
\pgfpathcurveto{\pgfqpoint{1.867304in}{1.926436in}}{\pgfqpoint{1.871695in}{1.915837in}}{\pgfqpoint{1.879508in}{1.908024in}}%
\pgfpathcurveto{\pgfqpoint{1.887322in}{1.900210in}}{\pgfqpoint{1.897921in}{1.895820in}}{\pgfqpoint{1.908971in}{1.895820in}}%
\pgfpathclose%
\pgfusepath{stroke,fill}%
\end{pgfscope}%
\begin{pgfscope}%
\pgfpathrectangle{\pgfqpoint{0.888750in}{0.419100in}}{\pgfqpoint{2.504659in}{2.933700in}} %
\pgfusepath{clip}%
\pgfsetbuttcap%
\pgfsetroundjoin%
\definecolor{currentfill}{rgb}{1.000000,0.549020,0.000000}%
\pgfsetfillcolor{currentfill}%
\pgfsetlinewidth{1.003750pt}%
\definecolor{currentstroke}{rgb}{1.000000,0.549020,0.000000}%
\pgfsetstrokecolor{currentstroke}%
\pgfsetdash{}{0pt}%
\pgfpathmoveto{\pgfqpoint{2.222050in}{1.333620in}}%
\pgfpathcurveto{\pgfqpoint{2.233100in}{1.333620in}}{\pgfqpoint{2.243699in}{1.338011in}}{\pgfqpoint{2.251513in}{1.345824in}}%
\pgfpathcurveto{\pgfqpoint{2.259327in}{1.353638in}}{\pgfqpoint{2.263717in}{1.364237in}}{\pgfqpoint{2.263717in}{1.375287in}}%
\pgfpathcurveto{\pgfqpoint{2.263717in}{1.386337in}}{\pgfqpoint{2.259327in}{1.396936in}}{\pgfqpoint{2.251513in}{1.404750in}}%
\pgfpathcurveto{\pgfqpoint{2.243699in}{1.412564in}}{\pgfqpoint{2.233100in}{1.416954in}}{\pgfqpoint{2.222050in}{1.416954in}}%
\pgfpathcurveto{\pgfqpoint{2.211000in}{1.416954in}}{\pgfqpoint{2.200401in}{1.412564in}}{\pgfqpoint{2.192587in}{1.404750in}}%
\pgfpathcurveto{\pgfqpoint{2.184774in}{1.396936in}}{\pgfqpoint{2.180384in}{1.386337in}}{\pgfqpoint{2.180384in}{1.375287in}}%
\pgfpathcurveto{\pgfqpoint{2.180384in}{1.364237in}}{\pgfqpoint{2.184774in}{1.353638in}}{\pgfqpoint{2.192587in}{1.345824in}}%
\pgfpathcurveto{\pgfqpoint{2.200401in}{1.338011in}}{\pgfqpoint{2.211000in}{1.333620in}}{\pgfqpoint{2.222050in}{1.333620in}}%
\pgfpathclose%
\pgfusepath{stroke,fill}%
\end{pgfscope}%
\begin{pgfscope}%
\pgfpathrectangle{\pgfqpoint{0.888750in}{0.419100in}}{\pgfqpoint{2.504659in}{2.933700in}} %
\pgfusepath{clip}%
\pgfsetbuttcap%
\pgfsetroundjoin%
\definecolor{currentfill}{rgb}{1.000000,0.549020,0.000000}%
\pgfsetfillcolor{currentfill}%
\pgfsetlinewidth{1.003750pt}%
\definecolor{currentstroke}{rgb}{1.000000,0.549020,0.000000}%
\pgfsetstrokecolor{currentstroke}%
\pgfsetdash{}{0pt}%
\pgfpathmoveto{\pgfqpoint{2.499571in}{1.347211in}}%
\pgfpathcurveto{\pgfqpoint{2.510621in}{1.347211in}}{\pgfqpoint{2.521220in}{1.351601in}}{\pgfqpoint{2.529034in}{1.359415in}}%
\pgfpathcurveto{\pgfqpoint{2.536847in}{1.367228in}}{\pgfqpoint{2.541238in}{1.377827in}}{\pgfqpoint{2.541238in}{1.388878in}}%
\pgfpathcurveto{\pgfqpoint{2.541238in}{1.399928in}}{\pgfqpoint{2.536847in}{1.410527in}}{\pgfqpoint{2.529034in}{1.418340in}}%
\pgfpathcurveto{\pgfqpoint{2.521220in}{1.426154in}}{\pgfqpoint{2.510621in}{1.430544in}}{\pgfqpoint{2.499571in}{1.430544in}}%
\pgfpathcurveto{\pgfqpoint{2.488521in}{1.430544in}}{\pgfqpoint{2.477922in}{1.426154in}}{\pgfqpoint{2.470108in}{1.418340in}}%
\pgfpathcurveto{\pgfqpoint{2.462295in}{1.410527in}}{\pgfqpoint{2.457904in}{1.399928in}}{\pgfqpoint{2.457904in}{1.388878in}}%
\pgfpathcurveto{\pgfqpoint{2.457904in}{1.377827in}}{\pgfqpoint{2.462295in}{1.367228in}}{\pgfqpoint{2.470108in}{1.359415in}}%
\pgfpathcurveto{\pgfqpoint{2.477922in}{1.351601in}}{\pgfqpoint{2.488521in}{1.347211in}}{\pgfqpoint{2.499571in}{1.347211in}}%
\pgfpathclose%
\pgfusepath{stroke,fill}%
\end{pgfscope}%
\begin{pgfscope}%
\pgfpathrectangle{\pgfqpoint{0.888750in}{0.419100in}}{\pgfqpoint{2.504659in}{2.933700in}} %
\pgfusepath{clip}%
\pgfsetbuttcap%
\pgfsetroundjoin%
\definecolor{currentfill}{rgb}{1.000000,0.549020,0.000000}%
\pgfsetfillcolor{currentfill}%
\pgfsetlinewidth{1.003750pt}%
\definecolor{currentstroke}{rgb}{1.000000,0.549020,0.000000}%
\pgfsetstrokecolor{currentstroke}%
\pgfsetdash{}{0pt}%
\pgfpathmoveto{\pgfqpoint{1.698190in}{1.836983in}}%
\pgfpathcurveto{\pgfqpoint{1.709240in}{1.836983in}}{\pgfqpoint{1.719839in}{1.841373in}}{\pgfqpoint{1.727653in}{1.849187in}}%
\pgfpathcurveto{\pgfqpoint{1.735467in}{1.857000in}}{\pgfqpoint{1.739857in}{1.867599in}}{\pgfqpoint{1.739857in}{1.878649in}}%
\pgfpathcurveto{\pgfqpoint{1.739857in}{1.889700in}}{\pgfqpoint{1.735467in}{1.900299in}}{\pgfqpoint{1.727653in}{1.908112in}}%
\pgfpathcurveto{\pgfqpoint{1.719839in}{1.915926in}}{\pgfqpoint{1.709240in}{1.920316in}}{\pgfqpoint{1.698190in}{1.920316in}}%
\pgfpathcurveto{\pgfqpoint{1.687140in}{1.920316in}}{\pgfqpoint{1.676541in}{1.915926in}}{\pgfqpoint{1.668727in}{1.908112in}}%
\pgfpathcurveto{\pgfqpoint{1.660914in}{1.900299in}}{\pgfqpoint{1.656524in}{1.889700in}}{\pgfqpoint{1.656524in}{1.878649in}}%
\pgfpathcurveto{\pgfqpoint{1.656524in}{1.867599in}}{\pgfqpoint{1.660914in}{1.857000in}}{\pgfqpoint{1.668727in}{1.849187in}}%
\pgfpathcurveto{\pgfqpoint{1.676541in}{1.841373in}}{\pgfqpoint{1.687140in}{1.836983in}}{\pgfqpoint{1.698190in}{1.836983in}}%
\pgfpathclose%
\pgfusepath{stroke,fill}%
\end{pgfscope}%
\begin{pgfscope}%
\pgfpathrectangle{\pgfqpoint{0.888750in}{0.419100in}}{\pgfqpoint{2.504659in}{2.933700in}} %
\pgfusepath{clip}%
\pgfsetbuttcap%
\pgfsetroundjoin%
\definecolor{currentfill}{rgb}{1.000000,0.549020,0.000000}%
\pgfsetfillcolor{currentfill}%
\pgfsetlinewidth{1.003750pt}%
\definecolor{currentstroke}{rgb}{1.000000,0.549020,0.000000}%
\pgfsetstrokecolor{currentstroke}%
\pgfsetdash{}{0pt}%
\pgfpathmoveto{\pgfqpoint{2.249205in}{2.059859in}}%
\pgfpathcurveto{\pgfqpoint{2.260255in}{2.059859in}}{\pgfqpoint{2.270854in}{2.064249in}}{\pgfqpoint{2.278668in}{2.072063in}}%
\pgfpathcurveto{\pgfqpoint{2.286482in}{2.079877in}}{\pgfqpoint{2.290872in}{2.090476in}}{\pgfqpoint{2.290872in}{2.101526in}}%
\pgfpathcurveto{\pgfqpoint{2.290872in}{2.112576in}}{\pgfqpoint{2.286482in}{2.123175in}}{\pgfqpoint{2.278668in}{2.130988in}}%
\pgfpathcurveto{\pgfqpoint{2.270854in}{2.138802in}}{\pgfqpoint{2.260255in}{2.143192in}}{\pgfqpoint{2.249205in}{2.143192in}}%
\pgfpathcurveto{\pgfqpoint{2.238155in}{2.143192in}}{\pgfqpoint{2.227556in}{2.138802in}}{\pgfqpoint{2.219743in}{2.130988in}}%
\pgfpathcurveto{\pgfqpoint{2.211929in}{2.123175in}}{\pgfqpoint{2.207539in}{2.112576in}}{\pgfqpoint{2.207539in}{2.101526in}}%
\pgfpathcurveto{\pgfqpoint{2.207539in}{2.090476in}}{\pgfqpoint{2.211929in}{2.079877in}}{\pgfqpoint{2.219743in}{2.072063in}}%
\pgfpathcurveto{\pgfqpoint{2.227556in}{2.064249in}}{\pgfqpoint{2.238155in}{2.059859in}}{\pgfqpoint{2.249205in}{2.059859in}}%
\pgfpathclose%
\pgfusepath{stroke,fill}%
\end{pgfscope}%
\begin{pgfscope}%
\pgfpathrectangle{\pgfqpoint{0.888750in}{0.419100in}}{\pgfqpoint{2.504659in}{2.933700in}} %
\pgfusepath{clip}%
\pgfsetbuttcap%
\pgfsetroundjoin%
\definecolor{currentfill}{rgb}{1.000000,0.549020,0.000000}%
\pgfsetfillcolor{currentfill}%
\pgfsetlinewidth{1.003750pt}%
\definecolor{currentstroke}{rgb}{1.000000,0.549020,0.000000}%
\pgfsetstrokecolor{currentstroke}%
\pgfsetdash{}{0pt}%
\pgfpathmoveto{\pgfqpoint{1.962770in}{2.311576in}}%
\pgfpathcurveto{\pgfqpoint{1.973820in}{2.311576in}}{\pgfqpoint{1.984419in}{2.315967in}}{\pgfqpoint{1.992232in}{2.323780in}}%
\pgfpathcurveto{\pgfqpoint{2.000046in}{2.331594in}}{\pgfqpoint{2.004436in}{2.342193in}}{\pgfqpoint{2.004436in}{2.353243in}}%
\pgfpathcurveto{\pgfqpoint{2.004436in}{2.364293in}}{\pgfqpoint{2.000046in}{2.374892in}}{\pgfqpoint{1.992232in}{2.382706in}}%
\pgfpathcurveto{\pgfqpoint{1.984419in}{2.390519in}}{\pgfqpoint{1.973820in}{2.394910in}}{\pgfqpoint{1.962770in}{2.394910in}}%
\pgfpathcurveto{\pgfqpoint{1.951719in}{2.394910in}}{\pgfqpoint{1.941120in}{2.390519in}}{\pgfqpoint{1.933307in}{2.382706in}}%
\pgfpathcurveto{\pgfqpoint{1.925493in}{2.374892in}}{\pgfqpoint{1.921103in}{2.364293in}}{\pgfqpoint{1.921103in}{2.353243in}}%
\pgfpathcurveto{\pgfqpoint{1.921103in}{2.342193in}}{\pgfqpoint{1.925493in}{2.331594in}}{\pgfqpoint{1.933307in}{2.323780in}}%
\pgfpathcurveto{\pgfqpoint{1.941120in}{2.315967in}}{\pgfqpoint{1.951719in}{2.311576in}}{\pgfqpoint{1.962770in}{2.311576in}}%
\pgfpathclose%
\pgfusepath{stroke,fill}%
\end{pgfscope}%
\begin{pgfscope}%
\pgfpathrectangle{\pgfqpoint{0.888750in}{0.419100in}}{\pgfqpoint{2.504659in}{2.933700in}} %
\pgfusepath{clip}%
\pgfsetbuttcap%
\pgfsetroundjoin%
\definecolor{currentfill}{rgb}{1.000000,0.549020,0.000000}%
\pgfsetfillcolor{currentfill}%
\pgfsetlinewidth{1.003750pt}%
\definecolor{currentstroke}{rgb}{1.000000,0.549020,0.000000}%
\pgfsetstrokecolor{currentstroke}%
\pgfsetdash{}{0pt}%
\pgfpathmoveto{\pgfqpoint{2.460155in}{1.675139in}}%
\pgfpathcurveto{\pgfqpoint{2.471205in}{1.675139in}}{\pgfqpoint{2.481804in}{1.679529in}}{\pgfqpoint{2.489617in}{1.687343in}}%
\pgfpathcurveto{\pgfqpoint{2.497431in}{1.695156in}}{\pgfqpoint{2.501821in}{1.705755in}}{\pgfqpoint{2.501821in}{1.716806in}}%
\pgfpathcurveto{\pgfqpoint{2.501821in}{1.727856in}}{\pgfqpoint{2.497431in}{1.738455in}}{\pgfqpoint{2.489617in}{1.746268in}}%
\pgfpathcurveto{\pgfqpoint{2.481804in}{1.754082in}}{\pgfqpoint{2.471205in}{1.758472in}}{\pgfqpoint{2.460155in}{1.758472in}}%
\pgfpathcurveto{\pgfqpoint{2.449105in}{1.758472in}}{\pgfqpoint{2.438506in}{1.754082in}}{\pgfqpoint{2.430692in}{1.746268in}}%
\pgfpathcurveto{\pgfqpoint{2.422878in}{1.738455in}}{\pgfqpoint{2.418488in}{1.727856in}}{\pgfqpoint{2.418488in}{1.716806in}}%
\pgfpathcurveto{\pgfqpoint{2.418488in}{1.705755in}}{\pgfqpoint{2.422878in}{1.695156in}}{\pgfqpoint{2.430692in}{1.687343in}}%
\pgfpathcurveto{\pgfqpoint{2.438506in}{1.679529in}}{\pgfqpoint{2.449105in}{1.675139in}}{\pgfqpoint{2.460155in}{1.675139in}}%
\pgfpathclose%
\pgfusepath{stroke,fill}%
\end{pgfscope}%
\begin{pgfscope}%
\pgfpathrectangle{\pgfqpoint{0.888750in}{0.419100in}}{\pgfqpoint{2.504659in}{2.933700in}} %
\pgfusepath{clip}%
\pgfsetbuttcap%
\pgfsetroundjoin%
\definecolor{currentfill}{rgb}{1.000000,0.549020,0.000000}%
\pgfsetfillcolor{currentfill}%
\pgfsetlinewidth{1.003750pt}%
\definecolor{currentstroke}{rgb}{1.000000,0.549020,0.000000}%
\pgfsetstrokecolor{currentstroke}%
\pgfsetdash{}{0pt}%
\pgfpathmoveto{\pgfqpoint{2.181452in}{1.708197in}}%
\pgfpathcurveto{\pgfqpoint{2.192502in}{1.708197in}}{\pgfqpoint{2.203101in}{1.712587in}}{\pgfqpoint{2.210915in}{1.720400in}}%
\pgfpathcurveto{\pgfqpoint{2.218729in}{1.728214in}}{\pgfqpoint{2.223119in}{1.738813in}}{\pgfqpoint{2.223119in}{1.749863in}}%
\pgfpathcurveto{\pgfqpoint{2.223119in}{1.760913in}}{\pgfqpoint{2.218729in}{1.771512in}}{\pgfqpoint{2.210915in}{1.779326in}}%
\pgfpathcurveto{\pgfqpoint{2.203101in}{1.787140in}}{\pgfqpoint{2.192502in}{1.791530in}}{\pgfqpoint{2.181452in}{1.791530in}}%
\pgfpathcurveto{\pgfqpoint{2.170402in}{1.791530in}}{\pgfqpoint{2.159803in}{1.787140in}}{\pgfqpoint{2.151990in}{1.779326in}}%
\pgfpathcurveto{\pgfqpoint{2.144176in}{1.771512in}}{\pgfqpoint{2.139786in}{1.760913in}}{\pgfqpoint{2.139786in}{1.749863in}}%
\pgfpathcurveto{\pgfqpoint{2.139786in}{1.738813in}}{\pgfqpoint{2.144176in}{1.728214in}}{\pgfqpoint{2.151990in}{1.720400in}}%
\pgfpathcurveto{\pgfqpoint{2.159803in}{1.712587in}}{\pgfqpoint{2.170402in}{1.708197in}}{\pgfqpoint{2.181452in}{1.708197in}}%
\pgfpathclose%
\pgfusepath{stroke,fill}%
\end{pgfscope}%
\begin{pgfscope}%
\pgfpathrectangle{\pgfqpoint{0.888750in}{0.419100in}}{\pgfqpoint{2.504659in}{2.933700in}} %
\pgfusepath{clip}%
\pgfsetbuttcap%
\pgfsetroundjoin%
\definecolor{currentfill}{rgb}{1.000000,0.549020,0.000000}%
\pgfsetfillcolor{currentfill}%
\pgfsetlinewidth{1.003750pt}%
\definecolor{currentstroke}{rgb}{1.000000,0.549020,0.000000}%
\pgfsetstrokecolor{currentstroke}%
\pgfsetdash{}{0pt}%
\pgfpathmoveto{\pgfqpoint{2.220578in}{2.011684in}}%
\pgfpathcurveto{\pgfqpoint{2.231628in}{2.011684in}}{\pgfqpoint{2.242227in}{2.016074in}}{\pgfqpoint{2.250041in}{2.023887in}}%
\pgfpathcurveto{\pgfqpoint{2.257854in}{2.031701in}}{\pgfqpoint{2.262245in}{2.042300in}}{\pgfqpoint{2.262245in}{2.053350in}}%
\pgfpathcurveto{\pgfqpoint{2.262245in}{2.064400in}}{\pgfqpoint{2.257854in}{2.074999in}}{\pgfqpoint{2.250041in}{2.082813in}}%
\pgfpathcurveto{\pgfqpoint{2.242227in}{2.090627in}}{\pgfqpoint{2.231628in}{2.095017in}}{\pgfqpoint{2.220578in}{2.095017in}}%
\pgfpathcurveto{\pgfqpoint{2.209528in}{2.095017in}}{\pgfqpoint{2.198929in}{2.090627in}}{\pgfqpoint{2.191115in}{2.082813in}}%
\pgfpathcurveto{\pgfqpoint{2.183301in}{2.074999in}}{\pgfqpoint{2.178911in}{2.064400in}}{\pgfqpoint{2.178911in}{2.053350in}}%
\pgfpathcurveto{\pgfqpoint{2.178911in}{2.042300in}}{\pgfqpoint{2.183301in}{2.031701in}}{\pgfqpoint{2.191115in}{2.023887in}}%
\pgfpathcurveto{\pgfqpoint{2.198929in}{2.016074in}}{\pgfqpoint{2.209528in}{2.011684in}}{\pgfqpoint{2.220578in}{2.011684in}}%
\pgfpathclose%
\pgfusepath{stroke,fill}%
\end{pgfscope}%
\begin{pgfscope}%
\pgfpathrectangle{\pgfqpoint{0.888750in}{0.419100in}}{\pgfqpoint{2.504659in}{2.933700in}} %
\pgfusepath{clip}%
\pgfsetbuttcap%
\pgfsetroundjoin%
\definecolor{currentfill}{rgb}{1.000000,0.549020,0.000000}%
\pgfsetfillcolor{currentfill}%
\pgfsetlinewidth{1.003750pt}%
\definecolor{currentstroke}{rgb}{1.000000,0.549020,0.000000}%
\pgfsetstrokecolor{currentstroke}%
\pgfsetdash{}{0pt}%
\pgfpathmoveto{\pgfqpoint{2.496122in}{1.847096in}}%
\pgfpathcurveto{\pgfqpoint{2.507172in}{1.847096in}}{\pgfqpoint{2.517771in}{1.851486in}}{\pgfqpoint{2.525585in}{1.859300in}}%
\pgfpathcurveto{\pgfqpoint{2.533399in}{1.867113in}}{\pgfqpoint{2.537789in}{1.877712in}}{\pgfqpoint{2.537789in}{1.888763in}}%
\pgfpathcurveto{\pgfqpoint{2.537789in}{1.899813in}}{\pgfqpoint{2.533399in}{1.910412in}}{\pgfqpoint{2.525585in}{1.918225in}}%
\pgfpathcurveto{\pgfqpoint{2.517771in}{1.926039in}}{\pgfqpoint{2.507172in}{1.930429in}}{\pgfqpoint{2.496122in}{1.930429in}}%
\pgfpathcurveto{\pgfqpoint{2.485072in}{1.930429in}}{\pgfqpoint{2.474473in}{1.926039in}}{\pgfqpoint{2.466659in}{1.918225in}}%
\pgfpathcurveto{\pgfqpoint{2.458846in}{1.910412in}}{\pgfqpoint{2.454456in}{1.899813in}}{\pgfqpoint{2.454456in}{1.888763in}}%
\pgfpathcurveto{\pgfqpoint{2.454456in}{1.877712in}}{\pgfqpoint{2.458846in}{1.867113in}}{\pgfqpoint{2.466659in}{1.859300in}}%
\pgfpathcurveto{\pgfqpoint{2.474473in}{1.851486in}}{\pgfqpoint{2.485072in}{1.847096in}}{\pgfqpoint{2.496122in}{1.847096in}}%
\pgfpathclose%
\pgfusepath{stroke,fill}%
\end{pgfscope}%
\begin{pgfscope}%
\pgfpathrectangle{\pgfqpoint{0.888750in}{0.419100in}}{\pgfqpoint{2.504659in}{2.933700in}} %
\pgfusepath{clip}%
\pgfsetbuttcap%
\pgfsetroundjoin%
\definecolor{currentfill}{rgb}{1.000000,0.549020,0.000000}%
\pgfsetfillcolor{currentfill}%
\pgfsetlinewidth{1.003750pt}%
\definecolor{currentstroke}{rgb}{1.000000,0.549020,0.000000}%
\pgfsetstrokecolor{currentstroke}%
\pgfsetdash{}{0pt}%
\pgfpathmoveto{\pgfqpoint{1.723339in}{2.308470in}}%
\pgfpathcurveto{\pgfqpoint{1.734389in}{2.308470in}}{\pgfqpoint{1.744988in}{2.312860in}}{\pgfqpoint{1.752802in}{2.320674in}}%
\pgfpathcurveto{\pgfqpoint{1.760616in}{2.328487in}}{\pgfqpoint{1.765006in}{2.339086in}}{\pgfqpoint{1.765006in}{2.350136in}}%
\pgfpathcurveto{\pgfqpoint{1.765006in}{2.361186in}}{\pgfqpoint{1.760616in}{2.371786in}}{\pgfqpoint{1.752802in}{2.379599in}}%
\pgfpathcurveto{\pgfqpoint{1.744988in}{2.387413in}}{\pgfqpoint{1.734389in}{2.391803in}}{\pgfqpoint{1.723339in}{2.391803in}}%
\pgfpathcurveto{\pgfqpoint{1.712289in}{2.391803in}}{\pgfqpoint{1.701690in}{2.387413in}}{\pgfqpoint{1.693877in}{2.379599in}}%
\pgfpathcurveto{\pgfqpoint{1.686063in}{2.371786in}}{\pgfqpoint{1.681673in}{2.361186in}}{\pgfqpoint{1.681673in}{2.350136in}}%
\pgfpathcurveto{\pgfqpoint{1.681673in}{2.339086in}}{\pgfqpoint{1.686063in}{2.328487in}}{\pgfqpoint{1.693877in}{2.320674in}}%
\pgfpathcurveto{\pgfqpoint{1.701690in}{2.312860in}}{\pgfqpoint{1.712289in}{2.308470in}}{\pgfqpoint{1.723339in}{2.308470in}}%
\pgfpathclose%
\pgfusepath{stroke,fill}%
\end{pgfscope}%
\begin{pgfscope}%
\pgfpathrectangle{\pgfqpoint{0.888750in}{0.419100in}}{\pgfqpoint{2.504659in}{2.933700in}} %
\pgfusepath{clip}%
\pgfsetbuttcap%
\pgfsetroundjoin%
\definecolor{currentfill}{rgb}{1.000000,0.549020,0.000000}%
\pgfsetfillcolor{currentfill}%
\pgfsetlinewidth{1.003750pt}%
\definecolor{currentstroke}{rgb}{1.000000,0.549020,0.000000}%
\pgfsetstrokecolor{currentstroke}%
\pgfsetdash{}{0pt}%
\pgfpathmoveto{\pgfqpoint{1.627510in}{1.780404in}}%
\pgfpathcurveto{\pgfqpoint{1.638560in}{1.780404in}}{\pgfqpoint{1.649159in}{1.784794in}}{\pgfqpoint{1.656973in}{1.792608in}}%
\pgfpathcurveto{\pgfqpoint{1.664786in}{1.800421in}}{\pgfqpoint{1.669177in}{1.811021in}}{\pgfqpoint{1.669177in}{1.822071in}}%
\pgfpathcurveto{\pgfqpoint{1.669177in}{1.833121in}}{\pgfqpoint{1.664786in}{1.843720in}}{\pgfqpoint{1.656973in}{1.851533in}}%
\pgfpathcurveto{\pgfqpoint{1.649159in}{1.859347in}}{\pgfqpoint{1.638560in}{1.863737in}}{\pgfqpoint{1.627510in}{1.863737in}}%
\pgfpathcurveto{\pgfqpoint{1.616460in}{1.863737in}}{\pgfqpoint{1.605861in}{1.859347in}}{\pgfqpoint{1.598047in}{1.851533in}}%
\pgfpathcurveto{\pgfqpoint{1.590234in}{1.843720in}}{\pgfqpoint{1.585843in}{1.833121in}}{\pgfqpoint{1.585843in}{1.822071in}}%
\pgfpathcurveto{\pgfqpoint{1.585843in}{1.811021in}}{\pgfqpoint{1.590234in}{1.800421in}}{\pgfqpoint{1.598047in}{1.792608in}}%
\pgfpathcurveto{\pgfqpoint{1.605861in}{1.784794in}}{\pgfqpoint{1.616460in}{1.780404in}}{\pgfqpoint{1.627510in}{1.780404in}}%
\pgfpathclose%
\pgfusepath{stroke,fill}%
\end{pgfscope}%
\begin{pgfscope}%
\pgfpathrectangle{\pgfqpoint{0.888750in}{0.419100in}}{\pgfqpoint{2.504659in}{2.933700in}} %
\pgfusepath{clip}%
\pgfsetbuttcap%
\pgfsetroundjoin%
\definecolor{currentfill}{rgb}{1.000000,0.549020,0.000000}%
\pgfsetfillcolor{currentfill}%
\pgfsetlinewidth{1.003750pt}%
\definecolor{currentstroke}{rgb}{1.000000,0.549020,0.000000}%
\pgfsetstrokecolor{currentstroke}%
\pgfsetdash{}{0pt}%
\pgfpathmoveto{\pgfqpoint{1.995110in}{1.965028in}}%
\pgfpathcurveto{\pgfqpoint{2.006160in}{1.965028in}}{\pgfqpoint{2.016759in}{1.969418in}}{\pgfqpoint{2.024573in}{1.977232in}}%
\pgfpathcurveto{\pgfqpoint{2.032387in}{1.985045in}}{\pgfqpoint{2.036777in}{1.995644in}}{\pgfqpoint{2.036777in}{2.006695in}}%
\pgfpathcurveto{\pgfqpoint{2.036777in}{2.017745in}}{\pgfqpoint{2.032387in}{2.028344in}}{\pgfqpoint{2.024573in}{2.036157in}}%
\pgfpathcurveto{\pgfqpoint{2.016759in}{2.043971in}}{\pgfqpoint{2.006160in}{2.048361in}}{\pgfqpoint{1.995110in}{2.048361in}}%
\pgfpathcurveto{\pgfqpoint{1.984060in}{2.048361in}}{\pgfqpoint{1.973461in}{2.043971in}}{\pgfqpoint{1.965647in}{2.036157in}}%
\pgfpathcurveto{\pgfqpoint{1.957834in}{2.028344in}}{\pgfqpoint{1.953444in}{2.017745in}}{\pgfqpoint{1.953444in}{2.006695in}}%
\pgfpathcurveto{\pgfqpoint{1.953444in}{1.995644in}}{\pgfqpoint{1.957834in}{1.985045in}}{\pgfqpoint{1.965647in}{1.977232in}}%
\pgfpathcurveto{\pgfqpoint{1.973461in}{1.969418in}}{\pgfqpoint{1.984060in}{1.965028in}}{\pgfqpoint{1.995110in}{1.965028in}}%
\pgfpathclose%
\pgfusepath{stroke,fill}%
\end{pgfscope}%
\begin{pgfscope}%
\pgfpathrectangle{\pgfqpoint{0.888750in}{0.419100in}}{\pgfqpoint{2.504659in}{2.933700in}} %
\pgfusepath{clip}%
\pgfsetbuttcap%
\pgfsetroundjoin%
\definecolor{currentfill}{rgb}{1.000000,0.549020,0.000000}%
\pgfsetfillcolor{currentfill}%
\pgfsetlinewidth{1.003750pt}%
\definecolor{currentstroke}{rgb}{1.000000,0.549020,0.000000}%
\pgfsetstrokecolor{currentstroke}%
\pgfsetdash{}{0pt}%
\pgfpathmoveto{\pgfqpoint{2.191001in}{2.027598in}}%
\pgfpathcurveto{\pgfqpoint{2.202052in}{2.027598in}}{\pgfqpoint{2.212651in}{2.031988in}}{\pgfqpoint{2.220464in}{2.039802in}}%
\pgfpathcurveto{\pgfqpoint{2.228278in}{2.047616in}}{\pgfqpoint{2.232668in}{2.058215in}}{\pgfqpoint{2.232668in}{2.069265in}}%
\pgfpathcurveto{\pgfqpoint{2.232668in}{2.080315in}}{\pgfqpoint{2.228278in}{2.090914in}}{\pgfqpoint{2.220464in}{2.098728in}}%
\pgfpathcurveto{\pgfqpoint{2.212651in}{2.106541in}}{\pgfqpoint{2.202052in}{2.110932in}}{\pgfqpoint{2.191001in}{2.110932in}}%
\pgfpathcurveto{\pgfqpoint{2.179951in}{2.110932in}}{\pgfqpoint{2.169352in}{2.106541in}}{\pgfqpoint{2.161539in}{2.098728in}}%
\pgfpathcurveto{\pgfqpoint{2.153725in}{2.090914in}}{\pgfqpoint{2.149335in}{2.080315in}}{\pgfqpoint{2.149335in}{2.069265in}}%
\pgfpathcurveto{\pgfqpoint{2.149335in}{2.058215in}}{\pgfqpoint{2.153725in}{2.047616in}}{\pgfqpoint{2.161539in}{2.039802in}}%
\pgfpathcurveto{\pgfqpoint{2.169352in}{2.031988in}}{\pgfqpoint{2.179951in}{2.027598in}}{\pgfqpoint{2.191001in}{2.027598in}}%
\pgfpathclose%
\pgfusepath{stroke,fill}%
\end{pgfscope}%
\begin{pgfscope}%
\pgfpathrectangle{\pgfqpoint{0.888750in}{0.419100in}}{\pgfqpoint{2.504659in}{2.933700in}} %
\pgfusepath{clip}%
\pgfsetbuttcap%
\pgfsetroundjoin%
\definecolor{currentfill}{rgb}{1.000000,0.549020,0.000000}%
\pgfsetfillcolor{currentfill}%
\pgfsetlinewidth{1.003750pt}%
\definecolor{currentstroke}{rgb}{1.000000,0.549020,0.000000}%
\pgfsetstrokecolor{currentstroke}%
\pgfsetdash{}{0pt}%
\pgfpathmoveto{\pgfqpoint{2.463545in}{1.917815in}}%
\pgfpathcurveto{\pgfqpoint{2.474595in}{1.917815in}}{\pgfqpoint{2.485194in}{1.922206in}}{\pgfqpoint{2.493007in}{1.930019in}}%
\pgfpathcurveto{\pgfqpoint{2.500821in}{1.937833in}}{\pgfqpoint{2.505211in}{1.948432in}}{\pgfqpoint{2.505211in}{1.959482in}}%
\pgfpathcurveto{\pgfqpoint{2.505211in}{1.970532in}}{\pgfqpoint{2.500821in}{1.981131in}}{\pgfqpoint{2.493007in}{1.988945in}}%
\pgfpathcurveto{\pgfqpoint{2.485194in}{1.996758in}}{\pgfqpoint{2.474595in}{2.001149in}}{\pgfqpoint{2.463545in}{2.001149in}}%
\pgfpathcurveto{\pgfqpoint{2.452495in}{2.001149in}}{\pgfqpoint{2.441896in}{1.996758in}}{\pgfqpoint{2.434082in}{1.988945in}}%
\pgfpathcurveto{\pgfqpoint{2.426268in}{1.981131in}}{\pgfqpoint{2.421878in}{1.970532in}}{\pgfqpoint{2.421878in}{1.959482in}}%
\pgfpathcurveto{\pgfqpoint{2.421878in}{1.948432in}}{\pgfqpoint{2.426268in}{1.937833in}}{\pgfqpoint{2.434082in}{1.930019in}}%
\pgfpathcurveto{\pgfqpoint{2.441896in}{1.922206in}}{\pgfqpoint{2.452495in}{1.917815in}}{\pgfqpoint{2.463545in}{1.917815in}}%
\pgfpathclose%
\pgfusepath{stroke,fill}%
\end{pgfscope}%
\begin{pgfscope}%
\pgfpathrectangle{\pgfqpoint{0.888750in}{0.419100in}}{\pgfqpoint{2.504659in}{2.933700in}} %
\pgfusepath{clip}%
\pgfsetbuttcap%
\pgfsetroundjoin%
\definecolor{currentfill}{rgb}{1.000000,0.549020,0.000000}%
\pgfsetfillcolor{currentfill}%
\pgfsetlinewidth{1.003750pt}%
\definecolor{currentstroke}{rgb}{1.000000,0.549020,0.000000}%
\pgfsetstrokecolor{currentstroke}%
\pgfsetdash{}{0pt}%
\pgfpathmoveto{\pgfqpoint{2.676948in}{1.257937in}}%
\pgfpathcurveto{\pgfqpoint{2.687998in}{1.257937in}}{\pgfqpoint{2.698597in}{1.262327in}}{\pgfqpoint{2.706411in}{1.270141in}}%
\pgfpathcurveto{\pgfqpoint{2.714224in}{1.277955in}}{\pgfqpoint{2.718615in}{1.288554in}}{\pgfqpoint{2.718615in}{1.299604in}}%
\pgfpathcurveto{\pgfqpoint{2.718615in}{1.310654in}}{\pgfqpoint{2.714224in}{1.321253in}}{\pgfqpoint{2.706411in}{1.329067in}}%
\pgfpathcurveto{\pgfqpoint{2.698597in}{1.336880in}}{\pgfqpoint{2.687998in}{1.341270in}}{\pgfqpoint{2.676948in}{1.341270in}}%
\pgfpathcurveto{\pgfqpoint{2.665898in}{1.341270in}}{\pgfqpoint{2.655299in}{1.336880in}}{\pgfqpoint{2.647485in}{1.329067in}}%
\pgfpathcurveto{\pgfqpoint{2.639672in}{1.321253in}}{\pgfqpoint{2.635281in}{1.310654in}}{\pgfqpoint{2.635281in}{1.299604in}}%
\pgfpathcurveto{\pgfqpoint{2.635281in}{1.288554in}}{\pgfqpoint{2.639672in}{1.277955in}}{\pgfqpoint{2.647485in}{1.270141in}}%
\pgfpathcurveto{\pgfqpoint{2.655299in}{1.262327in}}{\pgfqpoint{2.665898in}{1.257937in}}{\pgfqpoint{2.676948in}{1.257937in}}%
\pgfpathclose%
\pgfusepath{stroke,fill}%
\end{pgfscope}%
\begin{pgfscope}%
\pgfpathrectangle{\pgfqpoint{0.888750in}{0.419100in}}{\pgfqpoint{2.504659in}{2.933700in}} %
\pgfusepath{clip}%
\pgfsetbuttcap%
\pgfsetroundjoin%
\definecolor{currentfill}{rgb}{1.000000,0.549020,0.000000}%
\pgfsetfillcolor{currentfill}%
\pgfsetlinewidth{1.003750pt}%
\definecolor{currentstroke}{rgb}{1.000000,0.549020,0.000000}%
\pgfsetstrokecolor{currentstroke}%
\pgfsetdash{}{0pt}%
\pgfpathmoveto{\pgfqpoint{2.679834in}{1.764030in}}%
\pgfpathcurveto{\pgfqpoint{2.690884in}{1.764030in}}{\pgfqpoint{2.701483in}{1.768421in}}{\pgfqpoint{2.709297in}{1.776234in}}%
\pgfpathcurveto{\pgfqpoint{2.717110in}{1.784048in}}{\pgfqpoint{2.721501in}{1.794647in}}{\pgfqpoint{2.721501in}{1.805697in}}%
\pgfpathcurveto{\pgfqpoint{2.721501in}{1.816747in}}{\pgfqpoint{2.717110in}{1.827346in}}{\pgfqpoint{2.709297in}{1.835160in}}%
\pgfpathcurveto{\pgfqpoint{2.701483in}{1.842974in}}{\pgfqpoint{2.690884in}{1.847364in}}{\pgfqpoint{2.679834in}{1.847364in}}%
\pgfpathcurveto{\pgfqpoint{2.668784in}{1.847364in}}{\pgfqpoint{2.658185in}{1.842974in}}{\pgfqpoint{2.650371in}{1.835160in}}%
\pgfpathcurveto{\pgfqpoint{2.642558in}{1.827346in}}{\pgfqpoint{2.638167in}{1.816747in}}{\pgfqpoint{2.638167in}{1.805697in}}%
\pgfpathcurveto{\pgfqpoint{2.638167in}{1.794647in}}{\pgfqpoint{2.642558in}{1.784048in}}{\pgfqpoint{2.650371in}{1.776234in}}%
\pgfpathcurveto{\pgfqpoint{2.658185in}{1.768421in}}{\pgfqpoint{2.668784in}{1.764030in}}{\pgfqpoint{2.679834in}{1.764030in}}%
\pgfpathclose%
\pgfusepath{stroke,fill}%
\end{pgfscope}%
\begin{pgfscope}%
\pgfpathrectangle{\pgfqpoint{0.888750in}{0.419100in}}{\pgfqpoint{2.504659in}{2.933700in}} %
\pgfusepath{clip}%
\pgfsetbuttcap%
\pgfsetroundjoin%
\definecolor{currentfill}{rgb}{1.000000,0.549020,0.000000}%
\pgfsetfillcolor{currentfill}%
\pgfsetlinewidth{1.003750pt}%
\definecolor{currentstroke}{rgb}{1.000000,0.549020,0.000000}%
\pgfsetstrokecolor{currentstroke}%
\pgfsetdash{}{0pt}%
\pgfpathmoveto{\pgfqpoint{1.757468in}{1.737457in}}%
\pgfpathcurveto{\pgfqpoint{1.768518in}{1.737457in}}{\pgfqpoint{1.779118in}{1.741847in}}{\pgfqpoint{1.786931in}{1.749661in}}%
\pgfpathcurveto{\pgfqpoint{1.794745in}{1.757475in}}{\pgfqpoint{1.799135in}{1.768074in}}{\pgfqpoint{1.799135in}{1.779124in}}%
\pgfpathcurveto{\pgfqpoint{1.799135in}{1.790174in}}{\pgfqpoint{1.794745in}{1.800773in}}{\pgfqpoint{1.786931in}{1.808586in}}%
\pgfpathcurveto{\pgfqpoint{1.779118in}{1.816400in}}{\pgfqpoint{1.768518in}{1.820790in}}{\pgfqpoint{1.757468in}{1.820790in}}%
\pgfpathcurveto{\pgfqpoint{1.746418in}{1.820790in}}{\pgfqpoint{1.735819in}{1.816400in}}{\pgfqpoint{1.728006in}{1.808586in}}%
\pgfpathcurveto{\pgfqpoint{1.720192in}{1.800773in}}{\pgfqpoint{1.715802in}{1.790174in}}{\pgfqpoint{1.715802in}{1.779124in}}%
\pgfpathcurveto{\pgfqpoint{1.715802in}{1.768074in}}{\pgfqpoint{1.720192in}{1.757475in}}{\pgfqpoint{1.728006in}{1.749661in}}%
\pgfpathcurveto{\pgfqpoint{1.735819in}{1.741847in}}{\pgfqpoint{1.746418in}{1.737457in}}{\pgfqpoint{1.757468in}{1.737457in}}%
\pgfpathclose%
\pgfusepath{stroke,fill}%
\end{pgfscope}%
\begin{pgfscope}%
\pgfpathrectangle{\pgfqpoint{0.888750in}{0.419100in}}{\pgfqpoint{2.504659in}{2.933700in}} %
\pgfusepath{clip}%
\pgfsetbuttcap%
\pgfsetroundjoin%
\definecolor{currentfill}{rgb}{1.000000,0.549020,0.000000}%
\pgfsetfillcolor{currentfill}%
\pgfsetlinewidth{1.003750pt}%
\definecolor{currentstroke}{rgb}{1.000000,0.549020,0.000000}%
\pgfsetstrokecolor{currentstroke}%
\pgfsetdash{}{0pt}%
\pgfpathmoveto{\pgfqpoint{1.706195in}{1.812689in}}%
\pgfpathcurveto{\pgfqpoint{1.717245in}{1.812689in}}{\pgfqpoint{1.727844in}{1.817080in}}{\pgfqpoint{1.735658in}{1.824893in}}%
\pgfpathcurveto{\pgfqpoint{1.743471in}{1.832707in}}{\pgfqpoint{1.747862in}{1.843306in}}{\pgfqpoint{1.747862in}{1.854356in}}%
\pgfpathcurveto{\pgfqpoint{1.747862in}{1.865406in}}{\pgfqpoint{1.743471in}{1.876005in}}{\pgfqpoint{1.735658in}{1.883819in}}%
\pgfpathcurveto{\pgfqpoint{1.727844in}{1.891632in}}{\pgfqpoint{1.717245in}{1.896023in}}{\pgfqpoint{1.706195in}{1.896023in}}%
\pgfpathcurveto{\pgfqpoint{1.695145in}{1.896023in}}{\pgfqpoint{1.684546in}{1.891632in}}{\pgfqpoint{1.676732in}{1.883819in}}%
\pgfpathcurveto{\pgfqpoint{1.668918in}{1.876005in}}{\pgfqpoint{1.664528in}{1.865406in}}{\pgfqpoint{1.664528in}{1.854356in}}%
\pgfpathcurveto{\pgfqpoint{1.664528in}{1.843306in}}{\pgfqpoint{1.668918in}{1.832707in}}{\pgfqpoint{1.676732in}{1.824893in}}%
\pgfpathcurveto{\pgfqpoint{1.684546in}{1.817080in}}{\pgfqpoint{1.695145in}{1.812689in}}{\pgfqpoint{1.706195in}{1.812689in}}%
\pgfpathclose%
\pgfusepath{stroke,fill}%
\end{pgfscope}%
\begin{pgfscope}%
\pgfpathrectangle{\pgfqpoint{0.888750in}{0.419100in}}{\pgfqpoint{2.504659in}{2.933700in}} %
\pgfusepath{clip}%
\pgfsetbuttcap%
\pgfsetroundjoin%
\definecolor{currentfill}{rgb}{1.000000,0.549020,0.000000}%
\pgfsetfillcolor{currentfill}%
\pgfsetlinewidth{1.003750pt}%
\definecolor{currentstroke}{rgb}{1.000000,0.549020,0.000000}%
\pgfsetstrokecolor{currentstroke}%
\pgfsetdash{}{0pt}%
\pgfpathmoveto{\pgfqpoint{2.977326in}{1.947823in}}%
\pgfpathcurveto{\pgfqpoint{2.988376in}{1.947823in}}{\pgfqpoint{2.998975in}{1.952213in}}{\pgfqpoint{3.006789in}{1.960027in}}%
\pgfpathcurveto{\pgfqpoint{3.014602in}{1.967840in}}{\pgfqpoint{3.018992in}{1.978439in}}{\pgfqpoint{3.018992in}{1.989490in}}%
\pgfpathcurveto{\pgfqpoint{3.018992in}{2.000540in}}{\pgfqpoint{3.014602in}{2.011139in}}{\pgfqpoint{3.006789in}{2.018952in}}%
\pgfpathcurveto{\pgfqpoint{2.998975in}{2.026766in}}{\pgfqpoint{2.988376in}{2.031156in}}{\pgfqpoint{2.977326in}{2.031156in}}%
\pgfpathcurveto{\pgfqpoint{2.966276in}{2.031156in}}{\pgfqpoint{2.955677in}{2.026766in}}{\pgfqpoint{2.947863in}{2.018952in}}%
\pgfpathcurveto{\pgfqpoint{2.940049in}{2.011139in}}{\pgfqpoint{2.935659in}{2.000540in}}{\pgfqpoint{2.935659in}{1.989490in}}%
\pgfpathcurveto{\pgfqpoint{2.935659in}{1.978439in}}{\pgfqpoint{2.940049in}{1.967840in}}{\pgfqpoint{2.947863in}{1.960027in}}%
\pgfpathcurveto{\pgfqpoint{2.955677in}{1.952213in}}{\pgfqpoint{2.966276in}{1.947823in}}{\pgfqpoint{2.977326in}{1.947823in}}%
\pgfpathclose%
\pgfusepath{stroke,fill}%
\end{pgfscope}%
\begin{pgfscope}%
\pgfpathrectangle{\pgfqpoint{0.888750in}{0.419100in}}{\pgfqpoint{2.504659in}{2.933700in}} %
\pgfusepath{clip}%
\pgfsetbuttcap%
\pgfsetroundjoin%
\definecolor{currentfill}{rgb}{1.000000,0.549020,0.000000}%
\pgfsetfillcolor{currentfill}%
\pgfsetlinewidth{1.003750pt}%
\definecolor{currentstroke}{rgb}{1.000000,0.549020,0.000000}%
\pgfsetstrokecolor{currentstroke}%
\pgfsetdash{}{0pt}%
\pgfpathmoveto{\pgfqpoint{2.705960in}{2.573591in}}%
\pgfpathcurveto{\pgfqpoint{2.717010in}{2.573591in}}{\pgfqpoint{2.727609in}{2.577981in}}{\pgfqpoint{2.735423in}{2.585795in}}%
\pgfpathcurveto{\pgfqpoint{2.743236in}{2.593609in}}{\pgfqpoint{2.747627in}{2.604208in}}{\pgfqpoint{2.747627in}{2.615258in}}%
\pgfpathcurveto{\pgfqpoint{2.747627in}{2.626308in}}{\pgfqpoint{2.743236in}{2.636907in}}{\pgfqpoint{2.735423in}{2.644721in}}%
\pgfpathcurveto{\pgfqpoint{2.727609in}{2.652534in}}{\pgfqpoint{2.717010in}{2.656924in}}{\pgfqpoint{2.705960in}{2.656924in}}%
\pgfpathcurveto{\pgfqpoint{2.694910in}{2.656924in}}{\pgfqpoint{2.684311in}{2.652534in}}{\pgfqpoint{2.676497in}{2.644721in}}%
\pgfpathcurveto{\pgfqpoint{2.668683in}{2.636907in}}{\pgfqpoint{2.664293in}{2.626308in}}{\pgfqpoint{2.664293in}{2.615258in}}%
\pgfpathcurveto{\pgfqpoint{2.664293in}{2.604208in}}{\pgfqpoint{2.668683in}{2.593609in}}{\pgfqpoint{2.676497in}{2.585795in}}%
\pgfpathcurveto{\pgfqpoint{2.684311in}{2.577981in}}{\pgfqpoint{2.694910in}{2.573591in}}{\pgfqpoint{2.705960in}{2.573591in}}%
\pgfpathclose%
\pgfusepath{stroke,fill}%
\end{pgfscope}%
\begin{pgfscope}%
\pgfpathrectangle{\pgfqpoint{0.888750in}{0.419100in}}{\pgfqpoint{2.504659in}{2.933700in}} %
\pgfusepath{clip}%
\pgfsetbuttcap%
\pgfsetroundjoin%
\definecolor{currentfill}{rgb}{1.000000,0.549020,0.000000}%
\pgfsetfillcolor{currentfill}%
\pgfsetlinewidth{1.003750pt}%
\definecolor{currentstroke}{rgb}{1.000000,0.549020,0.000000}%
\pgfsetstrokecolor{currentstroke}%
\pgfsetdash{}{0pt}%
\pgfpathmoveto{\pgfqpoint{2.184339in}{1.785010in}}%
\pgfpathcurveto{\pgfqpoint{2.195389in}{1.785010in}}{\pgfqpoint{2.205988in}{1.789400in}}{\pgfqpoint{2.213801in}{1.797214in}}%
\pgfpathcurveto{\pgfqpoint{2.221615in}{1.805028in}}{\pgfqpoint{2.226005in}{1.815627in}}{\pgfqpoint{2.226005in}{1.826677in}}%
\pgfpathcurveto{\pgfqpoint{2.226005in}{1.837727in}}{\pgfqpoint{2.221615in}{1.848326in}}{\pgfqpoint{2.213801in}{1.856139in}}%
\pgfpathcurveto{\pgfqpoint{2.205988in}{1.863953in}}{\pgfqpoint{2.195389in}{1.868343in}}{\pgfqpoint{2.184339in}{1.868343in}}%
\pgfpathcurveto{\pgfqpoint{2.173289in}{1.868343in}}{\pgfqpoint{2.162690in}{1.863953in}}{\pgfqpoint{2.154876in}{1.856139in}}%
\pgfpathcurveto{\pgfqpoint{2.147062in}{1.848326in}}{\pgfqpoint{2.142672in}{1.837727in}}{\pgfqpoint{2.142672in}{1.826677in}}%
\pgfpathcurveto{\pgfqpoint{2.142672in}{1.815627in}}{\pgfqpoint{2.147062in}{1.805028in}}{\pgfqpoint{2.154876in}{1.797214in}}%
\pgfpathcurveto{\pgfqpoint{2.162690in}{1.789400in}}{\pgfqpoint{2.173289in}{1.785010in}}{\pgfqpoint{2.184339in}{1.785010in}}%
\pgfpathclose%
\pgfusepath{stroke,fill}%
\end{pgfscope}%
\begin{pgfscope}%
\pgfpathrectangle{\pgfqpoint{0.888750in}{0.419100in}}{\pgfqpoint{2.504659in}{2.933700in}} %
\pgfusepath{clip}%
\pgfsetbuttcap%
\pgfsetroundjoin%
\definecolor{currentfill}{rgb}{1.000000,0.549020,0.000000}%
\pgfsetfillcolor{currentfill}%
\pgfsetlinewidth{1.003750pt}%
\definecolor{currentstroke}{rgb}{1.000000,0.549020,0.000000}%
\pgfsetstrokecolor{currentstroke}%
\pgfsetdash{}{0pt}%
\pgfpathmoveto{\pgfqpoint{1.740354in}{1.857784in}}%
\pgfpathcurveto{\pgfqpoint{1.751404in}{1.857784in}}{\pgfqpoint{1.762003in}{1.862174in}}{\pgfqpoint{1.769817in}{1.869988in}}%
\pgfpathcurveto{\pgfqpoint{1.777631in}{1.877801in}}{\pgfqpoint{1.782021in}{1.888400in}}{\pgfqpoint{1.782021in}{1.899450in}}%
\pgfpathcurveto{\pgfqpoint{1.782021in}{1.910501in}}{\pgfqpoint{1.777631in}{1.921100in}}{\pgfqpoint{1.769817in}{1.928913in}}%
\pgfpathcurveto{\pgfqpoint{1.762003in}{1.936727in}}{\pgfqpoint{1.751404in}{1.941117in}}{\pgfqpoint{1.740354in}{1.941117in}}%
\pgfpathcurveto{\pgfqpoint{1.729304in}{1.941117in}}{\pgfqpoint{1.718705in}{1.936727in}}{\pgfqpoint{1.710891in}{1.928913in}}%
\pgfpathcurveto{\pgfqpoint{1.703078in}{1.921100in}}{\pgfqpoint{1.698688in}{1.910501in}}{\pgfqpoint{1.698688in}{1.899450in}}%
\pgfpathcurveto{\pgfqpoint{1.698688in}{1.888400in}}{\pgfqpoint{1.703078in}{1.877801in}}{\pgfqpoint{1.710891in}{1.869988in}}%
\pgfpathcurveto{\pgfqpoint{1.718705in}{1.862174in}}{\pgfqpoint{1.729304in}{1.857784in}}{\pgfqpoint{1.740354in}{1.857784in}}%
\pgfpathclose%
\pgfusepath{stroke,fill}%
\end{pgfscope}%
\begin{pgfscope}%
\pgfpathrectangle{\pgfqpoint{0.888750in}{0.419100in}}{\pgfqpoint{2.504659in}{2.933700in}} %
\pgfusepath{clip}%
\pgfsetbuttcap%
\pgfsetroundjoin%
\definecolor{currentfill}{rgb}{1.000000,0.549020,0.000000}%
\pgfsetfillcolor{currentfill}%
\pgfsetlinewidth{1.003750pt}%
\definecolor{currentstroke}{rgb}{1.000000,0.549020,0.000000}%
\pgfsetstrokecolor{currentstroke}%
\pgfsetdash{}{0pt}%
\pgfpathmoveto{\pgfqpoint{2.140787in}{2.167298in}}%
\pgfpathcurveto{\pgfqpoint{2.151837in}{2.167298in}}{\pgfqpoint{2.162436in}{2.171688in}}{\pgfqpoint{2.170250in}{2.179502in}}%
\pgfpathcurveto{\pgfqpoint{2.178064in}{2.187316in}}{\pgfqpoint{2.182454in}{2.197915in}}{\pgfqpoint{2.182454in}{2.208965in}}%
\pgfpathcurveto{\pgfqpoint{2.182454in}{2.220015in}}{\pgfqpoint{2.178064in}{2.230614in}}{\pgfqpoint{2.170250in}{2.238428in}}%
\pgfpathcurveto{\pgfqpoint{2.162436in}{2.246241in}}{\pgfqpoint{2.151837in}{2.250631in}}{\pgfqpoint{2.140787in}{2.250631in}}%
\pgfpathcurveto{\pgfqpoint{2.129737in}{2.250631in}}{\pgfqpoint{2.119138in}{2.246241in}}{\pgfqpoint{2.111324in}{2.238428in}}%
\pgfpathcurveto{\pgfqpoint{2.103511in}{2.230614in}}{\pgfqpoint{2.099121in}{2.220015in}}{\pgfqpoint{2.099121in}{2.208965in}}%
\pgfpathcurveto{\pgfqpoint{2.099121in}{2.197915in}}{\pgfqpoint{2.103511in}{2.187316in}}{\pgfqpoint{2.111324in}{2.179502in}}%
\pgfpathcurveto{\pgfqpoint{2.119138in}{2.171688in}}{\pgfqpoint{2.129737in}{2.167298in}}{\pgfqpoint{2.140787in}{2.167298in}}%
\pgfpathclose%
\pgfusepath{stroke,fill}%
\end{pgfscope}%
\begin{pgfscope}%
\pgfpathrectangle{\pgfqpoint{0.888750in}{0.419100in}}{\pgfqpoint{2.504659in}{2.933700in}} %
\pgfusepath{clip}%
\pgfsetbuttcap%
\pgfsetroundjoin%
\definecolor{currentfill}{rgb}{1.000000,0.549020,0.000000}%
\pgfsetfillcolor{currentfill}%
\pgfsetlinewidth{1.003750pt}%
\definecolor{currentstroke}{rgb}{1.000000,0.549020,0.000000}%
\pgfsetstrokecolor{currentstroke}%
\pgfsetdash{}{0pt}%
\pgfpathmoveto{\pgfqpoint{2.289341in}{1.834536in}}%
\pgfpathcurveto{\pgfqpoint{2.300391in}{1.834536in}}{\pgfqpoint{2.310990in}{1.838927in}}{\pgfqpoint{2.318804in}{1.846740in}}%
\pgfpathcurveto{\pgfqpoint{2.326617in}{1.854554in}}{\pgfqpoint{2.331007in}{1.865153in}}{\pgfqpoint{2.331007in}{1.876203in}}%
\pgfpathcurveto{\pgfqpoint{2.331007in}{1.887253in}}{\pgfqpoint{2.326617in}{1.897852in}}{\pgfqpoint{2.318804in}{1.905666in}}%
\pgfpathcurveto{\pgfqpoint{2.310990in}{1.913480in}}{\pgfqpoint{2.300391in}{1.917870in}}{\pgfqpoint{2.289341in}{1.917870in}}%
\pgfpathcurveto{\pgfqpoint{2.278291in}{1.917870in}}{\pgfqpoint{2.267692in}{1.913480in}}{\pgfqpoint{2.259878in}{1.905666in}}%
\pgfpathcurveto{\pgfqpoint{2.252064in}{1.897852in}}{\pgfqpoint{2.247674in}{1.887253in}}{\pgfqpoint{2.247674in}{1.876203in}}%
\pgfpathcurveto{\pgfqpoint{2.247674in}{1.865153in}}{\pgfqpoint{2.252064in}{1.854554in}}{\pgfqpoint{2.259878in}{1.846740in}}%
\pgfpathcurveto{\pgfqpoint{2.267692in}{1.838927in}}{\pgfqpoint{2.278291in}{1.834536in}}{\pgfqpoint{2.289341in}{1.834536in}}%
\pgfpathclose%
\pgfusepath{stroke,fill}%
\end{pgfscope}%
\begin{pgfscope}%
\pgfpathrectangle{\pgfqpoint{0.888750in}{0.419100in}}{\pgfqpoint{2.504659in}{2.933700in}} %
\pgfusepath{clip}%
\pgfsetbuttcap%
\pgfsetroundjoin%
\definecolor{currentfill}{rgb}{1.000000,0.549020,0.000000}%
\pgfsetfillcolor{currentfill}%
\pgfsetlinewidth{1.003750pt}%
\definecolor{currentstroke}{rgb}{1.000000,0.549020,0.000000}%
\pgfsetstrokecolor{currentstroke}%
\pgfsetdash{}{0pt}%
\pgfpathmoveto{\pgfqpoint{1.935899in}{1.811549in}}%
\pgfpathcurveto{\pgfqpoint{1.946949in}{1.811549in}}{\pgfqpoint{1.957548in}{1.815940in}}{\pgfqpoint{1.965362in}{1.823753in}}%
\pgfpathcurveto{\pgfqpoint{1.973176in}{1.831567in}}{\pgfqpoint{1.977566in}{1.842166in}}{\pgfqpoint{1.977566in}{1.853216in}}%
\pgfpathcurveto{\pgfqpoint{1.977566in}{1.864266in}}{\pgfqpoint{1.973176in}{1.874865in}}{\pgfqpoint{1.965362in}{1.882679in}}%
\pgfpathcurveto{\pgfqpoint{1.957548in}{1.890492in}}{\pgfqpoint{1.946949in}{1.894883in}}{\pgfqpoint{1.935899in}{1.894883in}}%
\pgfpathcurveto{\pgfqpoint{1.924849in}{1.894883in}}{\pgfqpoint{1.914250in}{1.890492in}}{\pgfqpoint{1.906436in}{1.882679in}}%
\pgfpathcurveto{\pgfqpoint{1.898623in}{1.874865in}}{\pgfqpoint{1.894233in}{1.864266in}}{\pgfqpoint{1.894233in}{1.853216in}}%
\pgfpathcurveto{\pgfqpoint{1.894233in}{1.842166in}}{\pgfqpoint{1.898623in}{1.831567in}}{\pgfqpoint{1.906436in}{1.823753in}}%
\pgfpathcurveto{\pgfqpoint{1.914250in}{1.815940in}}{\pgfqpoint{1.924849in}{1.811549in}}{\pgfqpoint{1.935899in}{1.811549in}}%
\pgfpathclose%
\pgfusepath{stroke,fill}%
\end{pgfscope}%
\begin{pgfscope}%
\pgfpathrectangle{\pgfqpoint{0.888750in}{0.419100in}}{\pgfqpoint{2.504659in}{2.933700in}} %
\pgfusepath{clip}%
\pgfsetbuttcap%
\pgfsetroundjoin%
\definecolor{currentfill}{rgb}{1.000000,0.549020,0.000000}%
\pgfsetfillcolor{currentfill}%
\pgfsetlinewidth{1.003750pt}%
\definecolor{currentstroke}{rgb}{1.000000,0.549020,0.000000}%
\pgfsetstrokecolor{currentstroke}%
\pgfsetdash{}{0pt}%
\pgfpathmoveto{\pgfqpoint{2.348420in}{1.645796in}}%
\pgfpathcurveto{\pgfqpoint{2.359470in}{1.645796in}}{\pgfqpoint{2.370069in}{1.650186in}}{\pgfqpoint{2.377883in}{1.658000in}}%
\pgfpathcurveto{\pgfqpoint{2.385696in}{1.665813in}}{\pgfqpoint{2.390087in}{1.676412in}}{\pgfqpoint{2.390087in}{1.687462in}}%
\pgfpathcurveto{\pgfqpoint{2.390087in}{1.698512in}}{\pgfqpoint{2.385696in}{1.709112in}}{\pgfqpoint{2.377883in}{1.716925in}}%
\pgfpathcurveto{\pgfqpoint{2.370069in}{1.724739in}}{\pgfqpoint{2.359470in}{1.729129in}}{\pgfqpoint{2.348420in}{1.729129in}}%
\pgfpathcurveto{\pgfqpoint{2.337370in}{1.729129in}}{\pgfqpoint{2.326771in}{1.724739in}}{\pgfqpoint{2.318957in}{1.716925in}}%
\pgfpathcurveto{\pgfqpoint{2.311144in}{1.709112in}}{\pgfqpoint{2.306753in}{1.698512in}}{\pgfqpoint{2.306753in}{1.687462in}}%
\pgfpathcurveto{\pgfqpoint{2.306753in}{1.676412in}}{\pgfqpoint{2.311144in}{1.665813in}}{\pgfqpoint{2.318957in}{1.658000in}}%
\pgfpathcurveto{\pgfqpoint{2.326771in}{1.650186in}}{\pgfqpoint{2.337370in}{1.645796in}}{\pgfqpoint{2.348420in}{1.645796in}}%
\pgfpathclose%
\pgfusepath{stroke,fill}%
\end{pgfscope}%
\begin{pgfscope}%
\pgfpathrectangle{\pgfqpoint{0.888750in}{0.419100in}}{\pgfqpoint{2.504659in}{2.933700in}} %
\pgfusepath{clip}%
\pgfsetbuttcap%
\pgfsetroundjoin%
\definecolor{currentfill}{rgb}{1.000000,0.549020,0.000000}%
\pgfsetfillcolor{currentfill}%
\pgfsetlinewidth{1.003750pt}%
\definecolor{currentstroke}{rgb}{1.000000,0.549020,0.000000}%
\pgfsetstrokecolor{currentstroke}%
\pgfsetdash{}{0pt}%
\pgfpathmoveto{\pgfqpoint{1.739879in}{1.410848in}}%
\pgfpathcurveto{\pgfqpoint{1.750929in}{1.410848in}}{\pgfqpoint{1.761529in}{1.415238in}}{\pgfqpoint{1.769342in}{1.423052in}}%
\pgfpathcurveto{\pgfqpoint{1.777156in}{1.430866in}}{\pgfqpoint{1.781546in}{1.441465in}}{\pgfqpoint{1.781546in}{1.452515in}}%
\pgfpathcurveto{\pgfqpoint{1.781546in}{1.463565in}}{\pgfqpoint{1.777156in}{1.474164in}}{\pgfqpoint{1.769342in}{1.481978in}}%
\pgfpathcurveto{\pgfqpoint{1.761529in}{1.489791in}}{\pgfqpoint{1.750929in}{1.494181in}}{\pgfqpoint{1.739879in}{1.494181in}}%
\pgfpathcurveto{\pgfqpoint{1.728829in}{1.494181in}}{\pgfqpoint{1.718230in}{1.489791in}}{\pgfqpoint{1.710417in}{1.481978in}}%
\pgfpathcurveto{\pgfqpoint{1.702603in}{1.474164in}}{\pgfqpoint{1.698213in}{1.463565in}}{\pgfqpoint{1.698213in}{1.452515in}}%
\pgfpathcurveto{\pgfqpoint{1.698213in}{1.441465in}}{\pgfqpoint{1.702603in}{1.430866in}}{\pgfqpoint{1.710417in}{1.423052in}}%
\pgfpathcurveto{\pgfqpoint{1.718230in}{1.415238in}}{\pgfqpoint{1.728829in}{1.410848in}}{\pgfqpoint{1.739879in}{1.410848in}}%
\pgfpathclose%
\pgfusepath{stroke,fill}%
\end{pgfscope}%
\begin{pgfscope}%
\pgfpathrectangle{\pgfqpoint{0.888750in}{0.419100in}}{\pgfqpoint{2.504659in}{2.933700in}} %
\pgfusepath{clip}%
\pgfsetbuttcap%
\pgfsetroundjoin%
\definecolor{currentfill}{rgb}{1.000000,0.549020,0.000000}%
\pgfsetfillcolor{currentfill}%
\pgfsetlinewidth{1.003750pt}%
\definecolor{currentstroke}{rgb}{1.000000,0.549020,0.000000}%
\pgfsetstrokecolor{currentstroke}%
\pgfsetdash{}{0pt}%
\pgfpathmoveto{\pgfqpoint{1.935877in}{1.640280in}}%
\pgfpathcurveto{\pgfqpoint{1.946927in}{1.640280in}}{\pgfqpoint{1.957526in}{1.644671in}}{\pgfqpoint{1.965339in}{1.652484in}}%
\pgfpathcurveto{\pgfqpoint{1.973153in}{1.660298in}}{\pgfqpoint{1.977543in}{1.670897in}}{\pgfqpoint{1.977543in}{1.681947in}}%
\pgfpathcurveto{\pgfqpoint{1.977543in}{1.692997in}}{\pgfqpoint{1.973153in}{1.703596in}}{\pgfqpoint{1.965339in}{1.711410in}}%
\pgfpathcurveto{\pgfqpoint{1.957526in}{1.719223in}}{\pgfqpoint{1.946927in}{1.723614in}}{\pgfqpoint{1.935877in}{1.723614in}}%
\pgfpathcurveto{\pgfqpoint{1.924826in}{1.723614in}}{\pgfqpoint{1.914227in}{1.719223in}}{\pgfqpoint{1.906414in}{1.711410in}}%
\pgfpathcurveto{\pgfqpoint{1.898600in}{1.703596in}}{\pgfqpoint{1.894210in}{1.692997in}}{\pgfqpoint{1.894210in}{1.681947in}}%
\pgfpathcurveto{\pgfqpoint{1.894210in}{1.670897in}}{\pgfqpoint{1.898600in}{1.660298in}}{\pgfqpoint{1.906414in}{1.652484in}}%
\pgfpathcurveto{\pgfqpoint{1.914227in}{1.644671in}}{\pgfqpoint{1.924826in}{1.640280in}}{\pgfqpoint{1.935877in}{1.640280in}}%
\pgfpathclose%
\pgfusepath{stroke,fill}%
\end{pgfscope}%
\begin{pgfscope}%
\pgfpathrectangle{\pgfqpoint{0.888750in}{0.419100in}}{\pgfqpoint{2.504659in}{2.933700in}} %
\pgfusepath{clip}%
\pgfsetbuttcap%
\pgfsetroundjoin%
\definecolor{currentfill}{rgb}{1.000000,0.549020,0.000000}%
\pgfsetfillcolor{currentfill}%
\pgfsetlinewidth{1.003750pt}%
\definecolor{currentstroke}{rgb}{1.000000,0.549020,0.000000}%
\pgfsetstrokecolor{currentstroke}%
\pgfsetdash{}{0pt}%
\pgfpathmoveto{\pgfqpoint{1.918260in}{1.654823in}}%
\pgfpathcurveto{\pgfqpoint{1.929310in}{1.654823in}}{\pgfqpoint{1.939909in}{1.659213in}}{\pgfqpoint{1.947723in}{1.667027in}}%
\pgfpathcurveto{\pgfqpoint{1.955536in}{1.674841in}}{\pgfqpoint{1.959926in}{1.685440in}}{\pgfqpoint{1.959926in}{1.696490in}}%
\pgfpathcurveto{\pgfqpoint{1.959926in}{1.707540in}}{\pgfqpoint{1.955536in}{1.718139in}}{\pgfqpoint{1.947723in}{1.725953in}}%
\pgfpathcurveto{\pgfqpoint{1.939909in}{1.733766in}}{\pgfqpoint{1.929310in}{1.738157in}}{\pgfqpoint{1.918260in}{1.738157in}}%
\pgfpathcurveto{\pgfqpoint{1.907210in}{1.738157in}}{\pgfqpoint{1.896611in}{1.733766in}}{\pgfqpoint{1.888797in}{1.725953in}}%
\pgfpathcurveto{\pgfqpoint{1.880983in}{1.718139in}}{\pgfqpoint{1.876593in}{1.707540in}}{\pgfqpoint{1.876593in}{1.696490in}}%
\pgfpathcurveto{\pgfqpoint{1.876593in}{1.685440in}}{\pgfqpoint{1.880983in}{1.674841in}}{\pgfqpoint{1.888797in}{1.667027in}}%
\pgfpathcurveto{\pgfqpoint{1.896611in}{1.659213in}}{\pgfqpoint{1.907210in}{1.654823in}}{\pgfqpoint{1.918260in}{1.654823in}}%
\pgfpathclose%
\pgfusepath{stroke,fill}%
\end{pgfscope}%
\begin{pgfscope}%
\pgfpathrectangle{\pgfqpoint{0.888750in}{0.419100in}}{\pgfqpoint{2.504659in}{2.933700in}} %
\pgfusepath{clip}%
\pgfsetbuttcap%
\pgfsetroundjoin%
\definecolor{currentfill}{rgb}{1.000000,0.549020,0.000000}%
\pgfsetfillcolor{currentfill}%
\pgfsetlinewidth{1.003750pt}%
\definecolor{currentstroke}{rgb}{1.000000,0.549020,0.000000}%
\pgfsetstrokecolor{currentstroke}%
\pgfsetdash{}{0pt}%
\pgfpathmoveto{\pgfqpoint{1.757873in}{1.988857in}}%
\pgfpathcurveto{\pgfqpoint{1.768923in}{1.988857in}}{\pgfqpoint{1.779522in}{1.993247in}}{\pgfqpoint{1.787336in}{2.001060in}}%
\pgfpathcurveto{\pgfqpoint{1.795149in}{2.008874in}}{\pgfqpoint{1.799540in}{2.019473in}}{\pgfqpoint{1.799540in}{2.030523in}}%
\pgfpathcurveto{\pgfqpoint{1.799540in}{2.041573in}}{\pgfqpoint{1.795149in}{2.052172in}}{\pgfqpoint{1.787336in}{2.059986in}}%
\pgfpathcurveto{\pgfqpoint{1.779522in}{2.067800in}}{\pgfqpoint{1.768923in}{2.072190in}}{\pgfqpoint{1.757873in}{2.072190in}}%
\pgfpathcurveto{\pgfqpoint{1.746823in}{2.072190in}}{\pgfqpoint{1.736224in}{2.067800in}}{\pgfqpoint{1.728410in}{2.059986in}}%
\pgfpathcurveto{\pgfqpoint{1.720597in}{2.052172in}}{\pgfqpoint{1.716206in}{2.041573in}}{\pgfqpoint{1.716206in}{2.030523in}}%
\pgfpathcurveto{\pgfqpoint{1.716206in}{2.019473in}}{\pgfqpoint{1.720597in}{2.008874in}}{\pgfqpoint{1.728410in}{2.001060in}}%
\pgfpathcurveto{\pgfqpoint{1.736224in}{1.993247in}}{\pgfqpoint{1.746823in}{1.988857in}}{\pgfqpoint{1.757873in}{1.988857in}}%
\pgfpathclose%
\pgfusepath{stroke,fill}%
\end{pgfscope}%
\begin{pgfscope}%
\pgfpathrectangle{\pgfqpoint{0.888750in}{0.419100in}}{\pgfqpoint{2.504659in}{2.933700in}} %
\pgfusepath{clip}%
\pgfsetbuttcap%
\pgfsetroundjoin%
\definecolor{currentfill}{rgb}{1.000000,0.549020,0.000000}%
\pgfsetfillcolor{currentfill}%
\pgfsetlinewidth{1.003750pt}%
\definecolor{currentstroke}{rgb}{1.000000,0.549020,0.000000}%
\pgfsetstrokecolor{currentstroke}%
\pgfsetdash{}{0pt}%
\pgfpathmoveto{\pgfqpoint{2.427870in}{2.423662in}}%
\pgfpathcurveto{\pgfqpoint{2.438920in}{2.423662in}}{\pgfqpoint{2.449519in}{2.428052in}}{\pgfqpoint{2.457333in}{2.435866in}}%
\pgfpathcurveto{\pgfqpoint{2.465147in}{2.443680in}}{\pgfqpoint{2.469537in}{2.454279in}}{\pgfqpoint{2.469537in}{2.465329in}}%
\pgfpathcurveto{\pgfqpoint{2.469537in}{2.476379in}}{\pgfqpoint{2.465147in}{2.486978in}}{\pgfqpoint{2.457333in}{2.494792in}}%
\pgfpathcurveto{\pgfqpoint{2.449519in}{2.502605in}}{\pgfqpoint{2.438920in}{2.506995in}}{\pgfqpoint{2.427870in}{2.506995in}}%
\pgfpathcurveto{\pgfqpoint{2.416820in}{2.506995in}}{\pgfqpoint{2.406221in}{2.502605in}}{\pgfqpoint{2.398408in}{2.494792in}}%
\pgfpathcurveto{\pgfqpoint{2.390594in}{2.486978in}}{\pgfqpoint{2.386204in}{2.476379in}}{\pgfqpoint{2.386204in}{2.465329in}}%
\pgfpathcurveto{\pgfqpoint{2.386204in}{2.454279in}}{\pgfqpoint{2.390594in}{2.443680in}}{\pgfqpoint{2.398408in}{2.435866in}}%
\pgfpathcurveto{\pgfqpoint{2.406221in}{2.428052in}}{\pgfqpoint{2.416820in}{2.423662in}}{\pgfqpoint{2.427870in}{2.423662in}}%
\pgfpathclose%
\pgfusepath{stroke,fill}%
\end{pgfscope}%
\begin{pgfscope}%
\pgfpathrectangle{\pgfqpoint{0.888750in}{0.419100in}}{\pgfqpoint{2.504659in}{2.933700in}} %
\pgfusepath{clip}%
\pgfsetbuttcap%
\pgfsetroundjoin%
\definecolor{currentfill}{rgb}{1.000000,0.549020,0.000000}%
\pgfsetfillcolor{currentfill}%
\pgfsetlinewidth{1.003750pt}%
\definecolor{currentstroke}{rgb}{1.000000,0.549020,0.000000}%
\pgfsetstrokecolor{currentstroke}%
\pgfsetdash{}{0pt}%
\pgfpathmoveto{\pgfqpoint{1.809658in}{1.902277in}}%
\pgfpathcurveto{\pgfqpoint{1.820708in}{1.902277in}}{\pgfqpoint{1.831307in}{1.906667in}}{\pgfqpoint{1.839121in}{1.914481in}}%
\pgfpathcurveto{\pgfqpoint{1.846934in}{1.922294in}}{\pgfqpoint{1.851324in}{1.932893in}}{\pgfqpoint{1.851324in}{1.943944in}}%
\pgfpathcurveto{\pgfqpoint{1.851324in}{1.954994in}}{\pgfqpoint{1.846934in}{1.965593in}}{\pgfqpoint{1.839121in}{1.973406in}}%
\pgfpathcurveto{\pgfqpoint{1.831307in}{1.981220in}}{\pgfqpoint{1.820708in}{1.985610in}}{\pgfqpoint{1.809658in}{1.985610in}}%
\pgfpathcurveto{\pgfqpoint{1.798608in}{1.985610in}}{\pgfqpoint{1.788009in}{1.981220in}}{\pgfqpoint{1.780195in}{1.973406in}}%
\pgfpathcurveto{\pgfqpoint{1.772381in}{1.965593in}}{\pgfqpoint{1.767991in}{1.954994in}}{\pgfqpoint{1.767991in}{1.943944in}}%
\pgfpathcurveto{\pgfqpoint{1.767991in}{1.932893in}}{\pgfqpoint{1.772381in}{1.922294in}}{\pgfqpoint{1.780195in}{1.914481in}}%
\pgfpathcurveto{\pgfqpoint{1.788009in}{1.906667in}}{\pgfqpoint{1.798608in}{1.902277in}}{\pgfqpoint{1.809658in}{1.902277in}}%
\pgfpathclose%
\pgfusepath{stroke,fill}%
\end{pgfscope}%
\begin{pgfscope}%
\pgfpathrectangle{\pgfqpoint{0.888750in}{0.419100in}}{\pgfqpoint{2.504659in}{2.933700in}} %
\pgfusepath{clip}%
\pgfsetbuttcap%
\pgfsetroundjoin%
\definecolor{currentfill}{rgb}{1.000000,0.549020,0.000000}%
\pgfsetfillcolor{currentfill}%
\pgfsetlinewidth{1.003750pt}%
\definecolor{currentstroke}{rgb}{1.000000,0.549020,0.000000}%
\pgfsetstrokecolor{currentstroke}%
\pgfsetdash{}{0pt}%
\pgfpathmoveto{\pgfqpoint{1.993073in}{1.643572in}}%
\pgfpathcurveto{\pgfqpoint{2.004123in}{1.643572in}}{\pgfqpoint{2.014722in}{1.647963in}}{\pgfqpoint{2.022536in}{1.655776in}}%
\pgfpathcurveto{\pgfqpoint{2.030349in}{1.663590in}}{\pgfqpoint{2.034740in}{1.674189in}}{\pgfqpoint{2.034740in}{1.685239in}}%
\pgfpathcurveto{\pgfqpoint{2.034740in}{1.696289in}}{\pgfqpoint{2.030349in}{1.706888in}}{\pgfqpoint{2.022536in}{1.714702in}}%
\pgfpathcurveto{\pgfqpoint{2.014722in}{1.722516in}}{\pgfqpoint{2.004123in}{1.726906in}}{\pgfqpoint{1.993073in}{1.726906in}}%
\pgfpathcurveto{\pgfqpoint{1.982023in}{1.726906in}}{\pgfqpoint{1.971424in}{1.722516in}}{\pgfqpoint{1.963610in}{1.714702in}}%
\pgfpathcurveto{\pgfqpoint{1.955797in}{1.706888in}}{\pgfqpoint{1.951406in}{1.696289in}}{\pgfqpoint{1.951406in}{1.685239in}}%
\pgfpathcurveto{\pgfqpoint{1.951406in}{1.674189in}}{\pgfqpoint{1.955797in}{1.663590in}}{\pgfqpoint{1.963610in}{1.655776in}}%
\pgfpathcurveto{\pgfqpoint{1.971424in}{1.647963in}}{\pgfqpoint{1.982023in}{1.643572in}}{\pgfqpoint{1.993073in}{1.643572in}}%
\pgfpathclose%
\pgfusepath{stroke,fill}%
\end{pgfscope}%
\begin{pgfscope}%
\pgfpathrectangle{\pgfqpoint{0.888750in}{0.419100in}}{\pgfqpoint{2.504659in}{2.933700in}} %
\pgfusepath{clip}%
\pgfsetbuttcap%
\pgfsetroundjoin%
\definecolor{currentfill}{rgb}{1.000000,0.549020,0.000000}%
\pgfsetfillcolor{currentfill}%
\pgfsetlinewidth{1.003750pt}%
\definecolor{currentstroke}{rgb}{1.000000,0.549020,0.000000}%
\pgfsetstrokecolor{currentstroke}%
\pgfsetdash{}{0pt}%
\pgfpathmoveto{\pgfqpoint{1.628852in}{2.160545in}}%
\pgfpathcurveto{\pgfqpoint{1.639903in}{2.160545in}}{\pgfqpoint{1.650502in}{2.164935in}}{\pgfqpoint{1.658315in}{2.172748in}}%
\pgfpathcurveto{\pgfqpoint{1.666129in}{2.180562in}}{\pgfqpoint{1.670519in}{2.191161in}}{\pgfqpoint{1.670519in}{2.202211in}}%
\pgfpathcurveto{\pgfqpoint{1.670519in}{2.213261in}}{\pgfqpoint{1.666129in}{2.223860in}}{\pgfqpoint{1.658315in}{2.231674in}}%
\pgfpathcurveto{\pgfqpoint{1.650502in}{2.239488in}}{\pgfqpoint{1.639903in}{2.243878in}}{\pgfqpoint{1.628852in}{2.243878in}}%
\pgfpathcurveto{\pgfqpoint{1.617802in}{2.243878in}}{\pgfqpoint{1.607203in}{2.239488in}}{\pgfqpoint{1.599390in}{2.231674in}}%
\pgfpathcurveto{\pgfqpoint{1.591576in}{2.223860in}}{\pgfqpoint{1.587186in}{2.213261in}}{\pgfqpoint{1.587186in}{2.202211in}}%
\pgfpathcurveto{\pgfqpoint{1.587186in}{2.191161in}}{\pgfqpoint{1.591576in}{2.180562in}}{\pgfqpoint{1.599390in}{2.172748in}}%
\pgfpathcurveto{\pgfqpoint{1.607203in}{2.164935in}}{\pgfqpoint{1.617802in}{2.160545in}}{\pgfqpoint{1.628852in}{2.160545in}}%
\pgfpathclose%
\pgfusepath{stroke,fill}%
\end{pgfscope}%
\begin{pgfscope}%
\pgfpathrectangle{\pgfqpoint{0.888750in}{0.419100in}}{\pgfqpoint{2.504659in}{2.933700in}} %
\pgfusepath{clip}%
\pgfsetbuttcap%
\pgfsetroundjoin%
\definecolor{currentfill}{rgb}{1.000000,0.549020,0.000000}%
\pgfsetfillcolor{currentfill}%
\pgfsetlinewidth{1.003750pt}%
\definecolor{currentstroke}{rgb}{1.000000,0.549020,0.000000}%
\pgfsetstrokecolor{currentstroke}%
\pgfsetdash{}{0pt}%
\pgfpathmoveto{\pgfqpoint{2.468455in}{1.415949in}}%
\pgfpathcurveto{\pgfqpoint{2.479505in}{1.415949in}}{\pgfqpoint{2.490104in}{1.420339in}}{\pgfqpoint{2.497918in}{1.428152in}}%
\pgfpathcurveto{\pgfqpoint{2.505731in}{1.435966in}}{\pgfqpoint{2.510121in}{1.446565in}}{\pgfqpoint{2.510121in}{1.457615in}}%
\pgfpathcurveto{\pgfqpoint{2.510121in}{1.468665in}}{\pgfqpoint{2.505731in}{1.479264in}}{\pgfqpoint{2.497918in}{1.487078in}}%
\pgfpathcurveto{\pgfqpoint{2.490104in}{1.494892in}}{\pgfqpoint{2.479505in}{1.499282in}}{\pgfqpoint{2.468455in}{1.499282in}}%
\pgfpathcurveto{\pgfqpoint{2.457405in}{1.499282in}}{\pgfqpoint{2.446806in}{1.494892in}}{\pgfqpoint{2.438992in}{1.487078in}}%
\pgfpathcurveto{\pgfqpoint{2.431178in}{1.479264in}}{\pgfqpoint{2.426788in}{1.468665in}}{\pgfqpoint{2.426788in}{1.457615in}}%
\pgfpathcurveto{\pgfqpoint{2.426788in}{1.446565in}}{\pgfqpoint{2.431178in}{1.435966in}}{\pgfqpoint{2.438992in}{1.428152in}}%
\pgfpathcurveto{\pgfqpoint{2.446806in}{1.420339in}}{\pgfqpoint{2.457405in}{1.415949in}}{\pgfqpoint{2.468455in}{1.415949in}}%
\pgfpathclose%
\pgfusepath{stroke,fill}%
\end{pgfscope}%
\begin{pgfscope}%
\pgfpathrectangle{\pgfqpoint{0.888750in}{0.419100in}}{\pgfqpoint{2.504659in}{2.933700in}} %
\pgfusepath{clip}%
\pgfsetbuttcap%
\pgfsetroundjoin%
\definecolor{currentfill}{rgb}{1.000000,0.549020,0.000000}%
\pgfsetfillcolor{currentfill}%
\pgfsetlinewidth{1.003750pt}%
\definecolor{currentstroke}{rgb}{1.000000,0.549020,0.000000}%
\pgfsetstrokecolor{currentstroke}%
\pgfsetdash{}{0pt}%
\pgfpathmoveto{\pgfqpoint{2.115504in}{1.278778in}}%
\pgfpathcurveto{\pgfqpoint{2.126555in}{1.278778in}}{\pgfqpoint{2.137154in}{1.283168in}}{\pgfqpoint{2.144967in}{1.290981in}}%
\pgfpathcurveto{\pgfqpoint{2.152781in}{1.298795in}}{\pgfqpoint{2.157171in}{1.309394in}}{\pgfqpoint{2.157171in}{1.320444in}}%
\pgfpathcurveto{\pgfqpoint{2.157171in}{1.331494in}}{\pgfqpoint{2.152781in}{1.342093in}}{\pgfqpoint{2.144967in}{1.349907in}}%
\pgfpathcurveto{\pgfqpoint{2.137154in}{1.357721in}}{\pgfqpoint{2.126555in}{1.362111in}}{\pgfqpoint{2.115504in}{1.362111in}}%
\pgfpathcurveto{\pgfqpoint{2.104454in}{1.362111in}}{\pgfqpoint{2.093855in}{1.357721in}}{\pgfqpoint{2.086042in}{1.349907in}}%
\pgfpathcurveto{\pgfqpoint{2.078228in}{1.342093in}}{\pgfqpoint{2.073838in}{1.331494in}}{\pgfqpoint{2.073838in}{1.320444in}}%
\pgfpathcurveto{\pgfqpoint{2.073838in}{1.309394in}}{\pgfqpoint{2.078228in}{1.298795in}}{\pgfqpoint{2.086042in}{1.290981in}}%
\pgfpathcurveto{\pgfqpoint{2.093855in}{1.283168in}}{\pgfqpoint{2.104454in}{1.278778in}}{\pgfqpoint{2.115504in}{1.278778in}}%
\pgfpathclose%
\pgfusepath{stroke,fill}%
\end{pgfscope}%
\begin{pgfscope}%
\pgfpathrectangle{\pgfqpoint{0.888750in}{0.419100in}}{\pgfqpoint{2.504659in}{2.933700in}} %
\pgfusepath{clip}%
\pgfsetbuttcap%
\pgfsetroundjoin%
\definecolor{currentfill}{rgb}{1.000000,0.549020,0.000000}%
\pgfsetfillcolor{currentfill}%
\pgfsetlinewidth{1.003750pt}%
\definecolor{currentstroke}{rgb}{1.000000,0.549020,0.000000}%
\pgfsetstrokecolor{currentstroke}%
\pgfsetdash{}{0pt}%
\pgfpathmoveto{\pgfqpoint{2.284822in}{1.578333in}}%
\pgfpathcurveto{\pgfqpoint{2.295872in}{1.578333in}}{\pgfqpoint{2.306471in}{1.582723in}}{\pgfqpoint{2.314285in}{1.590537in}}%
\pgfpathcurveto{\pgfqpoint{2.322098in}{1.598351in}}{\pgfqpoint{2.326488in}{1.608950in}}{\pgfqpoint{2.326488in}{1.620000in}}%
\pgfpathcurveto{\pgfqpoint{2.326488in}{1.631050in}}{\pgfqpoint{2.322098in}{1.641649in}}{\pgfqpoint{2.314285in}{1.649463in}}%
\pgfpathcurveto{\pgfqpoint{2.306471in}{1.657276in}}{\pgfqpoint{2.295872in}{1.661666in}}{\pgfqpoint{2.284822in}{1.661666in}}%
\pgfpathcurveto{\pgfqpoint{2.273772in}{1.661666in}}{\pgfqpoint{2.263173in}{1.657276in}}{\pgfqpoint{2.255359in}{1.649463in}}%
\pgfpathcurveto{\pgfqpoint{2.247545in}{1.641649in}}{\pgfqpoint{2.243155in}{1.631050in}}{\pgfqpoint{2.243155in}{1.620000in}}%
\pgfpathcurveto{\pgfqpoint{2.243155in}{1.608950in}}{\pgfqpoint{2.247545in}{1.598351in}}{\pgfqpoint{2.255359in}{1.590537in}}%
\pgfpathcurveto{\pgfqpoint{2.263173in}{1.582723in}}{\pgfqpoint{2.273772in}{1.578333in}}{\pgfqpoint{2.284822in}{1.578333in}}%
\pgfpathclose%
\pgfusepath{stroke,fill}%
\end{pgfscope}%
\begin{pgfscope}%
\pgfpathrectangle{\pgfqpoint{0.888750in}{0.419100in}}{\pgfqpoint{2.504659in}{2.933700in}} %
\pgfusepath{clip}%
\pgfsetbuttcap%
\pgfsetroundjoin%
\definecolor{currentfill}{rgb}{1.000000,0.549020,0.000000}%
\pgfsetfillcolor{currentfill}%
\pgfsetlinewidth{1.003750pt}%
\definecolor{currentstroke}{rgb}{1.000000,0.549020,0.000000}%
\pgfsetstrokecolor{currentstroke}%
\pgfsetdash{}{0pt}%
\pgfpathmoveto{\pgfqpoint{1.805155in}{1.620035in}}%
\pgfpathcurveto{\pgfqpoint{1.816205in}{1.620035in}}{\pgfqpoint{1.826804in}{1.624425in}}{\pgfqpoint{1.834618in}{1.632239in}}%
\pgfpathcurveto{\pgfqpoint{1.842432in}{1.640052in}}{\pgfqpoint{1.846822in}{1.650651in}}{\pgfqpoint{1.846822in}{1.661702in}}%
\pgfpathcurveto{\pgfqpoint{1.846822in}{1.672752in}}{\pgfqpoint{1.842432in}{1.683351in}}{\pgfqpoint{1.834618in}{1.691164in}}%
\pgfpathcurveto{\pgfqpoint{1.826804in}{1.698978in}}{\pgfqpoint{1.816205in}{1.703368in}}{\pgfqpoint{1.805155in}{1.703368in}}%
\pgfpathcurveto{\pgfqpoint{1.794105in}{1.703368in}}{\pgfqpoint{1.783506in}{1.698978in}}{\pgfqpoint{1.775693in}{1.691164in}}%
\pgfpathcurveto{\pgfqpoint{1.767879in}{1.683351in}}{\pgfqpoint{1.763489in}{1.672752in}}{\pgfqpoint{1.763489in}{1.661702in}}%
\pgfpathcurveto{\pgfqpoint{1.763489in}{1.650651in}}{\pgfqpoint{1.767879in}{1.640052in}}{\pgfqpoint{1.775693in}{1.632239in}}%
\pgfpathcurveto{\pgfqpoint{1.783506in}{1.624425in}}{\pgfqpoint{1.794105in}{1.620035in}}{\pgfqpoint{1.805155in}{1.620035in}}%
\pgfpathclose%
\pgfusepath{stroke,fill}%
\end{pgfscope}%
\begin{pgfscope}%
\pgfpathrectangle{\pgfqpoint{0.888750in}{0.419100in}}{\pgfqpoint{2.504659in}{2.933700in}} %
\pgfusepath{clip}%
\pgfsetbuttcap%
\pgfsetroundjoin%
\definecolor{currentfill}{rgb}{1.000000,0.549020,0.000000}%
\pgfsetfillcolor{currentfill}%
\pgfsetlinewidth{1.003750pt}%
\definecolor{currentstroke}{rgb}{1.000000,0.549020,0.000000}%
\pgfsetstrokecolor{currentstroke}%
\pgfsetdash{}{0pt}%
\pgfpathmoveto{\pgfqpoint{2.362986in}{1.892177in}}%
\pgfpathcurveto{\pgfqpoint{2.374036in}{1.892177in}}{\pgfqpoint{2.384635in}{1.896568in}}{\pgfqpoint{2.392449in}{1.904381in}}%
\pgfpathcurveto{\pgfqpoint{2.400262in}{1.912195in}}{\pgfqpoint{2.404652in}{1.922794in}}{\pgfqpoint{2.404652in}{1.933844in}}%
\pgfpathcurveto{\pgfqpoint{2.404652in}{1.944894in}}{\pgfqpoint{2.400262in}{1.955493in}}{\pgfqpoint{2.392449in}{1.963307in}}%
\pgfpathcurveto{\pgfqpoint{2.384635in}{1.971120in}}{\pgfqpoint{2.374036in}{1.975511in}}{\pgfqpoint{2.362986in}{1.975511in}}%
\pgfpathcurveto{\pgfqpoint{2.351936in}{1.975511in}}{\pgfqpoint{2.341337in}{1.971120in}}{\pgfqpoint{2.333523in}{1.963307in}}%
\pgfpathcurveto{\pgfqpoint{2.325709in}{1.955493in}}{\pgfqpoint{2.321319in}{1.944894in}}{\pgfqpoint{2.321319in}{1.933844in}}%
\pgfpathcurveto{\pgfqpoint{2.321319in}{1.922794in}}{\pgfqpoint{2.325709in}{1.912195in}}{\pgfqpoint{2.333523in}{1.904381in}}%
\pgfpathcurveto{\pgfqpoint{2.341337in}{1.896568in}}{\pgfqpoint{2.351936in}{1.892177in}}{\pgfqpoint{2.362986in}{1.892177in}}%
\pgfpathclose%
\pgfusepath{stroke,fill}%
\end{pgfscope}%
\begin{pgfscope}%
\pgfpathrectangle{\pgfqpoint{0.888750in}{0.419100in}}{\pgfqpoint{2.504659in}{2.933700in}} %
\pgfusepath{clip}%
\pgfsetbuttcap%
\pgfsetroundjoin%
\definecolor{currentfill}{rgb}{0.400000,0.600000,0.800000}%
\pgfsetfillcolor{currentfill}%
\pgfsetlinewidth{1.003750pt}%
\definecolor{currentstroke}{rgb}{0.400000,0.600000,0.800000}%
\pgfsetstrokecolor{currentstroke}%
\pgfsetdash{}{0pt}%
\pgfpathmoveto{\pgfqpoint{3.083070in}{1.814271in}}%
\pgfpathcurveto{\pgfqpoint{3.094120in}{1.814271in}}{\pgfqpoint{3.104719in}{1.818661in}}{\pgfqpoint{3.112532in}{1.826475in}}%
\pgfpathcurveto{\pgfqpoint{3.120346in}{1.834289in}}{\pgfqpoint{3.124736in}{1.844888in}}{\pgfqpoint{3.124736in}{1.855938in}}%
\pgfpathcurveto{\pgfqpoint{3.124736in}{1.866988in}}{\pgfqpoint{3.120346in}{1.877587in}}{\pgfqpoint{3.112532in}{1.885401in}}%
\pgfpathcurveto{\pgfqpoint{3.104719in}{1.893214in}}{\pgfqpoint{3.094120in}{1.897604in}}{\pgfqpoint{3.083070in}{1.897604in}}%
\pgfpathcurveto{\pgfqpoint{3.072019in}{1.897604in}}{\pgfqpoint{3.061420in}{1.893214in}}{\pgfqpoint{3.053607in}{1.885401in}}%
\pgfpathcurveto{\pgfqpoint{3.045793in}{1.877587in}}{\pgfqpoint{3.041403in}{1.866988in}}{\pgfqpoint{3.041403in}{1.855938in}}%
\pgfpathcurveto{\pgfqpoint{3.041403in}{1.844888in}}{\pgfqpoint{3.045793in}{1.834289in}}{\pgfqpoint{3.053607in}{1.826475in}}%
\pgfpathcurveto{\pgfqpoint{3.061420in}{1.818661in}}{\pgfqpoint{3.072019in}{1.814271in}}{\pgfqpoint{3.083070in}{1.814271in}}%
\pgfpathclose%
\pgfusepath{stroke,fill}%
\end{pgfscope}%
\begin{pgfscope}%
\pgfpathrectangle{\pgfqpoint{0.888750in}{0.419100in}}{\pgfqpoint{2.504659in}{2.933700in}} %
\pgfusepath{clip}%
\pgfsetbuttcap%
\pgfsetroundjoin%
\definecolor{currentfill}{rgb}{0.400000,0.600000,0.800000}%
\pgfsetfillcolor{currentfill}%
\pgfsetlinewidth{1.003750pt}%
\definecolor{currentstroke}{rgb}{0.400000,0.600000,0.800000}%
\pgfsetstrokecolor{currentstroke}%
\pgfsetdash{}{0pt}%
\pgfpathmoveto{\pgfqpoint{2.769595in}{2.654615in}}%
\pgfpathcurveto{\pgfqpoint{2.780646in}{2.654615in}}{\pgfqpoint{2.791245in}{2.659005in}}{\pgfqpoint{2.799058in}{2.666819in}}%
\pgfpathcurveto{\pgfqpoint{2.806872in}{2.674632in}}{\pgfqpoint{2.811262in}{2.685231in}}{\pgfqpoint{2.811262in}{2.696281in}}%
\pgfpathcurveto{\pgfqpoint{2.811262in}{2.707332in}}{\pgfqpoint{2.806872in}{2.717931in}}{\pgfqpoint{2.799058in}{2.725744in}}%
\pgfpathcurveto{\pgfqpoint{2.791245in}{2.733558in}}{\pgfqpoint{2.780646in}{2.737948in}}{\pgfqpoint{2.769595in}{2.737948in}}%
\pgfpathcurveto{\pgfqpoint{2.758545in}{2.737948in}}{\pgfqpoint{2.747946in}{2.733558in}}{\pgfqpoint{2.740133in}{2.725744in}}%
\pgfpathcurveto{\pgfqpoint{2.732319in}{2.717931in}}{\pgfqpoint{2.727929in}{2.707332in}}{\pgfqpoint{2.727929in}{2.696281in}}%
\pgfpathcurveto{\pgfqpoint{2.727929in}{2.685231in}}{\pgfqpoint{2.732319in}{2.674632in}}{\pgfqpoint{2.740133in}{2.666819in}}%
\pgfpathcurveto{\pgfqpoint{2.747946in}{2.659005in}}{\pgfqpoint{2.758545in}{2.654615in}}{\pgfqpoint{2.769595in}{2.654615in}}%
\pgfpathclose%
\pgfusepath{stroke,fill}%
\end{pgfscope}%
\begin{pgfscope}%
\pgfpathrectangle{\pgfqpoint{0.888750in}{0.419100in}}{\pgfqpoint{2.504659in}{2.933700in}} %
\pgfusepath{clip}%
\pgfsetbuttcap%
\pgfsetroundjoin%
\definecolor{currentfill}{rgb}{0.400000,0.600000,0.800000}%
\pgfsetfillcolor{currentfill}%
\pgfsetlinewidth{1.003750pt}%
\definecolor{currentstroke}{rgb}{0.400000,0.600000,0.800000}%
\pgfsetstrokecolor{currentstroke}%
\pgfsetdash{}{0pt}%
\pgfpathmoveto{\pgfqpoint{2.456121in}{2.875955in}}%
\pgfpathcurveto{\pgfqpoint{2.467171in}{2.875955in}}{\pgfqpoint{2.477770in}{2.880346in}}{\pgfqpoint{2.485584in}{2.888159in}}%
\pgfpathcurveto{\pgfqpoint{2.493398in}{2.895973in}}{\pgfqpoint{2.497788in}{2.906572in}}{\pgfqpoint{2.497788in}{2.917622in}}%
\pgfpathcurveto{\pgfqpoint{2.497788in}{2.928672in}}{\pgfqpoint{2.493398in}{2.939271in}}{\pgfqpoint{2.485584in}{2.947085in}}%
\pgfpathcurveto{\pgfqpoint{2.477770in}{2.954898in}}{\pgfqpoint{2.467171in}{2.959289in}}{\pgfqpoint{2.456121in}{2.959289in}}%
\pgfpathcurveto{\pgfqpoint{2.445071in}{2.959289in}}{\pgfqpoint{2.434472in}{2.954898in}}{\pgfqpoint{2.426658in}{2.947085in}}%
\pgfpathcurveto{\pgfqpoint{2.418845in}{2.939271in}}{\pgfqpoint{2.414454in}{2.928672in}}{\pgfqpoint{2.414454in}{2.917622in}}%
\pgfpathcurveto{\pgfqpoint{2.414454in}{2.906572in}}{\pgfqpoint{2.418845in}{2.895973in}}{\pgfqpoint{2.426658in}{2.888159in}}%
\pgfpathcurveto{\pgfqpoint{2.434472in}{2.880346in}}{\pgfqpoint{2.445071in}{2.875955in}}{\pgfqpoint{2.456121in}{2.875955in}}%
\pgfpathclose%
\pgfusepath{stroke,fill}%
\end{pgfscope}%
\begin{pgfscope}%
\pgfpathrectangle{\pgfqpoint{0.888750in}{0.419100in}}{\pgfqpoint{2.504659in}{2.933700in}} %
\pgfusepath{clip}%
\pgfsetbuttcap%
\pgfsetroundjoin%
\definecolor{currentfill}{rgb}{0.400000,0.600000,0.800000}%
\pgfsetfillcolor{currentfill}%
\pgfsetlinewidth{1.003750pt}%
\definecolor{currentstroke}{rgb}{0.400000,0.600000,0.800000}%
\pgfsetstrokecolor{currentstroke}%
\pgfsetdash{}{0pt}%
\pgfpathmoveto{\pgfqpoint{2.142647in}{2.939731in}}%
\pgfpathcurveto{\pgfqpoint{2.153697in}{2.939731in}}{\pgfqpoint{2.164296in}{2.944122in}}{\pgfqpoint{2.172110in}{2.951935in}}%
\pgfpathcurveto{\pgfqpoint{2.179923in}{2.959749in}}{\pgfqpoint{2.184314in}{2.970348in}}{\pgfqpoint{2.184314in}{2.981398in}}%
\pgfpathcurveto{\pgfqpoint{2.184314in}{2.992448in}}{\pgfqpoint{2.179923in}{3.003047in}}{\pgfqpoint{2.172110in}{3.010861in}}%
\pgfpathcurveto{\pgfqpoint{2.164296in}{3.018674in}}{\pgfqpoint{2.153697in}{3.023065in}}{\pgfqpoint{2.142647in}{3.023065in}}%
\pgfpathcurveto{\pgfqpoint{2.131597in}{3.023065in}}{\pgfqpoint{2.120998in}{3.018674in}}{\pgfqpoint{2.113184in}{3.010861in}}%
\pgfpathcurveto{\pgfqpoint{2.105371in}{3.003047in}}{\pgfqpoint{2.100980in}{2.992448in}}{\pgfqpoint{2.100980in}{2.981398in}}%
\pgfpathcurveto{\pgfqpoint{2.100980in}{2.970348in}}{\pgfqpoint{2.105371in}{2.959749in}}{\pgfqpoint{2.113184in}{2.951935in}}%
\pgfpathcurveto{\pgfqpoint{2.120998in}{2.944122in}}{\pgfqpoint{2.131597in}{2.939731in}}{\pgfqpoint{2.142647in}{2.939731in}}%
\pgfpathclose%
\pgfusepath{stroke,fill}%
\end{pgfscope}%
\begin{pgfscope}%
\pgfpathrectangle{\pgfqpoint{0.888750in}{0.419100in}}{\pgfqpoint{2.504659in}{2.933700in}} %
\pgfusepath{clip}%
\pgfsetbuttcap%
\pgfsetroundjoin%
\definecolor{currentfill}{rgb}{0.400000,0.600000,0.800000}%
\pgfsetfillcolor{currentfill}%
\pgfsetlinewidth{1.003750pt}%
\definecolor{currentstroke}{rgb}{0.400000,0.600000,0.800000}%
\pgfsetstrokecolor{currentstroke}%
\pgfsetdash{}{0pt}%
\pgfpathmoveto{\pgfqpoint{1.829173in}{2.875955in}}%
\pgfpathcurveto{\pgfqpoint{1.840223in}{2.875955in}}{\pgfqpoint{1.850822in}{2.880346in}}{\pgfqpoint{1.858635in}{2.888159in}}%
\pgfpathcurveto{\pgfqpoint{1.866449in}{2.895973in}}{\pgfqpoint{1.870839in}{2.906572in}}{\pgfqpoint{1.870839in}{2.917622in}}%
\pgfpathcurveto{\pgfqpoint{1.870839in}{2.928672in}}{\pgfqpoint{1.866449in}{2.939271in}}{\pgfqpoint{1.858635in}{2.947085in}}%
\pgfpathcurveto{\pgfqpoint{1.850822in}{2.954898in}}{\pgfqpoint{1.840223in}{2.959289in}}{\pgfqpoint{1.829173in}{2.959289in}}%
\pgfpathcurveto{\pgfqpoint{1.818123in}{2.959289in}}{\pgfqpoint{1.807524in}{2.954898in}}{\pgfqpoint{1.799710in}{2.947085in}}%
\pgfpathcurveto{\pgfqpoint{1.791896in}{2.939271in}}{\pgfqpoint{1.787506in}{2.928672in}}{\pgfqpoint{1.787506in}{2.917622in}}%
\pgfpathcurveto{\pgfqpoint{1.787506in}{2.906572in}}{\pgfqpoint{1.791896in}{2.895973in}}{\pgfqpoint{1.799710in}{2.888159in}}%
\pgfpathcurveto{\pgfqpoint{1.807524in}{2.880346in}}{\pgfqpoint{1.818123in}{2.875955in}}{\pgfqpoint{1.829173in}{2.875955in}}%
\pgfpathclose%
\pgfusepath{stroke,fill}%
\end{pgfscope}%
\begin{pgfscope}%
\pgfpathrectangle{\pgfqpoint{0.888750in}{0.419100in}}{\pgfqpoint{2.504659in}{2.933700in}} %
\pgfusepath{clip}%
\pgfsetbuttcap%
\pgfsetroundjoin%
\definecolor{currentfill}{rgb}{0.400000,0.600000,0.800000}%
\pgfsetfillcolor{currentfill}%
\pgfsetlinewidth{1.003750pt}%
\definecolor{currentstroke}{rgb}{0.400000,0.600000,0.800000}%
\pgfsetstrokecolor{currentstroke}%
\pgfsetdash{}{0pt}%
\pgfpathmoveto{\pgfqpoint{1.515698in}{2.654615in}}%
\pgfpathcurveto{\pgfqpoint{1.526749in}{2.654615in}}{\pgfqpoint{1.537348in}{2.659005in}}{\pgfqpoint{1.545161in}{2.666819in}}%
\pgfpathcurveto{\pgfqpoint{1.552975in}{2.674632in}}{\pgfqpoint{1.557365in}{2.685231in}}{\pgfqpoint{1.557365in}{2.696281in}}%
\pgfpathcurveto{\pgfqpoint{1.557365in}{2.707332in}}{\pgfqpoint{1.552975in}{2.717931in}}{\pgfqpoint{1.545161in}{2.725744in}}%
\pgfpathcurveto{\pgfqpoint{1.537348in}{2.733558in}}{\pgfqpoint{1.526749in}{2.737948in}}{\pgfqpoint{1.515698in}{2.737948in}}%
\pgfpathcurveto{\pgfqpoint{1.504648in}{2.737948in}}{\pgfqpoint{1.494049in}{2.733558in}}{\pgfqpoint{1.486236in}{2.725744in}}%
\pgfpathcurveto{\pgfqpoint{1.478422in}{2.717931in}}{\pgfqpoint{1.474032in}{2.707332in}}{\pgfqpoint{1.474032in}{2.696281in}}%
\pgfpathcurveto{\pgfqpoint{1.474032in}{2.685231in}}{\pgfqpoint{1.478422in}{2.674632in}}{\pgfqpoint{1.486236in}{2.666819in}}%
\pgfpathcurveto{\pgfqpoint{1.494049in}{2.659005in}}{\pgfqpoint{1.504648in}{2.654615in}}{\pgfqpoint{1.515698in}{2.654615in}}%
\pgfpathclose%
\pgfusepath{stroke,fill}%
\end{pgfscope}%
\begin{pgfscope}%
\pgfpathrectangle{\pgfqpoint{0.888750in}{0.419100in}}{\pgfqpoint{2.504659in}{2.933700in}} %
\pgfusepath{clip}%
\pgfsetbuttcap%
\pgfsetroundjoin%
\definecolor{currentfill}{rgb}{0.400000,0.600000,0.800000}%
\pgfsetfillcolor{currentfill}%
\pgfsetlinewidth{1.003750pt}%
\definecolor{currentstroke}{rgb}{0.400000,0.600000,0.800000}%
\pgfsetstrokecolor{currentstroke}%
\pgfsetdash{}{0pt}%
\pgfpathmoveto{\pgfqpoint{1.202224in}{1.814271in}}%
\pgfpathcurveto{\pgfqpoint{1.213274in}{1.814271in}}{\pgfqpoint{1.223873in}{1.818661in}}{\pgfqpoint{1.231687in}{1.826475in}}%
\pgfpathcurveto{\pgfqpoint{1.239501in}{1.834289in}}{\pgfqpoint{1.243891in}{1.844888in}}{\pgfqpoint{1.243891in}{1.855938in}}%
\pgfpathcurveto{\pgfqpoint{1.243891in}{1.866988in}}{\pgfqpoint{1.239501in}{1.877587in}}{\pgfqpoint{1.231687in}{1.885401in}}%
\pgfpathcurveto{\pgfqpoint{1.223873in}{1.893214in}}{\pgfqpoint{1.213274in}{1.897604in}}{\pgfqpoint{1.202224in}{1.897604in}}%
\pgfpathcurveto{\pgfqpoint{1.191174in}{1.897604in}}{\pgfqpoint{1.180575in}{1.893214in}}{\pgfqpoint{1.172761in}{1.885401in}}%
\pgfpathcurveto{\pgfqpoint{1.164948in}{1.877587in}}{\pgfqpoint{1.160558in}{1.866988in}}{\pgfqpoint{1.160558in}{1.855938in}}%
\pgfpathcurveto{\pgfqpoint{1.160558in}{1.844888in}}{\pgfqpoint{1.164948in}{1.834289in}}{\pgfqpoint{1.172761in}{1.826475in}}%
\pgfpathcurveto{\pgfqpoint{1.180575in}{1.818661in}}{\pgfqpoint{1.191174in}{1.814271in}}{\pgfqpoint{1.202224in}{1.814271in}}%
\pgfpathclose%
\pgfusepath{stroke,fill}%
\end{pgfscope}%
\begin{pgfscope}%
\pgfpathrectangle{\pgfqpoint{0.888750in}{0.419100in}}{\pgfqpoint{2.504659in}{2.933700in}} %
\pgfusepath{clip}%
\pgfsetbuttcap%
\pgfsetroundjoin%
\definecolor{currentfill}{rgb}{0.400000,0.600000,0.800000}%
\pgfsetfillcolor{currentfill}%
\pgfsetlinewidth{1.003750pt}%
\definecolor{currentstroke}{rgb}{0.400000,0.600000,0.800000}%
\pgfsetstrokecolor{currentstroke}%
\pgfsetdash{}{0pt}%
\pgfpathmoveto{\pgfqpoint{1.829173in}{0.752587in}}%
\pgfpathcurveto{\pgfqpoint{1.840223in}{0.752587in}}{\pgfqpoint{1.850822in}{0.756977in}}{\pgfqpoint{1.858635in}{0.764791in}}%
\pgfpathcurveto{\pgfqpoint{1.866449in}{0.772604in}}{\pgfqpoint{1.870839in}{0.783203in}}{\pgfqpoint{1.870839in}{0.794253in}}%
\pgfpathcurveto{\pgfqpoint{1.870839in}{0.805304in}}{\pgfqpoint{1.866449in}{0.815903in}}{\pgfqpoint{1.858635in}{0.823716in}}%
\pgfpathcurveto{\pgfqpoint{1.850822in}{0.831530in}}{\pgfqpoint{1.840223in}{0.835920in}}{\pgfqpoint{1.829173in}{0.835920in}}%
\pgfpathcurveto{\pgfqpoint{1.818123in}{0.835920in}}{\pgfqpoint{1.807524in}{0.831530in}}{\pgfqpoint{1.799710in}{0.823716in}}%
\pgfpathcurveto{\pgfqpoint{1.791896in}{0.815903in}}{\pgfqpoint{1.787506in}{0.805304in}}{\pgfqpoint{1.787506in}{0.794253in}}%
\pgfpathcurveto{\pgfqpoint{1.787506in}{0.783203in}}{\pgfqpoint{1.791896in}{0.772604in}}{\pgfqpoint{1.799710in}{0.764791in}}%
\pgfpathcurveto{\pgfqpoint{1.807524in}{0.756977in}}{\pgfqpoint{1.818123in}{0.752587in}}{\pgfqpoint{1.829173in}{0.752587in}}%
\pgfpathclose%
\pgfusepath{stroke,fill}%
\end{pgfscope}%
\begin{pgfscope}%
\pgfpathrectangle{\pgfqpoint{0.888750in}{0.419100in}}{\pgfqpoint{2.504659in}{2.933700in}} %
\pgfusepath{clip}%
\pgfsetbuttcap%
\pgfsetroundjoin%
\definecolor{currentfill}{rgb}{0.400000,0.600000,0.800000}%
\pgfsetfillcolor{currentfill}%
\pgfsetlinewidth{1.003750pt}%
\definecolor{currentstroke}{rgb}{0.400000,0.600000,0.800000}%
\pgfsetstrokecolor{currentstroke}%
\pgfsetdash{}{0pt}%
\pgfpathmoveto{\pgfqpoint{1.515698in}{0.973927in}}%
\pgfpathcurveto{\pgfqpoint{1.526749in}{0.973927in}}{\pgfqpoint{1.537348in}{0.978318in}}{\pgfqpoint{1.545161in}{0.986131in}}%
\pgfpathcurveto{\pgfqpoint{1.552975in}{0.993945in}}{\pgfqpoint{1.557365in}{1.004544in}}{\pgfqpoint{1.557365in}{1.015594in}}%
\pgfpathcurveto{\pgfqpoint{1.557365in}{1.026644in}}{\pgfqpoint{1.552975in}{1.037243in}}{\pgfqpoint{1.545161in}{1.045057in}}%
\pgfpathcurveto{\pgfqpoint{1.537348in}{1.052870in}}{\pgfqpoint{1.526749in}{1.057261in}}{\pgfqpoint{1.515698in}{1.057261in}}%
\pgfpathcurveto{\pgfqpoint{1.504648in}{1.057261in}}{\pgfqpoint{1.494049in}{1.052870in}}{\pgfqpoint{1.486236in}{1.045057in}}%
\pgfpathcurveto{\pgfqpoint{1.478422in}{1.037243in}}{\pgfqpoint{1.474032in}{1.026644in}}{\pgfqpoint{1.474032in}{1.015594in}}%
\pgfpathcurveto{\pgfqpoint{1.474032in}{1.004544in}}{\pgfqpoint{1.478422in}{0.993945in}}{\pgfqpoint{1.486236in}{0.986131in}}%
\pgfpathcurveto{\pgfqpoint{1.494049in}{0.978318in}}{\pgfqpoint{1.504648in}{0.973927in}}{\pgfqpoint{1.515698in}{0.973927in}}%
\pgfpathclose%
\pgfusepath{stroke,fill}%
\end{pgfscope}%
\begin{pgfscope}%
\pgfpathrectangle{\pgfqpoint{0.888750in}{0.419100in}}{\pgfqpoint{2.504659in}{2.933700in}} %
\pgfusepath{clip}%
\pgfsetbuttcap%
\pgfsetroundjoin%
\definecolor{currentfill}{rgb}{0.400000,0.600000,0.800000}%
\pgfsetfillcolor{currentfill}%
\pgfsetlinewidth{1.003750pt}%
\definecolor{currentstroke}{rgb}{0.400000,0.600000,0.800000}%
\pgfsetstrokecolor{currentstroke}%
\pgfsetdash{}{0pt}%
\pgfpathmoveto{\pgfqpoint{2.456121in}{0.752587in}}%
\pgfpathcurveto{\pgfqpoint{2.467171in}{0.752587in}}{\pgfqpoint{2.477770in}{0.756977in}}{\pgfqpoint{2.485584in}{0.764791in}}%
\pgfpathcurveto{\pgfqpoint{2.493398in}{0.772604in}}{\pgfqpoint{2.497788in}{0.783203in}}{\pgfqpoint{2.497788in}{0.794253in}}%
\pgfpathcurveto{\pgfqpoint{2.497788in}{0.805304in}}{\pgfqpoint{2.493398in}{0.815903in}}{\pgfqpoint{2.485584in}{0.823716in}}%
\pgfpathcurveto{\pgfqpoint{2.477770in}{0.831530in}}{\pgfqpoint{2.467171in}{0.835920in}}{\pgfqpoint{2.456121in}{0.835920in}}%
\pgfpathcurveto{\pgfqpoint{2.445071in}{0.835920in}}{\pgfqpoint{2.434472in}{0.831530in}}{\pgfqpoint{2.426658in}{0.823716in}}%
\pgfpathcurveto{\pgfqpoint{2.418845in}{0.815903in}}{\pgfqpoint{2.414454in}{0.805304in}}{\pgfqpoint{2.414454in}{0.794253in}}%
\pgfpathcurveto{\pgfqpoint{2.414454in}{0.783203in}}{\pgfqpoint{2.418845in}{0.772604in}}{\pgfqpoint{2.426658in}{0.764791in}}%
\pgfpathcurveto{\pgfqpoint{2.434472in}{0.756977in}}{\pgfqpoint{2.445071in}{0.752587in}}{\pgfqpoint{2.456121in}{0.752587in}}%
\pgfpathclose%
\pgfusepath{stroke,fill}%
\end{pgfscope}%
\begin{pgfscope}%
\pgfpathrectangle{\pgfqpoint{0.888750in}{0.419100in}}{\pgfqpoint{2.504659in}{2.933700in}} %
\pgfusepath{clip}%
\pgfsetbuttcap%
\pgfsetroundjoin%
\definecolor{currentfill}{rgb}{0.400000,0.600000,0.800000}%
\pgfsetfillcolor{currentfill}%
\pgfsetlinewidth{1.003750pt}%
\definecolor{currentstroke}{rgb}{0.400000,0.600000,0.800000}%
\pgfsetstrokecolor{currentstroke}%
\pgfsetdash{}{0pt}%
\pgfpathmoveto{\pgfqpoint{2.769595in}{0.973927in}}%
\pgfpathcurveto{\pgfqpoint{2.780646in}{0.973927in}}{\pgfqpoint{2.791245in}{0.978318in}}{\pgfqpoint{2.799058in}{0.986131in}}%
\pgfpathcurveto{\pgfqpoint{2.806872in}{0.993945in}}{\pgfqpoint{2.811262in}{1.004544in}}{\pgfqpoint{2.811262in}{1.015594in}}%
\pgfpathcurveto{\pgfqpoint{2.811262in}{1.026644in}}{\pgfqpoint{2.806872in}{1.037243in}}{\pgfqpoint{2.799058in}{1.045057in}}%
\pgfpathcurveto{\pgfqpoint{2.791245in}{1.052870in}}{\pgfqpoint{2.780646in}{1.057261in}}{\pgfqpoint{2.769595in}{1.057261in}}%
\pgfpathcurveto{\pgfqpoint{2.758545in}{1.057261in}}{\pgfqpoint{2.747946in}{1.052870in}}{\pgfqpoint{2.740133in}{1.045057in}}%
\pgfpathcurveto{\pgfqpoint{2.732319in}{1.037243in}}{\pgfqpoint{2.727929in}{1.026644in}}{\pgfqpoint{2.727929in}{1.015594in}}%
\pgfpathcurveto{\pgfqpoint{2.727929in}{1.004544in}}{\pgfqpoint{2.732319in}{0.993945in}}{\pgfqpoint{2.740133in}{0.986131in}}%
\pgfpathcurveto{\pgfqpoint{2.747946in}{0.978318in}}{\pgfqpoint{2.758545in}{0.973927in}}{\pgfqpoint{2.769595in}{0.973927in}}%
\pgfpathclose%
\pgfusepath{stroke,fill}%
\end{pgfscope}%
\begin{pgfscope}%
\pgfsetbuttcap%
\pgfsetroundjoin%
\definecolor{currentfill}{rgb}{0.000000,0.000000,0.000000}%
\pgfsetfillcolor{currentfill}%
\pgfsetlinewidth{0.803000pt}%
\definecolor{currentstroke}{rgb}{0.000000,0.000000,0.000000}%
\pgfsetstrokecolor{currentstroke}%
\pgfsetdash{}{0pt}%
\pgfsys@defobject{currentmarker}{\pgfqpoint{0.000000in}{-0.048611in}}{\pgfqpoint{0.000000in}{0.000000in}}{%
\pgfpathmoveto{\pgfqpoint{0.000000in}{0.000000in}}%
\pgfpathlineto{\pgfqpoint{0.000000in}{-0.048611in}}%
\pgfusepath{stroke,fill}%
}%
\begin{pgfscope}%
\pgfsys@transformshift{0.888750in}{0.419100in}%
\pgfsys@useobject{currentmarker}{}%
\end{pgfscope}%
\end{pgfscope}%
\begin{pgfscope}%
\pgftext[x=0.888750in,y=0.321878in,,top]{\sffamily\fontsize{10.000000}{12.000000}\selectfont −4}%
\end{pgfscope}%
\begin{pgfscope}%
\pgfsetbuttcap%
\pgfsetroundjoin%
\definecolor{currentfill}{rgb}{0.000000,0.000000,0.000000}%
\pgfsetfillcolor{currentfill}%
\pgfsetlinewidth{0.803000pt}%
\definecolor{currentstroke}{rgb}{0.000000,0.000000,0.000000}%
\pgfsetstrokecolor{currentstroke}%
\pgfsetdash{}{0pt}%
\pgfsys@defobject{currentmarker}{\pgfqpoint{0.000000in}{-0.048611in}}{\pgfqpoint{0.000000in}{0.000000in}}{%
\pgfpathmoveto{\pgfqpoint{0.000000in}{0.000000in}}%
\pgfpathlineto{\pgfqpoint{0.000000in}{-0.048611in}}%
\pgfusepath{stroke,fill}%
}%
\begin{pgfscope}%
\pgfsys@transformshift{1.515698in}{0.419100in}%
\pgfsys@useobject{currentmarker}{}%
\end{pgfscope}%
\end{pgfscope}%
\begin{pgfscope}%
\pgftext[x=1.515698in,y=0.321878in,,top]{\sffamily\fontsize{10.000000}{12.000000}\selectfont −2}%
\end{pgfscope}%
\begin{pgfscope}%
\pgfsetbuttcap%
\pgfsetroundjoin%
\definecolor{currentfill}{rgb}{0.000000,0.000000,0.000000}%
\pgfsetfillcolor{currentfill}%
\pgfsetlinewidth{0.803000pt}%
\definecolor{currentstroke}{rgb}{0.000000,0.000000,0.000000}%
\pgfsetstrokecolor{currentstroke}%
\pgfsetdash{}{0pt}%
\pgfsys@defobject{currentmarker}{\pgfqpoint{0.000000in}{-0.048611in}}{\pgfqpoint{0.000000in}{0.000000in}}{%
\pgfpathmoveto{\pgfqpoint{0.000000in}{0.000000in}}%
\pgfpathlineto{\pgfqpoint{0.000000in}{-0.048611in}}%
\pgfusepath{stroke,fill}%
}%
\begin{pgfscope}%
\pgfsys@transformshift{2.142647in}{0.419100in}%
\pgfsys@useobject{currentmarker}{}%
\end{pgfscope}%
\end{pgfscope}%
\begin{pgfscope}%
\pgftext[x=2.142647in,y=0.321878in,,top]{\sffamily\fontsize{10.000000}{12.000000}\selectfont 0}%
\end{pgfscope}%
\begin{pgfscope}%
\pgfsetbuttcap%
\pgfsetroundjoin%
\definecolor{currentfill}{rgb}{0.000000,0.000000,0.000000}%
\pgfsetfillcolor{currentfill}%
\pgfsetlinewidth{0.803000pt}%
\definecolor{currentstroke}{rgb}{0.000000,0.000000,0.000000}%
\pgfsetstrokecolor{currentstroke}%
\pgfsetdash{}{0pt}%
\pgfsys@defobject{currentmarker}{\pgfqpoint{0.000000in}{-0.048611in}}{\pgfqpoint{0.000000in}{0.000000in}}{%
\pgfpathmoveto{\pgfqpoint{0.000000in}{0.000000in}}%
\pgfpathlineto{\pgfqpoint{0.000000in}{-0.048611in}}%
\pgfusepath{stroke,fill}%
}%
\begin{pgfscope}%
\pgfsys@transformshift{2.769595in}{0.419100in}%
\pgfsys@useobject{currentmarker}{}%
\end{pgfscope}%
\end{pgfscope}%
\begin{pgfscope}%
\pgftext[x=2.769595in,y=0.321878in,,top]{\sffamily\fontsize{10.000000}{12.000000}\selectfont 2}%
\end{pgfscope}%
\begin{pgfscope}%
\pgfsetbuttcap%
\pgfsetroundjoin%
\definecolor{currentfill}{rgb}{0.000000,0.000000,0.000000}%
\pgfsetfillcolor{currentfill}%
\pgfsetlinewidth{0.803000pt}%
\definecolor{currentstroke}{rgb}{0.000000,0.000000,0.000000}%
\pgfsetstrokecolor{currentstroke}%
\pgfsetdash{}{0pt}%
\pgfsys@defobject{currentmarker}{\pgfqpoint{0.000000in}{-0.048611in}}{\pgfqpoint{0.000000in}{0.000000in}}{%
\pgfpathmoveto{\pgfqpoint{0.000000in}{0.000000in}}%
\pgfpathlineto{\pgfqpoint{0.000000in}{-0.048611in}}%
\pgfusepath{stroke,fill}%
}%
\begin{pgfscope}%
\pgfsys@transformshift{3.396544in}{0.419100in}%
\pgfsys@useobject{currentmarker}{}%
\end{pgfscope}%
\end{pgfscope}%
\begin{pgfscope}%
\pgftext[x=3.396544in,y=0.321878in,,top]{\sffamily\fontsize{10.000000}{12.000000}\selectfont 4}%
\end{pgfscope}%
\begin{pgfscope}%
\pgftext[x=2.141080in,y=0.131909in,,top]{\sffamily\fontsize{10.000000}{12.000000}\selectfont x}%
\end{pgfscope}%
\begin{pgfscope}%
\pgfsetbuttcap%
\pgfsetroundjoin%
\definecolor{currentfill}{rgb}{0.000000,0.000000,0.000000}%
\pgfsetfillcolor{currentfill}%
\pgfsetlinewidth{0.803000pt}%
\definecolor{currentstroke}{rgb}{0.000000,0.000000,0.000000}%
\pgfsetstrokecolor{currentstroke}%
\pgfsetdash{}{0pt}%
\pgfsys@defobject{currentmarker}{\pgfqpoint{-0.048611in}{0.000000in}}{\pgfqpoint{0.000000in}{0.000000in}}{%
\pgfpathmoveto{\pgfqpoint{0.000000in}{0.000000in}}%
\pgfpathlineto{\pgfqpoint{-0.048611in}{0.000000in}}%
\pgfusepath{stroke,fill}%
}%
\begin{pgfscope}%
\pgfsys@transformshift{0.888750in}{0.730477in}%
\pgfsys@useobject{currentmarker}{}%
\end{pgfscope}%
\end{pgfscope}%
\begin{pgfscope}%
\pgftext[x=0.586789in,y=0.677716in,left,base]{\sffamily\fontsize{10.000000}{12.000000}\selectfont −3}%
\end{pgfscope}%
\begin{pgfscope}%
\pgfsetbuttcap%
\pgfsetroundjoin%
\definecolor{currentfill}{rgb}{0.000000,0.000000,0.000000}%
\pgfsetfillcolor{currentfill}%
\pgfsetlinewidth{0.803000pt}%
\definecolor{currentstroke}{rgb}{0.000000,0.000000,0.000000}%
\pgfsetstrokecolor{currentstroke}%
\pgfsetdash{}{0pt}%
\pgfsys@defobject{currentmarker}{\pgfqpoint{-0.048611in}{0.000000in}}{\pgfqpoint{0.000000in}{0.000000in}}{%
\pgfpathmoveto{\pgfqpoint{0.000000in}{0.000000in}}%
\pgfpathlineto{\pgfqpoint{-0.048611in}{0.000000in}}%
\pgfusepath{stroke,fill}%
}%
\begin{pgfscope}%
\pgfsys@transformshift{0.888750in}{1.105631in}%
\pgfsys@useobject{currentmarker}{}%
\end{pgfscope}%
\end{pgfscope}%
\begin{pgfscope}%
\pgftext[x=0.586789in,y=1.052869in,left,base]{\sffamily\fontsize{10.000000}{12.000000}\selectfont −2}%
\end{pgfscope}%
\begin{pgfscope}%
\pgfsetbuttcap%
\pgfsetroundjoin%
\definecolor{currentfill}{rgb}{0.000000,0.000000,0.000000}%
\pgfsetfillcolor{currentfill}%
\pgfsetlinewidth{0.803000pt}%
\definecolor{currentstroke}{rgb}{0.000000,0.000000,0.000000}%
\pgfsetstrokecolor{currentstroke}%
\pgfsetdash{}{0pt}%
\pgfsys@defobject{currentmarker}{\pgfqpoint{-0.048611in}{0.000000in}}{\pgfqpoint{0.000000in}{0.000000in}}{%
\pgfpathmoveto{\pgfqpoint{0.000000in}{0.000000in}}%
\pgfpathlineto{\pgfqpoint{-0.048611in}{0.000000in}}%
\pgfusepath{stroke,fill}%
}%
\begin{pgfscope}%
\pgfsys@transformshift{0.888750in}{1.480784in}%
\pgfsys@useobject{currentmarker}{}%
\end{pgfscope}%
\end{pgfscope}%
\begin{pgfscope}%
\pgftext[x=0.586789in,y=1.428023in,left,base]{\sffamily\fontsize{10.000000}{12.000000}\selectfont −1}%
\end{pgfscope}%
\begin{pgfscope}%
\pgfsetbuttcap%
\pgfsetroundjoin%
\definecolor{currentfill}{rgb}{0.000000,0.000000,0.000000}%
\pgfsetfillcolor{currentfill}%
\pgfsetlinewidth{0.803000pt}%
\definecolor{currentstroke}{rgb}{0.000000,0.000000,0.000000}%
\pgfsetstrokecolor{currentstroke}%
\pgfsetdash{}{0pt}%
\pgfsys@defobject{currentmarker}{\pgfqpoint{-0.048611in}{0.000000in}}{\pgfqpoint{0.000000in}{0.000000in}}{%
\pgfpathmoveto{\pgfqpoint{0.000000in}{0.000000in}}%
\pgfpathlineto{\pgfqpoint{-0.048611in}{0.000000in}}%
\pgfusepath{stroke,fill}%
}%
\begin{pgfscope}%
\pgfsys@transformshift{0.888750in}{1.855938in}%
\pgfsys@useobject{currentmarker}{}%
\end{pgfscope}%
\end{pgfscope}%
\begin{pgfscope}%
\pgftext[x=0.703163in,y=1.803176in,left,base]{\sffamily\fontsize{10.000000}{12.000000}\selectfont 0}%
\end{pgfscope}%
\begin{pgfscope}%
\pgfsetbuttcap%
\pgfsetroundjoin%
\definecolor{currentfill}{rgb}{0.000000,0.000000,0.000000}%
\pgfsetfillcolor{currentfill}%
\pgfsetlinewidth{0.803000pt}%
\definecolor{currentstroke}{rgb}{0.000000,0.000000,0.000000}%
\pgfsetstrokecolor{currentstroke}%
\pgfsetdash{}{0pt}%
\pgfsys@defobject{currentmarker}{\pgfqpoint{-0.048611in}{0.000000in}}{\pgfqpoint{0.000000in}{0.000000in}}{%
\pgfpathmoveto{\pgfqpoint{0.000000in}{0.000000in}}%
\pgfpathlineto{\pgfqpoint{-0.048611in}{0.000000in}}%
\pgfusepath{stroke,fill}%
}%
\begin{pgfscope}%
\pgfsys@transformshift{0.888750in}{2.231091in}%
\pgfsys@useobject{currentmarker}{}%
\end{pgfscope}%
\end{pgfscope}%
\begin{pgfscope}%
\pgftext[x=0.703163in,y=2.178330in,left,base]{\sffamily\fontsize{10.000000}{12.000000}\selectfont 1}%
\end{pgfscope}%
\begin{pgfscope}%
\pgfsetbuttcap%
\pgfsetroundjoin%
\definecolor{currentfill}{rgb}{0.000000,0.000000,0.000000}%
\pgfsetfillcolor{currentfill}%
\pgfsetlinewidth{0.803000pt}%
\definecolor{currentstroke}{rgb}{0.000000,0.000000,0.000000}%
\pgfsetstrokecolor{currentstroke}%
\pgfsetdash{}{0pt}%
\pgfsys@defobject{currentmarker}{\pgfqpoint{-0.048611in}{0.000000in}}{\pgfqpoint{0.000000in}{0.000000in}}{%
\pgfpathmoveto{\pgfqpoint{0.000000in}{0.000000in}}%
\pgfpathlineto{\pgfqpoint{-0.048611in}{0.000000in}}%
\pgfusepath{stroke,fill}%
}%
\begin{pgfscope}%
\pgfsys@transformshift{0.888750in}{2.606245in}%
\pgfsys@useobject{currentmarker}{}%
\end{pgfscope}%
\end{pgfscope}%
\begin{pgfscope}%
\pgftext[x=0.703163in,y=2.553483in,left,base]{\sffamily\fontsize{10.000000}{12.000000}\selectfont 2}%
\end{pgfscope}%
\begin{pgfscope}%
\pgfsetbuttcap%
\pgfsetroundjoin%
\definecolor{currentfill}{rgb}{0.000000,0.000000,0.000000}%
\pgfsetfillcolor{currentfill}%
\pgfsetlinewidth{0.803000pt}%
\definecolor{currentstroke}{rgb}{0.000000,0.000000,0.000000}%
\pgfsetstrokecolor{currentstroke}%
\pgfsetdash{}{0pt}%
\pgfsys@defobject{currentmarker}{\pgfqpoint{-0.048611in}{0.000000in}}{\pgfqpoint{0.000000in}{0.000000in}}{%
\pgfpathmoveto{\pgfqpoint{0.000000in}{0.000000in}}%
\pgfpathlineto{\pgfqpoint{-0.048611in}{0.000000in}}%
\pgfusepath{stroke,fill}%
}%
\begin{pgfscope}%
\pgfsys@transformshift{0.888750in}{2.981398in}%
\pgfsys@useobject{currentmarker}{}%
\end{pgfscope}%
\end{pgfscope}%
\begin{pgfscope}%
\pgftext[x=0.703163in,y=2.928637in,left,base]{\sffamily\fontsize{10.000000}{12.000000}\selectfont 3}%
\end{pgfscope}%
\begin{pgfscope}%
\pgfsetbuttcap%
\pgfsetroundjoin%
\definecolor{currentfill}{rgb}{0.000000,0.000000,0.000000}%
\pgfsetfillcolor{currentfill}%
\pgfsetlinewidth{0.803000pt}%
\definecolor{currentstroke}{rgb}{0.000000,0.000000,0.000000}%
\pgfsetstrokecolor{currentstroke}%
\pgfsetdash{}{0pt}%
\pgfsys@defobject{currentmarker}{\pgfqpoint{-0.048611in}{0.000000in}}{\pgfqpoint{0.000000in}{0.000000in}}{%
\pgfpathmoveto{\pgfqpoint{0.000000in}{0.000000in}}%
\pgfpathlineto{\pgfqpoint{-0.048611in}{0.000000in}}%
\pgfusepath{stroke,fill}%
}%
\begin{pgfscope}%
\pgfsys@transformshift{0.888750in}{3.356552in}%
\pgfsys@useobject{currentmarker}{}%
\end{pgfscope}%
\end{pgfscope}%
\begin{pgfscope}%
\pgftext[x=0.703163in,y=3.303790in,left,base]{\sffamily\fontsize{10.000000}{12.000000}\selectfont 4}%
\end{pgfscope}%
\begin{pgfscope}%
\pgftext[x=0.531233in,y=1.885950in,,bottom,rotate=90.000000]{\sffamily\fontsize{10.000000}{12.000000}\selectfont y}%
\end{pgfscope}%
\begin{pgfscope}%
\pgfpathrectangle{\pgfqpoint{0.888750in}{0.419100in}}{\pgfqpoint{2.504659in}{2.933700in}} %
\pgfusepath{clip}%
\pgfsetbuttcap%
\pgfsetroundjoin%
\pgfsetlinewidth{1.505625pt}%
\definecolor{currentstroke}{rgb}{1.000000,1.000000,1.000000}%
\pgfsetstrokecolor{currentstroke}%
\pgfsetdash{}{0pt}%
\pgfpathmoveto{\pgfqpoint{2.173994in}{0.898734in}}%
\pgfpathlineto{\pgfqpoint{2.167725in}{0.906237in}}%
\pgfpathlineto{\pgfqpoint{2.164120in}{0.906799in}}%
\pgfpathlineto{\pgfqpoint{2.152051in}{0.921243in}}%
\pgfpathlineto{\pgfqpoint{2.148446in}{0.921806in}}%
\pgfpathlineto{\pgfqpoint{2.139512in}{0.932498in}}%
\pgfpathlineto{\pgfqpoint{2.135907in}{0.933060in}}%
\pgfpathlineto{\pgfqpoint{2.126973in}{0.943752in}}%
\pgfpathlineto{\pgfqpoint{2.123368in}{0.944315in}}%
\pgfpathlineto{\pgfqpoint{2.114434in}{0.955007in}}%
\pgfpathlineto{\pgfqpoint{2.110829in}{0.955569in}}%
\pgfpathlineto{\pgfqpoint{2.101895in}{0.966261in}}%
\pgfpathlineto{\pgfqpoint{2.098290in}{0.966824in}}%
\pgfpathlineto{\pgfqpoint{2.086222in}{0.981267in}}%
\pgfpathlineto{\pgfqpoint{2.082617in}{0.981830in}}%
\pgfpathlineto{\pgfqpoint{2.073683in}{0.992522in}}%
\pgfpathlineto{\pgfqpoint{2.070078in}{0.993085in}}%
\pgfpathlineto{\pgfqpoint{2.061144in}{1.003777in}}%
\pgfpathlineto{\pgfqpoint{2.057539in}{1.004339in}}%
\pgfpathlineto{\pgfqpoint{2.048605in}{1.015031in}}%
\pgfpathlineto{\pgfqpoint{2.045000in}{1.015594in}}%
\pgfpathlineto{\pgfqpoint{2.036066in}{1.026286in}}%
\pgfpathlineto{\pgfqpoint{2.032461in}{1.026849in}}%
\pgfpathlineto{\pgfqpoint{2.023527in}{1.037540in}}%
\pgfpathlineto{\pgfqpoint{2.019922in}{1.038103in}}%
\pgfpathlineto{\pgfqpoint{2.007853in}{1.052547in}}%
\pgfpathlineto{\pgfqpoint{2.004248in}{1.053109in}}%
\pgfpathlineto{\pgfqpoint{1.995314in}{1.063801in}}%
\pgfpathlineto{\pgfqpoint{1.991709in}{1.064364in}}%
\pgfpathlineto{\pgfqpoint{1.982775in}{1.075056in}}%
\pgfpathlineto{\pgfqpoint{1.979170in}{1.075619in}}%
\pgfpathlineto{\pgfqpoint{1.970236in}{1.086310in}}%
\pgfpathlineto{\pgfqpoint{1.966631in}{1.086873in}}%
\pgfpathlineto{\pgfqpoint{1.957697in}{1.097565in}}%
\pgfpathlineto{\pgfqpoint{1.954092in}{1.098128in}}%
\pgfpathlineto{\pgfqpoint{1.945158in}{1.108820in}}%
\pgfpathlineto{\pgfqpoint{1.941553in}{1.109382in}}%
\pgfpathlineto{\pgfqpoint{1.929484in}{1.123826in}}%
\pgfpathlineto{\pgfqpoint{1.925879in}{1.124388in}}%
\pgfpathlineto{\pgfqpoint{1.916945in}{1.135080in}}%
\pgfpathlineto{\pgfqpoint{1.913341in}{1.135643in}}%
\pgfpathlineto{\pgfqpoint{1.904407in}{1.146335in}}%
\pgfpathlineto{\pgfqpoint{1.900802in}{1.146898in}}%
\pgfpathlineto{\pgfqpoint{1.891868in}{1.157590in}}%
\pgfpathlineto{\pgfqpoint{1.888263in}{1.158152in}}%
\pgfpathlineto{\pgfqpoint{1.879329in}{1.168844in}}%
\pgfpathlineto{\pgfqpoint{1.875724in}{1.169407in}}%
\pgfpathlineto{\pgfqpoint{1.866790in}{1.180099in}}%
\pgfpathlineto{\pgfqpoint{1.863185in}{1.180662in}}%
\pgfpathlineto{\pgfqpoint{1.851116in}{1.195105in}}%
\pgfpathlineto{\pgfqpoint{1.847511in}{1.195668in}}%
\pgfpathlineto{\pgfqpoint{1.838577in}{1.206360in}}%
\pgfpathlineto{\pgfqpoint{1.834972in}{1.206922in}}%
\pgfpathlineto{\pgfqpoint{1.826038in}{1.217614in}}%
\pgfpathlineto{\pgfqpoint{1.822433in}{1.218177in}}%
\pgfpathlineto{\pgfqpoint{1.813499in}{1.228869in}}%
\pgfpathlineto{\pgfqpoint{1.809894in}{1.229431in}}%
\pgfpathlineto{\pgfqpoint{1.800960in}{1.240123in}}%
\pgfpathlineto{\pgfqpoint{1.797355in}{1.240686in}}%
\pgfpathlineto{\pgfqpoint{1.785286in}{1.255129in}}%
\pgfpathlineto{\pgfqpoint{1.781681in}{1.255692in}}%
\pgfpathlineto{\pgfqpoint{1.772747in}{1.266384in}}%
\pgfpathlineto{\pgfqpoint{1.769142in}{1.266947in}}%
\pgfpathlineto{\pgfqpoint{1.760208in}{1.277639in}}%
\pgfpathlineto{\pgfqpoint{1.756603in}{1.278201in}}%
\pgfpathlineto{\pgfqpoint{1.747669in}{1.288893in}}%
\pgfpathlineto{\pgfqpoint{1.744064in}{1.289456in}}%
\pgfpathlineto{\pgfqpoint{1.735130in}{1.300148in}}%
\pgfpathlineto{\pgfqpoint{1.731525in}{1.300711in}}%
\pgfpathlineto{\pgfqpoint{1.722591in}{1.311402in}}%
\pgfpathlineto{\pgfqpoint{1.718986in}{1.311965in}}%
\pgfpathlineto{\pgfqpoint{1.706918in}{1.326409in}}%
\pgfpathlineto{\pgfqpoint{1.703313in}{1.326971in}}%
\pgfpathlineto{\pgfqpoint{1.694379in}{1.337663in}}%
\pgfpathlineto{\pgfqpoint{1.690774in}{1.338226in}}%
\pgfpathlineto{\pgfqpoint{1.681840in}{1.348918in}}%
\pgfpathlineto{\pgfqpoint{1.678235in}{1.349481in}}%
\pgfpathlineto{\pgfqpoint{1.669301in}{1.360172in}}%
\pgfpathlineto{\pgfqpoint{1.665696in}{1.360735in}}%
\pgfpathlineto{\pgfqpoint{1.656762in}{1.371427in}}%
\pgfpathlineto{\pgfqpoint{1.653157in}{1.371990in}}%
\pgfpathlineto{\pgfqpoint{1.644223in}{1.382682in}}%
\pgfpathlineto{\pgfqpoint{1.640618in}{1.383244in}}%
\pgfpathlineto{\pgfqpoint{1.628549in}{1.397688in}}%
\pgfpathlineto{\pgfqpoint{1.624944in}{1.398251in}}%
\pgfpathlineto{\pgfqpoint{1.616010in}{1.408942in}}%
\pgfpathlineto{\pgfqpoint{1.612405in}{1.409505in}}%
\pgfpathlineto{\pgfqpoint{1.603471in}{1.420197in}}%
\pgfpathlineto{\pgfqpoint{1.599866in}{1.420760in}}%
\pgfpathlineto{\pgfqpoint{1.590932in}{1.431452in}}%
\pgfpathlineto{\pgfqpoint{1.587327in}{1.432014in}}%
\pgfpathlineto{\pgfqpoint{1.578393in}{1.442706in}}%
\pgfpathlineto{\pgfqpoint{1.574788in}{1.443269in}}%
\pgfpathlineto{\pgfqpoint{1.565854in}{1.453961in}}%
\pgfpathlineto{\pgfqpoint{1.562249in}{1.454524in}}%
\pgfpathlineto{\pgfqpoint{1.550181in}{1.468967in}}%
\pgfpathlineto{\pgfqpoint{1.546576in}{1.469530in}}%
\pgfpathlineto{\pgfqpoint{1.537642in}{1.480222in}}%
\pgfpathlineto{\pgfqpoint{1.534037in}{1.480784in}}%
\pgfpathlineto{\pgfqpoint{1.525103in}{1.491476in}}%
\pgfpathlineto{\pgfqpoint{1.521498in}{1.492039in}}%
\pgfpathlineto{\pgfqpoint{1.512564in}{1.502731in}}%
\pgfpathlineto{\pgfqpoint{1.508959in}{1.503293in}}%
\pgfpathlineto{\pgfqpoint{1.500025in}{1.513985in}}%
\pgfpathlineto{\pgfqpoint{1.496420in}{1.514548in}}%
\pgfpathlineto{\pgfqpoint{1.484351in}{1.528991in}}%
\pgfpathlineto{\pgfqpoint{1.480746in}{1.529554in}}%
\pgfpathlineto{\pgfqpoint{1.471812in}{1.540246in}}%
\pgfpathlineto{\pgfqpoint{1.468207in}{1.540809in}}%
\pgfpathlineto{\pgfqpoint{1.459273in}{1.551501in}}%
\pgfpathlineto{\pgfqpoint{1.455668in}{1.552063in}}%
\pgfpathlineto{\pgfqpoint{1.446734in}{1.562755in}}%
\pgfpathlineto{\pgfqpoint{1.443129in}{1.563318in}}%
\pgfpathlineto{\pgfqpoint{1.434195in}{1.574010in}}%
\pgfpathlineto{\pgfqpoint{1.430590in}{1.574573in}}%
\pgfpathlineto{\pgfqpoint{1.421656in}{1.585265in}}%
\pgfpathlineto{\pgfqpoint{1.418051in}{1.585827in}}%
\pgfpathlineto{\pgfqpoint{1.405982in}{1.600271in}}%
\pgfpathlineto{\pgfqpoint{1.402378in}{1.600833in}}%
\pgfpathlineto{\pgfqpoint{1.393444in}{1.611525in}}%
\pgfpathlineto{\pgfqpoint{1.389839in}{1.612088in}}%
\pgfpathlineto{\pgfqpoint{1.380905in}{1.622780in}}%
\pgfpathlineto{\pgfqpoint{1.377300in}{1.623343in}}%
\pgfpathlineto{\pgfqpoint{1.368366in}{1.634034in}}%
\pgfpathlineto{\pgfqpoint{1.364761in}{1.634597in}}%
\pgfpathlineto{\pgfqpoint{1.355827in}{1.645289in}}%
\pgfpathlineto{\pgfqpoint{1.352222in}{1.645852in}}%
\pgfpathlineto{\pgfqpoint{1.343288in}{1.656544in}}%
\pgfpathlineto{\pgfqpoint{1.339683in}{1.657106in}}%
\pgfpathlineto{\pgfqpoint{1.327614in}{1.671550in}}%
\pgfpathlineto{\pgfqpoint{1.324009in}{1.672113in}}%
\pgfpathlineto{\pgfqpoint{1.315075in}{1.682804in}}%
\pgfpathlineto{\pgfqpoint{1.311470in}{1.683367in}}%
\pgfpathlineto{\pgfqpoint{1.302536in}{1.694059in}}%
\pgfpathlineto{\pgfqpoint{1.298931in}{1.694622in}}%
\pgfpathlineto{\pgfqpoint{1.292662in}{1.702125in}}%
\pgfpathlineto{\pgfqpoint{1.293132in}{1.706439in}}%
\pgfpathlineto{\pgfqpoint{1.295796in}{1.709628in}}%
\pgfpathlineto{\pgfqpoint{1.296266in}{1.721445in}}%
\pgfpathlineto{\pgfqpoint{1.298931in}{1.724634in}}%
\pgfpathlineto{\pgfqpoint{1.299401in}{1.732700in}}%
\pgfpathlineto{\pgfqpoint{1.302066in}{1.735889in}}%
\pgfpathlineto{\pgfqpoint{1.302536in}{1.747706in}}%
\pgfpathlineto{\pgfqpoint{1.305201in}{1.750895in}}%
\pgfpathlineto{\pgfqpoint{1.305671in}{1.758961in}}%
\pgfpathlineto{\pgfqpoint{1.308335in}{1.762149in}}%
\pgfpathlineto{\pgfqpoint{1.308805in}{1.773967in}}%
\pgfpathlineto{\pgfqpoint{1.311470in}{1.777155in}}%
\pgfpathlineto{\pgfqpoint{1.311940in}{1.785221in}}%
\pgfpathlineto{\pgfqpoint{1.314605in}{1.788410in}}%
\pgfpathlineto{\pgfqpoint{1.315075in}{1.800227in}}%
\pgfpathlineto{\pgfqpoint{1.317739in}{1.803416in}}%
\pgfpathlineto{\pgfqpoint{1.318210in}{1.811482in}}%
\pgfpathlineto{\pgfqpoint{1.320874in}{1.814671in}}%
\pgfpathlineto{\pgfqpoint{1.321344in}{1.822737in}}%
\pgfpathlineto{\pgfqpoint{1.324009in}{1.825925in}}%
\pgfpathlineto{\pgfqpoint{1.324479in}{1.837743in}}%
\pgfpathlineto{\pgfqpoint{1.327144in}{1.840932in}}%
\pgfpathlineto{\pgfqpoint{1.327614in}{1.848997in}}%
\pgfpathlineto{\pgfqpoint{1.330278in}{1.852186in}}%
\pgfpathlineto{\pgfqpoint{1.330749in}{1.864004in}}%
\pgfpathlineto{\pgfqpoint{1.333413in}{1.867192in}}%
\pgfpathlineto{\pgfqpoint{1.333883in}{1.875258in}}%
\pgfpathlineto{\pgfqpoint{1.336548in}{1.878447in}}%
\pgfpathlineto{\pgfqpoint{1.337018in}{1.890264in}}%
\pgfpathlineto{\pgfqpoint{1.339683in}{1.893453in}}%
\pgfpathlineto{\pgfqpoint{1.340153in}{1.901519in}}%
\pgfpathlineto{\pgfqpoint{1.342817in}{1.904708in}}%
\pgfpathlineto{\pgfqpoint{1.343288in}{1.916525in}}%
\pgfpathlineto{\pgfqpoint{1.345952in}{1.919714in}}%
\pgfpathlineto{\pgfqpoint{1.346422in}{1.927780in}}%
\pgfpathlineto{\pgfqpoint{1.349087in}{1.930968in}}%
\pgfpathlineto{\pgfqpoint{1.349557in}{1.942786in}}%
\pgfpathlineto{\pgfqpoint{1.352222in}{1.945975in}}%
\pgfpathlineto{\pgfqpoint{1.352692in}{1.954040in}}%
\pgfpathlineto{\pgfqpoint{1.355356in}{1.957229in}}%
\pgfpathlineto{\pgfqpoint{1.355827in}{1.969046in}}%
\pgfpathlineto{\pgfqpoint{1.358491in}{1.972235in}}%
\pgfpathlineto{\pgfqpoint{1.358961in}{1.980301in}}%
\pgfpathlineto{\pgfqpoint{1.361626in}{1.983490in}}%
\pgfpathlineto{\pgfqpoint{1.362096in}{1.995307in}}%
\pgfpathlineto{\pgfqpoint{1.364761in}{1.998496in}}%
\pgfpathlineto{\pgfqpoint{1.365231in}{2.006562in}}%
\pgfpathlineto{\pgfqpoint{1.367895in}{2.009751in}}%
\pgfpathlineto{\pgfqpoint{1.368366in}{2.021568in}}%
\pgfpathlineto{\pgfqpoint{1.371030in}{2.024757in}}%
\pgfpathlineto{\pgfqpoint{1.371500in}{2.032823in}}%
\pgfpathlineto{\pgfqpoint{1.374165in}{2.036011in}}%
\pgfpathlineto{\pgfqpoint{1.374635in}{2.044077in}}%
\pgfpathlineto{\pgfqpoint{1.377300in}{2.047266in}}%
\pgfpathlineto{\pgfqpoint{1.377770in}{2.059083in}}%
\pgfpathlineto{\pgfqpoint{1.380434in}{2.062272in}}%
\pgfpathlineto{\pgfqpoint{1.380905in}{2.070338in}}%
\pgfpathlineto{\pgfqpoint{1.383569in}{2.073527in}}%
\pgfpathlineto{\pgfqpoint{1.384039in}{2.085344in}}%
\pgfpathlineto{\pgfqpoint{1.386704in}{2.088533in}}%
\pgfpathlineto{\pgfqpoint{1.387174in}{2.096599in}}%
\pgfpathlineto{\pgfqpoint{1.389839in}{2.099787in}}%
\pgfpathlineto{\pgfqpoint{1.390309in}{2.111605in}}%
\pgfpathlineto{\pgfqpoint{1.392973in}{2.114794in}}%
\pgfpathlineto{\pgfqpoint{1.393444in}{2.122859in}}%
\pgfpathlineto{\pgfqpoint{1.396108in}{2.126048in}}%
\pgfpathlineto{\pgfqpoint{1.396578in}{2.137866in}}%
\pgfpathlineto{\pgfqpoint{1.399243in}{2.141054in}}%
\pgfpathlineto{\pgfqpoint{1.399713in}{2.149120in}}%
\pgfpathlineto{\pgfqpoint{1.402378in}{2.152309in}}%
\pgfpathlineto{\pgfqpoint{1.402848in}{2.164126in}}%
\pgfpathlineto{\pgfqpoint{1.405512in}{2.167315in}}%
\pgfpathlineto{\pgfqpoint{1.405982in}{2.175381in}}%
\pgfpathlineto{\pgfqpoint{1.408647in}{2.178570in}}%
\pgfpathlineto{\pgfqpoint{1.409117in}{2.190387in}}%
\pgfpathlineto{\pgfqpoint{1.411782in}{2.193576in}}%
\pgfpathlineto{\pgfqpoint{1.412252in}{2.201642in}}%
\pgfpathlineto{\pgfqpoint{1.414916in}{2.204830in}}%
\pgfpathlineto{\pgfqpoint{1.415387in}{2.216648in}}%
\pgfpathlineto{\pgfqpoint{1.418051in}{2.219837in}}%
\pgfpathlineto{\pgfqpoint{1.418521in}{2.227902in}}%
\pgfpathlineto{\pgfqpoint{1.421186in}{2.231091in}}%
\pgfpathlineto{\pgfqpoint{1.421656in}{2.242909in}}%
\pgfpathlineto{\pgfqpoint{1.424321in}{2.246097in}}%
\pgfpathlineto{\pgfqpoint{1.424791in}{2.254163in}}%
\pgfpathlineto{\pgfqpoint{1.427455in}{2.257352in}}%
\pgfpathlineto{\pgfqpoint{1.427926in}{2.265418in}}%
\pgfpathlineto{\pgfqpoint{1.430590in}{2.268607in}}%
\pgfpathlineto{\pgfqpoint{1.431060in}{2.280424in}}%
\pgfpathlineto{\pgfqpoint{1.433725in}{2.283613in}}%
\pgfpathlineto{\pgfqpoint{1.434195in}{2.291678in}}%
\pgfpathlineto{\pgfqpoint{1.436860in}{2.294867in}}%
\pgfpathlineto{\pgfqpoint{1.437330in}{2.306685in}}%
\pgfpathlineto{\pgfqpoint{1.439994in}{2.309873in}}%
\pgfpathlineto{\pgfqpoint{1.440465in}{2.317939in}}%
\pgfpathlineto{\pgfqpoint{1.443129in}{2.321128in}}%
\pgfpathlineto{\pgfqpoint{1.443599in}{2.332945in}}%
\pgfpathlineto{\pgfqpoint{1.446264in}{2.336134in}}%
\pgfpathlineto{\pgfqpoint{1.446734in}{2.344200in}}%
\pgfpathlineto{\pgfqpoint{1.449399in}{2.347389in}}%
\pgfpathlineto{\pgfqpoint{1.449869in}{2.359206in}}%
\pgfpathlineto{\pgfqpoint{1.452533in}{2.362395in}}%
\pgfpathlineto{\pgfqpoint{1.453004in}{2.370461in}}%
\pgfpathlineto{\pgfqpoint{1.455668in}{2.373649in}}%
\pgfpathlineto{\pgfqpoint{1.456138in}{2.385467in}}%
\pgfpathlineto{\pgfqpoint{1.458803in}{2.388656in}}%
\pgfpathlineto{\pgfqpoint{1.459273in}{2.396721in}}%
\pgfpathlineto{\pgfqpoint{1.461938in}{2.399910in}}%
\pgfpathlineto{\pgfqpoint{1.462408in}{2.404224in}}%
\pgfpathlineto{\pgfqpoint{1.465543in}{2.404224in}}%
\pgfpathlineto{\pgfqpoint{1.468677in}{2.407976in}}%
\pgfpathlineto{\pgfqpoint{1.478082in}{2.407976in}}%
\pgfpathlineto{\pgfqpoint{1.481216in}{2.411728in}}%
\pgfpathlineto{\pgfqpoint{1.490621in}{2.411728in}}%
\pgfpathlineto{\pgfqpoint{1.493755in}{2.415479in}}%
\pgfpathlineto{\pgfqpoint{1.503159in}{2.415479in}}%
\pgfpathlineto{\pgfqpoint{1.506294in}{2.419231in}}%
\pgfpathlineto{\pgfqpoint{1.515698in}{2.419231in}}%
\pgfpathlineto{\pgfqpoint{1.518833in}{2.422982in}}%
\pgfpathlineto{\pgfqpoint{1.531372in}{2.422982in}}%
\pgfpathlineto{\pgfqpoint{1.534507in}{2.426734in}}%
\pgfpathlineto{\pgfqpoint{1.543911in}{2.426734in}}%
\pgfpathlineto{\pgfqpoint{1.547046in}{2.430485in}}%
\pgfpathlineto{\pgfqpoint{1.556450in}{2.430485in}}%
\pgfpathlineto{\pgfqpoint{1.559585in}{2.434237in}}%
\pgfpathlineto{\pgfqpoint{1.568989in}{2.434237in}}%
\pgfpathlineto{\pgfqpoint{1.572124in}{2.437988in}}%
\pgfpathlineto{\pgfqpoint{1.584663in}{2.437988in}}%
\pgfpathlineto{\pgfqpoint{1.587798in}{2.441740in}}%
\pgfpathlineto{\pgfqpoint{1.597202in}{2.441740in}}%
\pgfpathlineto{\pgfqpoint{1.600337in}{2.445491in}}%
\pgfpathlineto{\pgfqpoint{1.609741in}{2.445491in}}%
\pgfpathlineto{\pgfqpoint{1.612875in}{2.449243in}}%
\pgfpathlineto{\pgfqpoint{1.622280in}{2.449243in}}%
\pgfpathlineto{\pgfqpoint{1.625414in}{2.452994in}}%
\pgfpathlineto{\pgfqpoint{1.637953in}{2.452994in}}%
\pgfpathlineto{\pgfqpoint{1.641088in}{2.456746in}}%
\pgfpathlineto{\pgfqpoint{1.650492in}{2.456746in}}%
\pgfpathlineto{\pgfqpoint{1.653627in}{2.460498in}}%
\pgfpathlineto{\pgfqpoint{1.663031in}{2.460498in}}%
\pgfpathlineto{\pgfqpoint{1.666166in}{2.464249in}}%
\pgfpathlineto{\pgfqpoint{1.675570in}{2.464249in}}%
\pgfpathlineto{\pgfqpoint{1.678705in}{2.468001in}}%
\pgfpathlineto{\pgfqpoint{1.688109in}{2.468001in}}%
\pgfpathlineto{\pgfqpoint{1.691244in}{2.471752in}}%
\pgfpathlineto{\pgfqpoint{1.703783in}{2.471752in}}%
\pgfpathlineto{\pgfqpoint{1.706918in}{2.475504in}}%
\pgfpathlineto{\pgfqpoint{1.716322in}{2.475504in}}%
\pgfpathlineto{\pgfqpoint{1.719457in}{2.479255in}}%
\pgfpathlineto{\pgfqpoint{1.728861in}{2.479255in}}%
\pgfpathlineto{\pgfqpoint{1.731996in}{2.483007in}}%
\pgfpathlineto{\pgfqpoint{1.741400in}{2.483007in}}%
\pgfpathlineto{\pgfqpoint{1.744535in}{2.486758in}}%
\pgfpathlineto{\pgfqpoint{1.757074in}{2.486758in}}%
\pgfpathlineto{\pgfqpoint{1.760208in}{2.490510in}}%
\pgfpathlineto{\pgfqpoint{1.769613in}{2.490510in}}%
\pgfpathlineto{\pgfqpoint{1.772747in}{2.494261in}}%
\pgfpathlineto{\pgfqpoint{1.782152in}{2.494261in}}%
\pgfpathlineto{\pgfqpoint{1.785286in}{2.498013in}}%
\pgfpathlineto{\pgfqpoint{1.794691in}{2.498013in}}%
\pgfpathlineto{\pgfqpoint{1.797825in}{2.501764in}}%
\pgfpathlineto{\pgfqpoint{1.810364in}{2.501764in}}%
\pgfpathlineto{\pgfqpoint{1.813499in}{2.505516in}}%
\pgfpathlineto{\pgfqpoint{1.822903in}{2.505516in}}%
\pgfpathlineto{\pgfqpoint{1.826038in}{2.509267in}}%
\pgfpathlineto{\pgfqpoint{1.835442in}{2.509267in}}%
\pgfpathlineto{\pgfqpoint{1.838577in}{2.513019in}}%
\pgfpathlineto{\pgfqpoint{1.847981in}{2.513019in}}%
\pgfpathlineto{\pgfqpoint{1.851116in}{2.516771in}}%
\pgfpathlineto{\pgfqpoint{1.860520in}{2.516771in}}%
\pgfpathlineto{\pgfqpoint{1.863655in}{2.520522in}}%
\pgfpathlineto{\pgfqpoint{1.876194in}{2.520522in}}%
\pgfpathlineto{\pgfqpoint{1.879329in}{2.524274in}}%
\pgfpathlineto{\pgfqpoint{1.888733in}{2.524274in}}%
\pgfpathlineto{\pgfqpoint{1.891868in}{2.528025in}}%
\pgfpathlineto{\pgfqpoint{1.901272in}{2.528025in}}%
\pgfpathlineto{\pgfqpoint{1.904407in}{2.531777in}}%
\pgfpathlineto{\pgfqpoint{1.913811in}{2.531777in}}%
\pgfpathlineto{\pgfqpoint{1.916945in}{2.535528in}}%
\pgfpathlineto{\pgfqpoint{1.929484in}{2.535528in}}%
\pgfpathlineto{\pgfqpoint{1.932619in}{2.539280in}}%
\pgfpathlineto{\pgfqpoint{1.942023in}{2.539280in}}%
\pgfpathlineto{\pgfqpoint{1.945158in}{2.543031in}}%
\pgfpathlineto{\pgfqpoint{1.954562in}{2.543031in}}%
\pgfpathlineto{\pgfqpoint{1.957697in}{2.546783in}}%
\pgfpathlineto{\pgfqpoint{1.967101in}{2.546783in}}%
\pgfpathlineto{\pgfqpoint{1.970236in}{2.550534in}}%
\pgfpathlineto{\pgfqpoint{1.979640in}{2.550534in}}%
\pgfpathlineto{\pgfqpoint{1.982775in}{2.554286in}}%
\pgfpathlineto{\pgfqpoint{1.995314in}{2.554286in}}%
\pgfpathlineto{\pgfqpoint{1.998449in}{2.558037in}}%
\pgfpathlineto{\pgfqpoint{2.007853in}{2.558037in}}%
\pgfpathlineto{\pgfqpoint{2.010988in}{2.561789in}}%
\pgfpathlineto{\pgfqpoint{2.020392in}{2.561789in}}%
\pgfpathlineto{\pgfqpoint{2.023527in}{2.565540in}}%
\pgfpathlineto{\pgfqpoint{2.032931in}{2.565540in}}%
\pgfpathlineto{\pgfqpoint{2.036066in}{2.569292in}}%
\pgfpathlineto{\pgfqpoint{2.048605in}{2.569292in}}%
\pgfpathlineto{\pgfqpoint{2.051739in}{2.573044in}}%
\pgfpathlineto{\pgfqpoint{2.061144in}{2.573044in}}%
\pgfpathlineto{\pgfqpoint{2.064278in}{2.576795in}}%
\pgfpathlineto{\pgfqpoint{2.073683in}{2.576795in}}%
\pgfpathlineto{\pgfqpoint{2.076817in}{2.580547in}}%
\pgfpathlineto{\pgfqpoint{2.086222in}{2.580547in}}%
\pgfpathlineto{\pgfqpoint{2.089356in}{2.584298in}}%
\pgfpathlineto{\pgfqpoint{2.101895in}{2.584298in}}%
\pgfpathlineto{\pgfqpoint{2.105030in}{2.588050in}}%
\pgfpathlineto{\pgfqpoint{2.114434in}{2.588050in}}%
\pgfpathlineto{\pgfqpoint{2.117569in}{2.591801in}}%
\pgfpathlineto{\pgfqpoint{2.126973in}{2.591801in}}%
\pgfpathlineto{\pgfqpoint{2.130108in}{2.595553in}}%
\pgfpathlineto{\pgfqpoint{2.139512in}{2.595553in}}%
\pgfpathlineto{\pgfqpoint{2.142647in}{2.599304in}}%
\pgfpathlineto{\pgfqpoint{2.152051in}{2.599304in}}%
\pgfpathlineto{\pgfqpoint{2.155186in}{2.603056in}}%
\pgfpathlineto{\pgfqpoint{2.167725in}{2.603056in}}%
\pgfpathlineto{\pgfqpoint{2.170860in}{2.606807in}}%
\pgfpathlineto{\pgfqpoint{2.180264in}{2.606807in}}%
\pgfpathlineto{\pgfqpoint{2.183399in}{2.610559in}}%
\pgfpathlineto{\pgfqpoint{2.192803in}{2.610559in}}%
\pgfpathlineto{\pgfqpoint{2.195938in}{2.614310in}}%
\pgfpathlineto{\pgfqpoint{2.205342in}{2.614310in}}%
\pgfpathlineto{\pgfqpoint{2.208477in}{2.618062in}}%
\pgfpathlineto{\pgfqpoint{2.221015in}{2.618062in}}%
\pgfpathlineto{\pgfqpoint{2.224150in}{2.621813in}}%
\pgfpathlineto{\pgfqpoint{2.233554in}{2.621813in}}%
\pgfpathlineto{\pgfqpoint{2.236689in}{2.625565in}}%
\pgfpathlineto{\pgfqpoint{2.246093in}{2.625565in}}%
\pgfpathlineto{\pgfqpoint{2.249228in}{2.629317in}}%
\pgfpathlineto{\pgfqpoint{2.258632in}{2.629317in}}%
\pgfpathlineto{\pgfqpoint{2.261767in}{2.633068in}}%
\pgfpathlineto{\pgfqpoint{2.274306in}{2.633068in}}%
\pgfpathlineto{\pgfqpoint{2.277441in}{2.636820in}}%
\pgfpathlineto{\pgfqpoint{2.286845in}{2.636820in}}%
\pgfpathlineto{\pgfqpoint{2.289980in}{2.640571in}}%
\pgfpathlineto{\pgfqpoint{2.299384in}{2.640571in}}%
\pgfpathlineto{\pgfqpoint{2.302519in}{2.644323in}}%
\pgfpathlineto{\pgfqpoint{2.311923in}{2.644323in}}%
\pgfpathlineto{\pgfqpoint{2.315058in}{2.648074in}}%
\pgfpathlineto{\pgfqpoint{2.324462in}{2.648074in}}%
\pgfpathlineto{\pgfqpoint{2.327597in}{2.651826in}}%
\pgfpathlineto{\pgfqpoint{2.340136in}{2.651826in}}%
\pgfpathlineto{\pgfqpoint{2.343270in}{2.655577in}}%
\pgfpathlineto{\pgfqpoint{2.352675in}{2.655577in}}%
\pgfpathlineto{\pgfqpoint{2.355809in}{2.659329in}}%
\pgfpathlineto{\pgfqpoint{2.365214in}{2.659329in}}%
\pgfpathlineto{\pgfqpoint{2.368348in}{2.663080in}}%
\pgfpathlineto{\pgfqpoint{2.377753in}{2.663080in}}%
\pgfpathlineto{\pgfqpoint{2.380887in}{2.666832in}}%
\pgfpathlineto{\pgfqpoint{2.393426in}{2.666832in}}%
\pgfpathlineto{\pgfqpoint{2.396561in}{2.670583in}}%
\pgfpathlineto{\pgfqpoint{2.405965in}{2.670583in}}%
\pgfpathlineto{\pgfqpoint{2.409100in}{2.674335in}}%
\pgfpathlineto{\pgfqpoint{2.418504in}{2.674335in}}%
\pgfpathlineto{\pgfqpoint{2.421639in}{2.678087in}}%
\pgfpathlineto{\pgfqpoint{2.431043in}{2.678087in}}%
\pgfpathlineto{\pgfqpoint{2.434178in}{2.681838in}}%
\pgfpathlineto{\pgfqpoint{2.446717in}{2.681838in}}%
\pgfpathlineto{\pgfqpoint{2.449852in}{2.685590in}}%
\pgfpathlineto{\pgfqpoint{2.459256in}{2.685590in}}%
\pgfpathlineto{\pgfqpoint{2.462391in}{2.689341in}}%
\pgfpathlineto{\pgfqpoint{2.471795in}{2.689341in}}%
\pgfpathlineto{\pgfqpoint{2.474930in}{2.693093in}}%
\pgfpathlineto{\pgfqpoint{2.484334in}{2.693093in}}%
\pgfpathlineto{\pgfqpoint{2.487469in}{2.696844in}}%
\pgfpathlineto{\pgfqpoint{2.496873in}{2.696844in}}%
\pgfpathlineto{\pgfqpoint{2.500008in}{2.700596in}}%
\pgfpathlineto{\pgfqpoint{2.512547in}{2.700596in}}%
\pgfpathlineto{\pgfqpoint{2.515681in}{2.704347in}}%
\pgfpathlineto{\pgfqpoint{2.525085in}{2.704347in}}%
\pgfpathlineto{\pgfqpoint{2.528220in}{2.708099in}}%
\pgfpathlineto{\pgfqpoint{2.537624in}{2.708099in}}%
\pgfpathlineto{\pgfqpoint{2.540759in}{2.711850in}}%
\pgfpathlineto{\pgfqpoint{2.550163in}{2.711850in}}%
\pgfpathlineto{\pgfqpoint{2.553298in}{2.715602in}}%
\pgfpathlineto{\pgfqpoint{2.565837in}{2.715602in}}%
\pgfpathlineto{\pgfqpoint{2.568972in}{2.719353in}}%
\pgfpathlineto{\pgfqpoint{2.578376in}{2.719353in}}%
\pgfpathlineto{\pgfqpoint{2.581511in}{2.723105in}}%
\pgfpathlineto{\pgfqpoint{2.590915in}{2.723105in}}%
\pgfpathlineto{\pgfqpoint{2.594050in}{2.726856in}}%
\pgfpathlineto{\pgfqpoint{2.603454in}{2.726856in}}%
\pgfpathlineto{\pgfqpoint{2.606589in}{2.730608in}}%
\pgfpathlineto{\pgfqpoint{2.615993in}{2.730608in}}%
\pgfpathlineto{\pgfqpoint{2.619128in}{2.734360in}}%
\pgfpathlineto{\pgfqpoint{2.631667in}{2.734360in}}%
\pgfpathlineto{\pgfqpoint{2.634801in}{2.738111in}}%
\pgfpathlineto{\pgfqpoint{2.644206in}{2.738111in}}%
\pgfpathlineto{\pgfqpoint{2.647340in}{2.741863in}}%
\pgfpathlineto{\pgfqpoint{2.656745in}{2.741863in}}%
\pgfpathlineto{\pgfqpoint{2.659879in}{2.745614in}}%
\pgfpathlineto{\pgfqpoint{2.672889in}{2.745051in}}%
\pgfpathlineto{\pgfqpoint{2.710505in}{2.700033in}}%
\pgfpathlineto{\pgfqpoint{2.710505in}{2.696281in}}%
\pgfpathlineto{\pgfqpoint{2.760661in}{2.636257in}}%
\pgfpathlineto{\pgfqpoint{2.760661in}{2.632505in}}%
\pgfpathlineto{\pgfqpoint{2.807682in}{2.576232in}}%
\pgfpathlineto{\pgfqpoint{2.807682in}{2.572481in}}%
\pgfpathlineto{\pgfqpoint{2.857838in}{2.512456in}}%
\pgfpathlineto{\pgfqpoint{2.857838in}{2.508705in}}%
\pgfpathlineto{\pgfqpoint{2.904860in}{2.452432in}}%
\pgfpathlineto{\pgfqpoint{2.904860in}{2.448680in}}%
\pgfpathlineto{\pgfqpoint{2.955015in}{2.388656in}}%
\pgfpathlineto{\pgfqpoint{2.955015in}{2.384904in}}%
\pgfpathlineto{\pgfqpoint{3.002037in}{2.328631in}}%
\pgfpathlineto{\pgfqpoint{3.002037in}{2.324880in}}%
\pgfpathlineto{\pgfqpoint{3.052192in}{2.264855in}}%
\pgfpathlineto{\pgfqpoint{3.052192in}{2.261103in}}%
\pgfpathlineto{\pgfqpoint{3.092944in}{2.212334in}}%
\pgfpathlineto{\pgfqpoint{3.092474in}{2.208019in}}%
\pgfpathlineto{\pgfqpoint{3.089809in}{2.204830in}}%
\pgfpathlineto{\pgfqpoint{3.089339in}{2.196765in}}%
\pgfpathlineto{\pgfqpoint{3.086675in}{2.193576in}}%
\pgfpathlineto{\pgfqpoint{3.086204in}{2.185510in}}%
\pgfpathlineto{\pgfqpoint{3.083540in}{2.182321in}}%
\pgfpathlineto{\pgfqpoint{3.083070in}{2.174255in}}%
\pgfpathlineto{\pgfqpoint{3.080405in}{2.171067in}}%
\pgfpathlineto{\pgfqpoint{3.079935in}{2.163001in}}%
\pgfpathlineto{\pgfqpoint{3.077270in}{2.159812in}}%
\pgfpathlineto{\pgfqpoint{3.076800in}{2.147995in}}%
\pgfpathlineto{\pgfqpoint{3.074136in}{2.144806in}}%
\pgfpathlineto{\pgfqpoint{3.073665in}{2.136740in}}%
\pgfpathlineto{\pgfqpoint{3.071001in}{2.133551in}}%
\pgfpathlineto{\pgfqpoint{3.070531in}{2.125485in}}%
\pgfpathlineto{\pgfqpoint{3.067866in}{2.122297in}}%
\pgfpathlineto{\pgfqpoint{3.067396in}{2.114231in}}%
\pgfpathlineto{\pgfqpoint{3.064731in}{2.111042in}}%
\pgfpathlineto{\pgfqpoint{3.064261in}{2.102976in}}%
\pgfpathlineto{\pgfqpoint{3.061597in}{2.099787in}}%
\pgfpathlineto{\pgfqpoint{3.061126in}{2.091722in}}%
\pgfpathlineto{\pgfqpoint{3.058462in}{2.088533in}}%
\pgfpathlineto{\pgfqpoint{3.057992in}{2.080467in}}%
\pgfpathlineto{\pgfqpoint{3.055327in}{2.077278in}}%
\pgfpathlineto{\pgfqpoint{3.054857in}{2.069212in}}%
\pgfpathlineto{\pgfqpoint{3.052192in}{2.066024in}}%
\pgfpathlineto{\pgfqpoint{3.051722in}{2.057958in}}%
\pgfpathlineto{\pgfqpoint{3.049058in}{2.054769in}}%
\pgfpathlineto{\pgfqpoint{3.048587in}{2.042952in}}%
\pgfpathlineto{\pgfqpoint{3.045923in}{2.039763in}}%
\pgfpathlineto{\pgfqpoint{3.045453in}{2.031697in}}%
\pgfpathlineto{\pgfqpoint{3.042788in}{2.028508in}}%
\pgfpathlineto{\pgfqpoint{3.042318in}{2.020443in}}%
\pgfpathlineto{\pgfqpoint{3.039653in}{2.017254in}}%
\pgfpathlineto{\pgfqpoint{3.039183in}{2.009188in}}%
\pgfpathlineto{\pgfqpoint{3.036519in}{2.005999in}}%
\pgfpathlineto{\pgfqpoint{3.036048in}{1.997933in}}%
\pgfpathlineto{\pgfqpoint{3.033384in}{1.994745in}}%
\pgfpathlineto{\pgfqpoint{3.032914in}{1.986679in}}%
\pgfpathlineto{\pgfqpoint{3.030249in}{1.983490in}}%
\pgfpathlineto{\pgfqpoint{3.029779in}{1.975424in}}%
\pgfpathlineto{\pgfqpoint{3.027114in}{1.972235in}}%
\pgfpathlineto{\pgfqpoint{3.026644in}{1.964169in}}%
\pgfpathlineto{\pgfqpoint{3.023980in}{1.960981in}}%
\pgfpathlineto{\pgfqpoint{3.023510in}{1.949163in}}%
\pgfpathlineto{\pgfqpoint{3.020845in}{1.945975in}}%
\pgfpathlineto{\pgfqpoint{3.020375in}{1.937909in}}%
\pgfpathlineto{\pgfqpoint{3.017710in}{1.934720in}}%
\pgfpathlineto{\pgfqpoint{3.017240in}{1.926654in}}%
\pgfpathlineto{\pgfqpoint{3.014575in}{1.923465in}}%
\pgfpathlineto{\pgfqpoint{3.014105in}{1.915400in}}%
\pgfpathlineto{\pgfqpoint{3.011441in}{1.912211in}}%
\pgfpathlineto{\pgfqpoint{3.010971in}{1.904145in}}%
\pgfpathlineto{\pgfqpoint{3.008306in}{1.900956in}}%
\pgfpathlineto{\pgfqpoint{3.007836in}{1.892890in}}%
\pgfpathlineto{\pgfqpoint{3.005171in}{1.889702in}}%
\pgfpathlineto{\pgfqpoint{3.004701in}{1.881636in}}%
\pgfpathlineto{\pgfqpoint{3.002037in}{1.878447in}}%
\pgfpathlineto{\pgfqpoint{3.001566in}{1.870381in}}%
\pgfpathlineto{\pgfqpoint{2.998902in}{1.867192in}}%
\pgfpathlineto{\pgfqpoint{2.998432in}{1.859127in}}%
\pgfpathlineto{\pgfqpoint{2.995767in}{1.855938in}}%
\pgfpathlineto{\pgfqpoint{2.995297in}{1.844120in}}%
\pgfpathlineto{\pgfqpoint{2.992632in}{1.840932in}}%
\pgfpathlineto{\pgfqpoint{2.992162in}{1.832866in}}%
\pgfpathlineto{\pgfqpoint{2.989498in}{1.829677in}}%
\pgfpathlineto{\pgfqpoint{2.989027in}{1.821611in}}%
\pgfpathlineto{\pgfqpoint{2.986363in}{1.818422in}}%
\pgfpathlineto{\pgfqpoint{2.985893in}{1.810357in}}%
\pgfpathlineto{\pgfqpoint{2.983228in}{1.807168in}}%
\pgfpathlineto{\pgfqpoint{2.982758in}{1.799102in}}%
\pgfpathlineto{\pgfqpoint{2.980093in}{1.795913in}}%
\pgfpathlineto{\pgfqpoint{2.979623in}{1.787847in}}%
\pgfpathlineto{\pgfqpoint{2.976959in}{1.784659in}}%
\pgfpathlineto{\pgfqpoint{2.976488in}{1.776593in}}%
\pgfpathlineto{\pgfqpoint{2.973824in}{1.773404in}}%
\pgfpathlineto{\pgfqpoint{2.973354in}{1.765338in}}%
\pgfpathlineto{\pgfqpoint{2.970689in}{1.762149in}}%
\pgfpathlineto{\pgfqpoint{2.970219in}{1.750332in}}%
\pgfpathlineto{\pgfqpoint{2.967554in}{1.747143in}}%
\pgfpathlineto{\pgfqpoint{2.967084in}{1.739077in}}%
\pgfpathlineto{\pgfqpoint{2.964420in}{1.735889in}}%
\pgfpathlineto{\pgfqpoint{2.963949in}{1.727823in}}%
\pgfpathlineto{\pgfqpoint{2.961285in}{1.724634in}}%
\pgfpathlineto{\pgfqpoint{2.960815in}{1.716568in}}%
\pgfpathlineto{\pgfqpoint{2.958150in}{1.713379in}}%
\pgfpathlineto{\pgfqpoint{2.957680in}{1.705314in}}%
\pgfpathlineto{\pgfqpoint{2.955015in}{1.702125in}}%
\pgfpathlineto{\pgfqpoint{2.954545in}{1.694059in}}%
\pgfpathlineto{\pgfqpoint{2.951881in}{1.690870in}}%
\pgfpathlineto{\pgfqpoint{2.951410in}{1.682804in}}%
\pgfpathlineto{\pgfqpoint{2.948746in}{1.679616in}}%
\pgfpathlineto{\pgfqpoint{2.948276in}{1.671550in}}%
\pgfpathlineto{\pgfqpoint{2.945611in}{1.668361in}}%
\pgfpathlineto{\pgfqpoint{2.945141in}{1.660295in}}%
\pgfpathlineto{\pgfqpoint{2.942476in}{1.657106in}}%
\pgfpathlineto{\pgfqpoint{2.942006in}{1.645289in}}%
\pgfpathlineto{\pgfqpoint{2.939342in}{1.642100in}}%
\pgfpathlineto{\pgfqpoint{2.938871in}{1.634034in}}%
\pgfpathlineto{\pgfqpoint{2.936207in}{1.630846in}}%
\pgfpathlineto{\pgfqpoint{2.935737in}{1.622780in}}%
\pgfpathlineto{\pgfqpoint{2.933072in}{1.619591in}}%
\pgfpathlineto{\pgfqpoint{2.932602in}{1.611525in}}%
\pgfpathlineto{\pgfqpoint{2.929937in}{1.608336in}}%
\pgfpathlineto{\pgfqpoint{2.929467in}{1.600271in}}%
\pgfpathlineto{\pgfqpoint{2.926803in}{1.597082in}}%
\pgfpathlineto{\pgfqpoint{2.926332in}{1.589016in}}%
\pgfpathlineto{\pgfqpoint{2.923668in}{1.585827in}}%
\pgfpathlineto{\pgfqpoint{2.923198in}{1.577761in}}%
\pgfpathlineto{\pgfqpoint{2.920533in}{1.574573in}}%
\pgfpathlineto{\pgfqpoint{2.920063in}{1.566507in}}%
\pgfpathlineto{\pgfqpoint{2.917398in}{1.563318in}}%
\pgfpathlineto{\pgfqpoint{2.916928in}{1.551501in}}%
\pgfpathlineto{\pgfqpoint{2.914264in}{1.548312in}}%
\pgfpathlineto{\pgfqpoint{2.913794in}{1.540246in}}%
\pgfpathlineto{\pgfqpoint{2.911129in}{1.537057in}}%
\pgfpathlineto{\pgfqpoint{2.910659in}{1.528991in}}%
\pgfpathlineto{\pgfqpoint{2.907994in}{1.525803in}}%
\pgfpathlineto{\pgfqpoint{2.907524in}{1.517737in}}%
\pgfpathlineto{\pgfqpoint{2.904860in}{1.514548in}}%
\pgfpathlineto{\pgfqpoint{2.904389in}{1.506482in}}%
\pgfpathlineto{\pgfqpoint{2.901725in}{1.503293in}}%
\pgfpathlineto{\pgfqpoint{2.901255in}{1.495228in}}%
\pgfpathlineto{\pgfqpoint{2.898590in}{1.492039in}}%
\pgfpathlineto{\pgfqpoint{2.898120in}{1.483973in}}%
\pgfpathlineto{\pgfqpoint{2.895455in}{1.480784in}}%
\pgfpathlineto{\pgfqpoint{2.894985in}{1.472718in}}%
\pgfpathlineto{\pgfqpoint{2.892321in}{1.469530in}}%
\pgfpathlineto{\pgfqpoint{2.891850in}{1.461464in}}%
\pgfpathlineto{\pgfqpoint{2.889186in}{1.458275in}}%
\pgfpathlineto{\pgfqpoint{2.888716in}{1.446458in}}%
\pgfpathlineto{\pgfqpoint{2.886051in}{1.443269in}}%
\pgfpathlineto{\pgfqpoint{2.885581in}{1.435203in}}%
\pgfpathlineto{\pgfqpoint{2.882916in}{1.432014in}}%
\pgfpathlineto{\pgfqpoint{2.882446in}{1.423949in}}%
\pgfpathlineto{\pgfqpoint{2.879782in}{1.420760in}}%
\pgfpathlineto{\pgfqpoint{2.879311in}{1.412694in}}%
\pgfpathlineto{\pgfqpoint{2.876647in}{1.409505in}}%
\pgfpathlineto{\pgfqpoint{2.876177in}{1.401439in}}%
\pgfpathlineto{\pgfqpoint{2.873512in}{1.398251in}}%
\pgfpathlineto{\pgfqpoint{2.873042in}{1.390185in}}%
\pgfpathlineto{\pgfqpoint{2.870377in}{1.386996in}}%
\pgfpathlineto{\pgfqpoint{2.869907in}{1.378930in}}%
\pgfpathlineto{\pgfqpoint{2.867243in}{1.375741in}}%
\pgfpathlineto{\pgfqpoint{2.866772in}{1.367676in}}%
\pgfpathlineto{\pgfqpoint{2.864108in}{1.364487in}}%
\pgfpathlineto{\pgfqpoint{2.863638in}{1.352669in}}%
\pgfpathlineto{\pgfqpoint{2.860973in}{1.349481in}}%
\pgfpathlineto{\pgfqpoint{2.860503in}{1.341415in}}%
\pgfpathlineto{\pgfqpoint{2.857838in}{1.338226in}}%
\pgfpathlineto{\pgfqpoint{2.857368in}{1.330160in}}%
\pgfpathlineto{\pgfqpoint{2.854704in}{1.326971in}}%
\pgfpathlineto{\pgfqpoint{2.854233in}{1.318906in}}%
\pgfpathlineto{\pgfqpoint{2.851569in}{1.315717in}}%
\pgfpathlineto{\pgfqpoint{2.851099in}{1.307651in}}%
\pgfpathlineto{\pgfqpoint{2.848434in}{1.304462in}}%
\pgfpathlineto{\pgfqpoint{2.847964in}{1.296396in}}%
\pgfpathlineto{\pgfqpoint{2.845299in}{1.293208in}}%
\pgfpathlineto{\pgfqpoint{2.844829in}{1.285142in}}%
\pgfpathlineto{\pgfqpoint{2.842165in}{1.281953in}}%
\pgfpathlineto{\pgfqpoint{2.841694in}{1.273887in}}%
\pgfpathlineto{\pgfqpoint{2.839030in}{1.270698in}}%
\pgfpathlineto{\pgfqpoint{2.838560in}{1.262633in}}%
\pgfpathlineto{\pgfqpoint{2.835895in}{1.259444in}}%
\pgfpathlineto{\pgfqpoint{2.835425in}{1.247626in}}%
\pgfpathlineto{\pgfqpoint{2.832760in}{1.244438in}}%
\pgfpathlineto{\pgfqpoint{2.832290in}{1.236372in}}%
\pgfpathlineto{\pgfqpoint{2.829626in}{1.233183in}}%
\pgfpathlineto{\pgfqpoint{2.829155in}{1.225117in}}%
\pgfpathlineto{\pgfqpoint{2.826491in}{1.221928in}}%
\pgfpathlineto{\pgfqpoint{2.826021in}{1.213863in}}%
\pgfpathlineto{\pgfqpoint{2.823356in}{1.210674in}}%
\pgfpathlineto{\pgfqpoint{2.822886in}{1.202608in}}%
\pgfpathlineto{\pgfqpoint{2.820221in}{1.199419in}}%
\pgfpathlineto{\pgfqpoint{2.819751in}{1.191353in}}%
\pgfpathlineto{\pgfqpoint{2.817087in}{1.188165in}}%
\pgfpathlineto{\pgfqpoint{2.816617in}{1.180099in}}%
\pgfpathlineto{\pgfqpoint{2.813952in}{1.176910in}}%
\pgfpathlineto{\pgfqpoint{2.813482in}{1.172596in}}%
\pgfpathlineto{\pgfqpoint{2.807212in}{1.172596in}}%
\pgfpathlineto{\pgfqpoint{2.804078in}{1.168844in}}%
\pgfpathlineto{\pgfqpoint{2.797808in}{1.168844in}}%
\pgfpathlineto{\pgfqpoint{2.794673in}{1.165093in}}%
\pgfpathlineto{\pgfqpoint{2.791539in}{1.165093in}}%
\pgfpathlineto{\pgfqpoint{2.788404in}{1.161341in}}%
\pgfpathlineto{\pgfqpoint{2.782134in}{1.161341in}}%
\pgfpathlineto{\pgfqpoint{2.779000in}{1.157590in}}%
\pgfpathlineto{\pgfqpoint{2.772730in}{1.157590in}}%
\pgfpathlineto{\pgfqpoint{2.769595in}{1.153838in}}%
\pgfpathlineto{\pgfqpoint{2.763326in}{1.153838in}}%
\pgfpathlineto{\pgfqpoint{2.760191in}{1.150087in}}%
\pgfpathlineto{\pgfqpoint{2.757056in}{1.150087in}}%
\pgfpathlineto{\pgfqpoint{2.753922in}{1.146335in}}%
\pgfpathlineto{\pgfqpoint{2.747652in}{1.146335in}}%
\pgfpathlineto{\pgfqpoint{2.744517in}{1.142583in}}%
\pgfpathlineto{\pgfqpoint{2.738248in}{1.142583in}}%
\pgfpathlineto{\pgfqpoint{2.735113in}{1.138832in}}%
\pgfpathlineto{\pgfqpoint{2.728844in}{1.138832in}}%
\pgfpathlineto{\pgfqpoint{2.725709in}{1.135080in}}%
\pgfpathlineto{\pgfqpoint{2.719439in}{1.135080in}}%
\pgfpathlineto{\pgfqpoint{2.716305in}{1.131329in}}%
\pgfpathlineto{\pgfqpoint{2.713170in}{1.131329in}}%
\pgfpathlineto{\pgfqpoint{2.710035in}{1.127577in}}%
\pgfpathlineto{\pgfqpoint{2.703766in}{1.127577in}}%
\pgfpathlineto{\pgfqpoint{2.700631in}{1.123826in}}%
\pgfpathlineto{\pgfqpoint{2.694362in}{1.123826in}}%
\pgfpathlineto{\pgfqpoint{2.691227in}{1.120074in}}%
\pgfpathlineto{\pgfqpoint{2.684957in}{1.120074in}}%
\pgfpathlineto{\pgfqpoint{2.681823in}{1.116323in}}%
\pgfpathlineto{\pgfqpoint{2.678688in}{1.116323in}}%
\pgfpathlineto{\pgfqpoint{2.675553in}{1.112571in}}%
\pgfpathlineto{\pgfqpoint{2.669284in}{1.112571in}}%
\pgfpathlineto{\pgfqpoint{2.666149in}{1.108820in}}%
\pgfpathlineto{\pgfqpoint{2.659879in}{1.108820in}}%
\pgfpathlineto{\pgfqpoint{2.656745in}{1.105068in}}%
\pgfpathlineto{\pgfqpoint{2.650475in}{1.105068in}}%
\pgfpathlineto{\pgfqpoint{2.647340in}{1.101317in}}%
\pgfpathlineto{\pgfqpoint{2.644206in}{1.101317in}}%
\pgfpathlineto{\pgfqpoint{2.641071in}{1.097565in}}%
\pgfpathlineto{\pgfqpoint{2.634801in}{1.097565in}}%
\pgfpathlineto{\pgfqpoint{2.631667in}{1.093813in}}%
\pgfpathlineto{\pgfqpoint{2.625397in}{1.093813in}}%
\pgfpathlineto{\pgfqpoint{2.622262in}{1.090062in}}%
\pgfpathlineto{\pgfqpoint{2.615993in}{1.090062in}}%
\pgfpathlineto{\pgfqpoint{2.612858in}{1.086310in}}%
\pgfpathlineto{\pgfqpoint{2.606589in}{1.086310in}}%
\pgfpathlineto{\pgfqpoint{2.603454in}{1.082559in}}%
\pgfpathlineto{\pgfqpoint{2.600319in}{1.082559in}}%
\pgfpathlineto{\pgfqpoint{2.597185in}{1.078807in}}%
\pgfpathlineto{\pgfqpoint{2.590915in}{1.078807in}}%
\pgfpathlineto{\pgfqpoint{2.587780in}{1.075056in}}%
\pgfpathlineto{\pgfqpoint{2.581511in}{1.075056in}}%
\pgfpathlineto{\pgfqpoint{2.578376in}{1.071304in}}%
\pgfpathlineto{\pgfqpoint{2.572107in}{1.071304in}}%
\pgfpathlineto{\pgfqpoint{2.568972in}{1.067553in}}%
\pgfpathlineto{\pgfqpoint{2.565837in}{1.067553in}}%
\pgfpathlineto{\pgfqpoint{2.562702in}{1.063801in}}%
\pgfpathlineto{\pgfqpoint{2.556433in}{1.063801in}}%
\pgfpathlineto{\pgfqpoint{2.553298in}{1.060050in}}%
\pgfpathlineto{\pgfqpoint{2.547029in}{1.060050in}}%
\pgfpathlineto{\pgfqpoint{2.543894in}{1.056298in}}%
\pgfpathlineto{\pgfqpoint{2.537624in}{1.056298in}}%
\pgfpathlineto{\pgfqpoint{2.534490in}{1.052547in}}%
\pgfpathlineto{\pgfqpoint{2.531355in}{1.052547in}}%
\pgfpathlineto{\pgfqpoint{2.528220in}{1.048795in}}%
\pgfpathlineto{\pgfqpoint{2.521951in}{1.048795in}}%
\pgfpathlineto{\pgfqpoint{2.518816in}{1.045044in}}%
\pgfpathlineto{\pgfqpoint{2.512547in}{1.045044in}}%
\pgfpathlineto{\pgfqpoint{2.509412in}{1.041292in}}%
\pgfpathlineto{\pgfqpoint{2.503142in}{1.041292in}}%
\pgfpathlineto{\pgfqpoint{2.500008in}{1.037540in}}%
\pgfpathlineto{\pgfqpoint{2.493738in}{1.037540in}}%
\pgfpathlineto{\pgfqpoint{2.490603in}{1.033789in}}%
\pgfpathlineto{\pgfqpoint{2.487469in}{1.033789in}}%
\pgfpathlineto{\pgfqpoint{2.484334in}{1.030037in}}%
\pgfpathlineto{\pgfqpoint{2.478064in}{1.030037in}}%
\pgfpathlineto{\pgfqpoint{2.474930in}{1.026286in}}%
\pgfpathlineto{\pgfqpoint{2.468660in}{1.026286in}}%
\pgfpathlineto{\pgfqpoint{2.465525in}{1.022534in}}%
\pgfpathlineto{\pgfqpoint{2.459256in}{1.022534in}}%
\pgfpathlineto{\pgfqpoint{2.456121in}{1.018783in}}%
\pgfpathlineto{\pgfqpoint{2.452986in}{1.018783in}}%
\pgfpathlineto{\pgfqpoint{2.449852in}{1.015031in}}%
\pgfpathlineto{\pgfqpoint{2.443582in}{1.015031in}}%
\pgfpathlineto{\pgfqpoint{2.440447in}{1.011280in}}%
\pgfpathlineto{\pgfqpoint{2.434178in}{1.011280in}}%
\pgfpathlineto{\pgfqpoint{2.431043in}{1.007528in}}%
\pgfpathlineto{\pgfqpoint{2.424774in}{1.007528in}}%
\pgfpathlineto{\pgfqpoint{2.421639in}{1.003777in}}%
\pgfpathlineto{\pgfqpoint{2.418504in}{1.003777in}}%
\pgfpathlineto{\pgfqpoint{2.415369in}{1.000025in}}%
\pgfpathlineto{\pgfqpoint{2.409100in}{1.000025in}}%
\pgfpathlineto{\pgfqpoint{2.405965in}{0.996274in}}%
\pgfpathlineto{\pgfqpoint{2.399696in}{0.996274in}}%
\pgfpathlineto{\pgfqpoint{2.396561in}{0.992522in}}%
\pgfpathlineto{\pgfqpoint{2.390292in}{0.992522in}}%
\pgfpathlineto{\pgfqpoint{2.387157in}{0.988771in}}%
\pgfpathlineto{\pgfqpoint{2.380887in}{0.988771in}}%
\pgfpathlineto{\pgfqpoint{2.377753in}{0.985019in}}%
\pgfpathlineto{\pgfqpoint{2.374618in}{0.985019in}}%
\pgfpathlineto{\pgfqpoint{2.371483in}{0.981267in}}%
\pgfpathlineto{\pgfqpoint{2.365214in}{0.981267in}}%
\pgfpathlineto{\pgfqpoint{2.362079in}{0.977516in}}%
\pgfpathlineto{\pgfqpoint{2.355809in}{0.977516in}}%
\pgfpathlineto{\pgfqpoint{2.352675in}{0.973764in}}%
\pgfpathlineto{\pgfqpoint{2.346405in}{0.973764in}}%
\pgfpathlineto{\pgfqpoint{2.343270in}{0.970013in}}%
\pgfpathlineto{\pgfqpoint{2.340136in}{0.970013in}}%
\pgfpathlineto{\pgfqpoint{2.337001in}{0.966261in}}%
\pgfpathlineto{\pgfqpoint{2.330731in}{0.966261in}}%
\pgfpathlineto{\pgfqpoint{2.327597in}{0.962510in}}%
\pgfpathlineto{\pgfqpoint{2.321327in}{0.962510in}}%
\pgfpathlineto{\pgfqpoint{2.318192in}{0.958758in}}%
\pgfpathlineto{\pgfqpoint{2.311923in}{0.958758in}}%
\pgfpathlineto{\pgfqpoint{2.308788in}{0.955007in}}%
\pgfpathlineto{\pgfqpoint{2.305654in}{0.955007in}}%
\pgfpathlineto{\pgfqpoint{2.302519in}{0.951255in}}%
\pgfpathlineto{\pgfqpoint{2.296249in}{0.951255in}}%
\pgfpathlineto{\pgfqpoint{2.293115in}{0.947504in}}%
\pgfpathlineto{\pgfqpoint{2.286845in}{0.947504in}}%
\pgfpathlineto{\pgfqpoint{2.283710in}{0.943752in}}%
\pgfpathlineto{\pgfqpoint{2.277441in}{0.943752in}}%
\pgfpathlineto{\pgfqpoint{2.274306in}{0.940001in}}%
\pgfpathlineto{\pgfqpoint{2.271171in}{0.940001in}}%
\pgfpathlineto{\pgfqpoint{2.268037in}{0.936249in}}%
\pgfpathlineto{\pgfqpoint{2.261767in}{0.936249in}}%
\pgfpathlineto{\pgfqpoint{2.258632in}{0.932498in}}%
\pgfpathlineto{\pgfqpoint{2.252363in}{0.932498in}}%
\pgfpathlineto{\pgfqpoint{2.249228in}{0.928746in}}%
\pgfpathlineto{\pgfqpoint{2.242959in}{0.928746in}}%
\pgfpathlineto{\pgfqpoint{2.239824in}{0.924994in}}%
\pgfpathlineto{\pgfqpoint{2.233554in}{0.924994in}}%
\pgfpathlineto{\pgfqpoint{2.230420in}{0.921243in}}%
\pgfpathlineto{\pgfqpoint{2.227285in}{0.921243in}}%
\pgfpathlineto{\pgfqpoint{2.224150in}{0.917491in}}%
\pgfpathlineto{\pgfqpoint{2.217881in}{0.917491in}}%
\pgfpathlineto{\pgfqpoint{2.214746in}{0.913740in}}%
\pgfpathlineto{\pgfqpoint{2.208477in}{0.913740in}}%
\pgfpathlineto{\pgfqpoint{2.205342in}{0.909988in}}%
\pgfpathlineto{\pgfqpoint{2.199072in}{0.909988in}}%
\pgfpathlineto{\pgfqpoint{2.195938in}{0.906237in}}%
\pgfpathlineto{\pgfqpoint{2.192803in}{0.906237in}}%
\pgfpathlineto{\pgfqpoint{2.189668in}{0.902485in}}%
\pgfpathlineto{\pgfqpoint{2.183399in}{0.902485in}}%
\pgfpathlineto{\pgfqpoint{2.180264in}{0.898734in}}%
\pgfpathlineto{\pgfqpoint{2.173994in}{0.898734in}}%
\pgfpathlineto{\pgfqpoint{2.173994in}{0.898734in}}%
\pgfusepath{stroke}%
\end{pgfscope}%
\begin{pgfscope}%
\pgfpathrectangle{\pgfqpoint{0.888750in}{0.419100in}}{\pgfqpoint{2.504659in}{2.933700in}} %
\pgfusepath{clip}%
\pgfsetbuttcap%
\pgfsetroundjoin%
\pgfsetlinewidth{1.505625pt}%
\definecolor{currentstroke}{rgb}{0.887059,0.887059,0.887059}%
\pgfsetstrokecolor{currentstroke}%
\pgfsetdash{}{0pt}%
\pgfpathmoveto{\pgfqpoint{2.173994in}{0.898171in}}%
\pgfpathlineto{\pgfqpoint{2.167725in}{0.905674in}}%
\pgfpathlineto{\pgfqpoint{2.164590in}{0.905674in}}%
\pgfpathlineto{\pgfqpoint{2.152051in}{0.920680in}}%
\pgfpathlineto{\pgfqpoint{2.148916in}{0.920680in}}%
\pgfpathlineto{\pgfqpoint{2.139512in}{0.931935in}}%
\pgfpathlineto{\pgfqpoint{2.136377in}{0.931935in}}%
\pgfpathlineto{\pgfqpoint{2.126973in}{0.943189in}}%
\pgfpathlineto{\pgfqpoint{2.123838in}{0.943189in}}%
\pgfpathlineto{\pgfqpoint{2.114434in}{0.954444in}}%
\pgfpathlineto{\pgfqpoint{2.111299in}{0.954444in}}%
\pgfpathlineto{\pgfqpoint{2.101895in}{0.965699in}}%
\pgfpathlineto{\pgfqpoint{2.098761in}{0.965699in}}%
\pgfpathlineto{\pgfqpoint{2.086222in}{0.980705in}}%
\pgfpathlineto{\pgfqpoint{2.083087in}{0.980705in}}%
\pgfpathlineto{\pgfqpoint{2.073683in}{0.991959in}}%
\pgfpathlineto{\pgfqpoint{2.070548in}{0.991959in}}%
\pgfpathlineto{\pgfqpoint{2.061144in}{1.003214in}}%
\pgfpathlineto{\pgfqpoint{2.058009in}{1.003214in}}%
\pgfpathlineto{\pgfqpoint{2.048605in}{1.014469in}}%
\pgfpathlineto{\pgfqpoint{2.045470in}{1.014469in}}%
\pgfpathlineto{\pgfqpoint{2.036066in}{1.025723in}}%
\pgfpathlineto{\pgfqpoint{2.032931in}{1.025723in}}%
\pgfpathlineto{\pgfqpoint{2.023527in}{1.036978in}}%
\pgfpathlineto{\pgfqpoint{2.020392in}{1.036978in}}%
\pgfpathlineto{\pgfqpoint{2.007853in}{1.051984in}}%
\pgfpathlineto{\pgfqpoint{2.004718in}{1.051984in}}%
\pgfpathlineto{\pgfqpoint{1.995314in}{1.063238in}}%
\pgfpathlineto{\pgfqpoint{1.992179in}{1.063238in}}%
\pgfpathlineto{\pgfqpoint{1.982775in}{1.074493in}}%
\pgfpathlineto{\pgfqpoint{1.979640in}{1.074493in}}%
\pgfpathlineto{\pgfqpoint{1.970236in}{1.085748in}}%
\pgfpathlineto{\pgfqpoint{1.967101in}{1.085748in}}%
\pgfpathlineto{\pgfqpoint{1.957697in}{1.097002in}}%
\pgfpathlineto{\pgfqpoint{1.954562in}{1.097002in}}%
\pgfpathlineto{\pgfqpoint{1.945158in}{1.108257in}}%
\pgfpathlineto{\pgfqpoint{1.942023in}{1.108257in}}%
\pgfpathlineto{\pgfqpoint{1.929484in}{1.123263in}}%
\pgfpathlineto{\pgfqpoint{1.926350in}{1.123263in}}%
\pgfpathlineto{\pgfqpoint{1.916945in}{1.134518in}}%
\pgfpathlineto{\pgfqpoint{1.913811in}{1.134518in}}%
\pgfpathlineto{\pgfqpoint{1.904407in}{1.145772in}}%
\pgfpathlineto{\pgfqpoint{1.901272in}{1.145772in}}%
\pgfpathlineto{\pgfqpoint{1.891868in}{1.157027in}}%
\pgfpathlineto{\pgfqpoint{1.888733in}{1.157027in}}%
\pgfpathlineto{\pgfqpoint{1.879329in}{1.168281in}}%
\pgfpathlineto{\pgfqpoint{1.876194in}{1.168281in}}%
\pgfpathlineto{\pgfqpoint{1.866790in}{1.179536in}}%
\pgfpathlineto{\pgfqpoint{1.863655in}{1.179536in}}%
\pgfpathlineto{\pgfqpoint{1.851116in}{1.194542in}}%
\pgfpathlineto{\pgfqpoint{1.847981in}{1.194542in}}%
\pgfpathlineto{\pgfqpoint{1.838577in}{1.205797in}}%
\pgfpathlineto{\pgfqpoint{1.835442in}{1.205797in}}%
\pgfpathlineto{\pgfqpoint{1.826038in}{1.217051in}}%
\pgfpathlineto{\pgfqpoint{1.822903in}{1.217051in}}%
\pgfpathlineto{\pgfqpoint{1.813499in}{1.228306in}}%
\pgfpathlineto{\pgfqpoint{1.810364in}{1.228306in}}%
\pgfpathlineto{\pgfqpoint{1.800960in}{1.239561in}}%
\pgfpathlineto{\pgfqpoint{1.797825in}{1.239561in}}%
\pgfpathlineto{\pgfqpoint{1.785286in}{1.254567in}}%
\pgfpathlineto{\pgfqpoint{1.782152in}{1.254567in}}%
\pgfpathlineto{\pgfqpoint{1.772747in}{1.265821in}}%
\pgfpathlineto{\pgfqpoint{1.769613in}{1.265821in}}%
\pgfpathlineto{\pgfqpoint{1.760208in}{1.277076in}}%
\pgfpathlineto{\pgfqpoint{1.757074in}{1.277076in}}%
\pgfpathlineto{\pgfqpoint{1.747669in}{1.288331in}}%
\pgfpathlineto{\pgfqpoint{1.744535in}{1.288331in}}%
\pgfpathlineto{\pgfqpoint{1.735130in}{1.299585in}}%
\pgfpathlineto{\pgfqpoint{1.731996in}{1.299585in}}%
\pgfpathlineto{\pgfqpoint{1.722591in}{1.310840in}}%
\pgfpathlineto{\pgfqpoint{1.719457in}{1.310840in}}%
\pgfpathlineto{\pgfqpoint{1.706918in}{1.325846in}}%
\pgfpathlineto{\pgfqpoint{1.703783in}{1.325846in}}%
\pgfpathlineto{\pgfqpoint{1.694379in}{1.337100in}}%
\pgfpathlineto{\pgfqpoint{1.691244in}{1.337100in}}%
\pgfpathlineto{\pgfqpoint{1.681840in}{1.348355in}}%
\pgfpathlineto{\pgfqpoint{1.678705in}{1.348355in}}%
\pgfpathlineto{\pgfqpoint{1.669301in}{1.359610in}}%
\pgfpathlineto{\pgfqpoint{1.666166in}{1.359610in}}%
\pgfpathlineto{\pgfqpoint{1.656762in}{1.370864in}}%
\pgfpathlineto{\pgfqpoint{1.653627in}{1.370864in}}%
\pgfpathlineto{\pgfqpoint{1.644223in}{1.382119in}}%
\pgfpathlineto{\pgfqpoint{1.641088in}{1.382119in}}%
\pgfpathlineto{\pgfqpoint{1.628549in}{1.397125in}}%
\pgfpathlineto{\pgfqpoint{1.625414in}{1.397125in}}%
\pgfpathlineto{\pgfqpoint{1.616010in}{1.408380in}}%
\pgfpathlineto{\pgfqpoint{1.612875in}{1.408380in}}%
\pgfpathlineto{\pgfqpoint{1.603471in}{1.419634in}}%
\pgfpathlineto{\pgfqpoint{1.600337in}{1.419634in}}%
\pgfpathlineto{\pgfqpoint{1.590932in}{1.430889in}}%
\pgfpathlineto{\pgfqpoint{1.587798in}{1.430889in}}%
\pgfpathlineto{\pgfqpoint{1.578393in}{1.442143in}}%
\pgfpathlineto{\pgfqpoint{1.575259in}{1.442143in}}%
\pgfpathlineto{\pgfqpoint{1.565854in}{1.453398in}}%
\pgfpathlineto{\pgfqpoint{1.562720in}{1.453398in}}%
\pgfpathlineto{\pgfqpoint{1.550181in}{1.468404in}}%
\pgfpathlineto{\pgfqpoint{1.547046in}{1.468404in}}%
\pgfpathlineto{\pgfqpoint{1.537642in}{1.479659in}}%
\pgfpathlineto{\pgfqpoint{1.534507in}{1.479659in}}%
\pgfpathlineto{\pgfqpoint{1.525103in}{1.490913in}}%
\pgfpathlineto{\pgfqpoint{1.521968in}{1.490913in}}%
\pgfpathlineto{\pgfqpoint{1.512564in}{1.502168in}}%
\pgfpathlineto{\pgfqpoint{1.509429in}{1.502168in}}%
\pgfpathlineto{\pgfqpoint{1.500025in}{1.513423in}}%
\pgfpathlineto{\pgfqpoint{1.496890in}{1.513423in}}%
\pgfpathlineto{\pgfqpoint{1.484351in}{1.528429in}}%
\pgfpathlineto{\pgfqpoint{1.481216in}{1.528429in}}%
\pgfpathlineto{\pgfqpoint{1.471812in}{1.539683in}}%
\pgfpathlineto{\pgfqpoint{1.468677in}{1.539683in}}%
\pgfpathlineto{\pgfqpoint{1.459273in}{1.550938in}}%
\pgfpathlineto{\pgfqpoint{1.456138in}{1.550938in}}%
\pgfpathlineto{\pgfqpoint{1.446734in}{1.562193in}}%
\pgfpathlineto{\pgfqpoint{1.443599in}{1.562193in}}%
\pgfpathlineto{\pgfqpoint{1.434195in}{1.573447in}}%
\pgfpathlineto{\pgfqpoint{1.431060in}{1.573447in}}%
\pgfpathlineto{\pgfqpoint{1.421656in}{1.584702in}}%
\pgfpathlineto{\pgfqpoint{1.418521in}{1.584702in}}%
\pgfpathlineto{\pgfqpoint{1.405982in}{1.599708in}}%
\pgfpathlineto{\pgfqpoint{1.402848in}{1.599708in}}%
\pgfpathlineto{\pgfqpoint{1.393444in}{1.610963in}}%
\pgfpathlineto{\pgfqpoint{1.390309in}{1.610963in}}%
\pgfpathlineto{\pgfqpoint{1.380905in}{1.622217in}}%
\pgfpathlineto{\pgfqpoint{1.377770in}{1.622217in}}%
\pgfpathlineto{\pgfqpoint{1.368366in}{1.633472in}}%
\pgfpathlineto{\pgfqpoint{1.365231in}{1.633472in}}%
\pgfpathlineto{\pgfqpoint{1.355827in}{1.644726in}}%
\pgfpathlineto{\pgfqpoint{1.352692in}{1.644726in}}%
\pgfpathlineto{\pgfqpoint{1.343288in}{1.655981in}}%
\pgfpathlineto{\pgfqpoint{1.340153in}{1.655981in}}%
\pgfpathlineto{\pgfqpoint{1.327614in}{1.670987in}}%
\pgfpathlineto{\pgfqpoint{1.324479in}{1.670987in}}%
\pgfpathlineto{\pgfqpoint{1.315075in}{1.682242in}}%
\pgfpathlineto{\pgfqpoint{1.311940in}{1.682242in}}%
\pgfpathlineto{\pgfqpoint{1.302536in}{1.693496in}}%
\pgfpathlineto{\pgfqpoint{1.299401in}{1.693496in}}%
\pgfpathlineto{\pgfqpoint{1.292191in}{1.702125in}}%
\pgfpathlineto{\pgfqpoint{1.293132in}{1.707002in}}%
\pgfpathlineto{\pgfqpoint{1.295326in}{1.709628in}}%
\pgfpathlineto{\pgfqpoint{1.296266in}{1.722008in}}%
\pgfpathlineto{\pgfqpoint{1.298461in}{1.724634in}}%
\pgfpathlineto{\pgfqpoint{1.299401in}{1.733263in}}%
\pgfpathlineto{\pgfqpoint{1.301596in}{1.735889in}}%
\pgfpathlineto{\pgfqpoint{1.302536in}{1.748269in}}%
\pgfpathlineto{\pgfqpoint{1.304730in}{1.750895in}}%
\pgfpathlineto{\pgfqpoint{1.305671in}{1.759523in}}%
\pgfpathlineto{\pgfqpoint{1.307865in}{1.762149in}}%
\pgfpathlineto{\pgfqpoint{1.308805in}{1.774529in}}%
\pgfpathlineto{\pgfqpoint{1.311000in}{1.777155in}}%
\pgfpathlineto{\pgfqpoint{1.311940in}{1.785784in}}%
\pgfpathlineto{\pgfqpoint{1.314135in}{1.788410in}}%
\pgfpathlineto{\pgfqpoint{1.315075in}{1.800790in}}%
\pgfpathlineto{\pgfqpoint{1.317269in}{1.803416in}}%
\pgfpathlineto{\pgfqpoint{1.318210in}{1.812045in}}%
\pgfpathlineto{\pgfqpoint{1.320404in}{1.814671in}}%
\pgfpathlineto{\pgfqpoint{1.321344in}{1.823299in}}%
\pgfpathlineto{\pgfqpoint{1.323539in}{1.825925in}}%
\pgfpathlineto{\pgfqpoint{1.324479in}{1.838306in}}%
\pgfpathlineto{\pgfqpoint{1.326673in}{1.840932in}}%
\pgfpathlineto{\pgfqpoint{1.327614in}{1.849560in}}%
\pgfpathlineto{\pgfqpoint{1.329808in}{1.852186in}}%
\pgfpathlineto{\pgfqpoint{1.330749in}{1.864566in}}%
\pgfpathlineto{\pgfqpoint{1.332943in}{1.867192in}}%
\pgfpathlineto{\pgfqpoint{1.333883in}{1.875821in}}%
\pgfpathlineto{\pgfqpoint{1.336078in}{1.878447in}}%
\pgfpathlineto{\pgfqpoint{1.337018in}{1.890827in}}%
\pgfpathlineto{\pgfqpoint{1.339212in}{1.893453in}}%
\pgfpathlineto{\pgfqpoint{1.340153in}{1.902082in}}%
\pgfpathlineto{\pgfqpoint{1.342347in}{1.904708in}}%
\pgfpathlineto{\pgfqpoint{1.343288in}{1.917088in}}%
\pgfpathlineto{\pgfqpoint{1.345482in}{1.919714in}}%
\pgfpathlineto{\pgfqpoint{1.346422in}{1.928342in}}%
\pgfpathlineto{\pgfqpoint{1.348617in}{1.930968in}}%
\pgfpathlineto{\pgfqpoint{1.349557in}{1.943348in}}%
\pgfpathlineto{\pgfqpoint{1.351751in}{1.945975in}}%
\pgfpathlineto{\pgfqpoint{1.352692in}{1.954603in}}%
\pgfpathlineto{\pgfqpoint{1.354886in}{1.957229in}}%
\pgfpathlineto{\pgfqpoint{1.355827in}{1.969609in}}%
\pgfpathlineto{\pgfqpoint{1.358021in}{1.972235in}}%
\pgfpathlineto{\pgfqpoint{1.358961in}{1.980864in}}%
\pgfpathlineto{\pgfqpoint{1.361156in}{1.983490in}}%
\pgfpathlineto{\pgfqpoint{1.362096in}{1.995870in}}%
\pgfpathlineto{\pgfqpoint{1.364290in}{1.998496in}}%
\pgfpathlineto{\pgfqpoint{1.365231in}{2.007125in}}%
\pgfpathlineto{\pgfqpoint{1.367425in}{2.009751in}}%
\pgfpathlineto{\pgfqpoint{1.368366in}{2.022131in}}%
\pgfpathlineto{\pgfqpoint{1.370560in}{2.024757in}}%
\pgfpathlineto{\pgfqpoint{1.371500in}{2.033385in}}%
\pgfpathlineto{\pgfqpoint{1.373695in}{2.036011in}}%
\pgfpathlineto{\pgfqpoint{1.374635in}{2.044640in}}%
\pgfpathlineto{\pgfqpoint{1.376829in}{2.047266in}}%
\pgfpathlineto{\pgfqpoint{1.377770in}{2.059646in}}%
\pgfpathlineto{\pgfqpoint{1.379964in}{2.062272in}}%
\pgfpathlineto{\pgfqpoint{1.380905in}{2.070901in}}%
\pgfpathlineto{\pgfqpoint{1.383099in}{2.073527in}}%
\pgfpathlineto{\pgfqpoint{1.384039in}{2.085907in}}%
\pgfpathlineto{\pgfqpoint{1.386234in}{2.088533in}}%
\pgfpathlineto{\pgfqpoint{1.387174in}{2.097161in}}%
\pgfpathlineto{\pgfqpoint{1.389368in}{2.099787in}}%
\pgfpathlineto{\pgfqpoint{1.390309in}{2.112168in}}%
\pgfpathlineto{\pgfqpoint{1.392503in}{2.114794in}}%
\pgfpathlineto{\pgfqpoint{1.393444in}{2.123422in}}%
\pgfpathlineto{\pgfqpoint{1.395638in}{2.126048in}}%
\pgfpathlineto{\pgfqpoint{1.396578in}{2.138428in}}%
\pgfpathlineto{\pgfqpoint{1.398773in}{2.141054in}}%
\pgfpathlineto{\pgfqpoint{1.399713in}{2.149683in}}%
\pgfpathlineto{\pgfqpoint{1.401907in}{2.152309in}}%
\pgfpathlineto{\pgfqpoint{1.402848in}{2.164689in}}%
\pgfpathlineto{\pgfqpoint{1.405042in}{2.167315in}}%
\pgfpathlineto{\pgfqpoint{1.405982in}{2.175944in}}%
\pgfpathlineto{\pgfqpoint{1.408177in}{2.178570in}}%
\pgfpathlineto{\pgfqpoint{1.409117in}{2.190950in}}%
\pgfpathlineto{\pgfqpoint{1.411312in}{2.193576in}}%
\pgfpathlineto{\pgfqpoint{1.412252in}{2.202204in}}%
\pgfpathlineto{\pgfqpoint{1.414446in}{2.204830in}}%
\pgfpathlineto{\pgfqpoint{1.415387in}{2.217210in}}%
\pgfpathlineto{\pgfqpoint{1.417581in}{2.219837in}}%
\pgfpathlineto{\pgfqpoint{1.418521in}{2.228465in}}%
\pgfpathlineto{\pgfqpoint{1.420716in}{2.231091in}}%
\pgfpathlineto{\pgfqpoint{1.421656in}{2.243471in}}%
\pgfpathlineto{\pgfqpoint{1.423851in}{2.246097in}}%
\pgfpathlineto{\pgfqpoint{1.424791in}{2.254726in}}%
\pgfpathlineto{\pgfqpoint{1.426985in}{2.257352in}}%
\pgfpathlineto{\pgfqpoint{1.427926in}{2.265980in}}%
\pgfpathlineto{\pgfqpoint{1.430120in}{2.268607in}}%
\pgfpathlineto{\pgfqpoint{1.431060in}{2.280987in}}%
\pgfpathlineto{\pgfqpoint{1.433255in}{2.283613in}}%
\pgfpathlineto{\pgfqpoint{1.434195in}{2.292241in}}%
\pgfpathlineto{\pgfqpoint{1.436389in}{2.294867in}}%
\pgfpathlineto{\pgfqpoint{1.437330in}{2.307247in}}%
\pgfpathlineto{\pgfqpoint{1.439524in}{2.309873in}}%
\pgfpathlineto{\pgfqpoint{1.440465in}{2.318502in}}%
\pgfpathlineto{\pgfqpoint{1.442659in}{2.321128in}}%
\pgfpathlineto{\pgfqpoint{1.443599in}{2.333508in}}%
\pgfpathlineto{\pgfqpoint{1.445794in}{2.336134in}}%
\pgfpathlineto{\pgfqpoint{1.446734in}{2.344763in}}%
\pgfpathlineto{\pgfqpoint{1.448928in}{2.347389in}}%
\pgfpathlineto{\pgfqpoint{1.449869in}{2.359769in}}%
\pgfpathlineto{\pgfqpoint{1.452063in}{2.362395in}}%
\pgfpathlineto{\pgfqpoint{1.453004in}{2.371023in}}%
\pgfpathlineto{\pgfqpoint{1.455198in}{2.373649in}}%
\pgfpathlineto{\pgfqpoint{1.456138in}{2.386030in}}%
\pgfpathlineto{\pgfqpoint{1.458333in}{2.388656in}}%
\pgfpathlineto{\pgfqpoint{1.459273in}{2.397284in}}%
\pgfpathlineto{\pgfqpoint{1.461467in}{2.399910in}}%
\pgfpathlineto{\pgfqpoint{1.462408in}{2.404787in}}%
\pgfpathlineto{\pgfqpoint{1.465543in}{2.404787in}}%
\pgfpathlineto{\pgfqpoint{1.468677in}{2.408539in}}%
\pgfpathlineto{\pgfqpoint{1.478082in}{2.408539in}}%
\pgfpathlineto{\pgfqpoint{1.481216in}{2.412290in}}%
\pgfpathlineto{\pgfqpoint{1.490621in}{2.412290in}}%
\pgfpathlineto{\pgfqpoint{1.493755in}{2.416042in}}%
\pgfpathlineto{\pgfqpoint{1.503159in}{2.416042in}}%
\pgfpathlineto{\pgfqpoint{1.506294in}{2.419793in}}%
\pgfpathlineto{\pgfqpoint{1.515698in}{2.419793in}}%
\pgfpathlineto{\pgfqpoint{1.518833in}{2.423545in}}%
\pgfpathlineto{\pgfqpoint{1.531372in}{2.423545in}}%
\pgfpathlineto{\pgfqpoint{1.534507in}{2.427296in}}%
\pgfpathlineto{\pgfqpoint{1.543911in}{2.427296in}}%
\pgfpathlineto{\pgfqpoint{1.547046in}{2.431048in}}%
\pgfpathlineto{\pgfqpoint{1.556450in}{2.431048in}}%
\pgfpathlineto{\pgfqpoint{1.559585in}{2.434800in}}%
\pgfpathlineto{\pgfqpoint{1.568989in}{2.434800in}}%
\pgfpathlineto{\pgfqpoint{1.572124in}{2.438551in}}%
\pgfpathlineto{\pgfqpoint{1.584663in}{2.438551in}}%
\pgfpathlineto{\pgfqpoint{1.587798in}{2.442303in}}%
\pgfpathlineto{\pgfqpoint{1.597202in}{2.442303in}}%
\pgfpathlineto{\pgfqpoint{1.600337in}{2.446054in}}%
\pgfpathlineto{\pgfqpoint{1.609741in}{2.446054in}}%
\pgfpathlineto{\pgfqpoint{1.612875in}{2.449806in}}%
\pgfpathlineto{\pgfqpoint{1.622280in}{2.449806in}}%
\pgfpathlineto{\pgfqpoint{1.625414in}{2.453557in}}%
\pgfpathlineto{\pgfqpoint{1.637953in}{2.453557in}}%
\pgfpathlineto{\pgfqpoint{1.641088in}{2.457309in}}%
\pgfpathlineto{\pgfqpoint{1.650492in}{2.457309in}}%
\pgfpathlineto{\pgfqpoint{1.653627in}{2.461060in}}%
\pgfpathlineto{\pgfqpoint{1.663031in}{2.461060in}}%
\pgfpathlineto{\pgfqpoint{1.666166in}{2.464812in}}%
\pgfpathlineto{\pgfqpoint{1.675570in}{2.464812in}}%
\pgfpathlineto{\pgfqpoint{1.678705in}{2.468563in}}%
\pgfpathlineto{\pgfqpoint{1.688109in}{2.468563in}}%
\pgfpathlineto{\pgfqpoint{1.691244in}{2.472315in}}%
\pgfpathlineto{\pgfqpoint{1.703783in}{2.472315in}}%
\pgfpathlineto{\pgfqpoint{1.706918in}{2.476066in}}%
\pgfpathlineto{\pgfqpoint{1.716322in}{2.476066in}}%
\pgfpathlineto{\pgfqpoint{1.719457in}{2.479818in}}%
\pgfpathlineto{\pgfqpoint{1.728861in}{2.479818in}}%
\pgfpathlineto{\pgfqpoint{1.731996in}{2.483569in}}%
\pgfpathlineto{\pgfqpoint{1.741400in}{2.483569in}}%
\pgfpathlineto{\pgfqpoint{1.744535in}{2.487321in}}%
\pgfpathlineto{\pgfqpoint{1.757074in}{2.487321in}}%
\pgfpathlineto{\pgfqpoint{1.760208in}{2.491073in}}%
\pgfpathlineto{\pgfqpoint{1.769613in}{2.491073in}}%
\pgfpathlineto{\pgfqpoint{1.772747in}{2.494824in}}%
\pgfpathlineto{\pgfqpoint{1.782152in}{2.494824in}}%
\pgfpathlineto{\pgfqpoint{1.785286in}{2.498576in}}%
\pgfpathlineto{\pgfqpoint{1.794691in}{2.498576in}}%
\pgfpathlineto{\pgfqpoint{1.797825in}{2.502327in}}%
\pgfpathlineto{\pgfqpoint{1.810364in}{2.502327in}}%
\pgfpathlineto{\pgfqpoint{1.813499in}{2.506079in}}%
\pgfpathlineto{\pgfqpoint{1.822903in}{2.506079in}}%
\pgfpathlineto{\pgfqpoint{1.826038in}{2.509830in}}%
\pgfpathlineto{\pgfqpoint{1.835442in}{2.509830in}}%
\pgfpathlineto{\pgfqpoint{1.838577in}{2.513582in}}%
\pgfpathlineto{\pgfqpoint{1.847981in}{2.513582in}}%
\pgfpathlineto{\pgfqpoint{1.851116in}{2.517333in}}%
\pgfpathlineto{\pgfqpoint{1.860520in}{2.517333in}}%
\pgfpathlineto{\pgfqpoint{1.863655in}{2.521085in}}%
\pgfpathlineto{\pgfqpoint{1.876194in}{2.521085in}}%
\pgfpathlineto{\pgfqpoint{1.879329in}{2.524836in}}%
\pgfpathlineto{\pgfqpoint{1.888733in}{2.524836in}}%
\pgfpathlineto{\pgfqpoint{1.891868in}{2.528588in}}%
\pgfpathlineto{\pgfqpoint{1.901272in}{2.528588in}}%
\pgfpathlineto{\pgfqpoint{1.904407in}{2.532339in}}%
\pgfpathlineto{\pgfqpoint{1.913811in}{2.532339in}}%
\pgfpathlineto{\pgfqpoint{1.916945in}{2.536091in}}%
\pgfpathlineto{\pgfqpoint{1.929484in}{2.536091in}}%
\pgfpathlineto{\pgfqpoint{1.932619in}{2.539842in}}%
\pgfpathlineto{\pgfqpoint{1.942023in}{2.539842in}}%
\pgfpathlineto{\pgfqpoint{1.945158in}{2.543594in}}%
\pgfpathlineto{\pgfqpoint{1.954562in}{2.543594in}}%
\pgfpathlineto{\pgfqpoint{1.957697in}{2.547346in}}%
\pgfpathlineto{\pgfqpoint{1.967101in}{2.547346in}}%
\pgfpathlineto{\pgfqpoint{1.970236in}{2.551097in}}%
\pgfpathlineto{\pgfqpoint{1.979640in}{2.551097in}}%
\pgfpathlineto{\pgfqpoint{1.982775in}{2.554849in}}%
\pgfpathlineto{\pgfqpoint{1.995314in}{2.554849in}}%
\pgfpathlineto{\pgfqpoint{1.998449in}{2.558600in}}%
\pgfpathlineto{\pgfqpoint{2.007853in}{2.558600in}}%
\pgfpathlineto{\pgfqpoint{2.010988in}{2.562352in}}%
\pgfpathlineto{\pgfqpoint{2.020392in}{2.562352in}}%
\pgfpathlineto{\pgfqpoint{2.023527in}{2.566103in}}%
\pgfpathlineto{\pgfqpoint{2.032931in}{2.566103in}}%
\pgfpathlineto{\pgfqpoint{2.036066in}{2.569855in}}%
\pgfpathlineto{\pgfqpoint{2.048605in}{2.569855in}}%
\pgfpathlineto{\pgfqpoint{2.051739in}{2.573606in}}%
\pgfpathlineto{\pgfqpoint{2.061144in}{2.573606in}}%
\pgfpathlineto{\pgfqpoint{2.064278in}{2.577358in}}%
\pgfpathlineto{\pgfqpoint{2.073683in}{2.577358in}}%
\pgfpathlineto{\pgfqpoint{2.076817in}{2.581109in}}%
\pgfpathlineto{\pgfqpoint{2.086222in}{2.581109in}}%
\pgfpathlineto{\pgfqpoint{2.089356in}{2.584861in}}%
\pgfpathlineto{\pgfqpoint{2.101895in}{2.584861in}}%
\pgfpathlineto{\pgfqpoint{2.105030in}{2.588612in}}%
\pgfpathlineto{\pgfqpoint{2.114434in}{2.588612in}}%
\pgfpathlineto{\pgfqpoint{2.117569in}{2.592364in}}%
\pgfpathlineto{\pgfqpoint{2.126973in}{2.592364in}}%
\pgfpathlineto{\pgfqpoint{2.130108in}{2.596115in}}%
\pgfpathlineto{\pgfqpoint{2.139512in}{2.596115in}}%
\pgfpathlineto{\pgfqpoint{2.142647in}{2.599867in}}%
\pgfpathlineto{\pgfqpoint{2.152051in}{2.599867in}}%
\pgfpathlineto{\pgfqpoint{2.155186in}{2.603619in}}%
\pgfpathlineto{\pgfqpoint{2.167725in}{2.603619in}}%
\pgfpathlineto{\pgfqpoint{2.170860in}{2.607370in}}%
\pgfpathlineto{\pgfqpoint{2.180264in}{2.607370in}}%
\pgfpathlineto{\pgfqpoint{2.183399in}{2.611122in}}%
\pgfpathlineto{\pgfqpoint{2.192803in}{2.611122in}}%
\pgfpathlineto{\pgfqpoint{2.195938in}{2.614873in}}%
\pgfpathlineto{\pgfqpoint{2.205342in}{2.614873in}}%
\pgfpathlineto{\pgfqpoint{2.208477in}{2.618625in}}%
\pgfpathlineto{\pgfqpoint{2.221015in}{2.618625in}}%
\pgfpathlineto{\pgfqpoint{2.224150in}{2.622376in}}%
\pgfpathlineto{\pgfqpoint{2.233554in}{2.622376in}}%
\pgfpathlineto{\pgfqpoint{2.236689in}{2.626128in}}%
\pgfpathlineto{\pgfqpoint{2.246093in}{2.626128in}}%
\pgfpathlineto{\pgfqpoint{2.249228in}{2.629879in}}%
\pgfpathlineto{\pgfqpoint{2.258632in}{2.629879in}}%
\pgfpathlineto{\pgfqpoint{2.261767in}{2.633631in}}%
\pgfpathlineto{\pgfqpoint{2.274306in}{2.633631in}}%
\pgfpathlineto{\pgfqpoint{2.277441in}{2.637382in}}%
\pgfpathlineto{\pgfqpoint{2.286845in}{2.637382in}}%
\pgfpathlineto{\pgfqpoint{2.289980in}{2.641134in}}%
\pgfpathlineto{\pgfqpoint{2.299384in}{2.641134in}}%
\pgfpathlineto{\pgfqpoint{2.302519in}{2.644885in}}%
\pgfpathlineto{\pgfqpoint{2.311923in}{2.644885in}}%
\pgfpathlineto{\pgfqpoint{2.315058in}{2.648637in}}%
\pgfpathlineto{\pgfqpoint{2.324462in}{2.648637in}}%
\pgfpathlineto{\pgfqpoint{2.327597in}{2.652389in}}%
\pgfpathlineto{\pgfqpoint{2.340136in}{2.652389in}}%
\pgfpathlineto{\pgfqpoint{2.343270in}{2.656140in}}%
\pgfpathlineto{\pgfqpoint{2.352675in}{2.656140in}}%
\pgfpathlineto{\pgfqpoint{2.355809in}{2.659892in}}%
\pgfpathlineto{\pgfqpoint{2.365214in}{2.659892in}}%
\pgfpathlineto{\pgfqpoint{2.368348in}{2.663643in}}%
\pgfpathlineto{\pgfqpoint{2.377753in}{2.663643in}}%
\pgfpathlineto{\pgfqpoint{2.380887in}{2.667395in}}%
\pgfpathlineto{\pgfqpoint{2.393426in}{2.667395in}}%
\pgfpathlineto{\pgfqpoint{2.396561in}{2.671146in}}%
\pgfpathlineto{\pgfqpoint{2.405965in}{2.671146in}}%
\pgfpathlineto{\pgfqpoint{2.409100in}{2.674898in}}%
\pgfpathlineto{\pgfqpoint{2.418504in}{2.674898in}}%
\pgfpathlineto{\pgfqpoint{2.421639in}{2.678649in}}%
\pgfpathlineto{\pgfqpoint{2.431043in}{2.678649in}}%
\pgfpathlineto{\pgfqpoint{2.434178in}{2.682401in}}%
\pgfpathlineto{\pgfqpoint{2.446717in}{2.682401in}}%
\pgfpathlineto{\pgfqpoint{2.449852in}{2.686152in}}%
\pgfpathlineto{\pgfqpoint{2.459256in}{2.686152in}}%
\pgfpathlineto{\pgfqpoint{2.462391in}{2.689904in}}%
\pgfpathlineto{\pgfqpoint{2.471795in}{2.689904in}}%
\pgfpathlineto{\pgfqpoint{2.474930in}{2.693655in}}%
\pgfpathlineto{\pgfqpoint{2.484334in}{2.693655in}}%
\pgfpathlineto{\pgfqpoint{2.487469in}{2.697407in}}%
\pgfpathlineto{\pgfqpoint{2.496873in}{2.697407in}}%
\pgfpathlineto{\pgfqpoint{2.500008in}{2.701158in}}%
\pgfpathlineto{\pgfqpoint{2.512547in}{2.701158in}}%
\pgfpathlineto{\pgfqpoint{2.515681in}{2.704910in}}%
\pgfpathlineto{\pgfqpoint{2.525085in}{2.704910in}}%
\pgfpathlineto{\pgfqpoint{2.528220in}{2.708662in}}%
\pgfpathlineto{\pgfqpoint{2.537624in}{2.708662in}}%
\pgfpathlineto{\pgfqpoint{2.540759in}{2.712413in}}%
\pgfpathlineto{\pgfqpoint{2.550163in}{2.712413in}}%
\pgfpathlineto{\pgfqpoint{2.553298in}{2.716165in}}%
\pgfpathlineto{\pgfqpoint{2.565837in}{2.716165in}}%
\pgfpathlineto{\pgfqpoint{2.568972in}{2.719916in}}%
\pgfpathlineto{\pgfqpoint{2.578376in}{2.719916in}}%
\pgfpathlineto{\pgfqpoint{2.581511in}{2.723668in}}%
\pgfpathlineto{\pgfqpoint{2.590915in}{2.723668in}}%
\pgfpathlineto{\pgfqpoint{2.594050in}{2.727419in}}%
\pgfpathlineto{\pgfqpoint{2.603454in}{2.727419in}}%
\pgfpathlineto{\pgfqpoint{2.606589in}{2.731171in}}%
\pgfpathlineto{\pgfqpoint{2.615993in}{2.731171in}}%
\pgfpathlineto{\pgfqpoint{2.619128in}{2.734922in}}%
\pgfpathlineto{\pgfqpoint{2.631667in}{2.734922in}}%
\pgfpathlineto{\pgfqpoint{2.634801in}{2.738674in}}%
\pgfpathlineto{\pgfqpoint{2.644206in}{2.738674in}}%
\pgfpathlineto{\pgfqpoint{2.647340in}{2.742425in}}%
\pgfpathlineto{\pgfqpoint{2.656745in}{2.742425in}}%
\pgfpathlineto{\pgfqpoint{2.659879in}{2.746177in}}%
\pgfpathlineto{\pgfqpoint{2.672418in}{2.746177in}}%
\pgfpathlineto{\pgfqpoint{2.710976in}{2.700033in}}%
\pgfpathlineto{\pgfqpoint{2.710976in}{2.696281in}}%
\pgfpathlineto{\pgfqpoint{2.761132in}{2.636257in}}%
\pgfpathlineto{\pgfqpoint{2.761132in}{2.632505in}}%
\pgfpathlineto{\pgfqpoint{2.808153in}{2.576232in}}%
\pgfpathlineto{\pgfqpoint{2.808153in}{2.572481in}}%
\pgfpathlineto{\pgfqpoint{2.858309in}{2.512456in}}%
\pgfpathlineto{\pgfqpoint{2.858309in}{2.508705in}}%
\pgfpathlineto{\pgfqpoint{2.905330in}{2.452432in}}%
\pgfpathlineto{\pgfqpoint{2.905330in}{2.448680in}}%
\pgfpathlineto{\pgfqpoint{2.955486in}{2.388656in}}%
\pgfpathlineto{\pgfqpoint{2.955486in}{2.384904in}}%
\pgfpathlineto{\pgfqpoint{3.002507in}{2.328631in}}%
\pgfpathlineto{\pgfqpoint{3.002507in}{2.324880in}}%
\pgfpathlineto{\pgfqpoint{3.052663in}{2.264855in}}%
\pgfpathlineto{\pgfqpoint{3.052663in}{2.261103in}}%
\pgfpathlineto{\pgfqpoint{3.093414in}{2.212334in}}%
\pgfpathlineto{\pgfqpoint{3.092474in}{2.207457in}}%
\pgfpathlineto{\pgfqpoint{3.090280in}{2.204830in}}%
\pgfpathlineto{\pgfqpoint{3.089339in}{2.196202in}}%
\pgfpathlineto{\pgfqpoint{3.087145in}{2.193576in}}%
\pgfpathlineto{\pgfqpoint{3.086204in}{2.184947in}}%
\pgfpathlineto{\pgfqpoint{3.084010in}{2.182321in}}%
\pgfpathlineto{\pgfqpoint{3.083070in}{2.173693in}}%
\pgfpathlineto{\pgfqpoint{3.080875in}{2.171067in}}%
\pgfpathlineto{\pgfqpoint{3.079935in}{2.162438in}}%
\pgfpathlineto{\pgfqpoint{3.077741in}{2.159812in}}%
\pgfpathlineto{\pgfqpoint{3.076800in}{2.147432in}}%
\pgfpathlineto{\pgfqpoint{3.074606in}{2.144806in}}%
\pgfpathlineto{\pgfqpoint{3.073665in}{2.136177in}}%
\pgfpathlineto{\pgfqpoint{3.071471in}{2.133551in}}%
\pgfpathlineto{\pgfqpoint{3.070531in}{2.124923in}}%
\pgfpathlineto{\pgfqpoint{3.068336in}{2.122297in}}%
\pgfpathlineto{\pgfqpoint{3.067396in}{2.113668in}}%
\pgfpathlineto{\pgfqpoint{3.065202in}{2.111042in}}%
\pgfpathlineto{\pgfqpoint{3.064261in}{2.102414in}}%
\pgfpathlineto{\pgfqpoint{3.062067in}{2.099787in}}%
\pgfpathlineto{\pgfqpoint{3.061126in}{2.091159in}}%
\pgfpathlineto{\pgfqpoint{3.058932in}{2.088533in}}%
\pgfpathlineto{\pgfqpoint{3.057992in}{2.079904in}}%
\pgfpathlineto{\pgfqpoint{3.055797in}{2.077278in}}%
\pgfpathlineto{\pgfqpoint{3.054857in}{2.068650in}}%
\pgfpathlineto{\pgfqpoint{3.052663in}{2.066024in}}%
\pgfpathlineto{\pgfqpoint{3.051722in}{2.057395in}}%
\pgfpathlineto{\pgfqpoint{3.049528in}{2.054769in}}%
\pgfpathlineto{\pgfqpoint{3.048587in}{2.042389in}}%
\pgfpathlineto{\pgfqpoint{3.046393in}{2.039763in}}%
\pgfpathlineto{\pgfqpoint{3.045453in}{2.031134in}}%
\pgfpathlineto{\pgfqpoint{3.043258in}{2.028508in}}%
\pgfpathlineto{\pgfqpoint{3.042318in}{2.019880in}}%
\pgfpathlineto{\pgfqpoint{3.040124in}{2.017254in}}%
\pgfpathlineto{\pgfqpoint{3.039183in}{2.008625in}}%
\pgfpathlineto{\pgfqpoint{3.036989in}{2.005999in}}%
\pgfpathlineto{\pgfqpoint{3.036048in}{1.997371in}}%
\pgfpathlineto{\pgfqpoint{3.033854in}{1.994745in}}%
\pgfpathlineto{\pgfqpoint{3.032914in}{1.986116in}}%
\pgfpathlineto{\pgfqpoint{3.030719in}{1.983490in}}%
\pgfpathlineto{\pgfqpoint{3.029779in}{1.974861in}}%
\pgfpathlineto{\pgfqpoint{3.027585in}{1.972235in}}%
\pgfpathlineto{\pgfqpoint{3.026644in}{1.963607in}}%
\pgfpathlineto{\pgfqpoint{3.024450in}{1.960981in}}%
\pgfpathlineto{\pgfqpoint{3.023510in}{1.948601in}}%
\pgfpathlineto{\pgfqpoint{3.021315in}{1.945975in}}%
\pgfpathlineto{\pgfqpoint{3.020375in}{1.937346in}}%
\pgfpathlineto{\pgfqpoint{3.018180in}{1.934720in}}%
\pgfpathlineto{\pgfqpoint{3.017240in}{1.926091in}}%
\pgfpathlineto{\pgfqpoint{3.015046in}{1.923465in}}%
\pgfpathlineto{\pgfqpoint{3.014105in}{1.914837in}}%
\pgfpathlineto{\pgfqpoint{3.011911in}{1.912211in}}%
\pgfpathlineto{\pgfqpoint{3.010971in}{1.903582in}}%
\pgfpathlineto{\pgfqpoint{3.008776in}{1.900956in}}%
\pgfpathlineto{\pgfqpoint{3.007836in}{1.892328in}}%
\pgfpathlineto{\pgfqpoint{3.005641in}{1.889702in}}%
\pgfpathlineto{\pgfqpoint{3.004701in}{1.881073in}}%
\pgfpathlineto{\pgfqpoint{3.002507in}{1.878447in}}%
\pgfpathlineto{\pgfqpoint{3.001566in}{1.869818in}}%
\pgfpathlineto{\pgfqpoint{2.999372in}{1.867192in}}%
\pgfpathlineto{\pgfqpoint{2.998432in}{1.858564in}}%
\pgfpathlineto{\pgfqpoint{2.996237in}{1.855938in}}%
\pgfpathlineto{\pgfqpoint{2.995297in}{1.843558in}}%
\pgfpathlineto{\pgfqpoint{2.993103in}{1.840932in}}%
\pgfpathlineto{\pgfqpoint{2.992162in}{1.832303in}}%
\pgfpathlineto{\pgfqpoint{2.989968in}{1.829677in}}%
\pgfpathlineto{\pgfqpoint{2.989027in}{1.821048in}}%
\pgfpathlineto{\pgfqpoint{2.986833in}{1.818422in}}%
\pgfpathlineto{\pgfqpoint{2.985893in}{1.809794in}}%
\pgfpathlineto{\pgfqpoint{2.983698in}{1.807168in}}%
\pgfpathlineto{\pgfqpoint{2.982758in}{1.798539in}}%
\pgfpathlineto{\pgfqpoint{2.980564in}{1.795913in}}%
\pgfpathlineto{\pgfqpoint{2.979623in}{1.787285in}}%
\pgfpathlineto{\pgfqpoint{2.977429in}{1.784659in}}%
\pgfpathlineto{\pgfqpoint{2.976488in}{1.776030in}}%
\pgfpathlineto{\pgfqpoint{2.974294in}{1.773404in}}%
\pgfpathlineto{\pgfqpoint{2.973354in}{1.764775in}}%
\pgfpathlineto{\pgfqpoint{2.971159in}{1.762149in}}%
\pgfpathlineto{\pgfqpoint{2.970219in}{1.749769in}}%
\pgfpathlineto{\pgfqpoint{2.968025in}{1.747143in}}%
\pgfpathlineto{\pgfqpoint{2.967084in}{1.738515in}}%
\pgfpathlineto{\pgfqpoint{2.964890in}{1.735889in}}%
\pgfpathlineto{\pgfqpoint{2.963949in}{1.727260in}}%
\pgfpathlineto{\pgfqpoint{2.961755in}{1.724634in}}%
\pgfpathlineto{\pgfqpoint{2.960815in}{1.716005in}}%
\pgfpathlineto{\pgfqpoint{2.958620in}{1.713379in}}%
\pgfpathlineto{\pgfqpoint{2.957680in}{1.704751in}}%
\pgfpathlineto{\pgfqpoint{2.955486in}{1.702125in}}%
\pgfpathlineto{\pgfqpoint{2.954545in}{1.693496in}}%
\pgfpathlineto{\pgfqpoint{2.952351in}{1.690870in}}%
\pgfpathlineto{\pgfqpoint{2.951410in}{1.682242in}}%
\pgfpathlineto{\pgfqpoint{2.949216in}{1.679616in}}%
\pgfpathlineto{\pgfqpoint{2.948276in}{1.670987in}}%
\pgfpathlineto{\pgfqpoint{2.946081in}{1.668361in}}%
\pgfpathlineto{\pgfqpoint{2.945141in}{1.659732in}}%
\pgfpathlineto{\pgfqpoint{2.942947in}{1.657106in}}%
\pgfpathlineto{\pgfqpoint{2.942006in}{1.644726in}}%
\pgfpathlineto{\pgfqpoint{2.939812in}{1.642100in}}%
\pgfpathlineto{\pgfqpoint{2.938871in}{1.633472in}}%
\pgfpathlineto{\pgfqpoint{2.936677in}{1.630846in}}%
\pgfpathlineto{\pgfqpoint{2.935737in}{1.622217in}}%
\pgfpathlineto{\pgfqpoint{2.933542in}{1.619591in}}%
\pgfpathlineto{\pgfqpoint{2.932602in}{1.610963in}}%
\pgfpathlineto{\pgfqpoint{2.930408in}{1.608336in}}%
\pgfpathlineto{\pgfqpoint{2.929467in}{1.599708in}}%
\pgfpathlineto{\pgfqpoint{2.927273in}{1.597082in}}%
\pgfpathlineto{\pgfqpoint{2.926332in}{1.588453in}}%
\pgfpathlineto{\pgfqpoint{2.924138in}{1.585827in}}%
\pgfpathlineto{\pgfqpoint{2.923198in}{1.577199in}}%
\pgfpathlineto{\pgfqpoint{2.921003in}{1.574573in}}%
\pgfpathlineto{\pgfqpoint{2.920063in}{1.565944in}}%
\pgfpathlineto{\pgfqpoint{2.917869in}{1.563318in}}%
\pgfpathlineto{\pgfqpoint{2.916928in}{1.550938in}}%
\pgfpathlineto{\pgfqpoint{2.914734in}{1.548312in}}%
\pgfpathlineto{\pgfqpoint{2.913794in}{1.539683in}}%
\pgfpathlineto{\pgfqpoint{2.911599in}{1.537057in}}%
\pgfpathlineto{\pgfqpoint{2.910659in}{1.528429in}}%
\pgfpathlineto{\pgfqpoint{2.908464in}{1.525803in}}%
\pgfpathlineto{\pgfqpoint{2.907524in}{1.517174in}}%
\pgfpathlineto{\pgfqpoint{2.905330in}{1.514548in}}%
\pgfpathlineto{\pgfqpoint{2.904389in}{1.505920in}}%
\pgfpathlineto{\pgfqpoint{2.902195in}{1.503293in}}%
\pgfpathlineto{\pgfqpoint{2.901255in}{1.494665in}}%
\pgfpathlineto{\pgfqpoint{2.899060in}{1.492039in}}%
\pgfpathlineto{\pgfqpoint{2.898120in}{1.483410in}}%
\pgfpathlineto{\pgfqpoint{2.895925in}{1.480784in}}%
\pgfpathlineto{\pgfqpoint{2.894985in}{1.472156in}}%
\pgfpathlineto{\pgfqpoint{2.892791in}{1.469530in}}%
\pgfpathlineto{\pgfqpoint{2.891850in}{1.460901in}}%
\pgfpathlineto{\pgfqpoint{2.889656in}{1.458275in}}%
\pgfpathlineto{\pgfqpoint{2.888716in}{1.445895in}}%
\pgfpathlineto{\pgfqpoint{2.886521in}{1.443269in}}%
\pgfpathlineto{\pgfqpoint{2.885581in}{1.434640in}}%
\pgfpathlineto{\pgfqpoint{2.883387in}{1.432014in}}%
\pgfpathlineto{\pgfqpoint{2.882446in}{1.423386in}}%
\pgfpathlineto{\pgfqpoint{2.880252in}{1.420760in}}%
\pgfpathlineto{\pgfqpoint{2.879311in}{1.412131in}}%
\pgfpathlineto{\pgfqpoint{2.877117in}{1.409505in}}%
\pgfpathlineto{\pgfqpoint{2.876177in}{1.400877in}}%
\pgfpathlineto{\pgfqpoint{2.873982in}{1.398251in}}%
\pgfpathlineto{\pgfqpoint{2.873042in}{1.389622in}}%
\pgfpathlineto{\pgfqpoint{2.870848in}{1.386996in}}%
\pgfpathlineto{\pgfqpoint{2.869907in}{1.378367in}}%
\pgfpathlineto{\pgfqpoint{2.867713in}{1.375741in}}%
\pgfpathlineto{\pgfqpoint{2.866772in}{1.367113in}}%
\pgfpathlineto{\pgfqpoint{2.864578in}{1.364487in}}%
\pgfpathlineto{\pgfqpoint{2.863638in}{1.352107in}}%
\pgfpathlineto{\pgfqpoint{2.861443in}{1.349481in}}%
\pgfpathlineto{\pgfqpoint{2.860503in}{1.340852in}}%
\pgfpathlineto{\pgfqpoint{2.858309in}{1.338226in}}%
\pgfpathlineto{\pgfqpoint{2.857368in}{1.329597in}}%
\pgfpathlineto{\pgfqpoint{2.855174in}{1.326971in}}%
\pgfpathlineto{\pgfqpoint{2.854233in}{1.318343in}}%
\pgfpathlineto{\pgfqpoint{2.852039in}{1.315717in}}%
\pgfpathlineto{\pgfqpoint{2.851099in}{1.307088in}}%
\pgfpathlineto{\pgfqpoint{2.848904in}{1.304462in}}%
\pgfpathlineto{\pgfqpoint{2.847964in}{1.295834in}}%
\pgfpathlineto{\pgfqpoint{2.845770in}{1.293208in}}%
\pgfpathlineto{\pgfqpoint{2.844829in}{1.284579in}}%
\pgfpathlineto{\pgfqpoint{2.842635in}{1.281953in}}%
\pgfpathlineto{\pgfqpoint{2.841694in}{1.273324in}}%
\pgfpathlineto{\pgfqpoint{2.839500in}{1.270698in}}%
\pgfpathlineto{\pgfqpoint{2.838560in}{1.262070in}}%
\pgfpathlineto{\pgfqpoint{2.836365in}{1.259444in}}%
\pgfpathlineto{\pgfqpoint{2.835425in}{1.247064in}}%
\pgfpathlineto{\pgfqpoint{2.833231in}{1.244438in}}%
\pgfpathlineto{\pgfqpoint{2.832290in}{1.235809in}}%
\pgfpathlineto{\pgfqpoint{2.830096in}{1.233183in}}%
\pgfpathlineto{\pgfqpoint{2.829155in}{1.224554in}}%
\pgfpathlineto{\pgfqpoint{2.826961in}{1.221928in}}%
\pgfpathlineto{\pgfqpoint{2.826021in}{1.213300in}}%
\pgfpathlineto{\pgfqpoint{2.823826in}{1.210674in}}%
\pgfpathlineto{\pgfqpoint{2.822886in}{1.202045in}}%
\pgfpathlineto{\pgfqpoint{2.820692in}{1.199419in}}%
\pgfpathlineto{\pgfqpoint{2.819751in}{1.190791in}}%
\pgfpathlineto{\pgfqpoint{2.817557in}{1.188165in}}%
\pgfpathlineto{\pgfqpoint{2.816617in}{1.179536in}}%
\pgfpathlineto{\pgfqpoint{2.814422in}{1.176910in}}%
\pgfpathlineto{\pgfqpoint{2.813482in}{1.172033in}}%
\pgfpathlineto{\pgfqpoint{2.807212in}{1.172033in}}%
\pgfpathlineto{\pgfqpoint{2.804078in}{1.168281in}}%
\pgfpathlineto{\pgfqpoint{2.797808in}{1.168281in}}%
\pgfpathlineto{\pgfqpoint{2.794673in}{1.164530in}}%
\pgfpathlineto{\pgfqpoint{2.791539in}{1.164530in}}%
\pgfpathlineto{\pgfqpoint{2.788404in}{1.160778in}}%
\pgfpathlineto{\pgfqpoint{2.782134in}{1.160778in}}%
\pgfpathlineto{\pgfqpoint{2.779000in}{1.157027in}}%
\pgfpathlineto{\pgfqpoint{2.772730in}{1.157027in}}%
\pgfpathlineto{\pgfqpoint{2.769595in}{1.153275in}}%
\pgfpathlineto{\pgfqpoint{2.763326in}{1.153275in}}%
\pgfpathlineto{\pgfqpoint{2.760191in}{1.149524in}}%
\pgfpathlineto{\pgfqpoint{2.757056in}{1.149524in}}%
\pgfpathlineto{\pgfqpoint{2.753922in}{1.145772in}}%
\pgfpathlineto{\pgfqpoint{2.747652in}{1.145772in}}%
\pgfpathlineto{\pgfqpoint{2.744517in}{1.142021in}}%
\pgfpathlineto{\pgfqpoint{2.738248in}{1.142021in}}%
\pgfpathlineto{\pgfqpoint{2.735113in}{1.138269in}}%
\pgfpathlineto{\pgfqpoint{2.728844in}{1.138269in}}%
\pgfpathlineto{\pgfqpoint{2.725709in}{1.134518in}}%
\pgfpathlineto{\pgfqpoint{2.719439in}{1.134518in}}%
\pgfpathlineto{\pgfqpoint{2.716305in}{1.130766in}}%
\pgfpathlineto{\pgfqpoint{2.713170in}{1.130766in}}%
\pgfpathlineto{\pgfqpoint{2.710035in}{1.127015in}}%
\pgfpathlineto{\pgfqpoint{2.703766in}{1.127015in}}%
\pgfpathlineto{\pgfqpoint{2.700631in}{1.123263in}}%
\pgfpathlineto{\pgfqpoint{2.694362in}{1.123263in}}%
\pgfpathlineto{\pgfqpoint{2.691227in}{1.119511in}}%
\pgfpathlineto{\pgfqpoint{2.684957in}{1.119511in}}%
\pgfpathlineto{\pgfqpoint{2.681823in}{1.115760in}}%
\pgfpathlineto{\pgfqpoint{2.678688in}{1.115760in}}%
\pgfpathlineto{\pgfqpoint{2.675553in}{1.112008in}}%
\pgfpathlineto{\pgfqpoint{2.669284in}{1.112008in}}%
\pgfpathlineto{\pgfqpoint{2.666149in}{1.108257in}}%
\pgfpathlineto{\pgfqpoint{2.659879in}{1.108257in}}%
\pgfpathlineto{\pgfqpoint{2.656745in}{1.104505in}}%
\pgfpathlineto{\pgfqpoint{2.650475in}{1.104505in}}%
\pgfpathlineto{\pgfqpoint{2.647340in}{1.100754in}}%
\pgfpathlineto{\pgfqpoint{2.644206in}{1.100754in}}%
\pgfpathlineto{\pgfqpoint{2.641071in}{1.097002in}}%
\pgfpathlineto{\pgfqpoint{2.634801in}{1.097002in}}%
\pgfpathlineto{\pgfqpoint{2.631667in}{1.093251in}}%
\pgfpathlineto{\pgfqpoint{2.625397in}{1.093251in}}%
\pgfpathlineto{\pgfqpoint{2.622262in}{1.089499in}}%
\pgfpathlineto{\pgfqpoint{2.615993in}{1.089499in}}%
\pgfpathlineto{\pgfqpoint{2.612858in}{1.085748in}}%
\pgfpathlineto{\pgfqpoint{2.606589in}{1.085748in}}%
\pgfpathlineto{\pgfqpoint{2.603454in}{1.081996in}}%
\pgfpathlineto{\pgfqpoint{2.600319in}{1.081996in}}%
\pgfpathlineto{\pgfqpoint{2.597185in}{1.078245in}}%
\pgfpathlineto{\pgfqpoint{2.590915in}{1.078245in}}%
\pgfpathlineto{\pgfqpoint{2.587780in}{1.074493in}}%
\pgfpathlineto{\pgfqpoint{2.581511in}{1.074493in}}%
\pgfpathlineto{\pgfqpoint{2.578376in}{1.070742in}}%
\pgfpathlineto{\pgfqpoint{2.572107in}{1.070742in}}%
\pgfpathlineto{\pgfqpoint{2.568972in}{1.066990in}}%
\pgfpathlineto{\pgfqpoint{2.565837in}{1.066990in}}%
\pgfpathlineto{\pgfqpoint{2.562702in}{1.063238in}}%
\pgfpathlineto{\pgfqpoint{2.556433in}{1.063238in}}%
\pgfpathlineto{\pgfqpoint{2.553298in}{1.059487in}}%
\pgfpathlineto{\pgfqpoint{2.547029in}{1.059487in}}%
\pgfpathlineto{\pgfqpoint{2.543894in}{1.055735in}}%
\pgfpathlineto{\pgfqpoint{2.537624in}{1.055735in}}%
\pgfpathlineto{\pgfqpoint{2.534490in}{1.051984in}}%
\pgfpathlineto{\pgfqpoint{2.531355in}{1.051984in}}%
\pgfpathlineto{\pgfqpoint{2.528220in}{1.048232in}}%
\pgfpathlineto{\pgfqpoint{2.521951in}{1.048232in}}%
\pgfpathlineto{\pgfqpoint{2.518816in}{1.044481in}}%
\pgfpathlineto{\pgfqpoint{2.512547in}{1.044481in}}%
\pgfpathlineto{\pgfqpoint{2.509412in}{1.040729in}}%
\pgfpathlineto{\pgfqpoint{2.503142in}{1.040729in}}%
\pgfpathlineto{\pgfqpoint{2.500008in}{1.036978in}}%
\pgfpathlineto{\pgfqpoint{2.493738in}{1.036978in}}%
\pgfpathlineto{\pgfqpoint{2.490603in}{1.033226in}}%
\pgfpathlineto{\pgfqpoint{2.487469in}{1.033226in}}%
\pgfpathlineto{\pgfqpoint{2.484334in}{1.029475in}}%
\pgfpathlineto{\pgfqpoint{2.478064in}{1.029475in}}%
\pgfpathlineto{\pgfqpoint{2.474930in}{1.025723in}}%
\pgfpathlineto{\pgfqpoint{2.468660in}{1.025723in}}%
\pgfpathlineto{\pgfqpoint{2.465525in}{1.021972in}}%
\pgfpathlineto{\pgfqpoint{2.459256in}{1.021972in}}%
\pgfpathlineto{\pgfqpoint{2.456121in}{1.018220in}}%
\pgfpathlineto{\pgfqpoint{2.452986in}{1.018220in}}%
\pgfpathlineto{\pgfqpoint{2.449852in}{1.014469in}}%
\pgfpathlineto{\pgfqpoint{2.443582in}{1.014469in}}%
\pgfpathlineto{\pgfqpoint{2.440447in}{1.010717in}}%
\pgfpathlineto{\pgfqpoint{2.434178in}{1.010717in}}%
\pgfpathlineto{\pgfqpoint{2.431043in}{1.006965in}}%
\pgfpathlineto{\pgfqpoint{2.424774in}{1.006965in}}%
\pgfpathlineto{\pgfqpoint{2.421639in}{1.003214in}}%
\pgfpathlineto{\pgfqpoint{2.418504in}{1.003214in}}%
\pgfpathlineto{\pgfqpoint{2.415369in}{0.999462in}}%
\pgfpathlineto{\pgfqpoint{2.409100in}{0.999462in}}%
\pgfpathlineto{\pgfqpoint{2.405965in}{0.995711in}}%
\pgfpathlineto{\pgfqpoint{2.399696in}{0.995711in}}%
\pgfpathlineto{\pgfqpoint{2.396561in}{0.991959in}}%
\pgfpathlineto{\pgfqpoint{2.390292in}{0.991959in}}%
\pgfpathlineto{\pgfqpoint{2.387157in}{0.988208in}}%
\pgfpathlineto{\pgfqpoint{2.380887in}{0.988208in}}%
\pgfpathlineto{\pgfqpoint{2.377753in}{0.984456in}}%
\pgfpathlineto{\pgfqpoint{2.374618in}{0.984456in}}%
\pgfpathlineto{\pgfqpoint{2.371483in}{0.980705in}}%
\pgfpathlineto{\pgfqpoint{2.365214in}{0.980705in}}%
\pgfpathlineto{\pgfqpoint{2.362079in}{0.976953in}}%
\pgfpathlineto{\pgfqpoint{2.355809in}{0.976953in}}%
\pgfpathlineto{\pgfqpoint{2.352675in}{0.973202in}}%
\pgfpathlineto{\pgfqpoint{2.346405in}{0.973202in}}%
\pgfpathlineto{\pgfqpoint{2.343270in}{0.969450in}}%
\pgfpathlineto{\pgfqpoint{2.340136in}{0.969450in}}%
\pgfpathlineto{\pgfqpoint{2.337001in}{0.965699in}}%
\pgfpathlineto{\pgfqpoint{2.330731in}{0.965699in}}%
\pgfpathlineto{\pgfqpoint{2.327597in}{0.961947in}}%
\pgfpathlineto{\pgfqpoint{2.321327in}{0.961947in}}%
\pgfpathlineto{\pgfqpoint{2.318192in}{0.958196in}}%
\pgfpathlineto{\pgfqpoint{2.311923in}{0.958196in}}%
\pgfpathlineto{\pgfqpoint{2.308788in}{0.954444in}}%
\pgfpathlineto{\pgfqpoint{2.305654in}{0.954444in}}%
\pgfpathlineto{\pgfqpoint{2.302519in}{0.950692in}}%
\pgfpathlineto{\pgfqpoint{2.296249in}{0.950692in}}%
\pgfpathlineto{\pgfqpoint{2.293115in}{0.946941in}}%
\pgfpathlineto{\pgfqpoint{2.286845in}{0.946941in}}%
\pgfpathlineto{\pgfqpoint{2.283710in}{0.943189in}}%
\pgfpathlineto{\pgfqpoint{2.277441in}{0.943189in}}%
\pgfpathlineto{\pgfqpoint{2.274306in}{0.939438in}}%
\pgfpathlineto{\pgfqpoint{2.271171in}{0.939438in}}%
\pgfpathlineto{\pgfqpoint{2.268037in}{0.935686in}}%
\pgfpathlineto{\pgfqpoint{2.261767in}{0.935686in}}%
\pgfpathlineto{\pgfqpoint{2.258632in}{0.931935in}}%
\pgfpathlineto{\pgfqpoint{2.252363in}{0.931935in}}%
\pgfpathlineto{\pgfqpoint{2.249228in}{0.928183in}}%
\pgfpathlineto{\pgfqpoint{2.242959in}{0.928183in}}%
\pgfpathlineto{\pgfqpoint{2.239824in}{0.924432in}}%
\pgfpathlineto{\pgfqpoint{2.233554in}{0.924432in}}%
\pgfpathlineto{\pgfqpoint{2.230420in}{0.920680in}}%
\pgfpathlineto{\pgfqpoint{2.227285in}{0.920680in}}%
\pgfpathlineto{\pgfqpoint{2.224150in}{0.916929in}}%
\pgfpathlineto{\pgfqpoint{2.217881in}{0.916929in}}%
\pgfpathlineto{\pgfqpoint{2.214746in}{0.913177in}}%
\pgfpathlineto{\pgfqpoint{2.208477in}{0.913177in}}%
\pgfpathlineto{\pgfqpoint{2.205342in}{0.909426in}}%
\pgfpathlineto{\pgfqpoint{2.199072in}{0.909426in}}%
\pgfpathlineto{\pgfqpoint{2.195938in}{0.905674in}}%
\pgfpathlineto{\pgfqpoint{2.192803in}{0.905674in}}%
\pgfpathlineto{\pgfqpoint{2.189668in}{0.901922in}}%
\pgfpathlineto{\pgfqpoint{2.183399in}{0.901922in}}%
\pgfpathlineto{\pgfqpoint{2.180264in}{0.898171in}}%
\pgfpathlineto{\pgfqpoint{2.173994in}{0.898171in}}%
\pgfpathlineto{\pgfqpoint{2.173994in}{0.898171in}}%
\pgfusepath{stroke}%
\end{pgfscope}%
\begin{pgfscope}%
\pgfpathrectangle{\pgfqpoint{0.888750in}{0.419100in}}{\pgfqpoint{2.504659in}{2.933700in}} %
\pgfusepath{clip}%
\pgfsetbuttcap%
\pgfsetroundjoin%
\pgfsetlinewidth{1.505625pt}%
\definecolor{currentstroke}{rgb}{0.710588,0.710588,0.710588}%
\pgfsetstrokecolor{currentstroke}%
\pgfsetdash{}{0pt}%
\pgfpathmoveto{\pgfqpoint{2.173994in}{0.897608in}}%
\pgfpathlineto{\pgfqpoint{2.167725in}{0.905111in}}%
\pgfpathlineto{\pgfqpoint{2.164590in}{0.905111in}}%
\pgfpathlineto{\pgfqpoint{2.152051in}{0.920117in}}%
\pgfpathlineto{\pgfqpoint{2.148916in}{0.920117in}}%
\pgfpathlineto{\pgfqpoint{2.139512in}{0.931372in}}%
\pgfpathlineto{\pgfqpoint{2.136377in}{0.931372in}}%
\pgfpathlineto{\pgfqpoint{2.126973in}{0.942627in}}%
\pgfpathlineto{\pgfqpoint{2.123838in}{0.942627in}}%
\pgfpathlineto{\pgfqpoint{2.114434in}{0.953881in}}%
\pgfpathlineto{\pgfqpoint{2.111299in}{0.953881in}}%
\pgfpathlineto{\pgfqpoint{2.101895in}{0.965136in}}%
\pgfpathlineto{\pgfqpoint{2.098761in}{0.965136in}}%
\pgfpathlineto{\pgfqpoint{2.086222in}{0.980142in}}%
\pgfpathlineto{\pgfqpoint{2.083087in}{0.980142in}}%
\pgfpathlineto{\pgfqpoint{2.073683in}{0.991397in}}%
\pgfpathlineto{\pgfqpoint{2.070548in}{0.991397in}}%
\pgfpathlineto{\pgfqpoint{2.061144in}{1.002651in}}%
\pgfpathlineto{\pgfqpoint{2.058009in}{1.002651in}}%
\pgfpathlineto{\pgfqpoint{2.048605in}{1.013906in}}%
\pgfpathlineto{\pgfqpoint{2.045470in}{1.013906in}}%
\pgfpathlineto{\pgfqpoint{2.036066in}{1.025160in}}%
\pgfpathlineto{\pgfqpoint{2.032931in}{1.025160in}}%
\pgfpathlineto{\pgfqpoint{2.023527in}{1.036415in}}%
\pgfpathlineto{\pgfqpoint{2.020392in}{1.036415in}}%
\pgfpathlineto{\pgfqpoint{2.007853in}{1.051421in}}%
\pgfpathlineto{\pgfqpoint{2.004718in}{1.051421in}}%
\pgfpathlineto{\pgfqpoint{1.995314in}{1.062676in}}%
\pgfpathlineto{\pgfqpoint{1.992179in}{1.062676in}}%
\pgfpathlineto{\pgfqpoint{1.982775in}{1.073930in}}%
\pgfpathlineto{\pgfqpoint{1.979640in}{1.073930in}}%
\pgfpathlineto{\pgfqpoint{1.970236in}{1.085185in}}%
\pgfpathlineto{\pgfqpoint{1.967101in}{1.085185in}}%
\pgfpathlineto{\pgfqpoint{1.957697in}{1.096440in}}%
\pgfpathlineto{\pgfqpoint{1.954562in}{1.096440in}}%
\pgfpathlineto{\pgfqpoint{1.945158in}{1.107694in}}%
\pgfpathlineto{\pgfqpoint{1.942023in}{1.107694in}}%
\pgfpathlineto{\pgfqpoint{1.929484in}{1.122700in}}%
\pgfpathlineto{\pgfqpoint{1.926350in}{1.122700in}}%
\pgfpathlineto{\pgfqpoint{1.916945in}{1.133955in}}%
\pgfpathlineto{\pgfqpoint{1.913811in}{1.133955in}}%
\pgfpathlineto{\pgfqpoint{1.904407in}{1.145210in}}%
\pgfpathlineto{\pgfqpoint{1.901272in}{1.145210in}}%
\pgfpathlineto{\pgfqpoint{1.891868in}{1.156464in}}%
\pgfpathlineto{\pgfqpoint{1.888733in}{1.156464in}}%
\pgfpathlineto{\pgfqpoint{1.879329in}{1.167719in}}%
\pgfpathlineto{\pgfqpoint{1.876194in}{1.167719in}}%
\pgfpathlineto{\pgfqpoint{1.866790in}{1.178973in}}%
\pgfpathlineto{\pgfqpoint{1.863655in}{1.178973in}}%
\pgfpathlineto{\pgfqpoint{1.851116in}{1.193979in}}%
\pgfpathlineto{\pgfqpoint{1.847981in}{1.193979in}}%
\pgfpathlineto{\pgfqpoint{1.838577in}{1.205234in}}%
\pgfpathlineto{\pgfqpoint{1.835442in}{1.205234in}}%
\pgfpathlineto{\pgfqpoint{1.826038in}{1.216489in}}%
\pgfpathlineto{\pgfqpoint{1.822903in}{1.216489in}}%
\pgfpathlineto{\pgfqpoint{1.813499in}{1.227743in}}%
\pgfpathlineto{\pgfqpoint{1.810364in}{1.227743in}}%
\pgfpathlineto{\pgfqpoint{1.800960in}{1.238998in}}%
\pgfpathlineto{\pgfqpoint{1.797825in}{1.238998in}}%
\pgfpathlineto{\pgfqpoint{1.785286in}{1.254004in}}%
\pgfpathlineto{\pgfqpoint{1.782152in}{1.254004in}}%
\pgfpathlineto{\pgfqpoint{1.772747in}{1.265259in}}%
\pgfpathlineto{\pgfqpoint{1.769613in}{1.265259in}}%
\pgfpathlineto{\pgfqpoint{1.760208in}{1.276513in}}%
\pgfpathlineto{\pgfqpoint{1.757074in}{1.276513in}}%
\pgfpathlineto{\pgfqpoint{1.747669in}{1.287768in}}%
\pgfpathlineto{\pgfqpoint{1.744535in}{1.287768in}}%
\pgfpathlineto{\pgfqpoint{1.735130in}{1.299022in}}%
\pgfpathlineto{\pgfqpoint{1.731996in}{1.299022in}}%
\pgfpathlineto{\pgfqpoint{1.722591in}{1.310277in}}%
\pgfpathlineto{\pgfqpoint{1.719457in}{1.310277in}}%
\pgfpathlineto{\pgfqpoint{1.706918in}{1.325283in}}%
\pgfpathlineto{\pgfqpoint{1.703783in}{1.325283in}}%
\pgfpathlineto{\pgfqpoint{1.694379in}{1.336538in}}%
\pgfpathlineto{\pgfqpoint{1.691244in}{1.336538in}}%
\pgfpathlineto{\pgfqpoint{1.681840in}{1.347792in}}%
\pgfpathlineto{\pgfqpoint{1.678705in}{1.347792in}}%
\pgfpathlineto{\pgfqpoint{1.669301in}{1.359047in}}%
\pgfpathlineto{\pgfqpoint{1.666166in}{1.359047in}}%
\pgfpathlineto{\pgfqpoint{1.656762in}{1.370302in}}%
\pgfpathlineto{\pgfqpoint{1.653627in}{1.370302in}}%
\pgfpathlineto{\pgfqpoint{1.644223in}{1.381556in}}%
\pgfpathlineto{\pgfqpoint{1.641088in}{1.381556in}}%
\pgfpathlineto{\pgfqpoint{1.628549in}{1.396562in}}%
\pgfpathlineto{\pgfqpoint{1.625414in}{1.396562in}}%
\pgfpathlineto{\pgfqpoint{1.616010in}{1.407817in}}%
\pgfpathlineto{\pgfqpoint{1.612875in}{1.407817in}}%
\pgfpathlineto{\pgfqpoint{1.603471in}{1.419072in}}%
\pgfpathlineto{\pgfqpoint{1.600337in}{1.419072in}}%
\pgfpathlineto{\pgfqpoint{1.590932in}{1.430326in}}%
\pgfpathlineto{\pgfqpoint{1.587798in}{1.430326in}}%
\pgfpathlineto{\pgfqpoint{1.578393in}{1.441581in}}%
\pgfpathlineto{\pgfqpoint{1.575259in}{1.441581in}}%
\pgfpathlineto{\pgfqpoint{1.565854in}{1.452835in}}%
\pgfpathlineto{\pgfqpoint{1.562720in}{1.452835in}}%
\pgfpathlineto{\pgfqpoint{1.550181in}{1.467841in}}%
\pgfpathlineto{\pgfqpoint{1.547046in}{1.467841in}}%
\pgfpathlineto{\pgfqpoint{1.537642in}{1.479096in}}%
\pgfpathlineto{\pgfqpoint{1.534507in}{1.479096in}}%
\pgfpathlineto{\pgfqpoint{1.525103in}{1.490351in}}%
\pgfpathlineto{\pgfqpoint{1.521968in}{1.490351in}}%
\pgfpathlineto{\pgfqpoint{1.512564in}{1.501605in}}%
\pgfpathlineto{\pgfqpoint{1.509429in}{1.501605in}}%
\pgfpathlineto{\pgfqpoint{1.500025in}{1.512860in}}%
\pgfpathlineto{\pgfqpoint{1.496890in}{1.512860in}}%
\pgfpathlineto{\pgfqpoint{1.484351in}{1.527866in}}%
\pgfpathlineto{\pgfqpoint{1.481216in}{1.527866in}}%
\pgfpathlineto{\pgfqpoint{1.471812in}{1.539121in}}%
\pgfpathlineto{\pgfqpoint{1.468677in}{1.539121in}}%
\pgfpathlineto{\pgfqpoint{1.459273in}{1.550375in}}%
\pgfpathlineto{\pgfqpoint{1.456138in}{1.550375in}}%
\pgfpathlineto{\pgfqpoint{1.446734in}{1.561630in}}%
\pgfpathlineto{\pgfqpoint{1.443599in}{1.561630in}}%
\pgfpathlineto{\pgfqpoint{1.434195in}{1.572884in}}%
\pgfpathlineto{\pgfqpoint{1.431060in}{1.572884in}}%
\pgfpathlineto{\pgfqpoint{1.421656in}{1.584139in}}%
\pgfpathlineto{\pgfqpoint{1.418521in}{1.584139in}}%
\pgfpathlineto{\pgfqpoint{1.405982in}{1.599145in}}%
\pgfpathlineto{\pgfqpoint{1.402848in}{1.599145in}}%
\pgfpathlineto{\pgfqpoint{1.393444in}{1.610400in}}%
\pgfpathlineto{\pgfqpoint{1.390309in}{1.610400in}}%
\pgfpathlineto{\pgfqpoint{1.380905in}{1.621654in}}%
\pgfpathlineto{\pgfqpoint{1.377770in}{1.621654in}}%
\pgfpathlineto{\pgfqpoint{1.368366in}{1.632909in}}%
\pgfpathlineto{\pgfqpoint{1.365231in}{1.632909in}}%
\pgfpathlineto{\pgfqpoint{1.355827in}{1.644164in}}%
\pgfpathlineto{\pgfqpoint{1.352692in}{1.644164in}}%
\pgfpathlineto{\pgfqpoint{1.343288in}{1.655418in}}%
\pgfpathlineto{\pgfqpoint{1.340153in}{1.655418in}}%
\pgfpathlineto{\pgfqpoint{1.327614in}{1.670424in}}%
\pgfpathlineto{\pgfqpoint{1.324479in}{1.670424in}}%
\pgfpathlineto{\pgfqpoint{1.315075in}{1.681679in}}%
\pgfpathlineto{\pgfqpoint{1.311940in}{1.681679in}}%
\pgfpathlineto{\pgfqpoint{1.302536in}{1.692934in}}%
\pgfpathlineto{\pgfqpoint{1.299401in}{1.692934in}}%
\pgfpathlineto{\pgfqpoint{1.291721in}{1.702125in}}%
\pgfpathlineto{\pgfqpoint{1.291721in}{1.705876in}}%
\pgfpathlineto{\pgfqpoint{1.294856in}{1.709628in}}%
\pgfpathlineto{\pgfqpoint{1.294856in}{1.720882in}}%
\pgfpathlineto{\pgfqpoint{1.297991in}{1.724634in}}%
\pgfpathlineto{\pgfqpoint{1.297991in}{1.732137in}}%
\pgfpathlineto{\pgfqpoint{1.301125in}{1.735889in}}%
\pgfpathlineto{\pgfqpoint{1.301125in}{1.747143in}}%
\pgfpathlineto{\pgfqpoint{1.304260in}{1.750895in}}%
\pgfpathlineto{\pgfqpoint{1.304260in}{1.758398in}}%
\pgfpathlineto{\pgfqpoint{1.307395in}{1.762149in}}%
\pgfpathlineto{\pgfqpoint{1.307395in}{1.773404in}}%
\pgfpathlineto{\pgfqpoint{1.310530in}{1.777155in}}%
\pgfpathlineto{\pgfqpoint{1.310530in}{1.784659in}}%
\pgfpathlineto{\pgfqpoint{1.313664in}{1.788410in}}%
\pgfpathlineto{\pgfqpoint{1.313664in}{1.799665in}}%
\pgfpathlineto{\pgfqpoint{1.316799in}{1.803416in}}%
\pgfpathlineto{\pgfqpoint{1.316799in}{1.810919in}}%
\pgfpathlineto{\pgfqpoint{1.319934in}{1.814671in}}%
\pgfpathlineto{\pgfqpoint{1.319934in}{1.822174in}}%
\pgfpathlineto{\pgfqpoint{1.323069in}{1.825925in}}%
\pgfpathlineto{\pgfqpoint{1.323069in}{1.837180in}}%
\pgfpathlineto{\pgfqpoint{1.326203in}{1.840932in}}%
\pgfpathlineto{\pgfqpoint{1.326203in}{1.848435in}}%
\pgfpathlineto{\pgfqpoint{1.329338in}{1.852186in}}%
\pgfpathlineto{\pgfqpoint{1.329338in}{1.863441in}}%
\pgfpathlineto{\pgfqpoint{1.332473in}{1.867192in}}%
\pgfpathlineto{\pgfqpoint{1.332473in}{1.874695in}}%
\pgfpathlineto{\pgfqpoint{1.335608in}{1.878447in}}%
\pgfpathlineto{\pgfqpoint{1.335608in}{1.889702in}}%
\pgfpathlineto{\pgfqpoint{1.338742in}{1.893453in}}%
\pgfpathlineto{\pgfqpoint{1.338742in}{1.900956in}}%
\pgfpathlineto{\pgfqpoint{1.341877in}{1.904708in}}%
\pgfpathlineto{\pgfqpoint{1.341877in}{1.915962in}}%
\pgfpathlineto{\pgfqpoint{1.345012in}{1.919714in}}%
\pgfpathlineto{\pgfqpoint{1.345012in}{1.927217in}}%
\pgfpathlineto{\pgfqpoint{1.348146in}{1.930968in}}%
\pgfpathlineto{\pgfqpoint{1.348146in}{1.942223in}}%
\pgfpathlineto{\pgfqpoint{1.351281in}{1.945975in}}%
\pgfpathlineto{\pgfqpoint{1.351281in}{1.953478in}}%
\pgfpathlineto{\pgfqpoint{1.354416in}{1.957229in}}%
\pgfpathlineto{\pgfqpoint{1.354416in}{1.968484in}}%
\pgfpathlineto{\pgfqpoint{1.357551in}{1.972235in}}%
\pgfpathlineto{\pgfqpoint{1.357551in}{1.979738in}}%
\pgfpathlineto{\pgfqpoint{1.360685in}{1.983490in}}%
\pgfpathlineto{\pgfqpoint{1.360685in}{1.994745in}}%
\pgfpathlineto{\pgfqpoint{1.363820in}{1.998496in}}%
\pgfpathlineto{\pgfqpoint{1.363820in}{2.005999in}}%
\pgfpathlineto{\pgfqpoint{1.366955in}{2.009751in}}%
\pgfpathlineto{\pgfqpoint{1.366955in}{2.021005in}}%
\pgfpathlineto{\pgfqpoint{1.370090in}{2.024757in}}%
\pgfpathlineto{\pgfqpoint{1.370090in}{2.032260in}}%
\pgfpathlineto{\pgfqpoint{1.373224in}{2.036011in}}%
\pgfpathlineto{\pgfqpoint{1.373224in}{2.043514in}}%
\pgfpathlineto{\pgfqpoint{1.376359in}{2.047266in}}%
\pgfpathlineto{\pgfqpoint{1.376359in}{2.058521in}}%
\pgfpathlineto{\pgfqpoint{1.379494in}{2.062272in}}%
\pgfpathlineto{\pgfqpoint{1.379494in}{2.069775in}}%
\pgfpathlineto{\pgfqpoint{1.382629in}{2.073527in}}%
\pgfpathlineto{\pgfqpoint{1.382629in}{2.084781in}}%
\pgfpathlineto{\pgfqpoint{1.385763in}{2.088533in}}%
\pgfpathlineto{\pgfqpoint{1.385763in}{2.096036in}}%
\pgfpathlineto{\pgfqpoint{1.388898in}{2.099787in}}%
\pgfpathlineto{\pgfqpoint{1.388898in}{2.111042in}}%
\pgfpathlineto{\pgfqpoint{1.392033in}{2.114794in}}%
\pgfpathlineto{\pgfqpoint{1.392033in}{2.122297in}}%
\pgfpathlineto{\pgfqpoint{1.395168in}{2.126048in}}%
\pgfpathlineto{\pgfqpoint{1.395168in}{2.137303in}}%
\pgfpathlineto{\pgfqpoint{1.398302in}{2.141054in}}%
\pgfpathlineto{\pgfqpoint{1.398302in}{2.148557in}}%
\pgfpathlineto{\pgfqpoint{1.401437in}{2.152309in}}%
\pgfpathlineto{\pgfqpoint{1.401437in}{2.163564in}}%
\pgfpathlineto{\pgfqpoint{1.404572in}{2.167315in}}%
\pgfpathlineto{\pgfqpoint{1.404572in}{2.174818in}}%
\pgfpathlineto{\pgfqpoint{1.407707in}{2.178570in}}%
\pgfpathlineto{\pgfqpoint{1.407707in}{2.189824in}}%
\pgfpathlineto{\pgfqpoint{1.410841in}{2.193576in}}%
\pgfpathlineto{\pgfqpoint{1.410841in}{2.201079in}}%
\pgfpathlineto{\pgfqpoint{1.413976in}{2.204830in}}%
\pgfpathlineto{\pgfqpoint{1.413976in}{2.216085in}}%
\pgfpathlineto{\pgfqpoint{1.417111in}{2.219837in}}%
\pgfpathlineto{\pgfqpoint{1.417111in}{2.227340in}}%
\pgfpathlineto{\pgfqpoint{1.420246in}{2.231091in}}%
\pgfpathlineto{\pgfqpoint{1.420246in}{2.242346in}}%
\pgfpathlineto{\pgfqpoint{1.423380in}{2.246097in}}%
\pgfpathlineto{\pgfqpoint{1.423380in}{2.253600in}}%
\pgfpathlineto{\pgfqpoint{1.426515in}{2.257352in}}%
\pgfpathlineto{\pgfqpoint{1.426515in}{2.264855in}}%
\pgfpathlineto{\pgfqpoint{1.429650in}{2.268607in}}%
\pgfpathlineto{\pgfqpoint{1.429650in}{2.279861in}}%
\pgfpathlineto{\pgfqpoint{1.432785in}{2.283613in}}%
\pgfpathlineto{\pgfqpoint{1.432785in}{2.291116in}}%
\pgfpathlineto{\pgfqpoint{1.435919in}{2.294867in}}%
\pgfpathlineto{\pgfqpoint{1.435919in}{2.306122in}}%
\pgfpathlineto{\pgfqpoint{1.439054in}{2.309873in}}%
\pgfpathlineto{\pgfqpoint{1.439054in}{2.317376in}}%
\pgfpathlineto{\pgfqpoint{1.442189in}{2.321128in}}%
\pgfpathlineto{\pgfqpoint{1.442189in}{2.332383in}}%
\pgfpathlineto{\pgfqpoint{1.445323in}{2.336134in}}%
\pgfpathlineto{\pgfqpoint{1.445323in}{2.343637in}}%
\pgfpathlineto{\pgfqpoint{1.448458in}{2.347389in}}%
\pgfpathlineto{\pgfqpoint{1.448458in}{2.358643in}}%
\pgfpathlineto{\pgfqpoint{1.451593in}{2.362395in}}%
\pgfpathlineto{\pgfqpoint{1.451593in}{2.369898in}}%
\pgfpathlineto{\pgfqpoint{1.454728in}{2.373649in}}%
\pgfpathlineto{\pgfqpoint{1.454728in}{2.384904in}}%
\pgfpathlineto{\pgfqpoint{1.457862in}{2.388656in}}%
\pgfpathlineto{\pgfqpoint{1.457862in}{2.396159in}}%
\pgfpathlineto{\pgfqpoint{1.460997in}{2.399910in}}%
\pgfpathlineto{\pgfqpoint{1.460997in}{2.403662in}}%
\pgfpathlineto{\pgfqpoint{1.462408in}{2.405350in}}%
\pgfpathlineto{\pgfqpoint{1.465543in}{2.405350in}}%
\pgfpathlineto{\pgfqpoint{1.468677in}{2.409101in}}%
\pgfpathlineto{\pgfqpoint{1.478082in}{2.409101in}}%
\pgfpathlineto{\pgfqpoint{1.481216in}{2.412853in}}%
\pgfpathlineto{\pgfqpoint{1.490621in}{2.412853in}}%
\pgfpathlineto{\pgfqpoint{1.493755in}{2.416605in}}%
\pgfpathlineto{\pgfqpoint{1.503159in}{2.416605in}}%
\pgfpathlineto{\pgfqpoint{1.506294in}{2.420356in}}%
\pgfpathlineto{\pgfqpoint{1.515698in}{2.420356in}}%
\pgfpathlineto{\pgfqpoint{1.518833in}{2.424108in}}%
\pgfpathlineto{\pgfqpoint{1.531372in}{2.424108in}}%
\pgfpathlineto{\pgfqpoint{1.534507in}{2.427859in}}%
\pgfpathlineto{\pgfqpoint{1.543911in}{2.427859in}}%
\pgfpathlineto{\pgfqpoint{1.547046in}{2.431611in}}%
\pgfpathlineto{\pgfqpoint{1.556450in}{2.431611in}}%
\pgfpathlineto{\pgfqpoint{1.559585in}{2.435362in}}%
\pgfpathlineto{\pgfqpoint{1.568989in}{2.435362in}}%
\pgfpathlineto{\pgfqpoint{1.572124in}{2.439114in}}%
\pgfpathlineto{\pgfqpoint{1.584663in}{2.439114in}}%
\pgfpathlineto{\pgfqpoint{1.587798in}{2.442865in}}%
\pgfpathlineto{\pgfqpoint{1.597202in}{2.442865in}}%
\pgfpathlineto{\pgfqpoint{1.600337in}{2.446617in}}%
\pgfpathlineto{\pgfqpoint{1.609741in}{2.446617in}}%
\pgfpathlineto{\pgfqpoint{1.612875in}{2.450368in}}%
\pgfpathlineto{\pgfqpoint{1.622280in}{2.450368in}}%
\pgfpathlineto{\pgfqpoint{1.625414in}{2.454120in}}%
\pgfpathlineto{\pgfqpoint{1.637953in}{2.454120in}}%
\pgfpathlineto{\pgfqpoint{1.641088in}{2.457871in}}%
\pgfpathlineto{\pgfqpoint{1.650492in}{2.457871in}}%
\pgfpathlineto{\pgfqpoint{1.653627in}{2.461623in}}%
\pgfpathlineto{\pgfqpoint{1.663031in}{2.461623in}}%
\pgfpathlineto{\pgfqpoint{1.666166in}{2.465375in}}%
\pgfpathlineto{\pgfqpoint{1.675570in}{2.465375in}}%
\pgfpathlineto{\pgfqpoint{1.678705in}{2.469126in}}%
\pgfpathlineto{\pgfqpoint{1.688109in}{2.469126in}}%
\pgfpathlineto{\pgfqpoint{1.691244in}{2.472878in}}%
\pgfpathlineto{\pgfqpoint{1.703783in}{2.472878in}}%
\pgfpathlineto{\pgfqpoint{1.706918in}{2.476629in}}%
\pgfpathlineto{\pgfqpoint{1.716322in}{2.476629in}}%
\pgfpathlineto{\pgfqpoint{1.719457in}{2.480381in}}%
\pgfpathlineto{\pgfqpoint{1.728861in}{2.480381in}}%
\pgfpathlineto{\pgfqpoint{1.731996in}{2.484132in}}%
\pgfpathlineto{\pgfqpoint{1.741400in}{2.484132in}}%
\pgfpathlineto{\pgfqpoint{1.744535in}{2.487884in}}%
\pgfpathlineto{\pgfqpoint{1.757074in}{2.487884in}}%
\pgfpathlineto{\pgfqpoint{1.760208in}{2.491635in}}%
\pgfpathlineto{\pgfqpoint{1.769613in}{2.491635in}}%
\pgfpathlineto{\pgfqpoint{1.772747in}{2.495387in}}%
\pgfpathlineto{\pgfqpoint{1.782152in}{2.495387in}}%
\pgfpathlineto{\pgfqpoint{1.785286in}{2.499138in}}%
\pgfpathlineto{\pgfqpoint{1.794691in}{2.499138in}}%
\pgfpathlineto{\pgfqpoint{1.797825in}{2.502890in}}%
\pgfpathlineto{\pgfqpoint{1.810364in}{2.502890in}}%
\pgfpathlineto{\pgfqpoint{1.813499in}{2.506641in}}%
\pgfpathlineto{\pgfqpoint{1.822903in}{2.506641in}}%
\pgfpathlineto{\pgfqpoint{1.826038in}{2.510393in}}%
\pgfpathlineto{\pgfqpoint{1.835442in}{2.510393in}}%
\pgfpathlineto{\pgfqpoint{1.838577in}{2.514144in}}%
\pgfpathlineto{\pgfqpoint{1.847981in}{2.514144in}}%
\pgfpathlineto{\pgfqpoint{1.851116in}{2.517896in}}%
\pgfpathlineto{\pgfqpoint{1.860520in}{2.517896in}}%
\pgfpathlineto{\pgfqpoint{1.863655in}{2.521648in}}%
\pgfpathlineto{\pgfqpoint{1.876194in}{2.521648in}}%
\pgfpathlineto{\pgfqpoint{1.879329in}{2.525399in}}%
\pgfpathlineto{\pgfqpoint{1.888733in}{2.525399in}}%
\pgfpathlineto{\pgfqpoint{1.891868in}{2.529151in}}%
\pgfpathlineto{\pgfqpoint{1.901272in}{2.529151in}}%
\pgfpathlineto{\pgfqpoint{1.904407in}{2.532902in}}%
\pgfpathlineto{\pgfqpoint{1.913811in}{2.532902in}}%
\pgfpathlineto{\pgfqpoint{1.916945in}{2.536654in}}%
\pgfpathlineto{\pgfqpoint{1.929484in}{2.536654in}}%
\pgfpathlineto{\pgfqpoint{1.932619in}{2.540405in}}%
\pgfpathlineto{\pgfqpoint{1.942023in}{2.540405in}}%
\pgfpathlineto{\pgfqpoint{1.945158in}{2.544157in}}%
\pgfpathlineto{\pgfqpoint{1.954562in}{2.544157in}}%
\pgfpathlineto{\pgfqpoint{1.957697in}{2.547908in}}%
\pgfpathlineto{\pgfqpoint{1.967101in}{2.547908in}}%
\pgfpathlineto{\pgfqpoint{1.970236in}{2.551660in}}%
\pgfpathlineto{\pgfqpoint{1.979640in}{2.551660in}}%
\pgfpathlineto{\pgfqpoint{1.982775in}{2.555411in}}%
\pgfpathlineto{\pgfqpoint{1.995314in}{2.555411in}}%
\pgfpathlineto{\pgfqpoint{1.998449in}{2.559163in}}%
\pgfpathlineto{\pgfqpoint{2.007853in}{2.559163in}}%
\pgfpathlineto{\pgfqpoint{2.010988in}{2.562914in}}%
\pgfpathlineto{\pgfqpoint{2.020392in}{2.562914in}}%
\pgfpathlineto{\pgfqpoint{2.023527in}{2.566666in}}%
\pgfpathlineto{\pgfqpoint{2.032931in}{2.566666in}}%
\pgfpathlineto{\pgfqpoint{2.036066in}{2.570417in}}%
\pgfpathlineto{\pgfqpoint{2.048605in}{2.570417in}}%
\pgfpathlineto{\pgfqpoint{2.051739in}{2.574169in}}%
\pgfpathlineto{\pgfqpoint{2.061144in}{2.574169in}}%
\pgfpathlineto{\pgfqpoint{2.064278in}{2.577921in}}%
\pgfpathlineto{\pgfqpoint{2.073683in}{2.577921in}}%
\pgfpathlineto{\pgfqpoint{2.076817in}{2.581672in}}%
\pgfpathlineto{\pgfqpoint{2.086222in}{2.581672in}}%
\pgfpathlineto{\pgfqpoint{2.089356in}{2.585424in}}%
\pgfpathlineto{\pgfqpoint{2.101895in}{2.585424in}}%
\pgfpathlineto{\pgfqpoint{2.105030in}{2.589175in}}%
\pgfpathlineto{\pgfqpoint{2.114434in}{2.589175in}}%
\pgfpathlineto{\pgfqpoint{2.117569in}{2.592927in}}%
\pgfpathlineto{\pgfqpoint{2.126973in}{2.592927in}}%
\pgfpathlineto{\pgfqpoint{2.130108in}{2.596678in}}%
\pgfpathlineto{\pgfqpoint{2.139512in}{2.596678in}}%
\pgfpathlineto{\pgfqpoint{2.142647in}{2.600430in}}%
\pgfpathlineto{\pgfqpoint{2.152051in}{2.600430in}}%
\pgfpathlineto{\pgfqpoint{2.155186in}{2.604181in}}%
\pgfpathlineto{\pgfqpoint{2.167725in}{2.604181in}}%
\pgfpathlineto{\pgfqpoint{2.170860in}{2.607933in}}%
\pgfpathlineto{\pgfqpoint{2.180264in}{2.607933in}}%
\pgfpathlineto{\pgfqpoint{2.183399in}{2.611684in}}%
\pgfpathlineto{\pgfqpoint{2.192803in}{2.611684in}}%
\pgfpathlineto{\pgfqpoint{2.195938in}{2.615436in}}%
\pgfpathlineto{\pgfqpoint{2.205342in}{2.615436in}}%
\pgfpathlineto{\pgfqpoint{2.208477in}{2.619187in}}%
\pgfpathlineto{\pgfqpoint{2.221015in}{2.619187in}}%
\pgfpathlineto{\pgfqpoint{2.224150in}{2.622939in}}%
\pgfpathlineto{\pgfqpoint{2.233554in}{2.622939in}}%
\pgfpathlineto{\pgfqpoint{2.236689in}{2.626690in}}%
\pgfpathlineto{\pgfqpoint{2.246093in}{2.626690in}}%
\pgfpathlineto{\pgfqpoint{2.249228in}{2.630442in}}%
\pgfpathlineto{\pgfqpoint{2.258632in}{2.630442in}}%
\pgfpathlineto{\pgfqpoint{2.261767in}{2.634194in}}%
\pgfpathlineto{\pgfqpoint{2.274306in}{2.634194in}}%
\pgfpathlineto{\pgfqpoint{2.277441in}{2.637945in}}%
\pgfpathlineto{\pgfqpoint{2.286845in}{2.637945in}}%
\pgfpathlineto{\pgfqpoint{2.289980in}{2.641697in}}%
\pgfpathlineto{\pgfqpoint{2.299384in}{2.641697in}}%
\pgfpathlineto{\pgfqpoint{2.302519in}{2.645448in}}%
\pgfpathlineto{\pgfqpoint{2.311923in}{2.645448in}}%
\pgfpathlineto{\pgfqpoint{2.315058in}{2.649200in}}%
\pgfpathlineto{\pgfqpoint{2.324462in}{2.649200in}}%
\pgfpathlineto{\pgfqpoint{2.327597in}{2.652951in}}%
\pgfpathlineto{\pgfqpoint{2.340136in}{2.652951in}}%
\pgfpathlineto{\pgfqpoint{2.343270in}{2.656703in}}%
\pgfpathlineto{\pgfqpoint{2.352675in}{2.656703in}}%
\pgfpathlineto{\pgfqpoint{2.355809in}{2.660454in}}%
\pgfpathlineto{\pgfqpoint{2.365214in}{2.660454in}}%
\pgfpathlineto{\pgfqpoint{2.368348in}{2.664206in}}%
\pgfpathlineto{\pgfqpoint{2.377753in}{2.664206in}}%
\pgfpathlineto{\pgfqpoint{2.380887in}{2.667957in}}%
\pgfpathlineto{\pgfqpoint{2.393426in}{2.667957in}}%
\pgfpathlineto{\pgfqpoint{2.396561in}{2.671709in}}%
\pgfpathlineto{\pgfqpoint{2.405965in}{2.671709in}}%
\pgfpathlineto{\pgfqpoint{2.409100in}{2.675460in}}%
\pgfpathlineto{\pgfqpoint{2.418504in}{2.675460in}}%
\pgfpathlineto{\pgfqpoint{2.421639in}{2.679212in}}%
\pgfpathlineto{\pgfqpoint{2.431043in}{2.679212in}}%
\pgfpathlineto{\pgfqpoint{2.434178in}{2.682964in}}%
\pgfpathlineto{\pgfqpoint{2.446717in}{2.682964in}}%
\pgfpathlineto{\pgfqpoint{2.449852in}{2.686715in}}%
\pgfpathlineto{\pgfqpoint{2.459256in}{2.686715in}}%
\pgfpathlineto{\pgfqpoint{2.462391in}{2.690467in}}%
\pgfpathlineto{\pgfqpoint{2.471795in}{2.690467in}}%
\pgfpathlineto{\pgfqpoint{2.474930in}{2.694218in}}%
\pgfpathlineto{\pgfqpoint{2.484334in}{2.694218in}}%
\pgfpathlineto{\pgfqpoint{2.487469in}{2.697970in}}%
\pgfpathlineto{\pgfqpoint{2.496873in}{2.697970in}}%
\pgfpathlineto{\pgfqpoint{2.500008in}{2.701721in}}%
\pgfpathlineto{\pgfqpoint{2.512547in}{2.701721in}}%
\pgfpathlineto{\pgfqpoint{2.515681in}{2.705473in}}%
\pgfpathlineto{\pgfqpoint{2.525085in}{2.705473in}}%
\pgfpathlineto{\pgfqpoint{2.528220in}{2.709224in}}%
\pgfpathlineto{\pgfqpoint{2.537624in}{2.709224in}}%
\pgfpathlineto{\pgfqpoint{2.540759in}{2.712976in}}%
\pgfpathlineto{\pgfqpoint{2.550163in}{2.712976in}}%
\pgfpathlineto{\pgfqpoint{2.553298in}{2.716727in}}%
\pgfpathlineto{\pgfqpoint{2.565837in}{2.716727in}}%
\pgfpathlineto{\pgfqpoint{2.568972in}{2.720479in}}%
\pgfpathlineto{\pgfqpoint{2.578376in}{2.720479in}}%
\pgfpathlineto{\pgfqpoint{2.581511in}{2.724230in}}%
\pgfpathlineto{\pgfqpoint{2.590915in}{2.724230in}}%
\pgfpathlineto{\pgfqpoint{2.594050in}{2.727982in}}%
\pgfpathlineto{\pgfqpoint{2.603454in}{2.727982in}}%
\pgfpathlineto{\pgfqpoint{2.606589in}{2.731733in}}%
\pgfpathlineto{\pgfqpoint{2.615993in}{2.731733in}}%
\pgfpathlineto{\pgfqpoint{2.619128in}{2.735485in}}%
\pgfpathlineto{\pgfqpoint{2.631667in}{2.735485in}}%
\pgfpathlineto{\pgfqpoint{2.634801in}{2.739237in}}%
\pgfpathlineto{\pgfqpoint{2.644206in}{2.739237in}}%
\pgfpathlineto{\pgfqpoint{2.647340in}{2.742988in}}%
\pgfpathlineto{\pgfqpoint{2.656745in}{2.742988in}}%
\pgfpathlineto{\pgfqpoint{2.659879in}{2.746740in}}%
\pgfpathlineto{\pgfqpoint{2.672418in}{2.746740in}}%
\pgfpathlineto{\pgfqpoint{2.711446in}{2.700033in}}%
\pgfpathlineto{\pgfqpoint{2.711446in}{2.696281in}}%
\pgfpathlineto{\pgfqpoint{2.761602in}{2.636257in}}%
\pgfpathlineto{\pgfqpoint{2.761602in}{2.632505in}}%
\pgfpathlineto{\pgfqpoint{2.808623in}{2.576232in}}%
\pgfpathlineto{\pgfqpoint{2.808623in}{2.572481in}}%
\pgfpathlineto{\pgfqpoint{2.858779in}{2.512456in}}%
\pgfpathlineto{\pgfqpoint{2.858779in}{2.508705in}}%
\pgfpathlineto{\pgfqpoint{2.905800in}{2.452432in}}%
\pgfpathlineto{\pgfqpoint{2.905800in}{2.448680in}}%
\pgfpathlineto{\pgfqpoint{2.955956in}{2.388656in}}%
\pgfpathlineto{\pgfqpoint{2.955956in}{2.384904in}}%
\pgfpathlineto{\pgfqpoint{3.002977in}{2.328631in}}%
\pgfpathlineto{\pgfqpoint{3.002977in}{2.324880in}}%
\pgfpathlineto{\pgfqpoint{3.053133in}{2.264855in}}%
\pgfpathlineto{\pgfqpoint{3.053133in}{2.261103in}}%
\pgfpathlineto{\pgfqpoint{3.093884in}{2.212334in}}%
\pgfpathlineto{\pgfqpoint{3.093884in}{2.208582in}}%
\pgfpathlineto{\pgfqpoint{3.090750in}{2.204830in}}%
\pgfpathlineto{\pgfqpoint{3.090750in}{2.197327in}}%
\pgfpathlineto{\pgfqpoint{3.087615in}{2.193576in}}%
\pgfpathlineto{\pgfqpoint{3.087615in}{2.186073in}}%
\pgfpathlineto{\pgfqpoint{3.084480in}{2.182321in}}%
\pgfpathlineto{\pgfqpoint{3.084480in}{2.174818in}}%
\pgfpathlineto{\pgfqpoint{3.081345in}{2.171067in}}%
\pgfpathlineto{\pgfqpoint{3.081345in}{2.163564in}}%
\pgfpathlineto{\pgfqpoint{3.078211in}{2.159812in}}%
\pgfpathlineto{\pgfqpoint{3.078211in}{2.148557in}}%
\pgfpathlineto{\pgfqpoint{3.075076in}{2.144806in}}%
\pgfpathlineto{\pgfqpoint{3.075076in}{2.137303in}}%
\pgfpathlineto{\pgfqpoint{3.071941in}{2.133551in}}%
\pgfpathlineto{\pgfqpoint{3.071941in}{2.126048in}}%
\pgfpathlineto{\pgfqpoint{3.068807in}{2.122297in}}%
\pgfpathlineto{\pgfqpoint{3.068807in}{2.114794in}}%
\pgfpathlineto{\pgfqpoint{3.065672in}{2.111042in}}%
\pgfpathlineto{\pgfqpoint{3.065672in}{2.103539in}}%
\pgfpathlineto{\pgfqpoint{3.062537in}{2.099787in}}%
\pgfpathlineto{\pgfqpoint{3.062537in}{2.092284in}}%
\pgfpathlineto{\pgfqpoint{3.059402in}{2.088533in}}%
\pgfpathlineto{\pgfqpoint{3.059402in}{2.081030in}}%
\pgfpathlineto{\pgfqpoint{3.056268in}{2.077278in}}%
\pgfpathlineto{\pgfqpoint{3.056268in}{2.069775in}}%
\pgfpathlineto{\pgfqpoint{3.053133in}{2.066024in}}%
\pgfpathlineto{\pgfqpoint{3.053133in}{2.058521in}}%
\pgfpathlineto{\pgfqpoint{3.049998in}{2.054769in}}%
\pgfpathlineto{\pgfqpoint{3.049998in}{2.043514in}}%
\pgfpathlineto{\pgfqpoint{3.046863in}{2.039763in}}%
\pgfpathlineto{\pgfqpoint{3.046863in}{2.032260in}}%
\pgfpathlineto{\pgfqpoint{3.043729in}{2.028508in}}%
\pgfpathlineto{\pgfqpoint{3.043729in}{2.021005in}}%
\pgfpathlineto{\pgfqpoint{3.040594in}{2.017254in}}%
\pgfpathlineto{\pgfqpoint{3.040594in}{2.009751in}}%
\pgfpathlineto{\pgfqpoint{3.037459in}{2.005999in}}%
\pgfpathlineto{\pgfqpoint{3.037459in}{1.998496in}}%
\pgfpathlineto{\pgfqpoint{3.034324in}{1.994745in}}%
\pgfpathlineto{\pgfqpoint{3.034324in}{1.987241in}}%
\pgfpathlineto{\pgfqpoint{3.031190in}{1.983490in}}%
\pgfpathlineto{\pgfqpoint{3.031190in}{1.975987in}}%
\pgfpathlineto{\pgfqpoint{3.028055in}{1.972235in}}%
\pgfpathlineto{\pgfqpoint{3.028055in}{1.964732in}}%
\pgfpathlineto{\pgfqpoint{3.024920in}{1.960981in}}%
\pgfpathlineto{\pgfqpoint{3.024920in}{1.949726in}}%
\pgfpathlineto{\pgfqpoint{3.021785in}{1.945975in}}%
\pgfpathlineto{\pgfqpoint{3.021785in}{1.938471in}}%
\pgfpathlineto{\pgfqpoint{3.018651in}{1.934720in}}%
\pgfpathlineto{\pgfqpoint{3.018651in}{1.927217in}}%
\pgfpathlineto{\pgfqpoint{3.015516in}{1.923465in}}%
\pgfpathlineto{\pgfqpoint{3.015516in}{1.915962in}}%
\pgfpathlineto{\pgfqpoint{3.012381in}{1.912211in}}%
\pgfpathlineto{\pgfqpoint{3.012381in}{1.904708in}}%
\pgfpathlineto{\pgfqpoint{3.009246in}{1.900956in}}%
\pgfpathlineto{\pgfqpoint{3.009246in}{1.893453in}}%
\pgfpathlineto{\pgfqpoint{3.006112in}{1.889702in}}%
\pgfpathlineto{\pgfqpoint{3.006112in}{1.882198in}}%
\pgfpathlineto{\pgfqpoint{3.002977in}{1.878447in}}%
\pgfpathlineto{\pgfqpoint{3.002977in}{1.870944in}}%
\pgfpathlineto{\pgfqpoint{2.999842in}{1.867192in}}%
\pgfpathlineto{\pgfqpoint{2.999842in}{1.859689in}}%
\pgfpathlineto{\pgfqpoint{2.996707in}{1.855938in}}%
\pgfpathlineto{\pgfqpoint{2.996707in}{1.844683in}}%
\pgfpathlineto{\pgfqpoint{2.993573in}{1.840932in}}%
\pgfpathlineto{\pgfqpoint{2.993573in}{1.833429in}}%
\pgfpathlineto{\pgfqpoint{2.990438in}{1.829677in}}%
\pgfpathlineto{\pgfqpoint{2.990438in}{1.822174in}}%
\pgfpathlineto{\pgfqpoint{2.987303in}{1.818422in}}%
\pgfpathlineto{\pgfqpoint{2.987303in}{1.810919in}}%
\pgfpathlineto{\pgfqpoint{2.984168in}{1.807168in}}%
\pgfpathlineto{\pgfqpoint{2.984168in}{1.799665in}}%
\pgfpathlineto{\pgfqpoint{2.981034in}{1.795913in}}%
\pgfpathlineto{\pgfqpoint{2.981034in}{1.788410in}}%
\pgfpathlineto{\pgfqpoint{2.977899in}{1.784659in}}%
\pgfpathlineto{\pgfqpoint{2.977899in}{1.777155in}}%
\pgfpathlineto{\pgfqpoint{2.974764in}{1.773404in}}%
\pgfpathlineto{\pgfqpoint{2.974764in}{1.765901in}}%
\pgfpathlineto{\pgfqpoint{2.971630in}{1.762149in}}%
\pgfpathlineto{\pgfqpoint{2.971630in}{1.750895in}}%
\pgfpathlineto{\pgfqpoint{2.968495in}{1.747143in}}%
\pgfpathlineto{\pgfqpoint{2.968495in}{1.739640in}}%
\pgfpathlineto{\pgfqpoint{2.965360in}{1.735889in}}%
\pgfpathlineto{\pgfqpoint{2.965360in}{1.728386in}}%
\pgfpathlineto{\pgfqpoint{2.962225in}{1.724634in}}%
\pgfpathlineto{\pgfqpoint{2.962225in}{1.717131in}}%
\pgfpathlineto{\pgfqpoint{2.959091in}{1.713379in}}%
\pgfpathlineto{\pgfqpoint{2.959091in}{1.705876in}}%
\pgfpathlineto{\pgfqpoint{2.955956in}{1.702125in}}%
\pgfpathlineto{\pgfqpoint{2.955956in}{1.694622in}}%
\pgfpathlineto{\pgfqpoint{2.952821in}{1.690870in}}%
\pgfpathlineto{\pgfqpoint{2.952821in}{1.683367in}}%
\pgfpathlineto{\pgfqpoint{2.949686in}{1.679616in}}%
\pgfpathlineto{\pgfqpoint{2.949686in}{1.672113in}}%
\pgfpathlineto{\pgfqpoint{2.946552in}{1.668361in}}%
\pgfpathlineto{\pgfqpoint{2.946552in}{1.660858in}}%
\pgfpathlineto{\pgfqpoint{2.943417in}{1.657106in}}%
\pgfpathlineto{\pgfqpoint{2.943417in}{1.645852in}}%
\pgfpathlineto{\pgfqpoint{2.940282in}{1.642100in}}%
\pgfpathlineto{\pgfqpoint{2.940282in}{1.634597in}}%
\pgfpathlineto{\pgfqpoint{2.937147in}{1.630846in}}%
\pgfpathlineto{\pgfqpoint{2.937147in}{1.623343in}}%
\pgfpathlineto{\pgfqpoint{2.934013in}{1.619591in}}%
\pgfpathlineto{\pgfqpoint{2.934013in}{1.612088in}}%
\pgfpathlineto{\pgfqpoint{2.930878in}{1.608336in}}%
\pgfpathlineto{\pgfqpoint{2.930878in}{1.600833in}}%
\pgfpathlineto{\pgfqpoint{2.927743in}{1.597082in}}%
\pgfpathlineto{\pgfqpoint{2.927743in}{1.589579in}}%
\pgfpathlineto{\pgfqpoint{2.924608in}{1.585827in}}%
\pgfpathlineto{\pgfqpoint{2.924608in}{1.578324in}}%
\pgfpathlineto{\pgfqpoint{2.921474in}{1.574573in}}%
\pgfpathlineto{\pgfqpoint{2.921474in}{1.567070in}}%
\pgfpathlineto{\pgfqpoint{2.918339in}{1.563318in}}%
\pgfpathlineto{\pgfqpoint{2.918339in}{1.552063in}}%
\pgfpathlineto{\pgfqpoint{2.915204in}{1.548312in}}%
\pgfpathlineto{\pgfqpoint{2.915204in}{1.540809in}}%
\pgfpathlineto{\pgfqpoint{2.912069in}{1.537057in}}%
\pgfpathlineto{\pgfqpoint{2.912069in}{1.529554in}}%
\pgfpathlineto{\pgfqpoint{2.908935in}{1.525803in}}%
\pgfpathlineto{\pgfqpoint{2.908935in}{1.518300in}}%
\pgfpathlineto{\pgfqpoint{2.905800in}{1.514548in}}%
\pgfpathlineto{\pgfqpoint{2.905800in}{1.507045in}}%
\pgfpathlineto{\pgfqpoint{2.902665in}{1.503293in}}%
\pgfpathlineto{\pgfqpoint{2.902665in}{1.495790in}}%
\pgfpathlineto{\pgfqpoint{2.899530in}{1.492039in}}%
\pgfpathlineto{\pgfqpoint{2.899530in}{1.484536in}}%
\pgfpathlineto{\pgfqpoint{2.896396in}{1.480784in}}%
\pgfpathlineto{\pgfqpoint{2.896396in}{1.473281in}}%
\pgfpathlineto{\pgfqpoint{2.893261in}{1.469530in}}%
\pgfpathlineto{\pgfqpoint{2.893261in}{1.462027in}}%
\pgfpathlineto{\pgfqpoint{2.890126in}{1.458275in}}%
\pgfpathlineto{\pgfqpoint{2.890126in}{1.447020in}}%
\pgfpathlineto{\pgfqpoint{2.886991in}{1.443269in}}%
\pgfpathlineto{\pgfqpoint{2.886991in}{1.435766in}}%
\pgfpathlineto{\pgfqpoint{2.883857in}{1.432014in}}%
\pgfpathlineto{\pgfqpoint{2.883857in}{1.424511in}}%
\pgfpathlineto{\pgfqpoint{2.880722in}{1.420760in}}%
\pgfpathlineto{\pgfqpoint{2.880722in}{1.413257in}}%
\pgfpathlineto{\pgfqpoint{2.877587in}{1.409505in}}%
\pgfpathlineto{\pgfqpoint{2.877587in}{1.402002in}}%
\pgfpathlineto{\pgfqpoint{2.874453in}{1.398251in}}%
\pgfpathlineto{\pgfqpoint{2.874453in}{1.390747in}}%
\pgfpathlineto{\pgfqpoint{2.871318in}{1.386996in}}%
\pgfpathlineto{\pgfqpoint{2.871318in}{1.379493in}}%
\pgfpathlineto{\pgfqpoint{2.868183in}{1.375741in}}%
\pgfpathlineto{\pgfqpoint{2.868183in}{1.368238in}}%
\pgfpathlineto{\pgfqpoint{2.865048in}{1.364487in}}%
\pgfpathlineto{\pgfqpoint{2.865048in}{1.353232in}}%
\pgfpathlineto{\pgfqpoint{2.861914in}{1.349481in}}%
\pgfpathlineto{\pgfqpoint{2.861914in}{1.341977in}}%
\pgfpathlineto{\pgfqpoint{2.858779in}{1.338226in}}%
\pgfpathlineto{\pgfqpoint{2.858779in}{1.330723in}}%
\pgfpathlineto{\pgfqpoint{2.855644in}{1.326971in}}%
\pgfpathlineto{\pgfqpoint{2.855644in}{1.319468in}}%
\pgfpathlineto{\pgfqpoint{2.852509in}{1.315717in}}%
\pgfpathlineto{\pgfqpoint{2.852509in}{1.308214in}}%
\pgfpathlineto{\pgfqpoint{2.849375in}{1.304462in}}%
\pgfpathlineto{\pgfqpoint{2.849375in}{1.296959in}}%
\pgfpathlineto{\pgfqpoint{2.846240in}{1.293208in}}%
\pgfpathlineto{\pgfqpoint{2.846240in}{1.285704in}}%
\pgfpathlineto{\pgfqpoint{2.843105in}{1.281953in}}%
\pgfpathlineto{\pgfqpoint{2.843105in}{1.274450in}}%
\pgfpathlineto{\pgfqpoint{2.839970in}{1.270698in}}%
\pgfpathlineto{\pgfqpoint{2.839970in}{1.263195in}}%
\pgfpathlineto{\pgfqpoint{2.836836in}{1.259444in}}%
\pgfpathlineto{\pgfqpoint{2.836836in}{1.248189in}}%
\pgfpathlineto{\pgfqpoint{2.833701in}{1.244438in}}%
\pgfpathlineto{\pgfqpoint{2.833701in}{1.236935in}}%
\pgfpathlineto{\pgfqpoint{2.830566in}{1.233183in}}%
\pgfpathlineto{\pgfqpoint{2.830566in}{1.225680in}}%
\pgfpathlineto{\pgfqpoint{2.827431in}{1.221928in}}%
\pgfpathlineto{\pgfqpoint{2.827431in}{1.214425in}}%
\pgfpathlineto{\pgfqpoint{2.824297in}{1.210674in}}%
\pgfpathlineto{\pgfqpoint{2.824297in}{1.203171in}}%
\pgfpathlineto{\pgfqpoint{2.821162in}{1.199419in}}%
\pgfpathlineto{\pgfqpoint{2.821162in}{1.191916in}}%
\pgfpathlineto{\pgfqpoint{2.818027in}{1.188165in}}%
\pgfpathlineto{\pgfqpoint{2.818027in}{1.180662in}}%
\pgfpathlineto{\pgfqpoint{2.814892in}{1.176910in}}%
\pgfpathlineto{\pgfqpoint{2.814892in}{1.173158in}}%
\pgfpathlineto{\pgfqpoint{2.813482in}{1.171470in}}%
\pgfpathlineto{\pgfqpoint{2.807212in}{1.171470in}}%
\pgfpathlineto{\pgfqpoint{2.804078in}{1.167719in}}%
\pgfpathlineto{\pgfqpoint{2.797808in}{1.167719in}}%
\pgfpathlineto{\pgfqpoint{2.794673in}{1.163967in}}%
\pgfpathlineto{\pgfqpoint{2.791539in}{1.163967in}}%
\pgfpathlineto{\pgfqpoint{2.788404in}{1.160216in}}%
\pgfpathlineto{\pgfqpoint{2.782134in}{1.160216in}}%
\pgfpathlineto{\pgfqpoint{2.779000in}{1.156464in}}%
\pgfpathlineto{\pgfqpoint{2.772730in}{1.156464in}}%
\pgfpathlineto{\pgfqpoint{2.769595in}{1.152713in}}%
\pgfpathlineto{\pgfqpoint{2.763326in}{1.152713in}}%
\pgfpathlineto{\pgfqpoint{2.760191in}{1.148961in}}%
\pgfpathlineto{\pgfqpoint{2.757056in}{1.148961in}}%
\pgfpathlineto{\pgfqpoint{2.753922in}{1.145210in}}%
\pgfpathlineto{\pgfqpoint{2.747652in}{1.145210in}}%
\pgfpathlineto{\pgfqpoint{2.744517in}{1.141458in}}%
\pgfpathlineto{\pgfqpoint{2.738248in}{1.141458in}}%
\pgfpathlineto{\pgfqpoint{2.735113in}{1.137706in}}%
\pgfpathlineto{\pgfqpoint{2.728844in}{1.137706in}}%
\pgfpathlineto{\pgfqpoint{2.725709in}{1.133955in}}%
\pgfpathlineto{\pgfqpoint{2.719439in}{1.133955in}}%
\pgfpathlineto{\pgfqpoint{2.716305in}{1.130203in}}%
\pgfpathlineto{\pgfqpoint{2.713170in}{1.130203in}}%
\pgfpathlineto{\pgfqpoint{2.710035in}{1.126452in}}%
\pgfpathlineto{\pgfqpoint{2.703766in}{1.126452in}}%
\pgfpathlineto{\pgfqpoint{2.700631in}{1.122700in}}%
\pgfpathlineto{\pgfqpoint{2.694362in}{1.122700in}}%
\pgfpathlineto{\pgfqpoint{2.691227in}{1.118949in}}%
\pgfpathlineto{\pgfqpoint{2.684957in}{1.118949in}}%
\pgfpathlineto{\pgfqpoint{2.681823in}{1.115197in}}%
\pgfpathlineto{\pgfqpoint{2.678688in}{1.115197in}}%
\pgfpathlineto{\pgfqpoint{2.675553in}{1.111446in}}%
\pgfpathlineto{\pgfqpoint{2.669284in}{1.111446in}}%
\pgfpathlineto{\pgfqpoint{2.666149in}{1.107694in}}%
\pgfpathlineto{\pgfqpoint{2.659879in}{1.107694in}}%
\pgfpathlineto{\pgfqpoint{2.656745in}{1.103943in}}%
\pgfpathlineto{\pgfqpoint{2.650475in}{1.103943in}}%
\pgfpathlineto{\pgfqpoint{2.647340in}{1.100191in}}%
\pgfpathlineto{\pgfqpoint{2.644206in}{1.100191in}}%
\pgfpathlineto{\pgfqpoint{2.641071in}{1.096440in}}%
\pgfpathlineto{\pgfqpoint{2.634801in}{1.096440in}}%
\pgfpathlineto{\pgfqpoint{2.631667in}{1.092688in}}%
\pgfpathlineto{\pgfqpoint{2.625397in}{1.092688in}}%
\pgfpathlineto{\pgfqpoint{2.622262in}{1.088936in}}%
\pgfpathlineto{\pgfqpoint{2.615993in}{1.088936in}}%
\pgfpathlineto{\pgfqpoint{2.612858in}{1.085185in}}%
\pgfpathlineto{\pgfqpoint{2.606589in}{1.085185in}}%
\pgfpathlineto{\pgfqpoint{2.603454in}{1.081433in}}%
\pgfpathlineto{\pgfqpoint{2.600319in}{1.081433in}}%
\pgfpathlineto{\pgfqpoint{2.597185in}{1.077682in}}%
\pgfpathlineto{\pgfqpoint{2.590915in}{1.077682in}}%
\pgfpathlineto{\pgfqpoint{2.587780in}{1.073930in}}%
\pgfpathlineto{\pgfqpoint{2.581511in}{1.073930in}}%
\pgfpathlineto{\pgfqpoint{2.578376in}{1.070179in}}%
\pgfpathlineto{\pgfqpoint{2.572107in}{1.070179in}}%
\pgfpathlineto{\pgfqpoint{2.568972in}{1.066427in}}%
\pgfpathlineto{\pgfqpoint{2.565837in}{1.066427in}}%
\pgfpathlineto{\pgfqpoint{2.562702in}{1.062676in}}%
\pgfpathlineto{\pgfqpoint{2.556433in}{1.062676in}}%
\pgfpathlineto{\pgfqpoint{2.553298in}{1.058924in}}%
\pgfpathlineto{\pgfqpoint{2.547029in}{1.058924in}}%
\pgfpathlineto{\pgfqpoint{2.543894in}{1.055173in}}%
\pgfpathlineto{\pgfqpoint{2.537624in}{1.055173in}}%
\pgfpathlineto{\pgfqpoint{2.534490in}{1.051421in}}%
\pgfpathlineto{\pgfqpoint{2.531355in}{1.051421in}}%
\pgfpathlineto{\pgfqpoint{2.528220in}{1.047670in}}%
\pgfpathlineto{\pgfqpoint{2.521951in}{1.047670in}}%
\pgfpathlineto{\pgfqpoint{2.518816in}{1.043918in}}%
\pgfpathlineto{\pgfqpoint{2.512547in}{1.043918in}}%
\pgfpathlineto{\pgfqpoint{2.509412in}{1.040167in}}%
\pgfpathlineto{\pgfqpoint{2.503142in}{1.040167in}}%
\pgfpathlineto{\pgfqpoint{2.500008in}{1.036415in}}%
\pgfpathlineto{\pgfqpoint{2.493738in}{1.036415in}}%
\pgfpathlineto{\pgfqpoint{2.490603in}{1.032663in}}%
\pgfpathlineto{\pgfqpoint{2.487469in}{1.032663in}}%
\pgfpathlineto{\pgfqpoint{2.484334in}{1.028912in}}%
\pgfpathlineto{\pgfqpoint{2.478064in}{1.028912in}}%
\pgfpathlineto{\pgfqpoint{2.474930in}{1.025160in}}%
\pgfpathlineto{\pgfqpoint{2.468660in}{1.025160in}}%
\pgfpathlineto{\pgfqpoint{2.465525in}{1.021409in}}%
\pgfpathlineto{\pgfqpoint{2.459256in}{1.021409in}}%
\pgfpathlineto{\pgfqpoint{2.456121in}{1.017657in}}%
\pgfpathlineto{\pgfqpoint{2.452986in}{1.017657in}}%
\pgfpathlineto{\pgfqpoint{2.449852in}{1.013906in}}%
\pgfpathlineto{\pgfqpoint{2.443582in}{1.013906in}}%
\pgfpathlineto{\pgfqpoint{2.440447in}{1.010154in}}%
\pgfpathlineto{\pgfqpoint{2.434178in}{1.010154in}}%
\pgfpathlineto{\pgfqpoint{2.431043in}{1.006403in}}%
\pgfpathlineto{\pgfqpoint{2.424774in}{1.006403in}}%
\pgfpathlineto{\pgfqpoint{2.421639in}{1.002651in}}%
\pgfpathlineto{\pgfqpoint{2.418504in}{1.002651in}}%
\pgfpathlineto{\pgfqpoint{2.415369in}{0.998900in}}%
\pgfpathlineto{\pgfqpoint{2.409100in}{0.998900in}}%
\pgfpathlineto{\pgfqpoint{2.405965in}{0.995148in}}%
\pgfpathlineto{\pgfqpoint{2.399696in}{0.995148in}}%
\pgfpathlineto{\pgfqpoint{2.396561in}{0.991397in}}%
\pgfpathlineto{\pgfqpoint{2.390292in}{0.991397in}}%
\pgfpathlineto{\pgfqpoint{2.387157in}{0.987645in}}%
\pgfpathlineto{\pgfqpoint{2.380887in}{0.987645in}}%
\pgfpathlineto{\pgfqpoint{2.377753in}{0.983894in}}%
\pgfpathlineto{\pgfqpoint{2.374618in}{0.983894in}}%
\pgfpathlineto{\pgfqpoint{2.371483in}{0.980142in}}%
\pgfpathlineto{\pgfqpoint{2.365214in}{0.980142in}}%
\pgfpathlineto{\pgfqpoint{2.362079in}{0.976390in}}%
\pgfpathlineto{\pgfqpoint{2.355809in}{0.976390in}}%
\pgfpathlineto{\pgfqpoint{2.352675in}{0.972639in}}%
\pgfpathlineto{\pgfqpoint{2.346405in}{0.972639in}}%
\pgfpathlineto{\pgfqpoint{2.343270in}{0.968887in}}%
\pgfpathlineto{\pgfqpoint{2.340136in}{0.968887in}}%
\pgfpathlineto{\pgfqpoint{2.337001in}{0.965136in}}%
\pgfpathlineto{\pgfqpoint{2.330731in}{0.965136in}}%
\pgfpathlineto{\pgfqpoint{2.327597in}{0.961384in}}%
\pgfpathlineto{\pgfqpoint{2.321327in}{0.961384in}}%
\pgfpathlineto{\pgfqpoint{2.318192in}{0.957633in}}%
\pgfpathlineto{\pgfqpoint{2.311923in}{0.957633in}}%
\pgfpathlineto{\pgfqpoint{2.308788in}{0.953881in}}%
\pgfpathlineto{\pgfqpoint{2.305654in}{0.953881in}}%
\pgfpathlineto{\pgfqpoint{2.302519in}{0.950130in}}%
\pgfpathlineto{\pgfqpoint{2.296249in}{0.950130in}}%
\pgfpathlineto{\pgfqpoint{2.293115in}{0.946378in}}%
\pgfpathlineto{\pgfqpoint{2.286845in}{0.946378in}}%
\pgfpathlineto{\pgfqpoint{2.283710in}{0.942627in}}%
\pgfpathlineto{\pgfqpoint{2.277441in}{0.942627in}}%
\pgfpathlineto{\pgfqpoint{2.274306in}{0.938875in}}%
\pgfpathlineto{\pgfqpoint{2.271171in}{0.938875in}}%
\pgfpathlineto{\pgfqpoint{2.268037in}{0.935124in}}%
\pgfpathlineto{\pgfqpoint{2.261767in}{0.935124in}}%
\pgfpathlineto{\pgfqpoint{2.258632in}{0.931372in}}%
\pgfpathlineto{\pgfqpoint{2.252363in}{0.931372in}}%
\pgfpathlineto{\pgfqpoint{2.249228in}{0.927621in}}%
\pgfpathlineto{\pgfqpoint{2.242959in}{0.927621in}}%
\pgfpathlineto{\pgfqpoint{2.239824in}{0.923869in}}%
\pgfpathlineto{\pgfqpoint{2.233554in}{0.923869in}}%
\pgfpathlineto{\pgfqpoint{2.230420in}{0.920117in}}%
\pgfpathlineto{\pgfqpoint{2.227285in}{0.920117in}}%
\pgfpathlineto{\pgfqpoint{2.224150in}{0.916366in}}%
\pgfpathlineto{\pgfqpoint{2.217881in}{0.916366in}}%
\pgfpathlineto{\pgfqpoint{2.214746in}{0.912614in}}%
\pgfpathlineto{\pgfqpoint{2.208477in}{0.912614in}}%
\pgfpathlineto{\pgfqpoint{2.205342in}{0.908863in}}%
\pgfpathlineto{\pgfqpoint{2.199072in}{0.908863in}}%
\pgfpathlineto{\pgfqpoint{2.195938in}{0.905111in}}%
\pgfpathlineto{\pgfqpoint{2.192803in}{0.905111in}}%
\pgfpathlineto{\pgfqpoint{2.189668in}{0.901360in}}%
\pgfpathlineto{\pgfqpoint{2.183399in}{0.901360in}}%
\pgfpathlineto{\pgfqpoint{2.180264in}{0.897608in}}%
\pgfpathlineto{\pgfqpoint{2.173994in}{0.897608in}}%
\pgfpathlineto{\pgfqpoint{2.173994in}{0.897608in}}%
\pgfusepath{stroke}%
\end{pgfscope}%
\begin{pgfscope}%
\pgfpathrectangle{\pgfqpoint{0.888750in}{0.419100in}}{\pgfqpoint{2.504659in}{2.933700in}} %
\pgfusepath{clip}%
\pgfsetbuttcap%
\pgfsetroundjoin%
\pgfsetlinewidth{1.505625pt}%
\definecolor{currentstroke}{rgb}{0.478431,0.478431,0.478431}%
\pgfsetstrokecolor{currentstroke}%
\pgfsetdash{}{0pt}%
\pgfpathmoveto{\pgfqpoint{2.173994in}{0.897045in}}%
\pgfpathlineto{\pgfqpoint{2.167725in}{0.904549in}}%
\pgfpathlineto{\pgfqpoint{2.164590in}{0.904549in}}%
\pgfpathlineto{\pgfqpoint{2.152051in}{0.919555in}}%
\pgfpathlineto{\pgfqpoint{2.148916in}{0.919555in}}%
\pgfpathlineto{\pgfqpoint{2.139512in}{0.930809in}}%
\pgfpathlineto{\pgfqpoint{2.136377in}{0.930809in}}%
\pgfpathlineto{\pgfqpoint{2.126973in}{0.942064in}}%
\pgfpathlineto{\pgfqpoint{2.123838in}{0.942064in}}%
\pgfpathlineto{\pgfqpoint{2.114434in}{0.953319in}}%
\pgfpathlineto{\pgfqpoint{2.111299in}{0.953319in}}%
\pgfpathlineto{\pgfqpoint{2.101895in}{0.964573in}}%
\pgfpathlineto{\pgfqpoint{2.098761in}{0.964573in}}%
\pgfpathlineto{\pgfqpoint{2.086222in}{0.979579in}}%
\pgfpathlineto{\pgfqpoint{2.083087in}{0.979579in}}%
\pgfpathlineto{\pgfqpoint{2.073683in}{0.990834in}}%
\pgfpathlineto{\pgfqpoint{2.070548in}{0.990834in}}%
\pgfpathlineto{\pgfqpoint{2.061144in}{1.002088in}}%
\pgfpathlineto{\pgfqpoint{2.058009in}{1.002088in}}%
\pgfpathlineto{\pgfqpoint{2.048605in}{1.013343in}}%
\pgfpathlineto{\pgfqpoint{2.045470in}{1.013343in}}%
\pgfpathlineto{\pgfqpoint{2.036066in}{1.024598in}}%
\pgfpathlineto{\pgfqpoint{2.032931in}{1.024598in}}%
\pgfpathlineto{\pgfqpoint{2.023527in}{1.035852in}}%
\pgfpathlineto{\pgfqpoint{2.020392in}{1.035852in}}%
\pgfpathlineto{\pgfqpoint{2.007853in}{1.050858in}}%
\pgfpathlineto{\pgfqpoint{2.004718in}{1.050858in}}%
\pgfpathlineto{\pgfqpoint{1.995314in}{1.062113in}}%
\pgfpathlineto{\pgfqpoint{1.992179in}{1.062113in}}%
\pgfpathlineto{\pgfqpoint{1.982775in}{1.073368in}}%
\pgfpathlineto{\pgfqpoint{1.979640in}{1.073368in}}%
\pgfpathlineto{\pgfqpoint{1.970236in}{1.084622in}}%
\pgfpathlineto{\pgfqpoint{1.967101in}{1.084622in}}%
\pgfpathlineto{\pgfqpoint{1.957697in}{1.095877in}}%
\pgfpathlineto{\pgfqpoint{1.954562in}{1.095877in}}%
\pgfpathlineto{\pgfqpoint{1.945158in}{1.107131in}}%
\pgfpathlineto{\pgfqpoint{1.942023in}{1.107131in}}%
\pgfpathlineto{\pgfqpoint{1.929484in}{1.122138in}}%
\pgfpathlineto{\pgfqpoint{1.926350in}{1.122138in}}%
\pgfpathlineto{\pgfqpoint{1.916945in}{1.133392in}}%
\pgfpathlineto{\pgfqpoint{1.913811in}{1.133392in}}%
\pgfpathlineto{\pgfqpoint{1.904407in}{1.144647in}}%
\pgfpathlineto{\pgfqpoint{1.901272in}{1.144647in}}%
\pgfpathlineto{\pgfqpoint{1.891868in}{1.155901in}}%
\pgfpathlineto{\pgfqpoint{1.888733in}{1.155901in}}%
\pgfpathlineto{\pgfqpoint{1.879329in}{1.167156in}}%
\pgfpathlineto{\pgfqpoint{1.876194in}{1.167156in}}%
\pgfpathlineto{\pgfqpoint{1.866790in}{1.178411in}}%
\pgfpathlineto{\pgfqpoint{1.863655in}{1.178411in}}%
\pgfpathlineto{\pgfqpoint{1.851116in}{1.193417in}}%
\pgfpathlineto{\pgfqpoint{1.847981in}{1.193417in}}%
\pgfpathlineto{\pgfqpoint{1.838577in}{1.204671in}}%
\pgfpathlineto{\pgfqpoint{1.835442in}{1.204671in}}%
\pgfpathlineto{\pgfqpoint{1.826038in}{1.215926in}}%
\pgfpathlineto{\pgfqpoint{1.822903in}{1.215926in}}%
\pgfpathlineto{\pgfqpoint{1.813499in}{1.227181in}}%
\pgfpathlineto{\pgfqpoint{1.810364in}{1.227181in}}%
\pgfpathlineto{\pgfqpoint{1.800960in}{1.238435in}}%
\pgfpathlineto{\pgfqpoint{1.797825in}{1.238435in}}%
\pgfpathlineto{\pgfqpoint{1.785286in}{1.253441in}}%
\pgfpathlineto{\pgfqpoint{1.782152in}{1.253441in}}%
\pgfpathlineto{\pgfqpoint{1.772747in}{1.264696in}}%
\pgfpathlineto{\pgfqpoint{1.769613in}{1.264696in}}%
\pgfpathlineto{\pgfqpoint{1.760208in}{1.275950in}}%
\pgfpathlineto{\pgfqpoint{1.757074in}{1.275950in}}%
\pgfpathlineto{\pgfqpoint{1.747669in}{1.287205in}}%
\pgfpathlineto{\pgfqpoint{1.744535in}{1.287205in}}%
\pgfpathlineto{\pgfqpoint{1.735130in}{1.298460in}}%
\pgfpathlineto{\pgfqpoint{1.731996in}{1.298460in}}%
\pgfpathlineto{\pgfqpoint{1.722591in}{1.309714in}}%
\pgfpathlineto{\pgfqpoint{1.719457in}{1.309714in}}%
\pgfpathlineto{\pgfqpoint{1.706918in}{1.324720in}}%
\pgfpathlineto{\pgfqpoint{1.703783in}{1.324720in}}%
\pgfpathlineto{\pgfqpoint{1.694379in}{1.335975in}}%
\pgfpathlineto{\pgfqpoint{1.691244in}{1.335975in}}%
\pgfpathlineto{\pgfqpoint{1.681840in}{1.347230in}}%
\pgfpathlineto{\pgfqpoint{1.678705in}{1.347230in}}%
\pgfpathlineto{\pgfqpoint{1.669301in}{1.358484in}}%
\pgfpathlineto{\pgfqpoint{1.666166in}{1.358484in}}%
\pgfpathlineto{\pgfqpoint{1.656762in}{1.369739in}}%
\pgfpathlineto{\pgfqpoint{1.653627in}{1.369739in}}%
\pgfpathlineto{\pgfqpoint{1.644223in}{1.380993in}}%
\pgfpathlineto{\pgfqpoint{1.641088in}{1.380993in}}%
\pgfpathlineto{\pgfqpoint{1.628549in}{1.396000in}}%
\pgfpathlineto{\pgfqpoint{1.625414in}{1.396000in}}%
\pgfpathlineto{\pgfqpoint{1.616010in}{1.407254in}}%
\pgfpathlineto{\pgfqpoint{1.612875in}{1.407254in}}%
\pgfpathlineto{\pgfqpoint{1.603471in}{1.418509in}}%
\pgfpathlineto{\pgfqpoint{1.600337in}{1.418509in}}%
\pgfpathlineto{\pgfqpoint{1.590932in}{1.429763in}}%
\pgfpathlineto{\pgfqpoint{1.587798in}{1.429763in}}%
\pgfpathlineto{\pgfqpoint{1.578393in}{1.441018in}}%
\pgfpathlineto{\pgfqpoint{1.575259in}{1.441018in}}%
\pgfpathlineto{\pgfqpoint{1.565854in}{1.452273in}}%
\pgfpathlineto{\pgfqpoint{1.562720in}{1.452273in}}%
\pgfpathlineto{\pgfqpoint{1.550181in}{1.467279in}}%
\pgfpathlineto{\pgfqpoint{1.547046in}{1.467279in}}%
\pgfpathlineto{\pgfqpoint{1.537642in}{1.478533in}}%
\pgfpathlineto{\pgfqpoint{1.534507in}{1.478533in}}%
\pgfpathlineto{\pgfqpoint{1.525103in}{1.489788in}}%
\pgfpathlineto{\pgfqpoint{1.521968in}{1.489788in}}%
\pgfpathlineto{\pgfqpoint{1.512564in}{1.501043in}}%
\pgfpathlineto{\pgfqpoint{1.509429in}{1.501043in}}%
\pgfpathlineto{\pgfqpoint{1.500025in}{1.512297in}}%
\pgfpathlineto{\pgfqpoint{1.496890in}{1.512297in}}%
\pgfpathlineto{\pgfqpoint{1.484351in}{1.527303in}}%
\pgfpathlineto{\pgfqpoint{1.481216in}{1.527303in}}%
\pgfpathlineto{\pgfqpoint{1.471812in}{1.538558in}}%
\pgfpathlineto{\pgfqpoint{1.468677in}{1.538558in}}%
\pgfpathlineto{\pgfqpoint{1.459273in}{1.549813in}}%
\pgfpathlineto{\pgfqpoint{1.456138in}{1.549813in}}%
\pgfpathlineto{\pgfqpoint{1.446734in}{1.561067in}}%
\pgfpathlineto{\pgfqpoint{1.443599in}{1.561067in}}%
\pgfpathlineto{\pgfqpoint{1.434195in}{1.572322in}}%
\pgfpathlineto{\pgfqpoint{1.431060in}{1.572322in}}%
\pgfpathlineto{\pgfqpoint{1.421656in}{1.583576in}}%
\pgfpathlineto{\pgfqpoint{1.418521in}{1.583576in}}%
\pgfpathlineto{\pgfqpoint{1.405982in}{1.598582in}}%
\pgfpathlineto{\pgfqpoint{1.402848in}{1.598582in}}%
\pgfpathlineto{\pgfqpoint{1.393444in}{1.609837in}}%
\pgfpathlineto{\pgfqpoint{1.390309in}{1.609837in}}%
\pgfpathlineto{\pgfqpoint{1.380905in}{1.621092in}}%
\pgfpathlineto{\pgfqpoint{1.377770in}{1.621092in}}%
\pgfpathlineto{\pgfqpoint{1.368366in}{1.632346in}}%
\pgfpathlineto{\pgfqpoint{1.365231in}{1.632346in}}%
\pgfpathlineto{\pgfqpoint{1.355827in}{1.643601in}}%
\pgfpathlineto{\pgfqpoint{1.352692in}{1.643601in}}%
\pgfpathlineto{\pgfqpoint{1.343288in}{1.654855in}}%
\pgfpathlineto{\pgfqpoint{1.340153in}{1.654855in}}%
\pgfpathlineto{\pgfqpoint{1.327614in}{1.669862in}}%
\pgfpathlineto{\pgfqpoint{1.324479in}{1.669862in}}%
\pgfpathlineto{\pgfqpoint{1.315075in}{1.681116in}}%
\pgfpathlineto{\pgfqpoint{1.311940in}{1.681116in}}%
\pgfpathlineto{\pgfqpoint{1.302536in}{1.692371in}}%
\pgfpathlineto{\pgfqpoint{1.299401in}{1.692371in}}%
\pgfpathlineto{\pgfqpoint{1.291251in}{1.702125in}}%
\pgfpathlineto{\pgfqpoint{1.291251in}{1.705876in}}%
\pgfpathlineto{\pgfqpoint{1.294386in}{1.709628in}}%
\pgfpathlineto{\pgfqpoint{1.294386in}{1.720882in}}%
\pgfpathlineto{\pgfqpoint{1.297520in}{1.724634in}}%
\pgfpathlineto{\pgfqpoint{1.297520in}{1.732137in}}%
\pgfpathlineto{\pgfqpoint{1.300655in}{1.735889in}}%
\pgfpathlineto{\pgfqpoint{1.300655in}{1.747143in}}%
\pgfpathlineto{\pgfqpoint{1.303790in}{1.750895in}}%
\pgfpathlineto{\pgfqpoint{1.303790in}{1.758398in}}%
\pgfpathlineto{\pgfqpoint{1.306925in}{1.762149in}}%
\pgfpathlineto{\pgfqpoint{1.306925in}{1.773404in}}%
\pgfpathlineto{\pgfqpoint{1.310059in}{1.777155in}}%
\pgfpathlineto{\pgfqpoint{1.310059in}{1.784659in}}%
\pgfpathlineto{\pgfqpoint{1.313194in}{1.788410in}}%
\pgfpathlineto{\pgfqpoint{1.313194in}{1.799665in}}%
\pgfpathlineto{\pgfqpoint{1.316329in}{1.803416in}}%
\pgfpathlineto{\pgfqpoint{1.316329in}{1.810919in}}%
\pgfpathlineto{\pgfqpoint{1.319464in}{1.814671in}}%
\pgfpathlineto{\pgfqpoint{1.319464in}{1.822174in}}%
\pgfpathlineto{\pgfqpoint{1.322598in}{1.825925in}}%
\pgfpathlineto{\pgfqpoint{1.322598in}{1.837180in}}%
\pgfpathlineto{\pgfqpoint{1.325733in}{1.840932in}}%
\pgfpathlineto{\pgfqpoint{1.325733in}{1.848435in}}%
\pgfpathlineto{\pgfqpoint{1.328868in}{1.852186in}}%
\pgfpathlineto{\pgfqpoint{1.328868in}{1.863441in}}%
\pgfpathlineto{\pgfqpoint{1.332003in}{1.867192in}}%
\pgfpathlineto{\pgfqpoint{1.332003in}{1.874695in}}%
\pgfpathlineto{\pgfqpoint{1.335137in}{1.878447in}}%
\pgfpathlineto{\pgfqpoint{1.335137in}{1.889702in}}%
\pgfpathlineto{\pgfqpoint{1.338272in}{1.893453in}}%
\pgfpathlineto{\pgfqpoint{1.338272in}{1.900956in}}%
\pgfpathlineto{\pgfqpoint{1.341407in}{1.904708in}}%
\pgfpathlineto{\pgfqpoint{1.341407in}{1.915962in}}%
\pgfpathlineto{\pgfqpoint{1.344542in}{1.919714in}}%
\pgfpathlineto{\pgfqpoint{1.344542in}{1.927217in}}%
\pgfpathlineto{\pgfqpoint{1.347676in}{1.930968in}}%
\pgfpathlineto{\pgfqpoint{1.347676in}{1.942223in}}%
\pgfpathlineto{\pgfqpoint{1.350811in}{1.945975in}}%
\pgfpathlineto{\pgfqpoint{1.350811in}{1.953478in}}%
\pgfpathlineto{\pgfqpoint{1.353946in}{1.957229in}}%
\pgfpathlineto{\pgfqpoint{1.353946in}{1.968484in}}%
\pgfpathlineto{\pgfqpoint{1.357080in}{1.972235in}}%
\pgfpathlineto{\pgfqpoint{1.357080in}{1.979738in}}%
\pgfpathlineto{\pgfqpoint{1.360215in}{1.983490in}}%
\pgfpathlineto{\pgfqpoint{1.360215in}{1.994745in}}%
\pgfpathlineto{\pgfqpoint{1.363350in}{1.998496in}}%
\pgfpathlineto{\pgfqpoint{1.363350in}{2.005999in}}%
\pgfpathlineto{\pgfqpoint{1.366485in}{2.009751in}}%
\pgfpathlineto{\pgfqpoint{1.366485in}{2.021005in}}%
\pgfpathlineto{\pgfqpoint{1.369619in}{2.024757in}}%
\pgfpathlineto{\pgfqpoint{1.369619in}{2.032260in}}%
\pgfpathlineto{\pgfqpoint{1.372754in}{2.036011in}}%
\pgfpathlineto{\pgfqpoint{1.372754in}{2.043514in}}%
\pgfpathlineto{\pgfqpoint{1.375889in}{2.047266in}}%
\pgfpathlineto{\pgfqpoint{1.375889in}{2.058521in}}%
\pgfpathlineto{\pgfqpoint{1.379024in}{2.062272in}}%
\pgfpathlineto{\pgfqpoint{1.379024in}{2.069775in}}%
\pgfpathlineto{\pgfqpoint{1.382158in}{2.073527in}}%
\pgfpathlineto{\pgfqpoint{1.382158in}{2.084781in}}%
\pgfpathlineto{\pgfqpoint{1.385293in}{2.088533in}}%
\pgfpathlineto{\pgfqpoint{1.385293in}{2.096036in}}%
\pgfpathlineto{\pgfqpoint{1.388428in}{2.099787in}}%
\pgfpathlineto{\pgfqpoint{1.388428in}{2.111042in}}%
\pgfpathlineto{\pgfqpoint{1.391563in}{2.114794in}}%
\pgfpathlineto{\pgfqpoint{1.391563in}{2.122297in}}%
\pgfpathlineto{\pgfqpoint{1.394697in}{2.126048in}}%
\pgfpathlineto{\pgfqpoint{1.394697in}{2.137303in}}%
\pgfpathlineto{\pgfqpoint{1.397832in}{2.141054in}}%
\pgfpathlineto{\pgfqpoint{1.397832in}{2.148557in}}%
\pgfpathlineto{\pgfqpoint{1.400967in}{2.152309in}}%
\pgfpathlineto{\pgfqpoint{1.400967in}{2.163564in}}%
\pgfpathlineto{\pgfqpoint{1.404102in}{2.167315in}}%
\pgfpathlineto{\pgfqpoint{1.404102in}{2.174818in}}%
\pgfpathlineto{\pgfqpoint{1.407236in}{2.178570in}}%
\pgfpathlineto{\pgfqpoint{1.407236in}{2.189824in}}%
\pgfpathlineto{\pgfqpoint{1.410371in}{2.193576in}}%
\pgfpathlineto{\pgfqpoint{1.410371in}{2.201079in}}%
\pgfpathlineto{\pgfqpoint{1.413506in}{2.204830in}}%
\pgfpathlineto{\pgfqpoint{1.413506in}{2.216085in}}%
\pgfpathlineto{\pgfqpoint{1.416641in}{2.219837in}}%
\pgfpathlineto{\pgfqpoint{1.416641in}{2.227340in}}%
\pgfpathlineto{\pgfqpoint{1.419775in}{2.231091in}}%
\pgfpathlineto{\pgfqpoint{1.419775in}{2.242346in}}%
\pgfpathlineto{\pgfqpoint{1.422910in}{2.246097in}}%
\pgfpathlineto{\pgfqpoint{1.422910in}{2.253600in}}%
\pgfpathlineto{\pgfqpoint{1.426045in}{2.257352in}}%
\pgfpathlineto{\pgfqpoint{1.426045in}{2.264855in}}%
\pgfpathlineto{\pgfqpoint{1.429180in}{2.268607in}}%
\pgfpathlineto{\pgfqpoint{1.429180in}{2.279861in}}%
\pgfpathlineto{\pgfqpoint{1.432314in}{2.283613in}}%
\pgfpathlineto{\pgfqpoint{1.432314in}{2.291116in}}%
\pgfpathlineto{\pgfqpoint{1.435449in}{2.294867in}}%
\pgfpathlineto{\pgfqpoint{1.435449in}{2.306122in}}%
\pgfpathlineto{\pgfqpoint{1.438584in}{2.309873in}}%
\pgfpathlineto{\pgfqpoint{1.438584in}{2.317376in}}%
\pgfpathlineto{\pgfqpoint{1.441719in}{2.321128in}}%
\pgfpathlineto{\pgfqpoint{1.441719in}{2.332383in}}%
\pgfpathlineto{\pgfqpoint{1.444853in}{2.336134in}}%
\pgfpathlineto{\pgfqpoint{1.444853in}{2.343637in}}%
\pgfpathlineto{\pgfqpoint{1.447988in}{2.347389in}}%
\pgfpathlineto{\pgfqpoint{1.447988in}{2.358643in}}%
\pgfpathlineto{\pgfqpoint{1.451123in}{2.362395in}}%
\pgfpathlineto{\pgfqpoint{1.451123in}{2.369898in}}%
\pgfpathlineto{\pgfqpoint{1.454258in}{2.373649in}}%
\pgfpathlineto{\pgfqpoint{1.454258in}{2.384904in}}%
\pgfpathlineto{\pgfqpoint{1.457392in}{2.388656in}}%
\pgfpathlineto{\pgfqpoint{1.457392in}{2.396159in}}%
\pgfpathlineto{\pgfqpoint{1.460527in}{2.399910in}}%
\pgfpathlineto{\pgfqpoint{1.460527in}{2.403662in}}%
\pgfpathlineto{\pgfqpoint{1.462408in}{2.405913in}}%
\pgfpathlineto{\pgfqpoint{1.465543in}{2.405913in}}%
\pgfpathlineto{\pgfqpoint{1.468677in}{2.409664in}}%
\pgfpathlineto{\pgfqpoint{1.478082in}{2.409664in}}%
\pgfpathlineto{\pgfqpoint{1.481216in}{2.413416in}}%
\pgfpathlineto{\pgfqpoint{1.490621in}{2.413416in}}%
\pgfpathlineto{\pgfqpoint{1.493755in}{2.417167in}}%
\pgfpathlineto{\pgfqpoint{1.503159in}{2.417167in}}%
\pgfpathlineto{\pgfqpoint{1.506294in}{2.420919in}}%
\pgfpathlineto{\pgfqpoint{1.515698in}{2.420919in}}%
\pgfpathlineto{\pgfqpoint{1.518833in}{2.424670in}}%
\pgfpathlineto{\pgfqpoint{1.531372in}{2.424670in}}%
\pgfpathlineto{\pgfqpoint{1.534507in}{2.428422in}}%
\pgfpathlineto{\pgfqpoint{1.543911in}{2.428422in}}%
\pgfpathlineto{\pgfqpoint{1.547046in}{2.432173in}}%
\pgfpathlineto{\pgfqpoint{1.556450in}{2.432173in}}%
\pgfpathlineto{\pgfqpoint{1.559585in}{2.435925in}}%
\pgfpathlineto{\pgfqpoint{1.568989in}{2.435925in}}%
\pgfpathlineto{\pgfqpoint{1.572124in}{2.439676in}}%
\pgfpathlineto{\pgfqpoint{1.584663in}{2.439676in}}%
\pgfpathlineto{\pgfqpoint{1.587798in}{2.443428in}}%
\pgfpathlineto{\pgfqpoint{1.597202in}{2.443428in}}%
\pgfpathlineto{\pgfqpoint{1.600337in}{2.447180in}}%
\pgfpathlineto{\pgfqpoint{1.609741in}{2.447180in}}%
\pgfpathlineto{\pgfqpoint{1.612875in}{2.450931in}}%
\pgfpathlineto{\pgfqpoint{1.622280in}{2.450931in}}%
\pgfpathlineto{\pgfqpoint{1.625414in}{2.454683in}}%
\pgfpathlineto{\pgfqpoint{1.637953in}{2.454683in}}%
\pgfpathlineto{\pgfqpoint{1.641088in}{2.458434in}}%
\pgfpathlineto{\pgfqpoint{1.650492in}{2.458434in}}%
\pgfpathlineto{\pgfqpoint{1.653627in}{2.462186in}}%
\pgfpathlineto{\pgfqpoint{1.663031in}{2.462186in}}%
\pgfpathlineto{\pgfqpoint{1.666166in}{2.465937in}}%
\pgfpathlineto{\pgfqpoint{1.675570in}{2.465937in}}%
\pgfpathlineto{\pgfqpoint{1.678705in}{2.469689in}}%
\pgfpathlineto{\pgfqpoint{1.688109in}{2.469689in}}%
\pgfpathlineto{\pgfqpoint{1.691244in}{2.473440in}}%
\pgfpathlineto{\pgfqpoint{1.703783in}{2.473440in}}%
\pgfpathlineto{\pgfqpoint{1.706918in}{2.477192in}}%
\pgfpathlineto{\pgfqpoint{1.716322in}{2.477192in}}%
\pgfpathlineto{\pgfqpoint{1.719457in}{2.480943in}}%
\pgfpathlineto{\pgfqpoint{1.728861in}{2.480943in}}%
\pgfpathlineto{\pgfqpoint{1.731996in}{2.484695in}}%
\pgfpathlineto{\pgfqpoint{1.741400in}{2.484695in}}%
\pgfpathlineto{\pgfqpoint{1.744535in}{2.488446in}}%
\pgfpathlineto{\pgfqpoint{1.757074in}{2.488446in}}%
\pgfpathlineto{\pgfqpoint{1.760208in}{2.492198in}}%
\pgfpathlineto{\pgfqpoint{1.769613in}{2.492198in}}%
\pgfpathlineto{\pgfqpoint{1.772747in}{2.495950in}}%
\pgfpathlineto{\pgfqpoint{1.782152in}{2.495950in}}%
\pgfpathlineto{\pgfqpoint{1.785286in}{2.499701in}}%
\pgfpathlineto{\pgfqpoint{1.794691in}{2.499701in}}%
\pgfpathlineto{\pgfqpoint{1.797825in}{2.503453in}}%
\pgfpathlineto{\pgfqpoint{1.810364in}{2.503453in}}%
\pgfpathlineto{\pgfqpoint{1.813499in}{2.507204in}}%
\pgfpathlineto{\pgfqpoint{1.822903in}{2.507204in}}%
\pgfpathlineto{\pgfqpoint{1.826038in}{2.510956in}}%
\pgfpathlineto{\pgfqpoint{1.835442in}{2.510956in}}%
\pgfpathlineto{\pgfqpoint{1.838577in}{2.514707in}}%
\pgfpathlineto{\pgfqpoint{1.847981in}{2.514707in}}%
\pgfpathlineto{\pgfqpoint{1.851116in}{2.518459in}}%
\pgfpathlineto{\pgfqpoint{1.860520in}{2.518459in}}%
\pgfpathlineto{\pgfqpoint{1.863655in}{2.522210in}}%
\pgfpathlineto{\pgfqpoint{1.876194in}{2.522210in}}%
\pgfpathlineto{\pgfqpoint{1.879329in}{2.525962in}}%
\pgfpathlineto{\pgfqpoint{1.888733in}{2.525962in}}%
\pgfpathlineto{\pgfqpoint{1.891868in}{2.529713in}}%
\pgfpathlineto{\pgfqpoint{1.901272in}{2.529713in}}%
\pgfpathlineto{\pgfqpoint{1.904407in}{2.533465in}}%
\pgfpathlineto{\pgfqpoint{1.913811in}{2.533465in}}%
\pgfpathlineto{\pgfqpoint{1.916945in}{2.537216in}}%
\pgfpathlineto{\pgfqpoint{1.929484in}{2.537216in}}%
\pgfpathlineto{\pgfqpoint{1.932619in}{2.540968in}}%
\pgfpathlineto{\pgfqpoint{1.942023in}{2.540968in}}%
\pgfpathlineto{\pgfqpoint{1.945158in}{2.544719in}}%
\pgfpathlineto{\pgfqpoint{1.954562in}{2.544719in}}%
\pgfpathlineto{\pgfqpoint{1.957697in}{2.548471in}}%
\pgfpathlineto{\pgfqpoint{1.967101in}{2.548471in}}%
\pgfpathlineto{\pgfqpoint{1.970236in}{2.552223in}}%
\pgfpathlineto{\pgfqpoint{1.979640in}{2.552223in}}%
\pgfpathlineto{\pgfqpoint{1.982775in}{2.555974in}}%
\pgfpathlineto{\pgfqpoint{1.995314in}{2.555974in}}%
\pgfpathlineto{\pgfqpoint{1.998449in}{2.559726in}}%
\pgfpathlineto{\pgfqpoint{2.007853in}{2.559726in}}%
\pgfpathlineto{\pgfqpoint{2.010988in}{2.563477in}}%
\pgfpathlineto{\pgfqpoint{2.020392in}{2.563477in}}%
\pgfpathlineto{\pgfqpoint{2.023527in}{2.567229in}}%
\pgfpathlineto{\pgfqpoint{2.032931in}{2.567229in}}%
\pgfpathlineto{\pgfqpoint{2.036066in}{2.570980in}}%
\pgfpathlineto{\pgfqpoint{2.048605in}{2.570980in}}%
\pgfpathlineto{\pgfqpoint{2.051739in}{2.574732in}}%
\pgfpathlineto{\pgfqpoint{2.061144in}{2.574732in}}%
\pgfpathlineto{\pgfqpoint{2.064278in}{2.578483in}}%
\pgfpathlineto{\pgfqpoint{2.073683in}{2.578483in}}%
\pgfpathlineto{\pgfqpoint{2.076817in}{2.582235in}}%
\pgfpathlineto{\pgfqpoint{2.086222in}{2.582235in}}%
\pgfpathlineto{\pgfqpoint{2.089356in}{2.585986in}}%
\pgfpathlineto{\pgfqpoint{2.101895in}{2.585986in}}%
\pgfpathlineto{\pgfqpoint{2.105030in}{2.589738in}}%
\pgfpathlineto{\pgfqpoint{2.114434in}{2.589738in}}%
\pgfpathlineto{\pgfqpoint{2.117569in}{2.593489in}}%
\pgfpathlineto{\pgfqpoint{2.126973in}{2.593489in}}%
\pgfpathlineto{\pgfqpoint{2.130108in}{2.597241in}}%
\pgfpathlineto{\pgfqpoint{2.139512in}{2.597241in}}%
\pgfpathlineto{\pgfqpoint{2.142647in}{2.600992in}}%
\pgfpathlineto{\pgfqpoint{2.152051in}{2.600992in}}%
\pgfpathlineto{\pgfqpoint{2.155186in}{2.604744in}}%
\pgfpathlineto{\pgfqpoint{2.167725in}{2.604744in}}%
\pgfpathlineto{\pgfqpoint{2.170860in}{2.608496in}}%
\pgfpathlineto{\pgfqpoint{2.180264in}{2.608496in}}%
\pgfpathlineto{\pgfqpoint{2.183399in}{2.612247in}}%
\pgfpathlineto{\pgfqpoint{2.192803in}{2.612247in}}%
\pgfpathlineto{\pgfqpoint{2.195938in}{2.615999in}}%
\pgfpathlineto{\pgfqpoint{2.205342in}{2.615999in}}%
\pgfpathlineto{\pgfqpoint{2.208477in}{2.619750in}}%
\pgfpathlineto{\pgfqpoint{2.221015in}{2.619750in}}%
\pgfpathlineto{\pgfqpoint{2.224150in}{2.623502in}}%
\pgfpathlineto{\pgfqpoint{2.233554in}{2.623502in}}%
\pgfpathlineto{\pgfqpoint{2.236689in}{2.627253in}}%
\pgfpathlineto{\pgfqpoint{2.246093in}{2.627253in}}%
\pgfpathlineto{\pgfqpoint{2.249228in}{2.631005in}}%
\pgfpathlineto{\pgfqpoint{2.258632in}{2.631005in}}%
\pgfpathlineto{\pgfqpoint{2.261767in}{2.634756in}}%
\pgfpathlineto{\pgfqpoint{2.274306in}{2.634756in}}%
\pgfpathlineto{\pgfqpoint{2.277441in}{2.638508in}}%
\pgfpathlineto{\pgfqpoint{2.286845in}{2.638508in}}%
\pgfpathlineto{\pgfqpoint{2.289980in}{2.642259in}}%
\pgfpathlineto{\pgfqpoint{2.299384in}{2.642259in}}%
\pgfpathlineto{\pgfqpoint{2.302519in}{2.646011in}}%
\pgfpathlineto{\pgfqpoint{2.311923in}{2.646011in}}%
\pgfpathlineto{\pgfqpoint{2.315058in}{2.649762in}}%
\pgfpathlineto{\pgfqpoint{2.324462in}{2.649762in}}%
\pgfpathlineto{\pgfqpoint{2.327597in}{2.653514in}}%
\pgfpathlineto{\pgfqpoint{2.340136in}{2.653514in}}%
\pgfpathlineto{\pgfqpoint{2.343270in}{2.657265in}}%
\pgfpathlineto{\pgfqpoint{2.352675in}{2.657265in}}%
\pgfpathlineto{\pgfqpoint{2.355809in}{2.661017in}}%
\pgfpathlineto{\pgfqpoint{2.365214in}{2.661017in}}%
\pgfpathlineto{\pgfqpoint{2.368348in}{2.664769in}}%
\pgfpathlineto{\pgfqpoint{2.377753in}{2.664769in}}%
\pgfpathlineto{\pgfqpoint{2.380887in}{2.668520in}}%
\pgfpathlineto{\pgfqpoint{2.393426in}{2.668520in}}%
\pgfpathlineto{\pgfqpoint{2.396561in}{2.672272in}}%
\pgfpathlineto{\pgfqpoint{2.405965in}{2.672272in}}%
\pgfpathlineto{\pgfqpoint{2.409100in}{2.676023in}}%
\pgfpathlineto{\pgfqpoint{2.418504in}{2.676023in}}%
\pgfpathlineto{\pgfqpoint{2.421639in}{2.679775in}}%
\pgfpathlineto{\pgfqpoint{2.431043in}{2.679775in}}%
\pgfpathlineto{\pgfqpoint{2.434178in}{2.683526in}}%
\pgfpathlineto{\pgfqpoint{2.446717in}{2.683526in}}%
\pgfpathlineto{\pgfqpoint{2.449852in}{2.687278in}}%
\pgfpathlineto{\pgfqpoint{2.459256in}{2.687278in}}%
\pgfpathlineto{\pgfqpoint{2.462391in}{2.691029in}}%
\pgfpathlineto{\pgfqpoint{2.471795in}{2.691029in}}%
\pgfpathlineto{\pgfqpoint{2.474930in}{2.694781in}}%
\pgfpathlineto{\pgfqpoint{2.484334in}{2.694781in}}%
\pgfpathlineto{\pgfqpoint{2.487469in}{2.698532in}}%
\pgfpathlineto{\pgfqpoint{2.496873in}{2.698532in}}%
\pgfpathlineto{\pgfqpoint{2.500008in}{2.702284in}}%
\pgfpathlineto{\pgfqpoint{2.512547in}{2.702284in}}%
\pgfpathlineto{\pgfqpoint{2.515681in}{2.706035in}}%
\pgfpathlineto{\pgfqpoint{2.525085in}{2.706035in}}%
\pgfpathlineto{\pgfqpoint{2.528220in}{2.709787in}}%
\pgfpathlineto{\pgfqpoint{2.537624in}{2.709787in}}%
\pgfpathlineto{\pgfqpoint{2.540759in}{2.713539in}}%
\pgfpathlineto{\pgfqpoint{2.550163in}{2.713539in}}%
\pgfpathlineto{\pgfqpoint{2.553298in}{2.717290in}}%
\pgfpathlineto{\pgfqpoint{2.565837in}{2.717290in}}%
\pgfpathlineto{\pgfqpoint{2.568972in}{2.721042in}}%
\pgfpathlineto{\pgfqpoint{2.578376in}{2.721042in}}%
\pgfpathlineto{\pgfqpoint{2.581511in}{2.724793in}}%
\pgfpathlineto{\pgfqpoint{2.590915in}{2.724793in}}%
\pgfpathlineto{\pgfqpoint{2.594050in}{2.728545in}}%
\pgfpathlineto{\pgfqpoint{2.603454in}{2.728545in}}%
\pgfpathlineto{\pgfqpoint{2.606589in}{2.732296in}}%
\pgfpathlineto{\pgfqpoint{2.615993in}{2.732296in}}%
\pgfpathlineto{\pgfqpoint{2.619128in}{2.736048in}}%
\pgfpathlineto{\pgfqpoint{2.631667in}{2.736048in}}%
\pgfpathlineto{\pgfqpoint{2.634801in}{2.739799in}}%
\pgfpathlineto{\pgfqpoint{2.644206in}{2.739799in}}%
\pgfpathlineto{\pgfqpoint{2.647340in}{2.743551in}}%
\pgfpathlineto{\pgfqpoint{2.656745in}{2.743551in}}%
\pgfpathlineto{\pgfqpoint{2.659879in}{2.747302in}}%
\pgfpathlineto{\pgfqpoint{2.672418in}{2.747302in}}%
\pgfpathlineto{\pgfqpoint{2.711916in}{2.700033in}}%
\pgfpathlineto{\pgfqpoint{2.711916in}{2.696281in}}%
\pgfpathlineto{\pgfqpoint{2.762072in}{2.636257in}}%
\pgfpathlineto{\pgfqpoint{2.762072in}{2.632505in}}%
\pgfpathlineto{\pgfqpoint{2.809093in}{2.576232in}}%
\pgfpathlineto{\pgfqpoint{2.809093in}{2.572481in}}%
\pgfpathlineto{\pgfqpoint{2.859249in}{2.512456in}}%
\pgfpathlineto{\pgfqpoint{2.859249in}{2.508705in}}%
\pgfpathlineto{\pgfqpoint{2.906270in}{2.452432in}}%
\pgfpathlineto{\pgfqpoint{2.906270in}{2.448680in}}%
\pgfpathlineto{\pgfqpoint{2.956426in}{2.388656in}}%
\pgfpathlineto{\pgfqpoint{2.956426in}{2.384904in}}%
\pgfpathlineto{\pgfqpoint{3.003447in}{2.328631in}}%
\pgfpathlineto{\pgfqpoint{3.003447in}{2.324880in}}%
\pgfpathlineto{\pgfqpoint{3.053603in}{2.264855in}}%
\pgfpathlineto{\pgfqpoint{3.053603in}{2.261103in}}%
\pgfpathlineto{\pgfqpoint{3.094355in}{2.212334in}}%
\pgfpathlineto{\pgfqpoint{3.094355in}{2.208582in}}%
\pgfpathlineto{\pgfqpoint{3.091220in}{2.204830in}}%
\pgfpathlineto{\pgfqpoint{3.091220in}{2.197327in}}%
\pgfpathlineto{\pgfqpoint{3.088085in}{2.193576in}}%
\pgfpathlineto{\pgfqpoint{3.088085in}{2.186073in}}%
\pgfpathlineto{\pgfqpoint{3.084950in}{2.182321in}}%
\pgfpathlineto{\pgfqpoint{3.084950in}{2.174818in}}%
\pgfpathlineto{\pgfqpoint{3.081816in}{2.171067in}}%
\pgfpathlineto{\pgfqpoint{3.081816in}{2.163564in}}%
\pgfpathlineto{\pgfqpoint{3.078681in}{2.159812in}}%
\pgfpathlineto{\pgfqpoint{3.078681in}{2.148557in}}%
\pgfpathlineto{\pgfqpoint{3.075546in}{2.144806in}}%
\pgfpathlineto{\pgfqpoint{3.075546in}{2.137303in}}%
\pgfpathlineto{\pgfqpoint{3.072411in}{2.133551in}}%
\pgfpathlineto{\pgfqpoint{3.072411in}{2.126048in}}%
\pgfpathlineto{\pgfqpoint{3.069277in}{2.122297in}}%
\pgfpathlineto{\pgfqpoint{3.069277in}{2.114794in}}%
\pgfpathlineto{\pgfqpoint{3.066142in}{2.111042in}}%
\pgfpathlineto{\pgfqpoint{3.066142in}{2.103539in}}%
\pgfpathlineto{\pgfqpoint{3.063007in}{2.099787in}}%
\pgfpathlineto{\pgfqpoint{3.063007in}{2.092284in}}%
\pgfpathlineto{\pgfqpoint{3.059873in}{2.088533in}}%
\pgfpathlineto{\pgfqpoint{3.059873in}{2.081030in}}%
\pgfpathlineto{\pgfqpoint{3.056738in}{2.077278in}}%
\pgfpathlineto{\pgfqpoint{3.056738in}{2.069775in}}%
\pgfpathlineto{\pgfqpoint{3.053603in}{2.066024in}}%
\pgfpathlineto{\pgfqpoint{3.053603in}{2.058521in}}%
\pgfpathlineto{\pgfqpoint{3.050468in}{2.054769in}}%
\pgfpathlineto{\pgfqpoint{3.050468in}{2.043514in}}%
\pgfpathlineto{\pgfqpoint{3.047334in}{2.039763in}}%
\pgfpathlineto{\pgfqpoint{3.047334in}{2.032260in}}%
\pgfpathlineto{\pgfqpoint{3.044199in}{2.028508in}}%
\pgfpathlineto{\pgfqpoint{3.044199in}{2.021005in}}%
\pgfpathlineto{\pgfqpoint{3.041064in}{2.017254in}}%
\pgfpathlineto{\pgfqpoint{3.041064in}{2.009751in}}%
\pgfpathlineto{\pgfqpoint{3.037929in}{2.005999in}}%
\pgfpathlineto{\pgfqpoint{3.037929in}{1.998496in}}%
\pgfpathlineto{\pgfqpoint{3.034795in}{1.994745in}}%
\pgfpathlineto{\pgfqpoint{3.034795in}{1.987241in}}%
\pgfpathlineto{\pgfqpoint{3.031660in}{1.983490in}}%
\pgfpathlineto{\pgfqpoint{3.031660in}{1.975987in}}%
\pgfpathlineto{\pgfqpoint{3.028525in}{1.972235in}}%
\pgfpathlineto{\pgfqpoint{3.028525in}{1.964732in}}%
\pgfpathlineto{\pgfqpoint{3.025390in}{1.960981in}}%
\pgfpathlineto{\pgfqpoint{3.025390in}{1.949726in}}%
\pgfpathlineto{\pgfqpoint{3.022256in}{1.945975in}}%
\pgfpathlineto{\pgfqpoint{3.022256in}{1.938471in}}%
\pgfpathlineto{\pgfqpoint{3.019121in}{1.934720in}}%
\pgfpathlineto{\pgfqpoint{3.019121in}{1.927217in}}%
\pgfpathlineto{\pgfqpoint{3.015986in}{1.923465in}}%
\pgfpathlineto{\pgfqpoint{3.015986in}{1.915962in}}%
\pgfpathlineto{\pgfqpoint{3.012851in}{1.912211in}}%
\pgfpathlineto{\pgfqpoint{3.012851in}{1.904708in}}%
\pgfpathlineto{\pgfqpoint{3.009717in}{1.900956in}}%
\pgfpathlineto{\pgfqpoint{3.009717in}{1.893453in}}%
\pgfpathlineto{\pgfqpoint{3.006582in}{1.889702in}}%
\pgfpathlineto{\pgfqpoint{3.006582in}{1.882198in}}%
\pgfpathlineto{\pgfqpoint{3.003447in}{1.878447in}}%
\pgfpathlineto{\pgfqpoint{3.003447in}{1.870944in}}%
\pgfpathlineto{\pgfqpoint{3.000312in}{1.867192in}}%
\pgfpathlineto{\pgfqpoint{3.000312in}{1.859689in}}%
\pgfpathlineto{\pgfqpoint{2.997178in}{1.855938in}}%
\pgfpathlineto{\pgfqpoint{2.997178in}{1.844683in}}%
\pgfpathlineto{\pgfqpoint{2.994043in}{1.840932in}}%
\pgfpathlineto{\pgfqpoint{2.994043in}{1.833429in}}%
\pgfpathlineto{\pgfqpoint{2.990908in}{1.829677in}}%
\pgfpathlineto{\pgfqpoint{2.990908in}{1.822174in}}%
\pgfpathlineto{\pgfqpoint{2.987773in}{1.818422in}}%
\pgfpathlineto{\pgfqpoint{2.987773in}{1.810919in}}%
\pgfpathlineto{\pgfqpoint{2.984639in}{1.807168in}}%
\pgfpathlineto{\pgfqpoint{2.984639in}{1.799665in}}%
\pgfpathlineto{\pgfqpoint{2.981504in}{1.795913in}}%
\pgfpathlineto{\pgfqpoint{2.981504in}{1.788410in}}%
\pgfpathlineto{\pgfqpoint{2.978369in}{1.784659in}}%
\pgfpathlineto{\pgfqpoint{2.978369in}{1.777155in}}%
\pgfpathlineto{\pgfqpoint{2.975234in}{1.773404in}}%
\pgfpathlineto{\pgfqpoint{2.975234in}{1.765901in}}%
\pgfpathlineto{\pgfqpoint{2.972100in}{1.762149in}}%
\pgfpathlineto{\pgfqpoint{2.972100in}{1.750895in}}%
\pgfpathlineto{\pgfqpoint{2.968965in}{1.747143in}}%
\pgfpathlineto{\pgfqpoint{2.968965in}{1.739640in}}%
\pgfpathlineto{\pgfqpoint{2.965830in}{1.735889in}}%
\pgfpathlineto{\pgfqpoint{2.965830in}{1.728386in}}%
\pgfpathlineto{\pgfqpoint{2.962696in}{1.724634in}}%
\pgfpathlineto{\pgfqpoint{2.962696in}{1.717131in}}%
\pgfpathlineto{\pgfqpoint{2.959561in}{1.713379in}}%
\pgfpathlineto{\pgfqpoint{2.959561in}{1.705876in}}%
\pgfpathlineto{\pgfqpoint{2.956426in}{1.702125in}}%
\pgfpathlineto{\pgfqpoint{2.956426in}{1.694622in}}%
\pgfpathlineto{\pgfqpoint{2.953291in}{1.690870in}}%
\pgfpathlineto{\pgfqpoint{2.953291in}{1.683367in}}%
\pgfpathlineto{\pgfqpoint{2.950157in}{1.679616in}}%
\pgfpathlineto{\pgfqpoint{2.950157in}{1.672113in}}%
\pgfpathlineto{\pgfqpoint{2.947022in}{1.668361in}}%
\pgfpathlineto{\pgfqpoint{2.947022in}{1.660858in}}%
\pgfpathlineto{\pgfqpoint{2.943887in}{1.657106in}}%
\pgfpathlineto{\pgfqpoint{2.943887in}{1.645852in}}%
\pgfpathlineto{\pgfqpoint{2.940752in}{1.642100in}}%
\pgfpathlineto{\pgfqpoint{2.940752in}{1.634597in}}%
\pgfpathlineto{\pgfqpoint{2.937618in}{1.630846in}}%
\pgfpathlineto{\pgfqpoint{2.937618in}{1.623343in}}%
\pgfpathlineto{\pgfqpoint{2.934483in}{1.619591in}}%
\pgfpathlineto{\pgfqpoint{2.934483in}{1.612088in}}%
\pgfpathlineto{\pgfqpoint{2.931348in}{1.608336in}}%
\pgfpathlineto{\pgfqpoint{2.931348in}{1.600833in}}%
\pgfpathlineto{\pgfqpoint{2.928213in}{1.597082in}}%
\pgfpathlineto{\pgfqpoint{2.928213in}{1.589579in}}%
\pgfpathlineto{\pgfqpoint{2.925079in}{1.585827in}}%
\pgfpathlineto{\pgfqpoint{2.925079in}{1.578324in}}%
\pgfpathlineto{\pgfqpoint{2.921944in}{1.574573in}}%
\pgfpathlineto{\pgfqpoint{2.921944in}{1.567070in}}%
\pgfpathlineto{\pgfqpoint{2.918809in}{1.563318in}}%
\pgfpathlineto{\pgfqpoint{2.918809in}{1.552063in}}%
\pgfpathlineto{\pgfqpoint{2.915674in}{1.548312in}}%
\pgfpathlineto{\pgfqpoint{2.915674in}{1.540809in}}%
\pgfpathlineto{\pgfqpoint{2.912540in}{1.537057in}}%
\pgfpathlineto{\pgfqpoint{2.912540in}{1.529554in}}%
\pgfpathlineto{\pgfqpoint{2.909405in}{1.525803in}}%
\pgfpathlineto{\pgfqpoint{2.909405in}{1.518300in}}%
\pgfpathlineto{\pgfqpoint{2.906270in}{1.514548in}}%
\pgfpathlineto{\pgfqpoint{2.906270in}{1.507045in}}%
\pgfpathlineto{\pgfqpoint{2.903135in}{1.503293in}}%
\pgfpathlineto{\pgfqpoint{2.903135in}{1.495790in}}%
\pgfpathlineto{\pgfqpoint{2.900001in}{1.492039in}}%
\pgfpathlineto{\pgfqpoint{2.900001in}{1.484536in}}%
\pgfpathlineto{\pgfqpoint{2.896866in}{1.480784in}}%
\pgfpathlineto{\pgfqpoint{2.896866in}{1.473281in}}%
\pgfpathlineto{\pgfqpoint{2.893731in}{1.469530in}}%
\pgfpathlineto{\pgfqpoint{2.893731in}{1.462027in}}%
\pgfpathlineto{\pgfqpoint{2.890596in}{1.458275in}}%
\pgfpathlineto{\pgfqpoint{2.890596in}{1.447020in}}%
\pgfpathlineto{\pgfqpoint{2.887462in}{1.443269in}}%
\pgfpathlineto{\pgfqpoint{2.887462in}{1.435766in}}%
\pgfpathlineto{\pgfqpoint{2.884327in}{1.432014in}}%
\pgfpathlineto{\pgfqpoint{2.884327in}{1.424511in}}%
\pgfpathlineto{\pgfqpoint{2.881192in}{1.420760in}}%
\pgfpathlineto{\pgfqpoint{2.881192in}{1.413257in}}%
\pgfpathlineto{\pgfqpoint{2.878057in}{1.409505in}}%
\pgfpathlineto{\pgfqpoint{2.878057in}{1.402002in}}%
\pgfpathlineto{\pgfqpoint{2.874923in}{1.398251in}}%
\pgfpathlineto{\pgfqpoint{2.874923in}{1.390747in}}%
\pgfpathlineto{\pgfqpoint{2.871788in}{1.386996in}}%
\pgfpathlineto{\pgfqpoint{2.871788in}{1.379493in}}%
\pgfpathlineto{\pgfqpoint{2.868653in}{1.375741in}}%
\pgfpathlineto{\pgfqpoint{2.868653in}{1.368238in}}%
\pgfpathlineto{\pgfqpoint{2.865518in}{1.364487in}}%
\pgfpathlineto{\pgfqpoint{2.865518in}{1.353232in}}%
\pgfpathlineto{\pgfqpoint{2.862384in}{1.349481in}}%
\pgfpathlineto{\pgfqpoint{2.862384in}{1.341977in}}%
\pgfpathlineto{\pgfqpoint{2.859249in}{1.338226in}}%
\pgfpathlineto{\pgfqpoint{2.859249in}{1.330723in}}%
\pgfpathlineto{\pgfqpoint{2.856114in}{1.326971in}}%
\pgfpathlineto{\pgfqpoint{2.856114in}{1.319468in}}%
\pgfpathlineto{\pgfqpoint{2.852980in}{1.315717in}}%
\pgfpathlineto{\pgfqpoint{2.852980in}{1.308214in}}%
\pgfpathlineto{\pgfqpoint{2.849845in}{1.304462in}}%
\pgfpathlineto{\pgfqpoint{2.849845in}{1.296959in}}%
\pgfpathlineto{\pgfqpoint{2.846710in}{1.293208in}}%
\pgfpathlineto{\pgfqpoint{2.846710in}{1.285704in}}%
\pgfpathlineto{\pgfqpoint{2.843575in}{1.281953in}}%
\pgfpathlineto{\pgfqpoint{2.843575in}{1.274450in}}%
\pgfpathlineto{\pgfqpoint{2.840441in}{1.270698in}}%
\pgfpathlineto{\pgfqpoint{2.840441in}{1.263195in}}%
\pgfpathlineto{\pgfqpoint{2.837306in}{1.259444in}}%
\pgfpathlineto{\pgfqpoint{2.837306in}{1.248189in}}%
\pgfpathlineto{\pgfqpoint{2.834171in}{1.244438in}}%
\pgfpathlineto{\pgfqpoint{2.834171in}{1.236935in}}%
\pgfpathlineto{\pgfqpoint{2.831036in}{1.233183in}}%
\pgfpathlineto{\pgfqpoint{2.831036in}{1.225680in}}%
\pgfpathlineto{\pgfqpoint{2.827902in}{1.221928in}}%
\pgfpathlineto{\pgfqpoint{2.827902in}{1.214425in}}%
\pgfpathlineto{\pgfqpoint{2.824767in}{1.210674in}}%
\pgfpathlineto{\pgfqpoint{2.824767in}{1.203171in}}%
\pgfpathlineto{\pgfqpoint{2.821632in}{1.199419in}}%
\pgfpathlineto{\pgfqpoint{2.821632in}{1.191916in}}%
\pgfpathlineto{\pgfqpoint{2.818497in}{1.188165in}}%
\pgfpathlineto{\pgfqpoint{2.818497in}{1.180662in}}%
\pgfpathlineto{\pgfqpoint{2.815363in}{1.176910in}}%
\pgfpathlineto{\pgfqpoint{2.815363in}{1.173158in}}%
\pgfpathlineto{\pgfqpoint{2.813482in}{1.170908in}}%
\pgfpathlineto{\pgfqpoint{2.807212in}{1.170908in}}%
\pgfpathlineto{\pgfqpoint{2.804078in}{1.167156in}}%
\pgfpathlineto{\pgfqpoint{2.797808in}{1.167156in}}%
\pgfpathlineto{\pgfqpoint{2.794673in}{1.163404in}}%
\pgfpathlineto{\pgfqpoint{2.791539in}{1.163404in}}%
\pgfpathlineto{\pgfqpoint{2.788404in}{1.159653in}}%
\pgfpathlineto{\pgfqpoint{2.782134in}{1.159653in}}%
\pgfpathlineto{\pgfqpoint{2.779000in}{1.155901in}}%
\pgfpathlineto{\pgfqpoint{2.772730in}{1.155901in}}%
\pgfpathlineto{\pgfqpoint{2.769595in}{1.152150in}}%
\pgfpathlineto{\pgfqpoint{2.763326in}{1.152150in}}%
\pgfpathlineto{\pgfqpoint{2.760191in}{1.148398in}}%
\pgfpathlineto{\pgfqpoint{2.757056in}{1.148398in}}%
\pgfpathlineto{\pgfqpoint{2.753922in}{1.144647in}}%
\pgfpathlineto{\pgfqpoint{2.747652in}{1.144647in}}%
\pgfpathlineto{\pgfqpoint{2.744517in}{1.140895in}}%
\pgfpathlineto{\pgfqpoint{2.738248in}{1.140895in}}%
\pgfpathlineto{\pgfqpoint{2.735113in}{1.137144in}}%
\pgfpathlineto{\pgfqpoint{2.728844in}{1.137144in}}%
\pgfpathlineto{\pgfqpoint{2.725709in}{1.133392in}}%
\pgfpathlineto{\pgfqpoint{2.719439in}{1.133392in}}%
\pgfpathlineto{\pgfqpoint{2.716305in}{1.129641in}}%
\pgfpathlineto{\pgfqpoint{2.713170in}{1.129641in}}%
\pgfpathlineto{\pgfqpoint{2.710035in}{1.125889in}}%
\pgfpathlineto{\pgfqpoint{2.703766in}{1.125889in}}%
\pgfpathlineto{\pgfqpoint{2.700631in}{1.122138in}}%
\pgfpathlineto{\pgfqpoint{2.694362in}{1.122138in}}%
\pgfpathlineto{\pgfqpoint{2.691227in}{1.118386in}}%
\pgfpathlineto{\pgfqpoint{2.684957in}{1.118386in}}%
\pgfpathlineto{\pgfqpoint{2.681823in}{1.114635in}}%
\pgfpathlineto{\pgfqpoint{2.678688in}{1.114635in}}%
\pgfpathlineto{\pgfqpoint{2.675553in}{1.110883in}}%
\pgfpathlineto{\pgfqpoint{2.669284in}{1.110883in}}%
\pgfpathlineto{\pgfqpoint{2.666149in}{1.107131in}}%
\pgfpathlineto{\pgfqpoint{2.659879in}{1.107131in}}%
\pgfpathlineto{\pgfqpoint{2.656745in}{1.103380in}}%
\pgfpathlineto{\pgfqpoint{2.650475in}{1.103380in}}%
\pgfpathlineto{\pgfqpoint{2.647340in}{1.099628in}}%
\pgfpathlineto{\pgfqpoint{2.644206in}{1.099628in}}%
\pgfpathlineto{\pgfqpoint{2.641071in}{1.095877in}}%
\pgfpathlineto{\pgfqpoint{2.634801in}{1.095877in}}%
\pgfpathlineto{\pgfqpoint{2.631667in}{1.092125in}}%
\pgfpathlineto{\pgfqpoint{2.625397in}{1.092125in}}%
\pgfpathlineto{\pgfqpoint{2.622262in}{1.088374in}}%
\pgfpathlineto{\pgfqpoint{2.615993in}{1.088374in}}%
\pgfpathlineto{\pgfqpoint{2.612858in}{1.084622in}}%
\pgfpathlineto{\pgfqpoint{2.606589in}{1.084622in}}%
\pgfpathlineto{\pgfqpoint{2.603454in}{1.080871in}}%
\pgfpathlineto{\pgfqpoint{2.600319in}{1.080871in}}%
\pgfpathlineto{\pgfqpoint{2.597185in}{1.077119in}}%
\pgfpathlineto{\pgfqpoint{2.590915in}{1.077119in}}%
\pgfpathlineto{\pgfqpoint{2.587780in}{1.073368in}}%
\pgfpathlineto{\pgfqpoint{2.581511in}{1.073368in}}%
\pgfpathlineto{\pgfqpoint{2.578376in}{1.069616in}}%
\pgfpathlineto{\pgfqpoint{2.572107in}{1.069616in}}%
\pgfpathlineto{\pgfqpoint{2.568972in}{1.065865in}}%
\pgfpathlineto{\pgfqpoint{2.565837in}{1.065865in}}%
\pgfpathlineto{\pgfqpoint{2.562702in}{1.062113in}}%
\pgfpathlineto{\pgfqpoint{2.556433in}{1.062113in}}%
\pgfpathlineto{\pgfqpoint{2.553298in}{1.058361in}}%
\pgfpathlineto{\pgfqpoint{2.547029in}{1.058361in}}%
\pgfpathlineto{\pgfqpoint{2.543894in}{1.054610in}}%
\pgfpathlineto{\pgfqpoint{2.537624in}{1.054610in}}%
\pgfpathlineto{\pgfqpoint{2.534490in}{1.050858in}}%
\pgfpathlineto{\pgfqpoint{2.531355in}{1.050858in}}%
\pgfpathlineto{\pgfqpoint{2.528220in}{1.047107in}}%
\pgfpathlineto{\pgfqpoint{2.521951in}{1.047107in}}%
\pgfpathlineto{\pgfqpoint{2.518816in}{1.043355in}}%
\pgfpathlineto{\pgfqpoint{2.512547in}{1.043355in}}%
\pgfpathlineto{\pgfqpoint{2.509412in}{1.039604in}}%
\pgfpathlineto{\pgfqpoint{2.503142in}{1.039604in}}%
\pgfpathlineto{\pgfqpoint{2.500008in}{1.035852in}}%
\pgfpathlineto{\pgfqpoint{2.493738in}{1.035852in}}%
\pgfpathlineto{\pgfqpoint{2.490603in}{1.032101in}}%
\pgfpathlineto{\pgfqpoint{2.487469in}{1.032101in}}%
\pgfpathlineto{\pgfqpoint{2.484334in}{1.028349in}}%
\pgfpathlineto{\pgfqpoint{2.478064in}{1.028349in}}%
\pgfpathlineto{\pgfqpoint{2.474930in}{1.024598in}}%
\pgfpathlineto{\pgfqpoint{2.468660in}{1.024598in}}%
\pgfpathlineto{\pgfqpoint{2.465525in}{1.020846in}}%
\pgfpathlineto{\pgfqpoint{2.459256in}{1.020846in}}%
\pgfpathlineto{\pgfqpoint{2.456121in}{1.017095in}}%
\pgfpathlineto{\pgfqpoint{2.452986in}{1.017095in}}%
\pgfpathlineto{\pgfqpoint{2.449852in}{1.013343in}}%
\pgfpathlineto{\pgfqpoint{2.443582in}{1.013343in}}%
\pgfpathlineto{\pgfqpoint{2.440447in}{1.009592in}}%
\pgfpathlineto{\pgfqpoint{2.434178in}{1.009592in}}%
\pgfpathlineto{\pgfqpoint{2.431043in}{1.005840in}}%
\pgfpathlineto{\pgfqpoint{2.424774in}{1.005840in}}%
\pgfpathlineto{\pgfqpoint{2.421639in}{1.002088in}}%
\pgfpathlineto{\pgfqpoint{2.418504in}{1.002088in}}%
\pgfpathlineto{\pgfqpoint{2.415369in}{0.998337in}}%
\pgfpathlineto{\pgfqpoint{2.409100in}{0.998337in}}%
\pgfpathlineto{\pgfqpoint{2.405965in}{0.994585in}}%
\pgfpathlineto{\pgfqpoint{2.399696in}{0.994585in}}%
\pgfpathlineto{\pgfqpoint{2.396561in}{0.990834in}}%
\pgfpathlineto{\pgfqpoint{2.390292in}{0.990834in}}%
\pgfpathlineto{\pgfqpoint{2.387157in}{0.987082in}}%
\pgfpathlineto{\pgfqpoint{2.380887in}{0.987082in}}%
\pgfpathlineto{\pgfqpoint{2.377753in}{0.983331in}}%
\pgfpathlineto{\pgfqpoint{2.374618in}{0.983331in}}%
\pgfpathlineto{\pgfqpoint{2.371483in}{0.979579in}}%
\pgfpathlineto{\pgfqpoint{2.365214in}{0.979579in}}%
\pgfpathlineto{\pgfqpoint{2.362079in}{0.975828in}}%
\pgfpathlineto{\pgfqpoint{2.355809in}{0.975828in}}%
\pgfpathlineto{\pgfqpoint{2.352675in}{0.972076in}}%
\pgfpathlineto{\pgfqpoint{2.346405in}{0.972076in}}%
\pgfpathlineto{\pgfqpoint{2.343270in}{0.968325in}}%
\pgfpathlineto{\pgfqpoint{2.340136in}{0.968325in}}%
\pgfpathlineto{\pgfqpoint{2.337001in}{0.964573in}}%
\pgfpathlineto{\pgfqpoint{2.330731in}{0.964573in}}%
\pgfpathlineto{\pgfqpoint{2.327597in}{0.960822in}}%
\pgfpathlineto{\pgfqpoint{2.321327in}{0.960822in}}%
\pgfpathlineto{\pgfqpoint{2.318192in}{0.957070in}}%
\pgfpathlineto{\pgfqpoint{2.311923in}{0.957070in}}%
\pgfpathlineto{\pgfqpoint{2.308788in}{0.953319in}}%
\pgfpathlineto{\pgfqpoint{2.305654in}{0.953319in}}%
\pgfpathlineto{\pgfqpoint{2.302519in}{0.949567in}}%
\pgfpathlineto{\pgfqpoint{2.296249in}{0.949567in}}%
\pgfpathlineto{\pgfqpoint{2.293115in}{0.945815in}}%
\pgfpathlineto{\pgfqpoint{2.286845in}{0.945815in}}%
\pgfpathlineto{\pgfqpoint{2.283710in}{0.942064in}}%
\pgfpathlineto{\pgfqpoint{2.277441in}{0.942064in}}%
\pgfpathlineto{\pgfqpoint{2.274306in}{0.938312in}}%
\pgfpathlineto{\pgfqpoint{2.271171in}{0.938312in}}%
\pgfpathlineto{\pgfqpoint{2.268037in}{0.934561in}}%
\pgfpathlineto{\pgfqpoint{2.261767in}{0.934561in}}%
\pgfpathlineto{\pgfqpoint{2.258632in}{0.930809in}}%
\pgfpathlineto{\pgfqpoint{2.252363in}{0.930809in}}%
\pgfpathlineto{\pgfqpoint{2.249228in}{0.927058in}}%
\pgfpathlineto{\pgfqpoint{2.242959in}{0.927058in}}%
\pgfpathlineto{\pgfqpoint{2.239824in}{0.923306in}}%
\pgfpathlineto{\pgfqpoint{2.233554in}{0.923306in}}%
\pgfpathlineto{\pgfqpoint{2.230420in}{0.919555in}}%
\pgfpathlineto{\pgfqpoint{2.227285in}{0.919555in}}%
\pgfpathlineto{\pgfqpoint{2.224150in}{0.915803in}}%
\pgfpathlineto{\pgfqpoint{2.217881in}{0.915803in}}%
\pgfpathlineto{\pgfqpoint{2.214746in}{0.912052in}}%
\pgfpathlineto{\pgfqpoint{2.208477in}{0.912052in}}%
\pgfpathlineto{\pgfqpoint{2.205342in}{0.908300in}}%
\pgfpathlineto{\pgfqpoint{2.199072in}{0.908300in}}%
\pgfpathlineto{\pgfqpoint{2.195938in}{0.904549in}}%
\pgfpathlineto{\pgfqpoint{2.192803in}{0.904549in}}%
\pgfpathlineto{\pgfqpoint{2.189668in}{0.900797in}}%
\pgfpathlineto{\pgfqpoint{2.183399in}{0.900797in}}%
\pgfpathlineto{\pgfqpoint{2.180264in}{0.897045in}}%
\pgfpathlineto{\pgfqpoint{2.173994in}{0.897045in}}%
\pgfpathlineto{\pgfqpoint{2.173994in}{0.897045in}}%
\pgfusepath{stroke}%
\end{pgfscope}%
\begin{pgfscope}%
\pgfpathrectangle{\pgfqpoint{0.888750in}{0.419100in}}{\pgfqpoint{2.504659in}{2.933700in}} %
\pgfusepath{clip}%
\pgfsetbuttcap%
\pgfsetroundjoin%
\pgfsetlinewidth{1.505625pt}%
\definecolor{currentstroke}{rgb}{0.250980,0.250980,0.250980}%
\pgfsetstrokecolor{currentstroke}%
\pgfsetdash{}{0pt}%
\pgfpathmoveto{\pgfqpoint{2.173994in}{0.896483in}}%
\pgfpathlineto{\pgfqpoint{2.167725in}{0.903986in}}%
\pgfpathlineto{\pgfqpoint{2.164590in}{0.903986in}}%
\pgfpathlineto{\pgfqpoint{2.152051in}{0.918992in}}%
\pgfpathlineto{\pgfqpoint{2.148916in}{0.918992in}}%
\pgfpathlineto{\pgfqpoint{2.139512in}{0.930247in}}%
\pgfpathlineto{\pgfqpoint{2.136377in}{0.930247in}}%
\pgfpathlineto{\pgfqpoint{2.126973in}{0.941501in}}%
\pgfpathlineto{\pgfqpoint{2.123838in}{0.941501in}}%
\pgfpathlineto{\pgfqpoint{2.114434in}{0.952756in}}%
\pgfpathlineto{\pgfqpoint{2.111299in}{0.952756in}}%
\pgfpathlineto{\pgfqpoint{2.101895in}{0.964010in}}%
\pgfpathlineto{\pgfqpoint{2.098761in}{0.964010in}}%
\pgfpathlineto{\pgfqpoint{2.086222in}{0.979017in}}%
\pgfpathlineto{\pgfqpoint{2.083087in}{0.979017in}}%
\pgfpathlineto{\pgfqpoint{2.073683in}{0.990271in}}%
\pgfpathlineto{\pgfqpoint{2.070548in}{0.990271in}}%
\pgfpathlineto{\pgfqpoint{2.061144in}{1.001526in}}%
\pgfpathlineto{\pgfqpoint{2.058009in}{1.001526in}}%
\pgfpathlineto{\pgfqpoint{2.048605in}{1.012780in}}%
\pgfpathlineto{\pgfqpoint{2.045470in}{1.012780in}}%
\pgfpathlineto{\pgfqpoint{2.036066in}{1.024035in}}%
\pgfpathlineto{\pgfqpoint{2.032931in}{1.024035in}}%
\pgfpathlineto{\pgfqpoint{2.023527in}{1.035290in}}%
\pgfpathlineto{\pgfqpoint{2.020392in}{1.035290in}}%
\pgfpathlineto{\pgfqpoint{2.007853in}{1.050296in}}%
\pgfpathlineto{\pgfqpoint{2.004718in}{1.050296in}}%
\pgfpathlineto{\pgfqpoint{1.995314in}{1.061550in}}%
\pgfpathlineto{\pgfqpoint{1.992179in}{1.061550in}}%
\pgfpathlineto{\pgfqpoint{1.982775in}{1.072805in}}%
\pgfpathlineto{\pgfqpoint{1.979640in}{1.072805in}}%
\pgfpathlineto{\pgfqpoint{1.970236in}{1.084059in}}%
\pgfpathlineto{\pgfqpoint{1.967101in}{1.084059in}}%
\pgfpathlineto{\pgfqpoint{1.957697in}{1.095314in}}%
\pgfpathlineto{\pgfqpoint{1.954562in}{1.095314in}}%
\pgfpathlineto{\pgfqpoint{1.945158in}{1.106569in}}%
\pgfpathlineto{\pgfqpoint{1.942023in}{1.106569in}}%
\pgfpathlineto{\pgfqpoint{1.929484in}{1.121575in}}%
\pgfpathlineto{\pgfqpoint{1.926350in}{1.121575in}}%
\pgfpathlineto{\pgfqpoint{1.916945in}{1.132829in}}%
\pgfpathlineto{\pgfqpoint{1.913811in}{1.132829in}}%
\pgfpathlineto{\pgfqpoint{1.904407in}{1.144084in}}%
\pgfpathlineto{\pgfqpoint{1.901272in}{1.144084in}}%
\pgfpathlineto{\pgfqpoint{1.891868in}{1.155339in}}%
\pgfpathlineto{\pgfqpoint{1.888733in}{1.155339in}}%
\pgfpathlineto{\pgfqpoint{1.879329in}{1.166593in}}%
\pgfpathlineto{\pgfqpoint{1.876194in}{1.166593in}}%
\pgfpathlineto{\pgfqpoint{1.866790in}{1.177848in}}%
\pgfpathlineto{\pgfqpoint{1.863655in}{1.177848in}}%
\pgfpathlineto{\pgfqpoint{1.851116in}{1.192854in}}%
\pgfpathlineto{\pgfqpoint{1.847981in}{1.192854in}}%
\pgfpathlineto{\pgfqpoint{1.838577in}{1.204109in}}%
\pgfpathlineto{\pgfqpoint{1.835442in}{1.204109in}}%
\pgfpathlineto{\pgfqpoint{1.826038in}{1.215363in}}%
\pgfpathlineto{\pgfqpoint{1.822903in}{1.215363in}}%
\pgfpathlineto{\pgfqpoint{1.813499in}{1.226618in}}%
\pgfpathlineto{\pgfqpoint{1.810364in}{1.226618in}}%
\pgfpathlineto{\pgfqpoint{1.800960in}{1.237872in}}%
\pgfpathlineto{\pgfqpoint{1.797825in}{1.237872in}}%
\pgfpathlineto{\pgfqpoint{1.785286in}{1.252879in}}%
\pgfpathlineto{\pgfqpoint{1.782152in}{1.252879in}}%
\pgfpathlineto{\pgfqpoint{1.772747in}{1.264133in}}%
\pgfpathlineto{\pgfqpoint{1.769613in}{1.264133in}}%
\pgfpathlineto{\pgfqpoint{1.760208in}{1.275388in}}%
\pgfpathlineto{\pgfqpoint{1.757074in}{1.275388in}}%
\pgfpathlineto{\pgfqpoint{1.747669in}{1.286642in}}%
\pgfpathlineto{\pgfqpoint{1.744535in}{1.286642in}}%
\pgfpathlineto{\pgfqpoint{1.735130in}{1.297897in}}%
\pgfpathlineto{\pgfqpoint{1.731996in}{1.297897in}}%
\pgfpathlineto{\pgfqpoint{1.722591in}{1.309152in}}%
\pgfpathlineto{\pgfqpoint{1.719457in}{1.309152in}}%
\pgfpathlineto{\pgfqpoint{1.706918in}{1.324158in}}%
\pgfpathlineto{\pgfqpoint{1.703783in}{1.324158in}}%
\pgfpathlineto{\pgfqpoint{1.694379in}{1.335412in}}%
\pgfpathlineto{\pgfqpoint{1.691244in}{1.335412in}}%
\pgfpathlineto{\pgfqpoint{1.681840in}{1.346667in}}%
\pgfpathlineto{\pgfqpoint{1.678705in}{1.346667in}}%
\pgfpathlineto{\pgfqpoint{1.669301in}{1.357922in}}%
\pgfpathlineto{\pgfqpoint{1.666166in}{1.357922in}}%
\pgfpathlineto{\pgfqpoint{1.656762in}{1.369176in}}%
\pgfpathlineto{\pgfqpoint{1.653627in}{1.369176in}}%
\pgfpathlineto{\pgfqpoint{1.644223in}{1.380431in}}%
\pgfpathlineto{\pgfqpoint{1.641088in}{1.380431in}}%
\pgfpathlineto{\pgfqpoint{1.628549in}{1.395437in}}%
\pgfpathlineto{\pgfqpoint{1.625414in}{1.395437in}}%
\pgfpathlineto{\pgfqpoint{1.616010in}{1.406691in}}%
\pgfpathlineto{\pgfqpoint{1.612875in}{1.406691in}}%
\pgfpathlineto{\pgfqpoint{1.603471in}{1.417946in}}%
\pgfpathlineto{\pgfqpoint{1.600337in}{1.417946in}}%
\pgfpathlineto{\pgfqpoint{1.590932in}{1.429201in}}%
\pgfpathlineto{\pgfqpoint{1.587798in}{1.429201in}}%
\pgfpathlineto{\pgfqpoint{1.578393in}{1.440455in}}%
\pgfpathlineto{\pgfqpoint{1.575259in}{1.440455in}}%
\pgfpathlineto{\pgfqpoint{1.565854in}{1.451710in}}%
\pgfpathlineto{\pgfqpoint{1.562720in}{1.451710in}}%
\pgfpathlineto{\pgfqpoint{1.550181in}{1.466716in}}%
\pgfpathlineto{\pgfqpoint{1.547046in}{1.466716in}}%
\pgfpathlineto{\pgfqpoint{1.537642in}{1.477971in}}%
\pgfpathlineto{\pgfqpoint{1.534507in}{1.477971in}}%
\pgfpathlineto{\pgfqpoint{1.525103in}{1.489225in}}%
\pgfpathlineto{\pgfqpoint{1.521968in}{1.489225in}}%
\pgfpathlineto{\pgfqpoint{1.512564in}{1.500480in}}%
\pgfpathlineto{\pgfqpoint{1.509429in}{1.500480in}}%
\pgfpathlineto{\pgfqpoint{1.500025in}{1.511734in}}%
\pgfpathlineto{\pgfqpoint{1.496890in}{1.511734in}}%
\pgfpathlineto{\pgfqpoint{1.484351in}{1.526741in}}%
\pgfpathlineto{\pgfqpoint{1.481216in}{1.526741in}}%
\pgfpathlineto{\pgfqpoint{1.471812in}{1.537995in}}%
\pgfpathlineto{\pgfqpoint{1.468677in}{1.537995in}}%
\pgfpathlineto{\pgfqpoint{1.459273in}{1.549250in}}%
\pgfpathlineto{\pgfqpoint{1.456138in}{1.549250in}}%
\pgfpathlineto{\pgfqpoint{1.446734in}{1.560504in}}%
\pgfpathlineto{\pgfqpoint{1.443599in}{1.560504in}}%
\pgfpathlineto{\pgfqpoint{1.434195in}{1.571759in}}%
\pgfpathlineto{\pgfqpoint{1.431060in}{1.571759in}}%
\pgfpathlineto{\pgfqpoint{1.421656in}{1.583014in}}%
\pgfpathlineto{\pgfqpoint{1.418521in}{1.583014in}}%
\pgfpathlineto{\pgfqpoint{1.405982in}{1.598020in}}%
\pgfpathlineto{\pgfqpoint{1.402848in}{1.598020in}}%
\pgfpathlineto{\pgfqpoint{1.393444in}{1.609274in}}%
\pgfpathlineto{\pgfqpoint{1.390309in}{1.609274in}}%
\pgfpathlineto{\pgfqpoint{1.380905in}{1.620529in}}%
\pgfpathlineto{\pgfqpoint{1.377770in}{1.620529in}}%
\pgfpathlineto{\pgfqpoint{1.368366in}{1.631784in}}%
\pgfpathlineto{\pgfqpoint{1.365231in}{1.631784in}}%
\pgfpathlineto{\pgfqpoint{1.355827in}{1.643038in}}%
\pgfpathlineto{\pgfqpoint{1.352692in}{1.643038in}}%
\pgfpathlineto{\pgfqpoint{1.343288in}{1.654293in}}%
\pgfpathlineto{\pgfqpoint{1.340153in}{1.654293in}}%
\pgfpathlineto{\pgfqpoint{1.327614in}{1.669299in}}%
\pgfpathlineto{\pgfqpoint{1.324479in}{1.669299in}}%
\pgfpathlineto{\pgfqpoint{1.315075in}{1.680553in}}%
\pgfpathlineto{\pgfqpoint{1.311940in}{1.680553in}}%
\pgfpathlineto{\pgfqpoint{1.302536in}{1.691808in}}%
\pgfpathlineto{\pgfqpoint{1.299401in}{1.691808in}}%
\pgfpathlineto{\pgfqpoint{1.290781in}{1.702125in}}%
\pgfpathlineto{\pgfqpoint{1.290781in}{1.705876in}}%
\pgfpathlineto{\pgfqpoint{1.293915in}{1.709628in}}%
\pgfpathlineto{\pgfqpoint{1.293915in}{1.720882in}}%
\pgfpathlineto{\pgfqpoint{1.297050in}{1.724634in}}%
\pgfpathlineto{\pgfqpoint{1.297050in}{1.732137in}}%
\pgfpathlineto{\pgfqpoint{1.300185in}{1.735889in}}%
\pgfpathlineto{\pgfqpoint{1.300185in}{1.747143in}}%
\pgfpathlineto{\pgfqpoint{1.303320in}{1.750895in}}%
\pgfpathlineto{\pgfqpoint{1.303320in}{1.758398in}}%
\pgfpathlineto{\pgfqpoint{1.306454in}{1.762149in}}%
\pgfpathlineto{\pgfqpoint{1.306454in}{1.773404in}}%
\pgfpathlineto{\pgfqpoint{1.309589in}{1.777155in}}%
\pgfpathlineto{\pgfqpoint{1.309589in}{1.784659in}}%
\pgfpathlineto{\pgfqpoint{1.312724in}{1.788410in}}%
\pgfpathlineto{\pgfqpoint{1.312724in}{1.799665in}}%
\pgfpathlineto{\pgfqpoint{1.315859in}{1.803416in}}%
\pgfpathlineto{\pgfqpoint{1.315859in}{1.810919in}}%
\pgfpathlineto{\pgfqpoint{1.318993in}{1.814671in}}%
\pgfpathlineto{\pgfqpoint{1.318993in}{1.822174in}}%
\pgfpathlineto{\pgfqpoint{1.322128in}{1.825925in}}%
\pgfpathlineto{\pgfqpoint{1.322128in}{1.837180in}}%
\pgfpathlineto{\pgfqpoint{1.325263in}{1.840932in}}%
\pgfpathlineto{\pgfqpoint{1.325263in}{1.848435in}}%
\pgfpathlineto{\pgfqpoint{1.328398in}{1.852186in}}%
\pgfpathlineto{\pgfqpoint{1.328398in}{1.863441in}}%
\pgfpathlineto{\pgfqpoint{1.331532in}{1.867192in}}%
\pgfpathlineto{\pgfqpoint{1.331532in}{1.874695in}}%
\pgfpathlineto{\pgfqpoint{1.334667in}{1.878447in}}%
\pgfpathlineto{\pgfqpoint{1.334667in}{1.889702in}}%
\pgfpathlineto{\pgfqpoint{1.337802in}{1.893453in}}%
\pgfpathlineto{\pgfqpoint{1.337802in}{1.900956in}}%
\pgfpathlineto{\pgfqpoint{1.340937in}{1.904708in}}%
\pgfpathlineto{\pgfqpoint{1.340937in}{1.915962in}}%
\pgfpathlineto{\pgfqpoint{1.344071in}{1.919714in}}%
\pgfpathlineto{\pgfqpoint{1.344071in}{1.927217in}}%
\pgfpathlineto{\pgfqpoint{1.347206in}{1.930968in}}%
\pgfpathlineto{\pgfqpoint{1.347206in}{1.942223in}}%
\pgfpathlineto{\pgfqpoint{1.350341in}{1.945975in}}%
\pgfpathlineto{\pgfqpoint{1.350341in}{1.953478in}}%
\pgfpathlineto{\pgfqpoint{1.353476in}{1.957229in}}%
\pgfpathlineto{\pgfqpoint{1.353476in}{1.968484in}}%
\pgfpathlineto{\pgfqpoint{1.356610in}{1.972235in}}%
\pgfpathlineto{\pgfqpoint{1.356610in}{1.979738in}}%
\pgfpathlineto{\pgfqpoint{1.359745in}{1.983490in}}%
\pgfpathlineto{\pgfqpoint{1.359745in}{1.994745in}}%
\pgfpathlineto{\pgfqpoint{1.362880in}{1.998496in}}%
\pgfpathlineto{\pgfqpoint{1.362880in}{2.005999in}}%
\pgfpathlineto{\pgfqpoint{1.366015in}{2.009751in}}%
\pgfpathlineto{\pgfqpoint{1.366015in}{2.021005in}}%
\pgfpathlineto{\pgfqpoint{1.369149in}{2.024757in}}%
\pgfpathlineto{\pgfqpoint{1.369149in}{2.032260in}}%
\pgfpathlineto{\pgfqpoint{1.372284in}{2.036011in}}%
\pgfpathlineto{\pgfqpoint{1.372284in}{2.043514in}}%
\pgfpathlineto{\pgfqpoint{1.375419in}{2.047266in}}%
\pgfpathlineto{\pgfqpoint{1.375419in}{2.058521in}}%
\pgfpathlineto{\pgfqpoint{1.378553in}{2.062272in}}%
\pgfpathlineto{\pgfqpoint{1.378553in}{2.069775in}}%
\pgfpathlineto{\pgfqpoint{1.381688in}{2.073527in}}%
\pgfpathlineto{\pgfqpoint{1.381688in}{2.084781in}}%
\pgfpathlineto{\pgfqpoint{1.384823in}{2.088533in}}%
\pgfpathlineto{\pgfqpoint{1.384823in}{2.096036in}}%
\pgfpathlineto{\pgfqpoint{1.387958in}{2.099787in}}%
\pgfpathlineto{\pgfqpoint{1.387958in}{2.111042in}}%
\pgfpathlineto{\pgfqpoint{1.391092in}{2.114794in}}%
\pgfpathlineto{\pgfqpoint{1.391092in}{2.122297in}}%
\pgfpathlineto{\pgfqpoint{1.394227in}{2.126048in}}%
\pgfpathlineto{\pgfqpoint{1.394227in}{2.137303in}}%
\pgfpathlineto{\pgfqpoint{1.397362in}{2.141054in}}%
\pgfpathlineto{\pgfqpoint{1.397362in}{2.148557in}}%
\pgfpathlineto{\pgfqpoint{1.400497in}{2.152309in}}%
\pgfpathlineto{\pgfqpoint{1.400497in}{2.163564in}}%
\pgfpathlineto{\pgfqpoint{1.403631in}{2.167315in}}%
\pgfpathlineto{\pgfqpoint{1.403631in}{2.174818in}}%
\pgfpathlineto{\pgfqpoint{1.406766in}{2.178570in}}%
\pgfpathlineto{\pgfqpoint{1.406766in}{2.189824in}}%
\pgfpathlineto{\pgfqpoint{1.409901in}{2.193576in}}%
\pgfpathlineto{\pgfqpoint{1.409901in}{2.201079in}}%
\pgfpathlineto{\pgfqpoint{1.413036in}{2.204830in}}%
\pgfpathlineto{\pgfqpoint{1.413036in}{2.216085in}}%
\pgfpathlineto{\pgfqpoint{1.416170in}{2.219837in}}%
\pgfpathlineto{\pgfqpoint{1.416170in}{2.227340in}}%
\pgfpathlineto{\pgfqpoint{1.419305in}{2.231091in}}%
\pgfpathlineto{\pgfqpoint{1.419305in}{2.242346in}}%
\pgfpathlineto{\pgfqpoint{1.422440in}{2.246097in}}%
\pgfpathlineto{\pgfqpoint{1.422440in}{2.253600in}}%
\pgfpathlineto{\pgfqpoint{1.425575in}{2.257352in}}%
\pgfpathlineto{\pgfqpoint{1.425575in}{2.264855in}}%
\pgfpathlineto{\pgfqpoint{1.428709in}{2.268607in}}%
\pgfpathlineto{\pgfqpoint{1.428709in}{2.279861in}}%
\pgfpathlineto{\pgfqpoint{1.431844in}{2.283613in}}%
\pgfpathlineto{\pgfqpoint{1.431844in}{2.291116in}}%
\pgfpathlineto{\pgfqpoint{1.434979in}{2.294867in}}%
\pgfpathlineto{\pgfqpoint{1.434979in}{2.306122in}}%
\pgfpathlineto{\pgfqpoint{1.438114in}{2.309873in}}%
\pgfpathlineto{\pgfqpoint{1.438114in}{2.317376in}}%
\pgfpathlineto{\pgfqpoint{1.441248in}{2.321128in}}%
\pgfpathlineto{\pgfqpoint{1.441248in}{2.332383in}}%
\pgfpathlineto{\pgfqpoint{1.444383in}{2.336134in}}%
\pgfpathlineto{\pgfqpoint{1.444383in}{2.343637in}}%
\pgfpathlineto{\pgfqpoint{1.447518in}{2.347389in}}%
\pgfpathlineto{\pgfqpoint{1.447518in}{2.358643in}}%
\pgfpathlineto{\pgfqpoint{1.450653in}{2.362395in}}%
\pgfpathlineto{\pgfqpoint{1.450653in}{2.369898in}}%
\pgfpathlineto{\pgfqpoint{1.453787in}{2.373649in}}%
\pgfpathlineto{\pgfqpoint{1.453787in}{2.384904in}}%
\pgfpathlineto{\pgfqpoint{1.456922in}{2.388656in}}%
\pgfpathlineto{\pgfqpoint{1.456922in}{2.396159in}}%
\pgfpathlineto{\pgfqpoint{1.460057in}{2.399910in}}%
\pgfpathlineto{\pgfqpoint{1.460057in}{2.403662in}}%
\pgfpathlineto{\pgfqpoint{1.462408in}{2.406475in}}%
\pgfpathlineto{\pgfqpoint{1.466326in}{2.407413in}}%
\pgfpathlineto{\pgfqpoint{1.468677in}{2.410227in}}%
\pgfpathlineto{\pgfqpoint{1.478865in}{2.411165in}}%
\pgfpathlineto{\pgfqpoint{1.481216in}{2.413978in}}%
\pgfpathlineto{\pgfqpoint{1.491404in}{2.414916in}}%
\pgfpathlineto{\pgfqpoint{1.493755in}{2.417730in}}%
\pgfpathlineto{\pgfqpoint{1.503943in}{2.418668in}}%
\pgfpathlineto{\pgfqpoint{1.506294in}{2.421482in}}%
\pgfpathlineto{\pgfqpoint{1.516482in}{2.422419in}}%
\pgfpathlineto{\pgfqpoint{1.518833in}{2.425233in}}%
\pgfpathlineto{\pgfqpoint{1.532156in}{2.426171in}}%
\pgfpathlineto{\pgfqpoint{1.534507in}{2.428985in}}%
\pgfpathlineto{\pgfqpoint{1.544695in}{2.429923in}}%
\pgfpathlineto{\pgfqpoint{1.547046in}{2.432736in}}%
\pgfpathlineto{\pgfqpoint{1.557234in}{2.433674in}}%
\pgfpathlineto{\pgfqpoint{1.559585in}{2.436488in}}%
\pgfpathlineto{\pgfqpoint{1.569773in}{2.437426in}}%
\pgfpathlineto{\pgfqpoint{1.572124in}{2.440239in}}%
\pgfpathlineto{\pgfqpoint{1.585446in}{2.441177in}}%
\pgfpathlineto{\pgfqpoint{1.587798in}{2.443991in}}%
\pgfpathlineto{\pgfqpoint{1.597985in}{2.444929in}}%
\pgfpathlineto{\pgfqpoint{1.600337in}{2.447742in}}%
\pgfpathlineto{\pgfqpoint{1.610524in}{2.448680in}}%
\pgfpathlineto{\pgfqpoint{1.612875in}{2.451494in}}%
\pgfpathlineto{\pgfqpoint{1.623063in}{2.452432in}}%
\pgfpathlineto{\pgfqpoint{1.625414in}{2.455245in}}%
\pgfpathlineto{\pgfqpoint{1.638737in}{2.456183in}}%
\pgfpathlineto{\pgfqpoint{1.641088in}{2.458997in}}%
\pgfpathlineto{\pgfqpoint{1.651276in}{2.459935in}}%
\pgfpathlineto{\pgfqpoint{1.653627in}{2.462748in}}%
\pgfpathlineto{\pgfqpoint{1.663815in}{2.463686in}}%
\pgfpathlineto{\pgfqpoint{1.666166in}{2.466500in}}%
\pgfpathlineto{\pgfqpoint{1.676354in}{2.467438in}}%
\pgfpathlineto{\pgfqpoint{1.678705in}{2.470252in}}%
\pgfpathlineto{\pgfqpoint{1.688893in}{2.471189in}}%
\pgfpathlineto{\pgfqpoint{1.691244in}{2.474003in}}%
\pgfpathlineto{\pgfqpoint{1.704567in}{2.474941in}}%
\pgfpathlineto{\pgfqpoint{1.706918in}{2.477755in}}%
\pgfpathlineto{\pgfqpoint{1.717106in}{2.478692in}}%
\pgfpathlineto{\pgfqpoint{1.719457in}{2.481506in}}%
\pgfpathlineto{\pgfqpoint{1.729645in}{2.482444in}}%
\pgfpathlineto{\pgfqpoint{1.731996in}{2.485258in}}%
\pgfpathlineto{\pgfqpoint{1.742184in}{2.486196in}}%
\pgfpathlineto{\pgfqpoint{1.744535in}{2.489009in}}%
\pgfpathlineto{\pgfqpoint{1.757857in}{2.489947in}}%
\pgfpathlineto{\pgfqpoint{1.760208in}{2.492761in}}%
\pgfpathlineto{\pgfqpoint{1.770396in}{2.493699in}}%
\pgfpathlineto{\pgfqpoint{1.772747in}{2.496512in}}%
\pgfpathlineto{\pgfqpoint{1.782935in}{2.497450in}}%
\pgfpathlineto{\pgfqpoint{1.785286in}{2.500264in}}%
\pgfpathlineto{\pgfqpoint{1.795474in}{2.501202in}}%
\pgfpathlineto{\pgfqpoint{1.797825in}{2.504015in}}%
\pgfpathlineto{\pgfqpoint{1.811148in}{2.504953in}}%
\pgfpathlineto{\pgfqpoint{1.813499in}{2.507767in}}%
\pgfpathlineto{\pgfqpoint{1.823687in}{2.508705in}}%
\pgfpathlineto{\pgfqpoint{1.826038in}{2.511518in}}%
\pgfpathlineto{\pgfqpoint{1.836226in}{2.512456in}}%
\pgfpathlineto{\pgfqpoint{1.838577in}{2.515270in}}%
\pgfpathlineto{\pgfqpoint{1.848765in}{2.516208in}}%
\pgfpathlineto{\pgfqpoint{1.851116in}{2.519021in}}%
\pgfpathlineto{\pgfqpoint{1.861304in}{2.519959in}}%
\pgfpathlineto{\pgfqpoint{1.863655in}{2.522773in}}%
\pgfpathlineto{\pgfqpoint{1.876978in}{2.523711in}}%
\pgfpathlineto{\pgfqpoint{1.879329in}{2.526525in}}%
\pgfpathlineto{\pgfqpoint{1.889516in}{2.527462in}}%
\pgfpathlineto{\pgfqpoint{1.891868in}{2.530276in}}%
\pgfpathlineto{\pgfqpoint{1.902055in}{2.531214in}}%
\pgfpathlineto{\pgfqpoint{1.904407in}{2.534028in}}%
\pgfpathlineto{\pgfqpoint{1.914594in}{2.534965in}}%
\pgfpathlineto{\pgfqpoint{1.916945in}{2.537779in}}%
\pgfpathlineto{\pgfqpoint{1.930268in}{2.538717in}}%
\pgfpathlineto{\pgfqpoint{1.932619in}{2.541531in}}%
\pgfpathlineto{\pgfqpoint{1.942807in}{2.542469in}}%
\pgfpathlineto{\pgfqpoint{1.945158in}{2.545282in}}%
\pgfpathlineto{\pgfqpoint{1.955346in}{2.546220in}}%
\pgfpathlineto{\pgfqpoint{1.957697in}{2.549034in}}%
\pgfpathlineto{\pgfqpoint{1.967885in}{2.549972in}}%
\pgfpathlineto{\pgfqpoint{1.970236in}{2.552785in}}%
\pgfpathlineto{\pgfqpoint{1.980424in}{2.553723in}}%
\pgfpathlineto{\pgfqpoint{1.982775in}{2.556537in}}%
\pgfpathlineto{\pgfqpoint{1.996098in}{2.557475in}}%
\pgfpathlineto{\pgfqpoint{1.998449in}{2.560288in}}%
\pgfpathlineto{\pgfqpoint{2.008637in}{2.561226in}}%
\pgfpathlineto{\pgfqpoint{2.010988in}{2.564040in}}%
\pgfpathlineto{\pgfqpoint{2.021176in}{2.564978in}}%
\pgfpathlineto{\pgfqpoint{2.023527in}{2.567791in}}%
\pgfpathlineto{\pgfqpoint{2.033715in}{2.568729in}}%
\pgfpathlineto{\pgfqpoint{2.036066in}{2.571543in}}%
\pgfpathlineto{\pgfqpoint{2.049388in}{2.572481in}}%
\pgfpathlineto{\pgfqpoint{2.051739in}{2.575294in}}%
\pgfpathlineto{\pgfqpoint{2.061927in}{2.576232in}}%
\pgfpathlineto{\pgfqpoint{2.064278in}{2.579046in}}%
\pgfpathlineto{\pgfqpoint{2.074466in}{2.579984in}}%
\pgfpathlineto{\pgfqpoint{2.076817in}{2.582798in}}%
\pgfpathlineto{\pgfqpoint{2.087005in}{2.583735in}}%
\pgfpathlineto{\pgfqpoint{2.089356in}{2.586549in}}%
\pgfpathlineto{\pgfqpoint{2.102679in}{2.587487in}}%
\pgfpathlineto{\pgfqpoint{2.105030in}{2.590301in}}%
\pgfpathlineto{\pgfqpoint{2.115218in}{2.591238in}}%
\pgfpathlineto{\pgfqpoint{2.117569in}{2.594052in}}%
\pgfpathlineto{\pgfqpoint{2.127757in}{2.594990in}}%
\pgfpathlineto{\pgfqpoint{2.130108in}{2.597804in}}%
\pgfpathlineto{\pgfqpoint{2.140296in}{2.598742in}}%
\pgfpathlineto{\pgfqpoint{2.142647in}{2.601555in}}%
\pgfpathlineto{\pgfqpoint{2.152835in}{2.602493in}}%
\pgfpathlineto{\pgfqpoint{2.155186in}{2.605307in}}%
\pgfpathlineto{\pgfqpoint{2.168509in}{2.606245in}}%
\pgfpathlineto{\pgfqpoint{2.170860in}{2.609058in}}%
\pgfpathlineto{\pgfqpoint{2.181048in}{2.609996in}}%
\pgfpathlineto{\pgfqpoint{2.183399in}{2.612810in}}%
\pgfpathlineto{\pgfqpoint{2.193586in}{2.613748in}}%
\pgfpathlineto{\pgfqpoint{2.195938in}{2.616561in}}%
\pgfpathlineto{\pgfqpoint{2.206125in}{2.617499in}}%
\pgfpathlineto{\pgfqpoint{2.208477in}{2.620313in}}%
\pgfpathlineto{\pgfqpoint{2.221799in}{2.621251in}}%
\pgfpathlineto{\pgfqpoint{2.224150in}{2.624064in}}%
\pgfpathlineto{\pgfqpoint{2.234338in}{2.625002in}}%
\pgfpathlineto{\pgfqpoint{2.236689in}{2.627816in}}%
\pgfpathlineto{\pgfqpoint{2.246877in}{2.628754in}}%
\pgfpathlineto{\pgfqpoint{2.249228in}{2.631567in}}%
\pgfpathlineto{\pgfqpoint{2.259416in}{2.632505in}}%
\pgfpathlineto{\pgfqpoint{2.261767in}{2.635319in}}%
\pgfpathlineto{\pgfqpoint{2.275090in}{2.636257in}}%
\pgfpathlineto{\pgfqpoint{2.277441in}{2.639071in}}%
\pgfpathlineto{\pgfqpoint{2.287629in}{2.640008in}}%
\pgfpathlineto{\pgfqpoint{2.289980in}{2.642822in}}%
\pgfpathlineto{\pgfqpoint{2.300168in}{2.643760in}}%
\pgfpathlineto{\pgfqpoint{2.302519in}{2.646574in}}%
\pgfpathlineto{\pgfqpoint{2.312707in}{2.647512in}}%
\pgfpathlineto{\pgfqpoint{2.315058in}{2.650325in}}%
\pgfpathlineto{\pgfqpoint{2.325246in}{2.651263in}}%
\pgfpathlineto{\pgfqpoint{2.327597in}{2.654077in}}%
\pgfpathlineto{\pgfqpoint{2.340919in}{2.655015in}}%
\pgfpathlineto{\pgfqpoint{2.343270in}{2.657828in}}%
\pgfpathlineto{\pgfqpoint{2.353458in}{2.658766in}}%
\pgfpathlineto{\pgfqpoint{2.355809in}{2.661580in}}%
\pgfpathlineto{\pgfqpoint{2.365997in}{2.662518in}}%
\pgfpathlineto{\pgfqpoint{2.368348in}{2.665331in}}%
\pgfpathlineto{\pgfqpoint{2.378536in}{2.666269in}}%
\pgfpathlineto{\pgfqpoint{2.380887in}{2.669083in}}%
\pgfpathlineto{\pgfqpoint{2.394210in}{2.670021in}}%
\pgfpathlineto{\pgfqpoint{2.396561in}{2.672834in}}%
\pgfpathlineto{\pgfqpoint{2.406749in}{2.673772in}}%
\pgfpathlineto{\pgfqpoint{2.409100in}{2.676586in}}%
\pgfpathlineto{\pgfqpoint{2.419288in}{2.677524in}}%
\pgfpathlineto{\pgfqpoint{2.421639in}{2.680337in}}%
\pgfpathlineto{\pgfqpoint{2.431827in}{2.681275in}}%
\pgfpathlineto{\pgfqpoint{2.434178in}{2.684089in}}%
\pgfpathlineto{\pgfqpoint{2.447501in}{2.685027in}}%
\pgfpathlineto{\pgfqpoint{2.449852in}{2.687841in}}%
\pgfpathlineto{\pgfqpoint{2.460040in}{2.688778in}}%
\pgfpathlineto{\pgfqpoint{2.462391in}{2.691592in}}%
\pgfpathlineto{\pgfqpoint{2.472579in}{2.692530in}}%
\pgfpathlineto{\pgfqpoint{2.474930in}{2.695344in}}%
\pgfpathlineto{\pgfqpoint{2.485118in}{2.696281in}}%
\pgfpathlineto{\pgfqpoint{2.487469in}{2.699095in}}%
\pgfpathlineto{\pgfqpoint{2.497656in}{2.700033in}}%
\pgfpathlineto{\pgfqpoint{2.500008in}{2.702847in}}%
\pgfpathlineto{\pgfqpoint{2.513330in}{2.703785in}}%
\pgfpathlineto{\pgfqpoint{2.515681in}{2.706598in}}%
\pgfpathlineto{\pgfqpoint{2.525869in}{2.707536in}}%
\pgfpathlineto{\pgfqpoint{2.528220in}{2.710350in}}%
\pgfpathlineto{\pgfqpoint{2.538408in}{2.711288in}}%
\pgfpathlineto{\pgfqpoint{2.540759in}{2.714101in}}%
\pgfpathlineto{\pgfqpoint{2.550947in}{2.715039in}}%
\pgfpathlineto{\pgfqpoint{2.553298in}{2.717853in}}%
\pgfpathlineto{\pgfqpoint{2.566621in}{2.718791in}}%
\pgfpathlineto{\pgfqpoint{2.568972in}{2.721604in}}%
\pgfpathlineto{\pgfqpoint{2.579160in}{2.722542in}}%
\pgfpathlineto{\pgfqpoint{2.581511in}{2.725356in}}%
\pgfpathlineto{\pgfqpoint{2.591699in}{2.726294in}}%
\pgfpathlineto{\pgfqpoint{2.594050in}{2.729107in}}%
\pgfpathlineto{\pgfqpoint{2.604238in}{2.730045in}}%
\pgfpathlineto{\pgfqpoint{2.606589in}{2.732859in}}%
\pgfpathlineto{\pgfqpoint{2.616777in}{2.733797in}}%
\pgfpathlineto{\pgfqpoint{2.619128in}{2.736610in}}%
\pgfpathlineto{\pgfqpoint{2.632450in}{2.737548in}}%
\pgfpathlineto{\pgfqpoint{2.634801in}{2.740362in}}%
\pgfpathlineto{\pgfqpoint{2.644989in}{2.741300in}}%
\pgfpathlineto{\pgfqpoint{2.647340in}{2.744114in}}%
\pgfpathlineto{\pgfqpoint{2.657528in}{2.745051in}}%
\pgfpathlineto{\pgfqpoint{2.659879in}{2.747865in}}%
\pgfpathlineto{\pgfqpoint{2.672418in}{2.747865in}}%
\pgfpathlineto{\pgfqpoint{2.712386in}{2.700033in}}%
\pgfpathlineto{\pgfqpoint{2.713170in}{2.695344in}}%
\pgfpathlineto{\pgfqpoint{2.762542in}{2.636257in}}%
\pgfpathlineto{\pgfqpoint{2.763326in}{2.631567in}}%
\pgfpathlineto{\pgfqpoint{2.809563in}{2.576232in}}%
\pgfpathlineto{\pgfqpoint{2.810347in}{2.571543in}}%
\pgfpathlineto{\pgfqpoint{2.859719in}{2.512456in}}%
\pgfpathlineto{\pgfqpoint{2.860503in}{2.507767in}}%
\pgfpathlineto{\pgfqpoint{2.906740in}{2.452432in}}%
\pgfpathlineto{\pgfqpoint{2.907524in}{2.447742in}}%
\pgfpathlineto{\pgfqpoint{2.956896in}{2.388656in}}%
\pgfpathlineto{\pgfqpoint{2.957680in}{2.383966in}}%
\pgfpathlineto{\pgfqpoint{3.003917in}{2.328631in}}%
\pgfpathlineto{\pgfqpoint{3.004701in}{2.323942in}}%
\pgfpathlineto{\pgfqpoint{3.054073in}{2.264855in}}%
\pgfpathlineto{\pgfqpoint{3.054857in}{2.260166in}}%
\pgfpathlineto{\pgfqpoint{3.094825in}{2.212334in}}%
\pgfpathlineto{\pgfqpoint{3.094825in}{2.208582in}}%
\pgfpathlineto{\pgfqpoint{3.091690in}{2.204830in}}%
\pgfpathlineto{\pgfqpoint{3.091690in}{2.197327in}}%
\pgfpathlineto{\pgfqpoint{3.088555in}{2.193576in}}%
\pgfpathlineto{\pgfqpoint{3.088555in}{2.186073in}}%
\pgfpathlineto{\pgfqpoint{3.085421in}{2.182321in}}%
\pgfpathlineto{\pgfqpoint{3.085421in}{2.174818in}}%
\pgfpathlineto{\pgfqpoint{3.082286in}{2.171067in}}%
\pgfpathlineto{\pgfqpoint{3.082286in}{2.163564in}}%
\pgfpathlineto{\pgfqpoint{3.079151in}{2.159812in}}%
\pgfpathlineto{\pgfqpoint{3.079151in}{2.148557in}}%
\pgfpathlineto{\pgfqpoint{3.076016in}{2.144806in}}%
\pgfpathlineto{\pgfqpoint{3.076016in}{2.137303in}}%
\pgfpathlineto{\pgfqpoint{3.072882in}{2.133551in}}%
\pgfpathlineto{\pgfqpoint{3.072882in}{2.126048in}}%
\pgfpathlineto{\pgfqpoint{3.069747in}{2.122297in}}%
\pgfpathlineto{\pgfqpoint{3.069747in}{2.114794in}}%
\pgfpathlineto{\pgfqpoint{3.066612in}{2.111042in}}%
\pgfpathlineto{\pgfqpoint{3.066612in}{2.103539in}}%
\pgfpathlineto{\pgfqpoint{3.063477in}{2.099787in}}%
\pgfpathlineto{\pgfqpoint{3.063477in}{2.092284in}}%
\pgfpathlineto{\pgfqpoint{3.060343in}{2.088533in}}%
\pgfpathlineto{\pgfqpoint{3.060343in}{2.081030in}}%
\pgfpathlineto{\pgfqpoint{3.057208in}{2.077278in}}%
\pgfpathlineto{\pgfqpoint{3.057208in}{2.069775in}}%
\pgfpathlineto{\pgfqpoint{3.054073in}{2.066024in}}%
\pgfpathlineto{\pgfqpoint{3.054073in}{2.058521in}}%
\pgfpathlineto{\pgfqpoint{3.050938in}{2.054769in}}%
\pgfpathlineto{\pgfqpoint{3.050938in}{2.043514in}}%
\pgfpathlineto{\pgfqpoint{3.047804in}{2.039763in}}%
\pgfpathlineto{\pgfqpoint{3.047804in}{2.032260in}}%
\pgfpathlineto{\pgfqpoint{3.044669in}{2.028508in}}%
\pgfpathlineto{\pgfqpoint{3.044669in}{2.021005in}}%
\pgfpathlineto{\pgfqpoint{3.041534in}{2.017254in}}%
\pgfpathlineto{\pgfqpoint{3.041534in}{2.009751in}}%
\pgfpathlineto{\pgfqpoint{3.038400in}{2.005999in}}%
\pgfpathlineto{\pgfqpoint{3.038400in}{1.998496in}}%
\pgfpathlineto{\pgfqpoint{3.035265in}{1.994745in}}%
\pgfpathlineto{\pgfqpoint{3.035265in}{1.987241in}}%
\pgfpathlineto{\pgfqpoint{3.032130in}{1.983490in}}%
\pgfpathlineto{\pgfqpoint{3.032130in}{1.975987in}}%
\pgfpathlineto{\pgfqpoint{3.028995in}{1.972235in}}%
\pgfpathlineto{\pgfqpoint{3.028995in}{1.964732in}}%
\pgfpathlineto{\pgfqpoint{3.025861in}{1.960981in}}%
\pgfpathlineto{\pgfqpoint{3.025861in}{1.949726in}}%
\pgfpathlineto{\pgfqpoint{3.022726in}{1.945975in}}%
\pgfpathlineto{\pgfqpoint{3.022726in}{1.938471in}}%
\pgfpathlineto{\pgfqpoint{3.019591in}{1.934720in}}%
\pgfpathlineto{\pgfqpoint{3.019591in}{1.927217in}}%
\pgfpathlineto{\pgfqpoint{3.016456in}{1.923465in}}%
\pgfpathlineto{\pgfqpoint{3.016456in}{1.915962in}}%
\pgfpathlineto{\pgfqpoint{3.013322in}{1.912211in}}%
\pgfpathlineto{\pgfqpoint{3.013322in}{1.904708in}}%
\pgfpathlineto{\pgfqpoint{3.010187in}{1.900956in}}%
\pgfpathlineto{\pgfqpoint{3.010187in}{1.893453in}}%
\pgfpathlineto{\pgfqpoint{3.007052in}{1.889702in}}%
\pgfpathlineto{\pgfqpoint{3.007052in}{1.882198in}}%
\pgfpathlineto{\pgfqpoint{3.003917in}{1.878447in}}%
\pgfpathlineto{\pgfqpoint{3.003917in}{1.870944in}}%
\pgfpathlineto{\pgfqpoint{3.000783in}{1.867192in}}%
\pgfpathlineto{\pgfqpoint{3.000783in}{1.859689in}}%
\pgfpathlineto{\pgfqpoint{2.997648in}{1.855938in}}%
\pgfpathlineto{\pgfqpoint{2.997648in}{1.844683in}}%
\pgfpathlineto{\pgfqpoint{2.994513in}{1.840932in}}%
\pgfpathlineto{\pgfqpoint{2.994513in}{1.833429in}}%
\pgfpathlineto{\pgfqpoint{2.991378in}{1.829677in}}%
\pgfpathlineto{\pgfqpoint{2.991378in}{1.822174in}}%
\pgfpathlineto{\pgfqpoint{2.988244in}{1.818422in}}%
\pgfpathlineto{\pgfqpoint{2.988244in}{1.810919in}}%
\pgfpathlineto{\pgfqpoint{2.985109in}{1.807168in}}%
\pgfpathlineto{\pgfqpoint{2.985109in}{1.799665in}}%
\pgfpathlineto{\pgfqpoint{2.981974in}{1.795913in}}%
\pgfpathlineto{\pgfqpoint{2.981974in}{1.788410in}}%
\pgfpathlineto{\pgfqpoint{2.978839in}{1.784659in}}%
\pgfpathlineto{\pgfqpoint{2.978839in}{1.777155in}}%
\pgfpathlineto{\pgfqpoint{2.975705in}{1.773404in}}%
\pgfpathlineto{\pgfqpoint{2.975705in}{1.765901in}}%
\pgfpathlineto{\pgfqpoint{2.972570in}{1.762149in}}%
\pgfpathlineto{\pgfqpoint{2.972570in}{1.750895in}}%
\pgfpathlineto{\pgfqpoint{2.969435in}{1.747143in}}%
\pgfpathlineto{\pgfqpoint{2.969435in}{1.739640in}}%
\pgfpathlineto{\pgfqpoint{2.966300in}{1.735889in}}%
\pgfpathlineto{\pgfqpoint{2.966300in}{1.728386in}}%
\pgfpathlineto{\pgfqpoint{2.963166in}{1.724634in}}%
\pgfpathlineto{\pgfqpoint{2.963166in}{1.717131in}}%
\pgfpathlineto{\pgfqpoint{2.960031in}{1.713379in}}%
\pgfpathlineto{\pgfqpoint{2.960031in}{1.705876in}}%
\pgfpathlineto{\pgfqpoint{2.956896in}{1.702125in}}%
\pgfpathlineto{\pgfqpoint{2.956896in}{1.694622in}}%
\pgfpathlineto{\pgfqpoint{2.953761in}{1.690870in}}%
\pgfpathlineto{\pgfqpoint{2.953761in}{1.683367in}}%
\pgfpathlineto{\pgfqpoint{2.950627in}{1.679616in}}%
\pgfpathlineto{\pgfqpoint{2.950627in}{1.672113in}}%
\pgfpathlineto{\pgfqpoint{2.947492in}{1.668361in}}%
\pgfpathlineto{\pgfqpoint{2.947492in}{1.660858in}}%
\pgfpathlineto{\pgfqpoint{2.944357in}{1.657106in}}%
\pgfpathlineto{\pgfqpoint{2.944357in}{1.645852in}}%
\pgfpathlineto{\pgfqpoint{2.941223in}{1.642100in}}%
\pgfpathlineto{\pgfqpoint{2.941223in}{1.634597in}}%
\pgfpathlineto{\pgfqpoint{2.938088in}{1.630846in}}%
\pgfpathlineto{\pgfqpoint{2.938088in}{1.623343in}}%
\pgfpathlineto{\pgfqpoint{2.934953in}{1.619591in}}%
\pgfpathlineto{\pgfqpoint{2.934953in}{1.612088in}}%
\pgfpathlineto{\pgfqpoint{2.931818in}{1.608336in}}%
\pgfpathlineto{\pgfqpoint{2.931818in}{1.600833in}}%
\pgfpathlineto{\pgfqpoint{2.928684in}{1.597082in}}%
\pgfpathlineto{\pgfqpoint{2.928684in}{1.589579in}}%
\pgfpathlineto{\pgfqpoint{2.925549in}{1.585827in}}%
\pgfpathlineto{\pgfqpoint{2.925549in}{1.578324in}}%
\pgfpathlineto{\pgfqpoint{2.922414in}{1.574573in}}%
\pgfpathlineto{\pgfqpoint{2.922414in}{1.567070in}}%
\pgfpathlineto{\pgfqpoint{2.919279in}{1.563318in}}%
\pgfpathlineto{\pgfqpoint{2.919279in}{1.552063in}}%
\pgfpathlineto{\pgfqpoint{2.916145in}{1.548312in}}%
\pgfpathlineto{\pgfqpoint{2.916145in}{1.540809in}}%
\pgfpathlineto{\pgfqpoint{2.913010in}{1.537057in}}%
\pgfpathlineto{\pgfqpoint{2.913010in}{1.529554in}}%
\pgfpathlineto{\pgfqpoint{2.909875in}{1.525803in}}%
\pgfpathlineto{\pgfqpoint{2.909875in}{1.518300in}}%
\pgfpathlineto{\pgfqpoint{2.906740in}{1.514548in}}%
\pgfpathlineto{\pgfqpoint{2.906740in}{1.507045in}}%
\pgfpathlineto{\pgfqpoint{2.903606in}{1.503293in}}%
\pgfpathlineto{\pgfqpoint{2.903606in}{1.495790in}}%
\pgfpathlineto{\pgfqpoint{2.900471in}{1.492039in}}%
\pgfpathlineto{\pgfqpoint{2.900471in}{1.484536in}}%
\pgfpathlineto{\pgfqpoint{2.897336in}{1.480784in}}%
\pgfpathlineto{\pgfqpoint{2.897336in}{1.473281in}}%
\pgfpathlineto{\pgfqpoint{2.894201in}{1.469530in}}%
\pgfpathlineto{\pgfqpoint{2.894201in}{1.462027in}}%
\pgfpathlineto{\pgfqpoint{2.891067in}{1.458275in}}%
\pgfpathlineto{\pgfqpoint{2.891067in}{1.447020in}}%
\pgfpathlineto{\pgfqpoint{2.887932in}{1.443269in}}%
\pgfpathlineto{\pgfqpoint{2.887932in}{1.435766in}}%
\pgfpathlineto{\pgfqpoint{2.884797in}{1.432014in}}%
\pgfpathlineto{\pgfqpoint{2.884797in}{1.424511in}}%
\pgfpathlineto{\pgfqpoint{2.881662in}{1.420760in}}%
\pgfpathlineto{\pgfqpoint{2.881662in}{1.413257in}}%
\pgfpathlineto{\pgfqpoint{2.878528in}{1.409505in}}%
\pgfpathlineto{\pgfqpoint{2.878528in}{1.402002in}}%
\pgfpathlineto{\pgfqpoint{2.875393in}{1.398251in}}%
\pgfpathlineto{\pgfqpoint{2.875393in}{1.390747in}}%
\pgfpathlineto{\pgfqpoint{2.872258in}{1.386996in}}%
\pgfpathlineto{\pgfqpoint{2.872258in}{1.379493in}}%
\pgfpathlineto{\pgfqpoint{2.869123in}{1.375741in}}%
\pgfpathlineto{\pgfqpoint{2.869123in}{1.368238in}}%
\pgfpathlineto{\pgfqpoint{2.865989in}{1.364487in}}%
\pgfpathlineto{\pgfqpoint{2.865989in}{1.353232in}}%
\pgfpathlineto{\pgfqpoint{2.862854in}{1.349481in}}%
\pgfpathlineto{\pgfqpoint{2.862854in}{1.341977in}}%
\pgfpathlineto{\pgfqpoint{2.859719in}{1.338226in}}%
\pgfpathlineto{\pgfqpoint{2.859719in}{1.330723in}}%
\pgfpathlineto{\pgfqpoint{2.856584in}{1.326971in}}%
\pgfpathlineto{\pgfqpoint{2.856584in}{1.319468in}}%
\pgfpathlineto{\pgfqpoint{2.853450in}{1.315717in}}%
\pgfpathlineto{\pgfqpoint{2.853450in}{1.308214in}}%
\pgfpathlineto{\pgfqpoint{2.850315in}{1.304462in}}%
\pgfpathlineto{\pgfqpoint{2.850315in}{1.296959in}}%
\pgfpathlineto{\pgfqpoint{2.847180in}{1.293208in}}%
\pgfpathlineto{\pgfqpoint{2.847180in}{1.285704in}}%
\pgfpathlineto{\pgfqpoint{2.844046in}{1.281953in}}%
\pgfpathlineto{\pgfqpoint{2.844046in}{1.274450in}}%
\pgfpathlineto{\pgfqpoint{2.840911in}{1.270698in}}%
\pgfpathlineto{\pgfqpoint{2.840911in}{1.263195in}}%
\pgfpathlineto{\pgfqpoint{2.837776in}{1.259444in}}%
\pgfpathlineto{\pgfqpoint{2.837776in}{1.248189in}}%
\pgfpathlineto{\pgfqpoint{2.834641in}{1.244438in}}%
\pgfpathlineto{\pgfqpoint{2.834641in}{1.236935in}}%
\pgfpathlineto{\pgfqpoint{2.831507in}{1.233183in}}%
\pgfpathlineto{\pgfqpoint{2.831507in}{1.225680in}}%
\pgfpathlineto{\pgfqpoint{2.828372in}{1.221928in}}%
\pgfpathlineto{\pgfqpoint{2.828372in}{1.214425in}}%
\pgfpathlineto{\pgfqpoint{2.825237in}{1.210674in}}%
\pgfpathlineto{\pgfqpoint{2.825237in}{1.203171in}}%
\pgfpathlineto{\pgfqpoint{2.822102in}{1.199419in}}%
\pgfpathlineto{\pgfqpoint{2.822102in}{1.191916in}}%
\pgfpathlineto{\pgfqpoint{2.818968in}{1.188165in}}%
\pgfpathlineto{\pgfqpoint{2.818968in}{1.180662in}}%
\pgfpathlineto{\pgfqpoint{2.815833in}{1.176910in}}%
\pgfpathlineto{\pgfqpoint{2.815833in}{1.173158in}}%
\pgfpathlineto{\pgfqpoint{2.813482in}{1.170345in}}%
\pgfpathlineto{\pgfqpoint{2.806429in}{1.169407in}}%
\pgfpathlineto{\pgfqpoint{2.804078in}{1.166593in}}%
\pgfpathlineto{\pgfqpoint{2.797024in}{1.165655in}}%
\pgfpathlineto{\pgfqpoint{2.794673in}{1.162842in}}%
\pgfpathlineto{\pgfqpoint{2.790755in}{1.161904in}}%
\pgfpathlineto{\pgfqpoint{2.788404in}{1.159090in}}%
\pgfpathlineto{\pgfqpoint{2.781351in}{1.158152in}}%
\pgfpathlineto{\pgfqpoint{2.779000in}{1.155339in}}%
\pgfpathlineto{\pgfqpoint{2.771946in}{1.154401in}}%
\pgfpathlineto{\pgfqpoint{2.769595in}{1.151587in}}%
\pgfpathlineto{\pgfqpoint{2.762542in}{1.150649in}}%
\pgfpathlineto{\pgfqpoint{2.760191in}{1.147836in}}%
\pgfpathlineto{\pgfqpoint{2.756273in}{1.146898in}}%
\pgfpathlineto{\pgfqpoint{2.753922in}{1.144084in}}%
\pgfpathlineto{\pgfqpoint{2.746868in}{1.143146in}}%
\pgfpathlineto{\pgfqpoint{2.744517in}{1.140333in}}%
\pgfpathlineto{\pgfqpoint{2.737464in}{1.139395in}}%
\pgfpathlineto{\pgfqpoint{2.735113in}{1.136581in}}%
\pgfpathlineto{\pgfqpoint{2.728060in}{1.135643in}}%
\pgfpathlineto{\pgfqpoint{2.725709in}{1.132829in}}%
\pgfpathlineto{\pgfqpoint{2.718656in}{1.131892in}}%
\pgfpathlineto{\pgfqpoint{2.716305in}{1.129078in}}%
\pgfpathlineto{\pgfqpoint{2.712386in}{1.128140in}}%
\pgfpathlineto{\pgfqpoint{2.710035in}{1.125326in}}%
\pgfpathlineto{\pgfqpoint{2.702982in}{1.124388in}}%
\pgfpathlineto{\pgfqpoint{2.700631in}{1.121575in}}%
\pgfpathlineto{\pgfqpoint{2.693578in}{1.120637in}}%
\pgfpathlineto{\pgfqpoint{2.691227in}{1.117823in}}%
\pgfpathlineto{\pgfqpoint{2.684174in}{1.116885in}}%
\pgfpathlineto{\pgfqpoint{2.681823in}{1.114072in}}%
\pgfpathlineto{\pgfqpoint{2.677904in}{1.113134in}}%
\pgfpathlineto{\pgfqpoint{2.675553in}{1.110320in}}%
\pgfpathlineto{\pgfqpoint{2.668500in}{1.109382in}}%
\pgfpathlineto{\pgfqpoint{2.666149in}{1.106569in}}%
\pgfpathlineto{\pgfqpoint{2.659096in}{1.105631in}}%
\pgfpathlineto{\pgfqpoint{2.656745in}{1.102817in}}%
\pgfpathlineto{\pgfqpoint{2.649691in}{1.101879in}}%
\pgfpathlineto{\pgfqpoint{2.647340in}{1.099066in}}%
\pgfpathlineto{\pgfqpoint{2.643422in}{1.098128in}}%
\pgfpathlineto{\pgfqpoint{2.641071in}{1.095314in}}%
\pgfpathlineto{\pgfqpoint{2.634018in}{1.094376in}}%
\pgfpathlineto{\pgfqpoint{2.631667in}{1.091563in}}%
\pgfpathlineto{\pgfqpoint{2.624614in}{1.090625in}}%
\pgfpathlineto{\pgfqpoint{2.622262in}{1.087811in}}%
\pgfpathlineto{\pgfqpoint{2.615209in}{1.086873in}}%
\pgfpathlineto{\pgfqpoint{2.612858in}{1.084059in}}%
\pgfpathlineto{\pgfqpoint{2.605805in}{1.083122in}}%
\pgfpathlineto{\pgfqpoint{2.603454in}{1.080308in}}%
\pgfpathlineto{\pgfqpoint{2.599536in}{1.079370in}}%
\pgfpathlineto{\pgfqpoint{2.597185in}{1.076556in}}%
\pgfpathlineto{\pgfqpoint{2.590131in}{1.075619in}}%
\pgfpathlineto{\pgfqpoint{2.587780in}{1.072805in}}%
\pgfpathlineto{\pgfqpoint{2.580727in}{1.071867in}}%
\pgfpathlineto{\pgfqpoint{2.578376in}{1.069053in}}%
\pgfpathlineto{\pgfqpoint{2.571323in}{1.068115in}}%
\pgfpathlineto{\pgfqpoint{2.568972in}{1.065302in}}%
\pgfpathlineto{\pgfqpoint{2.565053in}{1.064364in}}%
\pgfpathlineto{\pgfqpoint{2.562702in}{1.061550in}}%
\pgfpathlineto{\pgfqpoint{2.555649in}{1.060612in}}%
\pgfpathlineto{\pgfqpoint{2.553298in}{1.057799in}}%
\pgfpathlineto{\pgfqpoint{2.546245in}{1.056861in}}%
\pgfpathlineto{\pgfqpoint{2.543894in}{1.054047in}}%
\pgfpathlineto{\pgfqpoint{2.536841in}{1.053109in}}%
\pgfpathlineto{\pgfqpoint{2.534490in}{1.050296in}}%
\pgfpathlineto{\pgfqpoint{2.530571in}{1.049358in}}%
\pgfpathlineto{\pgfqpoint{2.528220in}{1.046544in}}%
\pgfpathlineto{\pgfqpoint{2.521167in}{1.045606in}}%
\pgfpathlineto{\pgfqpoint{2.518816in}{1.042793in}}%
\pgfpathlineto{\pgfqpoint{2.511763in}{1.041855in}}%
\pgfpathlineto{\pgfqpoint{2.509412in}{1.039041in}}%
\pgfpathlineto{\pgfqpoint{2.502359in}{1.038103in}}%
\pgfpathlineto{\pgfqpoint{2.500008in}{1.035290in}}%
\pgfpathlineto{\pgfqpoint{2.492954in}{1.034352in}}%
\pgfpathlineto{\pgfqpoint{2.490603in}{1.031538in}}%
\pgfpathlineto{\pgfqpoint{2.486685in}{1.030600in}}%
\pgfpathlineto{\pgfqpoint{2.484334in}{1.027786in}}%
\pgfpathlineto{\pgfqpoint{2.477281in}{1.026849in}}%
\pgfpathlineto{\pgfqpoint{2.474930in}{1.024035in}}%
\pgfpathlineto{\pgfqpoint{2.467876in}{1.023097in}}%
\pgfpathlineto{\pgfqpoint{2.465525in}{1.020283in}}%
\pgfpathlineto{\pgfqpoint{2.458472in}{1.019346in}}%
\pgfpathlineto{\pgfqpoint{2.456121in}{1.016532in}}%
\pgfpathlineto{\pgfqpoint{2.452203in}{1.015594in}}%
\pgfpathlineto{\pgfqpoint{2.449852in}{1.012780in}}%
\pgfpathlineto{\pgfqpoint{2.442798in}{1.011842in}}%
\pgfpathlineto{\pgfqpoint{2.440447in}{1.009029in}}%
\pgfpathlineto{\pgfqpoint{2.433394in}{1.008091in}}%
\pgfpathlineto{\pgfqpoint{2.431043in}{1.005277in}}%
\pgfpathlineto{\pgfqpoint{2.423990in}{1.004339in}}%
\pgfpathlineto{\pgfqpoint{2.421639in}{1.001526in}}%
\pgfpathlineto{\pgfqpoint{2.417721in}{1.000588in}}%
\pgfpathlineto{\pgfqpoint{2.415369in}{0.997774in}}%
\pgfpathlineto{\pgfqpoint{2.408316in}{0.996836in}}%
\pgfpathlineto{\pgfqpoint{2.405965in}{0.994023in}}%
\pgfpathlineto{\pgfqpoint{2.398912in}{0.993085in}}%
\pgfpathlineto{\pgfqpoint{2.396561in}{0.990271in}}%
\pgfpathlineto{\pgfqpoint{2.389508in}{0.989333in}}%
\pgfpathlineto{\pgfqpoint{2.387157in}{0.986520in}}%
\pgfpathlineto{\pgfqpoint{2.380104in}{0.985582in}}%
\pgfpathlineto{\pgfqpoint{2.377753in}{0.982768in}}%
\pgfpathlineto{\pgfqpoint{2.373834in}{0.981830in}}%
\pgfpathlineto{\pgfqpoint{2.371483in}{0.979017in}}%
\pgfpathlineto{\pgfqpoint{2.364430in}{0.978079in}}%
\pgfpathlineto{\pgfqpoint{2.362079in}{0.975265in}}%
\pgfpathlineto{\pgfqpoint{2.355026in}{0.974327in}}%
\pgfpathlineto{\pgfqpoint{2.352675in}{0.971513in}}%
\pgfpathlineto{\pgfqpoint{2.345621in}{0.970576in}}%
\pgfpathlineto{\pgfqpoint{2.343270in}{0.967762in}}%
\pgfpathlineto{\pgfqpoint{2.339352in}{0.966824in}}%
\pgfpathlineto{\pgfqpoint{2.337001in}{0.964010in}}%
\pgfpathlineto{\pgfqpoint{2.329948in}{0.963073in}}%
\pgfpathlineto{\pgfqpoint{2.327597in}{0.960259in}}%
\pgfpathlineto{\pgfqpoint{2.320544in}{0.959321in}}%
\pgfpathlineto{\pgfqpoint{2.318192in}{0.956507in}}%
\pgfpathlineto{\pgfqpoint{2.311139in}{0.955569in}}%
\pgfpathlineto{\pgfqpoint{2.308788in}{0.952756in}}%
\pgfpathlineto{\pgfqpoint{2.304870in}{0.951818in}}%
\pgfpathlineto{\pgfqpoint{2.302519in}{0.949004in}}%
\pgfpathlineto{\pgfqpoint{2.295466in}{0.948066in}}%
\pgfpathlineto{\pgfqpoint{2.293115in}{0.945253in}}%
\pgfpathlineto{\pgfqpoint{2.286061in}{0.944315in}}%
\pgfpathlineto{\pgfqpoint{2.283710in}{0.941501in}}%
\pgfpathlineto{\pgfqpoint{2.276657in}{0.940563in}}%
\pgfpathlineto{\pgfqpoint{2.274306in}{0.937750in}}%
\pgfpathlineto{\pgfqpoint{2.270388in}{0.936812in}}%
\pgfpathlineto{\pgfqpoint{2.268037in}{0.933998in}}%
\pgfpathlineto{\pgfqpoint{2.260983in}{0.933060in}}%
\pgfpathlineto{\pgfqpoint{2.258632in}{0.930247in}}%
\pgfpathlineto{\pgfqpoint{2.251579in}{0.929309in}}%
\pgfpathlineto{\pgfqpoint{2.249228in}{0.926495in}}%
\pgfpathlineto{\pgfqpoint{2.242175in}{0.925557in}}%
\pgfpathlineto{\pgfqpoint{2.239824in}{0.922744in}}%
\pgfpathlineto{\pgfqpoint{2.232771in}{0.921806in}}%
\pgfpathlineto{\pgfqpoint{2.230420in}{0.918992in}}%
\pgfpathlineto{\pgfqpoint{2.226501in}{0.918054in}}%
\pgfpathlineto{\pgfqpoint{2.224150in}{0.915240in}}%
\pgfpathlineto{\pgfqpoint{2.217097in}{0.914303in}}%
\pgfpathlineto{\pgfqpoint{2.214746in}{0.911489in}}%
\pgfpathlineto{\pgfqpoint{2.207693in}{0.910551in}}%
\pgfpathlineto{\pgfqpoint{2.205342in}{0.907737in}}%
\pgfpathlineto{\pgfqpoint{2.198289in}{0.906799in}}%
\pgfpathlineto{\pgfqpoint{2.195938in}{0.903986in}}%
\pgfpathlineto{\pgfqpoint{2.192019in}{0.903048in}}%
\pgfpathlineto{\pgfqpoint{2.189668in}{0.900234in}}%
\pgfpathlineto{\pgfqpoint{2.182615in}{0.899296in}}%
\pgfpathlineto{\pgfqpoint{2.180264in}{0.896483in}}%
\pgfpathlineto{\pgfqpoint{2.173994in}{0.896483in}}%
\pgfpathlineto{\pgfqpoint{2.173994in}{0.896483in}}%
\pgfusepath{stroke}%
\end{pgfscope}%
\begin{pgfscope}%
\pgfpathrectangle{\pgfqpoint{0.888750in}{0.419100in}}{\pgfqpoint{2.504659in}{2.933700in}} %
\pgfusepath{clip}%
\pgfsetbuttcap%
\pgfsetroundjoin%
\pgfsetlinewidth{1.505625pt}%
\definecolor{currentstroke}{rgb}{0.000000,0.000000,0.000000}%
\pgfsetstrokecolor{currentstroke}%
\pgfsetdash{}{0pt}%
\pgfpathmoveto{\pgfqpoint{2.173994in}{0.895920in}}%
\pgfpathlineto{\pgfqpoint{2.167725in}{0.903423in}}%
\pgfpathlineto{\pgfqpoint{2.164590in}{0.903423in}}%
\pgfpathlineto{\pgfqpoint{2.152051in}{0.918429in}}%
\pgfpathlineto{\pgfqpoint{2.148916in}{0.918429in}}%
\pgfpathlineto{\pgfqpoint{2.139512in}{0.929684in}}%
\pgfpathlineto{\pgfqpoint{2.136377in}{0.929684in}}%
\pgfpathlineto{\pgfqpoint{2.126973in}{0.940938in}}%
\pgfpathlineto{\pgfqpoint{2.123838in}{0.940938in}}%
\pgfpathlineto{\pgfqpoint{2.114434in}{0.952193in}}%
\pgfpathlineto{\pgfqpoint{2.111299in}{0.952193in}}%
\pgfpathlineto{\pgfqpoint{2.101895in}{0.963448in}}%
\pgfpathlineto{\pgfqpoint{2.098761in}{0.963448in}}%
\pgfpathlineto{\pgfqpoint{2.086222in}{0.978454in}}%
\pgfpathlineto{\pgfqpoint{2.083087in}{0.978454in}}%
\pgfpathlineto{\pgfqpoint{2.073683in}{0.989708in}}%
\pgfpathlineto{\pgfqpoint{2.070548in}{0.989708in}}%
\pgfpathlineto{\pgfqpoint{2.061144in}{1.000963in}}%
\pgfpathlineto{\pgfqpoint{2.058009in}{1.000963in}}%
\pgfpathlineto{\pgfqpoint{2.048605in}{1.012218in}}%
\pgfpathlineto{\pgfqpoint{2.045470in}{1.012218in}}%
\pgfpathlineto{\pgfqpoint{2.036066in}{1.023472in}}%
\pgfpathlineto{\pgfqpoint{2.032931in}{1.023472in}}%
\pgfpathlineto{\pgfqpoint{2.023527in}{1.034727in}}%
\pgfpathlineto{\pgfqpoint{2.020392in}{1.034727in}}%
\pgfpathlineto{\pgfqpoint{2.007853in}{1.049733in}}%
\pgfpathlineto{\pgfqpoint{2.004718in}{1.049733in}}%
\pgfpathlineto{\pgfqpoint{1.995314in}{1.060988in}}%
\pgfpathlineto{\pgfqpoint{1.992179in}{1.060988in}}%
\pgfpathlineto{\pgfqpoint{1.982775in}{1.072242in}}%
\pgfpathlineto{\pgfqpoint{1.979640in}{1.072242in}}%
\pgfpathlineto{\pgfqpoint{1.970236in}{1.083497in}}%
\pgfpathlineto{\pgfqpoint{1.967101in}{1.083497in}}%
\pgfpathlineto{\pgfqpoint{1.957697in}{1.094751in}}%
\pgfpathlineto{\pgfqpoint{1.954562in}{1.094751in}}%
\pgfpathlineto{\pgfqpoint{1.945158in}{1.106006in}}%
\pgfpathlineto{\pgfqpoint{1.942023in}{1.106006in}}%
\pgfpathlineto{\pgfqpoint{1.929484in}{1.121012in}}%
\pgfpathlineto{\pgfqpoint{1.926350in}{1.121012in}}%
\pgfpathlineto{\pgfqpoint{1.916945in}{1.132267in}}%
\pgfpathlineto{\pgfqpoint{1.913811in}{1.132267in}}%
\pgfpathlineto{\pgfqpoint{1.904407in}{1.143521in}}%
\pgfpathlineto{\pgfqpoint{1.901272in}{1.143521in}}%
\pgfpathlineto{\pgfqpoint{1.891868in}{1.154776in}}%
\pgfpathlineto{\pgfqpoint{1.888733in}{1.154776in}}%
\pgfpathlineto{\pgfqpoint{1.879329in}{1.166031in}}%
\pgfpathlineto{\pgfqpoint{1.876194in}{1.166031in}}%
\pgfpathlineto{\pgfqpoint{1.866790in}{1.177285in}}%
\pgfpathlineto{\pgfqpoint{1.863655in}{1.177285in}}%
\pgfpathlineto{\pgfqpoint{1.851116in}{1.192291in}}%
\pgfpathlineto{\pgfqpoint{1.847981in}{1.192291in}}%
\pgfpathlineto{\pgfqpoint{1.838577in}{1.203546in}}%
\pgfpathlineto{\pgfqpoint{1.835442in}{1.203546in}}%
\pgfpathlineto{\pgfqpoint{1.826038in}{1.214800in}}%
\pgfpathlineto{\pgfqpoint{1.822903in}{1.214800in}}%
\pgfpathlineto{\pgfqpoint{1.813499in}{1.226055in}}%
\pgfpathlineto{\pgfqpoint{1.810364in}{1.226055in}}%
\pgfpathlineto{\pgfqpoint{1.800960in}{1.237310in}}%
\pgfpathlineto{\pgfqpoint{1.797825in}{1.237310in}}%
\pgfpathlineto{\pgfqpoint{1.785286in}{1.252316in}}%
\pgfpathlineto{\pgfqpoint{1.782152in}{1.252316in}}%
\pgfpathlineto{\pgfqpoint{1.772747in}{1.263570in}}%
\pgfpathlineto{\pgfqpoint{1.769613in}{1.263570in}}%
\pgfpathlineto{\pgfqpoint{1.760208in}{1.274825in}}%
\pgfpathlineto{\pgfqpoint{1.757074in}{1.274825in}}%
\pgfpathlineto{\pgfqpoint{1.747669in}{1.286080in}}%
\pgfpathlineto{\pgfqpoint{1.744535in}{1.286080in}}%
\pgfpathlineto{\pgfqpoint{1.735130in}{1.297334in}}%
\pgfpathlineto{\pgfqpoint{1.731996in}{1.297334in}}%
\pgfpathlineto{\pgfqpoint{1.722591in}{1.308589in}}%
\pgfpathlineto{\pgfqpoint{1.719457in}{1.308589in}}%
\pgfpathlineto{\pgfqpoint{1.706918in}{1.323595in}}%
\pgfpathlineto{\pgfqpoint{1.703783in}{1.323595in}}%
\pgfpathlineto{\pgfqpoint{1.694379in}{1.334850in}}%
\pgfpathlineto{\pgfqpoint{1.691244in}{1.334850in}}%
\pgfpathlineto{\pgfqpoint{1.681840in}{1.346104in}}%
\pgfpathlineto{\pgfqpoint{1.678705in}{1.346104in}}%
\pgfpathlineto{\pgfqpoint{1.669301in}{1.357359in}}%
\pgfpathlineto{\pgfqpoint{1.666166in}{1.357359in}}%
\pgfpathlineto{\pgfqpoint{1.656762in}{1.368613in}}%
\pgfpathlineto{\pgfqpoint{1.653627in}{1.368613in}}%
\pgfpathlineto{\pgfqpoint{1.644223in}{1.379868in}}%
\pgfpathlineto{\pgfqpoint{1.641088in}{1.379868in}}%
\pgfpathlineto{\pgfqpoint{1.628549in}{1.394874in}}%
\pgfpathlineto{\pgfqpoint{1.625414in}{1.394874in}}%
\pgfpathlineto{\pgfqpoint{1.616010in}{1.406129in}}%
\pgfpathlineto{\pgfqpoint{1.612875in}{1.406129in}}%
\pgfpathlineto{\pgfqpoint{1.603471in}{1.417383in}}%
\pgfpathlineto{\pgfqpoint{1.600337in}{1.417383in}}%
\pgfpathlineto{\pgfqpoint{1.590932in}{1.428638in}}%
\pgfpathlineto{\pgfqpoint{1.587798in}{1.428638in}}%
\pgfpathlineto{\pgfqpoint{1.578393in}{1.439893in}}%
\pgfpathlineto{\pgfqpoint{1.575259in}{1.439893in}}%
\pgfpathlineto{\pgfqpoint{1.565854in}{1.451147in}}%
\pgfpathlineto{\pgfqpoint{1.562720in}{1.451147in}}%
\pgfpathlineto{\pgfqpoint{1.550181in}{1.466153in}}%
\pgfpathlineto{\pgfqpoint{1.547046in}{1.466153in}}%
\pgfpathlineto{\pgfqpoint{1.537642in}{1.477408in}}%
\pgfpathlineto{\pgfqpoint{1.534507in}{1.477408in}}%
\pgfpathlineto{\pgfqpoint{1.525103in}{1.488662in}}%
\pgfpathlineto{\pgfqpoint{1.521968in}{1.488662in}}%
\pgfpathlineto{\pgfqpoint{1.512564in}{1.499917in}}%
\pgfpathlineto{\pgfqpoint{1.509429in}{1.499917in}}%
\pgfpathlineto{\pgfqpoint{1.500025in}{1.511172in}}%
\pgfpathlineto{\pgfqpoint{1.496890in}{1.511172in}}%
\pgfpathlineto{\pgfqpoint{1.484351in}{1.526178in}}%
\pgfpathlineto{\pgfqpoint{1.481216in}{1.526178in}}%
\pgfpathlineto{\pgfqpoint{1.471812in}{1.537432in}}%
\pgfpathlineto{\pgfqpoint{1.468677in}{1.537432in}}%
\pgfpathlineto{\pgfqpoint{1.459273in}{1.548687in}}%
\pgfpathlineto{\pgfqpoint{1.456138in}{1.548687in}}%
\pgfpathlineto{\pgfqpoint{1.446734in}{1.559942in}}%
\pgfpathlineto{\pgfqpoint{1.443599in}{1.559942in}}%
\pgfpathlineto{\pgfqpoint{1.434195in}{1.571196in}}%
\pgfpathlineto{\pgfqpoint{1.431060in}{1.571196in}}%
\pgfpathlineto{\pgfqpoint{1.421656in}{1.582451in}}%
\pgfpathlineto{\pgfqpoint{1.418521in}{1.582451in}}%
\pgfpathlineto{\pgfqpoint{1.405982in}{1.597457in}}%
\pgfpathlineto{\pgfqpoint{1.402848in}{1.597457in}}%
\pgfpathlineto{\pgfqpoint{1.393444in}{1.608712in}}%
\pgfpathlineto{\pgfqpoint{1.390309in}{1.608712in}}%
\pgfpathlineto{\pgfqpoint{1.380905in}{1.619966in}}%
\pgfpathlineto{\pgfqpoint{1.377770in}{1.619966in}}%
\pgfpathlineto{\pgfqpoint{1.368366in}{1.631221in}}%
\pgfpathlineto{\pgfqpoint{1.365231in}{1.631221in}}%
\pgfpathlineto{\pgfqpoint{1.355827in}{1.642475in}}%
\pgfpathlineto{\pgfqpoint{1.352692in}{1.642475in}}%
\pgfpathlineto{\pgfqpoint{1.343288in}{1.653730in}}%
\pgfpathlineto{\pgfqpoint{1.340153in}{1.653730in}}%
\pgfpathlineto{\pgfqpoint{1.327614in}{1.668736in}}%
\pgfpathlineto{\pgfqpoint{1.324479in}{1.668736in}}%
\pgfpathlineto{\pgfqpoint{1.315075in}{1.679991in}}%
\pgfpathlineto{\pgfqpoint{1.311940in}{1.679991in}}%
\pgfpathlineto{\pgfqpoint{1.302536in}{1.691245in}}%
\pgfpathlineto{\pgfqpoint{1.299401in}{1.691245in}}%
\pgfpathlineto{\pgfqpoint{1.290310in}{1.702125in}}%
\pgfpathlineto{\pgfqpoint{1.290310in}{1.705876in}}%
\pgfpathlineto{\pgfqpoint{1.293445in}{1.709628in}}%
\pgfpathlineto{\pgfqpoint{1.293445in}{1.720882in}}%
\pgfpathlineto{\pgfqpoint{1.296580in}{1.724634in}}%
\pgfpathlineto{\pgfqpoint{1.296580in}{1.732137in}}%
\pgfpathlineto{\pgfqpoint{1.299715in}{1.735889in}}%
\pgfpathlineto{\pgfqpoint{1.299715in}{1.747143in}}%
\pgfpathlineto{\pgfqpoint{1.302849in}{1.750895in}}%
\pgfpathlineto{\pgfqpoint{1.302849in}{1.758398in}}%
\pgfpathlineto{\pgfqpoint{1.305984in}{1.762149in}}%
\pgfpathlineto{\pgfqpoint{1.305984in}{1.773404in}}%
\pgfpathlineto{\pgfqpoint{1.309119in}{1.777155in}}%
\pgfpathlineto{\pgfqpoint{1.309119in}{1.784659in}}%
\pgfpathlineto{\pgfqpoint{1.312254in}{1.788410in}}%
\pgfpathlineto{\pgfqpoint{1.312254in}{1.799665in}}%
\pgfpathlineto{\pgfqpoint{1.315388in}{1.803416in}}%
\pgfpathlineto{\pgfqpoint{1.315388in}{1.810919in}}%
\pgfpathlineto{\pgfqpoint{1.318523in}{1.814671in}}%
\pgfpathlineto{\pgfqpoint{1.318523in}{1.822174in}}%
\pgfpathlineto{\pgfqpoint{1.321658in}{1.825925in}}%
\pgfpathlineto{\pgfqpoint{1.321658in}{1.837180in}}%
\pgfpathlineto{\pgfqpoint{1.324793in}{1.840932in}}%
\pgfpathlineto{\pgfqpoint{1.324793in}{1.848435in}}%
\pgfpathlineto{\pgfqpoint{1.327927in}{1.852186in}}%
\pgfpathlineto{\pgfqpoint{1.327927in}{1.863441in}}%
\pgfpathlineto{\pgfqpoint{1.331062in}{1.867192in}}%
\pgfpathlineto{\pgfqpoint{1.331062in}{1.874695in}}%
\pgfpathlineto{\pgfqpoint{1.334197in}{1.878447in}}%
\pgfpathlineto{\pgfqpoint{1.334197in}{1.889702in}}%
\pgfpathlineto{\pgfqpoint{1.337332in}{1.893453in}}%
\pgfpathlineto{\pgfqpoint{1.337332in}{1.900956in}}%
\pgfpathlineto{\pgfqpoint{1.340466in}{1.904708in}}%
\pgfpathlineto{\pgfqpoint{1.340466in}{1.915962in}}%
\pgfpathlineto{\pgfqpoint{1.343601in}{1.919714in}}%
\pgfpathlineto{\pgfqpoint{1.343601in}{1.927217in}}%
\pgfpathlineto{\pgfqpoint{1.346736in}{1.930968in}}%
\pgfpathlineto{\pgfqpoint{1.346736in}{1.942223in}}%
\pgfpathlineto{\pgfqpoint{1.349871in}{1.945975in}}%
\pgfpathlineto{\pgfqpoint{1.349871in}{1.953478in}}%
\pgfpathlineto{\pgfqpoint{1.353005in}{1.957229in}}%
\pgfpathlineto{\pgfqpoint{1.353005in}{1.968484in}}%
\pgfpathlineto{\pgfqpoint{1.356140in}{1.972235in}}%
\pgfpathlineto{\pgfqpoint{1.356140in}{1.979738in}}%
\pgfpathlineto{\pgfqpoint{1.359275in}{1.983490in}}%
\pgfpathlineto{\pgfqpoint{1.359275in}{1.994745in}}%
\pgfpathlineto{\pgfqpoint{1.362410in}{1.998496in}}%
\pgfpathlineto{\pgfqpoint{1.362410in}{2.005999in}}%
\pgfpathlineto{\pgfqpoint{1.365544in}{2.009751in}}%
\pgfpathlineto{\pgfqpoint{1.365544in}{2.021005in}}%
\pgfpathlineto{\pgfqpoint{1.368679in}{2.024757in}}%
\pgfpathlineto{\pgfqpoint{1.368679in}{2.032260in}}%
\pgfpathlineto{\pgfqpoint{1.371814in}{2.036011in}}%
\pgfpathlineto{\pgfqpoint{1.371814in}{2.043514in}}%
\pgfpathlineto{\pgfqpoint{1.374949in}{2.047266in}}%
\pgfpathlineto{\pgfqpoint{1.374949in}{2.058521in}}%
\pgfpathlineto{\pgfqpoint{1.378083in}{2.062272in}}%
\pgfpathlineto{\pgfqpoint{1.378083in}{2.069775in}}%
\pgfpathlineto{\pgfqpoint{1.381218in}{2.073527in}}%
\pgfpathlineto{\pgfqpoint{1.381218in}{2.084781in}}%
\pgfpathlineto{\pgfqpoint{1.384353in}{2.088533in}}%
\pgfpathlineto{\pgfqpoint{1.384353in}{2.096036in}}%
\pgfpathlineto{\pgfqpoint{1.387487in}{2.099787in}}%
\pgfpathlineto{\pgfqpoint{1.387487in}{2.111042in}}%
\pgfpathlineto{\pgfqpoint{1.390622in}{2.114794in}}%
\pgfpathlineto{\pgfqpoint{1.390622in}{2.122297in}}%
\pgfpathlineto{\pgfqpoint{1.393757in}{2.126048in}}%
\pgfpathlineto{\pgfqpoint{1.393757in}{2.137303in}}%
\pgfpathlineto{\pgfqpoint{1.396892in}{2.141054in}}%
\pgfpathlineto{\pgfqpoint{1.396892in}{2.148557in}}%
\pgfpathlineto{\pgfqpoint{1.400026in}{2.152309in}}%
\pgfpathlineto{\pgfqpoint{1.400026in}{2.163564in}}%
\pgfpathlineto{\pgfqpoint{1.403161in}{2.167315in}}%
\pgfpathlineto{\pgfqpoint{1.403161in}{2.174818in}}%
\pgfpathlineto{\pgfqpoint{1.406296in}{2.178570in}}%
\pgfpathlineto{\pgfqpoint{1.406296in}{2.189824in}}%
\pgfpathlineto{\pgfqpoint{1.409431in}{2.193576in}}%
\pgfpathlineto{\pgfqpoint{1.409431in}{2.201079in}}%
\pgfpathlineto{\pgfqpoint{1.412565in}{2.204830in}}%
\pgfpathlineto{\pgfqpoint{1.412565in}{2.216085in}}%
\pgfpathlineto{\pgfqpoint{1.415700in}{2.219837in}}%
\pgfpathlineto{\pgfqpoint{1.415700in}{2.227340in}}%
\pgfpathlineto{\pgfqpoint{1.418835in}{2.231091in}}%
\pgfpathlineto{\pgfqpoint{1.418835in}{2.242346in}}%
\pgfpathlineto{\pgfqpoint{1.421970in}{2.246097in}}%
\pgfpathlineto{\pgfqpoint{1.421970in}{2.253600in}}%
\pgfpathlineto{\pgfqpoint{1.425104in}{2.257352in}}%
\pgfpathlineto{\pgfqpoint{1.425104in}{2.264855in}}%
\pgfpathlineto{\pgfqpoint{1.428239in}{2.268607in}}%
\pgfpathlineto{\pgfqpoint{1.428239in}{2.279861in}}%
\pgfpathlineto{\pgfqpoint{1.431374in}{2.283613in}}%
\pgfpathlineto{\pgfqpoint{1.431374in}{2.291116in}}%
\pgfpathlineto{\pgfqpoint{1.434509in}{2.294867in}}%
\pgfpathlineto{\pgfqpoint{1.434509in}{2.306122in}}%
\pgfpathlineto{\pgfqpoint{1.437643in}{2.309873in}}%
\pgfpathlineto{\pgfqpoint{1.437643in}{2.317376in}}%
\pgfpathlineto{\pgfqpoint{1.440778in}{2.321128in}}%
\pgfpathlineto{\pgfqpoint{1.440778in}{2.332383in}}%
\pgfpathlineto{\pgfqpoint{1.443913in}{2.336134in}}%
\pgfpathlineto{\pgfqpoint{1.443913in}{2.343637in}}%
\pgfpathlineto{\pgfqpoint{1.447048in}{2.347389in}}%
\pgfpathlineto{\pgfqpoint{1.447048in}{2.358643in}}%
\pgfpathlineto{\pgfqpoint{1.450182in}{2.362395in}}%
\pgfpathlineto{\pgfqpoint{1.450182in}{2.369898in}}%
\pgfpathlineto{\pgfqpoint{1.453317in}{2.373649in}}%
\pgfpathlineto{\pgfqpoint{1.453317in}{2.384904in}}%
\pgfpathlineto{\pgfqpoint{1.456452in}{2.388656in}}%
\pgfpathlineto{\pgfqpoint{1.456452in}{2.396159in}}%
\pgfpathlineto{\pgfqpoint{1.459587in}{2.399910in}}%
\pgfpathlineto{\pgfqpoint{1.459587in}{2.403662in}}%
\pgfpathlineto{\pgfqpoint{1.462408in}{2.407038in}}%
\pgfpathlineto{\pgfqpoint{1.465856in}{2.407413in}}%
\pgfpathlineto{\pgfqpoint{1.468677in}{2.410790in}}%
\pgfpathlineto{\pgfqpoint{1.478395in}{2.411165in}}%
\pgfpathlineto{\pgfqpoint{1.481216in}{2.414541in}}%
\pgfpathlineto{\pgfqpoint{1.490934in}{2.414916in}}%
\pgfpathlineto{\pgfqpoint{1.493755in}{2.418293in}}%
\pgfpathlineto{\pgfqpoint{1.503473in}{2.418668in}}%
\pgfpathlineto{\pgfqpoint{1.506294in}{2.422044in}}%
\pgfpathlineto{\pgfqpoint{1.516012in}{2.422419in}}%
\pgfpathlineto{\pgfqpoint{1.518833in}{2.425796in}}%
\pgfpathlineto{\pgfqpoint{1.531686in}{2.426171in}}%
\pgfpathlineto{\pgfqpoint{1.534507in}{2.429547in}}%
\pgfpathlineto{\pgfqpoint{1.544225in}{2.429923in}}%
\pgfpathlineto{\pgfqpoint{1.547046in}{2.433299in}}%
\pgfpathlineto{\pgfqpoint{1.556764in}{2.433674in}}%
\pgfpathlineto{\pgfqpoint{1.559585in}{2.437050in}}%
\pgfpathlineto{\pgfqpoint{1.569303in}{2.437426in}}%
\pgfpathlineto{\pgfqpoint{1.572124in}{2.440802in}}%
\pgfpathlineto{\pgfqpoint{1.584976in}{2.441177in}}%
\pgfpathlineto{\pgfqpoint{1.587798in}{2.444553in}}%
\pgfpathlineto{\pgfqpoint{1.597515in}{2.444929in}}%
\pgfpathlineto{\pgfqpoint{1.600337in}{2.448305in}}%
\pgfpathlineto{\pgfqpoint{1.610054in}{2.448680in}}%
\pgfpathlineto{\pgfqpoint{1.612875in}{2.452057in}}%
\pgfpathlineto{\pgfqpoint{1.622593in}{2.452432in}}%
\pgfpathlineto{\pgfqpoint{1.625414in}{2.455808in}}%
\pgfpathlineto{\pgfqpoint{1.638267in}{2.456183in}}%
\pgfpathlineto{\pgfqpoint{1.641088in}{2.459560in}}%
\pgfpathlineto{\pgfqpoint{1.650806in}{2.459935in}}%
\pgfpathlineto{\pgfqpoint{1.653627in}{2.463311in}}%
\pgfpathlineto{\pgfqpoint{1.663345in}{2.463686in}}%
\pgfpathlineto{\pgfqpoint{1.666166in}{2.467063in}}%
\pgfpathlineto{\pgfqpoint{1.675884in}{2.467438in}}%
\pgfpathlineto{\pgfqpoint{1.678705in}{2.470814in}}%
\pgfpathlineto{\pgfqpoint{1.688423in}{2.471189in}}%
\pgfpathlineto{\pgfqpoint{1.691244in}{2.474566in}}%
\pgfpathlineto{\pgfqpoint{1.704096in}{2.474941in}}%
\pgfpathlineto{\pgfqpoint{1.706918in}{2.478317in}}%
\pgfpathlineto{\pgfqpoint{1.716635in}{2.478692in}}%
\pgfpathlineto{\pgfqpoint{1.719457in}{2.482069in}}%
\pgfpathlineto{\pgfqpoint{1.729174in}{2.482444in}}%
\pgfpathlineto{\pgfqpoint{1.731996in}{2.485820in}}%
\pgfpathlineto{\pgfqpoint{1.741713in}{2.486196in}}%
\pgfpathlineto{\pgfqpoint{1.744535in}{2.489572in}}%
\pgfpathlineto{\pgfqpoint{1.757387in}{2.489947in}}%
\pgfpathlineto{\pgfqpoint{1.760208in}{2.493323in}}%
\pgfpathlineto{\pgfqpoint{1.769926in}{2.493699in}}%
\pgfpathlineto{\pgfqpoint{1.772747in}{2.497075in}}%
\pgfpathlineto{\pgfqpoint{1.782465in}{2.497450in}}%
\pgfpathlineto{\pgfqpoint{1.785286in}{2.500827in}}%
\pgfpathlineto{\pgfqpoint{1.795004in}{2.501202in}}%
\pgfpathlineto{\pgfqpoint{1.797825in}{2.504578in}}%
\pgfpathlineto{\pgfqpoint{1.810678in}{2.504953in}}%
\pgfpathlineto{\pgfqpoint{1.813499in}{2.508330in}}%
\pgfpathlineto{\pgfqpoint{1.823217in}{2.508705in}}%
\pgfpathlineto{\pgfqpoint{1.826038in}{2.512081in}}%
\pgfpathlineto{\pgfqpoint{1.835756in}{2.512456in}}%
\pgfpathlineto{\pgfqpoint{1.838577in}{2.515833in}}%
\pgfpathlineto{\pgfqpoint{1.848295in}{2.516208in}}%
\pgfpathlineto{\pgfqpoint{1.851116in}{2.519584in}}%
\pgfpathlineto{\pgfqpoint{1.860834in}{2.519959in}}%
\pgfpathlineto{\pgfqpoint{1.863655in}{2.523336in}}%
\pgfpathlineto{\pgfqpoint{1.876507in}{2.523711in}}%
\pgfpathlineto{\pgfqpoint{1.879329in}{2.527087in}}%
\pgfpathlineto{\pgfqpoint{1.889046in}{2.527462in}}%
\pgfpathlineto{\pgfqpoint{1.891868in}{2.530839in}}%
\pgfpathlineto{\pgfqpoint{1.901585in}{2.531214in}}%
\pgfpathlineto{\pgfqpoint{1.904407in}{2.534590in}}%
\pgfpathlineto{\pgfqpoint{1.914124in}{2.534965in}}%
\pgfpathlineto{\pgfqpoint{1.916945in}{2.538342in}}%
\pgfpathlineto{\pgfqpoint{1.929798in}{2.538717in}}%
\pgfpathlineto{\pgfqpoint{1.932619in}{2.542093in}}%
\pgfpathlineto{\pgfqpoint{1.942337in}{2.542469in}}%
\pgfpathlineto{\pgfqpoint{1.945158in}{2.545845in}}%
\pgfpathlineto{\pgfqpoint{1.954876in}{2.546220in}}%
\pgfpathlineto{\pgfqpoint{1.957697in}{2.549596in}}%
\pgfpathlineto{\pgfqpoint{1.967415in}{2.549972in}}%
\pgfpathlineto{\pgfqpoint{1.970236in}{2.553348in}}%
\pgfpathlineto{\pgfqpoint{1.979954in}{2.553723in}}%
\pgfpathlineto{\pgfqpoint{1.982775in}{2.557100in}}%
\pgfpathlineto{\pgfqpoint{1.995628in}{2.557475in}}%
\pgfpathlineto{\pgfqpoint{1.998449in}{2.560851in}}%
\pgfpathlineto{\pgfqpoint{2.008166in}{2.561226in}}%
\pgfpathlineto{\pgfqpoint{2.010988in}{2.564603in}}%
\pgfpathlineto{\pgfqpoint{2.020705in}{2.564978in}}%
\pgfpathlineto{\pgfqpoint{2.023527in}{2.568354in}}%
\pgfpathlineto{\pgfqpoint{2.033244in}{2.568729in}}%
\pgfpathlineto{\pgfqpoint{2.036066in}{2.572106in}}%
\pgfpathlineto{\pgfqpoint{2.048918in}{2.572481in}}%
\pgfpathlineto{\pgfqpoint{2.051739in}{2.575857in}}%
\pgfpathlineto{\pgfqpoint{2.061457in}{2.576232in}}%
\pgfpathlineto{\pgfqpoint{2.064278in}{2.579609in}}%
\pgfpathlineto{\pgfqpoint{2.073996in}{2.579984in}}%
\pgfpathlineto{\pgfqpoint{2.076817in}{2.583360in}}%
\pgfpathlineto{\pgfqpoint{2.086535in}{2.583735in}}%
\pgfpathlineto{\pgfqpoint{2.089356in}{2.587112in}}%
\pgfpathlineto{\pgfqpoint{2.102209in}{2.587487in}}%
\pgfpathlineto{\pgfqpoint{2.105030in}{2.590863in}}%
\pgfpathlineto{\pgfqpoint{2.114748in}{2.591238in}}%
\pgfpathlineto{\pgfqpoint{2.117569in}{2.594615in}}%
\pgfpathlineto{\pgfqpoint{2.127287in}{2.594990in}}%
\pgfpathlineto{\pgfqpoint{2.130108in}{2.598366in}}%
\pgfpathlineto{\pgfqpoint{2.139826in}{2.598742in}}%
\pgfpathlineto{\pgfqpoint{2.142647in}{2.602118in}}%
\pgfpathlineto{\pgfqpoint{2.152365in}{2.602493in}}%
\pgfpathlineto{\pgfqpoint{2.155186in}{2.605869in}}%
\pgfpathlineto{\pgfqpoint{2.168038in}{2.606245in}}%
\pgfpathlineto{\pgfqpoint{2.170860in}{2.609621in}}%
\pgfpathlineto{\pgfqpoint{2.180577in}{2.609996in}}%
\pgfpathlineto{\pgfqpoint{2.183399in}{2.613373in}}%
\pgfpathlineto{\pgfqpoint{2.193116in}{2.613748in}}%
\pgfpathlineto{\pgfqpoint{2.195938in}{2.617124in}}%
\pgfpathlineto{\pgfqpoint{2.205655in}{2.617499in}}%
\pgfpathlineto{\pgfqpoint{2.208477in}{2.620876in}}%
\pgfpathlineto{\pgfqpoint{2.221329in}{2.621251in}}%
\pgfpathlineto{\pgfqpoint{2.224150in}{2.624627in}}%
\pgfpathlineto{\pgfqpoint{2.233868in}{2.625002in}}%
\pgfpathlineto{\pgfqpoint{2.236689in}{2.628379in}}%
\pgfpathlineto{\pgfqpoint{2.246407in}{2.628754in}}%
\pgfpathlineto{\pgfqpoint{2.249228in}{2.632130in}}%
\pgfpathlineto{\pgfqpoint{2.258946in}{2.632505in}}%
\pgfpathlineto{\pgfqpoint{2.261767in}{2.635882in}}%
\pgfpathlineto{\pgfqpoint{2.274620in}{2.636257in}}%
\pgfpathlineto{\pgfqpoint{2.277441in}{2.639633in}}%
\pgfpathlineto{\pgfqpoint{2.287159in}{2.640008in}}%
\pgfpathlineto{\pgfqpoint{2.289980in}{2.643385in}}%
\pgfpathlineto{\pgfqpoint{2.299698in}{2.643760in}}%
\pgfpathlineto{\pgfqpoint{2.302519in}{2.647136in}}%
\pgfpathlineto{\pgfqpoint{2.312236in}{2.647512in}}%
\pgfpathlineto{\pgfqpoint{2.315058in}{2.650888in}}%
\pgfpathlineto{\pgfqpoint{2.324775in}{2.651263in}}%
\pgfpathlineto{\pgfqpoint{2.327597in}{2.654639in}}%
\pgfpathlineto{\pgfqpoint{2.340449in}{2.655015in}}%
\pgfpathlineto{\pgfqpoint{2.343270in}{2.658391in}}%
\pgfpathlineto{\pgfqpoint{2.352988in}{2.658766in}}%
\pgfpathlineto{\pgfqpoint{2.355809in}{2.662142in}}%
\pgfpathlineto{\pgfqpoint{2.365527in}{2.662518in}}%
\pgfpathlineto{\pgfqpoint{2.368348in}{2.665894in}}%
\pgfpathlineto{\pgfqpoint{2.378066in}{2.666269in}}%
\pgfpathlineto{\pgfqpoint{2.380887in}{2.669646in}}%
\pgfpathlineto{\pgfqpoint{2.393740in}{2.670021in}}%
\pgfpathlineto{\pgfqpoint{2.396561in}{2.673397in}}%
\pgfpathlineto{\pgfqpoint{2.406279in}{2.673772in}}%
\pgfpathlineto{\pgfqpoint{2.409100in}{2.677149in}}%
\pgfpathlineto{\pgfqpoint{2.418818in}{2.677524in}}%
\pgfpathlineto{\pgfqpoint{2.421639in}{2.680900in}}%
\pgfpathlineto{\pgfqpoint{2.431357in}{2.681275in}}%
\pgfpathlineto{\pgfqpoint{2.434178in}{2.684652in}}%
\pgfpathlineto{\pgfqpoint{2.447030in}{2.685027in}}%
\pgfpathlineto{\pgfqpoint{2.449852in}{2.688403in}}%
\pgfpathlineto{\pgfqpoint{2.459569in}{2.688778in}}%
\pgfpathlineto{\pgfqpoint{2.462391in}{2.692155in}}%
\pgfpathlineto{\pgfqpoint{2.472108in}{2.692530in}}%
\pgfpathlineto{\pgfqpoint{2.474930in}{2.695906in}}%
\pgfpathlineto{\pgfqpoint{2.484647in}{2.696281in}}%
\pgfpathlineto{\pgfqpoint{2.487469in}{2.699658in}}%
\pgfpathlineto{\pgfqpoint{2.497186in}{2.700033in}}%
\pgfpathlineto{\pgfqpoint{2.500008in}{2.703409in}}%
\pgfpathlineto{\pgfqpoint{2.512860in}{2.703785in}}%
\pgfpathlineto{\pgfqpoint{2.515681in}{2.707161in}}%
\pgfpathlineto{\pgfqpoint{2.525399in}{2.707536in}}%
\pgfpathlineto{\pgfqpoint{2.528220in}{2.710912in}}%
\pgfpathlineto{\pgfqpoint{2.537938in}{2.711288in}}%
\pgfpathlineto{\pgfqpoint{2.540759in}{2.714664in}}%
\pgfpathlineto{\pgfqpoint{2.550477in}{2.715039in}}%
\pgfpathlineto{\pgfqpoint{2.553298in}{2.718416in}}%
\pgfpathlineto{\pgfqpoint{2.566151in}{2.718791in}}%
\pgfpathlineto{\pgfqpoint{2.568972in}{2.722167in}}%
\pgfpathlineto{\pgfqpoint{2.578690in}{2.722542in}}%
\pgfpathlineto{\pgfqpoint{2.581511in}{2.725919in}}%
\pgfpathlineto{\pgfqpoint{2.591229in}{2.726294in}}%
\pgfpathlineto{\pgfqpoint{2.594050in}{2.729670in}}%
\pgfpathlineto{\pgfqpoint{2.603768in}{2.730045in}}%
\pgfpathlineto{\pgfqpoint{2.606589in}{2.733422in}}%
\pgfpathlineto{\pgfqpoint{2.616306in}{2.733797in}}%
\pgfpathlineto{\pgfqpoint{2.619128in}{2.737173in}}%
\pgfpathlineto{\pgfqpoint{2.631980in}{2.737548in}}%
\pgfpathlineto{\pgfqpoint{2.634801in}{2.740925in}}%
\pgfpathlineto{\pgfqpoint{2.644519in}{2.741300in}}%
\pgfpathlineto{\pgfqpoint{2.647340in}{2.744676in}}%
\pgfpathlineto{\pgfqpoint{2.657058in}{2.745051in}}%
\pgfpathlineto{\pgfqpoint{2.659879in}{2.748428in}}%
\pgfpathlineto{\pgfqpoint{2.672418in}{2.748428in}}%
\pgfpathlineto{\pgfqpoint{2.712857in}{2.700033in}}%
\pgfpathlineto{\pgfqpoint{2.713170in}{2.695906in}}%
\pgfpathlineto{\pgfqpoint{2.763012in}{2.636257in}}%
\pgfpathlineto{\pgfqpoint{2.763326in}{2.632130in}}%
\pgfpathlineto{\pgfqpoint{2.810034in}{2.576232in}}%
\pgfpathlineto{\pgfqpoint{2.810347in}{2.572106in}}%
\pgfpathlineto{\pgfqpoint{2.860189in}{2.512456in}}%
\pgfpathlineto{\pgfqpoint{2.860503in}{2.508330in}}%
\pgfpathlineto{\pgfqpoint{2.907211in}{2.452432in}}%
\pgfpathlineto{\pgfqpoint{2.907524in}{2.448305in}}%
\pgfpathlineto{\pgfqpoint{2.957366in}{2.388656in}}%
\pgfpathlineto{\pgfqpoint{2.957680in}{2.384529in}}%
\pgfpathlineto{\pgfqpoint{3.004388in}{2.328631in}}%
\pgfpathlineto{\pgfqpoint{3.004701in}{2.324504in}}%
\pgfpathlineto{\pgfqpoint{3.054543in}{2.264855in}}%
\pgfpathlineto{\pgfqpoint{3.054857in}{2.260728in}}%
\pgfpathlineto{\pgfqpoint{3.095295in}{2.212334in}}%
\pgfpathlineto{\pgfqpoint{3.095295in}{2.208582in}}%
\pgfpathlineto{\pgfqpoint{3.092160in}{2.204830in}}%
\pgfpathlineto{\pgfqpoint{3.092160in}{2.197327in}}%
\pgfpathlineto{\pgfqpoint{3.089026in}{2.193576in}}%
\pgfpathlineto{\pgfqpoint{3.089026in}{2.186073in}}%
\pgfpathlineto{\pgfqpoint{3.085891in}{2.182321in}}%
\pgfpathlineto{\pgfqpoint{3.085891in}{2.174818in}}%
\pgfpathlineto{\pgfqpoint{3.082756in}{2.171067in}}%
\pgfpathlineto{\pgfqpoint{3.082756in}{2.163564in}}%
\pgfpathlineto{\pgfqpoint{3.079621in}{2.159812in}}%
\pgfpathlineto{\pgfqpoint{3.079621in}{2.148557in}}%
\pgfpathlineto{\pgfqpoint{3.076487in}{2.144806in}}%
\pgfpathlineto{\pgfqpoint{3.076487in}{2.137303in}}%
\pgfpathlineto{\pgfqpoint{3.073352in}{2.133551in}}%
\pgfpathlineto{\pgfqpoint{3.073352in}{2.126048in}}%
\pgfpathlineto{\pgfqpoint{3.070217in}{2.122297in}}%
\pgfpathlineto{\pgfqpoint{3.070217in}{2.114794in}}%
\pgfpathlineto{\pgfqpoint{3.067082in}{2.111042in}}%
\pgfpathlineto{\pgfqpoint{3.067082in}{2.103539in}}%
\pgfpathlineto{\pgfqpoint{3.063948in}{2.099787in}}%
\pgfpathlineto{\pgfqpoint{3.063948in}{2.092284in}}%
\pgfpathlineto{\pgfqpoint{3.060813in}{2.088533in}}%
\pgfpathlineto{\pgfqpoint{3.060813in}{2.081030in}}%
\pgfpathlineto{\pgfqpoint{3.057678in}{2.077278in}}%
\pgfpathlineto{\pgfqpoint{3.057678in}{2.069775in}}%
\pgfpathlineto{\pgfqpoint{3.054543in}{2.066024in}}%
\pgfpathlineto{\pgfqpoint{3.054543in}{2.058521in}}%
\pgfpathlineto{\pgfqpoint{3.051409in}{2.054769in}}%
\pgfpathlineto{\pgfqpoint{3.051409in}{2.043514in}}%
\pgfpathlineto{\pgfqpoint{3.048274in}{2.039763in}}%
\pgfpathlineto{\pgfqpoint{3.048274in}{2.032260in}}%
\pgfpathlineto{\pgfqpoint{3.045139in}{2.028508in}}%
\pgfpathlineto{\pgfqpoint{3.045139in}{2.021005in}}%
\pgfpathlineto{\pgfqpoint{3.042004in}{2.017254in}}%
\pgfpathlineto{\pgfqpoint{3.042004in}{2.009751in}}%
\pgfpathlineto{\pgfqpoint{3.038870in}{2.005999in}}%
\pgfpathlineto{\pgfqpoint{3.038870in}{1.998496in}}%
\pgfpathlineto{\pgfqpoint{3.035735in}{1.994745in}}%
\pgfpathlineto{\pgfqpoint{3.035735in}{1.987241in}}%
\pgfpathlineto{\pgfqpoint{3.032600in}{1.983490in}}%
\pgfpathlineto{\pgfqpoint{3.032600in}{1.975987in}}%
\pgfpathlineto{\pgfqpoint{3.029466in}{1.972235in}}%
\pgfpathlineto{\pgfqpoint{3.029466in}{1.964732in}}%
\pgfpathlineto{\pgfqpoint{3.026331in}{1.960981in}}%
\pgfpathlineto{\pgfqpoint{3.026331in}{1.949726in}}%
\pgfpathlineto{\pgfqpoint{3.023196in}{1.945975in}}%
\pgfpathlineto{\pgfqpoint{3.023196in}{1.938471in}}%
\pgfpathlineto{\pgfqpoint{3.020061in}{1.934720in}}%
\pgfpathlineto{\pgfqpoint{3.020061in}{1.927217in}}%
\pgfpathlineto{\pgfqpoint{3.016927in}{1.923465in}}%
\pgfpathlineto{\pgfqpoint{3.016927in}{1.915962in}}%
\pgfpathlineto{\pgfqpoint{3.013792in}{1.912211in}}%
\pgfpathlineto{\pgfqpoint{3.013792in}{1.904708in}}%
\pgfpathlineto{\pgfqpoint{3.010657in}{1.900956in}}%
\pgfpathlineto{\pgfqpoint{3.010657in}{1.893453in}}%
\pgfpathlineto{\pgfqpoint{3.007522in}{1.889702in}}%
\pgfpathlineto{\pgfqpoint{3.007522in}{1.882198in}}%
\pgfpathlineto{\pgfqpoint{3.004388in}{1.878447in}}%
\pgfpathlineto{\pgfqpoint{3.004388in}{1.870944in}}%
\pgfpathlineto{\pgfqpoint{3.001253in}{1.867192in}}%
\pgfpathlineto{\pgfqpoint{3.001253in}{1.859689in}}%
\pgfpathlineto{\pgfqpoint{2.998118in}{1.855938in}}%
\pgfpathlineto{\pgfqpoint{2.998118in}{1.844683in}}%
\pgfpathlineto{\pgfqpoint{2.994983in}{1.840932in}}%
\pgfpathlineto{\pgfqpoint{2.994983in}{1.833429in}}%
\pgfpathlineto{\pgfqpoint{2.991849in}{1.829677in}}%
\pgfpathlineto{\pgfqpoint{2.991849in}{1.822174in}}%
\pgfpathlineto{\pgfqpoint{2.988714in}{1.818422in}}%
\pgfpathlineto{\pgfqpoint{2.988714in}{1.810919in}}%
\pgfpathlineto{\pgfqpoint{2.985579in}{1.807168in}}%
\pgfpathlineto{\pgfqpoint{2.985579in}{1.799665in}}%
\pgfpathlineto{\pgfqpoint{2.982444in}{1.795913in}}%
\pgfpathlineto{\pgfqpoint{2.982444in}{1.788410in}}%
\pgfpathlineto{\pgfqpoint{2.979310in}{1.784659in}}%
\pgfpathlineto{\pgfqpoint{2.979310in}{1.777155in}}%
\pgfpathlineto{\pgfqpoint{2.976175in}{1.773404in}}%
\pgfpathlineto{\pgfqpoint{2.976175in}{1.765901in}}%
\pgfpathlineto{\pgfqpoint{2.973040in}{1.762149in}}%
\pgfpathlineto{\pgfqpoint{2.973040in}{1.750895in}}%
\pgfpathlineto{\pgfqpoint{2.969905in}{1.747143in}}%
\pgfpathlineto{\pgfqpoint{2.969905in}{1.739640in}}%
\pgfpathlineto{\pgfqpoint{2.966771in}{1.735889in}}%
\pgfpathlineto{\pgfqpoint{2.966771in}{1.728386in}}%
\pgfpathlineto{\pgfqpoint{2.963636in}{1.724634in}}%
\pgfpathlineto{\pgfqpoint{2.963636in}{1.717131in}}%
\pgfpathlineto{\pgfqpoint{2.960501in}{1.713379in}}%
\pgfpathlineto{\pgfqpoint{2.960501in}{1.705876in}}%
\pgfpathlineto{\pgfqpoint{2.957366in}{1.702125in}}%
\pgfpathlineto{\pgfqpoint{2.957366in}{1.694622in}}%
\pgfpathlineto{\pgfqpoint{2.954232in}{1.690870in}}%
\pgfpathlineto{\pgfqpoint{2.954232in}{1.683367in}}%
\pgfpathlineto{\pgfqpoint{2.951097in}{1.679616in}}%
\pgfpathlineto{\pgfqpoint{2.951097in}{1.672113in}}%
\pgfpathlineto{\pgfqpoint{2.947962in}{1.668361in}}%
\pgfpathlineto{\pgfqpoint{2.947962in}{1.660858in}}%
\pgfpathlineto{\pgfqpoint{2.944827in}{1.657106in}}%
\pgfpathlineto{\pgfqpoint{2.944827in}{1.645852in}}%
\pgfpathlineto{\pgfqpoint{2.941693in}{1.642100in}}%
\pgfpathlineto{\pgfqpoint{2.941693in}{1.634597in}}%
\pgfpathlineto{\pgfqpoint{2.938558in}{1.630846in}}%
\pgfpathlineto{\pgfqpoint{2.938558in}{1.623343in}}%
\pgfpathlineto{\pgfqpoint{2.935423in}{1.619591in}}%
\pgfpathlineto{\pgfqpoint{2.935423in}{1.612088in}}%
\pgfpathlineto{\pgfqpoint{2.932288in}{1.608336in}}%
\pgfpathlineto{\pgfqpoint{2.932288in}{1.600833in}}%
\pgfpathlineto{\pgfqpoint{2.929154in}{1.597082in}}%
\pgfpathlineto{\pgfqpoint{2.929154in}{1.589579in}}%
\pgfpathlineto{\pgfqpoint{2.926019in}{1.585827in}}%
\pgfpathlineto{\pgfqpoint{2.926019in}{1.578324in}}%
\pgfpathlineto{\pgfqpoint{2.922884in}{1.574573in}}%
\pgfpathlineto{\pgfqpoint{2.922884in}{1.567070in}}%
\pgfpathlineto{\pgfqpoint{2.919750in}{1.563318in}}%
\pgfpathlineto{\pgfqpoint{2.919750in}{1.552063in}}%
\pgfpathlineto{\pgfqpoint{2.916615in}{1.548312in}}%
\pgfpathlineto{\pgfqpoint{2.916615in}{1.540809in}}%
\pgfpathlineto{\pgfqpoint{2.913480in}{1.537057in}}%
\pgfpathlineto{\pgfqpoint{2.913480in}{1.529554in}}%
\pgfpathlineto{\pgfqpoint{2.910345in}{1.525803in}}%
\pgfpathlineto{\pgfqpoint{2.910345in}{1.518300in}}%
\pgfpathlineto{\pgfqpoint{2.907211in}{1.514548in}}%
\pgfpathlineto{\pgfqpoint{2.907211in}{1.507045in}}%
\pgfpathlineto{\pgfqpoint{2.904076in}{1.503293in}}%
\pgfpathlineto{\pgfqpoint{2.904076in}{1.495790in}}%
\pgfpathlineto{\pgfqpoint{2.900941in}{1.492039in}}%
\pgfpathlineto{\pgfqpoint{2.900941in}{1.484536in}}%
\pgfpathlineto{\pgfqpoint{2.897806in}{1.480784in}}%
\pgfpathlineto{\pgfqpoint{2.897806in}{1.473281in}}%
\pgfpathlineto{\pgfqpoint{2.894672in}{1.469530in}}%
\pgfpathlineto{\pgfqpoint{2.894672in}{1.462027in}}%
\pgfpathlineto{\pgfqpoint{2.891537in}{1.458275in}}%
\pgfpathlineto{\pgfqpoint{2.891537in}{1.447020in}}%
\pgfpathlineto{\pgfqpoint{2.888402in}{1.443269in}}%
\pgfpathlineto{\pgfqpoint{2.888402in}{1.435766in}}%
\pgfpathlineto{\pgfqpoint{2.885267in}{1.432014in}}%
\pgfpathlineto{\pgfqpoint{2.885267in}{1.424511in}}%
\pgfpathlineto{\pgfqpoint{2.882133in}{1.420760in}}%
\pgfpathlineto{\pgfqpoint{2.882133in}{1.413257in}}%
\pgfpathlineto{\pgfqpoint{2.878998in}{1.409505in}}%
\pgfpathlineto{\pgfqpoint{2.878998in}{1.402002in}}%
\pgfpathlineto{\pgfqpoint{2.875863in}{1.398251in}}%
\pgfpathlineto{\pgfqpoint{2.875863in}{1.390747in}}%
\pgfpathlineto{\pgfqpoint{2.872728in}{1.386996in}}%
\pgfpathlineto{\pgfqpoint{2.872728in}{1.379493in}}%
\pgfpathlineto{\pgfqpoint{2.869594in}{1.375741in}}%
\pgfpathlineto{\pgfqpoint{2.869594in}{1.368238in}}%
\pgfpathlineto{\pgfqpoint{2.866459in}{1.364487in}}%
\pgfpathlineto{\pgfqpoint{2.866459in}{1.353232in}}%
\pgfpathlineto{\pgfqpoint{2.863324in}{1.349481in}}%
\pgfpathlineto{\pgfqpoint{2.863324in}{1.341977in}}%
\pgfpathlineto{\pgfqpoint{2.860189in}{1.338226in}}%
\pgfpathlineto{\pgfqpoint{2.860189in}{1.330723in}}%
\pgfpathlineto{\pgfqpoint{2.857055in}{1.326971in}}%
\pgfpathlineto{\pgfqpoint{2.857055in}{1.319468in}}%
\pgfpathlineto{\pgfqpoint{2.853920in}{1.315717in}}%
\pgfpathlineto{\pgfqpoint{2.853920in}{1.308214in}}%
\pgfpathlineto{\pgfqpoint{2.850785in}{1.304462in}}%
\pgfpathlineto{\pgfqpoint{2.850785in}{1.296959in}}%
\pgfpathlineto{\pgfqpoint{2.847650in}{1.293208in}}%
\pgfpathlineto{\pgfqpoint{2.847650in}{1.285704in}}%
\pgfpathlineto{\pgfqpoint{2.844516in}{1.281953in}}%
\pgfpathlineto{\pgfqpoint{2.844516in}{1.274450in}}%
\pgfpathlineto{\pgfqpoint{2.841381in}{1.270698in}}%
\pgfpathlineto{\pgfqpoint{2.841381in}{1.263195in}}%
\pgfpathlineto{\pgfqpoint{2.838246in}{1.259444in}}%
\pgfpathlineto{\pgfqpoint{2.838246in}{1.248189in}}%
\pgfpathlineto{\pgfqpoint{2.835111in}{1.244438in}}%
\pgfpathlineto{\pgfqpoint{2.835111in}{1.236935in}}%
\pgfpathlineto{\pgfqpoint{2.831977in}{1.233183in}}%
\pgfpathlineto{\pgfqpoint{2.831977in}{1.225680in}}%
\pgfpathlineto{\pgfqpoint{2.828842in}{1.221928in}}%
\pgfpathlineto{\pgfqpoint{2.828842in}{1.214425in}}%
\pgfpathlineto{\pgfqpoint{2.825707in}{1.210674in}}%
\pgfpathlineto{\pgfqpoint{2.825707in}{1.203171in}}%
\pgfpathlineto{\pgfqpoint{2.822573in}{1.199419in}}%
\pgfpathlineto{\pgfqpoint{2.822573in}{1.191916in}}%
\pgfpathlineto{\pgfqpoint{2.819438in}{1.188165in}}%
\pgfpathlineto{\pgfqpoint{2.819438in}{1.180662in}}%
\pgfpathlineto{\pgfqpoint{2.816303in}{1.176910in}}%
\pgfpathlineto{\pgfqpoint{2.816303in}{1.173158in}}%
\pgfpathlineto{\pgfqpoint{2.813482in}{1.169782in}}%
\pgfpathlineto{\pgfqpoint{2.806899in}{1.169407in}}%
\pgfpathlineto{\pgfqpoint{2.804078in}{1.166031in}}%
\pgfpathlineto{\pgfqpoint{2.797495in}{1.165655in}}%
\pgfpathlineto{\pgfqpoint{2.794673in}{1.162279in}}%
\pgfpathlineto{\pgfqpoint{2.791225in}{1.161904in}}%
\pgfpathlineto{\pgfqpoint{2.788404in}{1.158527in}}%
\pgfpathlineto{\pgfqpoint{2.781821in}{1.158152in}}%
\pgfpathlineto{\pgfqpoint{2.779000in}{1.154776in}}%
\pgfpathlineto{\pgfqpoint{2.772417in}{1.154401in}}%
\pgfpathlineto{\pgfqpoint{2.769595in}{1.151024in}}%
\pgfpathlineto{\pgfqpoint{2.763012in}{1.150649in}}%
\pgfpathlineto{\pgfqpoint{2.760191in}{1.147273in}}%
\pgfpathlineto{\pgfqpoint{2.756743in}{1.146898in}}%
\pgfpathlineto{\pgfqpoint{2.753922in}{1.143521in}}%
\pgfpathlineto{\pgfqpoint{2.747339in}{1.143146in}}%
\pgfpathlineto{\pgfqpoint{2.744517in}{1.139770in}}%
\pgfpathlineto{\pgfqpoint{2.737934in}{1.139395in}}%
\pgfpathlineto{\pgfqpoint{2.735113in}{1.136018in}}%
\pgfpathlineto{\pgfqpoint{2.728530in}{1.135643in}}%
\pgfpathlineto{\pgfqpoint{2.725709in}{1.132267in}}%
\pgfpathlineto{\pgfqpoint{2.719126in}{1.131892in}}%
\pgfpathlineto{\pgfqpoint{2.716305in}{1.128515in}}%
\pgfpathlineto{\pgfqpoint{2.712857in}{1.128140in}}%
\pgfpathlineto{\pgfqpoint{2.710035in}{1.124764in}}%
\pgfpathlineto{\pgfqpoint{2.703452in}{1.124388in}}%
\pgfpathlineto{\pgfqpoint{2.700631in}{1.121012in}}%
\pgfpathlineto{\pgfqpoint{2.694048in}{1.120637in}}%
\pgfpathlineto{\pgfqpoint{2.691227in}{1.117261in}}%
\pgfpathlineto{\pgfqpoint{2.684644in}{1.116885in}}%
\pgfpathlineto{\pgfqpoint{2.681823in}{1.113509in}}%
\pgfpathlineto{\pgfqpoint{2.678374in}{1.113134in}}%
\pgfpathlineto{\pgfqpoint{2.675553in}{1.109758in}}%
\pgfpathlineto{\pgfqpoint{2.668970in}{1.109382in}}%
\pgfpathlineto{\pgfqpoint{2.666149in}{1.106006in}}%
\pgfpathlineto{\pgfqpoint{2.659566in}{1.105631in}}%
\pgfpathlineto{\pgfqpoint{2.656745in}{1.102254in}}%
\pgfpathlineto{\pgfqpoint{2.650162in}{1.101879in}}%
\pgfpathlineto{\pgfqpoint{2.647340in}{1.098503in}}%
\pgfpathlineto{\pgfqpoint{2.643892in}{1.098128in}}%
\pgfpathlineto{\pgfqpoint{2.641071in}{1.094751in}}%
\pgfpathlineto{\pgfqpoint{2.634488in}{1.094376in}}%
\pgfpathlineto{\pgfqpoint{2.631667in}{1.091000in}}%
\pgfpathlineto{\pgfqpoint{2.625084in}{1.090625in}}%
\pgfpathlineto{\pgfqpoint{2.622262in}{1.087248in}}%
\pgfpathlineto{\pgfqpoint{2.615680in}{1.086873in}}%
\pgfpathlineto{\pgfqpoint{2.612858in}{1.083497in}}%
\pgfpathlineto{\pgfqpoint{2.606275in}{1.083122in}}%
\pgfpathlineto{\pgfqpoint{2.603454in}{1.079745in}}%
\pgfpathlineto{\pgfqpoint{2.600006in}{1.079370in}}%
\pgfpathlineto{\pgfqpoint{2.597185in}{1.075994in}}%
\pgfpathlineto{\pgfqpoint{2.590602in}{1.075619in}}%
\pgfpathlineto{\pgfqpoint{2.587780in}{1.072242in}}%
\pgfpathlineto{\pgfqpoint{2.581197in}{1.071867in}}%
\pgfpathlineto{\pgfqpoint{2.578376in}{1.068491in}}%
\pgfpathlineto{\pgfqpoint{2.571793in}{1.068115in}}%
\pgfpathlineto{\pgfqpoint{2.568972in}{1.064739in}}%
\pgfpathlineto{\pgfqpoint{2.565524in}{1.064364in}}%
\pgfpathlineto{\pgfqpoint{2.562702in}{1.060988in}}%
\pgfpathlineto{\pgfqpoint{2.556119in}{1.060612in}}%
\pgfpathlineto{\pgfqpoint{2.553298in}{1.057236in}}%
\pgfpathlineto{\pgfqpoint{2.546715in}{1.056861in}}%
\pgfpathlineto{\pgfqpoint{2.543894in}{1.053484in}}%
\pgfpathlineto{\pgfqpoint{2.537311in}{1.053109in}}%
\pgfpathlineto{\pgfqpoint{2.534490in}{1.049733in}}%
\pgfpathlineto{\pgfqpoint{2.531041in}{1.049358in}}%
\pgfpathlineto{\pgfqpoint{2.528220in}{1.045981in}}%
\pgfpathlineto{\pgfqpoint{2.521637in}{1.045606in}}%
\pgfpathlineto{\pgfqpoint{2.518816in}{1.042230in}}%
\pgfpathlineto{\pgfqpoint{2.512233in}{1.041855in}}%
\pgfpathlineto{\pgfqpoint{2.509412in}{1.038478in}}%
\pgfpathlineto{\pgfqpoint{2.502829in}{1.038103in}}%
\pgfpathlineto{\pgfqpoint{2.500008in}{1.034727in}}%
\pgfpathlineto{\pgfqpoint{2.493425in}{1.034352in}}%
\pgfpathlineto{\pgfqpoint{2.490603in}{1.030975in}}%
\pgfpathlineto{\pgfqpoint{2.487155in}{1.030600in}}%
\pgfpathlineto{\pgfqpoint{2.484334in}{1.027224in}}%
\pgfpathlineto{\pgfqpoint{2.477751in}{1.026849in}}%
\pgfpathlineto{\pgfqpoint{2.474930in}{1.023472in}}%
\pgfpathlineto{\pgfqpoint{2.468347in}{1.023097in}}%
\pgfpathlineto{\pgfqpoint{2.465525in}{1.019721in}}%
\pgfpathlineto{\pgfqpoint{2.458942in}{1.019346in}}%
\pgfpathlineto{\pgfqpoint{2.456121in}{1.015969in}}%
\pgfpathlineto{\pgfqpoint{2.452673in}{1.015594in}}%
\pgfpathlineto{\pgfqpoint{2.449852in}{1.012218in}}%
\pgfpathlineto{\pgfqpoint{2.443269in}{1.011842in}}%
\pgfpathlineto{\pgfqpoint{2.440447in}{1.008466in}}%
\pgfpathlineto{\pgfqpoint{2.433864in}{1.008091in}}%
\pgfpathlineto{\pgfqpoint{2.431043in}{1.004715in}}%
\pgfpathlineto{\pgfqpoint{2.424460in}{1.004339in}}%
\pgfpathlineto{\pgfqpoint{2.421639in}{1.000963in}}%
\pgfpathlineto{\pgfqpoint{2.418191in}{1.000588in}}%
\pgfpathlineto{\pgfqpoint{2.415369in}{0.997211in}}%
\pgfpathlineto{\pgfqpoint{2.408787in}{0.996836in}}%
\pgfpathlineto{\pgfqpoint{2.405965in}{0.993460in}}%
\pgfpathlineto{\pgfqpoint{2.399382in}{0.993085in}}%
\pgfpathlineto{\pgfqpoint{2.396561in}{0.989708in}}%
\pgfpathlineto{\pgfqpoint{2.389978in}{0.989333in}}%
\pgfpathlineto{\pgfqpoint{2.387157in}{0.985957in}}%
\pgfpathlineto{\pgfqpoint{2.380574in}{0.985582in}}%
\pgfpathlineto{\pgfqpoint{2.377753in}{0.982205in}}%
\pgfpathlineto{\pgfqpoint{2.374304in}{0.981830in}}%
\pgfpathlineto{\pgfqpoint{2.371483in}{0.978454in}}%
\pgfpathlineto{\pgfqpoint{2.364900in}{0.978079in}}%
\pgfpathlineto{\pgfqpoint{2.362079in}{0.974702in}}%
\pgfpathlineto{\pgfqpoint{2.355496in}{0.974327in}}%
\pgfpathlineto{\pgfqpoint{2.352675in}{0.970951in}}%
\pgfpathlineto{\pgfqpoint{2.346092in}{0.970576in}}%
\pgfpathlineto{\pgfqpoint{2.343270in}{0.967199in}}%
\pgfpathlineto{\pgfqpoint{2.339822in}{0.966824in}}%
\pgfpathlineto{\pgfqpoint{2.337001in}{0.963448in}}%
\pgfpathlineto{\pgfqpoint{2.330418in}{0.963073in}}%
\pgfpathlineto{\pgfqpoint{2.327597in}{0.959696in}}%
\pgfpathlineto{\pgfqpoint{2.321014in}{0.959321in}}%
\pgfpathlineto{\pgfqpoint{2.318192in}{0.955945in}}%
\pgfpathlineto{\pgfqpoint{2.311610in}{0.955569in}}%
\pgfpathlineto{\pgfqpoint{2.308788in}{0.952193in}}%
\pgfpathlineto{\pgfqpoint{2.305340in}{0.951818in}}%
\pgfpathlineto{\pgfqpoint{2.302519in}{0.948442in}}%
\pgfpathlineto{\pgfqpoint{2.295936in}{0.948066in}}%
\pgfpathlineto{\pgfqpoint{2.293115in}{0.944690in}}%
\pgfpathlineto{\pgfqpoint{2.286532in}{0.944315in}}%
\pgfpathlineto{\pgfqpoint{2.283710in}{0.940938in}}%
\pgfpathlineto{\pgfqpoint{2.277127in}{0.940563in}}%
\pgfpathlineto{\pgfqpoint{2.274306in}{0.937187in}}%
\pgfpathlineto{\pgfqpoint{2.270858in}{0.936812in}}%
\pgfpathlineto{\pgfqpoint{2.268037in}{0.933435in}}%
\pgfpathlineto{\pgfqpoint{2.261454in}{0.933060in}}%
\pgfpathlineto{\pgfqpoint{2.258632in}{0.929684in}}%
\pgfpathlineto{\pgfqpoint{2.252049in}{0.929309in}}%
\pgfpathlineto{\pgfqpoint{2.249228in}{0.925932in}}%
\pgfpathlineto{\pgfqpoint{2.242645in}{0.925557in}}%
\pgfpathlineto{\pgfqpoint{2.239824in}{0.922181in}}%
\pgfpathlineto{\pgfqpoint{2.233241in}{0.921806in}}%
\pgfpathlineto{\pgfqpoint{2.230420in}{0.918429in}}%
\pgfpathlineto{\pgfqpoint{2.226971in}{0.918054in}}%
\pgfpathlineto{\pgfqpoint{2.224150in}{0.914678in}}%
\pgfpathlineto{\pgfqpoint{2.217567in}{0.914303in}}%
\pgfpathlineto{\pgfqpoint{2.214746in}{0.910926in}}%
\pgfpathlineto{\pgfqpoint{2.208163in}{0.910551in}}%
\pgfpathlineto{\pgfqpoint{2.205342in}{0.907175in}}%
\pgfpathlineto{\pgfqpoint{2.198759in}{0.906799in}}%
\pgfpathlineto{\pgfqpoint{2.195938in}{0.903423in}}%
\pgfpathlineto{\pgfqpoint{2.192489in}{0.903048in}}%
\pgfpathlineto{\pgfqpoint{2.189668in}{0.899672in}}%
\pgfpathlineto{\pgfqpoint{2.183085in}{0.899296in}}%
\pgfpathlineto{\pgfqpoint{2.180264in}{0.895920in}}%
\pgfpathlineto{\pgfqpoint{2.173994in}{0.895920in}}%
\pgfpathlineto{\pgfqpoint{2.173994in}{0.895920in}}%
\pgfusepath{stroke}%
\end{pgfscope}%
\begin{pgfscope}%
\pgfsetrectcap%
\pgfsetmiterjoin%
\pgfsetlinewidth{0.803000pt}%
\definecolor{currentstroke}{rgb}{0.000000,0.000000,0.000000}%
\pgfsetstrokecolor{currentstroke}%
\pgfsetdash{}{0pt}%
\pgfpathmoveto{\pgfqpoint{0.888750in}{0.419100in}}%
\pgfpathlineto{\pgfqpoint{0.888750in}{3.352800in}}%
\pgfusepath{stroke}%
\end{pgfscope}%
\begin{pgfscope}%
\pgfsetrectcap%
\pgfsetmiterjoin%
\pgfsetlinewidth{0.803000pt}%
\definecolor{currentstroke}{rgb}{0.000000,0.000000,0.000000}%
\pgfsetstrokecolor{currentstroke}%
\pgfsetdash{}{0pt}%
\pgfpathmoveto{\pgfqpoint{3.393409in}{0.419100in}}%
\pgfpathlineto{\pgfqpoint{3.393409in}{3.352800in}}%
\pgfusepath{stroke}%
\end{pgfscope}%
\begin{pgfscope}%
\pgfsetrectcap%
\pgfsetmiterjoin%
\pgfsetlinewidth{0.803000pt}%
\definecolor{currentstroke}{rgb}{0.000000,0.000000,0.000000}%
\pgfsetstrokecolor{currentstroke}%
\pgfsetdash{}{0pt}%
\pgfpathmoveto{\pgfqpoint{0.888750in}{0.419100in}}%
\pgfpathlineto{\pgfqpoint{3.393409in}{0.419100in}}%
\pgfusepath{stroke}%
\end{pgfscope}%
\begin{pgfscope}%
\pgfsetrectcap%
\pgfsetmiterjoin%
\pgfsetlinewidth{0.803000pt}%
\definecolor{currentstroke}{rgb}{0.000000,0.000000,0.000000}%
\pgfsetstrokecolor{currentstroke}%
\pgfsetdash{}{0pt}%
\pgfpathmoveto{\pgfqpoint{0.888750in}{3.352800in}}%
\pgfpathlineto{\pgfqpoint{3.393409in}{3.352800in}}%
\pgfusepath{stroke}%
\end{pgfscope}%
\begin{pgfscope}%
\pgfsetbuttcap%
\pgfsetmiterjoin%
\definecolor{currentfill}{rgb}{1.000000,1.000000,1.000000}%
\pgfsetfillcolor{currentfill}%
\pgfsetlinewidth{0.000000pt}%
\definecolor{currentstroke}{rgb}{0.000000,0.000000,0.000000}%
\pgfsetstrokecolor{currentstroke}%
\pgfsetstrokeopacity{0.000000}%
\pgfsetdash{}{0pt}%
\pgfpathmoveto{\pgfqpoint{3.894341in}{0.419100in}}%
\pgfpathlineto{\pgfqpoint{6.399000in}{0.419100in}}%
\pgfpathlineto{\pgfqpoint{6.399000in}{3.352800in}}%
\pgfpathlineto{\pgfqpoint{3.894341in}{3.352800in}}%
\pgfpathclose%
\pgfusepath{fill}%
\end{pgfscope}%
\begin{pgfscope}%
\pgfsetbuttcap%
\pgfsetmiterjoin%
\definecolor{currentfill}{rgb}{0.950000,0.950000,0.950000}%
\pgfsetfillcolor{currentfill}%
\pgfsetfillopacity{0.500000}%
\pgfsetlinewidth{1.003750pt}%
\definecolor{currentstroke}{rgb}{0.950000,0.950000,0.950000}%
\pgfsetstrokecolor{currentstroke}%
\pgfsetstrokeopacity{0.500000}%
\pgfsetdash{}{0pt}%
\pgfpathmoveto{\pgfqpoint{4.185884in}{1.085659in}}%
\pgfpathlineto{\pgfqpoint{5.054537in}{1.227451in}}%
\pgfpathlineto{\pgfqpoint{5.053763in}{2.764539in}}%
\pgfpathlineto{\pgfqpoint{4.179260in}{2.750819in}}%
\pgfusepath{stroke,fill}%
\end{pgfscope}%
\begin{pgfscope}%
\pgfsetbuttcap%
\pgfsetmiterjoin%
\definecolor{currentfill}{rgb}{0.900000,0.900000,0.900000}%
\pgfsetfillcolor{currentfill}%
\pgfsetfillopacity{0.500000}%
\pgfsetlinewidth{1.003750pt}%
\definecolor{currentstroke}{rgb}{0.900000,0.900000,0.900000}%
\pgfsetstrokecolor{currentstroke}%
\pgfsetstrokeopacity{0.500000}%
\pgfsetdash{}{0pt}%
\pgfpathmoveto{\pgfqpoint{5.054537in}{1.227451in}}%
\pgfpathlineto{\pgfqpoint{6.155531in}{1.121999in}}%
\pgfpathlineto{\pgfqpoint{6.161895in}{2.754337in}}%
\pgfpathlineto{\pgfqpoint{5.053763in}{2.764539in}}%
\pgfusepath{stroke,fill}%
\end{pgfscope}%
\begin{pgfscope}%
\pgfsetbuttcap%
\pgfsetmiterjoin%
\definecolor{currentfill}{rgb}{0.925000,0.925000,0.925000}%
\pgfsetfillcolor{currentfill}%
\pgfsetfillopacity{0.500000}%
\pgfsetlinewidth{1.003750pt}%
\definecolor{currentstroke}{rgb}{0.925000,0.925000,0.925000}%
\pgfsetstrokecolor{currentstroke}%
\pgfsetstrokeopacity{0.500000}%
\pgfsetdash{}{0pt}%
\pgfpathmoveto{\pgfqpoint{4.185884in}{1.085659in}}%
\pgfpathlineto{\pgfqpoint{5.326219in}{0.961246in}}%
\pgfpathlineto{\pgfqpoint{6.155531in}{1.121999in}}%
\pgfpathlineto{\pgfqpoint{5.054537in}{1.227451in}}%
\pgfusepath{stroke,fill}%
\end{pgfscope}%
\begin{pgfscope}%
\pgfsetrectcap%
\pgfsetroundjoin%
\pgfsetlinewidth{0.803000pt}%
\definecolor{currentstroke}{rgb}{0.000000,0.000000,0.000000}%
\pgfsetstrokecolor{currentstroke}%
\pgfsetdash{}{0pt}%
\pgfpathmoveto{\pgfqpoint{4.185884in}{1.085659in}}%
\pgfpathlineto{\pgfqpoint{5.326219in}{0.961246in}}%
\pgfusepath{stroke}%
\end{pgfscope}%
\begin{pgfscope}%
\pgftext[x=4.507074in,y=0.557532in,,]{\sffamily\fontsize{10.000000}{12.000000}\selectfont x}%
\end{pgfscope}%
\begin{pgfscope}%
\pgfsetbuttcap%
\pgfsetroundjoin%
\pgfsetlinewidth{0.803000pt}%
\definecolor{currentstroke}{rgb}{0.690196,0.690196,0.690196}%
\pgfsetstrokecolor{currentstroke}%
\pgfsetdash{}{0pt}%
\pgfpathmoveto{\pgfqpoint{4.412229in}{1.060964in}}%
\pgfpathlineto{\pgfqpoint{5.273982in}{1.206433in}}%
\pgfpathlineto{\pgfqpoint{5.274563in}{2.762506in}}%
\pgfusepath{stroke}%
\end{pgfscope}%
\begin{pgfscope}%
\pgfsetbuttcap%
\pgfsetroundjoin%
\pgfsetlinewidth{0.803000pt}%
\definecolor{currentstroke}{rgb}{0.690196,0.690196,0.690196}%
\pgfsetstrokecolor{currentstroke}%
\pgfsetdash{}{0pt}%
\pgfpathmoveto{\pgfqpoint{4.737428in}{1.025484in}}%
\pgfpathlineto{\pgfqpoint{5.588479in}{1.176311in}}%
\pgfpathlineto{\pgfqpoint{5.591061in}{2.759592in}}%
\pgfusepath{stroke}%
\end{pgfscope}%
\begin{pgfscope}%
\pgfsetbuttcap%
\pgfsetroundjoin%
\pgfsetlinewidth{0.803000pt}%
\definecolor{currentstroke}{rgb}{0.690196,0.690196,0.690196}%
\pgfsetstrokecolor{currentstroke}%
\pgfsetdash{}{0pt}%
\pgfpathmoveto{\pgfqpoint{5.075228in}{0.988629in}}%
\pgfpathlineto{\pgfqpoint{5.914177in}{1.145116in}}%
\pgfpathlineto{\pgfqpoint{5.918904in}{2.756574in}}%
\pgfusepath{stroke}%
\end{pgfscope}%
\begin{pgfscope}%
\pgfsetrectcap%
\pgfsetroundjoin%
\pgfsetlinewidth{0.803000pt}%
\definecolor{currentstroke}{rgb}{0.000000,0.000000,0.000000}%
\pgfsetstrokecolor{currentstroke}%
\pgfsetdash{}{0pt}%
\pgfpathmoveto{\pgfqpoint{4.419700in}{1.062226in}}%
\pgfpathlineto{\pgfqpoint{4.397255in}{1.058437in}}%
\pgfusepath{stroke}%
\end{pgfscope}%
\begin{pgfscope}%
\pgftext[x=4.306482in,y=0.852307in,,top]{\sffamily\fontsize{10.000000}{12.000000}\selectfont −2}%
\end{pgfscope}%
\begin{pgfscope}%
\pgfsetrectcap%
\pgfsetroundjoin%
\pgfsetlinewidth{0.803000pt}%
\definecolor{currentstroke}{rgb}{0.000000,0.000000,0.000000}%
\pgfsetstrokecolor{currentstroke}%
\pgfsetdash{}{0pt}%
\pgfpathmoveto{\pgfqpoint{4.744818in}{1.026794in}}%
\pgfpathlineto{\pgfqpoint{4.722619in}{1.022860in}}%
\pgfusepath{stroke}%
\end{pgfscope}%
\begin{pgfscope}%
\pgftext[x=4.632588in,y=0.812506in,,top]{\sffamily\fontsize{10.000000}{12.000000}\selectfont 0}%
\end{pgfscope}%
\begin{pgfscope}%
\pgfsetrectcap%
\pgfsetroundjoin%
\pgfsetlinewidth{0.803000pt}%
\definecolor{currentstroke}{rgb}{0.000000,0.000000,0.000000}%
\pgfsetstrokecolor{currentstroke}%
\pgfsetdash{}{0pt}%
\pgfpathmoveto{\pgfqpoint{5.082523in}{0.989990in}}%
\pgfpathlineto{\pgfqpoint{5.060606in}{0.985902in}}%
\pgfusepath{stroke}%
\end{pgfscope}%
\begin{pgfscope}%
\pgftext[x=4.971444in,y=0.771148in,,top]{\sffamily\fontsize{10.000000}{12.000000}\selectfont 2}%
\end{pgfscope}%
\begin{pgfscope}%
\pgfsetrectcap%
\pgfsetroundjoin%
\pgfsetlinewidth{0.803000pt}%
\definecolor{currentstroke}{rgb}{0.000000,0.000000,0.000000}%
\pgfsetstrokecolor{currentstroke}%
\pgfsetdash{}{0pt}%
\pgfpathmoveto{\pgfqpoint{6.155531in}{1.121999in}}%
\pgfpathlineto{\pgfqpoint{5.326219in}{0.961246in}}%
\pgfusepath{stroke}%
\end{pgfscope}%
\begin{pgfscope}%
\pgftext[x=6.054378in,y=0.594382in,,]{\sffamily\fontsize{10.000000}{12.000000}\selectfont y}%
\end{pgfscope}%
\begin{pgfscope}%
\pgfsetbuttcap%
\pgfsetroundjoin%
\pgfsetlinewidth{0.803000pt}%
\definecolor{currentstroke}{rgb}{0.690196,0.690196,0.690196}%
\pgfsetstrokecolor{currentstroke}%
\pgfsetdash{}{0pt}%
\pgfpathmoveto{\pgfqpoint{4.354591in}{2.753570in}}%
\pgfpathlineto{\pgfqpoint{4.359971in}{1.114076in}}%
\pgfpathlineto{\pgfqpoint{5.493111in}{0.993596in}}%
\pgfusepath{stroke}%
\end{pgfscope}%
\begin{pgfscope}%
\pgfsetbuttcap%
\pgfsetroundjoin%
\pgfsetlinewidth{0.803000pt}%
\definecolor{currentstroke}{rgb}{0.690196,0.690196,0.690196}%
\pgfsetstrokecolor{currentstroke}%
\pgfsetdash{}{0pt}%
\pgfpathmoveto{\pgfqpoint{4.623002in}{2.757781in}}%
\pgfpathlineto{\pgfqpoint{4.626547in}{1.157589in}}%
\pgfpathlineto{\pgfqpoint{5.747998in}{1.043003in}}%
\pgfusepath{stroke}%
\end{pgfscope}%
\begin{pgfscope}%
\pgfsetbuttcap%
\pgfsetroundjoin%
\pgfsetlinewidth{0.803000pt}%
\definecolor{currentstroke}{rgb}{0.690196,0.690196,0.690196}%
\pgfsetstrokecolor{currentstroke}%
\pgfsetdash{}{0pt}%
\pgfpathmoveto{\pgfqpoint{4.878759in}{2.761793in}}%
\pgfpathlineto{\pgfqpoint{4.880632in}{1.199064in}}%
\pgfpathlineto{\pgfqpoint{5.990189in}{1.089949in}}%
\pgfusepath{stroke}%
\end{pgfscope}%
\begin{pgfscope}%
\pgfsetrectcap%
\pgfsetroundjoin%
\pgfsetlinewidth{0.803000pt}%
\definecolor{currentstroke}{rgb}{0.000000,0.000000,0.000000}%
\pgfsetstrokecolor{currentstroke}%
\pgfsetdash{}{0pt}%
\pgfpathmoveto{\pgfqpoint{5.483449in}{0.994623in}}%
\pgfpathlineto{\pgfqpoint{5.512466in}{0.991538in}}%
\pgfusepath{stroke}%
\end{pgfscope}%
\begin{pgfscope}%
\pgftext[x=5.630068in,y=0.781883in,,top]{\sffamily\fontsize{10.000000}{12.000000}\selectfont −2}%
\end{pgfscope}%
\begin{pgfscope}%
\pgfsetrectcap%
\pgfsetroundjoin%
\pgfsetlinewidth{0.803000pt}%
\definecolor{currentstroke}{rgb}{0.000000,0.000000,0.000000}%
\pgfsetstrokecolor{currentstroke}%
\pgfsetdash{}{0pt}%
\pgfpathmoveto{\pgfqpoint{5.738451in}{1.043979in}}%
\pgfpathlineto{\pgfqpoint{5.767123in}{1.041049in}}%
\pgfusepath{stroke}%
\end{pgfscope}%
\begin{pgfscope}%
\pgftext[x=5.883111in,y=0.837058in,,top]{\sffamily\fontsize{10.000000}{12.000000}\selectfont 0}%
\end{pgfscope}%
\begin{pgfscope}%
\pgfsetrectcap%
\pgfsetroundjoin%
\pgfsetlinewidth{0.803000pt}%
\definecolor{currentstroke}{rgb}{0.000000,0.000000,0.000000}%
\pgfsetstrokecolor{currentstroke}%
\pgfsetdash{}{0pt}%
\pgfpathmoveto{\pgfqpoint{5.980758in}{1.090877in}}%
\pgfpathlineto{\pgfqpoint{6.009082in}{1.088092in}}%
\pgfusepath{stroke}%
\end{pgfscope}%
\begin{pgfscope}%
\pgftext[x=6.123469in,y=0.889468in,,top]{\sffamily\fontsize{10.000000}{12.000000}\selectfont 2}%
\end{pgfscope}%
\begin{pgfscope}%
\pgfsetrectcap%
\pgfsetroundjoin%
\pgfsetlinewidth{0.803000pt}%
\definecolor{currentstroke}{rgb}{0.000000,0.000000,0.000000}%
\pgfsetstrokecolor{currentstroke}%
\pgfsetdash{}{0pt}%
\pgfpathmoveto{\pgfqpoint{6.155531in}{1.121999in}}%
\pgfpathlineto{\pgfqpoint{6.161895in}{2.754337in}}%
\pgfusepath{stroke}%
\end{pgfscope}%
\begin{pgfscope}%
\pgftext[x=6.667938in,y=1.506877in,left,base,rotate=89.776604]{\sffamily\fontsize{10.000000}{12.000000}\selectfont z = net(x, y)}%
\end{pgfscope}%
\begin{pgfscope}%
\pgfsetbuttcap%
\pgfsetroundjoin%
\pgfsetlinewidth{0.803000pt}%
\definecolor{currentstroke}{rgb}{0.690196,0.690196,0.690196}%
\pgfsetstrokecolor{currentstroke}%
\pgfsetdash{}{0pt}%
\pgfpathmoveto{\pgfqpoint{6.155703in}{1.166004in}}%
\pgfpathlineto{\pgfqpoint{5.054516in}{1.268904in}}%
\pgfpathlineto{\pgfqpoint{4.185706in}{1.130543in}}%
\pgfusepath{stroke}%
\end{pgfscope}%
\begin{pgfscope}%
\pgfsetbuttcap%
\pgfsetroundjoin%
\pgfsetlinewidth{0.803000pt}%
\definecolor{currentstroke}{rgb}{0.690196,0.690196,0.690196}%
\pgfsetstrokecolor{currentstroke}%
\pgfsetdash{}{0pt}%
\pgfpathmoveto{\pgfqpoint{6.156909in}{1.475504in}}%
\pgfpathlineto{\pgfqpoint{5.054369in}{1.560427in}}%
\pgfpathlineto{\pgfqpoint{4.184450in}{1.446235in}}%
\pgfusepath{stroke}%
\end{pgfscope}%
\begin{pgfscope}%
\pgfsetbuttcap%
\pgfsetroundjoin%
\pgfsetlinewidth{0.803000pt}%
\definecolor{currentstroke}{rgb}{0.690196,0.690196,0.690196}%
\pgfsetstrokecolor{currentstroke}%
\pgfsetdash{}{0pt}%
\pgfpathmoveto{\pgfqpoint{6.158119in}{1.785770in}}%
\pgfpathlineto{\pgfqpoint{5.054222in}{1.852631in}}%
\pgfpathlineto{\pgfqpoint{4.183191in}{1.762724in}}%
\pgfusepath{stroke}%
\end{pgfscope}%
\begin{pgfscope}%
\pgfsetbuttcap%
\pgfsetroundjoin%
\pgfsetlinewidth{0.803000pt}%
\definecolor{currentstroke}{rgb}{0.690196,0.690196,0.690196}%
\pgfsetstrokecolor{currentstroke}%
\pgfsetdash{}{0pt}%
\pgfpathmoveto{\pgfqpoint{6.159332in}{2.096806in}}%
\pgfpathlineto{\pgfqpoint{5.054075in}{2.145517in}}%
\pgfpathlineto{\pgfqpoint{4.181929in}{2.080014in}}%
\pgfusepath{stroke}%
\end{pgfscope}%
\begin{pgfscope}%
\pgfsetbuttcap%
\pgfsetroundjoin%
\pgfsetlinewidth{0.803000pt}%
\definecolor{currentstroke}{rgb}{0.690196,0.690196,0.690196}%
\pgfsetstrokecolor{currentstroke}%
\pgfsetdash{}{0pt}%
\pgfpathmoveto{\pgfqpoint{6.160548in}{2.408614in}}%
\pgfpathlineto{\pgfqpoint{5.053927in}{2.439088in}}%
\pgfpathlineto{\pgfqpoint{4.180663in}{2.398109in}}%
\pgfusepath{stroke}%
\end{pgfscope}%
\begin{pgfscope}%
\pgfsetbuttcap%
\pgfsetroundjoin%
\pgfsetlinewidth{0.803000pt}%
\definecolor{currentstroke}{rgb}{0.690196,0.690196,0.690196}%
\pgfsetstrokecolor{currentstroke}%
\pgfsetdash{}{0pt}%
\pgfpathmoveto{\pgfqpoint{6.161766in}{2.721198in}}%
\pgfpathlineto{\pgfqpoint{5.053779in}{2.733346in}}%
\pgfpathlineto{\pgfqpoint{4.179395in}{2.717010in}}%
\pgfusepath{stroke}%
\end{pgfscope}%
\begin{pgfscope}%
\pgfsetrectcap%
\pgfsetroundjoin%
\pgfsetlinewidth{0.803000pt}%
\definecolor{currentstroke}{rgb}{0.000000,0.000000,0.000000}%
\pgfsetstrokecolor{currentstroke}%
\pgfsetdash{}{0pt}%
\pgfpathmoveto{\pgfqpoint{6.146351in}{1.166878in}}%
\pgfpathlineto{\pgfqpoint{6.174433in}{1.164254in}}%
\pgfusepath{stroke}%
\end{pgfscope}%
\begin{pgfscope}%
\pgftext[x=6.373010in,y=1.169916in,,top]{\sffamily\fontsize{10.000000}{12.000000}\selectfont −7.5}%
\end{pgfscope}%
\begin{pgfscope}%
\pgfsetrectcap%
\pgfsetroundjoin%
\pgfsetlinewidth{0.803000pt}%
\definecolor{currentstroke}{rgb}{0.000000,0.000000,0.000000}%
\pgfsetstrokecolor{currentstroke}%
\pgfsetdash{}{0pt}%
\pgfpathmoveto{\pgfqpoint{6.147546in}{1.476225in}}%
\pgfpathlineto{\pgfqpoint{6.175664in}{1.474059in}}%
\pgfusepath{stroke}%
\end{pgfscope}%
\begin{pgfscope}%
\pgftext[x=6.374483in,y=1.478732in,,top]{\sffamily\fontsize{10.000000}{12.000000}\selectfont −5.0}%
\end{pgfscope}%
\begin{pgfscope}%
\pgfsetrectcap%
\pgfsetroundjoin%
\pgfsetlinewidth{0.803000pt}%
\definecolor{currentstroke}{rgb}{0.000000,0.000000,0.000000}%
\pgfsetstrokecolor{currentstroke}%
\pgfsetdash{}{0pt}%
\pgfpathmoveto{\pgfqpoint{6.148743in}{1.786338in}}%
\pgfpathlineto{\pgfqpoint{6.176898in}{1.784632in}}%
\pgfusepath{stroke}%
\end{pgfscope}%
\begin{pgfscope}%
\pgftext[x=6.375959in,y=1.788312in,,top]{\sffamily\fontsize{10.000000}{12.000000}\selectfont −2.5}%
\end{pgfscope}%
\begin{pgfscope}%
\pgfsetrectcap%
\pgfsetroundjoin%
\pgfsetlinewidth{0.803000pt}%
\definecolor{currentstroke}{rgb}{0.000000,0.000000,0.000000}%
\pgfsetstrokecolor{currentstroke}%
\pgfsetdash{}{0pt}%
\pgfpathmoveto{\pgfqpoint{6.149944in}{2.097219in}}%
\pgfpathlineto{\pgfqpoint{6.178135in}{2.095977in}}%
\pgfusepath{stroke}%
\end{pgfscope}%
\begin{pgfscope}%
\pgftext[x=6.377438in,y=2.098658in,,top]{\sffamily\fontsize{10.000000}{12.000000}\selectfont 0.0}%
\end{pgfscope}%
\begin{pgfscope}%
\pgfsetrectcap%
\pgfsetroundjoin%
\pgfsetlinewidth{0.803000pt}%
\definecolor{currentstroke}{rgb}{0.000000,0.000000,0.000000}%
\pgfsetstrokecolor{currentstroke}%
\pgfsetdash{}{0pt}%
\pgfpathmoveto{\pgfqpoint{6.151147in}{2.408873in}}%
\pgfpathlineto{\pgfqpoint{6.179376in}{2.408096in}}%
\pgfusepath{stroke}%
\end{pgfscope}%
\begin{pgfscope}%
\pgftext[x=6.378922in,y=2.409773in,,top]{\sffamily\fontsize{10.000000}{12.000000}\selectfont 2.5}%
\end{pgfscope}%
\begin{pgfscope}%
\pgfsetrectcap%
\pgfsetroundjoin%
\pgfsetlinewidth{0.803000pt}%
\definecolor{currentstroke}{rgb}{0.000000,0.000000,0.000000}%
\pgfsetstrokecolor{currentstroke}%
\pgfsetdash{}{0pt}%
\pgfpathmoveto{\pgfqpoint{6.152354in}{2.721302in}}%
\pgfpathlineto{\pgfqpoint{6.180619in}{2.720992in}}%
\pgfusepath{stroke}%
\end{pgfscope}%
\begin{pgfscope}%
\pgftext[x=6.380409in,y=2.721660in,,top]{\sffamily\fontsize{10.000000}{12.000000}\selectfont 5.0}%
\end{pgfscope}%
\begin{pgfscope}%
\pgfpathrectangle{\pgfqpoint{3.894341in}{0.419100in}}{\pgfqpoint{2.504659in}{2.933700in}} %
\pgfusepath{clip}%
\pgfsetbuttcap%
\pgfsetroundjoin%
\definecolor{currentfill}{rgb}{1.000000,0.549020,0.000000}%
\pgfsetfillcolor{currentfill}%
\pgfsetfillopacity{0.596863}%
\pgfsetlinewidth{1.003750pt}%
\definecolor{currentstroke}{rgb}{1.000000,0.549020,0.000000}%
\pgfsetstrokecolor{currentstroke}%
\pgfsetstrokeopacity{0.596863}%
\pgfsetdash{}{0pt}%
\pgfpathmoveto{\pgfqpoint{5.315490in}{1.172688in}}%
\pgfpathcurveto{\pgfqpoint{5.323726in}{1.172688in}}{\pgfqpoint{5.331626in}{1.175960in}}{\pgfqpoint{5.337450in}{1.181784in}}%
\pgfpathcurveto{\pgfqpoint{5.343274in}{1.187608in}}{\pgfqpoint{5.346546in}{1.195508in}}{\pgfqpoint{5.346546in}{1.203744in}}%
\pgfpathcurveto{\pgfqpoint{5.346546in}{1.211981in}}{\pgfqpoint{5.343274in}{1.219881in}}{\pgfqpoint{5.337450in}{1.225705in}}%
\pgfpathcurveto{\pgfqpoint{5.331626in}{1.231528in}}{\pgfqpoint{5.323726in}{1.234801in}}{\pgfqpoint{5.315490in}{1.234801in}}%
\pgfpathcurveto{\pgfqpoint{5.307254in}{1.234801in}}{\pgfqpoint{5.299354in}{1.231528in}}{\pgfqpoint{5.293530in}{1.225705in}}%
\pgfpathcurveto{\pgfqpoint{5.287706in}{1.219881in}}{\pgfqpoint{5.284433in}{1.211981in}}{\pgfqpoint{5.284433in}{1.203744in}}%
\pgfpathcurveto{\pgfqpoint{5.284433in}{1.195508in}}{\pgfqpoint{5.287706in}{1.187608in}}{\pgfqpoint{5.293530in}{1.181784in}}%
\pgfpathcurveto{\pgfqpoint{5.299354in}{1.175960in}}{\pgfqpoint{5.307254in}{1.172688in}}{\pgfqpoint{5.315490in}{1.172688in}}%
\pgfpathclose%
\pgfusepath{stroke,fill}%
\end{pgfscope}%
\begin{pgfscope}%
\pgfpathrectangle{\pgfqpoint{3.894341in}{0.419100in}}{\pgfqpoint{2.504659in}{2.933700in}} %
\pgfusepath{clip}%
\pgfsetbuttcap%
\pgfsetroundjoin%
\definecolor{currentfill}{rgb}{1.000000,0.549020,0.000000}%
\pgfsetfillcolor{currentfill}%
\pgfsetfillopacity{0.474896}%
\pgfsetlinewidth{1.003750pt}%
\definecolor{currentstroke}{rgb}{1.000000,0.549020,0.000000}%
\pgfsetstrokecolor{currentstroke}%
\pgfsetstrokeopacity{0.474896}%
\pgfsetdash{}{0pt}%
\pgfpathmoveto{\pgfqpoint{5.093020in}{1.326609in}}%
\pgfpathcurveto{\pgfqpoint{5.101256in}{1.326609in}}{\pgfqpoint{5.109156in}{1.329881in}}{\pgfqpoint{5.114980in}{1.335705in}}%
\pgfpathcurveto{\pgfqpoint{5.120804in}{1.341529in}}{\pgfqpoint{5.124077in}{1.349429in}}{\pgfqpoint{5.124077in}{1.357665in}}%
\pgfpathcurveto{\pgfqpoint{5.124077in}{1.365902in}}{\pgfqpoint{5.120804in}{1.373802in}}{\pgfqpoint{5.114980in}{1.379626in}}%
\pgfpathcurveto{\pgfqpoint{5.109156in}{1.385449in}}{\pgfqpoint{5.101256in}{1.388722in}}{\pgfqpoint{5.093020in}{1.388722in}}%
\pgfpathcurveto{\pgfqpoint{5.084784in}{1.388722in}}{\pgfqpoint{5.076884in}{1.385449in}}{\pgfqpoint{5.071060in}{1.379626in}}%
\pgfpathcurveto{\pgfqpoint{5.065236in}{1.373802in}}{\pgfqpoint{5.061964in}{1.365902in}}{\pgfqpoint{5.061964in}{1.357665in}}%
\pgfpathcurveto{\pgfqpoint{5.061964in}{1.349429in}}{\pgfqpoint{5.065236in}{1.341529in}}{\pgfqpoint{5.071060in}{1.335705in}}%
\pgfpathcurveto{\pgfqpoint{5.076884in}{1.329881in}}{\pgfqpoint{5.084784in}{1.326609in}}{\pgfqpoint{5.093020in}{1.326609in}}%
\pgfpathclose%
\pgfusepath{stroke,fill}%
\end{pgfscope}%
\begin{pgfscope}%
\pgfpathrectangle{\pgfqpoint{3.894341in}{0.419100in}}{\pgfqpoint{2.504659in}{2.933700in}} %
\pgfusepath{clip}%
\pgfsetbuttcap%
\pgfsetroundjoin%
\definecolor{currentfill}{rgb}{1.000000,0.549020,0.000000}%
\pgfsetfillcolor{currentfill}%
\pgfsetfillopacity{0.796471}%
\pgfsetlinewidth{1.003750pt}%
\definecolor{currentstroke}{rgb}{1.000000,0.549020,0.000000}%
\pgfsetstrokecolor{currentstroke}%
\pgfsetstrokeopacity{0.796471}%
\pgfsetdash{}{0pt}%
\pgfpathmoveto{\pgfqpoint{4.920797in}{1.765222in}}%
\pgfpathcurveto{\pgfqpoint{4.929033in}{1.765222in}}{\pgfqpoint{4.936933in}{1.768494in}}{\pgfqpoint{4.942757in}{1.774318in}}%
\pgfpathcurveto{\pgfqpoint{4.948581in}{1.780142in}}{\pgfqpoint{4.951853in}{1.788042in}}{\pgfqpoint{4.951853in}{1.796278in}}%
\pgfpathcurveto{\pgfqpoint{4.951853in}{1.804515in}}{\pgfqpoint{4.948581in}{1.812415in}}{\pgfqpoint{4.942757in}{1.818239in}}%
\pgfpathcurveto{\pgfqpoint{4.936933in}{1.824062in}}{\pgfqpoint{4.929033in}{1.827335in}}{\pgfqpoint{4.920797in}{1.827335in}}%
\pgfpathcurveto{\pgfqpoint{4.912560in}{1.827335in}}{\pgfqpoint{4.904660in}{1.824062in}}{\pgfqpoint{4.898836in}{1.818239in}}%
\pgfpathcurveto{\pgfqpoint{4.893013in}{1.812415in}}{\pgfqpoint{4.889740in}{1.804515in}}{\pgfqpoint{4.889740in}{1.796278in}}%
\pgfpathcurveto{\pgfqpoint{4.889740in}{1.788042in}}{\pgfqpoint{4.893013in}{1.780142in}}{\pgfqpoint{4.898836in}{1.774318in}}%
\pgfpathcurveto{\pgfqpoint{4.904660in}{1.768494in}}{\pgfqpoint{4.912560in}{1.765222in}}{\pgfqpoint{4.920797in}{1.765222in}}%
\pgfpathclose%
\pgfusepath{stroke,fill}%
\end{pgfscope}%
\begin{pgfscope}%
\pgfpathrectangle{\pgfqpoint{3.894341in}{0.419100in}}{\pgfqpoint{2.504659in}{2.933700in}} %
\pgfusepath{clip}%
\pgfsetbuttcap%
\pgfsetroundjoin%
\definecolor{currentfill}{rgb}{1.000000,0.549020,0.000000}%
\pgfsetfillcolor{currentfill}%
\pgfsetfillopacity{0.484683}%
\pgfsetlinewidth{1.003750pt}%
\definecolor{currentstroke}{rgb}{1.000000,0.549020,0.000000}%
\pgfsetstrokecolor{currentstroke}%
\pgfsetstrokeopacity{0.484683}%
\pgfsetdash{}{0pt}%
\pgfpathmoveto{\pgfqpoint{5.142065in}{1.324759in}}%
\pgfpathcurveto{\pgfqpoint{5.150301in}{1.324759in}}{\pgfqpoint{5.158201in}{1.328032in}}{\pgfqpoint{5.164025in}{1.333856in}}%
\pgfpathcurveto{\pgfqpoint{5.169849in}{1.339680in}}{\pgfqpoint{5.173121in}{1.347580in}}{\pgfqpoint{5.173121in}{1.355816in}}%
\pgfpathcurveto{\pgfqpoint{5.173121in}{1.364052in}}{\pgfqpoint{5.169849in}{1.371952in}}{\pgfqpoint{5.164025in}{1.377776in}}%
\pgfpathcurveto{\pgfqpoint{5.158201in}{1.383600in}}{\pgfqpoint{5.150301in}{1.386872in}}{\pgfqpoint{5.142065in}{1.386872in}}%
\pgfpathcurveto{\pgfqpoint{5.133828in}{1.386872in}}{\pgfqpoint{5.125928in}{1.383600in}}{\pgfqpoint{5.120104in}{1.377776in}}%
\pgfpathcurveto{\pgfqpoint{5.114281in}{1.371952in}}{\pgfqpoint{5.111008in}{1.364052in}}{\pgfqpoint{5.111008in}{1.355816in}}%
\pgfpathcurveto{\pgfqpoint{5.111008in}{1.347580in}}{\pgfqpoint{5.114281in}{1.339680in}}{\pgfqpoint{5.120104in}{1.333856in}}%
\pgfpathcurveto{\pgfqpoint{5.125928in}{1.328032in}}{\pgfqpoint{5.133828in}{1.324759in}}{\pgfqpoint{5.142065in}{1.324759in}}%
\pgfpathclose%
\pgfusepath{stroke,fill}%
\end{pgfscope}%
\begin{pgfscope}%
\pgfpathrectangle{\pgfqpoint{3.894341in}{0.419100in}}{\pgfqpoint{2.504659in}{2.933700in}} %
\pgfusepath{clip}%
\pgfsetbuttcap%
\pgfsetroundjoin%
\definecolor{currentfill}{rgb}{1.000000,0.549020,0.000000}%
\pgfsetfillcolor{currentfill}%
\pgfsetfillopacity{0.587693}%
\pgfsetlinewidth{1.003750pt}%
\definecolor{currentstroke}{rgb}{1.000000,0.549020,0.000000}%
\pgfsetstrokecolor{currentstroke}%
\pgfsetstrokeopacity{0.587693}%
\pgfsetdash{}{0pt}%
\pgfpathmoveto{\pgfqpoint{5.040634in}{1.391594in}}%
\pgfpathcurveto{\pgfqpoint{5.048870in}{1.391594in}}{\pgfqpoint{5.056770in}{1.394867in}}{\pgfqpoint{5.062594in}{1.400691in}}%
\pgfpathcurveto{\pgfqpoint{5.068418in}{1.406514in}}{\pgfqpoint{5.071691in}{1.414415in}}{\pgfqpoint{5.071691in}{1.422651in}}%
\pgfpathcurveto{\pgfqpoint{5.071691in}{1.430887in}}{\pgfqpoint{5.068418in}{1.438787in}}{\pgfqpoint{5.062594in}{1.444611in}}%
\pgfpathcurveto{\pgfqpoint{5.056770in}{1.450435in}}{\pgfqpoint{5.048870in}{1.453707in}}{\pgfqpoint{5.040634in}{1.453707in}}%
\pgfpathcurveto{\pgfqpoint{5.032398in}{1.453707in}}{\pgfqpoint{5.024498in}{1.450435in}}{\pgfqpoint{5.018674in}{1.444611in}}%
\pgfpathcurveto{\pgfqpoint{5.012850in}{1.438787in}}{\pgfqpoint{5.009578in}{1.430887in}}{\pgfqpoint{5.009578in}{1.422651in}}%
\pgfpathcurveto{\pgfqpoint{5.009578in}{1.414415in}}{\pgfqpoint{5.012850in}{1.406514in}}{\pgfqpoint{5.018674in}{1.400691in}}%
\pgfpathcurveto{\pgfqpoint{5.024498in}{1.394867in}}{\pgfqpoint{5.032398in}{1.391594in}}{\pgfqpoint{5.040634in}{1.391594in}}%
\pgfpathclose%
\pgfusepath{stroke,fill}%
\end{pgfscope}%
\begin{pgfscope}%
\pgfpathrectangle{\pgfqpoint{3.894341in}{0.419100in}}{\pgfqpoint{2.504659in}{2.933700in}} %
\pgfusepath{clip}%
\pgfsetbuttcap%
\pgfsetroundjoin%
\definecolor{currentfill}{rgb}{1.000000,0.549020,0.000000}%
\pgfsetfillcolor{currentfill}%
\pgfsetfillopacity{0.502355}%
\pgfsetlinewidth{1.003750pt}%
\definecolor{currentstroke}{rgb}{1.000000,0.549020,0.000000}%
\pgfsetstrokecolor{currentstroke}%
\pgfsetstrokeopacity{0.502355}%
\pgfsetdash{}{0pt}%
\pgfpathmoveto{\pgfqpoint{5.654497in}{1.844530in}}%
\pgfpathcurveto{\pgfqpoint{5.662733in}{1.844530in}}{\pgfqpoint{5.670633in}{1.847802in}}{\pgfqpoint{5.676457in}{1.853626in}}%
\pgfpathcurveto{\pgfqpoint{5.682281in}{1.859450in}}{\pgfqpoint{5.685553in}{1.867350in}}{\pgfqpoint{5.685553in}{1.875586in}}%
\pgfpathcurveto{\pgfqpoint{5.685553in}{1.883823in}}{\pgfqpoint{5.682281in}{1.891723in}}{\pgfqpoint{5.676457in}{1.897547in}}%
\pgfpathcurveto{\pgfqpoint{5.670633in}{1.903370in}}{\pgfqpoint{5.662733in}{1.906643in}}{\pgfqpoint{5.654497in}{1.906643in}}%
\pgfpathcurveto{\pgfqpoint{5.646261in}{1.906643in}}{\pgfqpoint{5.638361in}{1.903370in}}{\pgfqpoint{5.632537in}{1.897547in}}%
\pgfpathcurveto{\pgfqpoint{5.626713in}{1.891723in}}{\pgfqpoint{5.623440in}{1.883823in}}{\pgfqpoint{5.623440in}{1.875586in}}%
\pgfpathcurveto{\pgfqpoint{5.623440in}{1.867350in}}{\pgfqpoint{5.626713in}{1.859450in}}{\pgfqpoint{5.632537in}{1.853626in}}%
\pgfpathcurveto{\pgfqpoint{5.638361in}{1.847802in}}{\pgfqpoint{5.646261in}{1.844530in}}{\pgfqpoint{5.654497in}{1.844530in}}%
\pgfpathclose%
\pgfusepath{stroke,fill}%
\end{pgfscope}%
\begin{pgfscope}%
\pgfpathrectangle{\pgfqpoint{3.894341in}{0.419100in}}{\pgfqpoint{2.504659in}{2.933700in}} %
\pgfusepath{clip}%
\pgfsetbuttcap%
\pgfsetroundjoin%
\definecolor{currentfill}{rgb}{1.000000,0.549020,0.000000}%
\pgfsetfillcolor{currentfill}%
\pgfsetfillopacity{0.779852}%
\pgfsetlinewidth{1.003750pt}%
\definecolor{currentstroke}{rgb}{1.000000,0.549020,0.000000}%
\pgfsetstrokecolor{currentstroke}%
\pgfsetstrokeopacity{0.779852}%
\pgfsetdash{}{0pt}%
\pgfpathmoveto{\pgfqpoint{4.934561in}{1.726259in}}%
\pgfpathcurveto{\pgfqpoint{4.942798in}{1.726259in}}{\pgfqpoint{4.950698in}{1.729531in}}{\pgfqpoint{4.956522in}{1.735355in}}%
\pgfpathcurveto{\pgfqpoint{4.962346in}{1.741179in}}{\pgfqpoint{4.965618in}{1.749079in}}{\pgfqpoint{4.965618in}{1.757315in}}%
\pgfpathcurveto{\pgfqpoint{4.965618in}{1.765552in}}{\pgfqpoint{4.962346in}{1.773452in}}{\pgfqpoint{4.956522in}{1.779276in}}%
\pgfpathcurveto{\pgfqpoint{4.950698in}{1.785100in}}{\pgfqpoint{4.942798in}{1.788372in}}{\pgfqpoint{4.934561in}{1.788372in}}%
\pgfpathcurveto{\pgfqpoint{4.926325in}{1.788372in}}{\pgfqpoint{4.918425in}{1.785100in}}{\pgfqpoint{4.912601in}{1.779276in}}%
\pgfpathcurveto{\pgfqpoint{4.906777in}{1.773452in}}{\pgfqpoint{4.903505in}{1.765552in}}{\pgfqpoint{4.903505in}{1.757315in}}%
\pgfpathcurveto{\pgfqpoint{4.903505in}{1.749079in}}{\pgfqpoint{4.906777in}{1.741179in}}{\pgfqpoint{4.912601in}{1.735355in}}%
\pgfpathcurveto{\pgfqpoint{4.918425in}{1.729531in}}{\pgfqpoint{4.926325in}{1.726259in}}{\pgfqpoint{4.934561in}{1.726259in}}%
\pgfpathclose%
\pgfusepath{stroke,fill}%
\end{pgfscope}%
\begin{pgfscope}%
\pgfpathrectangle{\pgfqpoint{3.894341in}{0.419100in}}{\pgfqpoint{2.504659in}{2.933700in}} %
\pgfusepath{clip}%
\pgfsetbuttcap%
\pgfsetroundjoin%
\definecolor{currentfill}{rgb}{1.000000,0.549020,0.000000}%
\pgfsetfillcolor{currentfill}%
\pgfsetfillopacity{0.635173}%
\pgfsetlinewidth{1.003750pt}%
\definecolor{currentstroke}{rgb}{1.000000,0.549020,0.000000}%
\pgfsetstrokecolor{currentstroke}%
\pgfsetstrokeopacity{0.635173}%
\pgfsetdash{}{0pt}%
\pgfpathmoveto{\pgfqpoint{5.250726in}{1.166775in}}%
\pgfpathcurveto{\pgfqpoint{5.258962in}{1.166775in}}{\pgfqpoint{5.266862in}{1.170048in}}{\pgfqpoint{5.272686in}{1.175872in}}%
\pgfpathcurveto{\pgfqpoint{5.278510in}{1.181696in}}{\pgfqpoint{5.281782in}{1.189596in}}{\pgfqpoint{5.281782in}{1.197832in}}%
\pgfpathcurveto{\pgfqpoint{5.281782in}{1.206068in}}{\pgfqpoint{5.278510in}{1.213968in}}{\pgfqpoint{5.272686in}{1.219792in}}%
\pgfpathcurveto{\pgfqpoint{5.266862in}{1.225616in}}{\pgfqpoint{5.258962in}{1.228888in}}{\pgfqpoint{5.250726in}{1.228888in}}%
\pgfpathcurveto{\pgfqpoint{5.242489in}{1.228888in}}{\pgfqpoint{5.234589in}{1.225616in}}{\pgfqpoint{5.228765in}{1.219792in}}%
\pgfpathcurveto{\pgfqpoint{5.222942in}{1.213968in}}{\pgfqpoint{5.219669in}{1.206068in}}{\pgfqpoint{5.219669in}{1.197832in}}%
\pgfpathcurveto{\pgfqpoint{5.219669in}{1.189596in}}{\pgfqpoint{5.222942in}{1.181696in}}{\pgfqpoint{5.228765in}{1.175872in}}%
\pgfpathcurveto{\pgfqpoint{5.234589in}{1.170048in}}{\pgfqpoint{5.242489in}{1.166775in}}{\pgfqpoint{5.250726in}{1.166775in}}%
\pgfpathclose%
\pgfusepath{stroke,fill}%
\end{pgfscope}%
\begin{pgfscope}%
\pgfpathrectangle{\pgfqpoint{3.894341in}{0.419100in}}{\pgfqpoint{2.504659in}{2.933700in}} %
\pgfusepath{clip}%
\pgfsetbuttcap%
\pgfsetroundjoin%
\definecolor{currentfill}{rgb}{1.000000,0.549020,0.000000}%
\pgfsetfillcolor{currentfill}%
\pgfsetfillopacity{0.612870}%
\pgfsetlinewidth{1.003750pt}%
\definecolor{currentstroke}{rgb}{1.000000,0.549020,0.000000}%
\pgfsetstrokecolor{currentstroke}%
\pgfsetstrokeopacity{0.612870}%
\pgfsetdash{}{0pt}%
\pgfpathmoveto{\pgfqpoint{5.113058in}{1.247053in}}%
\pgfpathcurveto{\pgfqpoint{5.121294in}{1.247053in}}{\pgfqpoint{5.129194in}{1.250325in}}{\pgfqpoint{5.135018in}{1.256149in}}%
\pgfpathcurveto{\pgfqpoint{5.140842in}{1.261973in}}{\pgfqpoint{5.144114in}{1.269873in}}{\pgfqpoint{5.144114in}{1.278109in}}%
\pgfpathcurveto{\pgfqpoint{5.144114in}{1.286346in}}{\pgfqpoint{5.140842in}{1.294246in}}{\pgfqpoint{5.135018in}{1.300070in}}%
\pgfpathcurveto{\pgfqpoint{5.129194in}{1.305894in}}{\pgfqpoint{5.121294in}{1.309166in}}{\pgfqpoint{5.113058in}{1.309166in}}%
\pgfpathcurveto{\pgfqpoint{5.104821in}{1.309166in}}{\pgfqpoint{5.096921in}{1.305894in}}{\pgfqpoint{5.091097in}{1.300070in}}%
\pgfpathcurveto{\pgfqpoint{5.085273in}{1.294246in}}{\pgfqpoint{5.082001in}{1.286346in}}{\pgfqpoint{5.082001in}{1.278109in}}%
\pgfpathcurveto{\pgfqpoint{5.082001in}{1.269873in}}{\pgfqpoint{5.085273in}{1.261973in}}{\pgfqpoint{5.091097in}{1.256149in}}%
\pgfpathcurveto{\pgfqpoint{5.096921in}{1.250325in}}{\pgfqpoint{5.104821in}{1.247053in}}{\pgfqpoint{5.113058in}{1.247053in}}%
\pgfpathclose%
\pgfusepath{stroke,fill}%
\end{pgfscope}%
\begin{pgfscope}%
\pgfpathrectangle{\pgfqpoint{3.894341in}{0.419100in}}{\pgfqpoint{2.504659in}{2.933700in}} %
\pgfusepath{clip}%
\pgfsetbuttcap%
\pgfsetroundjoin%
\definecolor{currentfill}{rgb}{1.000000,0.549020,0.000000}%
\pgfsetfillcolor{currentfill}%
\pgfsetfillopacity{0.842867}%
\pgfsetlinewidth{1.003750pt}%
\definecolor{currentstroke}{rgb}{1.000000,0.549020,0.000000}%
\pgfsetstrokecolor{currentstroke}%
\pgfsetstrokeopacity{0.842867}%
\pgfsetdash{}{0pt}%
\pgfpathmoveto{\pgfqpoint{5.069726in}{1.472847in}}%
\pgfpathcurveto{\pgfqpoint{5.077962in}{1.472847in}}{\pgfqpoint{5.085862in}{1.476119in}}{\pgfqpoint{5.091686in}{1.481943in}}%
\pgfpathcurveto{\pgfqpoint{5.097510in}{1.487767in}}{\pgfqpoint{5.100782in}{1.495667in}}{\pgfqpoint{5.100782in}{1.503904in}}%
\pgfpathcurveto{\pgfqpoint{5.100782in}{1.512140in}}{\pgfqpoint{5.097510in}{1.520040in}}{\pgfqpoint{5.091686in}{1.525864in}}%
\pgfpathcurveto{\pgfqpoint{5.085862in}{1.531688in}}{\pgfqpoint{5.077962in}{1.534960in}}{\pgfqpoint{5.069726in}{1.534960in}}%
\pgfpathcurveto{\pgfqpoint{5.061489in}{1.534960in}}{\pgfqpoint{5.053589in}{1.531688in}}{\pgfqpoint{5.047765in}{1.525864in}}%
\pgfpathcurveto{\pgfqpoint{5.041942in}{1.520040in}}{\pgfqpoint{5.038669in}{1.512140in}}{\pgfqpoint{5.038669in}{1.503904in}}%
\pgfpathcurveto{\pgfqpoint{5.038669in}{1.495667in}}{\pgfqpoint{5.041942in}{1.487767in}}{\pgfqpoint{5.047765in}{1.481943in}}%
\pgfpathcurveto{\pgfqpoint{5.053589in}{1.476119in}}{\pgfqpoint{5.061489in}{1.472847in}}{\pgfqpoint{5.069726in}{1.472847in}}%
\pgfpathclose%
\pgfusepath{stroke,fill}%
\end{pgfscope}%
\begin{pgfscope}%
\pgfpathrectangle{\pgfqpoint{3.894341in}{0.419100in}}{\pgfqpoint{2.504659in}{2.933700in}} %
\pgfusepath{clip}%
\pgfsetbuttcap%
\pgfsetroundjoin%
\definecolor{currentfill}{rgb}{1.000000,0.549020,0.000000}%
\pgfsetfillcolor{currentfill}%
\pgfsetfillopacity{0.480686}%
\pgfsetlinewidth{1.003750pt}%
\definecolor{currentstroke}{rgb}{1.000000,0.549020,0.000000}%
\pgfsetstrokecolor{currentstroke}%
\pgfsetstrokeopacity{0.480686}%
\pgfsetdash{}{0pt}%
\pgfpathmoveto{\pgfqpoint{5.076702in}{1.335210in}}%
\pgfpathcurveto{\pgfqpoint{5.084938in}{1.335210in}}{\pgfqpoint{5.092838in}{1.338483in}}{\pgfqpoint{5.098662in}{1.344306in}}%
\pgfpathcurveto{\pgfqpoint{5.104486in}{1.350130in}}{\pgfqpoint{5.107758in}{1.358030in}}{\pgfqpoint{5.107758in}{1.366267in}}%
\pgfpathcurveto{\pgfqpoint{5.107758in}{1.374503in}}{\pgfqpoint{5.104486in}{1.382403in}}{\pgfqpoint{5.098662in}{1.388227in}}%
\pgfpathcurveto{\pgfqpoint{5.092838in}{1.394051in}}{\pgfqpoint{5.084938in}{1.397323in}}{\pgfqpoint{5.076702in}{1.397323in}}%
\pgfpathcurveto{\pgfqpoint{5.068465in}{1.397323in}}{\pgfqpoint{5.060565in}{1.394051in}}{\pgfqpoint{5.054741in}{1.388227in}}%
\pgfpathcurveto{\pgfqpoint{5.048918in}{1.382403in}}{\pgfqpoint{5.045645in}{1.374503in}}{\pgfqpoint{5.045645in}{1.366267in}}%
\pgfpathcurveto{\pgfqpoint{5.045645in}{1.358030in}}{\pgfqpoint{5.048918in}{1.350130in}}{\pgfqpoint{5.054741in}{1.344306in}}%
\pgfpathcurveto{\pgfqpoint{5.060565in}{1.338483in}}{\pgfqpoint{5.068465in}{1.335210in}}{\pgfqpoint{5.076702in}{1.335210in}}%
\pgfpathclose%
\pgfusepath{stroke,fill}%
\end{pgfscope}%
\begin{pgfscope}%
\pgfpathrectangle{\pgfqpoint{3.894341in}{0.419100in}}{\pgfqpoint{2.504659in}{2.933700in}} %
\pgfusepath{clip}%
\pgfsetbuttcap%
\pgfsetroundjoin%
\definecolor{currentfill}{rgb}{1.000000,0.549020,0.000000}%
\pgfsetfillcolor{currentfill}%
\pgfsetfillopacity{0.801650}%
\pgfsetlinewidth{1.003750pt}%
\definecolor{currentstroke}{rgb}{1.000000,0.549020,0.000000}%
\pgfsetstrokecolor{currentstroke}%
\pgfsetstrokeopacity{0.801650}%
\pgfsetdash{}{0pt}%
\pgfpathmoveto{\pgfqpoint{5.045179in}{1.496547in}}%
\pgfpathcurveto{\pgfqpoint{5.053415in}{1.496547in}}{\pgfqpoint{5.061315in}{1.499819in}}{\pgfqpoint{5.067139in}{1.505643in}}%
\pgfpathcurveto{\pgfqpoint{5.072963in}{1.511467in}}{\pgfqpoint{5.076236in}{1.519367in}}{\pgfqpoint{5.076236in}{1.527603in}}%
\pgfpathcurveto{\pgfqpoint{5.076236in}{1.535839in}}{\pgfqpoint{5.072963in}{1.543739in}}{\pgfqpoint{5.067139in}{1.549563in}}%
\pgfpathcurveto{\pgfqpoint{5.061315in}{1.555387in}}{\pgfqpoint{5.053415in}{1.558660in}}{\pgfqpoint{5.045179in}{1.558660in}}%
\pgfpathcurveto{\pgfqpoint{5.036943in}{1.558660in}}{\pgfqpoint{5.029043in}{1.555387in}}{\pgfqpoint{5.023219in}{1.549563in}}%
\pgfpathcurveto{\pgfqpoint{5.017395in}{1.543739in}}{\pgfqpoint{5.014123in}{1.535839in}}{\pgfqpoint{5.014123in}{1.527603in}}%
\pgfpathcurveto{\pgfqpoint{5.014123in}{1.519367in}}{\pgfqpoint{5.017395in}{1.511467in}}{\pgfqpoint{5.023219in}{1.505643in}}%
\pgfpathcurveto{\pgfqpoint{5.029043in}{1.499819in}}{\pgfqpoint{5.036943in}{1.496547in}}{\pgfqpoint{5.045179in}{1.496547in}}%
\pgfpathclose%
\pgfusepath{stroke,fill}%
\end{pgfscope}%
\begin{pgfscope}%
\pgfpathrectangle{\pgfqpoint{3.894341in}{0.419100in}}{\pgfqpoint{2.504659in}{2.933700in}} %
\pgfusepath{clip}%
\pgfsetbuttcap%
\pgfsetroundjoin%
\definecolor{currentfill}{rgb}{1.000000,0.549020,0.000000}%
\pgfsetfillcolor{currentfill}%
\pgfsetfillopacity{0.892125}%
\pgfsetlinewidth{1.003750pt}%
\definecolor{currentstroke}{rgb}{1.000000,0.549020,0.000000}%
\pgfsetstrokecolor{currentstroke}%
\pgfsetstrokeopacity{0.892125}%
\pgfsetdash{}{0pt}%
\pgfpathmoveto{\pgfqpoint{5.197315in}{1.513051in}}%
\pgfpathcurveto{\pgfqpoint{5.205551in}{1.513051in}}{\pgfqpoint{5.213451in}{1.516324in}}{\pgfqpoint{5.219275in}{1.522148in}}%
\pgfpathcurveto{\pgfqpoint{5.225099in}{1.527972in}}{\pgfqpoint{5.228371in}{1.535872in}}{\pgfqpoint{5.228371in}{1.544108in}}%
\pgfpathcurveto{\pgfqpoint{5.228371in}{1.552344in}}{\pgfqpoint{5.225099in}{1.560244in}}{\pgfqpoint{5.219275in}{1.566068in}}%
\pgfpathcurveto{\pgfqpoint{5.213451in}{1.571892in}}{\pgfqpoint{5.205551in}{1.575164in}}{\pgfqpoint{5.197315in}{1.575164in}}%
\pgfpathcurveto{\pgfqpoint{5.189078in}{1.575164in}}{\pgfqpoint{5.181178in}{1.571892in}}{\pgfqpoint{5.175354in}{1.566068in}}%
\pgfpathcurveto{\pgfqpoint{5.169530in}{1.560244in}}{\pgfqpoint{5.166258in}{1.552344in}}{\pgfqpoint{5.166258in}{1.544108in}}%
\pgfpathcurveto{\pgfqpoint{5.166258in}{1.535872in}}{\pgfqpoint{5.169530in}{1.527972in}}{\pgfqpoint{5.175354in}{1.522148in}}%
\pgfpathcurveto{\pgfqpoint{5.181178in}{1.516324in}}{\pgfqpoint{5.189078in}{1.513051in}}{\pgfqpoint{5.197315in}{1.513051in}}%
\pgfpathclose%
\pgfusepath{stroke,fill}%
\end{pgfscope}%
\begin{pgfscope}%
\pgfpathrectangle{\pgfqpoint{3.894341in}{0.419100in}}{\pgfqpoint{2.504659in}{2.933700in}} %
\pgfusepath{clip}%
\pgfsetbuttcap%
\pgfsetroundjoin%
\definecolor{currentfill}{rgb}{1.000000,0.549020,0.000000}%
\pgfsetfillcolor{currentfill}%
\pgfsetfillopacity{0.446124}%
\pgfsetlinewidth{1.003750pt}%
\definecolor{currentstroke}{rgb}{1.000000,0.549020,0.000000}%
\pgfsetstrokecolor{currentstroke}%
\pgfsetstrokeopacity{0.446124}%
\pgfsetdash{}{0pt}%
\pgfpathmoveto{\pgfqpoint{4.948911in}{1.596350in}}%
\pgfpathcurveto{\pgfqpoint{4.957147in}{1.596350in}}{\pgfqpoint{4.965047in}{1.599622in}}{\pgfqpoint{4.970871in}{1.605446in}}%
\pgfpathcurveto{\pgfqpoint{4.976695in}{1.611270in}}{\pgfqpoint{4.979967in}{1.619170in}}{\pgfqpoint{4.979967in}{1.627407in}}%
\pgfpathcurveto{\pgfqpoint{4.979967in}{1.635643in}}{\pgfqpoint{4.976695in}{1.643543in}}{\pgfqpoint{4.970871in}{1.649367in}}%
\pgfpathcurveto{\pgfqpoint{4.965047in}{1.655191in}}{\pgfqpoint{4.957147in}{1.658463in}}{\pgfqpoint{4.948911in}{1.658463in}}%
\pgfpathcurveto{\pgfqpoint{4.940674in}{1.658463in}}{\pgfqpoint{4.932774in}{1.655191in}}{\pgfqpoint{4.926950in}{1.649367in}}%
\pgfpathcurveto{\pgfqpoint{4.921126in}{1.643543in}}{\pgfqpoint{4.917854in}{1.635643in}}{\pgfqpoint{4.917854in}{1.627407in}}%
\pgfpathcurveto{\pgfqpoint{4.917854in}{1.619170in}}{\pgfqpoint{4.921126in}{1.611270in}}{\pgfqpoint{4.926950in}{1.605446in}}%
\pgfpathcurveto{\pgfqpoint{4.932774in}{1.599622in}}{\pgfqpoint{4.940674in}{1.596350in}}{\pgfqpoint{4.948911in}{1.596350in}}%
\pgfpathclose%
\pgfusepath{stroke,fill}%
\end{pgfscope}%
\begin{pgfscope}%
\pgfpathrectangle{\pgfqpoint{3.894341in}{0.419100in}}{\pgfqpoint{2.504659in}{2.933700in}} %
\pgfusepath{clip}%
\pgfsetbuttcap%
\pgfsetroundjoin%
\definecolor{currentfill}{rgb}{1.000000,0.549020,0.000000}%
\pgfsetfillcolor{currentfill}%
\pgfsetfillopacity{0.531390}%
\pgfsetlinewidth{1.003750pt}%
\definecolor{currentstroke}{rgb}{1.000000,0.549020,0.000000}%
\pgfsetstrokecolor{currentstroke}%
\pgfsetstrokeopacity{0.531390}%
\pgfsetdash{}{0pt}%
\pgfpathmoveto{\pgfqpoint{5.307973in}{1.356932in}}%
\pgfpathcurveto{\pgfqpoint{5.316209in}{1.356932in}}{\pgfqpoint{5.324109in}{1.360205in}}{\pgfqpoint{5.329933in}{1.366029in}}%
\pgfpathcurveto{\pgfqpoint{5.335757in}{1.371853in}}{\pgfqpoint{5.339029in}{1.379753in}}{\pgfqpoint{5.339029in}{1.387989in}}%
\pgfpathcurveto{\pgfqpoint{5.339029in}{1.396225in}}{\pgfqpoint{5.335757in}{1.404125in}}{\pgfqpoint{5.329933in}{1.409949in}}%
\pgfpathcurveto{\pgfqpoint{5.324109in}{1.415773in}}{\pgfqpoint{5.316209in}{1.419045in}}{\pgfqpoint{5.307973in}{1.419045in}}%
\pgfpathcurveto{\pgfqpoint{5.299737in}{1.419045in}}{\pgfqpoint{5.291837in}{1.415773in}}{\pgfqpoint{5.286013in}{1.409949in}}%
\pgfpathcurveto{\pgfqpoint{5.280189in}{1.404125in}}{\pgfqpoint{5.276916in}{1.396225in}}{\pgfqpoint{5.276916in}{1.387989in}}%
\pgfpathcurveto{\pgfqpoint{5.276916in}{1.379753in}}{\pgfqpoint{5.280189in}{1.371853in}}{\pgfqpoint{5.286013in}{1.366029in}}%
\pgfpathcurveto{\pgfqpoint{5.291837in}{1.360205in}}{\pgfqpoint{5.299737in}{1.356932in}}{\pgfqpoint{5.307973in}{1.356932in}}%
\pgfpathclose%
\pgfusepath{stroke,fill}%
\end{pgfscope}%
\begin{pgfscope}%
\pgfpathrectangle{\pgfqpoint{3.894341in}{0.419100in}}{\pgfqpoint{2.504659in}{2.933700in}} %
\pgfusepath{clip}%
\pgfsetbuttcap%
\pgfsetroundjoin%
\definecolor{currentfill}{rgb}{1.000000,0.549020,0.000000}%
\pgfsetfillcolor{currentfill}%
\pgfsetfillopacity{0.369554}%
\pgfsetlinewidth{1.003750pt}%
\definecolor{currentstroke}{rgb}{1.000000,0.549020,0.000000}%
\pgfsetstrokecolor{currentstroke}%
\pgfsetstrokeopacity{0.369554}%
\pgfsetdash{}{0pt}%
\pgfpathmoveto{\pgfqpoint{5.243716in}{1.821847in}}%
\pgfpathcurveto{\pgfqpoint{5.251952in}{1.821847in}}{\pgfqpoint{5.259852in}{1.825119in}}{\pgfqpoint{5.265676in}{1.830943in}}%
\pgfpathcurveto{\pgfqpoint{5.271500in}{1.836767in}}{\pgfqpoint{5.274772in}{1.844667in}}{\pgfqpoint{5.274772in}{1.852903in}}%
\pgfpathcurveto{\pgfqpoint{5.274772in}{1.861140in}}{\pgfqpoint{5.271500in}{1.869040in}}{\pgfqpoint{5.265676in}{1.874864in}}%
\pgfpathcurveto{\pgfqpoint{5.259852in}{1.880688in}}{\pgfqpoint{5.251952in}{1.883960in}}{\pgfqpoint{5.243716in}{1.883960in}}%
\pgfpathcurveto{\pgfqpoint{5.235479in}{1.883960in}}{\pgfqpoint{5.227579in}{1.880688in}}{\pgfqpoint{5.221755in}{1.874864in}}%
\pgfpathcurveto{\pgfqpoint{5.215931in}{1.869040in}}{\pgfqpoint{5.212659in}{1.861140in}}{\pgfqpoint{5.212659in}{1.852903in}}%
\pgfpathcurveto{\pgfqpoint{5.212659in}{1.844667in}}{\pgfqpoint{5.215931in}{1.836767in}}{\pgfqpoint{5.221755in}{1.830943in}}%
\pgfpathcurveto{\pgfqpoint{5.227579in}{1.825119in}}{\pgfqpoint{5.235479in}{1.821847in}}{\pgfqpoint{5.243716in}{1.821847in}}%
\pgfpathclose%
\pgfusepath{stroke,fill}%
\end{pgfscope}%
\begin{pgfscope}%
\pgfpathrectangle{\pgfqpoint{3.894341in}{0.419100in}}{\pgfqpoint{2.504659in}{2.933700in}} %
\pgfusepath{clip}%
\pgfsetbuttcap%
\pgfsetroundjoin%
\definecolor{currentfill}{rgb}{1.000000,0.549020,0.000000}%
\pgfsetfillcolor{currentfill}%
\pgfsetfillopacity{0.740283}%
\pgfsetlinewidth{1.003750pt}%
\definecolor{currentstroke}{rgb}{1.000000,0.549020,0.000000}%
\pgfsetstrokecolor{currentstroke}%
\pgfsetstrokeopacity{0.740283}%
\pgfsetdash{}{0pt}%
\pgfpathmoveto{\pgfqpoint{5.289227in}{1.253589in}}%
\pgfpathcurveto{\pgfqpoint{5.297463in}{1.253589in}}{\pgfqpoint{5.305363in}{1.256861in}}{\pgfqpoint{5.311187in}{1.262685in}}%
\pgfpathcurveto{\pgfqpoint{5.317011in}{1.268509in}}{\pgfqpoint{5.320284in}{1.276409in}}{\pgfqpoint{5.320284in}{1.284646in}}%
\pgfpathcurveto{\pgfqpoint{5.320284in}{1.292882in}}{\pgfqpoint{5.317011in}{1.300782in}}{\pgfqpoint{5.311187in}{1.306606in}}%
\pgfpathcurveto{\pgfqpoint{5.305363in}{1.312430in}}{\pgfqpoint{5.297463in}{1.315702in}}{\pgfqpoint{5.289227in}{1.315702in}}%
\pgfpathcurveto{\pgfqpoint{5.280991in}{1.315702in}}{\pgfqpoint{5.273091in}{1.312430in}}{\pgfqpoint{5.267267in}{1.306606in}}%
\pgfpathcurveto{\pgfqpoint{5.261443in}{1.300782in}}{\pgfqpoint{5.258171in}{1.292882in}}{\pgfqpoint{5.258171in}{1.284646in}}%
\pgfpathcurveto{\pgfqpoint{5.258171in}{1.276409in}}{\pgfqpoint{5.261443in}{1.268509in}}{\pgfqpoint{5.267267in}{1.262685in}}%
\pgfpathcurveto{\pgfqpoint{5.273091in}{1.256861in}}{\pgfqpoint{5.280991in}{1.253589in}}{\pgfqpoint{5.289227in}{1.253589in}}%
\pgfpathclose%
\pgfusepath{stroke,fill}%
\end{pgfscope}%
\begin{pgfscope}%
\pgfpathrectangle{\pgfqpoint{3.894341in}{0.419100in}}{\pgfqpoint{2.504659in}{2.933700in}} %
\pgfusepath{clip}%
\pgfsetbuttcap%
\pgfsetroundjoin%
\definecolor{currentfill}{rgb}{1.000000,0.549020,0.000000}%
\pgfsetfillcolor{currentfill}%
\pgfsetfillopacity{0.631117}%
\pgfsetlinewidth{1.003750pt}%
\definecolor{currentstroke}{rgb}{1.000000,0.549020,0.000000}%
\pgfsetstrokecolor{currentstroke}%
\pgfsetstrokeopacity{0.631117}%
\pgfsetdash{}{0pt}%
\pgfpathmoveto{\pgfqpoint{5.153747in}{1.168047in}}%
\pgfpathcurveto{\pgfqpoint{5.161983in}{1.168047in}}{\pgfqpoint{5.169883in}{1.171320in}}{\pgfqpoint{5.175707in}{1.177144in}}%
\pgfpathcurveto{\pgfqpoint{5.181531in}{1.182968in}}{\pgfqpoint{5.184804in}{1.190868in}}{\pgfqpoint{5.184804in}{1.199104in}}%
\pgfpathcurveto{\pgfqpoint{5.184804in}{1.207340in}}{\pgfqpoint{5.181531in}{1.215240in}}{\pgfqpoint{5.175707in}{1.221064in}}%
\pgfpathcurveto{\pgfqpoint{5.169883in}{1.226888in}}{\pgfqpoint{5.161983in}{1.230160in}}{\pgfqpoint{5.153747in}{1.230160in}}%
\pgfpathcurveto{\pgfqpoint{5.145511in}{1.230160in}}{\pgfqpoint{5.137611in}{1.226888in}}{\pgfqpoint{5.131787in}{1.221064in}}%
\pgfpathcurveto{\pgfqpoint{5.125963in}{1.215240in}}{\pgfqpoint{5.122691in}{1.207340in}}{\pgfqpoint{5.122691in}{1.199104in}}%
\pgfpathcurveto{\pgfqpoint{5.122691in}{1.190868in}}{\pgfqpoint{5.125963in}{1.182968in}}{\pgfqpoint{5.131787in}{1.177144in}}%
\pgfpathcurveto{\pgfqpoint{5.137611in}{1.171320in}}{\pgfqpoint{5.145511in}{1.168047in}}{\pgfqpoint{5.153747in}{1.168047in}}%
\pgfpathclose%
\pgfusepath{stroke,fill}%
\end{pgfscope}%
\begin{pgfscope}%
\pgfpathrectangle{\pgfqpoint{3.894341in}{0.419100in}}{\pgfqpoint{2.504659in}{2.933700in}} %
\pgfusepath{clip}%
\pgfsetbuttcap%
\pgfsetroundjoin%
\definecolor{currentfill}{rgb}{1.000000,0.549020,0.000000}%
\pgfsetfillcolor{currentfill}%
\pgfsetfillopacity{0.537370}%
\pgfsetlinewidth{1.003750pt}%
\definecolor{currentstroke}{rgb}{1.000000,0.549020,0.000000}%
\pgfsetstrokecolor{currentstroke}%
\pgfsetstrokeopacity{0.537370}%
\pgfsetdash{}{0pt}%
\pgfpathmoveto{\pgfqpoint{5.276906in}{1.302001in}}%
\pgfpathcurveto{\pgfqpoint{5.285142in}{1.302001in}}{\pgfqpoint{5.293042in}{1.305273in}}{\pgfqpoint{5.298866in}{1.311097in}}%
\pgfpathcurveto{\pgfqpoint{5.304690in}{1.316921in}}{\pgfqpoint{5.307962in}{1.324821in}}{\pgfqpoint{5.307962in}{1.333057in}}%
\pgfpathcurveto{\pgfqpoint{5.307962in}{1.341293in}}{\pgfqpoint{5.304690in}{1.349194in}}{\pgfqpoint{5.298866in}{1.355017in}}%
\pgfpathcurveto{\pgfqpoint{5.293042in}{1.360841in}}{\pgfqpoint{5.285142in}{1.364114in}}{\pgfqpoint{5.276906in}{1.364114in}}%
\pgfpathcurveto{\pgfqpoint{5.268669in}{1.364114in}}{\pgfqpoint{5.260769in}{1.360841in}}{\pgfqpoint{5.254945in}{1.355017in}}%
\pgfpathcurveto{\pgfqpoint{5.249121in}{1.349194in}}{\pgfqpoint{5.245849in}{1.341293in}}{\pgfqpoint{5.245849in}{1.333057in}}%
\pgfpathcurveto{\pgfqpoint{5.245849in}{1.324821in}}{\pgfqpoint{5.249121in}{1.316921in}}{\pgfqpoint{5.254945in}{1.311097in}}%
\pgfpathcurveto{\pgfqpoint{5.260769in}{1.305273in}}{\pgfqpoint{5.268669in}{1.302001in}}{\pgfqpoint{5.276906in}{1.302001in}}%
\pgfpathclose%
\pgfusepath{stroke,fill}%
\end{pgfscope}%
\begin{pgfscope}%
\pgfpathrectangle{\pgfqpoint{3.894341in}{0.419100in}}{\pgfqpoint{2.504659in}{2.933700in}} %
\pgfusepath{clip}%
\pgfsetbuttcap%
\pgfsetroundjoin%
\definecolor{currentfill}{rgb}{1.000000,0.549020,0.000000}%
\pgfsetfillcolor{currentfill}%
\pgfsetfillopacity{0.686713}%
\pgfsetlinewidth{1.003750pt}%
\definecolor{currentstroke}{rgb}{1.000000,0.549020,0.000000}%
\pgfsetstrokecolor{currentstroke}%
\pgfsetstrokeopacity{0.686713}%
\pgfsetdash{}{0pt}%
\pgfpathmoveto{\pgfqpoint{5.366553in}{1.245346in}}%
\pgfpathcurveto{\pgfqpoint{5.374790in}{1.245346in}}{\pgfqpoint{5.382690in}{1.248619in}}{\pgfqpoint{5.388514in}{1.254443in}}%
\pgfpathcurveto{\pgfqpoint{5.394338in}{1.260267in}}{\pgfqpoint{5.397610in}{1.268167in}}{\pgfqpoint{5.397610in}{1.276403in}}%
\pgfpathcurveto{\pgfqpoint{5.397610in}{1.284639in}}{\pgfqpoint{5.394338in}{1.292539in}}{\pgfqpoint{5.388514in}{1.298363in}}%
\pgfpathcurveto{\pgfqpoint{5.382690in}{1.304187in}}{\pgfqpoint{5.374790in}{1.307459in}}{\pgfqpoint{5.366553in}{1.307459in}}%
\pgfpathcurveto{\pgfqpoint{5.358317in}{1.307459in}}{\pgfqpoint{5.350417in}{1.304187in}}{\pgfqpoint{5.344593in}{1.298363in}}%
\pgfpathcurveto{\pgfqpoint{5.338769in}{1.292539in}}{\pgfqpoint{5.335497in}{1.284639in}}{\pgfqpoint{5.335497in}{1.276403in}}%
\pgfpathcurveto{\pgfqpoint{5.335497in}{1.268167in}}{\pgfqpoint{5.338769in}{1.260267in}}{\pgfqpoint{5.344593in}{1.254443in}}%
\pgfpathcurveto{\pgfqpoint{5.350417in}{1.248619in}}{\pgfqpoint{5.358317in}{1.245346in}}{\pgfqpoint{5.366553in}{1.245346in}}%
\pgfpathclose%
\pgfusepath{stroke,fill}%
\end{pgfscope}%
\begin{pgfscope}%
\pgfpathrectangle{\pgfqpoint{3.894341in}{0.419100in}}{\pgfqpoint{2.504659in}{2.933700in}} %
\pgfusepath{clip}%
\pgfsetbuttcap%
\pgfsetroundjoin%
\definecolor{currentfill}{rgb}{1.000000,0.549020,0.000000}%
\pgfsetfillcolor{currentfill}%
\pgfsetfillopacity{0.300000}%
\pgfsetlinewidth{1.003750pt}%
\definecolor{currentstroke}{rgb}{1.000000,0.549020,0.000000}%
\pgfsetstrokecolor{currentstroke}%
\pgfsetstrokeopacity{0.300000}%
\pgfsetdash{}{0pt}%
\pgfpathmoveto{\pgfqpoint{5.120809in}{1.914092in}}%
\pgfpathcurveto{\pgfqpoint{5.129045in}{1.914092in}}{\pgfqpoint{5.136945in}{1.917364in}}{\pgfqpoint{5.142769in}{1.923188in}}%
\pgfpathcurveto{\pgfqpoint{5.148593in}{1.929012in}}{\pgfqpoint{5.151865in}{1.936912in}}{\pgfqpoint{5.151865in}{1.945148in}}%
\pgfpathcurveto{\pgfqpoint{5.151865in}{1.953384in}}{\pgfqpoint{5.148593in}{1.961284in}}{\pgfqpoint{5.142769in}{1.967108in}}%
\pgfpathcurveto{\pgfqpoint{5.136945in}{1.972932in}}{\pgfqpoint{5.129045in}{1.976205in}}{\pgfqpoint{5.120809in}{1.976205in}}%
\pgfpathcurveto{\pgfqpoint{5.112573in}{1.976205in}}{\pgfqpoint{5.104673in}{1.972932in}}{\pgfqpoint{5.098849in}{1.967108in}}%
\pgfpathcurveto{\pgfqpoint{5.093025in}{1.961284in}}{\pgfqpoint{5.089752in}{1.953384in}}{\pgfqpoint{5.089752in}{1.945148in}}%
\pgfpathcurveto{\pgfqpoint{5.089752in}{1.936912in}}{\pgfqpoint{5.093025in}{1.929012in}}{\pgfqpoint{5.098849in}{1.923188in}}%
\pgfpathcurveto{\pgfqpoint{5.104673in}{1.917364in}}{\pgfqpoint{5.112573in}{1.914092in}}{\pgfqpoint{5.120809in}{1.914092in}}%
\pgfpathclose%
\pgfusepath{stroke,fill}%
\end{pgfscope}%
\begin{pgfscope}%
\pgfpathrectangle{\pgfqpoint{3.894341in}{0.419100in}}{\pgfqpoint{2.504659in}{2.933700in}} %
\pgfusepath{clip}%
\pgfsetbuttcap%
\pgfsetroundjoin%
\definecolor{currentfill}{rgb}{1.000000,0.549020,0.000000}%
\pgfsetfillcolor{currentfill}%
\pgfsetfillopacity{0.447291}%
\pgfsetlinewidth{1.003750pt}%
\definecolor{currentstroke}{rgb}{1.000000,0.549020,0.000000}%
\pgfsetstrokecolor{currentstroke}%
\pgfsetstrokeopacity{0.447291}%
\pgfsetdash{}{0pt}%
\pgfpathmoveto{\pgfqpoint{4.893558in}{1.670324in}}%
\pgfpathcurveto{\pgfqpoint{4.901795in}{1.670324in}}{\pgfqpoint{4.909695in}{1.673596in}}{\pgfqpoint{4.915519in}{1.679420in}}%
\pgfpathcurveto{\pgfqpoint{4.921343in}{1.685244in}}{\pgfqpoint{4.924615in}{1.693144in}}{\pgfqpoint{4.924615in}{1.701380in}}%
\pgfpathcurveto{\pgfqpoint{4.924615in}{1.709617in}}{\pgfqpoint{4.921343in}{1.717517in}}{\pgfqpoint{4.915519in}{1.723341in}}%
\pgfpathcurveto{\pgfqpoint{4.909695in}{1.729165in}}{\pgfqpoint{4.901795in}{1.732437in}}{\pgfqpoint{4.893558in}{1.732437in}}%
\pgfpathcurveto{\pgfqpoint{4.885322in}{1.732437in}}{\pgfqpoint{4.877422in}{1.729165in}}{\pgfqpoint{4.871598in}{1.723341in}}%
\pgfpathcurveto{\pgfqpoint{4.865774in}{1.717517in}}{\pgfqpoint{4.862502in}{1.709617in}}{\pgfqpoint{4.862502in}{1.701380in}}%
\pgfpathcurveto{\pgfqpoint{4.862502in}{1.693144in}}{\pgfqpoint{4.865774in}{1.685244in}}{\pgfqpoint{4.871598in}{1.679420in}}%
\pgfpathcurveto{\pgfqpoint{4.877422in}{1.673596in}}{\pgfqpoint{4.885322in}{1.670324in}}{\pgfqpoint{4.893558in}{1.670324in}}%
\pgfpathclose%
\pgfusepath{stroke,fill}%
\end{pgfscope}%
\begin{pgfscope}%
\pgfpathrectangle{\pgfqpoint{3.894341in}{0.419100in}}{\pgfqpoint{2.504659in}{2.933700in}} %
\pgfusepath{clip}%
\pgfsetbuttcap%
\pgfsetroundjoin%
\definecolor{currentfill}{rgb}{1.000000,0.549020,0.000000}%
\pgfsetfillcolor{currentfill}%
\pgfsetfillopacity{0.483050}%
\pgfsetlinewidth{1.003750pt}%
\definecolor{currentstroke}{rgb}{1.000000,0.549020,0.000000}%
\pgfsetstrokecolor{currentstroke}%
\pgfsetstrokeopacity{0.483050}%
\pgfsetdash{}{0pt}%
\pgfpathmoveto{\pgfqpoint{5.144413in}{1.332833in}}%
\pgfpathcurveto{\pgfqpoint{5.152649in}{1.332833in}}{\pgfqpoint{5.160549in}{1.336105in}}{\pgfqpoint{5.166373in}{1.341929in}}%
\pgfpathcurveto{\pgfqpoint{5.172197in}{1.347753in}}{\pgfqpoint{5.175469in}{1.355653in}}{\pgfqpoint{5.175469in}{1.363889in}}%
\pgfpathcurveto{\pgfqpoint{5.175469in}{1.372125in}}{\pgfqpoint{5.172197in}{1.380025in}}{\pgfqpoint{5.166373in}{1.385849in}}%
\pgfpathcurveto{\pgfqpoint{5.160549in}{1.391673in}}{\pgfqpoint{5.152649in}{1.394946in}}{\pgfqpoint{5.144413in}{1.394946in}}%
\pgfpathcurveto{\pgfqpoint{5.136176in}{1.394946in}}{\pgfqpoint{5.128276in}{1.391673in}}{\pgfqpoint{5.122452in}{1.385849in}}%
\pgfpathcurveto{\pgfqpoint{5.116629in}{1.380025in}}{\pgfqpoint{5.113356in}{1.372125in}}{\pgfqpoint{5.113356in}{1.363889in}}%
\pgfpathcurveto{\pgfqpoint{5.113356in}{1.355653in}}{\pgfqpoint{5.116629in}{1.347753in}}{\pgfqpoint{5.122452in}{1.341929in}}%
\pgfpathcurveto{\pgfqpoint{5.128276in}{1.336105in}}{\pgfqpoint{5.136176in}{1.332833in}}{\pgfqpoint{5.144413in}{1.332833in}}%
\pgfpathclose%
\pgfusepath{stroke,fill}%
\end{pgfscope}%
\begin{pgfscope}%
\pgfpathrectangle{\pgfqpoint{3.894341in}{0.419100in}}{\pgfqpoint{2.504659in}{2.933700in}} %
\pgfusepath{clip}%
\pgfsetbuttcap%
\pgfsetroundjoin%
\definecolor{currentfill}{rgb}{1.000000,0.549020,0.000000}%
\pgfsetfillcolor{currentfill}%
\pgfsetfillopacity{0.523492}%
\pgfsetlinewidth{1.003750pt}%
\definecolor{currentstroke}{rgb}{1.000000,0.549020,0.000000}%
\pgfsetstrokecolor{currentstroke}%
\pgfsetstrokeopacity{0.523492}%
\pgfsetdash{}{0pt}%
\pgfpathmoveto{\pgfqpoint{5.266847in}{1.336736in}}%
\pgfpathcurveto{\pgfqpoint{5.275083in}{1.336736in}}{\pgfqpoint{5.282984in}{1.340009in}}{\pgfqpoint{5.288807in}{1.345833in}}%
\pgfpathcurveto{\pgfqpoint{5.294631in}{1.351657in}}{\pgfqpoint{5.297904in}{1.359557in}}{\pgfqpoint{5.297904in}{1.367793in}}%
\pgfpathcurveto{\pgfqpoint{5.297904in}{1.376029in}}{\pgfqpoint{5.294631in}{1.383929in}}{\pgfqpoint{5.288807in}{1.389753in}}%
\pgfpathcurveto{\pgfqpoint{5.282984in}{1.395577in}}{\pgfqpoint{5.275083in}{1.398849in}}{\pgfqpoint{5.266847in}{1.398849in}}%
\pgfpathcurveto{\pgfqpoint{5.258611in}{1.398849in}}{\pgfqpoint{5.250711in}{1.395577in}}{\pgfqpoint{5.244887in}{1.389753in}}%
\pgfpathcurveto{\pgfqpoint{5.239063in}{1.383929in}}{\pgfqpoint{5.235791in}{1.376029in}}{\pgfqpoint{5.235791in}{1.367793in}}%
\pgfpathcurveto{\pgfqpoint{5.235791in}{1.359557in}}{\pgfqpoint{5.239063in}{1.351657in}}{\pgfqpoint{5.244887in}{1.345833in}}%
\pgfpathcurveto{\pgfqpoint{5.250711in}{1.340009in}}{\pgfqpoint{5.258611in}{1.336736in}}{\pgfqpoint{5.266847in}{1.336736in}}%
\pgfpathclose%
\pgfusepath{stroke,fill}%
\end{pgfscope}%
\begin{pgfscope}%
\pgfpathrectangle{\pgfqpoint{3.894341in}{0.419100in}}{\pgfqpoint{2.504659in}{2.933700in}} %
\pgfusepath{clip}%
\pgfsetbuttcap%
\pgfsetroundjoin%
\definecolor{currentfill}{rgb}{1.000000,0.549020,0.000000}%
\pgfsetfillcolor{currentfill}%
\pgfsetfillopacity{0.645907}%
\pgfsetlinewidth{1.003750pt}%
\definecolor{currentstroke}{rgb}{1.000000,0.549020,0.000000}%
\pgfsetstrokecolor{currentstroke}%
\pgfsetstrokeopacity{0.645907}%
\pgfsetdash{}{0pt}%
\pgfpathmoveto{\pgfqpoint{5.372979in}{1.170084in}}%
\pgfpathcurveto{\pgfqpoint{5.381215in}{1.170084in}}{\pgfqpoint{5.389115in}{1.173356in}}{\pgfqpoint{5.394939in}{1.179180in}}%
\pgfpathcurveto{\pgfqpoint{5.400763in}{1.185004in}}{\pgfqpoint{5.404035in}{1.192904in}}{\pgfqpoint{5.404035in}{1.201140in}}%
\pgfpathcurveto{\pgfqpoint{5.404035in}{1.209377in}}{\pgfqpoint{5.400763in}{1.217277in}}{\pgfqpoint{5.394939in}{1.223101in}}%
\pgfpathcurveto{\pgfqpoint{5.389115in}{1.228925in}}{\pgfqpoint{5.381215in}{1.232197in}}{\pgfqpoint{5.372979in}{1.232197in}}%
\pgfpathcurveto{\pgfqpoint{5.364743in}{1.232197in}}{\pgfqpoint{5.356843in}{1.228925in}}{\pgfqpoint{5.351019in}{1.223101in}}%
\pgfpathcurveto{\pgfqpoint{5.345195in}{1.217277in}}{\pgfqpoint{5.341922in}{1.209377in}}{\pgfqpoint{5.341922in}{1.201140in}}%
\pgfpathcurveto{\pgfqpoint{5.341922in}{1.192904in}}{\pgfqpoint{5.345195in}{1.185004in}}{\pgfqpoint{5.351019in}{1.179180in}}%
\pgfpathcurveto{\pgfqpoint{5.356843in}{1.173356in}}{\pgfqpoint{5.364743in}{1.170084in}}{\pgfqpoint{5.372979in}{1.170084in}}%
\pgfpathclose%
\pgfusepath{stroke,fill}%
\end{pgfscope}%
\begin{pgfscope}%
\pgfpathrectangle{\pgfqpoint{3.894341in}{0.419100in}}{\pgfqpoint{2.504659in}{2.933700in}} %
\pgfusepath{clip}%
\pgfsetbuttcap%
\pgfsetroundjoin%
\definecolor{currentfill}{rgb}{1.000000,0.549020,0.000000}%
\pgfsetfillcolor{currentfill}%
\pgfsetlinewidth{1.003750pt}%
\definecolor{currentstroke}{rgb}{1.000000,0.549020,0.000000}%
\pgfsetstrokecolor{currentstroke}%
\pgfsetdash{}{0pt}%
\pgfpathmoveto{\pgfqpoint{5.261806in}{1.750319in}}%
\pgfpathcurveto{\pgfqpoint{5.270042in}{1.750319in}}{\pgfqpoint{5.277942in}{1.753591in}}{\pgfqpoint{5.283766in}{1.759415in}}%
\pgfpathcurveto{\pgfqpoint{5.289590in}{1.765239in}}{\pgfqpoint{5.292862in}{1.773139in}}{\pgfqpoint{5.292862in}{1.781375in}}%
\pgfpathcurveto{\pgfqpoint{5.292862in}{1.789612in}}{\pgfqpoint{5.289590in}{1.797512in}}{\pgfqpoint{5.283766in}{1.803336in}}%
\pgfpathcurveto{\pgfqpoint{5.277942in}{1.809159in}}{\pgfqpoint{5.270042in}{1.812432in}}{\pgfqpoint{5.261806in}{1.812432in}}%
\pgfpathcurveto{\pgfqpoint{5.253569in}{1.812432in}}{\pgfqpoint{5.245669in}{1.809159in}}{\pgfqpoint{5.239845in}{1.803336in}}%
\pgfpathcurveto{\pgfqpoint{5.234022in}{1.797512in}}{\pgfqpoint{5.230749in}{1.789612in}}{\pgfqpoint{5.230749in}{1.781375in}}%
\pgfpathcurveto{\pgfqpoint{5.230749in}{1.773139in}}{\pgfqpoint{5.234022in}{1.765239in}}{\pgfqpoint{5.239845in}{1.759415in}}%
\pgfpathcurveto{\pgfqpoint{5.245669in}{1.753591in}}{\pgfqpoint{5.253569in}{1.750319in}}{\pgfqpoint{5.261806in}{1.750319in}}%
\pgfpathclose%
\pgfusepath{stroke,fill}%
\end{pgfscope}%
\begin{pgfscope}%
\pgfpathrectangle{\pgfqpoint{3.894341in}{0.419100in}}{\pgfqpoint{2.504659in}{2.933700in}} %
\pgfusepath{clip}%
\pgfsetbuttcap%
\pgfsetroundjoin%
\definecolor{currentfill}{rgb}{1.000000,0.549020,0.000000}%
\pgfsetfillcolor{currentfill}%
\pgfsetfillopacity{0.794259}%
\pgfsetlinewidth{1.003750pt}%
\definecolor{currentstroke}{rgb}{1.000000,0.549020,0.000000}%
\pgfsetstrokecolor{currentstroke}%
\pgfsetstrokeopacity{0.794259}%
\pgfsetdash{}{0pt}%
\pgfpathmoveto{\pgfqpoint{5.437028in}{1.558605in}}%
\pgfpathcurveto{\pgfqpoint{5.445264in}{1.558605in}}{\pgfqpoint{5.453164in}{1.561877in}}{\pgfqpoint{5.458988in}{1.567701in}}%
\pgfpathcurveto{\pgfqpoint{5.464812in}{1.573525in}}{\pgfqpoint{5.468085in}{1.581425in}}{\pgfqpoint{5.468085in}{1.589662in}}%
\pgfpathcurveto{\pgfqpoint{5.468085in}{1.597898in}}{\pgfqpoint{5.464812in}{1.605798in}}{\pgfqpoint{5.458988in}{1.611622in}}%
\pgfpathcurveto{\pgfqpoint{5.453164in}{1.617446in}}{\pgfqpoint{5.445264in}{1.620718in}}{\pgfqpoint{5.437028in}{1.620718in}}%
\pgfpathcurveto{\pgfqpoint{5.428792in}{1.620718in}}{\pgfqpoint{5.420892in}{1.617446in}}{\pgfqpoint{5.415068in}{1.611622in}}%
\pgfpathcurveto{\pgfqpoint{5.409244in}{1.605798in}}{\pgfqpoint{5.405972in}{1.597898in}}{\pgfqpoint{5.405972in}{1.589662in}}%
\pgfpathcurveto{\pgfqpoint{5.405972in}{1.581425in}}{\pgfqpoint{5.409244in}{1.573525in}}{\pgfqpoint{5.415068in}{1.567701in}}%
\pgfpathcurveto{\pgfqpoint{5.420892in}{1.561877in}}{\pgfqpoint{5.428792in}{1.558605in}}{\pgfqpoint{5.437028in}{1.558605in}}%
\pgfpathclose%
\pgfusepath{stroke,fill}%
\end{pgfscope}%
\begin{pgfscope}%
\pgfpathrectangle{\pgfqpoint{3.894341in}{0.419100in}}{\pgfqpoint{2.504659in}{2.933700in}} %
\pgfusepath{clip}%
\pgfsetbuttcap%
\pgfsetroundjoin%
\definecolor{currentfill}{rgb}{1.000000,0.549020,0.000000}%
\pgfsetfillcolor{currentfill}%
\pgfsetfillopacity{0.499313}%
\pgfsetlinewidth{1.003750pt}%
\definecolor{currentstroke}{rgb}{1.000000,0.549020,0.000000}%
\pgfsetstrokecolor{currentstroke}%
\pgfsetstrokeopacity{0.499313}%
\pgfsetdash{}{0pt}%
\pgfpathmoveto{\pgfqpoint{4.944871in}{1.553140in}}%
\pgfpathcurveto{\pgfqpoint{4.953108in}{1.553140in}}{\pgfqpoint{4.961008in}{1.556413in}}{\pgfqpoint{4.966832in}{1.562237in}}%
\pgfpathcurveto{\pgfqpoint{4.972656in}{1.568060in}}{\pgfqpoint{4.975928in}{1.575960in}}{\pgfqpoint{4.975928in}{1.584197in}}%
\pgfpathcurveto{\pgfqpoint{4.975928in}{1.592433in}}{\pgfqpoint{4.972656in}{1.600333in}}{\pgfqpoint{4.966832in}{1.606157in}}%
\pgfpathcurveto{\pgfqpoint{4.961008in}{1.611981in}}{\pgfqpoint{4.953108in}{1.615253in}}{\pgfqpoint{4.944871in}{1.615253in}}%
\pgfpathcurveto{\pgfqpoint{4.936635in}{1.615253in}}{\pgfqpoint{4.928735in}{1.611981in}}{\pgfqpoint{4.922911in}{1.606157in}}%
\pgfpathcurveto{\pgfqpoint{4.917087in}{1.600333in}}{\pgfqpoint{4.913815in}{1.592433in}}{\pgfqpoint{4.913815in}{1.584197in}}%
\pgfpathcurveto{\pgfqpoint{4.913815in}{1.575960in}}{\pgfqpoint{4.917087in}{1.568060in}}{\pgfqpoint{4.922911in}{1.562237in}}%
\pgfpathcurveto{\pgfqpoint{4.928735in}{1.556413in}}{\pgfqpoint{4.936635in}{1.553140in}}{\pgfqpoint{4.944871in}{1.553140in}}%
\pgfpathclose%
\pgfusepath{stroke,fill}%
\end{pgfscope}%
\begin{pgfscope}%
\pgfpathrectangle{\pgfqpoint{3.894341in}{0.419100in}}{\pgfqpoint{2.504659in}{2.933700in}} %
\pgfusepath{clip}%
\pgfsetbuttcap%
\pgfsetroundjoin%
\definecolor{currentfill}{rgb}{1.000000,0.549020,0.000000}%
\pgfsetfillcolor{currentfill}%
\pgfsetfillopacity{0.456600}%
\pgfsetlinewidth{1.003750pt}%
\definecolor{currentstroke}{rgb}{1.000000,0.549020,0.000000}%
\pgfsetstrokecolor{currentstroke}%
\pgfsetstrokeopacity{0.456600}%
\pgfsetdash{}{0pt}%
\pgfpathmoveto{\pgfqpoint{4.944653in}{1.576851in}}%
\pgfpathcurveto{\pgfqpoint{4.952889in}{1.576851in}}{\pgfqpoint{4.960789in}{1.580124in}}{\pgfqpoint{4.966613in}{1.585948in}}%
\pgfpathcurveto{\pgfqpoint{4.972437in}{1.591771in}}{\pgfqpoint{4.975709in}{1.599671in}}{\pgfqpoint{4.975709in}{1.607908in}}%
\pgfpathcurveto{\pgfqpoint{4.975709in}{1.616144in}}{\pgfqpoint{4.972437in}{1.624044in}}{\pgfqpoint{4.966613in}{1.629868in}}%
\pgfpathcurveto{\pgfqpoint{4.960789in}{1.635692in}}{\pgfqpoint{4.952889in}{1.638964in}}{\pgfqpoint{4.944653in}{1.638964in}}%
\pgfpathcurveto{\pgfqpoint{4.936417in}{1.638964in}}{\pgfqpoint{4.928517in}{1.635692in}}{\pgfqpoint{4.922693in}{1.629868in}}%
\pgfpathcurveto{\pgfqpoint{4.916869in}{1.624044in}}{\pgfqpoint{4.913596in}{1.616144in}}{\pgfqpoint{4.913596in}{1.607908in}}%
\pgfpathcurveto{\pgfqpoint{4.913596in}{1.599671in}}{\pgfqpoint{4.916869in}{1.591771in}}{\pgfqpoint{4.922693in}{1.585948in}}%
\pgfpathcurveto{\pgfqpoint{4.928517in}{1.580124in}}{\pgfqpoint{4.936417in}{1.576851in}}{\pgfqpoint{4.944653in}{1.576851in}}%
\pgfpathclose%
\pgfusepath{stroke,fill}%
\end{pgfscope}%
\begin{pgfscope}%
\pgfpathrectangle{\pgfqpoint{3.894341in}{0.419100in}}{\pgfqpoint{2.504659in}{2.933700in}} %
\pgfusepath{clip}%
\pgfsetbuttcap%
\pgfsetroundjoin%
\definecolor{currentfill}{rgb}{1.000000,0.549020,0.000000}%
\pgfsetfillcolor{currentfill}%
\pgfsetfillopacity{0.842435}%
\pgfsetlinewidth{1.003750pt}%
\definecolor{currentstroke}{rgb}{1.000000,0.549020,0.000000}%
\pgfsetstrokecolor{currentstroke}%
\pgfsetstrokeopacity{0.842435}%
\pgfsetdash{}{0pt}%
\pgfpathmoveto{\pgfqpoint{5.660450in}{1.951132in}}%
\pgfpathcurveto{\pgfqpoint{5.668686in}{1.951132in}}{\pgfqpoint{5.676586in}{1.954404in}}{\pgfqpoint{5.682410in}{1.960228in}}%
\pgfpathcurveto{\pgfqpoint{5.688234in}{1.966052in}}{\pgfqpoint{5.691506in}{1.973952in}}{\pgfqpoint{5.691506in}{1.982188in}}%
\pgfpathcurveto{\pgfqpoint{5.691506in}{1.990424in}}{\pgfqpoint{5.688234in}{1.998324in}}{\pgfqpoint{5.682410in}{2.004148in}}%
\pgfpathcurveto{\pgfqpoint{5.676586in}{2.009972in}}{\pgfqpoint{5.668686in}{2.013245in}}{\pgfqpoint{5.660450in}{2.013245in}}%
\pgfpathcurveto{\pgfqpoint{5.652213in}{2.013245in}}{\pgfqpoint{5.644313in}{2.009972in}}{\pgfqpoint{5.638489in}{2.004148in}}%
\pgfpathcurveto{\pgfqpoint{5.632665in}{1.998324in}}{\pgfqpoint{5.629393in}{1.990424in}}{\pgfqpoint{5.629393in}{1.982188in}}%
\pgfpathcurveto{\pgfqpoint{5.629393in}{1.973952in}}{\pgfqpoint{5.632665in}{1.966052in}}{\pgfqpoint{5.638489in}{1.960228in}}%
\pgfpathcurveto{\pgfqpoint{5.644313in}{1.954404in}}{\pgfqpoint{5.652213in}{1.951132in}}{\pgfqpoint{5.660450in}{1.951132in}}%
\pgfpathclose%
\pgfusepath{stroke,fill}%
\end{pgfscope}%
\begin{pgfscope}%
\pgfpathrectangle{\pgfqpoint{3.894341in}{0.419100in}}{\pgfqpoint{2.504659in}{2.933700in}} %
\pgfusepath{clip}%
\pgfsetbuttcap%
\pgfsetroundjoin%
\definecolor{currentfill}{rgb}{1.000000,0.549020,0.000000}%
\pgfsetfillcolor{currentfill}%
\pgfsetfillopacity{0.518671}%
\pgfsetlinewidth{1.003750pt}%
\definecolor{currentstroke}{rgb}{1.000000,0.549020,0.000000}%
\pgfsetstrokecolor{currentstroke}%
\pgfsetstrokeopacity{0.518671}%
\pgfsetdash{}{0pt}%
\pgfpathmoveto{\pgfqpoint{5.717695in}{1.985660in}}%
\pgfpathcurveto{\pgfqpoint{5.725932in}{1.985660in}}{\pgfqpoint{5.733832in}{1.988932in}}{\pgfqpoint{5.739655in}{1.994756in}}%
\pgfpathcurveto{\pgfqpoint{5.745479in}{2.000580in}}{\pgfqpoint{5.748752in}{2.008480in}}{\pgfqpoint{5.748752in}{2.016716in}}%
\pgfpathcurveto{\pgfqpoint{5.748752in}{2.024953in}}{\pgfqpoint{5.745479in}{2.032853in}}{\pgfqpoint{5.739655in}{2.038677in}}%
\pgfpathcurveto{\pgfqpoint{5.733832in}{2.044500in}}{\pgfqpoint{5.725932in}{2.047773in}}{\pgfqpoint{5.717695in}{2.047773in}}%
\pgfpathcurveto{\pgfqpoint{5.709459in}{2.047773in}}{\pgfqpoint{5.701559in}{2.044500in}}{\pgfqpoint{5.695735in}{2.038677in}}%
\pgfpathcurveto{\pgfqpoint{5.689911in}{2.032853in}}{\pgfqpoint{5.686639in}{2.024953in}}{\pgfqpoint{5.686639in}{2.016716in}}%
\pgfpathcurveto{\pgfqpoint{5.686639in}{2.008480in}}{\pgfqpoint{5.689911in}{2.000580in}}{\pgfqpoint{5.695735in}{1.994756in}}%
\pgfpathcurveto{\pgfqpoint{5.701559in}{1.988932in}}{\pgfqpoint{5.709459in}{1.985660in}}{\pgfqpoint{5.717695in}{1.985660in}}%
\pgfpathclose%
\pgfusepath{stroke,fill}%
\end{pgfscope}%
\begin{pgfscope}%
\pgfpathrectangle{\pgfqpoint{3.894341in}{0.419100in}}{\pgfqpoint{2.504659in}{2.933700in}} %
\pgfusepath{clip}%
\pgfsetbuttcap%
\pgfsetroundjoin%
\definecolor{currentfill}{rgb}{1.000000,0.549020,0.000000}%
\pgfsetfillcolor{currentfill}%
\pgfsetfillopacity{0.603573}%
\pgfsetlinewidth{1.003750pt}%
\definecolor{currentstroke}{rgb}{1.000000,0.549020,0.000000}%
\pgfsetstrokecolor{currentstroke}%
\pgfsetstrokeopacity{0.603573}%
\pgfsetdash{}{0pt}%
\pgfpathmoveto{\pgfqpoint{5.181454in}{1.171652in}}%
\pgfpathcurveto{\pgfqpoint{5.189690in}{1.171652in}}{\pgfqpoint{5.197591in}{1.174925in}}{\pgfqpoint{5.203414in}{1.180748in}}%
\pgfpathcurveto{\pgfqpoint{5.209238in}{1.186572in}}{\pgfqpoint{5.212511in}{1.194472in}}{\pgfqpoint{5.212511in}{1.202709in}}%
\pgfpathcurveto{\pgfqpoint{5.212511in}{1.210945in}}{\pgfqpoint{5.209238in}{1.218845in}}{\pgfqpoint{5.203414in}{1.224669in}}%
\pgfpathcurveto{\pgfqpoint{5.197591in}{1.230493in}}{\pgfqpoint{5.189690in}{1.233765in}}{\pgfqpoint{5.181454in}{1.233765in}}%
\pgfpathcurveto{\pgfqpoint{5.173218in}{1.233765in}}{\pgfqpoint{5.165318in}{1.230493in}}{\pgfqpoint{5.159494in}{1.224669in}}%
\pgfpathcurveto{\pgfqpoint{5.153670in}{1.218845in}}{\pgfqpoint{5.150398in}{1.210945in}}{\pgfqpoint{5.150398in}{1.202709in}}%
\pgfpathcurveto{\pgfqpoint{5.150398in}{1.194472in}}{\pgfqpoint{5.153670in}{1.186572in}}{\pgfqpoint{5.159494in}{1.180748in}}%
\pgfpathcurveto{\pgfqpoint{5.165318in}{1.174925in}}{\pgfqpoint{5.173218in}{1.171652in}}{\pgfqpoint{5.181454in}{1.171652in}}%
\pgfpathclose%
\pgfusepath{stroke,fill}%
\end{pgfscope}%
\begin{pgfscope}%
\pgfpathrectangle{\pgfqpoint{3.894341in}{0.419100in}}{\pgfqpoint{2.504659in}{2.933700in}} %
\pgfusepath{clip}%
\pgfsetbuttcap%
\pgfsetroundjoin%
\definecolor{currentfill}{rgb}{1.000000,0.549020,0.000000}%
\pgfsetfillcolor{currentfill}%
\pgfsetfillopacity{0.449899}%
\pgfsetlinewidth{1.003750pt}%
\definecolor{currentstroke}{rgb}{1.000000,0.549020,0.000000}%
\pgfsetstrokecolor{currentstroke}%
\pgfsetstrokeopacity{0.449899}%
\pgfsetdash{}{0pt}%
\pgfpathmoveto{\pgfqpoint{4.977499in}{1.547525in}}%
\pgfpathcurveto{\pgfqpoint{4.985735in}{1.547525in}}{\pgfqpoint{4.993635in}{1.550798in}}{\pgfqpoint{4.999459in}{1.556622in}}%
\pgfpathcurveto{\pgfqpoint{5.005283in}{1.562445in}}{\pgfqpoint{5.008555in}{1.570346in}}{\pgfqpoint{5.008555in}{1.578582in}}%
\pgfpathcurveto{\pgfqpoint{5.008555in}{1.586818in}}{\pgfqpoint{5.005283in}{1.594718in}}{\pgfqpoint{4.999459in}{1.600542in}}%
\pgfpathcurveto{\pgfqpoint{4.993635in}{1.606366in}}{\pgfqpoint{4.985735in}{1.609638in}}{\pgfqpoint{4.977499in}{1.609638in}}%
\pgfpathcurveto{\pgfqpoint{4.969262in}{1.609638in}}{\pgfqpoint{4.961362in}{1.606366in}}{\pgfqpoint{4.955538in}{1.600542in}}%
\pgfpathcurveto{\pgfqpoint{4.949714in}{1.594718in}}{\pgfqpoint{4.946442in}{1.586818in}}{\pgfqpoint{4.946442in}{1.578582in}}%
\pgfpathcurveto{\pgfqpoint{4.946442in}{1.570346in}}{\pgfqpoint{4.949714in}{1.562445in}}{\pgfqpoint{4.955538in}{1.556622in}}%
\pgfpathcurveto{\pgfqpoint{4.961362in}{1.550798in}}{\pgfqpoint{4.969262in}{1.547525in}}{\pgfqpoint{4.977499in}{1.547525in}}%
\pgfpathclose%
\pgfusepath{stroke,fill}%
\end{pgfscope}%
\begin{pgfscope}%
\pgfpathrectangle{\pgfqpoint{3.894341in}{0.419100in}}{\pgfqpoint{2.504659in}{2.933700in}} %
\pgfusepath{clip}%
\pgfsetbuttcap%
\pgfsetroundjoin%
\definecolor{currentfill}{rgb}{1.000000,0.549020,0.000000}%
\pgfsetfillcolor{currentfill}%
\pgfsetfillopacity{0.465903}%
\pgfsetlinewidth{1.003750pt}%
\definecolor{currentstroke}{rgb}{1.000000,0.549020,0.000000}%
\pgfsetstrokecolor{currentstroke}%
\pgfsetstrokeopacity{0.465903}%
\pgfsetdash{}{0pt}%
\pgfpathmoveto{\pgfqpoint{5.287482in}{1.551281in}}%
\pgfpathcurveto{\pgfqpoint{5.295718in}{1.551281in}}{\pgfqpoint{5.303618in}{1.554553in}}{\pgfqpoint{5.309442in}{1.560377in}}%
\pgfpathcurveto{\pgfqpoint{5.315266in}{1.566201in}}{\pgfqpoint{5.318538in}{1.574101in}}{\pgfqpoint{5.318538in}{1.582337in}}%
\pgfpathcurveto{\pgfqpoint{5.318538in}{1.590574in}}{\pgfqpoint{5.315266in}{1.598474in}}{\pgfqpoint{5.309442in}{1.604298in}}%
\pgfpathcurveto{\pgfqpoint{5.303618in}{1.610122in}}{\pgfqpoint{5.295718in}{1.613394in}}{\pgfqpoint{5.287482in}{1.613394in}}%
\pgfpathcurveto{\pgfqpoint{5.279246in}{1.613394in}}{\pgfqpoint{5.271346in}{1.610122in}}{\pgfqpoint{5.265522in}{1.604298in}}%
\pgfpathcurveto{\pgfqpoint{5.259698in}{1.598474in}}{\pgfqpoint{5.256425in}{1.590574in}}{\pgfqpoint{5.256425in}{1.582337in}}%
\pgfpathcurveto{\pgfqpoint{5.256425in}{1.574101in}}{\pgfqpoint{5.259698in}{1.566201in}}{\pgfqpoint{5.265522in}{1.560377in}}%
\pgfpathcurveto{\pgfqpoint{5.271346in}{1.554553in}}{\pgfqpoint{5.279246in}{1.551281in}}{\pgfqpoint{5.287482in}{1.551281in}}%
\pgfpathclose%
\pgfusepath{stroke,fill}%
\end{pgfscope}%
\begin{pgfscope}%
\pgfpathrectangle{\pgfqpoint{3.894341in}{0.419100in}}{\pgfqpoint{2.504659in}{2.933700in}} %
\pgfusepath{clip}%
\pgfsetbuttcap%
\pgfsetroundjoin%
\definecolor{currentfill}{rgb}{1.000000,0.549020,0.000000}%
\pgfsetfillcolor{currentfill}%
\pgfsetfillopacity{0.619528}%
\pgfsetlinewidth{1.003750pt}%
\definecolor{currentstroke}{rgb}{1.000000,0.549020,0.000000}%
\pgfsetstrokecolor{currentstroke}%
\pgfsetstrokeopacity{0.619528}%
\pgfsetdash{}{0pt}%
\pgfpathmoveto{\pgfqpoint{5.253170in}{1.169190in}}%
\pgfpathcurveto{\pgfqpoint{5.261406in}{1.169190in}}{\pgfqpoint{5.269306in}{1.172462in}}{\pgfqpoint{5.275130in}{1.178286in}}%
\pgfpathcurveto{\pgfqpoint{5.280954in}{1.184110in}}{\pgfqpoint{5.284226in}{1.192010in}}{\pgfqpoint{5.284226in}{1.200246in}}%
\pgfpathcurveto{\pgfqpoint{5.284226in}{1.208483in}}{\pgfqpoint{5.280954in}{1.216383in}}{\pgfqpoint{5.275130in}{1.222207in}}%
\pgfpathcurveto{\pgfqpoint{5.269306in}{1.228031in}}{\pgfqpoint{5.261406in}{1.231303in}}{\pgfqpoint{5.253170in}{1.231303in}}%
\pgfpathcurveto{\pgfqpoint{5.244934in}{1.231303in}}{\pgfqpoint{5.237033in}{1.228031in}}{\pgfqpoint{5.231210in}{1.222207in}}%
\pgfpathcurveto{\pgfqpoint{5.225386in}{1.216383in}}{\pgfqpoint{5.222113in}{1.208483in}}{\pgfqpoint{5.222113in}{1.200246in}}%
\pgfpathcurveto{\pgfqpoint{5.222113in}{1.192010in}}{\pgfqpoint{5.225386in}{1.184110in}}{\pgfqpoint{5.231210in}{1.178286in}}%
\pgfpathcurveto{\pgfqpoint{5.237033in}{1.172462in}}{\pgfqpoint{5.244934in}{1.169190in}}{\pgfqpoint{5.253170in}{1.169190in}}%
\pgfpathclose%
\pgfusepath{stroke,fill}%
\end{pgfscope}%
\begin{pgfscope}%
\pgfpathrectangle{\pgfqpoint{3.894341in}{0.419100in}}{\pgfqpoint{2.504659in}{2.933700in}} %
\pgfusepath{clip}%
\pgfsetbuttcap%
\pgfsetroundjoin%
\definecolor{currentfill}{rgb}{1.000000,0.549020,0.000000}%
\pgfsetfillcolor{currentfill}%
\pgfsetfillopacity{0.518936}%
\pgfsetlinewidth{1.003750pt}%
\definecolor{currentstroke}{rgb}{1.000000,0.549020,0.000000}%
\pgfsetstrokecolor{currentstroke}%
\pgfsetstrokeopacity{0.518936}%
\pgfsetdash{}{0pt}%
\pgfpathmoveto{\pgfqpoint{5.061824in}{1.308435in}}%
\pgfpathcurveto{\pgfqpoint{5.070060in}{1.308435in}}{\pgfqpoint{5.077960in}{1.311707in}}{\pgfqpoint{5.083784in}{1.317531in}}%
\pgfpathcurveto{\pgfqpoint{5.089608in}{1.323355in}}{\pgfqpoint{5.092880in}{1.331255in}}{\pgfqpoint{5.092880in}{1.339491in}}%
\pgfpathcurveto{\pgfqpoint{5.092880in}{1.347728in}}{\pgfqpoint{5.089608in}{1.355628in}}{\pgfqpoint{5.083784in}{1.361452in}}%
\pgfpathcurveto{\pgfqpoint{5.077960in}{1.367275in}}{\pgfqpoint{5.070060in}{1.370548in}}{\pgfqpoint{5.061824in}{1.370548in}}%
\pgfpathcurveto{\pgfqpoint{5.053587in}{1.370548in}}{\pgfqpoint{5.045687in}{1.367275in}}{\pgfqpoint{5.039863in}{1.361452in}}%
\pgfpathcurveto{\pgfqpoint{5.034040in}{1.355628in}}{\pgfqpoint{5.030767in}{1.347728in}}{\pgfqpoint{5.030767in}{1.339491in}}%
\pgfpathcurveto{\pgfqpoint{5.030767in}{1.331255in}}{\pgfqpoint{5.034040in}{1.323355in}}{\pgfqpoint{5.039863in}{1.317531in}}%
\pgfpathcurveto{\pgfqpoint{5.045687in}{1.311707in}}{\pgfqpoint{5.053587in}{1.308435in}}{\pgfqpoint{5.061824in}{1.308435in}}%
\pgfpathclose%
\pgfusepath{stroke,fill}%
\end{pgfscope}%
\begin{pgfscope}%
\pgfpathrectangle{\pgfqpoint{3.894341in}{0.419100in}}{\pgfqpoint{2.504659in}{2.933700in}} %
\pgfusepath{clip}%
\pgfsetbuttcap%
\pgfsetroundjoin%
\definecolor{currentfill}{rgb}{1.000000,0.549020,0.000000}%
\pgfsetfillcolor{currentfill}%
\pgfsetfillopacity{0.709521}%
\pgfsetlinewidth{1.003750pt}%
\definecolor{currentstroke}{rgb}{1.000000,0.549020,0.000000}%
\pgfsetstrokecolor{currentstroke}%
\pgfsetstrokeopacity{0.709521}%
\pgfsetdash{}{0pt}%
\pgfpathmoveto{\pgfqpoint{5.220029in}{1.155302in}}%
\pgfpathcurveto{\pgfqpoint{5.228265in}{1.155302in}}{\pgfqpoint{5.236165in}{1.158574in}}{\pgfqpoint{5.241989in}{1.164398in}}%
\pgfpathcurveto{\pgfqpoint{5.247813in}{1.170222in}}{\pgfqpoint{5.251085in}{1.178122in}}{\pgfqpoint{5.251085in}{1.186358in}}%
\pgfpathcurveto{\pgfqpoint{5.251085in}{1.194594in}}{\pgfqpoint{5.247813in}{1.202494in}}{\pgfqpoint{5.241989in}{1.208318in}}%
\pgfpathcurveto{\pgfqpoint{5.236165in}{1.214142in}}{\pgfqpoint{5.228265in}{1.217415in}}{\pgfqpoint{5.220029in}{1.217415in}}%
\pgfpathcurveto{\pgfqpoint{5.211792in}{1.217415in}}{\pgfqpoint{5.203892in}{1.214142in}}{\pgfqpoint{5.198068in}{1.208318in}}%
\pgfpathcurveto{\pgfqpoint{5.192245in}{1.202494in}}{\pgfqpoint{5.188972in}{1.194594in}}{\pgfqpoint{5.188972in}{1.186358in}}%
\pgfpathcurveto{\pgfqpoint{5.188972in}{1.178122in}}{\pgfqpoint{5.192245in}{1.170222in}}{\pgfqpoint{5.198068in}{1.164398in}}%
\pgfpathcurveto{\pgfqpoint{5.203892in}{1.158574in}}{\pgfqpoint{5.211792in}{1.155302in}}{\pgfqpoint{5.220029in}{1.155302in}}%
\pgfpathclose%
\pgfusepath{stroke,fill}%
\end{pgfscope}%
\begin{pgfscope}%
\pgfpathrectangle{\pgfqpoint{3.894341in}{0.419100in}}{\pgfqpoint{2.504659in}{2.933700in}} %
\pgfusepath{clip}%
\pgfsetbuttcap%
\pgfsetroundjoin%
\definecolor{currentfill}{rgb}{1.000000,0.549020,0.000000}%
\pgfsetfillcolor{currentfill}%
\pgfsetfillopacity{0.629070}%
\pgfsetlinewidth{1.003750pt}%
\definecolor{currentstroke}{rgb}{1.000000,0.549020,0.000000}%
\pgfsetstrokecolor{currentstroke}%
\pgfsetstrokeopacity{0.629070}%
\pgfsetdash{}{0pt}%
\pgfpathmoveto{\pgfqpoint{4.821534in}{1.891980in}}%
\pgfpathcurveto{\pgfqpoint{4.829770in}{1.891980in}}{\pgfqpoint{4.837670in}{1.895253in}}{\pgfqpoint{4.843494in}{1.901077in}}%
\pgfpathcurveto{\pgfqpoint{4.849318in}{1.906900in}}{\pgfqpoint{4.852590in}{1.914801in}}{\pgfqpoint{4.852590in}{1.923037in}}%
\pgfpathcurveto{\pgfqpoint{4.852590in}{1.931273in}}{\pgfqpoint{4.849318in}{1.939173in}}{\pgfqpoint{4.843494in}{1.944997in}}%
\pgfpathcurveto{\pgfqpoint{4.837670in}{1.950821in}}{\pgfqpoint{4.829770in}{1.954093in}}{\pgfqpoint{4.821534in}{1.954093in}}%
\pgfpathcurveto{\pgfqpoint{4.813298in}{1.954093in}}{\pgfqpoint{4.805397in}{1.950821in}}{\pgfqpoint{4.799574in}{1.944997in}}%
\pgfpathcurveto{\pgfqpoint{4.793750in}{1.939173in}}{\pgfqpoint{4.790477in}{1.931273in}}{\pgfqpoint{4.790477in}{1.923037in}}%
\pgfpathcurveto{\pgfqpoint{4.790477in}{1.914801in}}{\pgfqpoint{4.793750in}{1.906900in}}{\pgfqpoint{4.799574in}{1.901077in}}%
\pgfpathcurveto{\pgfqpoint{4.805397in}{1.895253in}}{\pgfqpoint{4.813298in}{1.891980in}}{\pgfqpoint{4.821534in}{1.891980in}}%
\pgfpathclose%
\pgfusepath{stroke,fill}%
\end{pgfscope}%
\begin{pgfscope}%
\pgfpathrectangle{\pgfqpoint{3.894341in}{0.419100in}}{\pgfqpoint{2.504659in}{2.933700in}} %
\pgfusepath{clip}%
\pgfsetbuttcap%
\pgfsetroundjoin%
\definecolor{currentfill}{rgb}{1.000000,0.549020,0.000000}%
\pgfsetfillcolor{currentfill}%
\pgfsetfillopacity{0.589451}%
\pgfsetlinewidth{1.003750pt}%
\definecolor{currentstroke}{rgb}{1.000000,0.549020,0.000000}%
\pgfsetstrokecolor{currentstroke}%
\pgfsetstrokeopacity{0.589451}%
\pgfsetdash{}{0pt}%
\pgfpathmoveto{\pgfqpoint{5.002856in}{1.474985in}}%
\pgfpathcurveto{\pgfqpoint{5.011093in}{1.474985in}}{\pgfqpoint{5.018993in}{1.478258in}}{\pgfqpoint{5.024817in}{1.484082in}}%
\pgfpathcurveto{\pgfqpoint{5.030641in}{1.489906in}}{\pgfqpoint{5.033913in}{1.497806in}}{\pgfqpoint{5.033913in}{1.506042in}}%
\pgfpathcurveto{\pgfqpoint{5.033913in}{1.514278in}}{\pgfqpoint{5.030641in}{1.522178in}}{\pgfqpoint{5.024817in}{1.528002in}}%
\pgfpathcurveto{\pgfqpoint{5.018993in}{1.533826in}}{\pgfqpoint{5.011093in}{1.537098in}}{\pgfqpoint{5.002856in}{1.537098in}}%
\pgfpathcurveto{\pgfqpoint{4.994620in}{1.537098in}}{\pgfqpoint{4.986720in}{1.533826in}}{\pgfqpoint{4.980896in}{1.528002in}}%
\pgfpathcurveto{\pgfqpoint{4.975072in}{1.522178in}}{\pgfqpoint{4.971800in}{1.514278in}}{\pgfqpoint{4.971800in}{1.506042in}}%
\pgfpathcurveto{\pgfqpoint{4.971800in}{1.497806in}}{\pgfqpoint{4.975072in}{1.489906in}}{\pgfqpoint{4.980896in}{1.484082in}}%
\pgfpathcurveto{\pgfqpoint{4.986720in}{1.478258in}}{\pgfqpoint{4.994620in}{1.474985in}}{\pgfqpoint{5.002856in}{1.474985in}}%
\pgfpathclose%
\pgfusepath{stroke,fill}%
\end{pgfscope}%
\begin{pgfscope}%
\pgfpathrectangle{\pgfqpoint{3.894341in}{0.419100in}}{\pgfqpoint{2.504659in}{2.933700in}} %
\pgfusepath{clip}%
\pgfsetbuttcap%
\pgfsetroundjoin%
\definecolor{currentfill}{rgb}{1.000000,0.549020,0.000000}%
\pgfsetfillcolor{currentfill}%
\pgfsetfillopacity{0.578439}%
\pgfsetlinewidth{1.003750pt}%
\definecolor{currentstroke}{rgb}{1.000000,0.549020,0.000000}%
\pgfsetstrokecolor{currentstroke}%
\pgfsetstrokeopacity{0.578439}%
\pgfsetdash{}{0pt}%
\pgfpathmoveto{\pgfqpoint{4.998802in}{1.477922in}}%
\pgfpathcurveto{\pgfqpoint{5.007038in}{1.477922in}}{\pgfqpoint{5.014938in}{1.481194in}}{\pgfqpoint{5.020762in}{1.487018in}}%
\pgfpathcurveto{\pgfqpoint{5.026586in}{1.492842in}}{\pgfqpoint{5.029858in}{1.500742in}}{\pgfqpoint{5.029858in}{1.508978in}}%
\pgfpathcurveto{\pgfqpoint{5.029858in}{1.517215in}}{\pgfqpoint{5.026586in}{1.525115in}}{\pgfqpoint{5.020762in}{1.530939in}}%
\pgfpathcurveto{\pgfqpoint{5.014938in}{1.536762in}}{\pgfqpoint{5.007038in}{1.540035in}}{\pgfqpoint{4.998802in}{1.540035in}}%
\pgfpathcurveto{\pgfqpoint{4.990566in}{1.540035in}}{\pgfqpoint{4.982666in}{1.536762in}}{\pgfqpoint{4.976842in}{1.530939in}}%
\pgfpathcurveto{\pgfqpoint{4.971018in}{1.525115in}}{\pgfqpoint{4.967745in}{1.517215in}}{\pgfqpoint{4.967745in}{1.508978in}}%
\pgfpathcurveto{\pgfqpoint{4.967745in}{1.500742in}}{\pgfqpoint{4.971018in}{1.492842in}}{\pgfqpoint{4.976842in}{1.487018in}}%
\pgfpathcurveto{\pgfqpoint{4.982666in}{1.481194in}}{\pgfqpoint{4.990566in}{1.477922in}}{\pgfqpoint{4.998802in}{1.477922in}}%
\pgfpathclose%
\pgfusepath{stroke,fill}%
\end{pgfscope}%
\begin{pgfscope}%
\pgfpathrectangle{\pgfqpoint{3.894341in}{0.419100in}}{\pgfqpoint{2.504659in}{2.933700in}} %
\pgfusepath{clip}%
\pgfsetbuttcap%
\pgfsetroundjoin%
\definecolor{currentfill}{rgb}{1.000000,0.549020,0.000000}%
\pgfsetfillcolor{currentfill}%
\pgfsetfillopacity{0.409325}%
\pgfsetlinewidth{1.003750pt}%
\definecolor{currentstroke}{rgb}{1.000000,0.549020,0.000000}%
\pgfsetstrokecolor{currentstroke}%
\pgfsetstrokeopacity{0.409325}%
\pgfsetdash{}{0pt}%
\pgfpathmoveto{\pgfqpoint{5.031046in}{1.571624in}}%
\pgfpathcurveto{\pgfqpoint{5.039283in}{1.571624in}}{\pgfqpoint{5.047183in}{1.574897in}}{\pgfqpoint{5.053007in}{1.580721in}}%
\pgfpathcurveto{\pgfqpoint{5.058830in}{1.586545in}}{\pgfqpoint{5.062103in}{1.594445in}}{\pgfqpoint{5.062103in}{1.602681in}}%
\pgfpathcurveto{\pgfqpoint{5.062103in}{1.610917in}}{\pgfqpoint{5.058830in}{1.618817in}}{\pgfqpoint{5.053007in}{1.624641in}}%
\pgfpathcurveto{\pgfqpoint{5.047183in}{1.630465in}}{\pgfqpoint{5.039283in}{1.633737in}}{\pgfqpoint{5.031046in}{1.633737in}}%
\pgfpathcurveto{\pgfqpoint{5.022810in}{1.633737in}}{\pgfqpoint{5.014910in}{1.630465in}}{\pgfqpoint{5.009086in}{1.624641in}}%
\pgfpathcurveto{\pgfqpoint{5.003262in}{1.618817in}}{\pgfqpoint{4.999990in}{1.610917in}}{\pgfqpoint{4.999990in}{1.602681in}}%
\pgfpathcurveto{\pgfqpoint{4.999990in}{1.594445in}}{\pgfqpoint{5.003262in}{1.586545in}}{\pgfqpoint{5.009086in}{1.580721in}}%
\pgfpathcurveto{\pgfqpoint{5.014910in}{1.574897in}}{\pgfqpoint{5.022810in}{1.571624in}}{\pgfqpoint{5.031046in}{1.571624in}}%
\pgfpathclose%
\pgfusepath{stroke,fill}%
\end{pgfscope}%
\begin{pgfscope}%
\pgfpathrectangle{\pgfqpoint{3.894341in}{0.419100in}}{\pgfqpoint{2.504659in}{2.933700in}} %
\pgfusepath{clip}%
\pgfsetbuttcap%
\pgfsetroundjoin%
\definecolor{currentfill}{rgb}{1.000000,0.549020,0.000000}%
\pgfsetfillcolor{currentfill}%
\pgfsetfillopacity{0.475094}%
\pgfsetlinewidth{1.003750pt}%
\definecolor{currentstroke}{rgb}{1.000000,0.549020,0.000000}%
\pgfsetstrokecolor{currentstroke}%
\pgfsetstrokeopacity{0.475094}%
\pgfsetdash{}{0pt}%
\pgfpathmoveto{\pgfqpoint{5.521796in}{1.785265in}}%
\pgfpathcurveto{\pgfqpoint{5.530032in}{1.785265in}}{\pgfqpoint{5.537932in}{1.788537in}}{\pgfqpoint{5.543756in}{1.794361in}}%
\pgfpathcurveto{\pgfqpoint{5.549580in}{1.800185in}}{\pgfqpoint{5.552852in}{1.808085in}}{\pgfqpoint{5.552852in}{1.816321in}}%
\pgfpathcurveto{\pgfqpoint{5.552852in}{1.824557in}}{\pgfqpoint{5.549580in}{1.832457in}}{\pgfqpoint{5.543756in}{1.838281in}}%
\pgfpathcurveto{\pgfqpoint{5.537932in}{1.844105in}}{\pgfqpoint{5.530032in}{1.847378in}}{\pgfqpoint{5.521796in}{1.847378in}}%
\pgfpathcurveto{\pgfqpoint{5.513559in}{1.847378in}}{\pgfqpoint{5.505659in}{1.844105in}}{\pgfqpoint{5.499835in}{1.838281in}}%
\pgfpathcurveto{\pgfqpoint{5.494011in}{1.832457in}}{\pgfqpoint{5.490739in}{1.824557in}}{\pgfqpoint{5.490739in}{1.816321in}}%
\pgfpathcurveto{\pgfqpoint{5.490739in}{1.808085in}}{\pgfqpoint{5.494011in}{1.800185in}}{\pgfqpoint{5.499835in}{1.794361in}}%
\pgfpathcurveto{\pgfqpoint{5.505659in}{1.788537in}}{\pgfqpoint{5.513559in}{1.785265in}}{\pgfqpoint{5.521796in}{1.785265in}}%
\pgfpathclose%
\pgfusepath{stroke,fill}%
\end{pgfscope}%
\begin{pgfscope}%
\pgfpathrectangle{\pgfqpoint{3.894341in}{0.419100in}}{\pgfqpoint{2.504659in}{2.933700in}} %
\pgfusepath{clip}%
\pgfsetbuttcap%
\pgfsetroundjoin%
\definecolor{currentfill}{rgb}{1.000000,0.549020,0.000000}%
\pgfsetfillcolor{currentfill}%
\pgfsetfillopacity{0.452549}%
\pgfsetlinewidth{1.003750pt}%
\definecolor{currentstroke}{rgb}{1.000000,0.549020,0.000000}%
\pgfsetstrokecolor{currentstroke}%
\pgfsetstrokeopacity{0.452549}%
\pgfsetdash{}{0pt}%
\pgfpathmoveto{\pgfqpoint{5.028039in}{1.470971in}}%
\pgfpathcurveto{\pgfqpoint{5.036275in}{1.470971in}}{\pgfqpoint{5.044175in}{1.474244in}}{\pgfqpoint{5.049999in}{1.480067in}}%
\pgfpathcurveto{\pgfqpoint{5.055823in}{1.485891in}}{\pgfqpoint{5.059096in}{1.493791in}}{\pgfqpoint{5.059096in}{1.502028in}}%
\pgfpathcurveto{\pgfqpoint{5.059096in}{1.510264in}}{\pgfqpoint{5.055823in}{1.518164in}}{\pgfqpoint{5.049999in}{1.523988in}}%
\pgfpathcurveto{\pgfqpoint{5.044175in}{1.529812in}}{\pgfqpoint{5.036275in}{1.533084in}}{\pgfqpoint{5.028039in}{1.533084in}}%
\pgfpathcurveto{\pgfqpoint{5.019803in}{1.533084in}}{\pgfqpoint{5.011903in}{1.529812in}}{\pgfqpoint{5.006079in}{1.523988in}}%
\pgfpathcurveto{\pgfqpoint{5.000255in}{1.518164in}}{\pgfqpoint{4.996983in}{1.510264in}}{\pgfqpoint{4.996983in}{1.502028in}}%
\pgfpathcurveto{\pgfqpoint{4.996983in}{1.493791in}}{\pgfqpoint{5.000255in}{1.485891in}}{\pgfqpoint{5.006079in}{1.480067in}}%
\pgfpathcurveto{\pgfqpoint{5.011903in}{1.474244in}}{\pgfqpoint{5.019803in}{1.470971in}}{\pgfqpoint{5.028039in}{1.470971in}}%
\pgfpathclose%
\pgfusepath{stroke,fill}%
\end{pgfscope}%
\begin{pgfscope}%
\pgfpathrectangle{\pgfqpoint{3.894341in}{0.419100in}}{\pgfqpoint{2.504659in}{2.933700in}} %
\pgfusepath{clip}%
\pgfsetbuttcap%
\pgfsetroundjoin%
\definecolor{currentfill}{rgb}{1.000000,0.549020,0.000000}%
\pgfsetfillcolor{currentfill}%
\pgfsetfillopacity{0.604329}%
\pgfsetlinewidth{1.003750pt}%
\definecolor{currentstroke}{rgb}{1.000000,0.549020,0.000000}%
\pgfsetstrokecolor{currentstroke}%
\pgfsetstrokeopacity{0.604329}%
\pgfsetdash{}{0pt}%
\pgfpathmoveto{\pgfqpoint{5.033571in}{1.415941in}}%
\pgfpathcurveto{\pgfqpoint{5.041807in}{1.415941in}}{\pgfqpoint{5.049707in}{1.419214in}}{\pgfqpoint{5.055531in}{1.425037in}}%
\pgfpathcurveto{\pgfqpoint{5.061355in}{1.430861in}}{\pgfqpoint{5.064627in}{1.438761in}}{\pgfqpoint{5.064627in}{1.446998in}}%
\pgfpathcurveto{\pgfqpoint{5.064627in}{1.455234in}}{\pgfqpoint{5.061355in}{1.463134in}}{\pgfqpoint{5.055531in}{1.468958in}}%
\pgfpathcurveto{\pgfqpoint{5.049707in}{1.474782in}}{\pgfqpoint{5.041807in}{1.478054in}}{\pgfqpoint{5.033571in}{1.478054in}}%
\pgfpathcurveto{\pgfqpoint{5.025335in}{1.478054in}}{\pgfqpoint{5.017435in}{1.474782in}}{\pgfqpoint{5.011611in}{1.468958in}}%
\pgfpathcurveto{\pgfqpoint{5.005787in}{1.463134in}}{\pgfqpoint{5.002514in}{1.455234in}}{\pgfqpoint{5.002514in}{1.446998in}}%
\pgfpathcurveto{\pgfqpoint{5.002514in}{1.438761in}}{\pgfqpoint{5.005787in}{1.430861in}}{\pgfqpoint{5.011611in}{1.425037in}}%
\pgfpathcurveto{\pgfqpoint{5.017435in}{1.419214in}}{\pgfqpoint{5.025335in}{1.415941in}}{\pgfqpoint{5.033571in}{1.415941in}}%
\pgfpathclose%
\pgfusepath{stroke,fill}%
\end{pgfscope}%
\begin{pgfscope}%
\pgfpathrectangle{\pgfqpoint{3.894341in}{0.419100in}}{\pgfqpoint{2.504659in}{2.933700in}} %
\pgfusepath{clip}%
\pgfsetbuttcap%
\pgfsetroundjoin%
\definecolor{currentfill}{rgb}{1.000000,0.549020,0.000000}%
\pgfsetfillcolor{currentfill}%
\pgfsetfillopacity{0.317920}%
\pgfsetlinewidth{1.003750pt}%
\definecolor{currentstroke}{rgb}{1.000000,0.549020,0.000000}%
\pgfsetstrokecolor{currentstroke}%
\pgfsetstrokeopacity{0.317920}%
\pgfsetdash{}{0pt}%
\pgfpathmoveto{\pgfqpoint{5.023647in}{1.803568in}}%
\pgfpathcurveto{\pgfqpoint{5.031883in}{1.803568in}}{\pgfqpoint{5.039783in}{1.806841in}}{\pgfqpoint{5.045607in}{1.812664in}}%
\pgfpathcurveto{\pgfqpoint{5.051431in}{1.818488in}}{\pgfqpoint{5.054704in}{1.826388in}}{\pgfqpoint{5.054704in}{1.834625in}}%
\pgfpathcurveto{\pgfqpoint{5.054704in}{1.842861in}}{\pgfqpoint{5.051431in}{1.850761in}}{\pgfqpoint{5.045607in}{1.856585in}}%
\pgfpathcurveto{\pgfqpoint{5.039783in}{1.862409in}}{\pgfqpoint{5.031883in}{1.865681in}}{\pgfqpoint{5.023647in}{1.865681in}}%
\pgfpathcurveto{\pgfqpoint{5.015411in}{1.865681in}}{\pgfqpoint{5.007511in}{1.862409in}}{\pgfqpoint{5.001687in}{1.856585in}}%
\pgfpathcurveto{\pgfqpoint{4.995863in}{1.850761in}}{\pgfqpoint{4.992591in}{1.842861in}}{\pgfqpoint{4.992591in}{1.834625in}}%
\pgfpathcurveto{\pgfqpoint{4.992591in}{1.826388in}}{\pgfqpoint{4.995863in}{1.818488in}}{\pgfqpoint{5.001687in}{1.812664in}}%
\pgfpathcurveto{\pgfqpoint{5.007511in}{1.806841in}}{\pgfqpoint{5.015411in}{1.803568in}}{\pgfqpoint{5.023647in}{1.803568in}}%
\pgfpathclose%
\pgfusepath{stroke,fill}%
\end{pgfscope}%
\begin{pgfscope}%
\pgfpathrectangle{\pgfqpoint{3.894341in}{0.419100in}}{\pgfqpoint{2.504659in}{2.933700in}} %
\pgfusepath{clip}%
\pgfsetbuttcap%
\pgfsetroundjoin%
\definecolor{currentfill}{rgb}{1.000000,0.549020,0.000000}%
\pgfsetfillcolor{currentfill}%
\pgfsetfillopacity{0.849292}%
\pgfsetlinewidth{1.003750pt}%
\definecolor{currentstroke}{rgb}{1.000000,0.549020,0.000000}%
\pgfsetstrokecolor{currentstroke}%
\pgfsetstrokeopacity{0.849292}%
\pgfsetdash{}{0pt}%
\pgfpathmoveto{\pgfqpoint{5.204570in}{1.396044in}}%
\pgfpathcurveto{\pgfqpoint{5.212806in}{1.396044in}}{\pgfqpoint{5.220706in}{1.399316in}}{\pgfqpoint{5.226530in}{1.405140in}}%
\pgfpathcurveto{\pgfqpoint{5.232354in}{1.410964in}}{\pgfqpoint{5.235626in}{1.418864in}}{\pgfqpoint{5.235626in}{1.427101in}}%
\pgfpathcurveto{\pgfqpoint{5.235626in}{1.435337in}}{\pgfqpoint{5.232354in}{1.443237in}}{\pgfqpoint{5.226530in}{1.449061in}}%
\pgfpathcurveto{\pgfqpoint{5.220706in}{1.454885in}}{\pgfqpoint{5.212806in}{1.458157in}}{\pgfqpoint{5.204570in}{1.458157in}}%
\pgfpathcurveto{\pgfqpoint{5.196333in}{1.458157in}}{\pgfqpoint{5.188433in}{1.454885in}}{\pgfqpoint{5.182609in}{1.449061in}}%
\pgfpathcurveto{\pgfqpoint{5.176785in}{1.443237in}}{\pgfqpoint{5.173513in}{1.435337in}}{\pgfqpoint{5.173513in}{1.427101in}}%
\pgfpathcurveto{\pgfqpoint{5.173513in}{1.418864in}}{\pgfqpoint{5.176785in}{1.410964in}}{\pgfqpoint{5.182609in}{1.405140in}}%
\pgfpathcurveto{\pgfqpoint{5.188433in}{1.399316in}}{\pgfqpoint{5.196333in}{1.396044in}}{\pgfqpoint{5.204570in}{1.396044in}}%
\pgfpathclose%
\pgfusepath{stroke,fill}%
\end{pgfscope}%
\begin{pgfscope}%
\pgfpathrectangle{\pgfqpoint{3.894341in}{0.419100in}}{\pgfqpoint{2.504659in}{2.933700in}} %
\pgfusepath{clip}%
\pgfsetbuttcap%
\pgfsetroundjoin%
\definecolor{currentfill}{rgb}{1.000000,0.549020,0.000000}%
\pgfsetfillcolor{currentfill}%
\pgfsetfillopacity{0.793873}%
\pgfsetlinewidth{1.003750pt}%
\definecolor{currentstroke}{rgb}{1.000000,0.549020,0.000000}%
\pgfsetstrokecolor{currentstroke}%
\pgfsetstrokeopacity{0.793873}%
\pgfsetdash{}{0pt}%
\pgfpathmoveto{\pgfqpoint{4.969782in}{1.656920in}}%
\pgfpathcurveto{\pgfqpoint{4.978018in}{1.656920in}}{\pgfqpoint{4.985918in}{1.660192in}}{\pgfqpoint{4.991742in}{1.666016in}}%
\pgfpathcurveto{\pgfqpoint{4.997566in}{1.671840in}}{\pgfqpoint{5.000838in}{1.679740in}}{\pgfqpoint{5.000838in}{1.687977in}}%
\pgfpathcurveto{\pgfqpoint{5.000838in}{1.696213in}}{\pgfqpoint{4.997566in}{1.704113in}}{\pgfqpoint{4.991742in}{1.709937in}}%
\pgfpathcurveto{\pgfqpoint{4.985918in}{1.715761in}}{\pgfqpoint{4.978018in}{1.719033in}}{\pgfqpoint{4.969782in}{1.719033in}}%
\pgfpathcurveto{\pgfqpoint{4.961546in}{1.719033in}}{\pgfqpoint{4.953646in}{1.715761in}}{\pgfqpoint{4.947822in}{1.709937in}}%
\pgfpathcurveto{\pgfqpoint{4.941998in}{1.704113in}}{\pgfqpoint{4.938725in}{1.696213in}}{\pgfqpoint{4.938725in}{1.687977in}}%
\pgfpathcurveto{\pgfqpoint{4.938725in}{1.679740in}}{\pgfqpoint{4.941998in}{1.671840in}}{\pgfqpoint{4.947822in}{1.666016in}}%
\pgfpathcurveto{\pgfqpoint{4.953646in}{1.660192in}}{\pgfqpoint{4.961546in}{1.656920in}}{\pgfqpoint{4.969782in}{1.656920in}}%
\pgfpathclose%
\pgfusepath{stroke,fill}%
\end{pgfscope}%
\begin{pgfscope}%
\pgfpathrectangle{\pgfqpoint{3.894341in}{0.419100in}}{\pgfqpoint{2.504659in}{2.933700in}} %
\pgfusepath{clip}%
\pgfsetbuttcap%
\pgfsetroundjoin%
\definecolor{currentfill}{rgb}{1.000000,0.549020,0.000000}%
\pgfsetfillcolor{currentfill}%
\pgfsetfillopacity{0.715501}%
\pgfsetlinewidth{1.003750pt}%
\definecolor{currentstroke}{rgb}{1.000000,0.549020,0.000000}%
\pgfsetstrokecolor{currentstroke}%
\pgfsetstrokeopacity{0.715501}%
\pgfsetdash{}{0pt}%
\pgfpathmoveto{\pgfqpoint{5.163366in}{1.192360in}}%
\pgfpathcurveto{\pgfqpoint{5.171602in}{1.192360in}}{\pgfqpoint{5.179502in}{1.195632in}}{\pgfqpoint{5.185326in}{1.201456in}}%
\pgfpathcurveto{\pgfqpoint{5.191150in}{1.207280in}}{\pgfqpoint{5.194422in}{1.215180in}}{\pgfqpoint{5.194422in}{1.223417in}}%
\pgfpathcurveto{\pgfqpoint{5.194422in}{1.231653in}}{\pgfqpoint{5.191150in}{1.239553in}}{\pgfqpoint{5.185326in}{1.245377in}}%
\pgfpathcurveto{\pgfqpoint{5.179502in}{1.251201in}}{\pgfqpoint{5.171602in}{1.254473in}}{\pgfqpoint{5.163366in}{1.254473in}}%
\pgfpathcurveto{\pgfqpoint{5.155129in}{1.254473in}}{\pgfqpoint{5.147229in}{1.251201in}}{\pgfqpoint{5.141406in}{1.245377in}}%
\pgfpathcurveto{\pgfqpoint{5.135582in}{1.239553in}}{\pgfqpoint{5.132309in}{1.231653in}}{\pgfqpoint{5.132309in}{1.223417in}}%
\pgfpathcurveto{\pgfqpoint{5.132309in}{1.215180in}}{\pgfqpoint{5.135582in}{1.207280in}}{\pgfqpoint{5.141406in}{1.201456in}}%
\pgfpathcurveto{\pgfqpoint{5.147229in}{1.195632in}}{\pgfqpoint{5.155129in}{1.192360in}}{\pgfqpoint{5.163366in}{1.192360in}}%
\pgfpathclose%
\pgfusepath{stroke,fill}%
\end{pgfscope}%
\begin{pgfscope}%
\pgfpathrectangle{\pgfqpoint{3.894341in}{0.419100in}}{\pgfqpoint{2.504659in}{2.933700in}} %
\pgfusepath{clip}%
\pgfsetbuttcap%
\pgfsetroundjoin%
\definecolor{currentfill}{rgb}{1.000000,0.549020,0.000000}%
\pgfsetfillcolor{currentfill}%
\pgfsetfillopacity{0.560836}%
\pgfsetlinewidth{1.003750pt}%
\definecolor{currentstroke}{rgb}{1.000000,0.549020,0.000000}%
\pgfsetstrokecolor{currentstroke}%
\pgfsetstrokeopacity{0.560836}%
\pgfsetdash{}{0pt}%
\pgfpathmoveto{\pgfqpoint{4.928537in}{1.621820in}}%
\pgfpathcurveto{\pgfqpoint{4.936773in}{1.621820in}}{\pgfqpoint{4.944673in}{1.625093in}}{\pgfqpoint{4.950497in}{1.630917in}}%
\pgfpathcurveto{\pgfqpoint{4.956321in}{1.636741in}}{\pgfqpoint{4.959593in}{1.644641in}}{\pgfqpoint{4.959593in}{1.652877in}}%
\pgfpathcurveto{\pgfqpoint{4.959593in}{1.661113in}}{\pgfqpoint{4.956321in}{1.669013in}}{\pgfqpoint{4.950497in}{1.674837in}}%
\pgfpathcurveto{\pgfqpoint{4.944673in}{1.680661in}}{\pgfqpoint{4.936773in}{1.683933in}}{\pgfqpoint{4.928537in}{1.683933in}}%
\pgfpathcurveto{\pgfqpoint{4.920300in}{1.683933in}}{\pgfqpoint{4.912400in}{1.680661in}}{\pgfqpoint{4.906576in}{1.674837in}}%
\pgfpathcurveto{\pgfqpoint{4.900752in}{1.669013in}}{\pgfqpoint{4.897480in}{1.661113in}}{\pgfqpoint{4.897480in}{1.652877in}}%
\pgfpathcurveto{\pgfqpoint{4.897480in}{1.644641in}}{\pgfqpoint{4.900752in}{1.636741in}}{\pgfqpoint{4.906576in}{1.630917in}}%
\pgfpathcurveto{\pgfqpoint{4.912400in}{1.625093in}}{\pgfqpoint{4.920300in}{1.621820in}}{\pgfqpoint{4.928537in}{1.621820in}}%
\pgfpathclose%
\pgfusepath{stroke,fill}%
\end{pgfscope}%
\begin{pgfscope}%
\pgfpathrectangle{\pgfqpoint{3.894341in}{0.419100in}}{\pgfqpoint{2.504659in}{2.933700in}} %
\pgfusepath{clip}%
\pgfsetbuttcap%
\pgfsetroundjoin%
\definecolor{currentfill}{rgb}{1.000000,0.549020,0.000000}%
\pgfsetfillcolor{currentfill}%
\pgfsetfillopacity{0.622246}%
\pgfsetlinewidth{1.003750pt}%
\definecolor{currentstroke}{rgb}{1.000000,0.549020,0.000000}%
\pgfsetstrokecolor{currentstroke}%
\pgfsetstrokeopacity{0.622246}%
\pgfsetdash{}{0pt}%
\pgfpathmoveto{\pgfqpoint{5.311304in}{1.168770in}}%
\pgfpathcurveto{\pgfqpoint{5.319540in}{1.168770in}}{\pgfqpoint{5.327440in}{1.172043in}}{\pgfqpoint{5.333264in}{1.177867in}}%
\pgfpathcurveto{\pgfqpoint{5.339088in}{1.183691in}}{\pgfqpoint{5.342361in}{1.191591in}}{\pgfqpoint{5.342361in}{1.199827in}}%
\pgfpathcurveto{\pgfqpoint{5.342361in}{1.208063in}}{\pgfqpoint{5.339088in}{1.215963in}}{\pgfqpoint{5.333264in}{1.221787in}}%
\pgfpathcurveto{\pgfqpoint{5.327440in}{1.227611in}}{\pgfqpoint{5.319540in}{1.230883in}}{\pgfqpoint{5.311304in}{1.230883in}}%
\pgfpathcurveto{\pgfqpoint{5.303068in}{1.230883in}}{\pgfqpoint{5.295168in}{1.227611in}}{\pgfqpoint{5.289344in}{1.221787in}}%
\pgfpathcurveto{\pgfqpoint{5.283520in}{1.215963in}}{\pgfqpoint{5.280248in}{1.208063in}}{\pgfqpoint{5.280248in}{1.199827in}}%
\pgfpathcurveto{\pgfqpoint{5.280248in}{1.191591in}}{\pgfqpoint{5.283520in}{1.183691in}}{\pgfqpoint{5.289344in}{1.177867in}}%
\pgfpathcurveto{\pgfqpoint{5.295168in}{1.172043in}}{\pgfqpoint{5.303068in}{1.168770in}}{\pgfqpoint{5.311304in}{1.168770in}}%
\pgfpathclose%
\pgfusepath{stroke,fill}%
\end{pgfscope}%
\begin{pgfscope}%
\pgfpathrectangle{\pgfqpoint{3.894341in}{0.419100in}}{\pgfqpoint{2.504659in}{2.933700in}} %
\pgfusepath{clip}%
\pgfsetbuttcap%
\pgfsetroundjoin%
\definecolor{currentfill}{rgb}{0.400000,0.600000,0.800000}%
\pgfsetfillcolor{currentfill}%
\pgfsetfillopacity{0.844440}%
\pgfsetlinewidth{1.003750pt}%
\definecolor{currentstroke}{rgb}{0.400000,0.600000,0.800000}%
\pgfsetstrokecolor{currentstroke}%
\pgfsetstrokeopacity{0.844440}%
\pgfsetdash{}{0pt}%
\pgfpathmoveto{\pgfqpoint{5.674593in}{2.171984in}}%
\pgfpathcurveto{\pgfqpoint{5.682829in}{2.171984in}}{\pgfqpoint{5.690729in}{2.175256in}}{\pgfqpoint{5.696553in}{2.181080in}}%
\pgfpathcurveto{\pgfqpoint{5.702377in}{2.186904in}}{\pgfqpoint{5.705649in}{2.194804in}}{\pgfqpoint{5.705649in}{2.203040in}}%
\pgfpathcurveto{\pgfqpoint{5.705649in}{2.211277in}}{\pgfqpoint{5.702377in}{2.219177in}}{\pgfqpoint{5.696553in}{2.225001in}}%
\pgfpathcurveto{\pgfqpoint{5.690729in}{2.230825in}}{\pgfqpoint{5.682829in}{2.234097in}}{\pgfqpoint{5.674593in}{2.234097in}}%
\pgfpathcurveto{\pgfqpoint{5.666357in}{2.234097in}}{\pgfqpoint{5.658457in}{2.230825in}}{\pgfqpoint{5.652633in}{2.225001in}}%
\pgfpathcurveto{\pgfqpoint{5.646809in}{2.219177in}}{\pgfqpoint{5.643536in}{2.211277in}}{\pgfqpoint{5.643536in}{2.203040in}}%
\pgfpathcurveto{\pgfqpoint{5.643536in}{2.194804in}}{\pgfqpoint{5.646809in}{2.186904in}}{\pgfqpoint{5.652633in}{2.181080in}}%
\pgfpathcurveto{\pgfqpoint{5.658457in}{2.175256in}}{\pgfqpoint{5.666357in}{2.171984in}}{\pgfqpoint{5.674593in}{2.171984in}}%
\pgfpathclose%
\pgfusepath{stroke,fill}%
\end{pgfscope}%
\begin{pgfscope}%
\pgfpathrectangle{\pgfqpoint{3.894341in}{0.419100in}}{\pgfqpoint{2.504659in}{2.933700in}} %
\pgfusepath{clip}%
\pgfsetbuttcap%
\pgfsetroundjoin%
\definecolor{currentfill}{rgb}{0.400000,0.600000,0.800000}%
\pgfsetfillcolor{currentfill}%
\pgfsetfillopacity{0.555599}%
\pgfsetlinewidth{1.003750pt}%
\definecolor{currentstroke}{rgb}{0.400000,0.600000,0.800000}%
\pgfsetstrokecolor{currentstroke}%
\pgfsetstrokeopacity{0.555599}%
\pgfsetdash{}{0pt}%
\pgfpathmoveto{\pgfqpoint{5.777858in}{2.136094in}}%
\pgfpathcurveto{\pgfqpoint{5.786094in}{2.136094in}}{\pgfqpoint{5.793994in}{2.139367in}}{\pgfqpoint{5.799818in}{2.145191in}}%
\pgfpathcurveto{\pgfqpoint{5.805642in}{2.151014in}}{\pgfqpoint{5.808915in}{2.158915in}}{\pgfqpoint{5.808915in}{2.167151in}}%
\pgfpathcurveto{\pgfqpoint{5.808915in}{2.175387in}}{\pgfqpoint{5.805642in}{2.183287in}}{\pgfqpoint{5.799818in}{2.189111in}}%
\pgfpathcurveto{\pgfqpoint{5.793994in}{2.194935in}}{\pgfqpoint{5.786094in}{2.198207in}}{\pgfqpoint{5.777858in}{2.198207in}}%
\pgfpathcurveto{\pgfqpoint{5.769622in}{2.198207in}}{\pgfqpoint{5.761722in}{2.194935in}}{\pgfqpoint{5.755898in}{2.189111in}}%
\pgfpathcurveto{\pgfqpoint{5.750074in}{2.183287in}}{\pgfqpoint{5.746802in}{2.175387in}}{\pgfqpoint{5.746802in}{2.167151in}}%
\pgfpathcurveto{\pgfqpoint{5.746802in}{2.158915in}}{\pgfqpoint{5.750074in}{2.151014in}}{\pgfqpoint{5.755898in}{2.145191in}}%
\pgfpathcurveto{\pgfqpoint{5.761722in}{2.139367in}}{\pgfqpoint{5.769622in}{2.136094in}}{\pgfqpoint{5.777858in}{2.136094in}}%
\pgfpathclose%
\pgfusepath{stroke,fill}%
\end{pgfscope}%
\begin{pgfscope}%
\pgfpathrectangle{\pgfqpoint{3.894341in}{0.419100in}}{\pgfqpoint{2.504659in}{2.933700in}} %
\pgfusepath{clip}%
\pgfsetbuttcap%
\pgfsetroundjoin%
\definecolor{currentfill}{rgb}{0.400000,0.600000,0.800000}%
\pgfsetfillcolor{currentfill}%
\pgfsetfillopacity{0.441475}%
\pgfsetlinewidth{1.003750pt}%
\definecolor{currentstroke}{rgb}{0.400000,0.600000,0.800000}%
\pgfsetstrokecolor{currentstroke}%
\pgfsetstrokeopacity{0.441475}%
\pgfsetdash{}{0pt}%
\pgfpathmoveto{\pgfqpoint{5.682661in}{2.387637in}}%
\pgfpathcurveto{\pgfqpoint{5.690897in}{2.387637in}}{\pgfqpoint{5.698797in}{2.390909in}}{\pgfqpoint{5.704621in}{2.396733in}}%
\pgfpathcurveto{\pgfqpoint{5.710445in}{2.402557in}}{\pgfqpoint{5.713717in}{2.410457in}}{\pgfqpoint{5.713717in}{2.418693in}}%
\pgfpathcurveto{\pgfqpoint{5.713717in}{2.426930in}}{\pgfqpoint{5.710445in}{2.434830in}}{\pgfqpoint{5.704621in}{2.440654in}}%
\pgfpathcurveto{\pgfqpoint{5.698797in}{2.446478in}}{\pgfqpoint{5.690897in}{2.449750in}}{\pgfqpoint{5.682661in}{2.449750in}}%
\pgfpathcurveto{\pgfqpoint{5.674424in}{2.449750in}}{\pgfqpoint{5.666524in}{2.446478in}}{\pgfqpoint{5.660700in}{2.440654in}}%
\pgfpathcurveto{\pgfqpoint{5.654876in}{2.434830in}}{\pgfqpoint{5.651604in}{2.426930in}}{\pgfqpoint{5.651604in}{2.418693in}}%
\pgfpathcurveto{\pgfqpoint{5.651604in}{2.410457in}}{\pgfqpoint{5.654876in}{2.402557in}}{\pgfqpoint{5.660700in}{2.396733in}}%
\pgfpathcurveto{\pgfqpoint{5.666524in}{2.390909in}}{\pgfqpoint{5.674424in}{2.387637in}}{\pgfqpoint{5.682661in}{2.387637in}}%
\pgfpathclose%
\pgfusepath{stroke,fill}%
\end{pgfscope}%
\begin{pgfscope}%
\pgfpathrectangle{\pgfqpoint{3.894341in}{0.419100in}}{\pgfqpoint{2.504659in}{2.933700in}} %
\pgfusepath{clip}%
\pgfsetbuttcap%
\pgfsetroundjoin%
\definecolor{currentfill}{rgb}{0.400000,0.600000,0.800000}%
\pgfsetfillcolor{currentfill}%
\pgfsetfillopacity{0.366540}%
\pgfsetlinewidth{1.003750pt}%
\definecolor{currentstroke}{rgb}{0.400000,0.600000,0.800000}%
\pgfsetstrokecolor{currentstroke}%
\pgfsetstrokeopacity{0.366540}%
\pgfsetdash{}{0pt}%
\pgfpathmoveto{\pgfqpoint{5.540446in}{2.593092in}}%
\pgfpathcurveto{\pgfqpoint{5.548682in}{2.593092in}}{\pgfqpoint{5.556582in}{2.596365in}}{\pgfqpoint{5.562406in}{2.602189in}}%
\pgfpathcurveto{\pgfqpoint{5.568230in}{2.608013in}}{\pgfqpoint{5.571502in}{2.615913in}}{\pgfqpoint{5.571502in}{2.624149in}}%
\pgfpathcurveto{\pgfqpoint{5.571502in}{2.632385in}}{\pgfqpoint{5.568230in}{2.640285in}}{\pgfqpoint{5.562406in}{2.646109in}}%
\pgfpathcurveto{\pgfqpoint{5.556582in}{2.651933in}}{\pgfqpoint{5.548682in}{2.655205in}}{\pgfqpoint{5.540446in}{2.655205in}}%
\pgfpathcurveto{\pgfqpoint{5.532210in}{2.655205in}}{\pgfqpoint{5.524310in}{2.651933in}}{\pgfqpoint{5.518486in}{2.646109in}}%
\pgfpathcurveto{\pgfqpoint{5.512662in}{2.640285in}}{\pgfqpoint{5.509389in}{2.632385in}}{\pgfqpoint{5.509389in}{2.624149in}}%
\pgfpathcurveto{\pgfqpoint{5.509389in}{2.615913in}}{\pgfqpoint{5.512662in}{2.608013in}}{\pgfqpoint{5.518486in}{2.602189in}}%
\pgfpathcurveto{\pgfqpoint{5.524310in}{2.596365in}}{\pgfqpoint{5.532210in}{2.593092in}}{\pgfqpoint{5.540446in}{2.593092in}}%
\pgfpathclose%
\pgfusepath{stroke,fill}%
\end{pgfscope}%
\begin{pgfscope}%
\pgfpathrectangle{\pgfqpoint{3.894341in}{0.419100in}}{\pgfqpoint{2.504659in}{2.933700in}} %
\pgfusepath{clip}%
\pgfsetbuttcap%
\pgfsetroundjoin%
\definecolor{currentfill}{rgb}{0.400000,0.600000,0.800000}%
\pgfsetfillcolor{currentfill}%
\pgfsetfillopacity{0.317788}%
\pgfsetlinewidth{1.003750pt}%
\definecolor{currentstroke}{rgb}{0.400000,0.600000,0.800000}%
\pgfsetstrokecolor{currentstroke}%
\pgfsetstrokeopacity{0.317788}%
\pgfsetdash{}{0pt}%
\pgfpathmoveto{\pgfqpoint{5.360117in}{2.627641in}}%
\pgfpathcurveto{\pgfqpoint{5.368353in}{2.627641in}}{\pgfqpoint{5.376253in}{2.630914in}}{\pgfqpoint{5.382077in}{2.636738in}}%
\pgfpathcurveto{\pgfqpoint{5.387901in}{2.642561in}}{\pgfqpoint{5.391173in}{2.650461in}}{\pgfqpoint{5.391173in}{2.658698in}}%
\pgfpathcurveto{\pgfqpoint{5.391173in}{2.666934in}}{\pgfqpoint{5.387901in}{2.674834in}}{\pgfqpoint{5.382077in}{2.680658in}}%
\pgfpathcurveto{\pgfqpoint{5.376253in}{2.686482in}}{\pgfqpoint{5.368353in}{2.689754in}}{\pgfqpoint{5.360117in}{2.689754in}}%
\pgfpathcurveto{\pgfqpoint{5.351881in}{2.689754in}}{\pgfqpoint{5.343981in}{2.686482in}}{\pgfqpoint{5.338157in}{2.680658in}}%
\pgfpathcurveto{\pgfqpoint{5.332333in}{2.674834in}}{\pgfqpoint{5.329060in}{2.666934in}}{\pgfqpoint{5.329060in}{2.658698in}}%
\pgfpathcurveto{\pgfqpoint{5.329060in}{2.650461in}}{\pgfqpoint{5.332333in}{2.642561in}}{\pgfqpoint{5.338157in}{2.636738in}}%
\pgfpathcurveto{\pgfqpoint{5.343981in}{2.630914in}}{\pgfqpoint{5.351881in}{2.627641in}}{\pgfqpoint{5.360117in}{2.627641in}}%
\pgfpathclose%
\pgfusepath{stroke,fill}%
\end{pgfscope}%
\begin{pgfscope}%
\pgfpathrectangle{\pgfqpoint{3.894341in}{0.419100in}}{\pgfqpoint{2.504659in}{2.933700in}} %
\pgfusepath{clip}%
\pgfsetbuttcap%
\pgfsetroundjoin%
\definecolor{currentfill}{rgb}{0.400000,0.600000,0.800000}%
\pgfsetfillcolor{currentfill}%
\pgfsetfillopacity{0.300000}%
\pgfsetlinewidth{1.003750pt}%
\definecolor{currentstroke}{rgb}{0.400000,0.600000,0.800000}%
\pgfsetstrokecolor{currentstroke}%
\pgfsetstrokeopacity{0.300000}%
\pgfsetdash{}{0pt}%
\pgfpathmoveto{\pgfqpoint{5.130780in}{2.455632in}}%
\pgfpathcurveto{\pgfqpoint{5.139016in}{2.455632in}}{\pgfqpoint{5.146916in}{2.458904in}}{\pgfqpoint{5.152740in}{2.464728in}}%
\pgfpathcurveto{\pgfqpoint{5.158564in}{2.470552in}}{\pgfqpoint{5.161837in}{2.478452in}}{\pgfqpoint{5.161837in}{2.486688in}}%
\pgfpathcurveto{\pgfqpoint{5.161837in}{2.494925in}}{\pgfqpoint{5.158564in}{2.502825in}}{\pgfqpoint{5.152740in}{2.508649in}}%
\pgfpathcurveto{\pgfqpoint{5.146916in}{2.514472in}}{\pgfqpoint{5.139016in}{2.517745in}}{\pgfqpoint{5.130780in}{2.517745in}}%
\pgfpathcurveto{\pgfqpoint{5.122544in}{2.517745in}}{\pgfqpoint{5.114644in}{2.514472in}}{\pgfqpoint{5.108820in}{2.508649in}}%
\pgfpathcurveto{\pgfqpoint{5.102996in}{2.502825in}}{\pgfqpoint{5.099724in}{2.494925in}}{\pgfqpoint{5.099724in}{2.486688in}}%
\pgfpathcurveto{\pgfqpoint{5.099724in}{2.478452in}}{\pgfqpoint{5.102996in}{2.470552in}}{\pgfqpoint{5.108820in}{2.464728in}}%
\pgfpathcurveto{\pgfqpoint{5.114644in}{2.458904in}}{\pgfqpoint{5.122544in}{2.455632in}}{\pgfqpoint{5.130780in}{2.455632in}}%
\pgfpathclose%
\pgfusepath{stroke,fill}%
\end{pgfscope}%
\begin{pgfscope}%
\pgfpathrectangle{\pgfqpoint{3.894341in}{0.419100in}}{\pgfqpoint{2.504659in}{2.933700in}} %
\pgfusepath{clip}%
\pgfsetbuttcap%
\pgfsetroundjoin%
\definecolor{currentfill}{rgb}{0.400000,0.600000,0.800000}%
\pgfsetfillcolor{currentfill}%
\pgfsetfillopacity{0.426533}%
\pgfsetlinewidth{1.003750pt}%
\definecolor{currentstroke}{rgb}{0.400000,0.600000,0.800000}%
\pgfsetstrokecolor{currentstroke}%
\pgfsetstrokeopacity{0.426533}%
\pgfsetdash{}{0pt}%
\pgfpathmoveto{\pgfqpoint{4.691647in}{2.243639in}}%
\pgfpathcurveto{\pgfqpoint{4.699883in}{2.243639in}}{\pgfqpoint{4.707783in}{2.246911in}}{\pgfqpoint{4.713607in}{2.252735in}}%
\pgfpathcurveto{\pgfqpoint{4.719431in}{2.258559in}}{\pgfqpoint{4.722703in}{2.266459in}}{\pgfqpoint{4.722703in}{2.274695in}}%
\pgfpathcurveto{\pgfqpoint{4.722703in}{2.282932in}}{\pgfqpoint{4.719431in}{2.290832in}}{\pgfqpoint{4.713607in}{2.296656in}}%
\pgfpathcurveto{\pgfqpoint{4.707783in}{2.302480in}}{\pgfqpoint{4.699883in}{2.305752in}}{\pgfqpoint{4.691647in}{2.305752in}}%
\pgfpathcurveto{\pgfqpoint{4.683410in}{2.305752in}}{\pgfqpoint{4.675510in}{2.302480in}}{\pgfqpoint{4.669686in}{2.296656in}}%
\pgfpathcurveto{\pgfqpoint{4.663862in}{2.290832in}}{\pgfqpoint{4.660590in}{2.282932in}}{\pgfqpoint{4.660590in}{2.274695in}}%
\pgfpathcurveto{\pgfqpoint{4.660590in}{2.266459in}}{\pgfqpoint{4.663862in}{2.258559in}}{\pgfqpoint{4.669686in}{2.252735in}}%
\pgfpathcurveto{\pgfqpoint{4.675510in}{2.246911in}}{\pgfqpoint{4.683410in}{2.243639in}}{\pgfqpoint{4.691647in}{2.243639in}}%
\pgfpathclose%
\pgfusepath{stroke,fill}%
\end{pgfscope}%
\begin{pgfscope}%
\pgfpathrectangle{\pgfqpoint{3.894341in}{0.419100in}}{\pgfqpoint{2.504659in}{2.933700in}} %
\pgfusepath{clip}%
\pgfsetbuttcap%
\pgfsetroundjoin%
\definecolor{currentfill}{rgb}{0.400000,0.600000,0.800000}%
\pgfsetfillcolor{currentfill}%
\pgfsetfillopacity{0.842071}%
\pgfsetlinewidth{1.003750pt}%
\definecolor{currentstroke}{rgb}{0.400000,0.600000,0.800000}%
\pgfsetstrokecolor{currentstroke}%
\pgfsetstrokeopacity{0.842071}%
\pgfsetdash{}{0pt}%
\pgfpathmoveto{\pgfqpoint{4.628944in}{2.477912in}}%
\pgfpathcurveto{\pgfqpoint{4.637181in}{2.477912in}}{\pgfqpoint{4.645081in}{2.481184in}}{\pgfqpoint{4.650905in}{2.487008in}}%
\pgfpathcurveto{\pgfqpoint{4.656729in}{2.492832in}}{\pgfqpoint{4.660001in}{2.500732in}}{\pgfqpoint{4.660001in}{2.508968in}}%
\pgfpathcurveto{\pgfqpoint{4.660001in}{2.517204in}}{\pgfqpoint{4.656729in}{2.525104in}}{\pgfqpoint{4.650905in}{2.530928in}}%
\pgfpathcurveto{\pgfqpoint{4.645081in}{2.536752in}}{\pgfqpoint{4.637181in}{2.540025in}}{\pgfqpoint{4.628944in}{2.540025in}}%
\pgfpathcurveto{\pgfqpoint{4.620708in}{2.540025in}}{\pgfqpoint{4.612808in}{2.536752in}}{\pgfqpoint{4.606984in}{2.530928in}}%
\pgfpathcurveto{\pgfqpoint{4.601160in}{2.525104in}}{\pgfqpoint{4.597888in}{2.517204in}}{\pgfqpoint{4.597888in}{2.508968in}}%
\pgfpathcurveto{\pgfqpoint{4.597888in}{2.500732in}}{\pgfqpoint{4.601160in}{2.492832in}}{\pgfqpoint{4.606984in}{2.487008in}}%
\pgfpathcurveto{\pgfqpoint{4.612808in}{2.481184in}}{\pgfqpoint{4.620708in}{2.477912in}}{\pgfqpoint{4.628944in}{2.477912in}}%
\pgfpathclose%
\pgfusepath{stroke,fill}%
\end{pgfscope}%
\begin{pgfscope}%
\pgfpathrectangle{\pgfqpoint{3.894341in}{0.419100in}}{\pgfqpoint{2.504659in}{2.933700in}} %
\pgfusepath{clip}%
\pgfsetbuttcap%
\pgfsetroundjoin%
\definecolor{currentfill}{rgb}{0.400000,0.600000,0.800000}%
\pgfsetfillcolor{currentfill}%
\pgfsetfillopacity{0.713616}%
\pgfsetlinewidth{1.003750pt}%
\definecolor{currentstroke}{rgb}{0.400000,0.600000,0.800000}%
\pgfsetstrokecolor{currentstroke}%
\pgfsetstrokeopacity{0.713616}%
\pgfsetdash{}{0pt}%
\pgfpathmoveto{\pgfqpoint{4.548680in}{2.551656in}}%
\pgfpathcurveto{\pgfqpoint{4.556916in}{2.551656in}}{\pgfqpoint{4.564816in}{2.554929in}}{\pgfqpoint{4.570640in}{2.560753in}}%
\pgfpathcurveto{\pgfqpoint{4.576464in}{2.566576in}}{\pgfqpoint{4.579736in}{2.574477in}}{\pgfqpoint{4.579736in}{2.582713in}}%
\pgfpathcurveto{\pgfqpoint{4.579736in}{2.590949in}}{\pgfqpoint{4.576464in}{2.598849in}}{\pgfqpoint{4.570640in}{2.604673in}}%
\pgfpathcurveto{\pgfqpoint{4.564816in}{2.610497in}}{\pgfqpoint{4.556916in}{2.613769in}}{\pgfqpoint{4.548680in}{2.613769in}}%
\pgfpathcurveto{\pgfqpoint{4.540443in}{2.613769in}}{\pgfqpoint{4.532543in}{2.610497in}}{\pgfqpoint{4.526719in}{2.604673in}}%
\pgfpathcurveto{\pgfqpoint{4.520895in}{2.598849in}}{\pgfqpoint{4.517623in}{2.590949in}}{\pgfqpoint{4.517623in}{2.582713in}}%
\pgfpathcurveto{\pgfqpoint{4.517623in}{2.574477in}}{\pgfqpoint{4.520895in}{2.566576in}}{\pgfqpoint{4.526719in}{2.560753in}}%
\pgfpathcurveto{\pgfqpoint{4.532543in}{2.554929in}}{\pgfqpoint{4.540443in}{2.551656in}}{\pgfqpoint{4.548680in}{2.551656in}}%
\pgfpathclose%
\pgfusepath{stroke,fill}%
\end{pgfscope}%
\begin{pgfscope}%
\pgfpathrectangle{\pgfqpoint{3.894341in}{0.419100in}}{\pgfqpoint{2.504659in}{2.933700in}} %
\pgfusepath{clip}%
\pgfsetbuttcap%
\pgfsetroundjoin%
\definecolor{currentfill}{rgb}{0.400000,0.600000,0.800000}%
\pgfsetfillcolor{currentfill}%
\pgfsetfillopacity{0.988828}%
\pgfsetlinewidth{1.003750pt}%
\definecolor{currentstroke}{rgb}{0.400000,0.600000,0.800000}%
\pgfsetstrokecolor{currentstroke}%
\pgfsetstrokeopacity{0.988828}%
\pgfsetdash{}{0pt}%
\pgfpathmoveto{\pgfqpoint{4.961650in}{2.359606in}}%
\pgfpathcurveto{\pgfqpoint{4.969887in}{2.359606in}}{\pgfqpoint{4.977787in}{2.362879in}}{\pgfqpoint{4.983611in}{2.368703in}}%
\pgfpathcurveto{\pgfqpoint{4.989435in}{2.374527in}}{\pgfqpoint{4.992707in}{2.382427in}}{\pgfqpoint{4.992707in}{2.390663in}}%
\pgfpathcurveto{\pgfqpoint{4.992707in}{2.398899in}}{\pgfqpoint{4.989435in}{2.406799in}}{\pgfqpoint{4.983611in}{2.412623in}}%
\pgfpathcurveto{\pgfqpoint{4.977787in}{2.418447in}}{\pgfqpoint{4.969887in}{2.421719in}}{\pgfqpoint{4.961650in}{2.421719in}}%
\pgfpathcurveto{\pgfqpoint{4.953414in}{2.421719in}}{\pgfqpoint{4.945514in}{2.418447in}}{\pgfqpoint{4.939690in}{2.412623in}}%
\pgfpathcurveto{\pgfqpoint{4.933866in}{2.406799in}}{\pgfqpoint{4.930594in}{2.398899in}}{\pgfqpoint{4.930594in}{2.390663in}}%
\pgfpathcurveto{\pgfqpoint{4.930594in}{2.382427in}}{\pgfqpoint{4.933866in}{2.374527in}}{\pgfqpoint{4.939690in}{2.368703in}}%
\pgfpathcurveto{\pgfqpoint{4.945514in}{2.362879in}}{\pgfqpoint{4.953414in}{2.359606in}}{\pgfqpoint{4.961650in}{2.359606in}}%
\pgfpathclose%
\pgfusepath{stroke,fill}%
\end{pgfscope}%
\begin{pgfscope}%
\pgfpathrectangle{\pgfqpoint{3.894341in}{0.419100in}}{\pgfqpoint{2.504659in}{2.933700in}} %
\pgfusepath{clip}%
\pgfsetbuttcap%
\pgfsetroundjoin%
\definecolor{currentfill}{rgb}{0.400000,0.600000,0.800000}%
\pgfsetfillcolor{currentfill}%
\pgfsetlinewidth{1.003750pt}%
\definecolor{currentstroke}{rgb}{0.400000,0.600000,0.800000}%
\pgfsetstrokecolor{currentstroke}%
\pgfsetdash{}{0pt}%
\pgfpathmoveto{\pgfqpoint{5.212233in}{2.228622in}}%
\pgfpathcurveto{\pgfqpoint{5.220470in}{2.228622in}}{\pgfqpoint{5.228370in}{2.231894in}}{\pgfqpoint{5.234194in}{2.237718in}}%
\pgfpathcurveto{\pgfqpoint{5.240018in}{2.243542in}}{\pgfqpoint{5.243290in}{2.251442in}}{\pgfqpoint{5.243290in}{2.259678in}}%
\pgfpathcurveto{\pgfqpoint{5.243290in}{2.267914in}}{\pgfqpoint{5.240018in}{2.275814in}}{\pgfqpoint{5.234194in}{2.281638in}}%
\pgfpathcurveto{\pgfqpoint{5.228370in}{2.287462in}}{\pgfqpoint{5.220470in}{2.290735in}}{\pgfqpoint{5.212233in}{2.290735in}}%
\pgfpathcurveto{\pgfqpoint{5.203997in}{2.290735in}}{\pgfqpoint{5.196097in}{2.287462in}}{\pgfqpoint{5.190273in}{2.281638in}}%
\pgfpathcurveto{\pgfqpoint{5.184449in}{2.275814in}}{\pgfqpoint{5.181177in}{2.267914in}}{\pgfqpoint{5.181177in}{2.259678in}}%
\pgfpathcurveto{\pgfqpoint{5.181177in}{2.251442in}}{\pgfqpoint{5.184449in}{2.243542in}}{\pgfqpoint{5.190273in}{2.237718in}}%
\pgfpathcurveto{\pgfqpoint{5.196097in}{2.231894in}}{\pgfqpoint{5.203997in}{2.228622in}}{\pgfqpoint{5.212233in}{2.228622in}}%
\pgfpathclose%
\pgfusepath{stroke,fill}%
\end{pgfscope}%
\end{pgfpicture}%
\makeatother%
\endgroup%

		\end{center}
		\caption[]{\textbf{Left:} The problem from \ref{fig:perceptron} revisited. An 2-layer MLP with 2 input neurons, 5 hidden neurons and 2 output neurons was trained on the dataset for 10000 iterations using Backpropagation and Gradient Descent and classifies it correctly using a nonlinear classification boundary. \textbf{Right:} Visualization of the transformed data with $(x, y) \Rightarrow (x, y, out(x, y))$. The transformed dataset is then linearly separable by a plane.}
		\label{fig:mlp_trick}
	\end {figure}

\clearpage % workaround for \label in afterpage environment, see https://tex.stackexchange.com/questions/200585/label-is-undefined-when-used-in-afterpage