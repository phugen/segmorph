%---------------------------------------------------------------------------
%
%                          Vorlage der Arbeitsgruppe
%             Computer Vision and Pattern Recognition Group (CVPR)
%                           der Universität Münster
%                         http://cvpr.uni-muenster.de
%
%---------------------------------------------------------------------------
% Geeignet für:
%  - Seminararbeiten
%  - Bachelorarbeiten
%  - Masterarbeiten
%---------------------------------------------------------------------------
% Autoren:
%  - Daniel Tenbrinck
%  - Fabian Gigengack
%  - Michael Schmeing
%  - Lucas Franek
%---------------------------------------------------------------------------
% Version:
%  - 1.0.2 (09.09.2011)
%    - Titelblatt um Matrikelnummer und Studiengang ergänzt
%  - 1.0.1 (05.07.2011)
%---------------------------------------------------------------------------
% 
% "THE BEER-WARE LICENSE" (Revision 42):
% The above mentioned authors wrote this file. As long as you retain this
% notice you can do whatever you want with this stuff. If we meet some day,
% and you think this stuff is worth it, you can buy us a beer in return.
%  --------------------------------------------------------------------------

\documentclass[a4paper, twoside, 12pt, english, enabledeprecatedfontcommands]{scrbook} % Layout-Einstellungen für das Dokument
\usepackage[utf8]{inputenc} % UTF-8 Codierung
% \usepackage[ansinew]{inputenc} % bei Problemen mit Umlauten
\usepackage[english]{babel} % Deutsche Beschriftung

\usepackage{scrhack}
\usepackage{graphicx} % Um Bilder einzufügen
\usepackage{subfigure} % Um mehrere Bilder in eine figure einzufügen
\usepackage{amssymb, amsmath} % Für Mengensymbole und über Gleichheitszeichen schreiben
\usepackage{verbatim} % Um Quellcode in das Dokument einzufügen.
\usepackage{xcolor} % Für Farben
\usepackage[linkbordercolor=blue]{hyperref} % Für Links im Dokument
\usepackage{algorithmic} % Für Pseudo-Code
\usepackage{algorithm} % Wrapper für Pseudo-Code
\usepackage[font={small}, labelfont=bf]{caption} % kleine Bildunterschriften
\usepackage{geometry} % Für Feinanpassungen des Layouts
\usepackage{soul,listings,xcolor} % einzelne Zeilen in lstling highlighten

% MA
\usepackage{multicol} % multiple columns on one page.

\usepackage{mathtools} %rcases
\usepackage{tikz} % draw lines in matrix
\usetikzlibrary{arrows,matrix,positioning} % draw lines and arrow for matrix

\usepackage{listings} % Code-Listings
\usepackage{courier}
\lstset{
			 basicstyle=\footnotesize\ttfamily, % Standardschrift
			 numbers=left,               % Ort der Zeilennummern
			 numberstyle=\tiny,          % Stil der Zeilennummern
			 %stepnumber=2,               % Abstand zwischen den Zeilennummern
			 numbersep=5pt,              % Abstand der Nummern zum Text
			 tabsize=2,                  % Groesse von Tabs
			 extendedchars=true,         %
			 breaklines=true,            % Zeilen werden Umgebrochen
			 keywordstyle=\color{red},
					frame=b,         
%        keywordstyle=[1]\textbf,    % Stil der Keywords
%        keywordstyle=[2]\textbf,    %
%        keywordstyle=[3]\textbf,    %
%        keywordstyle=[4]\textbf,   \sqrt{\sqrt{}} %
			 stringstyle=\color{black}\ttfamily, % Farbe der String
			 showspaces=false,           % Leerzeichen anzeigen ?
			 showtabs=false,             % Tabs anzeigen ?
			 xleftmargin=17pt,
			 framexleftmargin=17pt,
			 framexrightmargin=5pt,
			 framexbottommargin=4pt,
			 %backgroundcolor=\color{lightgray},
			 showstringspaces=false      % Leerzeichen in Strings anzeigen ?        
}
\lstloadlanguages{% Check Dokumentation for further languages ...
			 %[Visual]Basic
			 %Pascal
			 %C
			 C++,
			 %XML
			 %HTML
			 %Java
			 Python
}
	%\DeclareCaptionFont{blue}{\color{blue}} 

%\captionsetup[lstlisting]{singlelinecheck=false, labelfont={blue}, textfont={blue}}
\usepackage{caption}
\DeclareCaptionFont{white}{\color{white}}
\DeclareCaptionFormat{listing}{\colorbox[cmyk]{0.43, 0.35, 0.35,0.01}{\parbox{\textwidth}{\hspace{15pt}#1#2#3}}}
\captionsetup[lstlisting]{format=listing,labelfont=white,textfont=white, singlelinecheck=false, margin=0pt, font={bf,footnotesize}}

\usepackage{qtree} % Bäume in LaTeX zeichnen
\lstset{escapeinside={(*@}{@*)}}

\usepackage{placeins} % FloatBarrier

\setcounter{tocdepth}{3} % paragraphs im Inhaltsverzeichnis anzeigen

% Einstellungen für Abstand an den Rändern
\geometry{a4paper,left=35mm,right=35mm,top=20mm,bottom=20mm, includeheadfoot}

\begin{document}
%\lstset{
	%captionpos=b,					% Titel von Listings unter dem Listing anzeigen,
	%frame=single,					% einfacher Rahmen
	%breaklines=true				% Zeilen automatisch umbrechen
%}
																										

%\pagenumbering{arabic}

% Titelblatt
\begin{titlepage}
\begin{centering}
\vspace*{\fill}
\includegraphics[width=12cm]{./img/wwu-logo-neu.pdf}\\[1.5cm]


{\textbf{ \Large
Multi-label Cell Segmentation in Fluorescence Microscopy Images using Deep Convolutional Neural Networks\\[1.2cm]
}}

{
Master's thesis submitted to the Department of Mathematics and Computer Science in fulfillment of the requirements for the degree of \\[2cm]
}

{\Large \textbf{Master of Science}}\\[4.5cm]

{
by:
}

{\textbf{ \large
Philipp Hugenroth\\[1cm]
}}

%{
%\textbf{Student ID}: 425967\\
%\textbf{E-Mail}: p\_huge01@uni-muenster.de\\
%\textbf{Degree}: MSc. Informatik\\[1cm]
%}

    

\hrulefill
                               
{
Supervised by:\\
\textbf{MSc. Aaron Scherzinger}\\
\textbf{Prof. Dr. Xiaoyi Jiang}\\[1cm]
}


{
Münster, \today
}
\vfill
\end{centering}
\end{titlepage}

\tableofcontents

\chapter*{Abstract}

\noindent This thesis presents the application of a Convolutional Neural Network (CNN), namely the \textit{U-Net} network, to a multi-class image segmentation problem on fluorescence microscopy data and also explores different optimization methods for preprocessing the data and training the network.\\

\noindent \textbf{Keywords}: Artifical intelligence, CNN, image segmentation, multi-class labelling
\chapter{Introduction}
Medical and biological image analysis usually is a time-consuming task that can only be carried out by domain experts. However, the sheer amount of data produced by contemporary experiments has become hard to handle manually, requiring computer-based assistance or even completely automatic image segmentation algorithms.\\

\noindent This thesis deals with the special case of automatically identifying and marking different parts of cells in images that were created using a form of fluorescence microscopy called \textit{Spinning Disk Confocal Microscopy}. The cells in question are macrophage cells of the common fruit fly, \textit{Drosophila melanogaster}, which has long been a research subject in biological experiments because of its suitability as a biological model organism. Regarding these cells, their migration and shape are of particular biological interest, requiring the observation of living cells. To obtain images while the cells are still alive (\textit{in vivo}), the macrophages are genetically altered so that they contain \textit{green fluorescent protein}, which emits green light when excited by blue or ultraviolet light. This fluorescence response is then filtered and used to create an image, or even a series of images to show the movement of these cells in form of a video. These microscopy images contain four regions of interest: The background, containing no cell material, the cell body, the \textit{Lamellipodium} and the \textit{Filopodia}. The latter two terms describe part of the cytoplasm of a cell, which can be exuded from the cell body in the form of a broad, translucent area, the Lamellipodium, and long, thin spikes traversing the Lamellipodium and the space beyond it, the Filopodia. These cell parts play roles in cell movement during wound healing and cell infection. Even for human experts, correctly identifying these areas is not always possible. Unfortunately, the samples are sensitive to light: if the samples are illuminated for too long, they deteriorate and become unusable. Also, samples are often stacked while performing microscopy to create 3D models. When viewed as individual 2D images, these circumstances lead to images that are often non-uniformly lit, sometimes noisy and often contain ``ghost'' cells that are actually parts of cells in the sample stack layer below the layer the actual image was taken from. In addition, since the cells are moving about, movement blur also occurs occasionally (see Figure \textbf{\ref{fig:cell_example}}). \cite{bioimage, bioimage2}

All of this combined makes automated image segmentation and cell analysis a difficult task.\\

\begin {figure}[!htb]	
	\centering
	\begin {subfigure}[t]{0.50\linewidth}
		%\textbf{TODO: Uncomment this}
		\includegraphics[scale=0.55]{img/fig_problems.png}
	\end {subfigure}
	\hspace{1cm}
	\begin {subfigure}[t]{0.40\linewidth}
		%\textbf{TODO: Uncomment this}
		\includegraphics[scale=1.470]{img/fig_cell_example.png}
	\end {subfigure}

	\caption[Segmentation challenges and cell parts.]{\textbf{Left:} Cropped part of a microscopy image, showing challenges segmentation algorithms must overcome: Obscured borders due to cell overlap (red), movement blur (green), clipped cells on the borders of the image (blue) and ``ghost cells'' from an image lower in the stack (orange). Additionally, the illumination in the image varies massively from cell to cell and the contrast between Lamellipodia, Filopodia and the background is generally low. \textbf{Right:} A Drosophila cell. Contrast and brightness are enhanced for clarity. The green arrow shows the cell body, the red arrow shows the Lamellopodium and the blue arrow shows one of many Filopodia.}
	\label{fig:cell_example}
\end {figure}

\noindent The remainder of this thesis is structured as follows: In Chapter \textbf{\ref{chap:concepts}}, different approaches to the image segmentation problem are explained. The following Chapter \textbf{\ref{chap:network}} describes the architecture of a neural network that was designed with a focus on cell segmentation, and also describes the individual layer types, activation functions and loss functions the network uses. Then, in Chapter \textbf{\ref{chap:training}}, information about the datasets used for training and validating the network as well as information about data pre-processing, training parameters and optimization methods is given. In Chapter \textbf{\ref{chap:results}}, the hardware details of the training- and testing environment are listed and the results of all proposed approaches are compared to each other. In Chapter \textbf{\ref{chap:conclusion}}, a conclusion is derived from the achieved results and finally, Chapter \textbf{\ref{chap:futurework}} presents some ideas that were considered worth examining but went untested in the course of this thesis due to time constraints. 
\chapter{Approaches}

	\section{Thresholding}

	\section{K-Means}

	\section{Gaussian Mixture Models}

	\section{Graph Cuts}

	\section{Convolutional Neural Networks}
\chapter {U-Net}
Although possible, it wasn't necessary to construct a fitting CNN from scratch. Since the task of image segmentation is popular, a number of propositions have been made; among them is the \textit{U-Net}. The \textit{U-Net} is a CNN architecture proposed by Ronneberger et. al in 2015\cite{unet} which aims to produce binary segmentation maps for images of cells. The network consists of a first, ``contracting'' path which is compromised by a series of convolutions, followed by max-pooling layers, and then a ``widening'' path which performs deconvolutions, giving the network a U-shape. The network also uses \textit{Dropout} layers\cite{dropout} for regularization of its weights.

	\section {Net structure}
	% Image of U-Net here

	\section {Loss layers}

		\subsection{Softmax}

The loss layer that was proposed originally for the U-Net architecture is a \textit{Softmax Loss} layer. The name \textit{Softmax Loss} can lead to misunderstandings, because the layer's internal implementation actually consists of two steps:

First, the input $z$ of the layer is transformed by the softmax function

\[\sigma(z) = \frac{e^{z}}{\sum \limits_{k=0}^{K} e^{z_k}},\]

\noindent where $e$ is the exponential function and $z$ has a dimension of $1 \times K$. The softmax function acts as a normalizer that ``squashes'' the values of its input vector into the range $[0, 1]$ with respect to the differences between the vector values. For example, let $z = [-2.432, 1.832, 0.299]$. Then $\sigma(z) = [0.011, 0.813, 0.176]$ is the softmax output for $z$.

These processed values can be interpreted as probabilities much more easily than the raw scores the loss layer receives, which in turns leads to the second step, in which the actual loss computation is done. The actual loss function the layer uses is the \textit{Cross-Entropy Loss} that compares a true probability distribution $p$ to a model probability distribution $q$. For a vector of calculated softmax probabilities $z_i$, it is defined as follows:

\[CE(p, q) = -\sum \limits_{i = 0}^{K} p(z) \log q(z)\]

\noindent Conveniently, the softmax function forces all input values to be positive, allowing the logarithm in the loss to be used. Even though $\log (0)$ isn't defined, the softmax function should never produce values that are exactly equal to zero or one. \textbf{TODO: WHY?} 

However, the above definition of the softmax function can become numerically unstable when the exponentiation yields large values because dividing by large numbers could potentially produce zero values. To prevent this, the softmax can be made robust by defining it as

\[z_{new} = z - \max \limits_{i = 0}(0, z_i)\]
\[\sigma(z_{new}) = \frac{e^{z_{new}}}{\sum \limits_{k=0}^{K} e^{z_{new_{k}}}}\]

\noindent Since each pixel of the input image belongs to one class only, this means that for a given pixel, the probability of its ground truth label is $1.0$, while all others are $0.0$. If, without loss of generality, the third class is assumed to be the ground-truth class, $p$ will be $[0.0, 0.0, 1.0, 0.0]$.  Hence, only the predicted probability of the true third class influences the loss, and all the other summands can be omitted, yielding the condensed general formula

\[CE'(q) = - \log(q_{true}),\]

\noindent where $q_{true}$ denotes the predicted probability for the true class. The loss of the entire image is then calculated as the sum of the losses of all pixels divided by the number of pixels.



		\subsection{Weighted Softmax}

		\subsection{F-Measure}

\chapter{Training}

	\section{Training Set}

	\section{Hardware Specifications}

	\section {Preprocessing and Data Augmentation}


		\subsection{Mean Substraction (Data Standardization)}

		\subsection{Cropping and Rotation}

		\subsection{Elastic Deformation}

	\section {Training Algorithms and Parameters}
	
	\section{Non-convex Functions and Momentum}

	\section {Pseudo-Labelling}

		\subsubsection {The Cluster Assumption}

	\section {Validation}
		
		\subsection{Early Stopping}
\chapter {Results}
\label{chap:results}

	\section{Hardware}
All variants of the U-Net were trained on a NVIDIA TITAN Xp GPU (12 GB VRAM) using Caffe's CUDA/cuDNN support.

	\section {Segmentation quality evaluation}

\noindent As there were many possible variations of the U-Net architecture, these variations were tested iteratively, choosing the best network of a number of networks and modifying it further. This was done because an exhaustive search for the best combination of weight initialization, activation functions, hyperparameters like the learning rate and techniques such as Dropout and Batch Normalization would have exceeded the time limit of this thesis. Also, at the time of writing, it was not yet known how these different approaches interact precisely. For example, Batch Normalization and specialized weight initialization schemes have the same goal but achieve it in different ways, making it unclear whether one or the other performs better in practice.\\

\textbf{TODO: Enable shuffling in tests https://valserb.wordpress.com/2016/05/15/hdf5-shuffle-caffe/}\\

\noindent To compare the performance of all methods on the validation set with each other, the \textit{Micro} and \textit{Macro} variants of the F-Measure \cite{micromacro} are a suitable way to quantize how well the segmentation works. The Micro F-Measure is defined by the Precision and Recall quantities (see Section \textbf{\ref{subsec:fmeasure}}) of the validation set:

\[ F_{1\mu} = 2 \left ( \frac{PR_\mu \cdot RC_\mu}{PR_\mu + RC_\mu} \right ) \]

\noindent Here, $PR_\mu$ and $RC_\mu$ denote the micro-average Precision and Recall over the entire validation set. $PR_\mu$ and $RC_\mu$ are calculated by taking the sum of all $TP$, $FP$ and $FN$ values for all images and deriving the Precision and Recall over all $n$ validation images from these sums, i.e.

\[ PR_\mu = \frac{\sum_{i=1}^{n} TP_i}{\sum_{i=1}^{n} (TP_i + FP_i)} \text{ and }  RC_\mu = \frac{\sum_{i=1}^{n} TP_i}{\sum_{i=1}^{n} (TP_i + FN_i)} \]

\noindent The Macro F-Measure likewise is defined as

\[ F_{1M} = 2 \left ( \frac{PR_M \cdot RC_M}{PR_M + RC_M} \right ) \]

\noindent where $PR_M$ and $RC_M$ are the macro-average Precision and Recall. These are calculated for each sample independently, summed, and averaged over all $n$ samples:

\[ PR_M = \frac{1}{n} \sum_{i=1}^{n} \frac{TP_i}{TP_i + FP_i} \text { and } RC_M = \frac{1}{n} \sum_{i=1}^{n} \frac{TP_i}{TP_i + FN_i} \] 

\noindent \cite[pp. 317-318]{information_retrieval} highlights that the Micro F-Measure is dominated by ``large'' classes, meaning classes that occur often in the ground truth data. This shifts the focus of the segmentation effectiveness evaluation towards whether the large classes are segmented correctly. As most pixels in the validation images are background pixels and the correct segmentation of the non-background class pixels is of more interest, the Macro F-Measure is therefore chosen for assessing which method performs best because it is biased towards smaller classes rather than large ones, but for completeness, both quantities are listed.\\

\noindent The first test run pitted two nearly identical U-Net networks against each other, using ReLU activations and Dropout with $p = 0.5$. The only difference was the choice of the loss function. The \textbf{Unet\_Weighted} networks used the weighted Cross-Entropy Loss, while the \textbf{Unet\_F1} networks employed the multi-class F-Measure. Both networks were trained for 70,000 iterations on both the 3-class and the 4-class training set, while testing the network on the respective validation set every 1,000 iterations.

\textbf{Unet\_Weighted} used an initial learning rate of 0.001, a step learning rate decay of a factor $\zeta = 0.1$ every 20,000 iterations and a momentum modifier $\gamma = 0.99$, while \textbf{Unet\_F1} used an initial learning rate of 0.0001, $\zeta = 0.3$ every 20,000 iterations and $\gamma = 0.99$. Both networks used L2 gradient regularization and a mini-batch size of 5.\\

\noindent The results of the training are shown in Table \textbf{\ref{tab:results1}}. They indicate that both networks perform similarly well, although using a Cross-Entropy loss function beats the F-Measure slightly. The Cross-Entropy networks achieved Macro F-Measure scores of $\approx$\textbf{0.877} for 3 classes and $\approx$\textbf{0.746} for 4 classes. Therefore, for the following tests, Cross-Entropy was used as the loss function.\\ 

\begin {table}
	\begin{flushleft}
		\begin {tabular}[!ht]{|l|c|c|c|c|}
			\hline\multicolumn{5}{|l|}{\textbf{3-class Micro F-Measure Scores}} \\ \hline
			\textbf{Network}& \textbf{Class 1}& \textbf{Class 2}& \textbf{Class 3}& \textbf{Overall} \\ \hline
			Unet\_Weighted\_3& \cellcolor{green!25}0.936013& \cellcolor{green!25}0.979221& \cellcolor{green!25}0.848447& \cellcolor{green!25}0.959854 \\ \hline
			Unet\_F1\_3& 0.933723& 0.968166& 0.786455&  0.941755\\ \hline
			\multicolumn{5}{|l|}{\textbf{3-class Macro F-Measure Scores}} \\ \hline
			\textbf{Network}& \textbf{Class 1}& \textbf{Class 2}& \textbf{Class 3}& \textbf{Overall} \\ \hline
			Unet\_Weighted\_3& \cellcolor{green!25}0.841432& \cellcolor{green!25}0.977186& \cellcolor{green!25}0.803589& \cellcolor{green!25}0.876880 \\ \hline
			Unet\_F1\_3& 0.837715& 0.964669& 0.761217& 0.858995 \\ \hline
		\end {tabular}
		\vspace{0.5cm}\\
		\begin {tabular}[!ht]{|l|c|c|c|c|c|}
			\hline\multicolumn{6}{|l|}{\textbf{4-class Micro F-Measure Scores}} \\ \hline
			\textbf{Network}& \textbf{Class 1}& \textbf{Class 2}& \textbf{Class 3}& \textbf{Class 4}& \textbf{Overall} \\ \hline
			Unet\_Weighted\_4& \cellcolor{green!25}0.63248& 0.978377& \cellcolor{green!25}0.660174& 0.927164& 0.934388 \\ \hline
			Unet\_F1\_4& 0.632356& \cellcolor{green!25}0.978707& 0.641875& \cellcolor{green!25}0.928546& \cellcolor{green!25}0.935324 \\ \hline
			\multicolumn{6}{|l|}{\textbf{4-class Macro F-Measure Scores}} \\ \hline
			\textbf{Network}& \textbf{Class 1}& \textbf{Class 2}& \textbf{Class 3}& \textbf{Class 4}& \textbf{Overall} \\ \hline
			Unet\_Weighted\_4& \cellcolor{green!25}0.59455& 0.975988& 0.565911& \cellcolor{green!25}0.827576& \cellcolor{green!25}0.746051 \\ \hline
			Unet\_F1\_4& 0.592614& \cellcolor{green!25}0.976762& \cellcolor{green!25}0.57009& 0.823417& 0.742489 \\ \hline
		\end {tabular}
	\end {flushleft}

\caption[]{Micro and Macro F-Measure scores of 3 and 4-class CNN segmentations, using the Weighted Cross-Entropy and the Multiclass F-Measure as loss functions. In the 4-class dataset, \textbf{class 1} is the background, \textbf{class 2} is the cell proper, \textbf{class 3} are the Filopodia and \textbf{class 4} are the Lamellopodia, while in the 3-class dataset, \textbf{class 3} represents both Filopodia and Lamellopodia. The best scores in each category, per class, are marked in green.}
\label{tab:results1}
\end {table}

\textbf{TODO: Evaluation of training progress, i.e. validation loss progession}\\

\noindent The next training case concerned whether using Batch Normalization provides benefits, either in convergence speed or overall score. Therefore, the \textbf{Unet\_Weighted} networks were modified to perform Batch Normalization before each ReLU activation, implemented in Caffe as a ``Batch Normalization'' layer that normalizes its input according to the mini-batch statistics, followed by a ``Scale'' layer that applies the affine transformation.

Again, the network was trained on the 3- and 4-class datasets, this time, as advised in \cite{batchnorm}, with a higher initial learning rate of 0.01, and a faster step learning rate decay that reduces the learning rate by $\zeta = 0.1$ each 3,000 iterations. Momentum was kept at $\gamma = 0.99$ and the network was trained for 70,000 iterations again, testing every 1,000 iterations, and using L2 regularization as well as a mini-batch size of 5. The results of the training are shown in Table \textbf{\ref{tab:results2}}.

\begin {table}
	\begin{flushleft}
		\begin {tabular}[!ht]{|l|c|c|c|c|}
			\hline\multicolumn{5}{|l|}{\textbf{3-class Micro F-Measure Scores}} \\ \hline
			\textbf{Network}& \textbf{Class 1}& \textbf{Class 2}& \textbf{Class 3}& \textbf{Overall} \\ \hline
			Unet\_Weighted\_3& 0.936013& 0.979221& 0.848447& 0.959854 \\ \hline
			Unet\_Weighted\_Batchnorm\_3& & & & \\ \hline
			\multicolumn{5}{|l|}{\textbf{3-class Macro F-Measure Scores}} \\ \hline
			\textbf{Network}& \textbf{Class 1}& \textbf{Class 2}& \textbf{Class 3}& \textbf{Overall} \\ \hline
			Unet\_Weighted\_3& 0.841432& 0.977186&0.803589& 0.876880 \\ \hline
			Unet\_Weighted\_Batchnorm\_3& & & & \\ \hline
		\end {tabular}
		\vspace{0.5cm}\\
		\begin {tabular}[!ht]{|l|c|c|c|c|c|}
			\hline\multicolumn{6}{|l|}{\textbf{4-class Micro F-Measure Scores}} \\ \hline
			\textbf{Network}& \textbf{Class 1}& \textbf{Class 2}& \textbf{Class 3}& \textbf{Class 4}& \textbf{Overall} \\ \hline
			Unet\_Weighted\_4& 0.63248& 0.978377& 0.660174& 0.927164& 0.934388 \\ \hline
			Unet\_Weighted\_Batchnorm\_4& & & & & \\ \hline
			\multicolumn{6}{|l|}{\textbf{4-class Macro F-Measure Scores}} \\ \hline
			\textbf{Network}& \textbf{Class 1}& \textbf{Class 2}& \textbf{Class 3}& \textbf{Class 4}& \textbf{Overall} \\ \hline
			Unet\_Weighted\_4& 0.59455& 0.975988& 0.565911& 0.827576& 0.746051 \\ \hline
			Unet\_Weighted\_Batchnorm\_4& & & & & \\ \hline
		\end {tabular}
	\end {flushleft}

\caption[]{Micro and Macro F-Measure scores of 3 and 4-class segmenations using a weighted Cross-Entropy loss with or without batch normalization.}
\label{tab:results2}
\end {table}

\noindent It is evident that \\


\begin {table}
	\centering
	\begin {tabular}[!ht]{|l|c|c|}
		\hline
		\textbf{Activation}& \textbf{3 classes}& \textbf{4 classes}\\ \hline
		ReLU& & \\ \hline
		LReLU& & \\ \hline
		PReLU& & \\ \hline
		ELU& & \\ \hline
	\end {tabular}
\caption[]{Multi-class F-Measure scores of 3 and 4-class segmentations for the \textbf{TODO} network, using different activations functions.}
\end {table}

\noindent Then, the effect of the weight initialization on the best network was compared, using the Xavier and the MSRA initializations.

\begin {table}
	\centering
	\begin {tabular}[!ht]{|l|c|c|}
		\hline
		\textbf{Init method}& \textbf{3 classes}& \textbf{4 classes}\\ \hline
		Xavier& & \\ \hline
		MSRA& & \\ \hline
	\end {tabular}
\caption[]{Multi-class F-Measure scores of 3 and 4-class segmentations for the \textbf{TODO} network, using \textbf{TODO} activations and either Xavier or MSRA weight initialization.}
\end {table}



\noindent Because Otsu thresholding, K-Means and Gaussian Mixture Models are all unsupervised methods, i.e. they do not depend on ground truth images, the labels they output have no direct relation to the ground truth labels used in the CNN training. Therefore, all combinations of matching the output labels with the ground truth labels are evaluated and for each, a multiclass F-Measure score is calculated. The assignment with the highest score is then assumed to be the correct one, which is then used for the overall evaluation using the Macro F-Measure over all segmentations.


\begin {table}
	\centering
	\begin {tabular}[!ht]{|l|c|c|}
		\hline
		\textbf{Method}& \textbf{3 classes}& \textbf{4 classes}\\ \hline
		Otsu& & \\ \hline
		K-Means& & \\ \hline
		GMM& & \\ \hline
		\textbf{TODO-network}& & \\ \hline
	\end {tabular}
\caption[]{Multi-class F-Measure scores of 3 and 4-class segmentations. The best network is compared to the outputs of unsupervised methods.}
\end {table}
\chapter {Future work}

	\section{Max-Out}
	\section{Denoising Auto-Encoder}
\include{chapter/8_addendum}

\cleardoublepage
\pagenumbering{arabic}


\listoffigures
\listoftables

\bibliographystyle{plain}
\bibliography{sources}

\end{document}